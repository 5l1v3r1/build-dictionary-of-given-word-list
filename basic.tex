\documentclass[10pt,a4paper,twoside]{article}%
\usepackage[T1]{fontenc}%
\usepackage[utf8]{inputenc}%
\usepackage{lmodern}%
\usepackage{textcomp}%
\usepackage{lastpage}%
\usepackage[top=2.3cm,bottom=2.0cm,left=2.5cm,right=2.0cm,columnsep=20pt]{geometry}%
\usepackage{palatino}%
\usepackage{microtype}%
\usepackage{multicol}%
\usepackage{fontspec}%
\usepackage{enumitem}%
\usepackage[bf, sf, center]{titlesec}%
\usepackage{fancyhdr}%
%
\fancyhead[L]{\textsf{\rightmark}}%
\fancyhead[R]{\textsf{\leftmark}}%
\renewcommand{\headrulewidth}{1.4pt}%
\fancyfoot[C]{\textbf{\textsf{\thepage}}}%
\renewcommand{\footrulewidth}{1.4pt}%
\pagestyle{fancy}%
%
\begin{document}%
\normalsize%
\newcommand{\entry}[4]{\textbf{#1}\markboth{#1}{#1}\ {{\fontspec{Doulos SIL} #2}}\  {{\fontspec{Kalpurush} #3}}\ $\bullet$\ {#4}}%
\begin{multicols}{2}%
\entry{abandon}{/əˈband(ə)n/}{বর্জন করা}{ \textsf{noun} \textbf{1} Complete lack of inhibition or restraint. {\fontspec{Times New Roman}◇} \textit{she sings and sways with total abandon} \textsf{verb} \textbf{1} Cease to support or look after (someone); desert. {\fontspec{Times New Roman}◇} \textit{her natural mother had abandoned her at an early age} \textbf{2} Give up completely (a practice or a course of action) {\fontspec{Times New Roman}◇} \textit{he had clearly abandoned all pretence of trying to succeed} \textbf{3} Allow oneself to indulge in (a desire or impulse) {\fontspec{Times New Roman}◇} \textit{they abandoned themselves to despair}}%
\par%
\entry{abduct}{/əbˈdʌkt/}{অপহরণ করা}{ \textsf{verb} \textbf{1} Take (someone) away illegally by force or deception; kidnap. {\fontspec{Times New Roman}◇} \textit{the millionaire who disappeared may have been abducted} \textbf{2} (of a muscle) move (a limb or part) away from the midline of the body or from another part. {\fontspec{Times New Roman}◇} \textit{the posterior rectus muscle, which abducts the eye}}%
\par%
\entry{abductor}{/əbˈdʌktə/}{অপহরণকারী}{ \textsf{noun} \textbf{1} A person who abducts another person. {\fontspec{Times New Roman}◇} \textit{she endured a two-hour ordeal at the hands of her abductors} \textbf{2}  {\fontspec{Times New Roman}◇} \textit{}}%
\par%
\entry{ablaze}{/əˈbleɪz/}{বহ্নিমান}{ \textsf{adjective} \textbf{1} Burning fiercely. {\fontspec{Times New Roman}◇} \textit{his clothes were ablaze}}%
\par%
\entry{abound}{/əˈbaʊnd/}{উড়া}{ \textsf{verb} \textbf{1} Exist in large numbers or amounts. {\fontspec{Times New Roman}◇} \textit{rumours of a further scandal abound}}%
\par%
\entry{absorb}{/əbˈzɔːb/}{শোষণ করা}{ \textsf{verb} \textbf{1} Take in or soak up (energy or a liquid or other substance) by chemical or physical action. {\fontspec{Times New Roman}◇} \textit{buildings can be designed to absorb and retain heat} \textbf{2} Take up the attention of (someone); interest greatly. {\fontspec{Times New Roman}◇} \textit{she sat in an armchair, absorbed in a book}}%
\par%
\entry{absurd}{/əbˈsəːd/}{কিম্ভুতকিমাকার}{ \textsf{adjective} \textbf{1} Wildly unreasonable, illogical, or inappropriate. {\fontspec{Times New Roman}◇} \textit{the allegations are patently absurd} \textsf{noun} \textbf{1} An absurd state of affairs. {\fontspec{Times New Roman}◇} \textit{the incidents that followed bordered on the absurd}}%
\par%
\entry{abundant}{/əˈbʌnd(ə)nt/}{প্রচুর}{ \textsf{adjective} \textbf{1} Existing or available in large quantities; plentiful. {\fontspec{Times New Roman}◇} \textit{there was abundant evidence to support the theory}}%
\par%
\entry{abundantly}{/əˈbʌnd(ə)ntli/}{প্রচুর পরিমাণে}{ \textsf{adverb} \textbf{1} In large quantities; plentifully. {\fontspec{Times New Roman}◇} \textit{the plant grows abundantly in the wild}}%
\par%
\entry{abysmal}{/əˈbɪzm(ə)l/}{অতল; ভয়ঙ্কর}{ \textsf{adjective} \textbf{1} Extremely bad; appalling. {\fontspec{Times New Roman}◇} \textit{the quality of her work is abysmal} \textbf{2} Very deep. {\fontspec{Times New Roman}◇} \textit{waterfalls that plunge into abysmal depths}}%
\par%
\entry{accomplice}{/əˈkʌmplɪs/}{যোগদানকারী}{ \textsf{noun} \textbf{1} A person who helps another commit a crime. {\fontspec{Times New Roman}◇} \textit{an accomplice in the murder}}%
\par%
\entry{accomplish}{/əˈkʌmplɪʃ/}{সাধা}{ \textsf{verb} \textbf{1} Achieve or complete successfully. {\fontspec{Times New Roman}◇} \textit{the planes accomplished their mission}}%
\par%
\entry{accord}{/əˈkɔːd/}{চুক্তি}{ \textsf{noun} \textbf{1} An official agreement or treaty. {\fontspec{Times New Roman}◇} \textit{opposition groups refused to sign the accord} \textsf{verb} \textbf{1} Give or grant someone (power, status, or recognition) {\fontspec{Times New Roman}◇} \textit{the powers accorded to the head of state} \textbf{2} (of a concept or fact) be harmonious or consistent with. {\fontspec{Times New Roman}◇} \textit{his views accorded well with those of Merivale}}%
\par%
\entry{account}{/əˈkaʊnt/}{হিসাব}{ \textsf{noun} \textbf{1} A report or description of an event or experience. {\fontspec{Times New Roman}◇} \textit{a detailed account of what has been achieved} \textbf{2} A record or statement of financial expenditure and receipts relating to a particular period or purpose. {\fontspec{Times New Roman}◇} \textit{the barman was doing his accounts} \textbf{3} An arrangement by which a body holds funds on behalf of a client or supplies goods or services to them on credit. {\fontspec{Times New Roman}◇} \textit{a bank account} \textbf{4} An arrangement by which a user is given personalized access to a computer, website, or application, typically by entering a username and password. {\fontspec{Times New Roman}◇} \textit{we've reset your password to prevent others from accessing your account} \textbf{5} Importance. {\fontspec{Times New Roman}◇} \textit{money was of no account to her} \textsf{verb} \textbf{1} Consider or regard in a specified way. {\fontspec{Times New Roman}◇} \textit{her visit could not be accounted a success} \textbf{2} Give or receive an account for money received. {\fontspec{Times New Roman}◇} \textit{after 1292 he accounted to the Westminster exchequer}}%
\par%
\entry{accuse}{/əˈkjuːz/}{অভিযুক্ত করা}{ \textsf{verb} \textbf{1} Charge (someone) with an offence or crime. {\fontspec{Times New Roman}◇} \textit{he was accused of murdering his wife's lover}}%
\par%
\entry{accustom}{/əˈkʌstəm/}{অভ্যস্ত করা}{ \textsf{verb} \textbf{1} Make someone or something accept (something) as normal or usual. {\fontspec{Times New Roman}◇} \textit{I accustomed my eyes to the lenses}}%
\par%
\entry{ace}{/eɪs/}{টেক্কা}{ \textsf{adjective} \textbf{1} Very good. {\fontspec{Times New Roman}◇} \textit{an ace swimmer} \textsf{noun} \textbf{1} A playing card with a single spot on it, ranked as the highest card in its suit in most card games. {\fontspec{Times New Roman}◇} \textit{the ace of diamonds} \textbf{2} A person who excels at a particular sport or other activity. {\fontspec{Times New Roman}◇} \textit{a motorcycle ace} \textbf{3} (in tennis and similar games) a service that an opponent is unable to return and thus wins a point. {\fontspec{Times New Roman}◇} \textit{Nadal banged down eight aces in the set} \textsf{verb} \textbf{1} (in tennis and similar games) serve an ace against (an opponent) {\fontspec{Times New Roman}◇} \textit{he can ace opponents with serves of no more than 62 mph} \textbf{2} Achieve high marks in (a test or exam) {\fontspec{Times New Roman}◇} \textit{I aced my grammar test}}%
\par%
\entry{ace}{/eɪs/}{টেক্কা}{ \textsf{adjective} \textbf{1} (of a person) having no sexual feelings or desires; asexual. {\fontspec{Times New Roman}◇} \textit{I didn't realize that I was ace for a long time} \textsf{noun} \textbf{1} A person who has no sexual feelings or desires. {\fontspec{Times New Roman}◇} \textit{both asexual, they have managed to connect with other aces offline}}%
\par%
\entry{ache}{/eɪk/}{ব্যাথা}{ \textsf{noun} \textbf{1} A continuous or prolonged dull pain in a part of one's body. {\fontspec{Times New Roman}◇} \textit{the ache in her head worsened} \textsf{verb} \textbf{1} Suffer from a continuous dull pain. {\fontspec{Times New Roman}◇} \textit{my legs ached from the previous day's exercise}}%
\par%
\entry{acquire}{/əˈkwʌɪə/}{অর্জন}{ \textsf{verb} \textbf{1} Buy or obtain (an asset or object) for oneself. {\fontspec{Times New Roman}◇} \textit{I managed to acquire all the books I needed} \textbf{2} Learn or develop (a skill, habit, or quality) {\fontspec{Times New Roman}◇} \textit{you must acquire the rudiments of Greek}}%
\par%
\entry{acquisition}{/ˌakwɪˈzɪʃ(ə)n/}{অর্জন; অধিগ্রহণ}{ \textsf{noun} \textbf{1} An asset or object bought or obtained, typically by a library or museum. {\fontspec{Times New Roman}◇} \textit{the legacy will be used for new acquisitions} \textbf{2} The learning or developing of a skill, habit, or quality. {\fontspec{Times New Roman}◇} \textit{the acquisition of management skills}}%
\par%
\entry{acting}{/ˈaktɪŋ/}{অভিনয়}{ \textsf{adjective} \textbf{1} Temporarily doing the duties of another person. {\fontspec{Times New Roman}◇} \textit{the acting supervisor} \textsf{noun} \textbf{1} The art or occupation of performing fictional roles in plays, films, or television. {\fontspec{Times New Roman}◇} \textit{she studied acting in New York}}%
\par%
\entry{adamant}{/ˈadəm(ə)nt/}{হীরক}{ \textsf{adjective} \textbf{1} Refusing to be persuaded or to change one's mind. {\fontspec{Times New Roman}◇} \textit{he is adamant that he is not going to resign} \textsf{noun} \textbf{1} A legendary rock or mineral to which many properties were attributed, formerly associated with diamond or lodestone. {\fontspec{Times New Roman}◇} \textit{As for the magical metal, asiceton, it sounds like adamant.}}%
\par%
\entry{adaptation}{/adəpˈteɪʃ(ə)n/}{অভিযোজন}{ \textsf{noun} \textbf{1} The action or process of adapting or being adapted. {\fontspec{Times New Roman}◇} \textit{the adaptation of teaching strategy to meet students' needs}}%
\par%
\entry{addendum}{/əˈdɛndəm/}{অভিযোজ্য বস্তু}{ \textsf{noun} \textbf{1} An item of additional material added at the end of a book or other publication. {\fontspec{Times New Roman}◇} \textit{} \textbf{2} The radial distance from the pitch circle of a cogwheel or wormwheel to the crests of the teeth or ridges. {\fontspec{Times New Roman}◇} \textit{}}%
\par%
\entry{adequate}{/ˈadɪkwət/}{পর্যাপ্ত}{ \textsf{adjective} \textbf{1} Satisfactory or acceptable in quality or quantity. {\fontspec{Times New Roman}◇} \textit{this office is perfectly adequate for my needs}}%
\par%
\entry{adhere}{/ədˈhɪə/}{মেনে চলে}{ \textsf{verb} \textbf{1} Stick fast to (a surface or substance) {\fontspec{Times New Roman}◇} \textit{paint won't adhere well to a greasy surface} \textbf{2} Believe in and follow the practices of. {\fontspec{Times New Roman}◇} \textit{I do not adhere to any organized religion}}%
\par%
\entry{adjourn}{/əˈdʒəːn/}{স্থগিত রাখা}{ \textsf{verb} \textbf{1} Break off (a meeting, legal case, or game) with the intention of resuming it later. {\fontspec{Times New Roman}◇} \textit{the meeting was adjourned until December 4}}%
\par%
\entry{admit}{/ədˈmɪt/}{সত্য বলিয়া স্বীকার করা}{ \textsf{verb} \textbf{1} Confess to be true or to be the case. {\fontspec{Times New Roman}◇} \textit{the Home Office finally admitted that several prisoners had been injured} \textbf{2} Allow (someone) to enter a place. {\fontspec{Times New Roman}◇} \textit{old-age pensioners are admitted free to the museum} \textbf{3} Accept as valid. {\fontspec{Times New Roman}◇} \textit{the courts can refuse to admit police evidence which has been illegally obtained} \textbf{4} Allow the possibility of. {\fontspec{Times New Roman}◇} \textit{the need to inform him was too urgent to admit of further delay}}%
\par%
\entry{aedes}{/eɪˈiːdiːz/}{এডিস; মশা বিশেষ}{ \textsf{noun} \textbf{1} A large and widespread genus of small mosquitoes (family Culicidae) including several vectors of human disease, notably Aedes aegypti, the principal carrier of yellow fever. Also (in form aedes): a mosquito of this genus (more fully "aedes mosquito"). {\fontspec{Times New Roman}◇} \textit{}}%
\par%
\entry{IED}{}{আইইডি}{ \textsf{noun} \textbf{1} A simple bomb made and used by unofficial or unauthorized forces. {\fontspec{Times New Roman}◇} \textit{}}%
\par%
\entry{mediterranean}{/ˌmɛdɪtəˈreɪnɪən/}{ভূমধ্য}{ \textsf{adjective} \textbf{1} Of or characteristic of the Mediterranean Sea, the countries bordering it, or their inhabitants. {\fontspec{Times New Roman}◇} \textit{a leisurely Mediterranean cruise} \textsf{noun} \textbf{1} The Mediterranean Sea or the countries bordering it. {\fontspec{Times New Roman}◇} \textit{a permanent American naval presence in the Mediterranean} \textbf{2} A native of a Mediterranean country. {\fontspec{Times New Roman}◇} \textit{an admiring audience of Mediterraneans}}%
\par%
\end{multicols}%
\end{document}