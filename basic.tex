\documentclass[10pt,a4paper,twoside]{book}%
\usepackage[T1]{fontenc}%
\usepackage[utf8]{inputenc}%
\usepackage{lmodern}%
\usepackage{textcomp}%
\usepackage{lastpage}%
\usepackage[top=2.3cm,bottom=2.0cm,left=2.5cm,right=2.0cm,columnsep=27pt]{geometry}%
\usepackage{palatino}%
\usepackage{microtype}%
\usepackage{multicol}%
\usepackage{fontspec}%
\usepackage{enumitem}%
\usepackage{graphicx}%
\usepackage{PTSerif}%
\usepackage[bf, sf, center]{titlesec}%
\usepackage{fancyhdr}%
%
\usepackage[svgnames]{xcolor}%
\setmainfont[Numbers=OldStyle]{TeX Gyre Pagella}%
\fancyhead[L]{\textsf{\rightmark}}%
\fancyhead[R]{\textsf{\leftmark}}%
\renewcommand{\headrulewidth}{1.4pt}%
\fancyfoot[C]{\textbf{\textsf{\thepage}}}%
\renewcommand{\footrulewidth}{1.4pt}%
\pagestyle{fancy}%
\newcommand{\entry}[7]{\textbf{#1}\markboth{#1}{#1}\ {{\fontspec{Doulos SIL} #2}}\  {{\fontspec{Kalpurush} #3}}\ {#4}\ {#5}\ {\textit{#6}}\ {#7}}%
\newcommand{\colorBulletS}[1]{\colorbox[RGB]{171,171,171}{\makebox(11,2){\textcolor{white}{{\tiny \textbf{#1}}}}}}%
\newcommand{\colorBullet}[1]{\colorbox[RGB]{171,171,171}{\makebox(22, 1){\textcolor{white}{{\tiny \textbf{#1}}}}}}%

\newcommand*{\rotrt}[1]{\rotatebox{90}{#1}} % Command to rotate right 90 degrees
\newcommand*{\rotlft}[1]{\rotatebox{-90}{#1}} % Command to rotate left 90 degrees

\title{\
	\def\CP{\textit{\Huge The Dictionary of Personal Words}} \	
	\settowidth{\unitlength}{\CP} \
	{\color{LightGoldenrod}\resizebox*{\unitlength}{\baselineskip}{\rotrt{$\}$}}} \\[\baselineskip]\
	\textcolor{Sienna}{\CP} \\[\baselineskip] \
	{\color{RosyBrown}\Large AN ILLUSTRATED COLLECTION} \\ \
	{\color{LightGoldenrod}\resizebox*{\unitlength}{\baselineskip}{\rotlft{$\}$}}} \
	}

\author{\
	\Large\textbf{Arafat Hasan} \
	}

\date{}
%
%
\begin{document}%
\normalsize%
\setlength{\parindent}{-0.7em}%
\maketitle%
\section*{A}%
\begin{multicols}{2}%
\entry{abandon}{/əˈband(ə)n/}{বর্জন করা}{\small{\textsf{\textit{noun, verb}}} \\{\fontspec{DejaVu Sans}▪ }\textsf{\textit{noun}}\\ \textbf{1} Complete lack of inhibition or restraint. {\fontspec{DejaVu Sans}◇} \textit{she sings and sways with total abandon} \colorBulletS{SYN} uninhibitedness, recklessness, lack of restraint, lack of inhibition, unruliness, wildness, impulsiveness, impetuosity, immoderation, wantonness \\{\fontspec{DejaVu Sans}▪ }\textsf{\textit{verb}}\\ \textbf{1} Cease to support or look after (someone); desert. {\fontspec{DejaVu Sans}◇} \textit{her natural mother had abandoned her at an early age} \colorBulletS{SYN} desert, leave, leave high and dry, turn one's back on, cast aside, break with, break up with \textbf{2} Give up completely (a practice or a course of action) {\fontspec{DejaVu Sans}◇} \textit{he had clearly abandoned all pretence of trying to succeed} \colorBulletS{SYN} renounce, relinquish, dispense with, forswear, disclaim, disown, disavow, discard, wash one's hands of \textbf{3} Allow oneself to indulge in (a desire or impulse) {\fontspec{DejaVu Sans}◇} \textit{they abandoned themselves to despair} \colorBulletS{SYN} indulge in, give way to, give oneself up to, yield to, lose oneself in, lose oneself to}{}{}{ \colorBullet{ORIGIN} Late Middle English from Old French abandoner, from a{-} (from Latin ad ‘to, at’) + bandon ‘control’ (related to ban). The original sense was ‘bring under control’, later ‘give in to the control of, surrender to’ (abandon (sense 3 of the verb)).}%
\par%
\entry{abduct}{/əbˈdʌkt/}{অপহরণ করা}{ \textsf{\textit{verb}}\ \textbf{1} Take (someone) away illegally by force or deception; kidnap. {\fontspec{DejaVu Sans}◇} \textit{the millionaire who disappeared may have been abducted} \colorBulletS{SYN} abduct, carry off, capture, seize, snatch, hold to ransom, take as hostage, hijack \textbf{2} (of a muscle) move (a limb or part) away from the midline of the body or from another part. {\fontspec{DejaVu Sans}◇} \textit{the posterior rectus muscle, which abducts the eye}}{}{}{ \colorBullet{ORIGIN} Early 17th century from Latin abduct{-} ‘led away’, from the verb abducere, from ab{-} ‘away, from’ + ducere ‘to lead’.}%
\par%
\entry{abductor}{/əbˈdʌktə/}{অপহরণকারী}{ \textsf{\textit{noun}}\ \textbf{1} A person who abducts another person. {\fontspec{DejaVu Sans}◇} \textit{she endured a two{-}hour ordeal at the hands of her abductors} \textbf{2}  {\fontspec{DejaVu Sans}◇} \textit{}}{}{Police rescued an abducted boy of comilla district and arrested the abductor from haji eidgah math area at dimla upazila in nilphamari on saturday, police sources said.}{ \colorBullet{ORIGIN} Early 17th century (as a term in anatomy): modern Latin (see abduct).}%
\par%
\entry{ablaze}{/əˈbleɪz/}{বহ্নিমান}{ \textsf{\textit{adjective}}\ \textbf{1} Burning fiercely. {\fontspec{DejaVu Sans}◇} \textit{his clothes were ablaze} \colorBulletS{SYN} alight, aflame, on fire, in flames, flaming, burning, blazing, raging, fiery, lit, lighted, ignited}{}{A housewife succumbed to her injuries today four days after she set herself ablaze as she was tortured by her husband allegedly for dowry. }{}%
\par%
\entry{abound}{/əˈbaʊnd/}{উড়া}{ \textsf{\textit{verb}}\ \textbf{1} Exist in large numbers or amounts. {\fontspec{DejaVu Sans}◇} \textit{rumours of a further scandal abound} \colorBulletS{SYN} be plentiful, be abundant, be numerous, proliferate, superabound, thrive, flourish, be thick on the ground}{}{Illegally modified vehicles abound}{ \colorBullet{ORIGIN} Middle English (in the sense ‘overflow, be abundant’): from Old French abunder, from Latin abundare ‘overflow’, from ab{-} ‘from’ + undare ‘surge’ (from unda ‘a wave’).}%
\par%
\entry{absorb}{/əbˈzɔːb/}{শোষণ করা}{ \textsf{\textit{verb}}\ \textbf{1} Take in or soak up (energy or a liquid or other substance) by chemical or physical action. {\fontspec{DejaVu Sans}◇} \textit{buildings can be designed to absorb and retain heat} \colorBulletS{SYN} soak up, suck up, draw in, draw up, take in, take up, blot up, mop up, sponge up, sop up \textbf{2} Take up the attention of (someone); interest greatly. {\fontspec{DejaVu Sans}◇} \textit{she sat in an armchair, absorbed in a book} \colorBulletS{SYN} preoccupy, engross, captivate, occupy, engage}{}{}{ \colorBullet{ORIGIN} Late Middle English from Latin absorbere, from ab{-} ‘from’ + sorbere ‘suck in’.}%
\par%
\entry{absurd}{/əbˈsəːd/}{কিম্ভুতকিমাকার}{\small{\textsf{\textit{adjective, noun}}} \\{\fontspec{DejaVu Sans}▪ }\textsf{\textit{adjective}}\\ \textbf{1} Wildly unreasonable, illogical, or inappropriate. {\fontspec{DejaVu Sans}◇} \textit{the allegations are patently absurd} \colorBulletS{SYN} preposterous, ridiculous, ludicrous, farcical, laughable, risible \\{\fontspec{DejaVu Sans}▪ }\textsf{\textit{noun}}\\ \textbf{1} An absurd state of affairs. {\fontspec{DejaVu Sans}◇} \textit{the incidents that followed bordered on the absurd}}{}{Don't be absurd}{ \colorBullet{ORIGIN} Mid 16th century from Latin absurdus ‘out of tune’, hence ‘irrational’; related to surdus ‘deaf, dull’.}%
\par%
\entry{abundant}{/əˈbʌnd(ə)nt/}{প্রচুর}{ \textsf{\textit{adjective}}\ \textbf{1} Existing or available in large quantities; plentiful. {\fontspec{DejaVu Sans}◇} \textit{there was abundant evidence to support the theory} \colorBulletS{SYN} plentiful, copious, ample, profuse, rich, lavish, liberal, generous, bountiful, large, huge, great, bumper, overflowing, superabundant, infinite, inexhaustible, opulent, prolific, teeming}{}{}{ \colorBullet{ORIGIN} Late Middle English from Latin abundant{-} ‘abounding’, from the verb abundare (see abound).}%
\par%
\entry{abundantly}{/əˈbʌnd(ə)ntli/}{প্রচুর পরিমাণে}{ \textsf{\textit{adverb}}\ \textbf{1} In large quantities; plentifully. {\fontspec{DejaVu Sans}◇} \textit{the plant grows abundantly in the wild} \colorBulletS{SYN} copiously, plentifully, amply, profusely, exuberantly, prolifically, luxuriantly, in profusion, in abundance, in great quantity, in large quantities, in plenty, aplenty, in huge numbers, freely, extensively, everywhere, all over the place}{}{}{}%
\par%
\entry{abysmal}{/əˈbɪzm(ə)l/}{অতল; ভয়ঙ্কর}{ \textsf{\textit{adjective}}\ \textbf{1} Extremely bad; appalling. {\fontspec{DejaVu Sans}◇} \textit{the quality of her work is abysmal} \colorBulletS{SYN} very bad, dreadful, awful, terrible, frightful, atrocious, disgraceful, deplorable, shameful, woeful, hopeless, lamentable, laughable, substandard, poor, inadequate, inferior, unsatisfactory \textbf{2} Very deep. {\fontspec{DejaVu Sans}◇} \textit{waterfalls that plunge into abysmal depths} \colorBulletS{SYN} profound, extreme, utter, complete, thorough, deep, endless, immeasurable, boundless, incalculable, unfathomable, bottomless}{}{"I think over the last few months the behaviour has been abysmal in international cricket," arthur, who is currently pakistan's head coach, said.}{ \colorBullet{ORIGIN} Mid 17th century (used literally as in abysmal (sense 2)): from abysm. abysmal (sense 1) dates from the early 19th century.}%
\par%
\entry{accomplice}{/əˈkʌmplɪs/}{যোগদানকারী}{ \textsf{\textit{noun}}\ \textbf{1} A person who helps another commit a crime. {\fontspec{DejaVu Sans}◇} \textit{an accomplice in the murder} \colorBulletS{SYN} abetter, accessory, partner in crime, associate, confederate, collaborator, fellow conspirator, co{-}conspirator}{}{}{ \colorBullet{ORIGIN} Mid 16th century alteration (probably by association with accompany) of Middle English complice ‘an associate’, via Old French from late Latin complex, complic{-} ‘allied’, from com{-} ‘together’ + the root of plicare ‘to fold’.}%
\par%
\entry{accomplish}{/əˈkʌmplɪʃ/}{সাধা}{ \textsf{\textit{verb}}\ \textbf{1} Achieve or complete successfully. {\fontspec{DejaVu Sans}◇} \textit{the planes accomplished their mission} \colorBulletS{SYN} fulfil, achieve, succeed in, realize, attain, manage, bring about, bring off, carry out, carry off, carry through, execute, conduct, effect, fix, engineer, perform, do, perpetrate, discharge, complete, finish, consummate, conclude}{}{}{ \colorBullet{ORIGIN} Late Middle English from Old French acompliss{-}, lengthened stem of acomplir, based on Latin ad{-} ‘to’ + complere ‘to complete’.}%
\par%
\entry{accord}{/əˈkɔːd/}{চুক্তি}{\small{\textsf{\textit{noun, verb}}} \\{\fontspec{DejaVu Sans}▪ }\textsf{\textit{noun}}\\ \textbf{1} An official agreement or treaty. {\fontspec{DejaVu Sans}◇} \textit{opposition groups refused to sign the accord} \colorBulletS{SYN} pact, treaty, agreement, settlement, deal, entente, concordat, concord, protocol, compact, contract, convention \\{\fontspec{DejaVu Sans}▪ }\textsf{\textit{verb}}\\ \textbf{1} Give or grant someone (power, status, or recognition) {\fontspec{DejaVu Sans}◇} \textit{the powers accorded to the head of state} \colorBulletS{SYN} give, grant, tender, present, award, hand, vouchsafe, concede, yield, cede \textbf{2} (of a concept or fact) be harmonious or consistent with. {\fontspec{DejaVu Sans}◇} \textit{his views accorded well with those of Merivale} \colorBulletS{SYN} correspond, agree, tally, match up, concur, coincide, be in agreement, be consistent, equate, harmonize, be in harmony, be compatible, be consonant, be congruous, be in tune, dovetail, correlate}{}{We have an accord}{ \colorBullet{ORIGIN} Old English, from Old French acorder ‘reconcile, be of one mind’, from Latin ad{-} ‘to’ + cor, cord{-} ‘heart’; influenced by concord.}%
\par%
\entry{account}{/əˈkaʊnt/}{হিসাব}{\small{\textsf{\textit{noun, verb}}} \\{\fontspec{DejaVu Sans}▪ }\textsf{\textit{noun}}\\ \textbf{1} A report or description of an event or experience. {\fontspec{DejaVu Sans}◇} \textit{a detailed account of what has been achieved} \colorBulletS{SYN} description, report, version, story, narration, narrative, statement, news, explanation, exposition, interpretation, communiqué, recital, rendition, sketch, delineation, portrayal, tale \textbf{2} A record or statement of financial expenditure and receipts relating to a particular period or purpose. {\fontspec{DejaVu Sans}◇} \textit{the barman was doing his accounts} \colorBulletS{SYN} financial record, book, ledger, journal, balance sheet, financial statement, results \textbf{3} An arrangement by which a body holds funds on behalf of a client or supplies goods or services to them on credit. {\fontspec{DejaVu Sans}◇} \textit{a bank account} \colorBulletS{SYN} bank account \textbf{4} An arrangement by which a user is given personalized access to a computer, website, or application, typically by entering a username and password. {\fontspec{DejaVu Sans}◇} \textit{we've reset your password to prevent others from accessing your account} \textbf{5} Importance. {\fontspec{DejaVu Sans}◇} \textit{money was of no account to her} \colorBulletS{SYN} importance, import, significance, consequence, moment, momentousness, substance, note, mark, prominence, value, weightiness, weight, concern, interest, gravity, seriousness \\{\fontspec{DejaVu Sans}▪ }\textsf{\textit{verb}}\\ \textbf{1} Consider or regard in a specified way. {\fontspec{DejaVu Sans}◇} \textit{her visit could not be accounted a success} \colorBulletS{SYN} consider, regard as, reckon, hold to be, think, think of as, look on as, view as, see as, take for, judge, adjudge, count, deem, rate, gauge, interpret as \textbf{2} Give or receive an account for money received. {\fontspec{DejaVu Sans}◇} \textit{after 1292 he accounted to the Westminster exchequer}}{}{}{ \colorBullet{ORIGIN} Middle English (in the sense ‘counting’, ‘to count’): from Old French acont (noun), aconter (verb), based on conter ‘to count’.}%
\par%
\entry{accuse}{/əˈkjuːz/}{অভিযুক্ত করা}{ \textsf{\textit{verb}}\ \textbf{1} Charge (someone) with an offence or crime. {\fontspec{DejaVu Sans}◇} \textit{he was accused of murdering his wife's lover} \colorBulletS{SYN} charge with, indict for, arraign for, take to court for, put on trial for, bring to trial for, prosecute for}{}{}{ \colorBullet{ORIGIN} Middle English from Old French acuser, from Latin accusare ‘call to account’, from ad{-} ‘towards’ + causa ‘reason, motive, lawsuit’.}%
\par%
\entry{accustom}{/əˈkʌstəm/}{অভ্যস্ত করা}{ \textsf{\textit{verb}}\ \textbf{1} Make someone or something accept (something) as normal or usual. {\fontspec{DejaVu Sans}◇} \textit{I accustomed my eyes to the lenses} \colorBulletS{SYN} adapt, adjust, acclimatize, attune, habituate, accommodate, assimilate, acculturate, inure, harden, condition, reconcile, become resigned, resign}{}{}{ \colorBullet{ORIGIN} Late Middle English from Old French acostumer, from a{-} (from Latin ad ‘to, at’) + costume ‘custom’.}%
\par%
\entry{ace}{/eɪs/}{টেক্কা}{\small{\textsf{\textit{adjective, noun, verb}}} \\{\fontspec{DejaVu Sans}▪ }\textsf{\textit{adjective}}\\ \textbf{1} Very good. {\fontspec{DejaVu Sans}◇} \textit{an ace swimmer} \colorBulletS{SYN} excellent, very good, first{-}rate, first{-}class, marvellous, wonderful, magnificent, outstanding, superlative, formidable, virtuoso, masterly, expert, champion, fine, consummate, skilful, adept \\{\fontspec{DejaVu Sans}▪ }\textsf{\textit{noun}}\\ \textbf{1} A playing card with a single spot on it, ranked as the highest card in its suit in most card games. {\fontspec{DejaVu Sans}◇} \textit{the ace of diamonds} \textbf{2} A person who excels at a particular sport or other activity. {\fontspec{DejaVu Sans}◇} \textit{a motorcycle ace} \colorBulletS{SYN} expert, master, genius, virtuoso, maestro, professional, adept, past master, doyen, champion, star, winner \textbf{3} (in tennis and similar games) a service that an opponent is unable to return and thus wins a point. {\fontspec{DejaVu Sans}◇} \textit{Nadal banged down eight aces in the set} \\{\fontspec{DejaVu Sans}▪ }\textsf{\textit{verb}}\\ \textbf{1} (in tennis and similar games) serve an ace against (an opponent) {\fontspec{DejaVu Sans}◇} \textit{he can ace opponents with serves of no more than 62 mph} \textbf{2} Achieve high marks in (a test or exam) {\fontspec{DejaVu Sans}◇} \textit{I aced my grammar test}}{}{}{ \colorBullet{ORIGIN} Middle English (denoting the ‘one’ on dice): via Old French from Latin as ‘unity, a unit’.}%
\par%
\entry{ace}{/eɪs/}{টেক্কা}{\small{\textsf{\textit{adjective, noun}}} \\{\fontspec{DejaVu Sans}▪ }\textsf{\textit{adjective}}\\ \textbf{1} (of a person) having no sexual feelings or desires; asexual. {\fontspec{DejaVu Sans}◇} \textit{I didn't realize that I was ace for a long time} \\{\fontspec{DejaVu Sans}▪ }\textsf{\textit{noun}}\\ \textbf{1} A person who has no sexual feelings or desires. {\fontspec{DejaVu Sans}◇} \textit{both asexual, they have managed to connect with other aces offline}}{}{}{ \colorBullet{ORIGIN} Early 21st century abbreviation of asexual, with alteration of spelling on the model of ace.}%
\par%
\entry{ache}{/eɪk/}{ব্যাথা}{\small{\textsf{\textit{noun, verb}}} \\{\fontspec{DejaVu Sans}▪ }\textsf{\textit{noun}}\\ \textbf{1} A continuous or prolonged dull pain in a part of one's body. {\fontspec{DejaVu Sans}◇} \textit{the ache in her head worsened} \colorBulletS{SYN} pain, dull pain, pang, twinge, throb \\{\fontspec{DejaVu Sans}▪ }\textsf{\textit{verb}}\\ \textbf{1} Suffer from a continuous dull pain. {\fontspec{DejaVu Sans}◇} \textit{my legs ached from the previous day's exercise} \colorBulletS{SYN} painful, achy, sore, stiff, hurt, tender, uncomfortable, troublesome}{}{}{ \colorBullet{ORIGIN} Old English æce (noun), acan (verb). In Middle English and early modern English the noun was spelled atche and rhymed with ‘batch’ and the verb was spelled and pronounced as it is today. The noun began to be pronounced like the verb around 1700. The modern spelling is largely due to Dr Johnson, who mistakenly assumed its derivation to be from Greek akhos ‘pain’.}%
\par%
\entry{acquire}{/əˈkwʌɪə/}{অর্জন}{ \textsf{\textit{verb}}\ \textbf{1} Buy or obtain (an asset or object) for oneself. {\fontspec{DejaVu Sans}◇} \textit{I managed to acquire all the books I needed} \colorBulletS{SYN} obtain, come by, come to have, get, receive, gain, earn, win, come into, come in for, take possession of, take receipt of, be given \textbf{2} Learn or develop (a skill, habit, or quality) {\fontspec{DejaVu Sans}◇} \textit{you must acquire the rudiments of Greek} \colorBulletS{SYN} learn, learn thoroughly, become proficient in, know inside out, know backwards, become expert in, acquire, pick up, grasp, understand}{}{}{ \colorBullet{ORIGIN} Late Middle English acquere, from Old French aquerre, based on Latin acquirere ‘get in addition’, from ad{-} ‘to’ + quaerere ‘seek’. The English spelling was modified (c1600) by association with the Latin word.}%
\par%
\entry{acquisition}{/ˌakwɪˈzɪʃ(ə)n/}{অর্জন; অধিগ্রহণ}{ \textsf{\textit{noun}}\ \textbf{1} An asset or object bought or obtained, typically by a library or museum. {\fontspec{DejaVu Sans}◇} \textit{the legacy will be used for new acquisitions} \colorBulletS{SYN} purchase, accession, addition, asset \textbf{2} The learning or developing of a skill, habit, or quality. {\fontspec{DejaVu Sans}◇} \textit{the acquisition of management skills} \colorBulletS{SYN} assumption, assuming, taking on, acquiring, acquisition, affecting, affectation, espousal, advocacy, promotion, appropriation, arrogation}{}{Land acquisition}{ \colorBullet{ORIGIN} Late Middle English (in the sense ‘act of acquiring something’): from Latin acquisitio(n{-}), from the verb acquirere (see acquire).}%
\par%
\entry{acting}{/ˈaktɪŋ/}{অভিনয়}{\small{\textsf{\textit{adjective, noun}}} \\{\fontspec{DejaVu Sans}▪ }\textsf{\textit{adjective}}\\ \textbf{1} Temporarily doing the duties of another person. {\fontspec{DejaVu Sans}◇} \textit{the acting supervisor} \colorBulletS{SYN} substitute, deputy, reserve, fill{-}in, stand{-}in, caretaker \\{\fontspec{DejaVu Sans}▪ }\textsf{\textit{noun}}\\ \textbf{1} The art or occupation of performing fictional roles in plays, films, or television. {\fontspec{DejaVu Sans}◇} \textit{she studied acting in New York} \colorBulletS{SYN} drama, the theatre, the stage, the performing arts, dramatic art, dramatics, dramaturgy, stagecraft, theatricals, theatrics, the thespian art, show business}{}{}{}%
\par%
\entry{adamant}{/ˈadəm(ə)nt/}{হীরক}{\small{\textsf{\textit{adjective, noun}}} \\{\fontspec{DejaVu Sans}▪ }\textsf{\textit{adjective}}\\ \textbf{1} Refusing to be persuaded or to change one's mind. {\fontspec{DejaVu Sans}◇} \textit{he is adamant that he is not going to resign} \colorBulletS{SYN} unshakeable, immovable, inflexible, unwavering, uncompromising, resolute, resolved, determined, firm, rigid, steadfast \\{\fontspec{DejaVu Sans}▪ }\textsf{\textit{noun}}\\ \textbf{1} A legendary rock or mineral to which many properties were attributed, formerly associated with diamond or lodestone. {\fontspec{DejaVu Sans}◇} \textit{As for the magical metal, asiceton, it sounds like adamant.}}{}{}{ \colorBullet{ORIGIN} Old English (as a noun), from Old French adamaunt{-}, via Latin from Greek adamas, adamant{-}, ‘untameable, invincible’ (later used to denote the hardest metal or stone, hence diamond), from a{-} ‘not’ + daman ‘to tame’. The phrase to be adamant dates from the 1930s, although adjectival use had been implied in such collocations as ‘an adamant heart’ since the 16th century.}%
\par%
\entry{adaptation}{/adəpˈteɪʃ(ə)n/}{অভিযোজন}{ \textsf{\textit{noun}}\ \textbf{1} The action or process of adapting or being adapted. {\fontspec{DejaVu Sans}◇} \textit{the adaptation of teaching strategy to meet students' needs} \colorBulletS{SYN} converting, conversion, alteration, modification, adjustment, changing, transformation}{}{}{ \colorBullet{ORIGIN} Early 17th century from French, from late Latin adaptatio(n{-}), from Latin adaptare (see adapt).}%
\par%
\entry{addendum}{/əˈdɛndəm/}{অভিযোজ্য বস্তু}{ \textsf{\textit{noun}}\ \textbf{1} An item of additional material added at the end of a book or other publication. {\fontspec{DejaVu Sans}◇} \textit{} \colorBulletS{SYN} appendix, codicil, postscript, afterword, tailpiece, rider, coda, supplement, accompaniment \textbf{2} The radial distance from the pitch circle of a cogwheel or wormwheel to the crests of the teeth or ridges. {\fontspec{DejaVu Sans}◇} \textit{}}{}{}{ \colorBullet{ORIGIN} Late 17th century Latin, ‘that which is to be added’, gerundive of addere (see add).}%
\par%
\entry{adequate}{/ˈadɪkwət/}{পর্যাপ্ত}{ \textsf{\textit{adjective}}\ \textbf{1} Satisfactory or acceptable in quality or quantity. {\fontspec{DejaVu Sans}◇} \textit{this office is perfectly adequate for my needs} \colorBulletS{SYN} sufficient, enough, ample, requisite, apposite, appropriate, suitable}{}{}{ \colorBullet{ORIGIN} Early 17th century from Latin adaequatus ‘made equal to’, past participle of the verb adaequare, from ad{-} ‘to’ + aequus ‘equal’.}%
\par%
\entry{adhere}{/ədˈhɪə/}{মেনে চলে}{ \textsf{\textit{verb}}\ \textbf{1} Stick fast to (a surface or substance) {\fontspec{DejaVu Sans}◇} \textit{paint won't adhere well to a greasy surface} \colorBulletS{SYN} stick, stick fast, cling, hold fast, cohere, bond, attach \textbf{2} Believe in and follow the practices of. {\fontspec{DejaVu Sans}◇} \textit{I do not adhere to any organized religion} \colorBulletS{SYN} get involved with, take up with, join up with, go around with, string along with, become friendly with, make friends with, strike up a friendship with, start seeing, make the acquaintance of}{}{}{ \colorBullet{ORIGIN} Late 15th century from Latin adhaerere, from ad{-} ‘to’ + haerere ‘to stick’.}%
\par%
\entry{adjourn}{/əˈdʒəːn/}{স্থগিত রাখা}{ \textsf{\textit{verb}}\ \textbf{1} Break off (a meeting, legal case, or game) with the intention of resuming it later. {\fontspec{DejaVu Sans}◇} \textit{the meeting was adjourned until December 4} \colorBulletS{SYN} end, bring to an end, come to an end, conclude, finish, terminate, wind up, break off, halt, call a halt to, discontinue, dissolve}{}{}{ \colorBullet{ORIGIN} Middle English (in the sense ‘summon someone to appear on a particular day’): from Old French ajorner, from the phrase a jorn (nome) ‘to an (appointed) day’.}%
\par%
\entry{admit}{/ədˈmɪt/}{সত্য বলিয়া স্বীকার করা}{ \textsf{\textit{verb}}\ \textbf{1} Confess to be true or to be the case. {\fontspec{DejaVu Sans}◇} \textit{the Home Office finally admitted that several prisoners had been injured} \colorBulletS{SYN} acknowledge, confess, reveal, make known, disclose, divulge, make public, avow, declare, profess, own up to, make a clean breast of, bring into the open, bring to light, give away, blurt out, leak \textbf{2} Allow (someone) to enter a place. {\fontspec{DejaVu Sans}◇} \textit{old{-}age pensioners are admitted free to the museum} \colorBulletS{SYN} let in, allow entry, permit entry, grant entrance to, give right of entry to, give access to, give admission to, accept, take in, usher in, show in, receive, welcome \textbf{3} Accept as valid. {\fontspec{DejaVu Sans}◇} \textit{the courts can refuse to admit police evidence which has been illegally obtained} \textbf{4} Allow the possibility of. {\fontspec{DejaVu Sans}◇} \textit{the need to inform him was too urgent to admit of further delay} \colorBulletS{SYN} allow, permit, authorize, sanction, condone, indulge, agree to, accede to, approve of}{}{}{ \colorBullet{ORIGIN} Late Middle English from Latin admittere, from ad{-} ‘to’ + mittere ‘send’.}%
\par%
\entry{adolescence}{/adəˈlɛs(ə)ns/}{কৈশোর}{ \textsf{\textit{noun}}\ \textbf{1} The period following the onset of puberty during which a young person develops from a child into an adult. {\fontspec{DejaVu Sans}◇} \textit{Mary spent her childhood and adolescence in Europe} \colorBulletS{SYN} teenage years, teens, youth, young adulthood, young days, early life}{}{}{ \colorBullet{ORIGIN} Late Middle English from French, from Latin adolescentia, from adolescere ‘grow to maturity’ (see adolescent).}%
\par%
\entry{adulterant}{/əˈdʌlt(ə)r(ə)nt/}{ভেজাল}{\small{\textsf{\textit{adjective, noun}}} \\{\fontspec{DejaVu Sans}▪ }\textsf{\textit{adjective}}\\ \textbf{1} Used in adulterating something. {\fontspec{DejaVu Sans}◇} \textit{They argued that because the bacteria is naturally occurring, it is not an "adulterant" substance subject to regulation by the government.} \\{\fontspec{DejaVu Sans}▪ }\textsf{\textit{noun}}\\ \textbf{1} A substance used to adulterate another. {\fontspec{DejaVu Sans}◇} \textit{} \colorBulletS{SYN} contaminant, adulterant, pollutant, foreign body}{}{}{ \colorBullet{ORIGIN} Mid 18th century from Latin adulterant{-} ‘corrupting’, from the verb adulterare (see adulterate).}%
\par%
\entry{adulteration}{/ədʌlt(ə)ˈreɪʃ(ə)n/}{ভেজাল দেয়া}{ \textsf{\textit{noun}}\ \textbf{1} The action of making something poorer in quality by the addition of another substance. {\fontspec{DejaVu Sans}◇} \textit{}}{}{}{ \colorBullet{ORIGIN} Early 16th century from Latin adulterat{-} ‘corrupted’, from the verb adulterare + {-}ion.}%
\par%
\entry{adverse}{/ˈadvəːs/}{প্রতিকূল}{ \textsf{\textit{adjective}}\ \textbf{1} Preventing success or development; harmful; unfavourable. {\fontspec{DejaVu Sans}◇} \textit{taxes are having an adverse effect on production} \colorBulletS{SYN} unfavourable, disadvantageous, inauspicious, unpropitious, unfortunate, unlucky, untimely, untoward}{}{Adverse impact}{ \colorBullet{ORIGIN} Late Middle English from Old French advers, from Latin adversus ‘against, opposite’, past participle of advertere, from ad{-} ‘to’ + vertere ‘to turn’. Compare with averse.}%
\par%
\entry{aedes}{/eɪˈiːdiːz/}{এডিস; মশা বিশেষ}{ \textsf{\textit{noun}}\ \textbf{1} A large and widespread genus of small mosquitoes (family Culicidae) including several vectors of human disease, notably Aedes aegypti, the principal carrier of yellow fever. Also (in form aedes): a mosquito of this genus (more fully "aedes mosquito"). {\fontspec{DejaVu Sans}◇} \textit{}}{}{}{ \colorBullet{ORIGIN} Mid 19th century. From scientific Latin Aedes from ancient Greek ἀηδής unpleasant, disagreeable from ἀ{-} + ἦδος delight, pleasure from the same Indo{-}European base as sweet.}%
\par%
\entry{aerial}{/ˈɛːrɪəl/}{বায়বীয়}{\small{\textsf{\textit{adjective, noun}}} \\{\fontspec{DejaVu Sans}▪ }\textsf{\textit{adjective}}\\ \textbf{1} Existing, happening, or operating in the air. {\fontspec{DejaVu Sans}◇} \textit{an aerial battle} \colorBulletS{SYN} raised, upraised, uplifted, lifted up, high up, aloft, aerial, overhead, hoisted \\{\fontspec{DejaVu Sans}▪ }\textsf{\textit{noun}}\\ \textbf{1} A rod, wire, or other structure by which signals are transmitted or received as part of a radio or television transmission or receiving system. {\fontspec{DejaVu Sans}◇} \textit{} \colorBulletS{SYN} flagpole, flagstaff, pole, post, rod, support, upright \textbf{2} A type of freestyle skiing in which the skier jumps from a ramp and carries out manoeuvres in the air. {\fontspec{DejaVu Sans}◇} \textit{}}{}{1. The dhaka city corporation (dcc) is now ready for aerial spraying of larvicide to combat mosquitoes. 2. The indian border security force yesterday said it has no plan to deploy unmanned aerial vehicles, popularly known as drone.}{ \colorBullet{ORIGIN} Late 16th century (in the sense ‘thin as air, imaginary’): via Latin aerius from Greek aerios (from aēr ‘air’) + {-}al.}%
\par%
\entry{aesthetic}{/iːsˈθɛtɪk/}{নান্দনিক}{\small{\textsf{\textit{adjective, noun}}} \\{\fontspec{DejaVu Sans}▪ }\textsf{\textit{adjective}}\\ \textbf{1} Concerned with beauty or the appreciation of beauty. {\fontspec{DejaVu Sans}◇} \textit{the pictures give great aesthetic pleasure} \\{\fontspec{DejaVu Sans}▪ }\textsf{\textit{noun}}\\ \textbf{1} A set of principles underlying the work of a particular artist or artistic movement. {\fontspec{DejaVu Sans}◇} \textit{the Cubist aesthetic}}{}{}{ \colorBullet{ORIGIN} Late 18th century (in the sense ‘relating to perception by the senses’): from Greek aisthētikos, from aisthēta ‘perceptible things’, from aisthesthai ‘perceive’. The sense ‘concerned with beauty’ was coined in German in the mid 18th century and adopted into English in the early 19th century, but its use was controversial until much later in the century.}%
\par%
\entry{aesthetically}{/iːsˈθɛtɪkli/}{নান্দনিক}{ \textsf{\textit{adverb}}\ \textbf{1} In a way that gives pleasure through beauty. {\fontspec{DejaVu Sans}◇} \textit{the buildings and gardens of the factory have been aesthetically designed and laid out}}{}{}{}%
\par%
\entry{affect}{/əˈfɛkt/}{প্রভাবিত}{ \textsf{\textit{verb}}\ \textbf{1} Have an effect on; make a difference to. {\fontspec{DejaVu Sans}◇} \textit{the dampness began to affect my health} \colorBulletS{SYN} affect, influence, exert influence on, act on, work on, condition, touch, interact with, have an impact on, impact on, take hold of, attack, infect, strike, strike at, hit}{}{}{ \colorBullet{ORIGIN} Late Middle English (in the sense ‘attack as a disease’): from French affecter or Latin affect{-} ‘influenced, affected’, from the verb afficere (see affect).}%
\par%
\entry{affect}{/əˈfɛkt/}{প্রভাবিত}{ \textsf{\textit{verb}}\ \textbf{1} Pretend to have or feel (something) {\fontspec{DejaVu Sans}◇} \textit{as usual I affected a supreme unconcern} \colorBulletS{SYN} pretend, feign, fake, counterfeit, sham, simulate, fabricate, give the appearance of, make a show of, make a pretence of, play at, go through the motions of}{}{}{ \colorBullet{ORIGIN} Late Middle English from French affecter or Latin affectare ‘aim at’, frequentative of afficere ‘work on, influence’, from ad{-} ‘at, to’ + facere ‘do’. The original sense was ‘like, love’, hence ‘(like to) use, assume, etc.’.}%
\par%
\entry{affect}{/ˈafɛkt/}{প্রভাবিত}{ \textsf{\textit{noun}}\ \textbf{1} Emotion or desire as influencing behaviour. {\fontspec{DejaVu Sans}◇} \textit{}}{}{}{ \colorBullet{ORIGIN} Late 19th century coined in German from Latin affectus ‘disposition’, from afficere ‘to influence’ (see affect).}%
\par%
\entry{affiliate}{/əˈfɪlɪeɪt/}{শাখা}{\small{\textsf{\textit{noun, verb}}} \\{\fontspec{DejaVu Sans}▪ }\textsf{\textit{noun}}\\ \textbf{1} A person or organization officially attached to a larger body. {\fontspec{DejaVu Sans}◇} \textit{the firm established links with American affiliates} \colorBulletS{SYN} office, bureau, agency \\{\fontspec{DejaVu Sans}▪ }\textsf{\textit{verb}}\\ \textbf{1} Officially attach or connect (a subsidiary group or a person) to an organization. {\fontspec{DejaVu Sans}◇} \textit{they are national associations affiliated to larger organizations} \colorBulletS{SYN} associate with, be in league with, unite with, combine with, join with, join up with, join forces with, ally with, form an alliance with, align with, amalgamate with, merge with, coalesce with, federate with, confederate with, form a federation with, form a confederation with, team up with, band together with, cooperate with}{}{}{ \colorBullet{ORIGIN} Mid 18th century from medieval Latin affiliat{-} ‘adopted as a son’, from the verb affiliare, from ad{-} ‘towards’ + filius ‘son’.}%
\par%
\entry{affiliation}{/əfɪlɪˈeɪʃ(ə)n/}{অন্তর্ভুক্তি}{ \textsf{\textit{noun}}\ \textbf{1} The state or process of affiliating or being affiliated. {\fontspec{DejaVu Sans}◇} \textit{the group has no affiliation to any preservation society} \colorBulletS{SYN} annexing, attaching, connecting, joining, bonding, uniting, combining, associating, aligning, allying, amalgamation, amalgamating, merging, incorporation, incorporating, integration, integrating, federating, federation, confederating, confederation, coupling, fusion}{}{}{ \colorBullet{ORIGIN} Late 18th century from French, from medieval Latin affiliatio(n{-}), from the verb affiliare (see affiliate).}%
\par%
\entry{affluent}{/ˈaflʊənt/}{ধনী}{\small{\textsf{\textit{adjective, noun}}} \\{\fontspec{DejaVu Sans}▪ }\textsf{\textit{adjective}}\\ \textbf{1} (especially of a group or area) having a great deal of money; wealthy. {\fontspec{DejaVu Sans}◇} \textit{the affluent societies of the western world} \colorBulletS{SYN} wealthy, rich, prosperous, opulent, well off, moneyed, cash rich, with deep pockets, well{-}to{-}do, comfortable \textbf{2} (of water) flowing freely or in great quantity. {\fontspec{DejaVu Sans}◇} \textit{He replied that the water was affluent and that they had not reviewed this in detail.} \\{\fontspec{DejaVu Sans}▪ }\textsf{\textit{noun}}\\ \textbf{1} A tributary stream. {\fontspec{DejaVu Sans}◇} \textit{}}{}{}{ \colorBullet{ORIGIN} Late Middle English (in affluent (sense 2 of the adjective)): via Old French from Latin affluent{-} ‘flowing towards, flowing freely’, from the verb affluere, from ad{-} ‘to’ + fluere ‘to flow’.}%
\par%
\entry{afire}{/əˈfʌɪə/}{আগুন}{ \textsf{\textit{adjective}}\ \textbf{1} On fire; burning. {\fontspec{DejaVu Sans}◇} \textit{the whole mill was afire} \colorBulletS{SYN} blazing, ablaze, burning, on fire, afire, in flames, aflame}{}{Set afire}{}%
\par%
\entry{aforementioned}{/əfɔːˈmɛnʃənd/}{উপরোক্ত}{ \textsf{\textit{adjective}}\ \textbf{1} Denoting a thing or person previously mentioned. {\fontspec{DejaVu Sans}◇} \textit{songs from the aforementioned album} \colorBulletS{SYN} foregoing, previous, prior, former, precursory, earlier, above, above{-}mentioned, aforementioned, above{-}stated, above{-}named, antecedent}{}{}{}%
\par%
\entry{aforethought}{}{পূর্বকল্পিত}{\small{\textsf{\textit{}}}}{}{}{}%
\par%
\entry{afraid}{/əˈfreɪd/}{}{ \textsf{\textit{adjective}}\ \textbf{1} Feeling fear or anxiety; frightened. {\fontspec{DejaVu Sans}◇} \textit{I'm afraid of dogs} \colorBulletS{SYN} frightened, scared, scared stiff, terrified, fearful, petrified, nervous, scared to death}{}{I'm afraid not}{ \colorBullet{ORIGIN} Middle English past participle of the obsolete verb affray, from Anglo{-}Norman French afrayer (see affray).}%
\par%
\entry{aftermath}{/ˈɑːftəmaθ/}{ভবিষ্যৎ ফল}{ \textsf{\textit{noun}}\ \textbf{1} The consequences or after{-}effects of a significant unpleasant event. {\fontspec{DejaVu Sans}◇} \textit{food prices soared in the aftermath of the drought} \colorBulletS{SYN} repercussions, after{-}effects, by{-}product, fallout, backwash, trail, wake, corollary \textbf{2} New grass growing after mowing or harvest. {\fontspec{DejaVu Sans}◇} \textit{}}{}{}{ \colorBullet{ORIGIN} Late 15th century (in aftermath (sense 2)): from after (as an adjective) + dialect math ‘mowing’, of Germanic origin; related to German Mahd.}%
\par%
\entry{aggravate}{/ˈaɡrəveɪt/}{বাড়া}{ \textsf{\textit{verb}}\ \textbf{1} Make (a problem, injury, or offence) worse or more serious. {\fontspec{DejaVu Sans}◇} \textit{military action would only aggravate the situation} \colorBulletS{SYN} worsen, make worse, exacerbate, inflame, compound \textbf{2} Annoy or exasperate. {\fontspec{DejaVu Sans}◇} \textit{} \colorBulletS{SYN} annoy, irritate, exasperate, anger, irk, vex, put out, nettle, provoke, incense, rile, infuriate, antagonize, get on someone's nerves, rub up the wrong way, make someone's blood boil, ruffle someone's feathers, ruffle, try someone's patience, make someone's hackles rise}{}{}{ \colorBullet{ORIGIN} Mid 16th century from Latin aggravat{-} ‘made heavy’, from the verb aggravare, from ad{-} (expressing increase) + gravis ‘heavy’.}%
\par%
\entry{aggregate}{/ˈaɡrɪɡət/}{দলা}{\small{\textsf{\textit{adjective, noun, verb}}} \\{\fontspec{DejaVu Sans}▪ }\textsf{\textit{adjective}}\\ \textbf{1} Formed or calculated by the combination of several separate elements; total. {\fontspec{DejaVu Sans}◇} \textit{the aggregate amount of grants made} \colorBulletS{SYN} total, combined, whole, gross, accumulated, added, entire, complete, full, comprehensive, overall, composite \\{\fontspec{DejaVu Sans}▪ }\textsf{\textit{noun}}\\ \textbf{1} A whole formed by combining several separate elements. {\fontspec{DejaVu Sans}◇} \textit{the council was an aggregate of three regional assemblies} \textbf{2} A material or structure formed from a mass of fragments or particles loosely compacted together. {\fontspec{DejaVu Sans}◇} \textit{the specimen is an aggregate of rock and mineral fragments} \colorBulletS{SYN} collection, mass, cluster, lump, clump, pile, heap, bundle, quantity \\{\fontspec{DejaVu Sans}▪ }\textsf{\textit{verb}}\\ \textbf{1} Form or group into a class or cluster. {\fontspec{DejaVu Sans}◇} \textit{socio{-}occupational groups aggregate men sharing similar kinds of occupation} \colorBulletS{SYN} combine, put, group, bunch, aggregate, unite, pool, mix, blend, merge, mass, join, fuse, conglomerate, coalesce, consolidate, collect, throw, consider together}{}{}{ \colorBullet{ORIGIN} Late Middle English from Latin aggregat{-} ‘herded together’, from the verb aggregare, from ad{-} ‘towards’ + grex, greg{-} ‘a flock’.}%
\par%
\entry{agitate}{/ˈadʒɪteɪt/}{উদ্বেগজনক}{ \textsf{\textit{verb}}\ \textbf{1} Make (someone) troubled or nervous. {\fontspec{DejaVu Sans}◇} \textit{the thought of questioning Toby agitated him extremely} \colorBulletS{SYN} upset, perturb, fluster, ruffle, disconcert, unnerve, disquiet, disturb, distress, unsettle, bother, concern, trouble, cause anxiety to, make anxious, alarm, work up, flurry, worry \textbf{2} Stir or disturb (something, especially a liquid) briskly. {\fontspec{DejaVu Sans}◇} \textit{agitate the water to disperse the oil} \colorBulletS{SYN} stir, whisk, beat, churn, shake, toss, blend, whip, whip up, fold, roil, jolt, disturb \textbf{3} Campaign to arouse public concern about an issue in the hope of prompting action. {\fontspec{DejaVu Sans}◇} \textit{they agitated for a reversal of the decision} \colorBulletS{SYN} campaign, strive, battle, fight, struggle, crusade, push, press}{}{}{ \colorBullet{ORIGIN} Late Middle English (in the sense ‘drive away’): from Latin agitat{-} ‘agitated, driven’, from agitare, frequentative of agere ‘do, drive’.}%
\par%
\entry{agitation}{/adʒɪˈteɪʃ(ə)n/}{চাগাড়}{ \textsf{\textit{noun}}\ \textbf{1} A state of anxiety or nervous excitement. {\fontspec{DejaVu Sans}◇} \textit{she was wringing her hands in agitation} \colorBulletS{SYN} anxiety, perturbation, disquiet, distress, concern, trouble, alarm, worry, upset \textbf{2} Brisk stirring or disturbance of a liquid. {\fontspec{DejaVu Sans}◇} \textit{the techniques mostly involve agitation by stirring} \colorBulletS{SYN} stirring, whisking, beating, churning, shaking, turbulence, tossing, blending, whipping, folding, rolling, jolting \textbf{3} The arousing of public concern about an issue and pressing for action on it. {\fontspec{DejaVu Sans}◇} \textit{widespread agitation for social reform} \colorBulletS{SYN} campaigning, striving, battling, fighting, struggling, crusading}{}{}{ \colorBullet{ORIGIN} Mid 16th century (in the sense ‘action, being active’): from Latin agitatio(n{-}), from the verb agitare (see agitate).}%
\par%
\entry{agitator}{/ˈadʒɪteɪtə/}{প্রচারক}{ \textsf{\textit{noun}}\ \textbf{1} A person who urges others to protest or rebel. {\fontspec{DejaVu Sans}◇} \textit{a political agitator} \colorBulletS{SYN} troublemaker, rabble{-}rouser, demagogue, soapbox orator, incendiary \textbf{2} An apparatus for stirring liquid. {\fontspec{DejaVu Sans}◇} \textit{}}{}{}{}%
\par%
\entry{agonize}{/ˈaɡənʌɪz/}{মানসিক যন্ত্রণাদায়ক}{ \textsf{\textit{verb}}\ \textbf{1} Undergo great mental anguish through worrying about something. {\fontspec{DejaVu Sans}◇} \textit{I didn't agonize over the problem} \colorBulletS{SYN} worry, fret, fuss, upset oneself, rack one's brains, wrestle with oneself, be worried, be anxious, feel uneasy, exercise oneself, brood, muse}{}{}{ \colorBullet{ORIGIN} Late 16th century from French agoniser or late Latin agonizare, from Greek agōnizesthai ‘contend’, from agōn ‘contest’.}%
\par%
\entry{agree to disagree}{}{To agree not to argue anymore about a difference of opinion }{\small{\textsf{\textit{}}}}{}{He likes golf and his wife likes tennis, so when it comes to sports, they have agreed to disagree.}{}%
\par%
\entry{aid}{/eɪd/}{সাহায্য}{\small{\textsf{\textit{noun, verb}}} \\{\fontspec{DejaVu Sans}▪ }\textsf{\textit{noun}}\\ \textbf{1} Help, typically of a practical nature. {\fontspec{DejaVu Sans}◇} \textit{he saw the pilot slumped in his cockpit and went to his aid} \colorBulletS{SYN} assistance, support \textbf{2} A grant of subsidy or tax to a king or queen. {\fontspec{DejaVu Sans}◇} \textit{} \\{\fontspec{DejaVu Sans}▪ }\textsf{\textit{verb}}\\ \textbf{1} Help or support (someone or something) in the achievement of something. {\fontspec{DejaVu Sans}◇} \textit{women were aided in childbirth by midwives} \colorBulletS{SYN} help, assist, abet, come to the aid of, give assistance to, lend a hand to, be of service to}{}{}{ \colorBullet{ORIGIN} Late Middle English from Old French aide (noun), aidier (verb), based on Latin adjuvare, from ad{-} ‘towards’ + juvare ‘to help’.}%
\par%
\entry{AID}{}{সাহায্য}{ \textsf{\textit{abbreviation}}\ \textbf{1} Artificial insemination by donor. {\fontspec{DejaVu Sans}◇} \textit{}}{}{}{}%
\par%
\entry{alibi}{/ˈalɪbʌɪ/}{অপরাধের অনুষ্ঠানকালে অন্যত্র থাকার অজুহাতে রেহাই পাইবার দাবি}{\small{\textsf{\textit{noun, verb}}} \\{\fontspec{DejaVu Sans}▪ }\textsf{\textit{noun}}\\ \textbf{1} A claim or piece of evidence that one was elsewhere when an act, typically a criminal one, is alleged to have taken place. {\fontspec{DejaVu Sans}◇} \textit{she has an alibi for the whole of yesterday evening} \\{\fontspec{DejaVu Sans}▪ }\textsf{\textit{verb}}\\ \textbf{1} Provide an alibi for. {\fontspec{DejaVu Sans}◇} \textit{her friend agreed to alibi her} \colorBulletS{SYN} cover for, give an alibi to, provide with an alibi, shield, protect}{}{}{ \colorBullet{ORIGIN} Late 17th century (as an adverb in the sense ‘elsewhere’): from Latin, ‘elsewhere’. The noun use dates from the late 18th century.}%
\par%
\entry{allegation}{/alɪˈɡeɪʃ(ə)n/}{অভিযোগ}{ \textsf{\textit{noun}}\ \textbf{1} A claim or assertion that someone has done something illegal or wrong, typically one made without proof. {\fontspec{DejaVu Sans}◇} \textit{he made allegations of corruption against the administration} \colorBulletS{SYN} claim, assertion, declaration, statement, proclamation, contention, argument, affirmation, averment, avowal, attestation, testimony, certification, evidence, witness, charge, accusation, suggestion, implication, hint, insinuation, indication, intimation, imputation, plea, pretence, profession}{}{}{ \colorBullet{ORIGIN} Late Middle English from Latin allegatio(n{-}), from allegare ‘allege’.}%
\par%
\entry{allege}{/əˈlɛdʒ/}{অভিযোগ করা}{ \textsf{\textit{verb}}\ \textbf{1} Claim or assert that someone has done something illegal or wrong, typically without proof. {\fontspec{DejaVu Sans}◇} \textit{he alleged that he had been assaulted} \colorBulletS{SYN} claim, assert, declare, state, proclaim, maintain, advance, contend, argue, affirm, aver, avow, attest, testify, swear, certify, give evidence, bear witness, charge, accuse, suggest, imply, hint, insinuate, indicate, intimate, impute, plead, pretend, profess}{}{}{ \colorBullet{ORIGIN} Middle English (in the sense ‘declare on oath’): from Old French esligier, based on Latin lis, lit{-} ‘lawsuit’; confused in sense with Latin allegare ‘allege’.}%
\par%
\entry{allegedly}{/əˈlɛdʒɪdli/}{অভিযোগে}{ \textsf{\textit{adverb}}\ \textbf{1} Used to convey that something is claimed to be the case or have taken place, although there is no proof. {\fontspec{DejaVu Sans}◇} \textit{he was allegedly a leading participant in the coup attempt} \colorBulletS{SYN} reportedly, supposedly, reputedly, purportedly, ostensibly, apparently, by all accounts, so the story goes, putatively, presumedly, presumably, assumedly, declaredly, avowedly}{}{}{}%
\par%
\entry{allegiance}{/əˈliːdʒ(ə)ns/}{আনুগত্য}{ \textsf{\textit{noun}}\ \textbf{1} Loyalty or commitment to a superior or to a group or cause. {\fontspec{DejaVu Sans}◇} \textit{those wishing to receive citizenship must swear allegiance to the republic} \colorBulletS{SYN} loyalty, faithfulness, fidelity, obedience, fealty, adherence, homage, devotion, bond}{}{}{ \colorBullet{ORIGIN} Late Middle English from Anglo{-}Norman French, variant of Old French ligeance, from lige, liege (see liege), perhaps by association with Anglo{-}Latin alligantia ‘alliance’.}%
\par%
\entry{alley}{/ˈali/}{সরু গলি}{ \textsf{\textit{noun}}\ \textbf{1} A narrow passageway between or behind buildings. {\fontspec{DejaVu Sans}◇} \textit{he took a short cut along an alley} \colorBulletS{SYN} passage, passageway, alleyway, back alley, backstreet, lane, path, pathway, walk}{}{}{ \colorBullet{ORIGIN} Late Middle English from Old French alee ‘walking or passage’, from aler ‘go’, from Latin ambulare ‘to walk’.}%
\par%
\entry{alley}{/ˈali/}{সরু গলি}{ \textsf{\textit{noun}}\ \textbf{1} A toy marble made of marble, alabaster, or glass. {\fontspec{DejaVu Sans}◇} \textit{}}{}{}{ \colorBullet{ORIGIN} Early 18th century perhaps a diminutive of alabaster.}%
\par%
\entry{alliance}{/əˈlʌɪəns/}{জোট}{ \textsf{\textit{noun}}\ \textbf{1} A union or association formed for mutual benefit, especially between countries or organizations. {\fontspec{DejaVu Sans}◇} \textit{a defensive alliance between Australia and New Zealand} \colorBulletS{SYN} association, union, league, treaty, pact, compact, entente, concordat}{}{}{ \colorBullet{ORIGIN} Middle English from Old French aliance, from alier ‘to ally’ (see ally).}%
\par%
\entry{ally}{/ˈalʌɪ/}{মিত্র}{\small{\textsf{\textit{noun, verb}}} \\{\fontspec{DejaVu Sans}▪ }\textsf{\textit{noun}}\\ \textbf{1} A state formally cooperating with another for a military or other purpose. {\fontspec{DejaVu Sans}◇} \textit{debate continued among NATO allies} \\{\fontspec{DejaVu Sans}▪ }\textsf{\textit{verb}}\\ \textbf{1} Combine or unite a resource or commodity with (another) for mutual benefit. {\fontspec{DejaVu Sans}◇} \textit{he allied his racing experience with his father's business acumen} \colorBulletS{SYN} combine, marry, couple, merge, amalgamate, join, pool, fuse, weld, knit}{}{}{ \colorBullet{ORIGIN} Middle English (as a verb): from Old French alier, from Latin alligare ‘bind together’, from ad{-} ‘to’ + ligare ‘to bind’; the noun is partly via Old French alie ‘allied’. Compare with alloy.}%
\par%
\entry{ally}{}{মিত্র}{\small{\textsf{\textit{}}}}{}{}{}%
\par%
\entry{altercation}{/ɒltəˈkeɪʃ(ə)n/}{ঝগড়াঝাঁটি}{ \textsf{\textit{noun}}\ \textbf{1} A noisy argument or disagreement, especially in public. {\fontspec{DejaVu Sans}◇} \textit{I had an altercation with the ticket collector} \colorBulletS{SYN} argument, quarrel, squabble, fight, shouting match, contretemps, disagreement, difference of opinion, dissension, falling{-}out, dispute, disputation, contention, clash, acrimonious exchange, war of words, wrangle}{}{}{ \colorBullet{ORIGIN} Late Middle English from Latin altercatio(n{-}), from the verb altercari (see altercate).}%
\par%
\entry{amalgam}{/əˈmalɡəm/}{মিশ্রণ}{ \textsf{\textit{noun}}\ \textbf{1} A mixture or blend. {\fontspec{DejaVu Sans}◇} \textit{a curious amalgam of the traditional and the modern} \colorBulletS{SYN} combination, union, merger, blend, mixture, mingling, compound, fusion, marriage, weave, coalescence, synthesis, composite, composition, concoction, amalgamation}{}{}{ \colorBullet{ORIGIN} Late 15th century from French amalgame or medieval Latin amalgama, from Greek malagma ‘an emollient’.}%
\par%
\entry{ambiguity}{/ambɪˈɡjuːɪti/}{অস্পষ্টতা}{ \textsf{\textit{noun}}\ \textbf{1} The quality of being open to more than one interpretation; inexactness. {\fontspec{DejaVu Sans}◇} \textit{we can detect no ambiguity in this section of the Act} \colorBulletS{SYN} ambivalence, equivocation}{}{}{ \colorBullet{ORIGIN} Late Middle English from Old French ambiguite or Latin ambiguitas, from ambiguus ‘doubtful’ (see ambiguous).}%
\par%
\entry{amendment}{/əˈmɛn(d)m(ə)nt/}{সংশোধন}{ \textsf{\textit{noun}}\ \textbf{1} A minor change or addition designed to improve a text, piece of legislation, etc. {\fontspec{DejaVu Sans}◇} \textit{an amendment to existing bail laws} \colorBulletS{SYN} revision, alteration, change, modification, qualification, adaptation, adjustment}{}{}{ \colorBullet{ORIGIN} Middle English (in the sense ‘improvement, correction’): from Old French amendement, from amender (see amend).}%
\par%
\entry{amiable}{/ˈeɪmɪəb(ə)l/}{বন্ধুসুলভ}{ \textsf{\textit{adjective}}\ \textbf{1} Having or displaying a friendly and pleasant manner. {\fontspec{DejaVu Sans}◇} \textit{the amiable young man greeted me enthusiastically} \colorBulletS{SYN} friendly, affable, amicable, cordial}{}{}{ \colorBullet{ORIGIN} Late Middle English (originally in the senses ‘kind’, and ‘lovely, lovable’): via Old French from late Latin amicabilis ‘amicable’. The current sense, influenced by modern French aimable ‘trying to please’, dates from the mid 18th century.}%
\par%
\entry{amid}{/əˈmɪd/}{মধ্যে}{ \textsf{\textit{preposition}}\ \textbf{1} Surrounded by; in the middle of. {\fontspec{DejaVu Sans}◇} \textit{our dream home, set amid magnificent rolling countryside} \colorBulletS{SYN} in the middle of, surrounded by, among, amongst, between, in the thick of}{}{}{ \colorBullet{ORIGIN} Middle English amidde(s)(see a, mid).}%
\par%
\entry{ample}{/ˈamp(ə)l/}{প্রশস্ত}{ \textsf{\textit{adjective}}\ \textbf{1} Enough or more than enough; plentiful. {\fontspec{DejaVu Sans}◇} \textit{there is ample time for discussion} \colorBulletS{SYN} enough, sufficient, adequate, plenty of, abundant, more than enough, enough and to spare}{}{}{ \colorBullet{ORIGIN} Late Middle English via French from Latin amplus ‘large, capacious, abundant’.}%
\par%
\entry{ancestral}{/anˈsɛstr(ə)l/}{পৈতৃক}{ \textsf{\textit{adjective}}\ \textbf{1} Of, belonging to, or inherited from an ancestor or ancestors. {\fontspec{DejaVu Sans}◇} \textit{the family's ancestral home} \colorBulletS{SYN} inherited, hereditary, familial}{}{}{ \colorBullet{ORIGIN} Late Middle English from Old French ancestrel, from ancestre (see ancestor).}%
\par%
\entry{anchorage}{/ˈaŋk(ə)rɪdʒ/}{নঙ্গর বাঁধিবার উপকরণ}{ \textsf{\textit{noun}}\ \textbf{1} An area off the coast which is suitable for a ship to anchor. {\fontspec{DejaVu Sans}◇} \textit{} \colorBulletS{SYN} moorings, harbour, port, roads \textbf{2} An anchorite's dwelling place. {\fontspec{DejaVu Sans}◇} \textit{}}{}{}{}%
\par%
\entry{Anchorage}{/ˈaŋk(ə)rɪdʒ/}{নঙ্গর বাঁধিবার উপকরণ}{ \textsf{\textit{proper noun}}\ \textbf{1} The largest city in Alaska, a seaport on an inlet of the Pacific Ocean; population 279,243 (est. 2008). {\fontspec{DejaVu Sans}◇} \textit{}}{}{}{}%
\par%
\entry{ankle}{/ˈaŋk(ə)l/}{গোড়ালি}{\small{\textsf{\textit{noun, verb}}} \\{\fontspec{DejaVu Sans}▪ }\textsf{\textit{noun}}\\ \textbf{1} The joint connecting the foot with the leg. {\fontspec{DejaVu Sans}◇} \textit{Jennie fell downstairs, breaking her ankle} \\{\fontspec{DejaVu Sans}▪ }\textsf{\textit{verb}}\\ \textbf{1} Walk. {\fontspec{DejaVu Sans}◇} \textit{we can ankle off to a new locale} \textbf{2} Flex the ankles while cycling in order to increase pedalling efficiency. {\fontspec{DejaVu Sans}◇} \textit{at higher cadences, the feet tend to flap when you are attempting to ankle}}{}{}{ \colorBullet{ORIGIN} Old English ancleow, of Germanic origin; superseded in Middle English by forms from Old Norse; related to Dutch enkel and German Enkel, from an Indo{-}European root shared by angle.}%
\par%
\entry{annoy}{/əˈnɔɪ/}{বিরক্ত করা}{ \textsf{\textit{verb}}\ \textbf{1} Make (someone) a little angry; irritate. {\fontspec{DejaVu Sans}◇} \textit{the decision really annoyed him} \colorBulletS{SYN} irritate, vex, make angry, make cross, anger, exasperate, irk, gall, pique, put out, displease, get someone's back up, put someone's back up, antagonize, get on someone's nerves, rub up the wrong way, ruffle, ruffle someone's feathers, make someone's hackles rise, raise someone's hackles \textbf{2} Harm or attack repeatedly. {\fontspec{DejaVu Sans}◇} \textit{a gallant Saxon, who annoyed this Coast}}{}{}{ \colorBullet{ORIGIN} Middle English (in the sense ‘be hateful to’): from Old French anoier (verb), anoi (noun), based on Latin in odio in the phrase mihi in odio est ‘it is hateful to me’.}%
\par%
\entry{anomaly}{/əˈnɒm(ə)li/}{ব্যতিক্রম}{ \textsf{\textit{noun}}\ \textbf{1} Something that deviates from what is standard, normal, or expected. {\fontspec{DejaVu Sans}◇} \textit{there are a number of anomalies in the present system} \colorBulletS{SYN} oddity, peculiarity, abnormality, irregularity, inconsistency, incongruity, deviation, aberration, quirk, freak, exception, departure, divergence, variation \textbf{2} The angular distance of a planet or satellite from its last perihelion or perigee. {\fontspec{DejaVu Sans}◇} \textit{}}{}{}{ \colorBullet{ORIGIN} Late 16th century via Latin from Greek anōmalia, from anōmalos (see anomalous).}%
\par%
\entry{anonymity}{/anəˈnɪmɪti/}{অপ্রকাশিতনামা}{ \textsf{\textit{noun}}\ \textbf{1} The condition of being anonymous. {\fontspec{DejaVu Sans}◇} \textit{the official spoke on condition of anonymity}}{}{}{}%
\par%
\entry{anticipate}{/anˈtɪsɪpeɪt/}{অপেক্ষা করা; কহা}{ \textsf{\textit{verb}}\ \textbf{1} Regard as probable; expect or predict. {\fontspec{DejaVu Sans}◇} \textit{she anticipated scorn on her return to the theatre} \colorBulletS{SYN} expect, foresee, predict, think likely, forecast, prophesy, foretell, contemplate the possibility of, allow for, be prepared for \textbf{2} Act as a forerunner or precursor of. {\fontspec{DejaVu Sans}◇} \textit{he anticipated Bates's theories on mimicry and protective coloration} \colorBulletS{SYN} foreshadow, precede, antedate, come before, go before, be earlier than}{}{Much{-}anticipated}{ \colorBullet{ORIGIN} Mid 16th century (in the senses ‘to take something into consideration’, ‘mention something before the proper time’): from Latin anticipat{-} ‘acted in advance’, from anticipare, based on ante{-} ‘before’ + capere ‘take’.}%
\par%
\entry{apart}{/əˈpɑːt/}{পাশাপাশি}{ \textsf{\textit{adverb}}\ \textbf{1} (of two or more people or things) separated by a specified distance in time or space. {\fontspec{DejaVu Sans}◇} \textit{two stone gateposts some thirty feet apart} \colorBulletS{SYN} away from each other, distant from each other \textbf{2} To or on one side; at a distance from the main body. {\fontspec{DejaVu Sans}◇} \textit{Isabel stepped away from Joanna and stood apart} \colorBulletS{SYN} to one side, aside, to the side \textbf{3} So as to be shattered; into pieces. {\fontspec{DejaVu Sans}◇} \textit{he leapt out of the car just before it was blown apart} \colorBulletS{SYN} to pieces, to bits, in pieces}{ \colorBullet{OTHER} apart from}{}{ \colorBullet{ORIGIN} Late Middle English from Old French, from Latin a parte ‘at the side’.}%
\par%
\entry{apathy}{/ˈapəθi/}{ঔদাসীন্য}{ \textsf{\textit{noun}}\ \textbf{1} Lack of interest, enthusiasm, or concern. {\fontspec{DejaVu Sans}◇} \textit{widespread apathy among students} \colorBulletS{SYN} indifference, lack of interest, lack of enthusiasm, lack of concern, unconcern, uninterestedness, unresponsiveness, impassivity, passivity, passiveness, detachment, dispassion, dispassionateness, lack of involvement, phlegm, coolness}{}{}{ \colorBullet{ORIGIN} Early 17th century from French apathie, via Latin from Greek apatheia, from apathēs ‘without feeling’, from a{-} ‘without’ + pathos ‘suffering’.}%
\par%
\entry{aphrodisiac}{/ˌafrəˈdɪzɪak/}{কামোদ্দীপক}{\small{\textsf{\textit{adjective, noun}}} \\{\fontspec{DejaVu Sans}▪ }\textsf{\textit{adjective}}\\ \textbf{1} Of the nature of an aphrodisiac; stimulating sexual desire. {\fontspec{DejaVu Sans}◇} \textit{the aphrodisiac effects of ylang{-}ylang oil} \colorBulletS{SYN} erotic, sexy, sexually arousing, stimulative, stimulant \\{\fontspec{DejaVu Sans}▪ }\textsf{\textit{noun}}\\ \textbf{1} A food, drink, or other thing that stimulates sexual desire. {\fontspec{DejaVu Sans}◇} \textit{power is the ultimate aphrodisiac} \colorBulletS{SYN} love potion, philtre}{}{}{ \colorBullet{ORIGIN} Early 18th century from Greek aphrodisiakos, from aphrodisios, from Aphroditē (see Aphrodite).}%
\par%
\entry{apparatus}{/ˌapəˈreɪtəs/}{যন্ত্রপাতি}{ \textsf{\textit{noun}}\ \textbf{1} The technical equipment or machinery needed for a particular activity or purpose. {\fontspec{DejaVu Sans}◇} \textit{firemen wearing breathing apparatus} \colorBulletS{SYN} equipment, gear, rig, tackle, gadgetry, paraphernalia \textbf{2} The complex structure of a particular organization or system. {\fontspec{DejaVu Sans}◇} \textit{the apparatus of government} \colorBulletS{SYN} structure, system, framework, organization, set{-}up, network \textbf{3}  {\fontspec{DejaVu Sans}◇} \textit{one thing about the book's apparatus does irritate: the absence of an index of titles}}{}{}{ \colorBullet{ORIGIN} Early 17th century Latin, from apparare ‘make ready for’, from ad{-} ‘towards’ + parare ‘make ready’.}%
\par%
\entry{apparel}{/əˈpar(ə)l/}{পোশাক}{\small{\textsf{\textit{noun, verb}}} \\{\fontspec{DejaVu Sans}▪ }\textsf{\textit{noun}}\\ \textbf{1} Clothing. {\fontspec{DejaVu Sans}◇} \textit{they were dressed in bright apparel} \colorBulletS{SYN} clothes, clothing, garments, dress, attire, wear, garb, wardrobe \\{\fontspec{DejaVu Sans}▪ }\textsf{\textit{verb}}\\ \textbf{1} Clothe (someone) {\fontspec{DejaVu Sans}◇} \textit{all the vestments in which they used to apparel their Deities} \colorBulletS{SYN} equip, kit out, fit out, fit up, rig out, supply, issue, furnish with, provide, provision, stock, arm}{}{}{ \colorBullet{ORIGIN} Middle English (as a verb in the sense ‘make ready or fit’; as a noun ‘furnishings, equipment’): from Old French apareillier, based on Latin ad{-} ‘to’ (expressing change) + par ‘equal’.}%
\par%
\entry{apparently}{/əˈparəntli/}{স্পষ্টতই}{ \textsf{\textit{adverb}}\ \textbf{1} As far as one knows or can see. {\fontspec{DejaVu Sans}◇} \textit{the child nodded, apparently content with the promise} \colorBulletS{SYN} seemingly, evidently, it seems, it seems that, it would seem, it would seem that, it appears, it appears that, it would appear, it would appear that, as far as one knows, by all accounts, so it seems}{}{}{}%
\par%
\entry{applause}{/əˈplɔːz/}{সাধুবাদ}{ \textsf{\textit{noun}}\ \textbf{1} Approval or praise expressed by clapping. {\fontspec{DejaVu Sans}◇} \textit{they gave him a round of applause} \colorBulletS{SYN} clapping, handclapping, cheering, whistling, ovation, standing ovation, acclamation, cheers, whistles, bravos}{}{}{ \colorBullet{ORIGIN} Late Middle English from medieval Latin applausus, from the verb applaudere (see applaud).}%
\par%
\entry{apprehension}{/aprɪˈhɛnʃ(ə)n/}{চেতনা}{ \textsf{\textit{noun}}\ \textbf{1} Anxiety or fear that something bad or unpleasant will happen. {\fontspec{DejaVu Sans}◇} \textit{he felt sick with apprehension} \colorBulletS{SYN} anxiety, angst, alarm, worry, uneasiness, unease, nervousness, misgiving, disquiet, concern, agitation, restlessness, edginess, fidgetiness, nerves, tension, trepidation, perturbation, consternation, panic, fearfulness, dread, fear, shock, horror, terror \textbf{2} Understanding; grasp. {\fontspec{DejaVu Sans}◇} \textit{his first apprehension of such large issues} \colorBulletS{SYN} understanding, grasp, comprehension, realization, recognition, appreciation, discernment, perception, awareness, cognizance, consciousness, penetration \textbf{3} The action of arresting someone. {\fontspec{DejaVu Sans}◇} \textit{they acted with intent to prevent lawful apprehension} \colorBulletS{SYN} arrest, capture, seizure, catching}{}{}{ \colorBullet{ORIGIN} Late Middle English (in the sense ‘learning, acquisition of knowledge’): from late Latin apprehensio(n{-}), from apprehendere ‘seize, grasp’ (see apprehend).}%
\par%
\entry{apprise}{/əˈprʌɪz/}{অবগত করান}{ \textsf{\textit{verb}}\ \textbf{1} Inform or tell (someone) {\fontspec{DejaVu Sans}◇} \textit{I thought it right to apprise Chris of what had happened} \colorBulletS{SYN} inform, notify, tell, let know, advise, brief, intimate, make aware of, send word to, update, keep posted, keep up to date, keep up to speed, enlighten}{ \colorBullet{OTHER} apprise of}{We are apprised of the sufferings and hardships of women in our society.}{ \colorBullet{ORIGIN} Late 17th century from French appris, apprise, past participle of apprendre ‘learn, teach’, from Latin apprehendere (see apprehend).}%
\par%
\entry{apropos}{/ˌaprəˈpəʊ/}{এতৎ সম্পর্কে}{\small{\textsf{\textit{adjective, preposition}}} \\{\fontspec{DejaVu Sans}▪ }\textsf{\textit{adjective}}\\ \textbf{1} Very appropriate to a particular situation. {\fontspec{DejaVu Sans}◇} \textit{the composer's reference to child's play is apropos} \colorBulletS{SYN} appropriate, pertinent, relevant, apposite, apt, applicable, suitable, germane, material, becoming, befitting, significant, to the point, to the purpose \\{\fontspec{DejaVu Sans}▪ }\textsf{\textit{preposition}}\\ \textbf{1} With reference to; concerning. {\fontspec{DejaVu Sans}◇} \textit{she remarked apropos of the initiative, ‘It's not going to stop the abuse’} \colorBulletS{SYN} with reference to, with regard to, with respect to, regarding, concerning, respecting, on the subject of, in the matter of, touching on, dealing with, connected with, in connection with, about, re}{}{That's not apropos}{ \colorBullet{ORIGIN} Mid 17th century from French à propos ‘(with regard) to (this) purpose’.}%
\par%
\entry{arbitration}{/ɑːbɪˈtreɪʃ(ə)n/}{সালিসি}{ \textsf{\textit{noun}}\ \textbf{1} The use of an arbitrator to settle a dispute. {\fontspec{DejaVu Sans}◇} \textit{Tayside Regional Council called for arbitration to settle the dispute} \colorBulletS{SYN} adjudication, mediation, mediatorship, negotiation, conciliation, intervention, interceding, interposition, peacemaking}{}{}{ \colorBullet{ORIGIN} Use an arbitrator to settle a dispute.}%
\par%
\entry{arduous}{/ˈɑːdjʊəs/}{শ্রমসাধ্য}{ \textsf{\textit{adjective}}\ \textbf{1} Involving or requiring strenuous effort; difficult and tiring. {\fontspec{DejaVu Sans}◇} \textit{an arduous journey} \colorBulletS{SYN} onerous, taxing, difficult, hard, heavy, laborious, burdensome, strenuous, vigorous, back{-}breaking, stiff, uphill, relentless, Herculean}{}{}{ \colorBullet{ORIGIN} Mid 16th century from Latin arduus ‘steep, difficult’ + {-}ous.}%
\par%
\entry{areola}{/əˈriːələ/}{ব্রণ বা ফোড়ার চারপাশের গোলাকার লালচে জায়গা}{ \textsf{\textit{noun}}\ \textbf{1} A small circular area, in particular the ring of pigmented skin surrounding a nipple. {\fontspec{DejaVu Sans}◇} \textit{}}{}{}{ \colorBullet{ORIGIN} Mid 17th century (in the sense ‘small space or interstice’): from Latin, literally ‘small open space’, diminutive of area (see area).}%
\par%
\entry{armor}{/ˈärmər/}{বর্ম}{\small{\textsf{\textit{noun, transitive verb}}} \\{\fontspec{DejaVu Sans}▪ }\textsf{\textit{noun}}\\ \textbf{1} The metal coverings formerly worn by soldiers or warriors to protect the body in battle. {\fontspec{DejaVu Sans}◇} \textit{knights in armor} \colorBulletS{SYN} protective covering, armour plate \\{\fontspec{DejaVu Sans}▪ }\textsf{\textit{transitive verb}}\\ \textbf{1} Provide (someone) with emotional, social, or other defenses. {\fontspec{DejaVu Sans}◇} \textit{the knowledge armored him against her}}{}{}{ \colorBullet{ORIGIN} Middle English from Old French armure, from Latin armatura, from armare ‘to arm’ (see arm).}%
\par%
\entry{armpit}{/ˈɑːmpɪt/}{বগল}{ \textsf{\textit{noun}}\ \textbf{1} A hollow under the arm at the shoulder. {\fontspec{DejaVu Sans}◇} \textit{}}{}{}{ \colorBullet{ORIGIN} Deeply involved in a particular unpleasant situation or enterprise.}%
\par%
\entry{arrears}{/əˈrɪəz/}{বকেয়া}{ \textsf{\textit{plural noun}}\ \textbf{1} Money that is owed and should have been paid earlier. {\fontspec{DejaVu Sans}◇} \textit{he was suing the lessee for the arrears of rent} \colorBulletS{SYN} money owing, outstanding payment, outstanding payments, debt, debts, liabilities, indebtedness, dues}{}{}{ \colorBullet{ORIGIN} Middle English (first used in the phrase in arrear): from arrear (adverb) ‘behind, overdue’, from Old French arere, from medieval Latin adretro, from ad{-} ‘towards’ + retro ‘backwards’.}%
\par%
\entry{arrestee}{/əˌrɛstˈiː/}{আটক}{ \textsf{\textit{noun}}\ \textbf{1} A person who has been or is being legally arrested. {\fontspec{DejaVu Sans}◇} \textit{}}{}{}{}%
\par%
\entry{arson}{/ˈɑːs(ə)n/}{অগ্নিসংযোগ}{ \textsf{\textit{noun}}\ \textbf{1} The criminal act of deliberately setting fire to property. {\fontspec{DejaVu Sans}◇} \textit{police are treating the fire as arson} \colorBulletS{SYN} incendiarism, pyromania, firebombing}{}{arson attack}{ \colorBullet{ORIGIN} Late 17th century an Anglo{-}Norman French legal term, from medieval Latin arsio(n{-}), from Latin ardere ‘to burn’.}%
\par%
\entry{ask out}{}{1. To ask someone to go on a date. 2. To invite someone to a social event or special occasion. 3. To invite someone to a distant location.}{\small{\textsf{\textit{}}}}{}{1. you are asking me out? 2. Bill still hasn't asked me out—maybe he doesn't have romantic feelings for me after all. 3. I'm sorry, we're busy on friday night—my boss has asked us out to the theater.4. I have asked hannah out to our new place, but she never wants to drive all the way from the city.}{}%
\par%
\entry{aspiration}{/aspəˈreɪʃ(ə)n/}{শ্বাসাঘাত}{ \textsf{\textit{noun}}\ \textbf{1} A hope or ambition of achieving something. {\fontspec{DejaVu Sans}◇} \textit{the needs and aspirations of the people} \colorBulletS{SYN} desire, hope, longing, yearning, hankering, urge, wish \textbf{2} The action or process of drawing breath. {\fontspec{DejaVu Sans}◇} \textit{These factors lead to either inhalation or aspiration of pathogens into the respiratory tract.} \textbf{3} The action of pronouncing a sound with an exhalation of breath. {\fontspec{DejaVu Sans}◇} \textit{there is no aspiration if the syllable begins with s}}{}{}{ \colorBullet{ORIGIN} Late Middle English (in aspiration (sense 3)): from Latin aspiratio(n{-}), from the verb aspirare (see aspire).}%
\par%
\entry{assailant}{/əˈseɪl(ə)nt/}{আততীয়}{ \textsf{\textit{noun}}\ \textbf{1} A person who physically attacks another. {\fontspec{DejaVu Sans}◇} \textit{the police have no firm leads about the identity of his assailant} \colorBulletS{SYN} attacker, mugger}{}{}{}%
\par%
\entry{assault}{/əˈsɔːlt/}{লাঞ্ছনা}{\small{\textsf{\textit{noun, verb}}} \\{\fontspec{DejaVu Sans}▪ }\textsf{\textit{noun}}\\ \textbf{1} A physical attack. {\fontspec{DejaVu Sans}◇} \textit{his imprisonment for an assault on the film director} \colorBulletS{SYN} violence, physical violence, battery, mugging, actual bodily harm, ABH \textbf{2} A concerted attempt to do something demanding. {\fontspec{DejaVu Sans}◇} \textit{a winter assault on Mt Everest} \\{\fontspec{DejaVu Sans}▪ }\textsf{\textit{verb}}\\ \textbf{1} Make a physical attack on. {\fontspec{DejaVu Sans}◇} \textit{he pleaded guilty to assaulting a police officer} \colorBulletS{SYN} hit, strike, physically attack, aim blows at, slap, smack, beat, thrash, spank, thump, thwack, punch, cuff, swat, knock, rap}{}{}{ \colorBullet{ORIGIN} Middle English from Old French asaut (noun), assauter (verb), based on Latin ad{-} ‘to’ + saltare, frequentative of salire ‘to leap’. Compare with assail.}%
\par%
\entry{assert}{/əˈsəːt/}{জাহির করা}{ \textsf{\textit{verb}}\ \textbf{1} State a fact or belief confidently and forcefully. {\fontspec{DejaVu Sans}◇} \textit{the company asserts that the cuts will not affect development} \colorBulletS{SYN} declare, maintain, contend, argue, state, claim, propound, submit, posit, postulate, adduce, move, advocate, venture, volunteer, aver, proclaim, announce, pronounce, attest, affirm, protest, profess, swear, insist, avow}{}{}{ \colorBullet{ORIGIN} Early 17th century from Latin asserere ‘claim, affirm’, from ad{-} ‘to’ + serere ‘to join’.}%
\par%
\entry{assess}{/əˈsɛs/}{পরিমাপ করা}{ \textsf{\textit{verb}}\ \textbf{1} Evaluate or estimate the nature, ability, or quality of. {\fontspec{DejaVu Sans}◇} \textit{the committee must assess the relative importance of the issues} \colorBulletS{SYN} evaluate, judge, gauge, rate, estimate, appraise, form an opinion of, check out, form an impression of, make up one's mind about, get the measure of, determine, weigh up, analyse}{}{}{ \colorBullet{ORIGIN} Late Middle English from Old French assesser, based on Latin assidere ‘sit by’ (in medieval Latin ‘levy tax’), from ad{-} ‘to, at’ + sedere ‘sit’. Compare with assize.}%
\par%
\entry{assume}{/əˈsjuːm/}{অনুমান}{ \textsf{\textit{verb}}\ \textbf{1} Suppose to be the case, without proof. {\fontspec{DejaVu Sans}◇} \textit{topics which assume detailed knowledge of local events} \colorBulletS{SYN} presume, suppose, take it, take for granted, take as read, take it as given, presuppose, conjecture, surmise, conclude, come to the conclusion, deduce, infer, draw the inference, reckon, reason, guess, imagine, think, fancy, suspect, expect, accept, believe, be of the opinion, understand, be given to understand, gather, glean \textbf{2} Take or begin to have (power or responsibility) {\fontspec{DejaVu Sans}◇} \textit{he assumed full responsibility for all organizational work} \colorBulletS{SYN} accept, shoulder, bear, undertake, take on, take up, take on oneself, manage, handle, deal with, get to grips with, turn one's hand to \textbf{3} Begin to have (a specified quality, appearance, or extent) {\fontspec{DejaVu Sans}◇} \textit{militant activity had assumed epidemic proportions} \colorBulletS{SYN} acquire, take on, adopt, come to have}{}{}{ \colorBullet{ORIGIN} Late Middle English from Latin assumere, from ad{-} ‘towards’ + sumere ‘take’.}%
\par%
\entry{assurance}{/əˈʃʊər(ə)ns/}{আশ্বাসন}{ \textsf{\textit{noun}}\ \textbf{1} A positive declaration intended to give confidence; a promise. {\fontspec{DejaVu Sans}◇} \textit{he gave an assurance that work would begin on Monday} \colorBulletS{SYN} word of honour, word, guarantee, promise, pledge, vow, avowal, oath, bond, affirmation, undertaking, commitment \textbf{2} Confidence or certainty in one's own abilities. {\fontspec{DejaVu Sans}◇} \textit{she drove with assurance} \colorBulletS{SYN} self{-}confidence, confidence, self{-}assurance, belief in oneself, faith in oneself, positiveness, assertiveness, self{-}possession, self{-}reliance, nerve, poise, aplomb, presence of mind, phlegm, level{-}headedness, cool{-}headedness \textbf{3} Insurance, specifically life insurance. {\fontspec{DejaVu Sans}◇} \textit{} \colorBulletS{SYN} insurance, indemnity, indemnification, protection, security, surety, cover, coverage}{}{}{ \colorBullet{ORIGIN} Late Middle English (in assurance (sense 2)): from Old French, from assurer ‘assure’.}%
\par%
\entry{assure}{/əˈʃʊə/}{নিশ্চিত করা}{ \textsf{\textit{verb}}\ \textbf{1} Tell someone something positively to dispel any doubts. {\fontspec{DejaVu Sans}◇} \textit{Tony assured me that there was a supermarket in the village} \colorBulletS{SYN} reassure, convince, satisfy, persuade, guarantee, promise, tell \textbf{2} Make (something) certain to happen. {\fontspec{DejaVu Sans}◇} \textit{victory was now assured} \textbf{3} Cover (a life) by assurance. {\fontspec{DejaVu Sans}◇} \textit{we guarantee to assure your life} \colorBulletS{SYN} insure, provide insurance for, cover, indemnify, guarantee, warrant}{}{}{ \colorBullet{ORIGIN} Late Middle English from Old French assurer, based on Latin ad{-} ‘to’ (expressing change) + securus (see secure).}%
\par%
\entry{assuredly}{/əˈʃʊərədli/}{নিশ্চয়}{ \textsf{\textit{adverb}}\ \textbf{1} Confidently. {\fontspec{DejaVu Sans}◇} \textit{the lad kept his cool and assuredly slipped the ball between the posts} \textbf{2} Used to express the speaker's certainty that something is true. {\fontspec{DejaVu Sans}◇} \textit{potted roses will most assuredly not survive winter without protection}}{}{}{}%
\par%
\entry{asylum}{/əˈsʌɪləm/}{আশ্রয়}{ \textsf{\textit{noun}}\ \textbf{1} The protection granted by a state to someone who has left their home country as a political refugee. {\fontspec{DejaVu Sans}◇} \textit{she applied for asylum and was granted refugee status} \textbf{2} An institution for the care of people who are mentally ill. {\fontspec{DejaVu Sans}◇} \textit{he'd been committed to an asylum} \colorBulletS{SYN} psychiatric hospital, mental hospital, mental institution, mental asylum, institution}{}{}{ \colorBullet{ORIGIN} Late Middle English (in the sense ‘place of refuge’, especially for criminals): via Latin from Greek asulon ‘refuge’, from asulos ‘inviolable’, from a{-} ‘without’ + sulon ‘right of seizure’. Current senses date from the 18th century.}%
\par%
\entry{attenuate}{/əˈtɛnjʊeɪt/}{কৃশ}{\small{\textsf{\textit{adjective, verb}}} \\{\fontspec{DejaVu Sans}▪ }\textsf{\textit{adjective}}\\ \textbf{1} Reduced in force, effect, or physical thickness. {\fontspec{DejaVu Sans}◇} \textit{the doctrines of Christianity became very attenuate and distorted} \colorBulletS{SYN} thin, slender, slim, skinny, spindly, bony, gaunt, skeletal \\{\fontspec{DejaVu Sans}▪ }\textsf{\textit{verb}}\\ \textbf{1} Reduce the force, effect, or value of. {\fontspec{DejaVu Sans}◇} \textit{her intolerance was attenuated by an unexpected liberalism} \colorBulletS{SYN} weakened, reduced, lessened, decreased, diminished, impaired, enervated \textbf{2} Reduce in thickness; make thin. {\fontspec{DejaVu Sans}◇} \textit{} \colorBulletS{SYN} thin, slender, slim, skinny, spindly, bony, gaunt, skeletal}{}{}{ \colorBullet{ORIGIN} Mid 16th century from Latin attenuat{-} ‘made slender’, from the verb attenuare, from ad{-} ‘to’ + tenuare ‘make thin’ (from tenuis ‘thin’).}%
\par%
\entry{attorney}{/əˈtəːni/}{অ্যাটর্নি}{ \textsf{\textit{noun}}\ \textbf{1} A person, typically a lawyer, appointed to act for another in business or legal matters. {\fontspec{DejaVu Sans}◇} \textit{} \colorBulletS{SYN} deputy, representative, substitute, delegate, agent, surrogate, stand{-}in, attorney, ambassador, emissary, go{-}between, envoy, frontman}{}{}{ \colorBullet{ORIGIN} Middle English from Old French atorne, past participle of atorner ‘assign’, from a ‘towards’ + torner ‘to turn’ (see attorn).}%
\par%
\entry{aubergine}{/ˈəʊbəʒiːn/}{বেগুন}{ \textsf{\textit{noun}}\ \textbf{1} The purple egg{-}shaped fruit of a tropical Old World plant, which is eaten as a vegetable. {\fontspec{DejaVu Sans}◇} \textit{a puree of aubergine} \textbf{2} The large plant of the nightshade family which bears aubergines. {\fontspec{DejaVu Sans}◇} \textit{The capsicums are a genus of the family Solanaceae, and are therefore related to the New World tomato and potato, and, in the Old World, to the aubergine and deadly nightshade.}}{}{}{ \colorBullet{ORIGIN} Late 18th century from French, from Catalan alberginia, from Arabic al{-}bāḏinjān (based on Persian bādingān, from Sanskrit vātiṃgaṇa).}%
\par%
\entry{auburn}{/ˈɔːbən/}{পিঙ্গল}{\small{\textsf{\textit{adjective, noun}}} \\{\fontspec{DejaVu Sans}▪ }\textsf{\textit{adjective}}\\ \textbf{1} (of hair) of a reddish{-}brown colour. {\fontspec{DejaVu Sans}◇} \textit{} \colorBulletS{SYN} reddish brown, red{-}brown, dark red, Titian, Titian red, tawny, russet, chestnut, chestnut{-}coloured, copper, coppery, copper{-}coloured, rust{-}coloured, rufous, henna, hennaed \\{\fontspec{DejaVu Sans}▪ }\textsf{\textit{noun}}\\ \textbf{1} A reddish{-}brown colour. {\fontspec{DejaVu Sans}◇} \textit{}}{}{}{ \colorBullet{ORIGIN} Late Middle English from Old French auborne, alborne, from Latin alburnus ‘whitish’, from albus ‘white’. The original sense was ‘yellowish white’, but the word became associated with brown because in the 16th and 17th centuries it was often written abrune or abroun.}%
\par%
\entry{augment}{/ɔːɡˈmɛnt/}{বৃদ্ধি}{\small{\textsf{\textit{noun, verb}}} \\{\fontspec{DejaVu Sans}▪ }\textsf{\textit{noun}}\\ \textbf{1} A vowel prefixed to past tenses of verbs in Greek and certain other Indo{-}European languages. {\fontspec{DejaVu Sans}◇} \textit{} \\{\fontspec{DejaVu Sans}▪ }\textsf{\textit{verb}}\\ \textbf{1} Make (something) greater by adding to it; increase. {\fontspec{DejaVu Sans}◇} \textit{he augmented his summer income by painting houses} \colorBulletS{SYN} increase, make larger, make bigger, make greater, add to, supplement, top up, build up, enlarge, expand, extend, raise, multiply, elevate, swell, inflate}{}{}{ \colorBullet{ORIGIN} Late Middle English from Old French augmenter (verb), augment (noun), or late Latin augmentare, from Latin augere ‘to increase’.}%
\par%
\entry{autocrat}{/ˈɔːtəkrat/}{একনায়ক}{ \textsf{\textit{noun}}\ \textbf{1} A ruler who has absolute power. {\fontspec{DejaVu Sans}◇} \textit{like many autocrats, Franco found the exercise of absolute power addictive} \colorBulletS{SYN} absolute ruler, dictator, despot, tyrant, monocrat}{}{}{ \colorBullet{ORIGIN} Early 19th century from French autocrate, from Greek autokratēs, from autos ‘self’ + kratos ‘power’.}%
\par%
\entry{autopsy}{/ˈɔːtɒpsi/}{ময়না}{\small{\textsf{\textit{noun, verb}}} \\{\fontspec{DejaVu Sans}▪ }\textsf{\textit{noun}}\\ \textbf{1} A post{-}mortem examination to discover the cause of death or the extent of disease. {\fontspec{DejaVu Sans}◇} \textit{a Home Office pathologist carried out the autopsy} \colorBulletS{SYN} post{-}mortem, PM, necropsy \\{\fontspec{DejaVu Sans}▪ }\textsf{\textit{verb}}\\ \textbf{1} Perform an autopsy on (a body or organ) {\fontspec{DejaVu Sans}◇} \textit{the animal must be autopsied as soon as possible}}{}{}{ \colorBullet{ORIGIN} Mid 17th century (in the sense ‘personal observation’): from French autopsie or modern Latin autopsia, from Greek, from autoptēs ‘eyewitness’, from autos ‘self’ + optos ‘seen’.}%
\par%
\entry{autotroph}{/ˈɔːtə(ʊ)trəʊf/}{}{ \textsf{\textit{noun}}\ \textbf{1} An organism that is able to form nutritional organic substances from simple inorganic substances such as carbon dioxide. {\fontspec{DejaVu Sans}◇} \textit{}}{}{}{}%
\par%
\entry{aversion}{/əˈvəːʃ(ə)n/}{বিরাগ}{ \textsf{\textit{noun}}\ \textbf{1} A strong dislike or disinclination. {\fontspec{DejaVu Sans}◇} \textit{they made plain their aversion to the use of force} \colorBulletS{SYN} dislike of, distaste for, disinclination, abhorrence, hatred, hate, loathing, detestation, odium, antipathy, hostility}{}{}{ \colorBullet{ORIGIN} Late 16th century (originally denoting the action of turning away or averting one's eyes): from Latin aversio(n{-}), from avertere ‘turn away from’ (see avert).}%
\par%
\entry{avert}{/əˈvəːt/}{প্রতিহত করা}{ \textsf{\textit{verb}}\ \textbf{1} Turn away (one's eyes or thoughts) {\fontspec{DejaVu Sans}◇} \textit{she averted her eyes while we made stilted conversation} \colorBulletS{SYN} turn aside, turn away, turn to one side \textbf{2} Prevent or ward off (an undesirable occurrence) {\fontspec{DejaVu Sans}◇} \textit{talks failed to avert a rail strike} \colorBulletS{SYN} prevent, stop, avoid, nip in the bud}{}{}{ \colorBullet{ORIGIN} Late Middle English (in the sense ‘divert or deter someone from a place or a course of action’): from Latin avertere, from ab{-} ‘from’ + vertere ‘to turn’; reinforced by Old French avertir.}%
\par%
\entry{awe}{/ɔː/}{সম্ভ্রম}{\small{\textsf{\textit{noun, verb}}} \\{\fontspec{DejaVu Sans}▪ }\textsf{\textit{noun}}\\ \textbf{1} A feeling of reverential respect mixed with fear or wonder. {\fontspec{DejaVu Sans}◇} \textit{they gazed in awe at the small mountain of diamonds} \colorBulletS{SYN} wonder, wonderment, amazement, astonishment \\{\fontspec{DejaVu Sans}▪ }\textsf{\textit{verb}}\\ \textbf{1} Inspire with awe. {\fontspec{DejaVu Sans}◇} \textit{they were both awed by the vastness of the forest} \colorBulletS{SYN} filled with wonder, wonderstruck, awestruck, amazed, filled with amazement, astonished, filled with astonishment, lost for words, open{-}mouthed}{}{}{ \colorBullet{ORIGIN} Old English ege ‘terror, dread, awe’, replaced in Middle English by forms related to Old Norse agi.}%
\par%
\end{multicols}%
\pagebreak%
\section*{B}%
\begin{multicols}{2}%
\entry{backdrop}{/ˈbakdrɒp/}{ব্যাকড্রপ}{\small{\textsf{\textit{noun, verb}}} \\{\fontspec{DejaVu Sans}▪ }\textsf{\textit{noun}}\\ \textbf{1} A painted cloth hung at the back of a theatre stage as part of the scenery. {\fontspec{DejaVu Sans}◇} \textit{} \colorBulletS{SYN} stage set, set, flats, backdrop, drop curtain \\{\fontspec{DejaVu Sans}▪ }\textsf{\textit{verb}}\\ \textbf{1} Lie behind or beyond; serve as a background to. {\fontspec{DejaVu Sans}◇} \textit{the rolling hills that backdropped our camp}}{}{}{}%
\par%
\entry{backwash}{/ˈbakwɒʃ/}{প্রতিক্রিয়া}{\small{\textsf{\textit{noun, verb}}} \\{\fontspec{DejaVu Sans}▪ }\textsf{\textit{noun}}\\ \textbf{1} The motion of receding waves. {\fontspec{DejaVu Sans}◇} \textit{the backwash is reduced in energy by the percolation of water into the shingle} \colorBulletS{SYN} wake, wash, slipstream, backflow \\{\fontspec{DejaVu Sans}▪ }\textsf{\textit{verb}}\\ \textbf{1} Clean (a filter) by reversing the flow of fluid through it. {\fontspec{DejaVu Sans}◇} \textit{the very fine mesh is backwashed to remove solids}}{}{}{}%
\par%
\entry{badly}{/ˈbadli/}{খারাপভাবে}{ \textsf{\textit{adverb}}\ \textbf{1} In an unsatisfactory, inadequate, or unsuccessful way. {\fontspec{DejaVu Sans}◇} \textit{England have played badly this year} \colorBulletS{SYN} poorly, incompetently, ineptly, inexpertly, inefficiently, imperfectly, deficiently, defectively, unsatisfactorily, inadequately, incorrectly, faultily, shoddily, amateurishly, carelessly, negligently \textbf{2} Used to emphasize the seriousness of an unpleasant event or action. {\fontspec{DejaVu Sans}◇} \textit{the building was badly damaged by fire} \colorBulletS{SYN} severely, gravely, badly, critically, acutely, sorely, grievously, desperately, alarmingly, dangerously, perilously \textbf{3} In a guilty or regretful way. {\fontspec{DejaVu Sans}◇} \textit{I felt badly about my unfriendliness}}{}{}{ \colorBullet{ORIGIN} At a disadvantage, especially by being poor.}%
\par%
\entry{baffling}{/ˈbaf(ə)lɪŋ/}{বিভ্রান্তিকর}{ \textsf{\textit{adjective}}\ \textbf{1} Impossible to understand; perplexing. {\fontspec{DejaVu Sans}◇} \textit{the crime is a baffling mystery for the police} \colorBulletS{SYN} puzzling, bewildering, perplexing, mystifying, bemusing, confusing, unclear, difficult to understand, hard to understand, beyond one, above one's head}{}{}{}%
\par%
\entry{bail}{/beɪl/}{জামিন}{\small{\textsf{\textit{noun, verb}}} \\{\fontspec{DejaVu Sans}▪ }\textsf{\textit{noun}}\\ \textbf{1} The temporary release of an accused person awaiting trial, sometimes on condition that a sum of money is lodged to guarantee their appearance in court. {\fontspec{DejaVu Sans}◇} \textit{he has been released on bail} \colorBulletS{SYN} surety, security, collateral, assurance, indemnity, indemnification \\{\fontspec{DejaVu Sans}▪ }\textsf{\textit{verb}}\\ \textbf{1} Release or secure the release of (a prisoner) on payment of bail. {\fontspec{DejaVu Sans}◇} \textit{nine were bailed on drugs charges}}{}{}{ \colorBullet{ORIGIN} Middle English from Old French, literally ‘custody, jurisdiction’, from bailler ‘take charge of’, from Latin bajulare ‘bear a burden’.}%
\par%
\entry{bail}{/beɪl/}{জামিন}{\small{\textsf{\textit{noun, verb}}} \\{\fontspec{DejaVu Sans}▪ }\textsf{\textit{noun}}\\ \textbf{1} Either of the two crosspieces bridging the stumps, which the bowler and fielders try to dislodge with the ball to get the batsman out. {\fontspec{DejaVu Sans}◇} \textit{the Lancashire captain was at full stretch as the wicketkeeper took off the bails} \textbf{2} A bar on a typewriter or computer printer which holds the paper steady. {\fontspec{DejaVu Sans}◇} \textit{} \textbf{3} A fastening that secures a crampon to the sole of a boot. {\fontspec{DejaVu Sans}◇} \textit{} \textbf{4} A bar or pole separating horses in an open stable. {\fontspec{DejaVu Sans}◇} \textit{} \\{\fontspec{DejaVu Sans}▪ }\textsf{\textit{verb}}\\ \textbf{1} Confront (someone) with the intention of robbing them. {\fontspec{DejaVu Sans}◇} \textit{they bailed up Mr Dyason and demanded his money} \textbf{2} Secure (a cow) during milking. {\fontspec{DejaVu Sans}◇} \textit{}}{}{}{ \colorBullet{ORIGIN} Middle English (denoting the outer wall of a castle): from Old French baile ‘palisade, enclosure’, baillier ‘enclose’, perhaps from Latin baculum ‘rod, stick’.}%
\par%
\entry{bail}{/beɪl/}{জামিন}{ \textsf{\textit{verb}}\ \textbf{1} Scoop water out of (a ship or boat) {\fontspec{DejaVu Sans}◇} \textit{the first priority is to bail out the boat with buckets} \textbf{2} Abandon a commitment, obligation, or activity. {\fontspec{DejaVu Sans}◇} \textit{}}{}{}{ \colorBullet{ORIGIN} Early 17th century from obsolete bail ‘bucket’, from French baille, based on Latin bajulus ‘carrier’.}%
\par%
\entry{banality}{/bəˈnalɪti/}{তুচ্ছতা}{ \textsf{\textit{noun}}\ \textbf{1} The fact or condition of being banal; unoriginality. {\fontspec{DejaVu Sans}◇} \textit{there is an essential banality to the story he tells} \colorBulletS{SYN} triteness, platitudinousness, vapidity, pedestrianism, conventionality, predictability, staleness, unimaginativeness, lack of originality, lack of inspiration, prosaicness, dullness, ordinariness}{}{}{}%
\par%
\entry{bandit}{/ˈbandɪt/}{ডাকাত}{ \textsf{\textit{noun}}\ \textbf{1} A robber or outlaw belonging to a gang and typically operating in an isolated or lawless area. {\fontspec{DejaVu Sans}◇} \textit{the bandit produced a weapon and demanded money} \colorBulletS{SYN} robber, raider, mugger}{}{}{ \colorBullet{ORIGIN} Late 16th century from Italian bandito, ‘banned’, past participle of bandire ‘to ban’.}%
\par%
\entry{bankroll}{/ˈbaŋkrəʊl/}{টাকা যোগান}{\small{\textsf{\textit{noun, verb}}} \\{\fontspec{DejaVu Sans}▪ }\textsf{\textit{noun}}\\ \textbf{1} A roll of banknotes. {\fontspec{DejaVu Sans}◇} \textit{} \colorBulletS{SYN} bundle, roll, bankroll, pile, stack, sheaf, pocketful, load \\{\fontspec{DejaVu Sans}▪ }\textsf{\textit{verb}}\\ \textbf{1} Support (a person, organization, or project) financially. {\fontspec{DejaVu Sans}◇} \textit{the project is bankrolled by wealthy expatriates} \colorBulletS{SYN} sponsor, support, back, insure, indemnify, provide security for, take the risk for, subsidize, contribute to, pay for, provide capital for, finance, fund}{}{}{}%
\par%
\entry{barber}{/ˈbɑːbə/}{নাপিত}{\small{\textsf{\textit{noun, verb}}} \\{\fontspec{DejaVu Sans}▪ }\textsf{\textit{noun}}\\ \textbf{1} A person who cuts men's hair and shaves or trims beards as an occupation. {\fontspec{DejaVu Sans}◇} \textit{he had his hair cut at the local barber's} \\{\fontspec{DejaVu Sans}▪ }\textsf{\textit{verb}}\\ \textbf{1} Cut or trim (a man's hair) {\fontspec{DejaVu Sans}◇} \textit{his hair was neatly barbered} \colorBulletS{SYN} cut short, cut, clip, trim, snip, shear, shave}{}{}{ \colorBullet{ORIGIN} Middle English via Anglo{-}Norman French from Old French barbe (see barb).}%
\par%
\entry{barbershop}{/ˈbɑːbəʃɒp/}{সেলুন}{ \textsf{\textit{noun}}\ \textbf{1} A shop where a barber works. {\fontspec{DejaVu Sans}◇} \textit{} \textbf{2} A popular style of close harmony singing, typically for four male voices. {\fontspec{DejaVu Sans}◇} \textit{a barbershop quartet}}{}{}{}%
\par%
\entry{bargain}{/ˈbɑːɡɪn/}{কারবারী}{\small{\textsf{\textit{noun, verb}}} \\{\fontspec{DejaVu Sans}▪ }\textsf{\textit{noun}}\\ \textbf{1} An agreement between two or more people or groups as to what each will do for the other. {\fontspec{DejaVu Sans}◇} \textit{bargains between political parties supporting the government} \colorBulletS{SYN} agreement, arrangement, understanding, deal \textbf{2} A thing bought or offered for sale much more cheaply than is usual or expected. {\fontspec{DejaVu Sans}◇} \textit{the table was a real bargain} \colorBulletS{SYN} good buy, cheap buy \\{\fontspec{DejaVu Sans}▪ }\textsf{\textit{verb}}\\ \textbf{1} Negotiate the terms and conditions of a transaction. {\fontspec{DejaVu Sans}◇} \textit{he bargained with the local council to rent the stadium} \colorBulletS{SYN} haggle, barter, negotiate, discuss terms, hold talks, deal, wheel and deal, trade, traffic \textbf{2} Be prepared for; expect. {\fontspec{DejaVu Sans}◇} \textit{I got more information than I'd bargained for} \colorBulletS{SYN} expect, anticipate, be prepared for, allow for, plan for, reckon with, take into account, take into consideration, contemplate, imagine, envisage, foresee, predict, look for, hope for, look to}{}{}{ \colorBullet{ORIGIN} Middle English from Old French bargaine (noun), bargaignier (verb); probably of Germanic origin and related to German borgen ‘borrow’.}%
\par%
\entry{barrage}{/ˈbarɑːʒ/}{বাঁধ}{\small{\textsf{\textit{noun, verb}}} \\{\fontspec{DejaVu Sans}▪ }\textsf{\textit{noun}}\\ \textbf{1} A concentrated artillery bombardment over a wide area. {\fontspec{DejaVu Sans}◇} \textit{his forces launched an artillery barrage on the city} \colorBulletS{SYN} bombardment, gunfire, cannonade, battery, blast, broadside, salvo, volley, fusillade \textbf{2} An artificial barrier across a river or estuary to prevent flooding, aid irrigation or navigation, or to generate electricity by tidal power. {\fontspec{DejaVu Sans}◇} \textit{they are considering a tidal barrage built across the Severn estuary} \colorBulletS{SYN} dam, weir, barrier, dyke, defence, embankment, wall, obstruction, gate, sluice \\{\fontspec{DejaVu Sans}▪ }\textsf{\textit{verb}}\\ \textbf{1} Bombard (someone) with questions, criticisms, complaints, etc. {\fontspec{DejaVu Sans}◇} \textit{his doctor was barraged with unsolicited advice}}{}{}{ \colorBullet{ORIGIN} Mid 19th century (in barrage (sense 2 of the noun)): from French, from barrer ‘to bar’, of unknown origin.}%
\par%
\entry{barring}{/ˈbɑːrɪŋ/}{ছাড়া}{ \textsf{\textit{preposition}}\ \textbf{1} Except for; if not for. {\fontspec{DejaVu Sans}◇} \textit{barring accidents, we should win} \colorBulletS{SYN} except for, with the exception of, excepting, if there are no, if there is no, bar, discounting, short of, apart from, but for, other than, aside from, excluding, omitting, leaving out, save for, saving}{}{}{ \colorBullet{ORIGIN} Late 15th century from the verb bar+ {-}ing.}%
\par%
\entry{bastard}{/ˈbɑːstəd/}{জারজ}{\small{\textsf{\textit{adjective, noun}}} \\{\fontspec{DejaVu Sans}▪ }\textsf{\textit{adjective}}\\ \textbf{1} (of a thing) no longer in its pure or original form; debased. {\fontspec{DejaVu Sans}◇} \textit{a bastard Darwinism} \colorBulletS{SYN} hybrid, alloyed \textbf{2} Born of parents not married to each other; illegitimate. {\fontspec{DejaVu Sans}◇} \textit{a bastard child} \colorBulletS{SYN} illegitimate, born out of wedlock \\{\fontspec{DejaVu Sans}▪ }\textsf{\textit{noun}}\\ \textbf{1} An unpleasant or despicable person. {\fontspec{DejaVu Sans}◇} \textit{he lied to me, the bastard!} \colorBulletS{SYN} scoundrel, villain, rogue, rascal, brute, animal, weasel, snake, monster, ogre, wretch, devil, good{-}for{-}nothing, reprobate, wrongdoer, evil{-}doer \textbf{2} A person born of parents not married to each other. {\fontspec{DejaVu Sans}◇} \textit{} \colorBulletS{SYN} illegitimate child, child born out of wedlock}{}{}{ \colorBullet{ORIGIN} Middle English via Old French from medieval Latin bastardus, probably from bastum ‘packsaddle’; compare with Old French fils de bast, ‘packsaddle son’ (i.e. the son of a mule driver who uses a packsaddle for a pillow and is gone by morning).}%
\par%
\entry{battered}{/ˈbatəd/}{ক্ষত}{ \textsf{\textit{adjective}}\ \textbf{1} Injured by repeated blows or punishment. {\fontspec{DejaVu Sans}◇} \textit{he finished the day battered and bruised}}{}{}{}%
\par%
\entry{battered}{/ˈbatəd/}{ক্ষত}{ \textsf{\textit{adjective}}\ \textbf{1} (of food) coated in batter and deep{-}fried until crisp. {\fontspec{DejaVu Sans}◇} \textit{}}{}{}{}%
\par%
\entry{bay}{/beɪ/}{উপসাগর}{ \textsf{\textit{noun}}\ \textbf{1} A broad inlet of the sea where the land curves inwards. {\fontspec{DejaVu Sans}◇} \textit{a boat trip round the bay} \colorBulletS{SYN} cove, inlet, estuary, indentation, natural harbour, gulf, basin, fjord, ria, sound, arm, bight, firth, anchorage}{}{}{ \colorBullet{ORIGIN} Late Middle English from Old French baie, from Old Spanish bahia, of unknown origin.}%
\par%
\entry{bay}{/beɪ/}{উপসাগর}{ \textsf{\textit{noun}}\ \textbf{1} An evergreen Mediterranean shrub with deep green leaves and purple berries. Its aromatic leaves are used in cooking and were formerly used to make triumphal crowns for victors. {\fontspec{DejaVu Sans}◇} \textit{}}{}{}{ \colorBullet{ORIGIN} Late Middle English (denoting the laurel berry): from Old French baie, from Latin baca ‘berry’.}%
\par%
\entry{bay}{/beɪ/}{উপসাগর}{ \textsf{\textit{noun}}\ \textbf{1} A space created by a window line projecting outwards from a wall. {\fontspec{DejaVu Sans}◇} \textit{} \colorBulletS{SYN} alcove, recess, niche, nook, cubbyhole, opening, hollow, cavity, corner, indentation, booth \textbf{2} A compartment with a specified function in a vehicle, aircraft, or ship. {\fontspec{DejaVu Sans}◇} \textit{a bomb bay}}{}{}{ \colorBullet{ORIGIN} Late Middle English from Old French baie, from baer ‘to gape’, from medieval Latin batare, of unknown origin.}%
\par%
\entry{bay}{/beɪ/}{উপসাগর}{\small{\textsf{\textit{adjective, noun}}} \\{\fontspec{DejaVu Sans}▪ }\textsf{\textit{adjective}}\\ \textbf{1} (of a horse) brown with black points. {\fontspec{DejaVu Sans}◇} \textit{} \\{\fontspec{DejaVu Sans}▪ }\textsf{\textit{noun}}\\ \textbf{1} A bay horse. {\fontspec{DejaVu Sans}◇} \textit{}}{}{}{ \colorBullet{ORIGIN} Middle English from Old French bai, from Latin badius.}%
\par%
\entry{bay}{/beɪ/}{উপসাগর}{\small{\textsf{\textit{noun, verb}}} \\{\fontspec{DejaVu Sans}▪ }\textsf{\textit{noun}}\\ \textbf{1} The sound of baying. {\fontspec{DejaVu Sans}◇} \textit{the bloodhounds' heavy bay} \colorBulletS{SYN} baying, howl, howling, bark, barking, cry, crying, growl, growling, bellow, bellowing, roar, roaring, clamour, clamouring \\{\fontspec{DejaVu Sans}▪ }\textsf{\textit{verb}}\\ \textbf{1} (of a dog, especially a large one) bark or howl loudly. {\fontspec{DejaVu Sans}◇} \textit{the dogs bayed} \colorBulletS{SYN} howl, bark, yelp, yap, cry, growl, bellow, roar, clamour, snarl}{}{}{ \colorBullet{ORIGIN} Middle English (as a noun): from Old French (a)bai (noun), (a)baiier (verb) ‘to bark’, of imitative origin.}%
\par%
\entry{bead}{/biːd/}{গুটিকা}{\small{\textsf{\textit{noun, verb}}} \\{\fontspec{DejaVu Sans}▪ }\textsf{\textit{noun}}\\ \textbf{1} A small piece of glass, stone, or similar material that is threaded with others to make a necklace or rosary or sewn on to fabric. {\fontspec{DejaVu Sans}◇} \textit{long strings of beads} \colorBulletS{SYN} ball, pellet, pill, globule, spheroid, spherule, sphere, oval, ovoid, orb, round, pearl \textbf{2} A drop of a liquid on a surface. {\fontspec{DejaVu Sans}◇} \textit{beads of sweat} \colorBulletS{SYN} droplet, drop, blob, bubble, dot, dewdrop, teardrop \textbf{3} A small knob forming the foresight of a gun. {\fontspec{DejaVu Sans}◇} \textit{} \textbf{4} The reinforced inner edge of a pneumatic tyre that grips the rim of the wheel. {\fontspec{DejaVu Sans}◇} \textit{} \textbf{5} An ornamental plaster moulding resembling a string of beads or having a semicircular cross section. {\fontspec{DejaVu Sans}◇} \textit{} \\{\fontspec{DejaVu Sans}▪ }\textsf{\textit{verb}}\\ \textbf{1} Decorate or cover with beads. {\fontspec{DejaVu Sans}◇} \textit{I beaded the jacket by hand} \textbf{2} Cover (a surface) with drops of moisture. {\fontspec{DejaVu Sans}◇} \textit{his face was beaded with perspiration}}{}{}{ \colorBullet{ORIGIN} Old English gebed ‘prayer’, of Germanic origin; related to Dutch bede and German Gebet, also to bid. Current senses derive from the use of a rosary, each bead representing a prayer.}%
\par%
\entry{bearable}{/ˈbɛːrəb(ə)l/}{সহনীয়}{ \textsf{\textit{adjective}}\ \textbf{1} Able to be endured. {\fontspec{DejaVu Sans}◇} \textit{things to make life in the tropics more bearable} \colorBulletS{SYN} tolerable, endurable, supportable, sufferable, brookable, sustainable}{}{}{}%
\par%
\entry{beaver}{/ˈbiːvə/}{বীবর}{\small{\textsf{\textit{noun, verb}}} \\{\fontspec{DejaVu Sans}▪ }\textsf{\textit{noun}}\\ \textbf{1} A large semiaquatic broad{-}tailed rodent native to North America and northern Eurasia. It is noted for its habit of gnawing through trees to fell them in order to make dams. {\fontspec{DejaVu Sans}◇} \textit{} \textbf{2}  {\fontspec{DejaVu Sans}◇} \textit{} \\{\fontspec{DejaVu Sans}▪ }\textsf{\textit{verb}}\\ \textbf{1} Work hard. {\fontspec{DejaVu Sans}◇} \textit{Bridget beavered away to keep things running smoothly} \colorBulletS{SYN} work hard, toil, labour, work one's fingers to the bone, work like a dog, work like a Trojan, work day and night, exert oneself, keep at it, keep one's nose to the grindstone, grind, slave, grub, plough, plod, peg}{}{}{ \colorBullet{ORIGIN} Old English beofor, befor, of Germanic origin; related to Dutch bever and German Biber, from an Indo{-}European root meaning ‘brown’.}%
\par%
\entry{beaver}{/ˈbiːvə/}{বীবর}{ \textsf{\textit{noun}}\ \textbf{1} The lower part of the face guard of a helmet in a suit of armour. The term is also used to refer to the upper part or visor, or to a single movable guard. {\fontspec{DejaVu Sans}◇} \textit{The ghost wears the beaver, or visor, of the helmet raised.}}{}{}{ \colorBullet{ORIGIN} Late 15th century from Old French baviere ‘bib’, from baver ‘slaver’.}%
\par%
\entry{beaver}{/ˈbiːvə/}{বীবর}{ \textsf{\textit{noun}}\ \textbf{1} A woman's genitals or pubic area. {\fontspec{DejaVu Sans}◇} \textit{} \textbf{2} A bearded man. {\fontspec{DejaVu Sans}◇} \textit{Skittish young girls would rush up to a bearded man in the street and tug his beard, yelling “Beaver!”.}}{}{}{ \colorBullet{ORIGIN} Early 20th century of unknown origin.}%
\par%
\entry{befall}{/bɪˈfɔːl/}{ঘটা}{ \textsf{\textit{verb}}\ \textbf{1} (especially of something bad) happen to (someone) {\fontspec{DejaVu Sans}◇} \textit{a tragedy befell his daughter} \colorBulletS{SYN} happen to, overtake, come upon, fall upon, hit, strike, be visited on}{}{}{ \colorBullet{ORIGIN} Old English befeallan ‘to fall’ (early use being chiefly figurative); related to German befallen.}%
\par%
\entry{befitting}{/bɪˈfɪtɪŋ/}{যুগোপযোগী}{ \textsf{\textit{adjective}}\ \textbf{1} Appropriate to the occasion. {\fontspec{DejaVu Sans}◇} \textit{a country which can run the prestigious tournament in a befitting manner}}{}{}{}%
\par%
\entry{befuddle}{/bɪˈfʌd(ə)l/}{বেহেড করা}{ \textsf{\textit{verb}}\ \textbf{1} Cause to become unable to think clearly. {\fontspec{DejaVu Sans}◇} \textit{even in my befuddled state I could see that they meant trouble} \colorBulletS{SYN} confused, muddled, addled, bewildered, disoriented, disorientated, all at sea, mixed up, fazed, perplexed, stunned, dazed, dizzy, stupefied, groggy, foggy, fuzzy, fuddled, benumbed, numbed, numb, vague}{}{}{}%
\par%
\entry{behave}{/bɪˈheɪv/}{আচরণ করা}{ \textsf{\textit{verb}}\ \textbf{1} Act or conduct oneself in a specified way, especially towards others. {\fontspec{DejaVu Sans}◇} \textit{he always behaved like a gentleman} \colorBulletS{SYN} conduct oneself, act, acquit oneself, bear oneself, carry oneself \textbf{2} Conduct oneself in accordance with the accepted norms of a society or group. {\fontspec{DejaVu Sans}◇} \textit{‘Just behave, Tom,’ he said} \colorBulletS{SYN} act correctly, act properly, conduct oneself well, act in a polite way, show good manners, mind one's manners, mind one's Ps and Qs}{}{}{ \colorBullet{ORIGIN} Late Middle English from be{-}‘thoroughly’ + have in the sense ‘have or bear (oneself) in a particular way’.}%
\par%
\entry{belligerent}{/bəˈlɪdʒ(ə)r(ə)nt/}{যুধ্যমান}{\small{\textsf{\textit{adjective, noun}}} \\{\fontspec{DejaVu Sans}▪ }\textsf{\textit{adjective}}\\ \textbf{1} Hostile and aggressive. {\fontspec{DejaVu Sans}◇} \textit{the mood at the meeting was belligerent} \colorBulletS{SYN} hostile, aggressive, threatening, antagonistic, pugnacious, bellicose, truculent, confrontational, argumentative, quarrelsome, disputatious, contentious, militant, combative \\{\fontspec{DejaVu Sans}▪ }\textsf{\textit{noun}}\\ \textbf{1} A nation or person engaged in war or conflict, as recognized by international law. {\fontspec{DejaVu Sans}◇} \textit{ships and goods captured at sea by a belligerent} \colorBulletS{SYN} militarist, hawk, jingoist, sabre{-}rattler, aggressor, provoker, belligerent}{}{}{ \colorBullet{ORIGIN} Late 16th century from Latin belligerant{-} ‘waging war’, from the verb belligerare, from bellum ‘war’.}%
\par%
\entry{benevolent}{/bɪˈnɛv(ə)l(ə)nt/}{হিতৈষী}{ \textsf{\textit{adjective}}\ \textbf{1} Well meaning and kindly. {\fontspec{DejaVu Sans}◇} \textit{he was something of a benevolent despot} \colorBulletS{SYN} kind, kindly, kind{-}hearted, warm{-}hearted, tender{-}hearted, big{-}hearted, good{-}natured, good, gracious, tolerant, benign, compassionate, caring, sympathetic, considerate, thoughtful, well meaning, obliging, accommodating, helpful, decent, neighbourly, public{-}spirited, charitable, altruistic, humane, humanitarian, philanthropic}{}{}{ \colorBullet{ORIGIN} Late Middle English from Old French benivolent, from Latin bene volent{-} ‘well wishing’, from bene ‘well’ + velle ‘to wish’.}%
\par%
\entry{berth}{/bəːθ/}{নোঙ্গরস্থান}{\small{\textsf{\textit{noun, verb}}} \\{\fontspec{DejaVu Sans}▪ }\textsf{\textit{noun}}\\ \textbf{1} A ship's allotted place at a wharf or dock. {\fontspec{DejaVu Sans}◇} \textit{the vessel had left its berth} \colorBulletS{SYN} docking site, anchorage, mooring \textbf{2} A fixed bunk on a ship, train, or other means of transport. {\fontspec{DejaVu Sans}◇} \textit{I'll sleep in the upper berth} \colorBulletS{SYN} bunk, bed, bunk bed, cot, couch, hammock \textbf{3} (often in a sporting context) a position in an organization or event. {\fontspec{DejaVu Sans}◇} \textit{he looked at home in an unfamiliar right{-}back berth} \\{\fontspec{DejaVu Sans}▪ }\textsf{\textit{verb}}\\ \textbf{1} Moor (a ship) in its allotted place. {\fontspec{DejaVu Sans}◇} \textit{they planned to berth HMS Impregnable at Portsmouth} \colorBulletS{SYN} moor, berth, harbour, be at anchor, tie up \textbf{2} (of a passenger ship) provide a sleeping place for (someone). {\fontspec{DejaVu Sans}◇} \textit{} \colorBulletS{SYN} accommodate, sleep, provide beds for, put up, house, shelter, lodge}{}{}{ \colorBullet{ORIGIN} Early 17th century (in the sense ‘adequate sea room’): probably from a nautical use of bear+ {-}th.}%
\par%
\entry{beyond}{/bɪˈjɒnd/}{তার পরেও}{\small{\textsf{\textit{noun, preposition \& adverb}}} \\{\fontspec{DejaVu Sans}▪ }\textsf{\textit{noun}}\\ \textbf{1} The unknown, especially in references to life after death. {\fontspec{DejaVu Sans}◇} \textit{messages from the beyond} \\{\fontspec{DejaVu Sans}▪ }\textsf{\textit{preposition \& adverb}}\\ \textbf{1} At or to the further side of. {\fontspec{DejaVu Sans}◇} \textit{he pointed to a spot beyond the concealing trees} \colorBulletS{SYN} on the far side of, on the farther side of, on the other side of, further on than, behind, past, after \textbf{2} Happening or continuing after (a specified time, stage, or event) {\fontspec{DejaVu Sans}◇} \textit{training beyond the age of 14} \colorBulletS{SYN} later than, past, after \textbf{3} Having progressed or achieved more than (a specified stage or level) {\fontspec{DejaVu Sans}◇} \textit{we need to get beyond square one} \textbf{4} To or in a degree or condition where a specified action is impossible. {\fontspec{DejaVu Sans}◇} \textit{the landscape has changed beyond recognition} \colorBulletS{SYN} outside the range of, beyond the capacity of, beyond the power of, outside the limitations of, surpassing \textbf{5} Apart from; except. {\fontspec{DejaVu Sans}◇} \textit{beyond telling us that she was well educated, he has nothing to say about her} \colorBulletS{SYN} apart from, except, other than}{}{}{ \colorBullet{ORIGIN} Old English begeondan, from be ‘by’ + geondan of Germanic origin (related to yon and yonder).}%
\par%
\entry{bid}{/bɪd/}{বিদার প্রস্তাব}{\small{\textsf{\textit{noun, verb}}} \\{\fontspec{DejaVu Sans}▪ }\textsf{\textit{noun}}\\ \textbf{1} An offer of a price, especially at an auction. {\fontspec{DejaVu Sans}◇} \textit{at the fur tables, several buyers make bids for the pelts} \colorBulletS{SYN} offer, tender, proposal, submission \textbf{2} An attempt or effort to achieve something. {\fontspec{DejaVu Sans}◇} \textit{he made a bid for power in 1984} \colorBulletS{SYN} attempt, effort, endeavour, try \\{\fontspec{DejaVu Sans}▪ }\textsf{\textit{verb}}\\ \textbf{1} Offer (a certain price) for something, especially at an auction. {\fontspec{DejaVu Sans}◇} \textit{a consortium of dealers bid a world record price for a snuff box} \colorBulletS{SYN} offer, make an offer of, put in a bid of, put up, tender, proffer, propose, submit, put forward, advance \textbf{2} Make an effort or attempt to achieve. {\fontspec{DejaVu Sans}◇} \textit{she's now bidding to become a top female model} \colorBulletS{SYN} try to obtain, try to get, make a pitch for, make a bid for}{}{}{ \colorBullet{ORIGIN} Old English bēodan ‘to offer, command’, of Germanic origin; related to Dutch bieden and German bieten.}%
\par%
\entry{bid}{/bɪd/}{বিদার প্রস্তাব}{ \textsf{\textit{verb}}\ \textbf{1} Utter (a greeting or farewell) to. {\fontspec{DejaVu Sans}◇} \textit{James bade a tearful farewell to his parents} \colorBulletS{SYN} wish \textbf{2} Command or order (someone) to do something. {\fontspec{DejaVu Sans}◇} \textit{I did as he bade me} \colorBulletS{SYN} order, command, tell, instruct, direct, require, enjoin, charge, demand, call upon}{}{}{ \colorBullet{ORIGIN} Old English biddan ‘ask’, of Germanic origin; related to German bitten.}%
\par%
\entry{bladder}{/ˈbladə/}{}{ \textsf{\textit{noun}}\ \textbf{1} A muscular membranous sac in the abdomen which receives urine from the kidneys and stores it for excretion. {\fontspec{DejaVu Sans}◇} \textit{patients were asked to empty their bladders before going to bed} \textbf{2} An inflated or hollow flexible bag or chamber. {\fontspec{DejaVu Sans}◇} \textit{a dried bladder of seaweed} \colorBulletS{SYN} bag, pouch, bladder, blister}{}{I have to void my bladder}{ \colorBullet{ORIGIN} Old English blǣdre, of Germanic origin; related to Dutch blaar and German Blatter, also to blow.}%
\par%
\entry{blanket}{/ˈblaŋkɪt/}{কম্বল}{\small{\textsf{\textit{adjective, noun, verb}}} \\{\fontspec{DejaVu Sans}▪ }\textsf{\textit{adjective}}\\ \textbf{1} Covering all cases or instances; total and inclusive. {\fontspec{DejaVu Sans}◇} \textit{a blanket ban on tobacco advertising} \colorBulletS{SYN} wholesale, across the board, outright, indiscriminate, overall, general, mass, umbrella, inclusive, all{-}inclusive, all{-}round, sweeping, total, complete, comprehensive, thorough, extensive, wide{-}ranging, far{-}reaching, large{-}scale, widespread \\{\fontspec{DejaVu Sans}▪ }\textsf{\textit{noun}}\\ \textbf{1} A large piece of woollen or similar material used as a covering on a bed or elsewhere for warmth. {\fontspec{DejaVu Sans}◇} \textit{I slept on the ground covered by my blanket} \colorBulletS{SYN} cover, covering, rug, afghan, quilt, eiderdown, duvet \textbf{2} A rubber surface used for transferring the image in ink from the plate to the paper in offset printing. {\fontspec{DejaVu Sans}◇} \textit{} \\{\fontspec{DejaVu Sans}▪ }\textsf{\textit{verb}}\\ \textbf{1} Cover completely with a thick layer of something. {\fontspec{DejaVu Sans}◇} \textit{the countryside was blanketed in snow} \colorBulletS{SYN} cover, coat, carpet, overlay, overlie, overspread, extend over, cap, top, crown \textbf{2} Take wind from the sails of (another craft) by passing to windward. {\fontspec{DejaVu Sans}◇} \textit{That should blanket the spinnaker behind the mainsail so that there is very little pressure on it.}}{}{}{ \colorBullet{ORIGIN} Middle English (denoting undyed woollen cloth): via Old Northern French from Old French blanc ‘white’, ultimately of Germanic origin.}%
\par%
\entry{blast}{/blɑːst/}{বিস্ফোরণ}{\small{\textsf{\textit{exclamation, noun, verb}}} \\{\fontspec{DejaVu Sans}▪ }\textsf{\textit{exclamation}}\\ \textbf{1} Expressing annoyance. {\fontspec{DejaVu Sans}◇} \textit{‘Blast! The car won't start!’} \colorBulletS{SYN} damn, damnation, blast, hell, heck, Gordon Bennett \\{\fontspec{DejaVu Sans}▪ }\textsf{\textit{noun}}\\ \textbf{1} A destructive wave of highly compressed air spreading outwards from an explosion. {\fontspec{DejaVu Sans}◇} \textit{they were thrown backwards by the blast} \colorBulletS{SYN} shock wave, pressure wave, bang, crash, crack \textbf{2} A strong gust of wind or air. {\fontspec{DejaVu Sans}◇} \textit{the icy blast hit them} \colorBulletS{SYN} gust, rush, blow, gale, squall, storm, wind, draught, waft, puff, flurry, breeze \textbf{3} A single loud note of a horn, whistle, or similar. {\fontspec{DejaVu Sans}◇} \textit{a blast of the ship's siren} \colorBulletS{SYN} blare, blaring, honk, bellow, boom, roar, screech, wail \textbf{4} A severe reprimand. {\fontspec{DejaVu Sans}◇} \textit{I braced myself for the inevitable blast} \colorBulletS{SYN} reprimand, rebuke, reproof, admonishment, admonition, reproach, reproval, scolding, remonstration, upbraiding, castigation, lambasting, lecture, criticism, censure \textbf{5} An enjoyable experience or lively party. {\fontspec{DejaVu Sans}◇} \textit{it could turn out to be a real blast} \colorBulletS{SYN} social gathering, gathering, social occasion, social event, social function, function, get{-}together, celebration, reunion, festivity, jamboree, reception, at{-}home, soirée, social \\{\fontspec{DejaVu Sans}▪ }\textsf{\textit{verb}}\\ \textbf{1} Blow up or break apart (something solid) with explosives. {\fontspec{DejaVu Sans}◇} \textit{the school was blasted by an explosion} \colorBulletS{SYN} blow up, bomb, blow, blow to pieces, dynamite, explode \textbf{2} Produce or cause to produce loud continuous music or other noise. {\fontspec{DejaVu Sans}◇} \textit{music blasted out at full volume} \colorBulletS{SYN} honk, sound loudly, trumpet, blare, boom, roar \textbf{3} Kick or strike (a ball) hard. {\fontspec{DejaVu Sans}◇} \textit{the striker blasted the free kick into the net} \textbf{4} Criticize fiercely. {\fontspec{DejaVu Sans}◇} \textit{the school was blasted by government inspectors} \colorBulletS{SYN} reprimand, rebuke, criticize, upbraid, berate, castigate, reprove, rail at, flay \textbf{5} (of a wind or other natural force) wither, shrivel, or blight (a plant) {\fontspec{DejaVu Sans}◇} \textit{corn blasted before it be grown up} \colorBulletS{SYN} blight, kill, destroy, wither, shrivel}{}{}{ \colorBullet{ORIGIN} Old English blǣst, of Germanic origin; related to blaze.}%
\par%
\entry{blatant}{/ˈbleɪt(ə)nt/}{স্থূল}{ \textsf{\textit{adjective}}\ \textbf{1} (of bad behaviour) done openly and unashamedly. {\fontspec{DejaVu Sans}◇} \textit{blatant lies} \colorBulletS{SYN} flagrant, glaring, obvious, undisguised, unconcealed, overt, open, transparent, patent, evident, manifest, palpable, unmistakable}{}{Blatant abuse of power}{ \colorBullet{ORIGIN} Late 16th century perhaps an alteration of Scots blatand ‘bleating’. It was first used by Spenser as an epithet for a thousand{-}tongued monster produced by Cerberus and Chimaera, a symbol of calumny, which he called the blatant beast. It was subsequently used to mean ‘clamorous, offensive to the ear’, first of people (mid 17th century), later of things (late 18th century); the sense ‘unashamedly conspicuous’ arose in the late 19th century.}%
\par%
\entry{bleak}{/bliːk/}{নিরানন্দ}{ \textsf{\textit{adjective}}\ \textbf{1} (of an area of land) lacking vegetation and exposed to the elements. {\fontspec{DejaVu Sans}◇} \textit{a bleak and barren moor} \colorBulletS{SYN} bare, exposed, desolate, stark, arid, desert, denuded, lunar, open, empty, windswept}{}{}{ \colorBullet{ORIGIN} Old English blāc ‘shining, white’, or in later use from synonymous Old Norse bleikr; ultimately of Germanic origin and related to bleach.}%
\par%
\entry{bleak}{/bliːk/}{নিরানন্দ}{ \textsf{\textit{noun}}\ \textbf{1} A small silvery shoaling fish of the carp family, found in Eurasian rivers. {\fontspec{DejaVu Sans}◇} \textit{}}{}{}{ \colorBullet{ORIGIN} Late 15th century from Old Norse bleikja.}%
\par%
\entry{blindfold}{/ˈblʌɪn(d)fəʊld/}{বেপরোয়াভাবে}{\small{\textsf{\textit{adjective, adverb, noun, verb}}} \\{\fontspec{DejaVu Sans}▪ }\textsf{\textit{adjective}}\\ \textbf{1} Wearing a blindfold. {\fontspec{DejaVu Sans}◇} \textit{} \\{\fontspec{DejaVu Sans}▪ }\textsf{\textit{adverb}}\\ \textbf{1} With a blindfold covering the eyes. {\fontspec{DejaVu Sans}◇} \textit{the reporter was driven blindfold to meet the gangster} \\{\fontspec{DejaVu Sans}▪ }\textsf{\textit{noun}}\\ \textbf{1} A piece of cloth tied round the head to cover someone's eyes. {\fontspec{DejaVu Sans}◇} \textit{} \\{\fontspec{DejaVu Sans}▪ }\textsf{\textit{verb}}\\ \textbf{1} Deprive (someone) of sight by tying a piece of cloth round the head so as to cover the eyes. {\fontspec{DejaVu Sans}◇} \textit{he was blindfolded and trussed up in a cupboard}}{}{}{ \colorBullet{ORIGIN} Mid 16th century alteration, by association with fold, of blindfeld, past participle of obsolete blindfell ‘strike blind, blindfold’, from Old English geblindfellan(see blind, fell).}%
\par%
\entry{blond}{/bländ/}{স্বর্ণকেশী}{\small{\textsf{\textit{adjective, noun}}} \\{\fontspec{DejaVu Sans}▪ }\textsf{\textit{adjective}}\\ \textbf{1} (of hair) fair or pale yellow. {\fontspec{DejaVu Sans}◇} \textit{short{-}cropped blond hair} \colorBulletS{SYN} fair, light, light{-}coloured, light{-}toned, yellow, flaxen, tow{-}coloured, strawberry blonde, yellowish, golden, silver, silvery, platinum, ash blonde \\{\fontspec{DejaVu Sans}▪ }\textsf{\textit{noun}}\\ \textbf{1} A person with fair or pale yellow hair (typically used of a woman). {\fontspec{DejaVu Sans}◇} \textit{}}{}{}{ \colorBullet{ORIGIN} Late 15th century from French blond, blonde, from medieval Latin blundus ‘yellow’, perhaps from Germanic.}%
\par%
\entry{bluff}{/blʌf/}{শুধু ধমকি}{\small{\textsf{\textit{noun, verb}}} \\{\fontspec{DejaVu Sans}▪ }\textsf{\textit{noun}}\\ \textbf{1} An attempt to deceive someone into believing that one can or is going to do something. {\fontspec{DejaVu Sans}◇} \textit{the offer was denounced as a bluff} \colorBulletS{SYN} deception, subterfuge, pretence, sham, fake, show, deceit, false show, idle boast, feint, delusion, hoax, fraud, masquerade, charade \\{\fontspec{DejaVu Sans}▪ }\textsf{\textit{verb}}\\ \textbf{1} Try to deceive someone as to one's abilities or intentions. {\fontspec{DejaVu Sans}◇} \textit{he's been bluffing all along} \colorBulletS{SYN} pretend, sham, fake, feign, put on an act, put it on, lie, hoax, pose, posture, masquerade, dissemble, dissimulate}{}{}{ \colorBullet{ORIGIN} Late 17th century (originally in the sense ‘blindfold, hoodwink’): from Dutch bluffen ‘brag’, or bluf ‘bragging’. The current sense (originally US, mid 19th century) originally referred to bluffing in the game of poker.}%
\par%
\entry{bluff}{/blʌf/}{শুধু ধমকি}{ \textsf{\textit{adjective}}\ \textbf{1} Direct in speech or behaviour but in a good{-}natured way. {\fontspec{DejaVu Sans}◇} \textit{a big, bluff, hearty man} \colorBulletS{SYN} plain{-}spoken, straightforward, blunt, direct, no{-}nonsense, frank, open, candid, outspoken, to the point, forthright, unequivocal, downright, hearty}{}{}{ \colorBullet{ORIGIN} Early 18th century (in the sense ‘surly, abrupt in manner’): figurative use of bluff. The current positive connotation dates from the early 19th century.}%
\par%
\entry{bluff}{/blʌf/}{শুধু ধমকি}{\small{\textsf{\textit{adjective, noun}}} \\{\fontspec{DejaVu Sans}▪ }\textsf{\textit{adjective}}\\ \textbf{1} (of a cliff or a ship's bows) having a vertical or steep broad front. {\fontspec{DejaVu Sans}◇} \textit{} \\{\fontspec{DejaVu Sans}▪ }\textsf{\textit{noun}}\\ \textbf{1} A steep cliff, bank, or promontory. {\fontspec{DejaVu Sans}◇} \textit{} \colorBulletS{SYN} cliff, ridge, promontory, headland, crag, bank, slope, height, peak, escarpment, scarp, precipice, rock face, overhang \textbf{2} A grove or clump of trees. {\fontspec{DejaVu Sans}◇} \textit{}}{}{}{ \colorBullet{ORIGIN} Early 17th century (as an adjective, originally in nautical use): of unknown origin.}%
\par%
\entry{blunt}{/blʌnt/}{ভোঁতা}{\small{\textsf{\textit{adjective, noun, verb}}} \\{\fontspec{DejaVu Sans}▪ }\textsf{\textit{adjective}}\\ \textbf{1} (of a cutting implement) not having a sharp edge or point. {\fontspec{DejaVu Sans}◇} \textit{a blunt knife} \colorBulletS{SYN} not sharp, unsharpened, dull, dulled, worn, worn down, edgeless \textbf{2} (of a person or remark) uncompromisingly forthright. {\fontspec{DejaVu Sans}◇} \textit{a blunt statement of fact} \colorBulletS{SYN} straightforward, frank, plain{-}spoken, candid, direct, bluff, to the point, forthright, unequivocal, point{-}blank, unceremonious, undiplomatic, indelicate \\{\fontspec{DejaVu Sans}▪ }\textsf{\textit{noun}}\\ \textbf{1} A hollowed{-}out cigar filled with cannabis. {\fontspec{DejaVu Sans}◇} \textit{} \colorBulletS{SYN} cannabis cigarette, marijuana cigarette \\{\fontspec{DejaVu Sans}▪ }\textsf{\textit{verb}}\\ \textbf{1} Make or become less sharp. {\fontspec{DejaVu Sans}◇} \textit{wood can blunt your axe} \colorBulletS{SYN} make less sharp, make blunt, make dull}{}{}{ \colorBullet{ORIGIN} Middle English (in the sense ‘dull, insensitive’): perhaps of Scandinavian origin and related to Old Norse blunda ‘shut the eyes’.}%
\par%
\entry{boast}{/bəʊst/}{দর্প}{\small{\textsf{\textit{noun, verb}}} \\{\fontspec{DejaVu Sans}▪ }\textsf{\textit{noun}}\\ \textbf{1} An act of talking with excessive pride and self{-}satisfaction. {\fontspec{DejaVu Sans}◇} \textit{I said I would win and it wasn't an idle boast} \colorBulletS{SYN} brag, self{-}praise \\{\fontspec{DejaVu Sans}▪ }\textsf{\textit{verb}}\\ \textbf{1} Talk with excessive pride and self{-}satisfaction about one's achievements, possessions, or abilities. {\fontspec{DejaVu Sans}◇} \textit{she boasted about her many conquests} \colorBulletS{SYN} brag, crow, swagger, swank, gloat, show off, blow one's own trumpet, sing one's own praises, congratulate oneself, pat oneself on the back \textbf{2} (of a person, place, or thing) possess (a feature that is a source of pride) {\fontspec{DejaVu Sans}◇} \textit{the hotel boasts high standards of comfort} \colorBulletS{SYN} possess, have, own, enjoy, pride itself on, pride oneself on, be the proud owner of}{}{}{ \colorBullet{ORIGIN} Middle English (as a noun): of unknown origin.}%
\par%
\entry{boast}{/bəʊst/}{দর্প}{ \textsf{\textit{noun}}\ \textbf{1} (in squash) a stroke in which the ball is made to hit one of the side walls before hitting the front wall. {\fontspec{DejaVu Sans}◇} \textit{}}{}{}{ \colorBullet{ORIGIN} Late 19th century perhaps from French bosse denoting a rounded projection in the wall of a court for real tennis.}%
\par%
\entry{bodacious}{/bəʊˈdeɪʃəs/}{very large or important, or something that people enjoy or admire}{ \textsf{\textit{adjective}}\ \textbf{1} Excellent, admirable, or attractive. {\fontspec{DejaVu Sans}◇} \textit{bodacious babes} \colorBulletS{SYN} delightful, pleasing, pleasant, agreeable, likeable, endearing, lovely, lovable, adorable, cute, sweet, appealing, attractive, good{-}looking, prepossessing}{}{It was a bodacious concert!}{ \colorBullet{ORIGIN} Mid 19th century (in sense ‘complete, thorough’): perhaps a variant of SW dialect boldacious, blend of bold and audacious.}%
\par%
\entry{bombard}{/bɒmˈbɑːd/}{বোমা ছুড়িয়া মারা}{\small{\textsf{\textit{noun, verb}}} \\{\fontspec{DejaVu Sans}▪ }\textsf{\textit{noun}}\\ \textbf{1} A cannon of the earliest type, which fired a stone ball or large shot. {\fontspec{DejaVu Sans}◇} \textit{} \\{\fontspec{DejaVu Sans}▪ }\textsf{\textit{verb}}\\ \textbf{1} Attack (a place or person) continuously with bombs, shells, or other missiles. {\fontspec{DejaVu Sans}◇} \textit{the city was bombarded by federal forces} \colorBulletS{SYN} shell, torpedo, pound, blitz, strafe, pepper, fire at, fire on, bomb}{}{}{ \colorBullet{ORIGIN} Late Middle English (as a noun denoting an early form of cannon, also a shawm) from Old French bombarde, probably based on Latin bombus ‘booming, humming’ (see bomb). The verb (late 16th century) is from French bombarder.}%
\par%
\entry{boo}{/buː/}{ছি{-}ছি}{\small{\textsf{\textit{exclamation, noun, verb}}} \\{\fontspec{DejaVu Sans}▪ }\textsf{\textit{exclamation}}\\ \textbf{1} Said suddenly to surprise someone who is unaware of one's presence. {\fontspec{DejaVu Sans}◇} \textit{‘Boo!’ she cried, jumping up to frighten him} \textbf{2} Said to show disapproval or contempt. {\fontspec{DejaVu Sans}◇} \textit{‘There's only one bar.’ ‘Boo!’} \\{\fontspec{DejaVu Sans}▪ }\textsf{\textit{noun}}\\ \textbf{1} An utterance of ‘boo’ to show disapproval of a speaker or performer. {\fontspec{DejaVu Sans}◇} \textit{the audience greeted this comment with boos and hisses} \colorBulletS{SYN} shout, yell, cry, howl, scream, shriek, whoop, whistle \\{\fontspec{DejaVu Sans}▪ }\textsf{\textit{verb}}\\ \textbf{1} Say ‘boo’ to show disapproval of a speaker or performer. {\fontspec{DejaVu Sans}◇} \textit{they booed and hissed when he stepped on stage} \colorBulletS{SYN} taunt, mock, scoff at, ridicule, laugh at, sneer at, deride, tease, insult, abuse, jibe, jibe at, scorn, shout disapproval, shout disapproval at}{}{}{ \colorBullet{ORIGIN} Early 19th century (in boo (sense 2 of the exclamation)): imitative of the lowing of oxen.}%
\par%
\entry{boo}{/buː/}{ছি{-}ছি}{ \textsf{\textit{noun}}\ \textbf{1} A person's boyfriend or girlfriend. {\fontspec{DejaVu Sans}◇} \textit{}}{}{}{ \colorBullet{ORIGIN} 1980s origin uncertain; probably an alteration of French beau ‘boyfriend, male admirer’.}%
\par%
\entry{borrow}{/ˈbɒrəʊ/}{ধার করা}{\small{\textsf{\textit{noun, verb}}} \\{\fontspec{DejaVu Sans}▪ }\textsf{\textit{noun}}\\ \textbf{1} A slope or other irregularity on a golf course which must be compensated for when playing a shot. {\fontspec{DejaVu Sans}◇} \textit{} \\{\fontspec{DejaVu Sans}▪ }\textsf{\textit{verb}}\\ \textbf{1} Take and use (something belonging to someone else) with the intention of returning it. {\fontspec{DejaVu Sans}◇} \textit{he had borrowed a car from one of his colleagues} \colorBulletS{SYN} take, take for oneself, help oneself to, use as one's own, abscond with, carry off, appropriate, commandeer, abstract \textbf{2} Allow (a certain distance) when playing a shot to compensate for sideways motion of the ball due to a slope or other irregularity. {\fontspec{DejaVu Sans}◇} \textit{}}{}{}{ \colorBullet{ORIGIN} Old English borgian ‘borrow against security’, of Germanic origin; related to Dutch and German borgen.}%
\par%
\entry{borrower}{/ˈbɒrəʊə/}{অধমর্ণ}{ \textsf{\textit{noun}}\ \textbf{1} A person or organization that takes and uses something belonging to someone else with the intention of returning it. {\fontspec{DejaVu Sans}◇} \textit{my last pair of secateurs were ruined by a careless borrower}}{}{}{}%
\par%
\entry{bosom}{/ˈbʊz(ə)m/}{বক্ষ}{\small{\textsf{\textit{adjective, noun}}} \\{\fontspec{DejaVu Sans}▪ }\textsf{\textit{adjective}}\\ \textbf{1} (of a friend) very close or intimate. {\fontspec{DejaVu Sans}◇} \textit{the two girls had become bosom friends} \colorBulletS{SYN} close, boon, intimate, confidential, inseparable, faithful, constant, devoted, loving \\{\fontspec{DejaVu Sans}▪ }\textsf{\textit{noun}}\\ \textbf{1} A woman's chest or breasts. {\fontspec{DejaVu Sans}◇} \textit{her ample bosom} \colorBulletS{SYN} bust, chest}{}{}{ \colorBullet{ORIGIN} Old English bōsm, of West Germanic origin; related to Dutch boezem and German Busen.}%
\par%
\entry{bouquet}{/bʊˈkeɪ/}{ফুলের তোড়া}{ \textsf{\textit{noun}}\ \textbf{1} An attractively arranged bunch of flowers, especially one presented as a gift or carried at a ceremony. {\fontspec{DejaVu Sans}◇} \textit{} \colorBulletS{SYN} bunch of flowers, posy, nosegay, spray, sprig \textbf{2} The characteristic scent of a wine or perfume. {\fontspec{DejaVu Sans}◇} \textit{the aperitif has a faint bouquet of almonds} \colorBulletS{SYN} aroma, nose, smell, fragrance, perfume, scent, odour, redolence, whiff, tang, savour}{}{}{ \colorBullet{ORIGIN} Early 18th century from French (earlier ‘clump of trees’), from a dialect variant of Old French bos ‘wood’. bouquet (sense 2) dates from the mid 19th century.}%
\par%
\entry{bow}{/bəʊ/}{নম}{\small{\textsf{\textit{noun, verb}}} \\{\fontspec{DejaVu Sans}▪ }\textsf{\textit{noun}}\\ \textbf{1} A knot tied with two loops and two loose ends, used especially for tying shoelaces and decorative ribbons. {\fontspec{DejaVu Sans}◇} \textit{a girl with long hair tied back in a bow} \colorBulletS{SYN} loop, knot \textbf{2} A weapon for shooting arrows, typically made of a curved piece of wood joined at both ends by a taut string. {\fontspec{DejaVu Sans}◇} \textit{} \colorBulletS{SYN} longbow, crossbow, recurve \textbf{3} A long, partially curved rod with horsehair stretched along its length, used for playing the violin and other stringed instruments. {\fontspec{DejaVu Sans}◇} \textit{} \textbf{4} A curved stroke forming part of a letter (e.g. b, p). {\fontspec{DejaVu Sans}◇} \textit{} \textbf{5} A metal ring forming the handle of a key or pair of scissors. {\fontspec{DejaVu Sans}◇} \textit{} \\{\fontspec{DejaVu Sans}▪ }\textsf{\textit{verb}}\\ \textbf{1} Play (a stringed instrument or music) using a bow. {\fontspec{DejaVu Sans}◇} \textit{the techniques by which the pieces were bowed}}{}{}{ \colorBullet{ORIGIN} Old English boga ‘bend, bow, arch’, of Germanic origin; related to Dutch boog and German Bogen, also to bow.}%
\par%
\entry{bow}{/baʊ/}{নম}{\small{\textsf{\textit{noun, verb}}} \\{\fontspec{DejaVu Sans}▪ }\textsf{\textit{noun}}\\ \textbf{1} An act of bending the head or upper body as a sign of respect or greeting. {\fontspec{DejaVu Sans}◇} \textit{the man gave a little bow} \colorBulletS{SYN} inclination, obeisance, nod, curtsy, bob, salaam, salutation \\{\fontspec{DejaVu Sans}▪ }\textsf{\textit{verb}}\\ \textbf{1} Bend the head or upper part of the body as a sign of respect, greeting, or shame. {\fontspec{DejaVu Sans}◇} \textit{he turned and bowed to his father} \colorBulletS{SYN} incline the body, incline the head, make an obeisance, make a bow, nod, curtsy, drop a curtsy, bob, salaam, genuflect, bend the knee, kowtow \textbf{2} Bend with age or under pressure. {\fontspec{DejaVu Sans}◇} \textit{the roof trusses bowed as the wind fought to rip the roof free} \textbf{3} (of a new film or product) be premiered or launched. {\fontspec{DejaVu Sans}◇} \textit{the trailer bowed in theaters nationwide on December 23}}{}{}{ \colorBullet{ORIGIN} Old English būgan ‘bend, stoop’, of Germanic origin; related to German biegen, also to bow.}%
\par%
\entry{bow}{/baʊ/}{নম}{ \textsf{\textit{noun}}\ \textbf{1} The front end of a ship. {\fontspec{DejaVu Sans}◇} \textit{water sprayed high over her bows} \colorBulletS{SYN} prow, front, forepart, stem, rostrum, ram, nose, head, bowsprit, cutwater}{}{}{ \colorBullet{ORIGIN} Late Middle English from Low German boog, Dutch boeg, ‘shoulder or ship's bow’; related to bough.}%
\par%
\entry{bowel}{/ˈbaʊəl/}{অন্ত্র}{ \textsf{\textit{noun}}\ \textbf{1}  {\fontspec{DejaVu Sans}◇} \textit{he felt his bowels loosen} \colorBulletS{SYN} intestine, intestines, small intestine, large intestine, colon}{}{}{ \colorBullet{ORIGIN} Middle English from Old French bouel, from Latin botellus, diminutive of botulus ‘sausage’.}%
\par%
\entry{bowel movement}{}{অন্ত্র{-}আন্দোলন; an act of passing usually solid waste through the rectum and anus}{\small{\textsf{\textit{}}}}{}{He had a two{-}day history of right lower abdominal pain… without bowel movements.}{}%
\par%
\entry{brace}{/breɪs/}{যুগল}{\small{\textsf{\textit{noun, verb}}} \\{\fontspec{DejaVu Sans}▪ }\textsf{\textit{noun}}\\ \textbf{1} A device fitted to something, in particular a weak or injured part of the body, to give support. {\fontspec{DejaVu Sans}◇} \textit{a neck brace} \colorBulletS{SYN} support, caliper, truss, surgical appliance \textbf{2} A pair of straps that pass over the shoulders and fasten to the top of trousers at the front and back to hold them up. {\fontspec{DejaVu Sans}◇} \textit{} \textbf{3} A pair of something, typically of birds or mammals killed in hunting. {\fontspec{DejaVu Sans}◇} \textit{thirty brace of grouse} \colorBulletS{SYN} pair, couple, duo, twosome, duology \textbf{4} Either of the two marks \{ and \}, used either to indicate that two or more items on one side have the same relationship as each other to the single item to which the other side points, or in pairs to show that words between them are connected. {\fontspec{DejaVu Sans}◇} \textit{} \colorBulletS{SYN} bracket, parenthesis \\{\fontspec{DejaVu Sans}▪ }\textsf{\textit{verb}}\\ \textbf{1} Make (a structure) stronger or firmer with wood, iron, or other forms of support. {\fontspec{DejaVu Sans}◇} \textit{the posts were braced by lengths of timber} \colorBulletS{SYN} support, shore up, prop up, hold up, buttress, carry, bear, underpin}{}{}{ \colorBullet{ORIGIN} Middle English (as a verb meaning ‘clasp, fasten tightly’): from Old French bracier ‘embrace’, from brace ‘two arms’, from Latin bracchia, plural of bracchium ‘arm’, from Greek brakhiōn.}%
\par%
\entry{bravery}{/ˈbreɪv(ə)ri/}{সাহস}{ \textsf{\textit{noun}}\ \textbf{1} Courageous behaviour or character. {\fontspec{DejaVu Sans}◇} \textit{perhaps I'll get a medal for bravery} \colorBulletS{SYN} courage, courageousness, pluck, pluckiness, braveness, valour, fearlessness, intrepidity, intrepidness, nerve, daring, audacity, boldness, dauntlessness, doughtiness, stout{-}heartedness, hardihood, manfulness, heroism, gallantry}{}{}{ \colorBullet{ORIGIN} Mid 16th century (in the sense ‘bravado’): from French braverie or Italian braveria ‘boldness’, based on Latin barbarus (see barbarous).}%
\par%
\entry{brawny}{/ˈbrɔːni/}{পেশীবহুল}{ \textsf{\textit{adjective}}\ \textbf{1} Physically strong; muscular. {\fontspec{DejaVu Sans}◇} \textit{a great brawny brute} \colorBulletS{SYN} strong, as strong as an ox, muscular, well muscled, muscly, muscle{-}bound, well built, powerfully built, powerful, mighty, Herculean, strapping, burly, robust, sturdy, husky, lusty, sinewy, well knit, rugged}{}{}{}%
\par%
\entry{breach}{/briːtʃ/}{লঙ্ঘন}{\small{\textsf{\textit{noun, verb}}} \\{\fontspec{DejaVu Sans}▪ }\textsf{\textit{noun}}\\ \textbf{1} An act of breaking or failing to observe a law, agreement, or code of conduct. {\fontspec{DejaVu Sans}◇} \textit{a breach of confidence} \colorBulletS{SYN} contravention, violation, breaking, non{-}observance, infringement, transgression, neglect, dereliction \textbf{2} A gap in a wall, barrier, or defence, especially one made by an attacking army. {\fontspec{DejaVu Sans}◇} \textit{a breach in the mountain wall} \colorBulletS{SYN} break, rupture, split, crack, fracture, rent, rift \\{\fontspec{DejaVu Sans}▪ }\textsf{\textit{verb}}\\ \textbf{1} Make a gap in and break through (a wall, barrier, or defence) {\fontspec{DejaVu Sans}◇} \textit{the river breached its bank} \colorBulletS{SYN} break, break through, burst, burst through, rupture, force itself through, split \textbf{2} (of a whale) rise and break through the surface of the water. {\fontspec{DejaVu Sans}◇} \textit{we saw whales breaching in the distance}}{}{}{ \colorBullet{ORIGIN} Middle English from Old French breche, ultimately of Germanic origin; related to break.}%
\par%
\entry{breather}{/ˈbriːðə/}{সাময়িক বিশ্রাম}{ \textsf{\textit{noun}}\ \textbf{1} A person or animal that breathes in a particular way. {\fontspec{DejaVu Sans}◇} \textit{a heavy breather} \textbf{2} A brief pause for rest. {\fontspec{DejaVu Sans}◇} \textit{let's take a breather} \colorBulletS{SYN} break, rest, pause, interval, respite, breathing space, lull, recess, time out \textbf{3} A vent or valve to release pressure or to allow air to move freely around something. {\fontspec{DejaVu Sans}◇} \textit{a cask breather} \colorBulletS{SYN} outlet, inlet, opening, aperture, vent hole, hole, gap, orifice, space, cavity, cleft, slit, pore, port}{}{}{}%
\par%
\entry{breathtaking}{/ˈbrɛθteɪkɪŋ/}{উত্তেজনাপূর্ণ}{ \textsf{\textit{adjective}}\ \textbf{1} Astonishing or awe{-}inspiring in quality, so as to take one's breath away. {\fontspec{DejaVu Sans}◇} \textit{the scene was one of breathtaking beauty} \colorBulletS{SYN} spectacular, magnificent, wonderful, awe{-}inspiring, awesome, astounding, astonishing, amazing, stunning, stupendous, incredible}{}{}{}%
\par%
\entry{brew}{/bruː/}{ফন্দি আঁটা}{\small{\textsf{\textit{noun, verb}}} \\{\fontspec{DejaVu Sans}▪ }\textsf{\textit{noun}}\\ \textbf{1} A kind of beer. {\fontspec{DejaVu Sans}◇} \textit{small breweries which are able to offer rare brews} \colorBulletS{SYN} beer, ale \textbf{2} A cup or mug of tea or coffee. {\fontspec{DejaVu Sans}◇} \textit{she took a sip of the hot reviving brew} \colorBulletS{SYN} drink \textbf{3} A mixture of events, people, or things which interact to form a more potent whole. {\fontspec{DejaVu Sans}◇} \textit{a dangerous brew of political turmoil and violent conflict} \colorBulletS{SYN} mixture, mix, blend, combination, compound, amalgam, concoction, pot{-}pourri, melange \\{\fontspec{DejaVu Sans}▪ }\textsf{\textit{verb}}\\ \textbf{1} Make (beer) by soaking, boiling, and fermentation. {\fontspec{DejaVu Sans}◇} \textit{within five years the company will brew as much beer in China as in Australia} \colorBulletS{SYN} ferment, make \textbf{2} Make (tea or coffee) by mixing it with hot water. {\fontspec{DejaVu Sans}◇} \textit{I've just brewed some coffee} \colorBulletS{SYN} prepare, infuse, make \textbf{3} (of an unwelcome event or situation) begin to develop. {\fontspec{DejaVu Sans}◇} \textit{there was more trouble brewing as the miners went on strike} \colorBulletS{SYN} develop, gather force, loom, be close, be ominously close, be on the way, be on the horizon, be in the offing, be in the wings, be imminent, be threatening, be impending, impend, be just around the corner}{}{}{ \colorBullet{ORIGIN} Old English brēowan (verb), of Germanic origin; related to Dutch brouwen and German brauen.}%
\par%
\entry{bribe}{/brʌɪb/}{ঘুষ}{\small{\textsf{\textit{noun, verb}}} \\{\fontspec{DejaVu Sans}▪ }\textsf{\textit{noun}}\\ \textbf{1} A sum of money or other inducement offered or given to bribe someone. {\fontspec{DejaVu Sans}◇} \textit{lawmakers were caught accepting bribes to bring in legalized gambling} \colorBulletS{SYN} inducement, incentive \\{\fontspec{DejaVu Sans}▪ }\textsf{\textit{verb}}\\ \textbf{1} Dishonestly persuade (someone) to act in one's favour by a gift of money or other inducement. {\fontspec{DejaVu Sans}◇} \textit{they attempted to bribe opponents into losing} \colorBulletS{SYN} buy off, pay off, suborn, give an inducement to, corrupt}{}{}{ \colorBullet{ORIGIN} Late Middle English from Old French briber, brimber ‘beg’, of unknown origin. The original sense was ‘rob, extort’, hence (as a noun) ‘theft, stolen goods’, also ‘money extorted or demanded for favours’, later ‘offer money as an inducement’ (early 16th century).}%
\par%
\entry{bribery}{/ˈbrʌɪbəri/}{উৎকোচ গ্রহণ}{ \textsf{\textit{noun}}\ \textbf{1} The giving or offering of a bribe. {\fontspec{DejaVu Sans}◇} \textit{his opponent had been guilty of bribery and corruption} \colorBulletS{SYN} corruption, subornation}{}{}{}%
\par%
\entry{bridal}{/ˈbrʌɪd(ə)l/}{দাম্পত্য}{ \textsf{\textit{adjective}}\ \textbf{1} Of or concerning a bride or a newly married couple. {\fontspec{DejaVu Sans}◇} \textit{her white bridal gown} \colorBulletS{SYN} nuptial, wedding, marriage, matrimonial, marital, connubial, conjugal}{}{}{ \colorBullet{ORIGIN} Late Middle English from Old English brȳd{-}ealu ‘wedding feast’, from brȳd ‘bride’ + ealu ‘ale{-}drinking’. Since the late 16th century, the word has been associated with adjectives ending in {-}al.}%
\par%
\entry{brim}{/brɪm/}{ধারি}{\small{\textsf{\textit{noun, verb}}} \\{\fontspec{DejaVu Sans}▪ }\textsf{\textit{noun}}\\ \textbf{1} The projecting edge around the bottom of a hat. {\fontspec{DejaVu Sans}◇} \textit{a soft hat with a turned{-}up brim} \colorBulletS{SYN} peak, visor, bill, projection, shield, shade \textbf{2} The upper edge or lip of a cup, bowl, or other container. {\fontspec{DejaVu Sans}◇} \textit{he filled her glass to the brim} \colorBulletS{SYN} rim, lip, brink, edge, margin \\{\fontspec{DejaVu Sans}▪ }\textsf{\textit{verb}}\\ \textbf{1} Be full to the point of overflowing. {\fontspec{DejaVu Sans}◇} \textit{my eyes brimmed with tears} \colorBulletS{SYN} be full, be filled up, be filled to the top, be full to capacity, be packed with, overflow, run over, well over}{}{}{ \colorBullet{ORIGIN} Middle English (denoting the edge of the sea or other body of water): perhaps related to German Bräme ‘trimming’.}%
\par%
\entry{brink}{/brɪŋk/}{কিনারা}{ \textsf{\textit{noun}}\ \textbf{1} The extreme edge of land before a steep slope or a body or water. {\fontspec{DejaVu Sans}◇} \textit{the brink of the cliffs} \colorBulletS{SYN} edge, verge, margin, rim, lip}{ \colorBullet{OTHER} brink of}{}{ \colorBullet{ORIGIN} Middle English of Scandinavian origin.}%
\par%
\entry{brisk}{/brɪsk/}{প্রাণবন্ত}{\small{\textsf{\textit{adjective, verb}}} \\{\fontspec{DejaVu Sans}▪ }\textsf{\textit{adjective}}\\ \textbf{1} Active and energetic. {\fontspec{DejaVu Sans}◇} \textit{a good brisk walk} \colorBulletS{SYN} quick, rapid, fast, swift, speedy, fleet{-}footed \\{\fontspec{DejaVu Sans}▪ }\textsf{\textit{verb}}\\ \textbf{1} Quicken something. {\fontspec{DejaVu Sans}◇} \textit{Mary brisked up her pace}}{}{}{ \colorBullet{ORIGIN} Late 16th century probably from French brusque (see brusque).}%
\par%
\entry{broker}{/ˈbrəʊkə/}{দালাল}{\small{\textsf{\textit{noun, verb}}} \\{\fontspec{DejaVu Sans}▪ }\textsf{\textit{noun}}\\ \textbf{1} A person who buys and sells goods or assets for others. {\fontspec{DejaVu Sans}◇} \textit{the centralized lenders operate through brokers} \colorBulletS{SYN} dealer, broker{-}dealer, agent, negotiator, trafficker \\{\fontspec{DejaVu Sans}▪ }\textsf{\textit{verb}}\\ \textbf{1} Arrange or negotiate (an agreement) {\fontspec{DejaVu Sans}◇} \textit{fighting continued despite attempts to broker a ceasefire} \colorBulletS{SYN} arrange, organize, orchestrate, work out, thrash out, hammer out, settle, clinch, contract, pull off, bring about, bring off}{}{}{ \colorBullet{ORIGIN} Middle English (denoting a retailer or pedlar): from Anglo{-}Norman French brocour, of unknown ultimate origin.}%
\par%
\entry{brutal}{/ˈbruːt(ə)l/}{পাশবিক}{ \textsf{\textit{adjective}}\ \textbf{1} Savagely violent. {\fontspec{DejaVu Sans}◇} \textit{a brutal murder} \colorBulletS{SYN} savage, cruel, bloodthirsty, vicious, ferocious, barbaric, barbarous, wicked, murderous, cold{-}blooded, hard{-}hearted, harsh}{}{}{ \colorBullet{ORIGIN} Late 15th century (in the sense ‘relating to the lower animals’): from Old French, or from medieval Latin brutalis, from brutus ‘dull, stupid’ (see brute).}%
\par%
\entry{bulky}{/ˈbʌlki/}{ভারী}{ \textsf{\textit{adjective}}\ \textbf{1} Taking up much space; large and unwieldy. {\fontspec{DejaVu Sans}◇} \textit{a bulky carrier bag} \colorBulletS{SYN} large, big, great, huge, of considerable size, sizeable, substantial, voluminous, girthy, immense, enormous, colossal, massive, mammoth, vast, goodly, prodigious, tremendous, gigantic, giant, monumental, stupendous, gargantuan, elephantine, titanic, mountainous, monstrous}{}{}{}%
\par%
\entry{bully}{/ˈbʊli/}{তর্জন}{\small{\textsf{\textit{noun, verb}}} \\{\fontspec{DejaVu Sans}▪ }\textsf{\textit{noun}}\\ \textbf{1} A person who habitually seeks to harm or intimidate those whom they perceive as vulnerable. {\fontspec{DejaVu Sans}◇} \textit{he is a ranting, domineering bully} \colorBulletS{SYN} persecutor, oppressor, tyrant, tormentor, browbeater, intimidator, coercer, subjugator \\{\fontspec{DejaVu Sans}▪ }\textsf{\textit{verb}}\\ \textbf{1} Seek to harm, intimidate, or coerce (someone perceived as vulnerable) {\fontspec{DejaVu Sans}◇} \textit{her 11{-} year{-}old son has been constantly bullied at school} \colorBulletS{SYN} persecute, oppress, tyrannize, torment, browbeat, intimidate, cow, coerce, strong{-}arm, subjugate, domineer}{}{}{ \colorBullet{ORIGIN} Mid 16th century probably from Middle Dutch boele ‘lover’. Original use was as a term of endearment applied to either sex; it later became a familiar form of address to a male friend. The current sense dates from the late 17th century.}%
\par%
\entry{bully}{/ˈbʊli/}{তর্জন}{ \textsf{\textit{adjective}}\ \textbf{1} Very good; excellent. {\fontspec{DejaVu Sans}◇} \textit{the statue really looked bully}}{}{}{ \colorBullet{ORIGIN} Late 16th century (originally used of a person, meaning ‘admirable, gallant, jolly’): from bully. The current sense dates from the mid 19th century.}%
\par%
\entry{bully}{/ˈbʊli/}{তর্জন}{ \textsf{\textit{noun}}\ \textbf{1} Corned beef. {\fontspec{DejaVu Sans}◇} \textit{}}{}{}{ \colorBullet{ORIGIN} Mid 18th century alteration of bouilli.}%
\par%
\entry{bully}{/ˈbʊli/}{তর্জন}{\small{\textsf{\textit{noun, verb}}} \\{\fontspec{DejaVu Sans}▪ }\textsf{\textit{noun}}\\ \textbf{1} An act of starting play in field hockey, in which two opponents strike each other's sticks three times and then go for the ball. {\fontspec{DejaVu Sans}◇} \textit{} \\{\fontspec{DejaVu Sans}▪ }\textsf{\textit{verb}}\\ \textbf{1} (in field hockey) start play with a bully. {\fontspec{DejaVu Sans}◇} \textit{}}{}{}{ \colorBullet{ORIGIN} Late 19th century (originally denoting a scrum in Eton football): of unknown origin.}%
\par%
\entry{bum}{/bʌm/}{পশ্চাদ্দেশ}{\small{\textsf{\textit{adjective, noun, verb}}} \\{\fontspec{DejaVu Sans}▪ }\textsf{\textit{adjective}}\\ \textbf{1} Of poor quality; bad or wrong. {\fontspec{DejaVu Sans}◇} \textit{not one bum note was played} \colorBulletS{SYN} bad, poor, inferior, second{-}rate, second{-}class, unsatisfactory, inadequate, unacceptable, substandard, not up to scratch, not up to par, deficient, imperfect, defective, faulty, shoddy, amateurish, careless, negligent \\{\fontspec{DejaVu Sans}▪ }\textsf{\textit{noun}}\\ \textbf{1} A vagrant. {\fontspec{DejaVu Sans}◇} \textit{bums had been known to wander up to their door and ask for a sandwich} \colorBulletS{SYN} tramp, vagrant, vagabond, homeless person, derelict, down{-}and{-}out \textbf{2} A person who devotes a great deal of time to a specified activity. {\fontspec{DejaVu Sans}◇} \textit{a ski bum} \\{\fontspec{DejaVu Sans}▪ }\textsf{\textit{verb}}\\ \textbf{1} Travel with no particular purpose. {\fontspec{DejaVu Sans}◇} \textit{he bummed around Florida for a few months} \colorBulletS{SYN} loaf, lounge, idle, laze, languish, moon, stooge, droop, dally, dawdle, amble, potter, wander, drift, meander \textbf{2} Get by asking or begging. {\fontspec{DejaVu Sans}◇} \textit{they tried to bum quarters off us} \colorBulletS{SYN} scrounge, beg, borrow}{}{}{ \colorBullet{ORIGIN} Mid 19th century probably from bummer.}%
\par%
\entry{bum}{/bʌm/}{পশ্চাদ্দেশ}{ \textsf{\textit{noun}}\ \textbf{1} A person's buttocks or anus. {\fontspec{DejaVu Sans}◇} \textit{if you sit there you'll get a cold bum} \colorBulletS{SYN} buttocks, bottom, cheeks, hindquarters, haunches, rear, rump, rear end, backside, seat}{}{}{ \colorBullet{ORIGIN} Late Middle English of unknown origin.}%
\par%
\entry{buoy}{/bɔɪ/}{বয়া; বজায় রাখা; ভাসাইয়া রাখা}{\small{\textsf{\textit{noun, verb}}} \\{\fontspec{DejaVu Sans}▪ }\textsf{\textit{noun}}\\ \textbf{1} An anchored float serving as a navigation mark, to show reefs or other hazards, or for mooring. {\fontspec{DejaVu Sans}◇} \textit{} \colorBulletS{SYN} marker, anchored float, navigation mark, guide, beacon, signal \\{\fontspec{DejaVu Sans}▪ }\textsf{\textit{verb}}\\ \textbf{1} Keep (someone or something) afloat. {\fontspec{DejaVu Sans}◇} \textit{the creatures could swim, both buoyed up and cooled by the water} \colorBulletS{SYN} buoyant, floating, buoyed up, non{-}submerged, suspended, drifting, above the surface, on the surface, above water, keeping one's head above water \textbf{2} Mark with a buoy. {\fontspec{DejaVu Sans}◇} \textit{the wreck is often buoyed during summer months}}{}{}{ \colorBullet{ORIGIN} Middle English probably from Middle Dutch boye, boeie, from a Germanic base meaning ‘signal’. The verb is from Spanish boyar ‘to float’, from boya ‘buoy’.}%
\par%
\entry{burden}{/ˈbəːd(ə)n/}{বোঝা}{\small{\textsf{\textit{noun, verb}}} \\{\fontspec{DejaVu Sans}▪ }\textsf{\textit{noun}}\\ \textbf{1} A load, typically a heavy one. {\fontspec{DejaVu Sans}◇} \textit{} \colorBulletS{SYN} load, cargo, freight, weight \textbf{2} The main theme or gist of a speech, book, or argument. {\fontspec{DejaVu Sans}◇} \textit{} \colorBulletS{SYN} gist, substance, drift, implication, intention, thrust, meaning, significance, signification, sense, essence, thesis, import, purport, tenor, message, spirit \textbf{3} The refrain or chorus of a song. {\fontspec{DejaVu Sans}◇} \textit{} \colorBulletS{SYN} refrain, burden, strain \\{\fontspec{DejaVu Sans}▪ }\textsf{\textit{verb}}\\ \textbf{1} Load heavily. {\fontspec{DejaVu Sans}◇} \textit{she walked forwards burdened with a wooden box} \colorBulletS{SYN} load, weight, charge}{}{}{ \colorBullet{ORIGIN} Old English byrthen, of West Germanic origin; related to bear.}%
\par%
\entry{burial}{/ˈbɛrɪəl/}{সমাধি}{ \textsf{\textit{noun}}\ \textbf{1} The action or practice of burying a dead body. {\fontspec{DejaVu Sans}◇} \textit{his remains were shipped home for burial} \colorBulletS{SYN} burial, burying, committal, entombment, inhumation}{}{}{ \colorBullet{ORIGIN} Old English byrgels ‘place of burial, grave’ (interpreted as plural in Middle English, hence the loss of the final {-}s), of Germanic origin; related to bury.}%
\par%
\entry{buried}{/ˈbɛrɪd/}{প্রোথিত}{ \textsf{\textit{adjective}}\ \textbf{1} Placed or hidden underground. {\fontspec{DejaVu Sans}◇} \textit{buried treasure}}{}{}{}%
\par%
\entry{burst}{/bəːst/}{বিস্ফোরণ}{\small{\textsf{\textit{noun, verb}}} \\{\fontspec{DejaVu Sans}▪ }\textsf{\textit{noun}}\\ \textbf{1} An instance of breaking or splitting as a result of internal pressure or puncturing; an explosion. {\fontspec{DejaVu Sans}◇} \textit{the mortar bursts were further away than before} \colorBulletS{SYN} rupture, breach, split, blowout \textbf{2} A sudden brief outbreak. {\fontspec{DejaVu Sans}◇} \textit{a burst of activity} \colorBulletS{SYN} outbreak, outburst, eruption, flare{-}up, explosion, blow{-}up, blast, blaze, attack, fit, spasm, paroxysm, access, rush, gale, flood, storm, hurricane, torrent, outpouring, surge, upsurge, spurt, effusion, outflow, outflowing, welling up \textbf{3} A period of continuous and intense effort. {\fontspec{DejaVu Sans}◇} \textit{he sailed 474 miles in one 24{-}hour burst} \colorBulletS{SYN} spell, period, time, stretch, stint, turn, run, session, round, cycle \\{\fontspec{DejaVu Sans}▪ }\textsf{\textit{verb}}\\ \textbf{1} Break open or apart suddenly and violently, especially as a result of an impact or internal pressure. {\fontspec{DejaVu Sans}◇} \textit{one of the balloons burst} \colorBulletS{SYN} split open, burst open, break open, tear open, rupture, crack, fracture, fragment, shatter, shiver, fly open \textbf{2} Issue suddenly and uncontrollably. {\fontspec{DejaVu Sans}◇} \textit{the words burst from him in an angry rush} \textbf{3} Suddenly begin doing or producing something. {\fontspec{DejaVu Sans}◇} \textit{Sophie burst out laughing} \colorBulletS{SYN} break out in, launch into, erupt in, have a fit of \textbf{4} Separate (continuous stationery) into single sheets. {\fontspec{DejaVu Sans}◇} \textit{}}{}{}{ \colorBullet{ORIGIN} Old English berstan, of Germanic origin; related to Dutch bersten, barsten.}%
\par%
\entry{bustling}{/ˈbʌslɪŋ/}{শশব্যস্ততা}{ \textsf{\textit{adjective}}\ \textbf{1} (of a place) full of activity. {\fontspec{DejaVu Sans}◇} \textit{the bustling little town}}{}{}{}%
\par%
\entry{buttock}{/ˈbʌtək/}{নিতম্ব}{ \textsf{\textit{noun}}\ \textbf{1} Either of the two round fleshy parts of the human body that form the bottom. {\fontspec{DejaVu Sans}◇} \textit{} \colorBulletS{SYN} backside, bottom, behind, seat, rump, rear, rear end, cheeks, hindquarters, haunches}{}{}{ \colorBullet{ORIGIN} Old English buttuc, probably from the base of butt+ {-}ock.}%
\par%
\end{multicols}%
\pagebreak%
\section*{C}%
\begin{multicols}{2}%
\entry{cakewalk}{/ˈkeɪkwɔːk/}{সহজ কাজ}{\small{\textsf{\textit{noun, verb}}} \\{\fontspec{DejaVu Sans}▪ }\textsf{\textit{noun}}\\ \textbf{1} An absurdly or surprisingly easy task. {\fontspec{DejaVu Sans}◇} \textit{winning the league won't be a cakewalk for them} \colorBulletS{SYN} easy task, easy job, child's play, five{-}finger exercise, gift, walkover, nothing, sinecure, gravy train \textbf{2} A dancing contest among black Americans in which a cake was awarded as a prize. {\fontspec{DejaVu Sans}◇} \textit{} \\{\fontspec{DejaVu Sans}▪ }\textsf{\textit{verb}}\\ \textbf{1} Achieve or win something easily. {\fontspec{DejaVu Sans}◇} \textit{he cakewalked to a 5–1 triumph} \textbf{2} Walk or dance in the manner of a cakewalk. {\fontspec{DejaVu Sans}◇} \textit{a troupe of clowns cakewalked by}}{}{}{}%
\par%
\entry{calf}{/kɑːf/}{বাছুর}{ \textsf{\textit{noun}}\ \textbf{1} A young bovine animal, especially a domestic cow or bull in its first year. {\fontspec{DejaVu Sans}◇} \textit{a heifer calf} \colorBulletS{SYN} cow, heifer, bull, bullock, calf, ox \textbf{2} A floating piece of ice detached from an iceberg. {\fontspec{DejaVu Sans}◇} \textit{}}{}{}{ \colorBullet{ORIGIN} Old English cælf, of Germanic origin; related to Dutch kalf and German Kalb.}%
\par%
\entry{calf}{/kɑːf/}{বাছুর}{ \textsf{\textit{noun}}\ \textbf{1} The fleshy part at the back of a person's leg below the knee. {\fontspec{DejaVu Sans}◇} \textit{the calf muscles}}{}{}{ \colorBullet{ORIGIN} Middle English from Old Norse kálfi, of unknown origin.}%
\par%
\entry{caliform}{}{}{\small{\textsf{\textit{}}}}{}{}{}%
\par%
\entry{call}{/kɔːl/}{ডাকা}{\small{\textsf{\textit{noun, verb}}} \\{\fontspec{DejaVu Sans}▪ }\textsf{\textit{noun}}\\ \textbf{1} A cry made as a summons or to attract someone's attention. {\fontspec{DejaVu Sans}◇} \textit{a nearby fisherman heard their calls for help} \colorBulletS{SYN} cry, shout, yell, whoop, roar, scream, shriek \textbf{2} The characteristic cry of a bird or other animal. {\fontspec{DejaVu Sans}◇} \textit{it is best distinguished by its call, a loud ‘pwit’} \colorBulletS{SYN} cry, song, sound \textbf{3} An instance of speaking to someone on the phone or attempting to contact someone by phone. {\fontspec{DejaVu Sans}◇} \textit{I'll give you a call at around five} \colorBulletS{SYN} phone call, telephone call \textbf{4} A brief visit, especially one made for social reasons. {\fontspec{DejaVu Sans}◇} \textit{we paid a call on Ben and his family} \colorBulletS{SYN} visit, social call \textbf{5} An appeal or demand for something to happen or be done. {\fontspec{DejaVu Sans}◇} \textit{the call for action was welcomed} \colorBulletS{SYN} appeal, request, plea, entreaty \textbf{6} An order or request for someone to be present. {\fontspec{DejaVu Sans}◇} \textit{he was delighted that so many former players had heeded the call to attend the conference} \colorBulletS{SYN} summons, request \textbf{7} (in sport) a decision or ruling made by an umpire or other official, traditionally conveyed by a shout, that the ball has gone out of play or that a rule has been breached. {\fontspec{DejaVu Sans}◇} \textit{he was visibly irritated with the umpire's calls} \textbf{8} A command to execute a subroutine. {\fontspec{DejaVu Sans}◇} \textit{parameter values may be changed by calls to a special purpose input specification subroutine} \textbf{9} A demand for payment of lent or unpaid capital. {\fontspec{DejaVu Sans}◇} \textit{} \textbf{10} (in a bar, club, etc.) denoting or made with relatively expensive brands of alcohol which customers request by name. {\fontspec{DejaVu Sans}◇} \textit{try wines by the glass for \$5, beer for \$3, and call drinks for \$8} \\{\fontspec{DejaVu Sans}▪ }\textsf{\textit{verb}}\\ \textbf{1} Give (a baby or animal) a specified name. {\fontspec{DejaVu Sans}◇} \textit{they called their daughter Hannah} \colorBulletS{SYN} name \textbf{2} Cry out (a word or words) {\fontspec{DejaVu Sans}◇} \textit{he heard an insistent voice calling his name} \colorBulletS{SYN} cry out, cry, shout, yell, sing out, whoop, bellow, roar, halloo, bawl, scream, shriek, screech \textbf{3} Contact or attempt to contact (a person or number) by phone. {\fontspec{DejaVu Sans}◇} \textit{could I call you back?} \colorBulletS{SYN} phone, telephone, get on the phone to, get someone on the phone, dial, make a call to, place a call to, get, reach \textbf{4} Order or request the attendance of. {\fontspec{DejaVu Sans}◇} \textit{representatives of all three teams have been called to appear before the Monaco stewards} \textbf{5} Announce or decide that (an event, especially a meeting, election, or strike) is to happen. {\fontspec{DejaVu Sans}◇} \textit{there appeared to be no alternative but to call a general election} \colorBulletS{SYN} convene, summon, call together, order, assemble \textbf{6} (of a person) pay a brief visit. {\fontspec{DejaVu Sans}◇} \textit{I've got to call at the bank to get some cash} \textbf{7} (of an umpire or other official in a game) pronounce (a ball, stroke, etc.) to be the thing specified. {\fontspec{DejaVu Sans}◇} \textit{the linesman called the ball wide} \textbf{8} Cause (a subroutine) to be executed. {\fontspec{DejaVu Sans}◇} \textit{one subroutine may call another subroutine (or itself)}}{ \colorBullet{OTHER} call off}{}{ \colorBullet{ORIGIN} Late Old English ceallian, from Old Norse kalla ‘summon loudly’.}%
\par%
\entry{cannabis}{/ˈkanəbɪs/}{ভাং}{ \textsf{\textit{noun}}\ \textbf{1} A tall plant with a stiff upright stem, divided serrated leaves, and glandular hairs. It is used to produce hemp fibre and as a drug. {\fontspec{DejaVu Sans}◇} \textit{}}{}{}{ \colorBullet{ORIGIN} From Latin, from Greek kannabis.}%
\par%
\entry{cannibalism}{/ˈkanɪbəˌlɪz(ə)m/}{নরমাংসভক্ষণপ্রথা}{ \textsf{\textit{noun}}\ \textbf{1} The practice of eating the flesh of one's own species. {\fontspec{DejaVu Sans}◇} \textit{the film is quite disturbing at points with references to cannibalism}}{}{}{}%
\par%
\entry{canny}{/ˈkani/}{মিতব্যয়ী}{ \textsf{\textit{adjective}}\ \textbf{1} Having or showing shrewdness and good judgement, especially in money or business matters. {\fontspec{DejaVu Sans}◇} \textit{canny investors will switch banks if they think they are getting a raw deal} \colorBulletS{SYN} shrewd, astute, sharp, sharp{-}witted, discerning, acute, penetrating, discriminating, perceptive, perspicacious, clever, intelligent, wise, sagacious, sensible, judicious, circumspect, careful, prudent, cautious \textbf{2} Pleasant; nice. {\fontspec{DejaVu Sans}◇} \textit{she's a canny lass} \colorBulletS{SYN} friendly, agreeable, amiable, affable, nice, genial, likeable, amicable, lovely, good{-}humoured, personable, congenial, hospitable, approachable, good{-}natured, companionable}{}{}{ \colorBullet{ORIGIN} Late 16th century (originally Scots): from can (in the obsolete sense ‘know’) + {-}y.}%
\par%
\entry{canyon}{/ˈkanjən/}{গভীর খাদ}{ \textsf{\textit{noun}}\ \textbf{1} A deep gorge, typically one with a river flowing through it, as found in North America. {\fontspec{DejaVu Sans}◇} \textit{the Grand Canyon} \colorBulletS{SYN} ravine, gorge, gully, pass, defile, couloir}{}{}{ \colorBullet{ORIGIN} Mid 19th century from Spanish cañón ‘tube’, based on Latin canna ‘reed, cane’.}%
\par%
\entry{capsize}{/kapˈsʌɪz/}{উলটান}{\small{\textsf{\textit{noun, verb}}} \\{\fontspec{DejaVu Sans}▪ }\textsf{\textit{noun}}\\ \textbf{1} An instance of capsizing. {\fontspec{DejaVu Sans}◇} \textit{do you know what to do in the event of a capsize?} \\{\fontspec{DejaVu Sans}▪ }\textsf{\textit{verb}}\\ \textbf{1} (of a boat) be overturned in the water. {\fontspec{DejaVu Sans}◇} \textit{the craft capsized in heavy seas} \colorBulletS{SYN} overturn, turn over, turn upside down, upset, upend, knock over, flip over, tip over, topple over, invert, keel over, turn turtle}{}{}{ \colorBullet{ORIGIN} Late 18th century perhaps based on Spanish capuzar ‘sink (a ship) by the head’, from cabo ‘head’ + chapuzar ‘to dive or duck’.}%
\par%
\entry{captive}{/ˈkaptɪv/}{বন্দী}{\small{\textsf{\textit{adjective, noun}}} \\{\fontspec{DejaVu Sans}▪ }\textsf{\textit{adjective}}\\ \textbf{1} Imprisoned or confined. {\fontspec{DejaVu Sans}◇} \textit{the farm was used to hold prisoners of war captive} \colorBulletS{SYN} confined, caged, incarcerated, locked up, penned up \textbf{2} (of a facility or service) controlled by, and typically for the sole use of, an organization. {\fontspec{DejaVu Sans}◇} \textit{a captive power plant} \\{\fontspec{DejaVu Sans}▪ }\textsf{\textit{noun}}\\ \textbf{1} A person who has been taken prisoner or an animal that has been confined. {\fontspec{DejaVu Sans}◇} \textit{the policeman put a pair of handcuffs on the captive} \colorBulletS{SYN} prisoner, convict, detainee, inmate}{}{}{ \colorBullet{ORIGIN} Late Middle English from Latin captivus, from capere ‘seize, take’.}%
\par%
\entry{captivity}{/kapˈtɪvɪti/}{বন্দিদশা}{ \textsf{\textit{noun}}\ \textbf{1} The condition of being imprisoned or confined. {\fontspec{DejaVu Sans}◇} \textit{he was released after 865 days in captivity} \colorBulletS{SYN} imprisonment, confinement, internment, incarceration, custody, detention, restraint, constraint, committal, arrest}{}{}{ \colorBullet{ORIGIN} Late Middle English from Latin captivitas, from captivus ‘taken captive’ (see captive).}%
\par%
\entry{caramel}{/ˈkarəm(ə)l/}{দগ্ধ শর্করা}{ \textsf{\textit{noun}}\ \textbf{1} Sugar or syrup heated until it turns brown, used as a flavouring or colouring for food or drink or combined with butter or cream to form a thick, sweet sauce. {\fontspec{DejaVu Sans}◇} \textit{a gateau frosted with caramel}}{}{}{ \colorBullet{ORIGIN} Early 18th century from French, from Spanish caramelo.}%
\par%
\entry{caravan}{/ˈkarəvan/}{ভ্রমণকারী মরূযাত্রিদল}{ \textsf{\textit{noun}}\ \textbf{1} A vehicle equipped for living in, typically towed by a car and used for holidays. {\fontspec{DejaVu Sans}◇} \textit{they spent a fishing holiday in a caravan} \colorBulletS{SYN} mobile home, camper, caravanette \textbf{2} A group of people, especially traders or pilgrims, travelling together across a desert in Asia or North Africa. {\fontspec{DejaVu Sans}◇} \textit{}}{}{}{ \colorBullet{ORIGIN} Late 15th century (in caravan (sense 2)): from French caravane, from Persian kārwān. The sense ‘covered horse{-}drawn wagon’ dates from the early 19th century.}%
\par%
\entry{cardamom}{/ˈkɑːdəməm/}{এলাচ}{ \textsf{\textit{noun}}\ \textbf{1} The aromatic seeds of a plant of the ginger family, used as a spice and also medicinally. {\fontspec{DejaVu Sans}◇} \textit{} \textbf{2} The SE Asian plant that bears cardamom seeds. {\fontspec{DejaVu Sans}◇} \textit{}}{}{}{ \colorBullet{ORIGIN} Late Middle English from Old French cardamome or Latin cardamomum, from Greek kardamōmon, from kardamon ‘cress’ + amōmon, the name of a kind of spice plant.}%
\par%
\entry{carpenter}{/ˈkɑːp(ə)ntə/}{সূত্রধর}{\small{\textsf{\textit{noun, verb}}} \\{\fontspec{DejaVu Sans}▪ }\textsf{\textit{noun}}\\ \textbf{1} A person who makes and repairs wooden objects and structures. {\fontspec{DejaVu Sans}◇} \textit{} \colorBulletS{SYN} woodworker, joiner, cabinetmaker \\{\fontspec{DejaVu Sans}▪ }\textsf{\textit{verb}}\\ \textbf{1} Make by shaping wood. {\fontspec{DejaVu Sans}◇} \textit{the rails were carpentered very skilfully}}{}{}{ \colorBullet{ORIGIN} Middle English from Anglo{-}Norman French, from Old French carpentier, charpentier, from late Latin carpentarius (artifex) ‘carriage (maker)’, from carpentum ‘wagon’, of Gaulish origin; related to car.}%
\par%
\entry{carry}{/ˈkari/}{বহা}{\small{\textsf{\textit{noun, verb}}} \\{\fontspec{DejaVu Sans}▪ }\textsf{\textit{noun}}\\ \textbf{1} An act of carrying something from one place to another. {\fontspec{DejaVu Sans}◇} \textit{we did a carry of equipment from the camp} \textbf{2} The range of a gun or similar weapon. {\fontspec{DejaVu Sans}◇} \textit{} \textbf{3} The maintenance of an investment position in a securities market, especially with regard to the costs or profits accruing. {\fontspec{DejaVu Sans}◇} \textit{if other short{-}term interest rates are higher than the current yield, the bond is said to involve a negative carry} \\{\fontspec{DejaVu Sans}▪ }\textsf{\textit{verb}}\\ \textbf{1} Support and move (someone or something) from one place to another. {\fontspec{DejaVu Sans}◇} \textit{medics were carrying a wounded man on a stretcher} \colorBulletS{SYN} convey, transfer, move, take, bring, bear, shift, switch, fetch, transport \textbf{2} Support the weight of. {\fontspec{DejaVu Sans}◇} \textit{the bridge is capable of carrying even the heaviest loads} \colorBulletS{SYN} support, sustain, stand, prop up, shore up, bolster, underpin, buttress \textbf{3} (of a sound, ball, missile, etc.) reach a specified point. {\fontspec{DejaVu Sans}◇} \textit{his voice carried clearly across the room} \colorBulletS{SYN} be audible, travel, reach, be transmitted \textbf{4} Assume or accept (responsibility or blame) {\fontspec{DejaVu Sans}◇} \textit{they must carry management responsibility for the mess they have got the company into} \colorBulletS{SYN} undertake, accept, assume, bear, shoulder, support, sustain \textbf{5} Have as a feature or consequence. {\fontspec{DejaVu Sans}◇} \textit{being a combat sport, karate carries with it the risk of injury} \colorBulletS{SYN} entail, involve, lead to, result in, occasion, have as a consequence, have \textbf{6} Approve (a proposed measure) by a majority of votes. {\fontspec{DejaVu Sans}◇} \textit{the resolution was carried by a two{-}to{-}one majority} \colorBulletS{SYN} approve, vote for, accept, endorse, ratify, authorize, mandate, support, back, uphold \textbf{7} Transfer (a figure) to an adjacent column during an arithmetical operation (e.g. when a column of digit adds up to more than ten). {\fontspec{DejaVu Sans}◇} \textit{}}{ \colorBullet{OTHER} carry away: সরান}{}{ \colorBullet{ORIGIN} Late Middle English from Anglo{-}Norman French and Old Northern French carier, based on Latin carrus ‘wheeled vehicle’.}%
\par%
\entry{cassava}{/kəˈsɑːvə/}{কাসাভা}{ \textsf{\textit{noun}}\ \textbf{1} The starchy tuberous root of a tropical tree, used as food in tropical countries. {\fontspec{DejaVu Sans}◇} \textit{} \textbf{2} The shrubby tree from which cassava is obtained, native to tropical America and cultivated throughout the tropics. {\fontspec{DejaVu Sans}◇} \textit{}}{}{}{ \colorBullet{ORIGIN} Mid 16th century from Taino casávi, cazábbi, influenced by French cassave.}%
\par%
\entry{casting}{/ˈkɑːstɪŋ/}{ঢালাই}{ \textsf{\textit{noun}}\ \textbf{1} An object made by pouring molten metal or other material into a mould. {\fontspec{DejaVu Sans}◇} \textit{bronze castings} \colorBulletS{SYN} expulsion, expelling, banishment, banishing, exile, exiling, transportation, transporting, extradition, extraditing, expatriation, expatriating, repatriation, repatriating, refoulement}{}{}{}%
\par%
\entry{castration}{/kaˈstreɪʃ(ə)n/}{খোজাকরণ}{ \textsf{\textit{noun}}\ \textbf{1} The removal of the testicles of a male animal or man. {\fontspec{DejaVu Sans}◇} \textit{the castration of male calves was initiated to reduce fighting}}{}{}{}%
\par%
\entry{casualty}{/ˈkaʒjʊəlti/}{দুর্ঘটনা}{ \textsf{\textit{noun}}\ \textbf{1} A person killed or injured in a war or accident. {\fontspec{DejaVu Sans}◇} \textit{the shelling caused thousands of civilian casualties} \colorBulletS{SYN} victim, fatality, mortality}{}{}{ \colorBullet{ORIGIN} Late Middle English (in the sense ‘chance, a chance occurrence’): from medieval Latin casualitas, from casualis (see casual), on the pattern of words such as penalty.}%
\par%
\entry{catastrophe}{/kəˈtastrəfi/}{বিপর্যয়কারী ঘটনা}{ \textsf{\textit{noun}}\ \textbf{1} An event causing great and usually sudden damage or suffering; a disaster. {\fontspec{DejaVu Sans}◇} \textit{an environmental catastrophe} \colorBulletS{SYN} disaster, calamity, cataclysm, crisis, holocaust, ruin, ruination, tragedy, blow, shock \textbf{2} The denouement of a drama, especially a classical tragedy. {\fontspec{DejaVu Sans}◇} \textit{This is an old insight, of course {-} as old as the domestic catastrophes of classical Greek drama.}}{}{}{ \colorBullet{ORIGIN} Mid 16th century (in the sense ‘denouement’): from Latin catastropha, from Greek katastrophē ‘overturning, sudden turn’, from kata{-} ‘down’ + strophē ‘turning’ (from strephein ‘to turn’).}%
\par%
\entry{catastrophic}{/katəˈstrɒfɪk/}{সর্বনাশা}{ \textsf{\textit{adjective}}\ \textbf{1} Involving or causing sudden great damage or suffering. {\fontspec{DejaVu Sans}◇} \textit{a catastrophic earthquake} \colorBulletS{SYN} destructive, ruinous, disastrous, catastrophic, calamitous, cataclysmic}{}{}{}%
\par%
\entry{categorically}{/ˌkatəˈɡɒrɪk(ə)li/}{সুনিশ্চিতভাবে}{ \textsf{\textit{adverb}}\ \textbf{1} In a way that is unambiguously explicit and direct. {\fontspec{DejaVu Sans}◇} \textit{the rules state categorically, 'No Violence'}}{}{Myanmar categorically denied the army's involvement in murder: মায়ানমার হত্যাকাণ্ডে সেনাবাহিনীর জড়িত থাকার বিষয়টি স্পষ্টভাবে অস্বীকার করেছে }{}%
\par%
\entry{cater}{/ˈkeɪtə/}{খাদ্যাদি পরিবেশন করা}{ \textsf{\textit{verb}}\ \textbf{1} Provide people with food and drink at a social event or other gathering. {\fontspec{DejaVu Sans}◇} \textit{my mother helped to cater for the party} \colorBulletS{SYN} provide food for, feed, serve, cook for, wine and dine, regale, provide for, provision \textbf{2} Provide with what is needed or required. {\fontspec{DejaVu Sans}◇} \textit{the school caters for children with learning difficulties} \colorBulletS{SYN} serve, provide for, oblige, meet the needs of, meet the wants of, accommodate, entertain, receive}{}{}{ \colorBullet{ORIGIN} Late 16th century from obsolete cater ‘caterer’, from Old French acateor ‘buyer’, from acater ‘buy’ (see cate).}%
\par%
\entry{catering}{/ˈkeɪtərɪŋ/}{ক্যাটারিং}{ \textsf{\textit{noun}}\ \textbf{1} The provision of food and drink at a social event or other gathering. {\fontspec{DejaVu Sans}◇} \textit{high standards of catering}}{}{}{}%
\par%
\entry{caterpillar}{/ˈkatəpɪlə/}{শুঁয়াপোকা}{ \textsf{\textit{noun}}\ \textbf{1} The larva of a butterfly or moth, which has a segmented wormlike body with three pairs of true legs and several pairs of appendages similar to legs. {\fontspec{DejaVu Sans}◇} \textit{} \textbf{2}  {\fontspec{DejaVu Sans}◇} \textit{}}{}{}{ \colorBullet{ORIGIN} Late Middle English perhaps from a variant of Old French chatepelose, literally ‘hairy cat’, influenced by obsolete piller ‘ravager’. The association with ‘cat’ is found in other languages, e.g. Swiss German Teufelskatz (literally ‘devil's cat’), Lombard gatta (literally ‘cat’). Compare with French chaton, English catkin, which resembles a hairy caterpillar.}%
\par%
\entry{cavort}{/kəˈvɔːt/}{তিড়িং{-}তিড়িং করিয়া লাফান}{ \textsf{\textit{verb}}\ \textbf{1} Jump or dance around excitedly. {\fontspec{DejaVu Sans}◇} \textit{the players cavorted about the pitch} \colorBulletS{SYN} skip, dance, romp, jig, caper, cut capers, frisk, gambol, prance, frolic, play, lark}{}{}{ \colorBullet{ORIGIN} Late 18th century (originally US): perhaps an alteration of curvet.}%
\par%
\entry{cease}{/siːs/}{ক্ষান্তি}{ \textsf{\textit{verb}}\ \textbf{1} Come or bring to an end. {\fontspec{DejaVu Sans}◇} \textit{the hostilities ceased and normal life was resumed} \colorBulletS{SYN} come to an end, come to a halt, come to a stop, end, halt, stop, conclude, terminate, finish, wind up, draw to a close, be over, come to a standstill}{}{}{ \colorBullet{ORIGIN} Middle English from Old French cesser, from Latin cessare ‘stop’, from cedere ‘to yield’.}%
\par%
\entry{cemetery}{/ˈsɛmɪtri/}{কবরস্থান}{ \textsf{\textit{noun}}\ \textbf{1} A large burial ground, especially one not in a churchyard. {\fontspec{DejaVu Sans}◇} \textit{a military cemetery} \colorBulletS{SYN} graveyard, churchyard, burial ground, burial place, burying place, burying ground, garden of remembrance}{}{}{ \colorBullet{ORIGIN} Late Middle English via late Latin from Greek koimētērion ‘dormitory’, from koiman ‘put to sleep’.}%
\par%
\entry{cereal}{/ˈsɪərɪəl/}{খাদ্যশস্য}{ \textsf{\textit{noun}}\ \textbf{1} A grain used for food, for example wheat, maize, or rye. {\fontspec{DejaVu Sans}◇} \textit{} \colorBulletS{SYN} cereal, cereal crops \textbf{2} A breakfast food made from roasted grain, typically eaten with milk. {\fontspec{DejaVu Sans}◇} \textit{a bowl of cereal}}{}{}{ \colorBullet{ORIGIN} Early 19th century (as an adjective): from Latin cerealis, from Ceres.}%
\par%
\entry{certainly}{/ˈsəːt(ə)nli/}{নিশ্চিত ভাবেই}{ \textsf{\textit{adverb}}\ \textbf{1} Used to emphasize the speaker's belief that what is said is true. {\fontspec{DejaVu Sans}◇} \textit{the prestigious address certainly adds to the firm's appeal} \colorBulletS{SYN} unquestionably, surely, assuredly, definitely, beyond question, without question, beyond doubt, unequivocally, indubitably, undeniably, irrefutably, indisputably, incontrovertibly, incontestably, obviously, patently, manifestly, evidently, plainly, clearly, transparently, palpably, unmistakably, conclusively, undisputedly, undoubtedly}{}{It certainly is…}{}%
\par%
\entry{cetacean}{/sɪˈteɪʃn/}{তিমি{-}সম্বন্ধীয়}{\small{\textsf{\textit{adjective, noun}}} \\{\fontspec{DejaVu Sans}▪ }\textsf{\textit{adjective}}\\ \textbf{1} Relating to or denoting cetaceans. {\fontspec{DejaVu Sans}◇} \textit{} \\{\fontspec{DejaVu Sans}▪ }\textsf{\textit{noun}}\\ \textbf{1} A marine mammal of the order Cetacea; a whale, dolphin, or porpoise. {\fontspec{DejaVu Sans}◇} \textit{}}{}{}{}%
\par%
\entry{chant}{/tʃɑːnt/}{}{\small{\textsf{\textit{noun, verb}}} \\{\fontspec{DejaVu Sans}▪ }\textsf{\textit{noun}}\\ \textbf{1} A repeated rhythmic phrase, typically one shouted or sung in unison by a crowd. {\fontspec{DejaVu Sans}◇} \textit{a group of young people set up a chant of ‘Why are we waiting?’} \colorBulletS{SYN} shout, cry, slogan, rallying call, war cry, chorus, chanting \textbf{2} A short musical passage in two or more phrases used for singing unmetrical words; a psalm or canticle sung to such music. {\fontspec{DejaVu Sans}◇} \textit{} \\{\fontspec{DejaVu Sans}▪ }\textsf{\textit{verb}}\\ \textbf{1} Say or shout repeatedly in a sing{-}song tone. {\fontspec{DejaVu Sans}◇} \textit{protesters were chanting slogans} \colorBulletS{SYN} shout, sing, chorus, carol}{}{Chanting slogans}{ \colorBullet{ORIGIN} Late Middle English (in the sense ‘sing’): from Old French chanter ‘sing’, from Latin cantare, frequentative of canere ‘sing’.}%
\par%
\entry{chaos}{/ˈkeɪɒs/}{বিশৃঙ্খলা}{ \textsf{\textit{noun}}\ \textbf{1} Complete disorder and confusion. {\fontspec{DejaVu Sans}◇} \textit{snow caused chaos in the region} \colorBulletS{SYN} disorder, disarray, disorganization, confusion, mayhem, bedlam, pandemonium, madness, havoc, turmoil, tumult, commotion, disruption, upheaval, furore, frenzy, uproar, hue and cry, babel, hurly{-}burly}{}{}{ \colorBullet{ORIGIN} Late 15th century (denoting a gaping void or chasm, later formless primordial matter): via French and Latin from Greek khaos ‘vast chasm, void’.}%
\par%
\entry{cheek}{/tʃiːk/}{গাল}{\small{\textsf{\textit{noun, verb}}} \\{\fontspec{DejaVu Sans}▪ }\textsf{\textit{noun}}\\ \textbf{1} Either side of the face below the eye. {\fontspec{DejaVu Sans}◇} \textit{tears rolled down her cheeks} \textbf{2} Talk or behaviour regarded as rude or lacking in respect. {\fontspec{DejaVu Sans}◇} \textit{he had the cheek to complain} \colorBulletS{SYN} impudence, impertinence, insolence, cheekiness, audacity, temerity, brazenness, presumption, effrontery, nerve, gall, pertness, boldness, shamelessness, impoliteness, disrespect, bad manners, unmannerliness, overfamiliarity \\{\fontspec{DejaVu Sans}▪ }\textsf{\textit{verb}}\\ \textbf{1} Speak impertinently to. {\fontspec{DejaVu Sans}◇} \textit{Frankie always got away with cheeking his elders} \colorBulletS{SYN} answer back to, talk back to, be cheeky to, be impertinent to}{}{}{ \colorBullet{ORIGIN} Old English cē(a)ce, cēoce ‘cheek, jaw’, of West Germanic origin; related to Dutch kaak.}%
\par%
\entry{cherish}{/ˈtʃɛrɪʃ/}{লালিত}{ \textsf{\textit{verb}}\ \textbf{1} Protect and care for (someone) lovingly. {\fontspec{DejaVu Sans}◇} \textit{he needed a woman he could cherish} \colorBulletS{SYN} adore, hold dear, love, care very much for, feel great affection for, dote on, be devoted to, revere, esteem, admire, appreciate}{}{}{ \colorBullet{ORIGIN} Middle English (in the sense ‘treat with affection’): from Old French cheriss{-}, lengthened stem of cherir, from cher ‘dear’, from Latin carus.}%
\par%
\entry{chill}{/tʃɪl/}{শীতলতা}{\small{\textsf{\textit{adjective, noun, verb}}} \\{\fontspec{DejaVu Sans}▪ }\textsf{\textit{adjective}}\\ \textbf{1} Chilly. {\fontspec{DejaVu Sans}◇} \textit{the chill grey dawn} \colorBulletS{SYN} cold, chilly, cool, crisp, fresh, brisk \textbf{2} Very relaxed or easy{-}going. {\fontspec{DejaVu Sans}◇} \textit{in general, I am a pretty chill guy} \\{\fontspec{DejaVu Sans}▪ }\textsf{\textit{noun}}\\ \textbf{1} An unpleasant feeling of coldness in the atmosphere, one's surroundings, or the body. {\fontspec{DejaVu Sans}◇} \textit{there was a chill in the air} \colorBulletS{SYN} coldness, chilliness, coolness, iciness, crispness, rawness, bitterness, nip, bite, sting, sharpness, keenness, harshness, wintriness, frigidity \textbf{2} A metal mould, often cooled, designed to ensure rapid or even cooling of metal during casting. {\fontspec{DejaVu Sans}◇} \textit{Thus, dry sand cores often are used in green sand molds, and metal chills can be used in sand molds to accelerate local cooling.} \\{\fontspec{DejaVu Sans}▪ }\textsf{\textit{verb}}\\ \textbf{1} Make (someone) cold. {\fontspec{DejaVu Sans}◇} \textit{they were chilled by a sudden wind} \textbf{2} Horrify or frighten (someone) {\fontspec{DejaVu Sans}◇} \textit{the city was chilled by the violence} \colorBulletS{SYN} scare, frighten, petrify, terrify, alarm, appal, disturb, disquiet, unsettle \textbf{3} Calm down and relax. {\fontspec{DejaVu Sans}◇} \textit{they like to get home, have a bath, and chill out} \colorBulletS{SYN} relax, unwind, loosen up, ease off, ease up, let up, slow down, de{-}stress, unbend, rest, repose, put one's feet up, take it easy, take time off, take time out, slack off, be at leisure, take one's leisure, take one's ease, laze, luxuriate, do nothing, sit back, lounge, loll, slump, flop, idle, loaf, enjoy oneself, amuse oneself, play, entertain oneself}{}{}{ \colorBullet{ORIGIN} Old English cele, ciele ‘cold, coldness’, of Germanic origin; related to cold.}%
\par%
\entry{chop}{/tʃɒp/}{চপ}{\small{\textsf{\textit{noun, verb}}} \\{\fontspec{DejaVu Sans}▪ }\textsf{\textit{noun}}\\ \textbf{1} A downward cutting blow or movement, typically with the hand. {\fontspec{DejaVu Sans}◇} \textit{an effective chop to the back of the neck} \textbf{2} A thick slice of meat, especially pork or lamb, adjacent to and often including a rib. {\fontspec{DejaVu Sans}◇} \textit{he lived on liver or chops} \textbf{3} A person's share of something. {\fontspec{DejaVu Sans}◇} \textit{} \textbf{4} Crushed or ground grain used as animal feed. {\fontspec{DejaVu Sans}◇} \textit{the pile of chop was dropped into the calves' feeder} \textbf{5} The broken motion of water, owing to the action of the wind against the tide. {\fontspec{DejaVu Sans}◇} \textit{we started our run into a two{-}foot chop} \\{\fontspec{DejaVu Sans}▪ }\textsf{\textit{verb}}\\ \textbf{1} Cut (something) into pieces with repeated sharp blows of an axe or knife. {\fontspec{DejaVu Sans}◇} \textit{they chopped up the pulpit for firewood} \colorBulletS{SYN} cut up, cut into pieces, chop up \textbf{2} Abolish or reduce the size of (something) in a way regarded as ruthless. {\fontspec{DejaVu Sans}◇} \textit{their training courses are to be chopped} \colorBulletS{SYN} reduce drastically, cut}{}{}{ \colorBullet{ORIGIN} Late Middle English variant of chap.}%
\par%
\entry{chop}{/tʃɒp/}{চপ}{ \textsf{\textit{verb}}\ \textbf{1} Change one's opinions or behaviour repeatedly and abruptly. {\fontspec{DejaVu Sans}◇} \textit{teachers are fed up with having to chop and change with every twist in government policy}}{}{}{ \colorBullet{ORIGIN} Late Middle English (in the sense ‘barter, exchange’): perhaps related to Old English cēap ‘bargaining, trade’; compare with chap{-} in chapman.}%
\par%
\entry{chop}{/tʃɒp/}{চপ}{ \textsf{\textit{noun}}\ \textbf{1} A trademark; a brand of goods. {\fontspec{DejaVu Sans}◇} \textit{}}{}{}{ \colorBullet{ORIGIN} Early 19th century from Hindi chāp ‘stamp, brand’ (see chaap).}%
\par%
\entry{chuckle}{/ˈtʃʌk(ə)l/}{মৃদুহাস্য}{\small{\textsf{\textit{noun, verb}}} \\{\fontspec{DejaVu Sans}▪ }\textsf{\textit{noun}}\\ \textbf{1} A quiet or suppressed laugh. {\fontspec{DejaVu Sans}◇} \textit{Melissa gave a chuckle} \colorBulletS{SYN} chuckle, chortle, guffaw, giggle, titter, ha{-}ha, tee{-}hee, snigger, roar of laughter, hoot of laughter, shriek of laughter, peal of laughter, belly laugh \\{\fontspec{DejaVu Sans}▪ }\textsf{\textit{verb}}\\ \textbf{1} Laugh quietly or inwardly. {\fontspec{DejaVu Sans}◇} \textit{I chuckled at the astonishment on her face} \colorBulletS{SYN} chortle, giggle, titter, laugh quietly, tee{-}hee, snicker, snigger}{}{}{ \colorBullet{ORIGIN} Late 16th century (in the sense ‘laugh convulsively’): from chuck meaning ‘to cluck’ in late Middle English.}%
\par%
\entry{chum}{/tʃʌm/}{অন্তরঙ্গ বন্ধু}{\small{\textsf{\textit{noun, verb}}} \\{\fontspec{DejaVu Sans}▪ }\textsf{\textit{noun}}\\ \textbf{1} A close friend. {\fontspec{DejaVu Sans}◇} \textit{she shared the cake with her chums} \colorBulletS{SYN} friend, companion, intimate, familiar, confidant, alter ego, second self \textbf{2} Used as a friendly or familiar form of address between men or boys. {\fontspec{DejaVu Sans}◇} \textit{it's your own fault, chum} \\{\fontspec{DejaVu Sans}▪ }\textsf{\textit{verb}}\\ \textbf{1} Form a friendship with someone. {\fontspec{DejaVu Sans}◇} \textit{his sister chummed up with Sally}}{}{}{ \colorBullet{ORIGIN} Late 17th century (originally Oxford University slang, denoting a room{-}mate): probably short for chamber{-}fellow. Compare with comrade and crony.}%
\par%
\entry{chum}{/tʃʌm/}{অন্তরঙ্গ বন্ধু}{\small{\textsf{\textit{noun, verb}}} \\{\fontspec{DejaVu Sans}▪ }\textsf{\textit{noun}}\\ \textbf{1} Chopped fish and other material thrown overboard as angling bait. {\fontspec{DejaVu Sans}◇} \textit{the anglers anchored down and put out their blood chum} \\{\fontspec{DejaVu Sans}▪ }\textsf{\textit{verb}}\\ \textbf{1} Fish using chum as bait. {\fontspec{DejaVu Sans}◇} \textit{chumming is always a must when flounder fishing}}{}{}{ \colorBullet{ORIGIN} Mid 19th century of unknown origin.}%
\par%
\entry{chum}{/tʃʌm/}{অন্তরঙ্গ বন্ধু}{ \textsf{\textit{noun}}\ \textbf{1} A large North Pacific salmon that is commercially important as a food fish. {\fontspec{DejaVu Sans}◇} \textit{}}{}{}{ \colorBullet{ORIGIN} Early 20th century from Chinook Jargon tzum (samun), literally ‘spotted (salmon)’.}%
\par%
\entry{cinnamon}{/ˈsɪnəmən/}{দারুচিনি}{ \textsf{\textit{noun}}\ \textbf{1} An aromatic spice made from the peeled, dried, and rolled bark of a SE Asian tree. {\fontspec{DejaVu Sans}◇} \textit{a teaspoon of ground cinnamon} \textbf{2} The tree which yields cinnamon. {\fontspec{DejaVu Sans}◇} \textit{A Daoist tradition in China holds that the source of immortality, or at least long life, is the cinnamon tree in the moon, a tree that no amount of chopping can fell.}}{}{}{ \colorBullet{ORIGIN} Late Middle English from Old French cinnamome (from Greek kinnamōmon), and Latin cinnamon (from Greek kinnamon), both from a Semitic language and perhaps based on Malay.}%
\par%
\entry{circumference}{/səˈkʌmf(ə)r(ə)ns/}{পরিধি}{ \textsf{\textit{noun}}\ \textbf{1} The enclosing boundary of a curved geometric figure, especially a circle. {\fontspec{DejaVu Sans}◇} \textit{} \colorBulletS{SYN} perimeter, border, boundary}{}{}{ \colorBullet{ORIGIN} Late Middle English from Old French circonference, from Latin circumferentia, from circum ‘around, about’ + ferre ‘carry, bear’.}%
\par%
\entry{cite}{/sʌɪt/}{উদ্ধৃত}{\small{\textsf{\textit{noun, verb}}} \\{\fontspec{DejaVu Sans}▪ }\textsf{\textit{noun}}\\ \textbf{1} A citation. {\fontspec{DejaVu Sans}◇} \textit{} \colorBulletS{SYN} citation, quote, reference, mention, allusion, excerpt, extract, selection, passage, line, cutting, clip, clipping, snippet, reading, section, piece, part, fragment, portion, paragraph, verse, stanza, canto, sentence, phrase \\{\fontspec{DejaVu Sans}▪ }\textsf{\textit{verb}}\\ \textbf{1} Refer to (a passage, book, or author) as evidence for or justification of an argument or statement, especially in a scholarly work. {\fontspec{DejaVu Sans}◇} \textit{authors who are highly regarded by their peers tend to be cited} \colorBulletS{SYN} quote, reproduce \textbf{2} Praise (someone, typically a member of the armed forces) in an official report for a courageous act. {\fontspec{DejaVu Sans}◇} \textit{he has been cited many times for his contributions in the intelligence area} \colorBulletS{SYN} commend, pay tribute to, praise, recognize, give recognition to \textbf{3} Summon (someone) to appear in court. {\fontspec{DejaVu Sans}◇} \textit{the writ cited only four of the signatories of the petition} \colorBulletS{SYN} summon, summons, serve with a summons, subpoena, serve with a writ, call}{}{}{ \colorBullet{ORIGIN} Late Middle English (in cite (sense 3 of the verb), originally with reference to a court of ecclesiastical law): from Old French citer, from Latin citare, from ciere, cire ‘to call’.}%
\par%
\entry{civility}{/sɪˈvɪlɪti/}{ভদ্রতা}{ \textsf{\textit{noun}}\ \textbf{1} Formal politeness and courtesy in behaviour or speech. {\fontspec{DejaVu Sans}◇} \textit{I hope we can treat each other with civility and respect} \colorBulletS{SYN} courtesy, courteousness, politeness, good manners, mannerliness, gentlemanliness, chivalry, gallantry, graciousness, consideration, respect, gentility}{}{}{ \colorBullet{ORIGIN} Late Middle English from Old French civilite, from Latin civilitas, from civilis ‘relating to citizens’ (see civil). In early use the term denoted the state of being a citizen and hence good citizenship or orderly behaviour. The sense ‘politeness’ arose in the mid 16th century.}%
\par%
\entry{clap}{/klap/}{হাততালি}{\small{\textsf{\textit{noun, verb}}} \\{\fontspec{DejaVu Sans}▪ }\textsf{\textit{noun}}\\ \textbf{1} An act of striking together the palms of the hands. {\fontspec{DejaVu Sans}◇} \textit{when they stop I give them a clap} \colorBulletS{SYN} round of applause, hand, handclap \textbf{2} An explosive sound, especially of thunder. {\fontspec{DejaVu Sans}◇} \textit{a clap of thunder echoed through the valley} \colorBulletS{SYN} crack, crash, bang, boom \\{\fontspec{DejaVu Sans}▪ }\textsf{\textit{verb}}\\ \textbf{1} Strike the palms of (one's hands) together repeatedly, typically in order to applaud someone or something. {\fontspec{DejaVu Sans}◇} \textit{Agnes clapped her hands in glee} \colorBulletS{SYN} applaud, clap one's hands, give someone a round of applause, put one's hands together \textbf{2} Slap (someone) encouragingly on the back or shoulder. {\fontspec{DejaVu Sans}◇} \textit{as they parted, he clapped Owen on the back} \colorBulletS{SYN} slap, strike, hit, smack, crack, bang, thump, cuff}{}{}{ \colorBullet{ORIGIN} Old English clappan ‘throb, beat’, of imitative origin. clap (sense 1 of the verb) dates from late Middle English.}%
\par%
\entry{clap}{/klap/}{হাততালি}{ \textsf{\textit{noun}}\ \textbf{1} A venereal disease, especially gonorrhoea. {\fontspec{DejaVu Sans}◇} \textit{she has given him the clap}}{}{}{ \colorBullet{ORIGIN} Late 16th century from Old French clapoir ‘venereal bubo’.}%
\par%
\entry{clash}{/klaʃ/}{সংঘর্ষ}{\small{\textsf{\textit{noun, verb}}} \\{\fontspec{DejaVu Sans}▪ }\textsf{\textit{noun}}\\ \textbf{1} A violent confrontation. {\fontspec{DejaVu Sans}◇} \textit{there have been minor clashes with security forces} \colorBulletS{SYN} confrontation, skirmish, brush, encounter, engagement, collision, incident, conflict, fight, battle \textbf{2} A mismatch of colours. {\fontspec{DejaVu Sans}◇} \textit{a clash of tweeds and a striped shirt} \colorBulletS{SYN} mismatch, discordance, discord, lack of harmony, incompatibility, jarring \textbf{3} A loud jarring sound, as of metal objects being struck together. {\fontspec{DejaVu Sans}◇} \textit{a clash of cymbals} \colorBulletS{SYN} striking, bang, clang, crash, clatter, clank \\{\fontspec{DejaVu Sans}▪ }\textsf{\textit{verb}}\\ \textbf{1} Meet and come into violent conflict. {\fontspec{DejaVu Sans}◇} \textit{protestors demanding self{-}rule clashed with police} \colorBulletS{SYN} fight, skirmish, contend, come to blows, be in conflict, come into conflict, engage, war, grapple \textbf{2} (of colours) appear discordant or ugly when placed close to each other. {\fontspec{DejaVu Sans}◇} \textit{the yellow shirt clashed with her purple skirt} \colorBulletS{SYN} be incompatible, not match, not go, be discordant, jar \textbf{3} Strike (cymbals) together, producing a loud discordant sound. {\fontspec{DejaVu Sans}◇} \textit{} \colorBulletS{SYN} bang, strike, clang, crash, smash, clank, clatter}{}{}{ \colorBullet{ORIGIN} Early 16th century imitative.}%
\par%
\entry{clatter}{/ˈklatə/}{ঝনঝন শব্দ}{\small{\textsf{\textit{noun, verb}}} \\{\fontspec{DejaVu Sans}▪ }\textsf{\textit{noun}}\\ \textbf{1} A continuous rattling sound as of hard objects falling or striking each other. {\fontspec{DejaVu Sans}◇} \textit{the horse spun round with a clatter of hooves} \colorBulletS{SYN} uproar, racket, loud noise, confused noise, commotion, cacophony, babel, hubbub, tumult, fracas, clangour, crash, clatter, clash \\{\fontspec{DejaVu Sans}▪ }\textsf{\textit{verb}}\\ \textbf{1} Make or cause to make a continuous rattling sound. {\fontspec{DejaVu Sans}◇} \textit{her coffee cup clattered in the saucer} \colorBulletS{SYN} rattle, clank, clink, clunk, clang, bang}{}{}{ \colorBullet{ORIGIN} Old English (as a verb), of imitative origin.}%
\par%
\entry{clause}{/klɔːz/}{দফা}{ \textsf{\textit{noun}}\ \textbf{1} A unit of grammatical organization next below the sentence in rank and in traditional grammar said to consist of a subject and predicate. {\fontspec{DejaVu Sans}◇} \textit{} \colorBulletS{SYN} expression, group of words, word group, construction, clause, locution, wording, term, turn of phrase, idiom, idiomatic expression, set phrase, phrasal idiom, phrasal verb \textbf{2} A particular and separate article, stipulation, or proviso in a treaty, bill, or contract. {\fontspec{DejaVu Sans}◇} \textit{} \colorBulletS{SYN} section, paragraph, article, subsection, note, item, point, passage, part, heading}{}{}{ \colorBullet{ORIGIN} Middle English via Old French clause, based on Latin claus{-} ‘shut, closed’, from the verb claudere.}%
\par%
\entry{cleavage}{/ˈkliːvɪdʒ/}{বিদারণ}{ \textsf{\textit{noun}}\ \textbf{1} A sharp division; a split. {\fontspec{DejaVu Sans}◇} \textit{the old cleavage between the forces of the right and left} \colorBulletS{SYN} breaking, breakage, cracking, cleavage, rupture, shattering, fragmentation, splintering, splitting, separation, bursting, disintegration \textbf{2} The hollow between a woman's breasts when supported, especially as exposed by a low{-}cut garment. {\fontspec{DejaVu Sans}◇} \textit{Holly and Bridget checked their cleavages and rearranged their hair}}{}{}{}%
\par%
\entry{clinical}{/ˈklɪnɪk(ə)l/}{রোগশয্যা}{ \textsf{\textit{adjective}}\ \textbf{1} Relating to the observation and treatment of actual patients rather than theoretical or laboratory studies. {\fontspec{DejaVu Sans}◇} \textit{clinical medicine} \textbf{2} Very efficient and without feeling; coldly detached. {\fontspec{DejaVu Sans}◇} \textit{nothing was left to chance—everything was clinical} \colorBulletS{SYN} detached, impersonal, dispassionate, objective, uninvolved, distant, remote, aloof, removed, cold, indifferent, neutral, unsympathetic, unfeeling, unemotional, non{-}emotional, unsentimental}{}{}{ \colorBullet{ORIGIN} Late 18th century from Greek klinikē ‘bedside’ (see clinic) + {-}al.}%
\par%
\entry{cluster}{/ˈklʌstə/}{গুচ্ছ}{\small{\textsf{\textit{noun, verb}}} \\{\fontspec{DejaVu Sans}▪ }\textsf{\textit{noun}}\\ \textbf{1} A group of similar things or people positioned or occurring closely together. {\fontspec{DejaVu Sans}◇} \textit{clusters of creamy{-}white flowers} \colorBulletS{SYN} bunch, clump, collection, mass, knot, group, clutch, bundle, nest \\{\fontspec{DejaVu Sans}▪ }\textsf{\textit{verb}}\\ \textbf{1} Form a cluster or clusters. {\fontspec{DejaVu Sans}◇} \textit{the children clustered round her skirts} \colorBulletS{SYN} congregate, gather, collect, group, come together, assemble}{}{}{ \colorBullet{ORIGIN} Old English clyster; probably related to clot.}%
\par%
\entry{clutch}{/klʌtʃ/}{নিষ্ঠুরতা}{\small{\textsf{\textit{adjective, noun, verb}}} \\{\fontspec{DejaVu Sans}▪ }\textsf{\textit{adjective}}\\ \textbf{1} (in sport) denoting or occurring at a critical situation in which the outcome of a game or competition is at stake. {\fontspec{DejaVu Sans}◇} \textit{they both are hard{-}nosed players who seem to thrive in clutch situations} \\{\fontspec{DejaVu Sans}▪ }\textsf{\textit{noun}}\\ \textbf{1} A tight grasp. {\fontspec{DejaVu Sans}◇} \textit{she made a clutch at his body} \textbf{2} A mechanism for connecting and disconnecting an engine and the transmission system in a vehicle, or the working parts of any machine. {\fontspec{DejaVu Sans}◇} \textit{she let the clutch in and the car surged forward} \textbf{3} A clutch bag. {\fontspec{DejaVu Sans}◇} \textit{} \\{\fontspec{DejaVu Sans}▪ }\textsf{\textit{verb}}\\ \textbf{1} Grasp (something) tightly. {\fontspec{DejaVu Sans}◇} \textit{he stood clutching a microphone} \colorBulletS{SYN} grip, grasp, clasp, cling to, hang on to, clench, hold}{}{}{ \colorBullet{ORIGIN} Middle English (in the sense ‘bend, crook’): variant of obsolete clitch ‘close the hand’, from Old English clyccan ‘crook, clench’, of Germanic origin.}%
\par%
\entry{clutch}{/klʌtʃ/}{নিষ্ঠুরতা}{ \textsf{\textit{noun}}\ \textbf{1} A group of eggs fertilized at the same time, laid in a single session and (in birds) incubated together. {\fontspec{DejaVu Sans}◇} \textit{they lay fewer than ten eggs in a clutch} \colorBulletS{SYN} group, batch, nestful}{}{}{ \colorBullet{ORIGIN} Early 18th century probably a southern variant of northern English dialect cletch, related to Middle English cleck ‘to hatch’, from Old Norse klekja.}%
\par%
\entry{cognitive}{/ˈkɒɡnɪtɪv/}{জ্ঞানীয়}{ \textsf{\textit{adjective}}\ \textbf{1} Relating to cognition. {\fontspec{DejaVu Sans}◇} \textit{the cognitive processes involved in reading} \colorBulletS{SYN} mental, emotional, intellectual, inner, non{-}physical, cerebral, brain, rational, cognitive, abstract, conceptual, theoretical}{}{}{ \colorBullet{ORIGIN} Late 16th century from medieval Latin cognitivus, from cognit{-} ‘known’, from the verb cognoscere.}%
\par%
\entry{cognizance}{/ˈkɒ(ɡ)nɪz(ə)ns/}{জ্ঞান}{ \textsf{\textit{noun}}\ \textbf{1} Knowledge or awareness. {\fontspec{DejaVu Sans}◇} \textit{the Renaissance cognizance of Greece was limited} \colorBulletS{SYN} awareness, notice, knowledge, consciousness, apprehension, perception, realization, recognition, appreciation \textbf{2} A distinctive emblem or badge formerly worn by retainers of a noble house. {\fontspec{DejaVu Sans}◇} \textit{}}{}{}{ \colorBullet{ORIGIN} Middle English conisance, from Old French conoisance, based on Latin cognoscere ‘get to know’. The spelling with g, influenced by Latin, arose in the 15th century and gradually affected the pronunciation.}%
\par%
\entry{coitus}{/ˈkəʊɪtəs/}{মৈথুন}{ \textsf{\textit{noun}}\ \textbf{1} Sexual intercourse. {\fontspec{DejaVu Sans}◇} \textit{} \colorBulletS{SYN} sexual intercourse, sex, lovemaking, making love, sex act, act of love, sexual relations, intimate relations, intimacy, coupling, mating, going to bed with someone, sleeping with someone}{}{}{ \colorBullet{ORIGIN} Mid 19th century from Latin, from coire ‘go together’ (see coition).}%
\par%
\entry{collagen}{/ˈkɒlədʒ(ə)n/}{কোলাজেন}{ \textsf{\textit{noun}}\ \textbf{1} The main structural protein found in skin and other connective tissues, widely used in purified form for cosmetic surgical treatments. {\fontspec{DejaVu Sans}◇} \textit{vitamin C plays a vital role in the formation of collagen}}{}{}{ \colorBullet{ORIGIN} Mid 19th century from French collagène, from Greek kolla ‘glue’ + French {-}gène (see {-}gen).}%
\par%
\entry{collapse}{/kəˈlaps/}{পতন}{\small{\textsf{\textit{noun, verb}}} \\{\fontspec{DejaVu Sans}▪ }\textsf{\textit{noun}}\\ \textbf{1} An instance of a structure falling down or giving way. {\fontspec{DejaVu Sans}◇} \textit{the collapse of a railway bridge} \colorBulletS{SYN} cave{-}in, giving way, subsidence, crumbling, disintegration \\{\fontspec{DejaVu Sans}▪ }\textsf{\textit{verb}}\\ \textbf{1} (of a structure) suddenly fall down or give way. {\fontspec{DejaVu Sans}◇} \textit{the roof collapsed on top of me} \colorBulletS{SYN} cave in, fall in, subside, fall down, sag, slump, settle, give, give way, crumble, crumple, disintegrate, fall to pieces, come apart \textbf{2} (of a person) fall down and become unconscious as a result of illness or injury. {\fontspec{DejaVu Sans}◇} \textit{he collapsed from loss of blood} \colorBulletS{SYN} faint, pass out, black out, lose consciousness, fall unconscious, keel over \textbf{3} Fail suddenly and completely. {\fontspec{DejaVu Sans}◇} \textit{the talks collapsed last week over territorial issues} \colorBulletS{SYN} break down, fail, fall through, fold, founder, fall flat, miscarry, go wrong, come to nothing, come to grief, be frustrated, be unsuccessful, not succeed, disintegrate \textbf{4} Fold or be foldable into a small space. {\fontspec{DejaVu Sans}◇} \textit{some cots collapse down to fit into a holdall}}{}{}{ \colorBullet{ORIGIN} Early 17th century (as collapsed): from medical Latin collapsus, past participle of collabi, from col{-} ‘together’ + labi ‘to slip’.}%
\par%
\entry{collateral}{/kəˈlat(ə)r(ə)l/}{সমান্তরাল}{\small{\textsf{\textit{adjective, noun}}} \\{\fontspec{DejaVu Sans}▪ }\textsf{\textit{adjective}}\\ \textbf{1} Additional but subordinate; secondary. {\fontspec{DejaVu Sans}◇} \textit{the collateral meanings of a word} \colorBulletS{SYN} incidental, accidental, unintended, secondary, subordinate, ancillary, collateral, concomitant, accompanying, contingent, resulting, resultant, consequential, derived, derivative \textbf{2} Descended from the same stock but by a different line. {\fontspec{DejaVu Sans}◇} \textit{a collateral descendant of Robert Burns} \textbf{3} Situated side by side; parallel. {\fontspec{DejaVu Sans}◇} \textit{collateral veins} \colorBulletS{SYN} side by side, aligned, collateral, equidistant \\{\fontspec{DejaVu Sans}▪ }\textsf{\textit{noun}}\\ \textbf{1} Something pledged as security for repayment of a loan, to be forfeited in the event of a default. {\fontspec{DejaVu Sans}◇} \textit{she put her house up as collateral for the bank loan} \colorBulletS{SYN} security, surety, guarantee, guaranty, pledge, bond, assurance, insurance, indemnity, indemnification, pawn, backing \textbf{2} A person having the same ancestor as another but through a different line. {\fontspec{DejaVu Sans}◇} \textit{A few days later, two powerful Sandhanvalia Sardars, Atar Singh and Ajit Singh, collaterals of the royal contenders for the throne, arrived in Lahore and took over control.}}{}{}{ \colorBullet{ORIGIN} Late Middle English (as an adjective): from medieval Latin collateralis, from col{-} ‘together with’ + lateralis (from latus, later{-} ‘side’). collateral (sense 1 of the noun) (originally US) is from the phrase collateral security, denoting something pledged in addition to the main obligation of a contract.}%
\par%
\entry{collide}{/kəˈlʌɪd/}{ধাক্কা লাগা}{ \textsf{\textit{verb}}\ \textbf{1} Hit by accident when moving. {\fontspec{DejaVu Sans}◇} \textit{she collided with someone} \colorBulletS{SYN} crash, crash into, come into collision, come into collision with, bang, bang into, slam, slam into, impact, impact with}{ \colorBullet{OTHER} collided with}{}{ \colorBullet{ORIGIN} Early 17th century (in the sense ‘cause to collide’): from Latin collidere, from col{-} ‘together’ + laedere ‘to strike’.}%
\par%
\entry{colonoscopy}{/kɒləˈnɒskəpi/}{}{ \textsf{\textit{noun}}\ \textbf{1} A procedure in which a flexible fibre{-}optic instrument is inserted through the anus in order to examine the colon. {\fontspec{DejaVu Sans}◇} \textit{a colonoscopy did not show any problem}}{}{}{}%
\par%
\entry{combat}{/ˈkɒmbat/}{বিরোধিতা}{\small{\textsf{\textit{noun, verb}}} \\{\fontspec{DejaVu Sans}▪ }\textsf{\textit{noun}}\\ \textbf{1} Fighting between armed forces. {\fontspec{DejaVu Sans}◇} \textit{five Hurricanes were shot down in combat} \colorBulletS{SYN} battle, fighting, action, hostilities, conflict, armed conflict, war, warfare, bloodshed \\{\fontspec{DejaVu Sans}▪ }\textsf{\textit{verb}}\\ \textbf{1} Take action to reduce or prevent (something bad or undesirable) {\fontspec{DejaVu Sans}◇} \textit{an effort to combat drug trafficking} \colorBulletS{SYN} fight, battle against, do battle with, wage war against, take up arms against, strive against, contend with, tackle, attack, counter, oppose, resist, withstand, stand up to, face up to, make a stand against, put up a fight against, confront, defy}{}{}{ \colorBullet{ORIGIN} Mid 16th century (originally denoting a fight between two people or parties): from French combattre (verb), from late Latin combattere, from com{-} ‘together with’ + battere, variant of Latin batuere ‘to fight’.}%
\par%
\entry{come}{/kʌm/}{আসা}{\small{\textsf{\textit{noun, preposition, verb}}} \\{\fontspec{DejaVu Sans}▪ }\textsf{\textit{noun}}\\ \textbf{1} Semen ejaculated at an orgasm. {\fontspec{DejaVu Sans}◇} \textit{} \\{\fontspec{DejaVu Sans}▪ }\textsf{\textit{preposition}}\\ \textbf{1} When a specified time is reached or event happens. {\fontspec{DejaVu Sans}◇} \textit{I don't think that they'll be far away from honours come the new season} \\{\fontspec{DejaVu Sans}▪ }\textsf{\textit{verb}}\\ \textbf{1} Move or travel towards or into a place thought of as near or familiar to the speaker. {\fontspec{DejaVu Sans}◇} \textit{Jess came into the kitchen} \colorBulletS{SYN} move nearer, move closer, approach, advance, near, draw nigh, draw close, draw closer, draw near, draw nearer \textbf{2} Occur; happen; take place. {\fontspec{DejaVu Sans}◇} \textit{twilight had not yet come} \colorBulletS{SYN} happen, occur, take place, come about, transpire, fall, present itself, crop up, materialize, arise, arrive, appear, surface, ensue, follow \textbf{3} Take or occupy a specified position in space, order, or priority. {\fontspec{DejaVu Sans}◇} \textit{prisons come well down the list of priorities} \textbf{4} Pass into a specified state, especially one of separation or disunion. {\fontspec{DejaVu Sans}◇} \textit{his shirt had come undone} \colorBulletS{SYN} break up, fall to bits, fall to pieces, come to bits, come to pieces, disintegrate, splinter, come unstuck, crumble, separate, split, tear, collapse, dissolve \textbf{5} Be sold, available, or found in a specified form. {\fontspec{DejaVu Sans}◇} \textit{the cars come with a variety of extras} \colorBulletS{SYN} be available, be made, be produced, be for sale, be on offer \textbf{6} Have an orgasm. {\fontspec{DejaVu Sans}◇} \textit{} \colorBulletS{SYN} climax, achieve orgasm, orgasm}{ \colorBullet{OTHER} come on in: ভিতরে আসো}{}{ \colorBullet{ORIGIN} Old English cuman, of Germanic origin; related to Dutch komen and German kommen.}%
\par%
\entry{commendable}{/kəˈmɛndəb(ə)l/}{প্রশংসনীয়}{ \textsf{\textit{adjective}}\ \textbf{1} Deserving praise. {\fontspec{DejaVu Sans}◇} \textit{he showed commendable restraint} \colorBulletS{SYN} admirable, praiseworthy, laudable, estimable, meritorious, creditable, exemplary, exceptional, noteworthy, notable, honourable, worthy, deserving, respectable, sterling, fine, excellent}{}{}{ \colorBullet{ORIGIN} Late Middle English via Old French from Latin commendabilis, from commendare (see commend).}%
\par%
\entry{commuter}{/kəˈmjuːtə/}{যে ব্যক্তি পরিবহনসংস্থাদির যানবাহনে যাতায়াত করে; নিত্যযাত্রী}{ \textsf{\textit{noun}}\ \textbf{1} A person who travels some distance to work on a regular basis. {\fontspec{DejaVu Sans}◇} \textit{a fault on the line caused widespread delays for commuters} \colorBulletS{SYN} daily traveller, traveller, passenger}{}{}{}%
\par%
\entry{compel}{/kəmˈpɛl/}{বাধ্য করা}{ \textsf{\textit{verb}}\ \textbf{1} Force or oblige (someone) to do something. {\fontspec{DejaVu Sans}◇} \textit{a sense of duty compelled Harry to answer her questions} \colorBulletS{SYN} force, coerce into, pressurize into, pressure, impel, drive, press, push, urge, prevail on}{}{}{ \colorBullet{ORIGIN} Late Middle English from Latin compellere, from com{-} ‘together’ + pellere ‘drive’.}%
\par%
\entry{compelling}{/kəmˈpɛlɪŋ/}{বাধ্যকারী}{ \textsf{\textit{adjective}}\ \textbf{1} Evoking interest, attention, or admiration in a powerfully irresistible way. {\fontspec{DejaVu Sans}◇} \textit{his eyes were strangely compelling} \colorBulletS{SYN} enthralling, captivating, gripping, engrossing, riveting, spellbinding, entrancing, transfixing, mesmerizing, hypnotic, mesmeric, absorbing, fascinating, thrilling, irresistible, addictive}{}{}{}%
\par%
\entry{compensate}{/ˈkɒmpɛnseɪt/}{ক্ষতিপূরণ করা}{ \textsf{\textit{verb}}\ \textbf{1} Give (someone) something, typically money, in recognition of loss, suffering, or injury incurred; recompense. {\fontspec{DejaVu Sans}◇} \textit{payments were made to farmers to compensate them for cuts in subsidies} \colorBulletS{SYN} recompense, repay, pay back, reimburse, remunerate, recoup, requite, indemnify \textbf{2} Reduce or counteract (something unwelcome or unpleasant) by exerting an opposite force or effect. {\fontspec{DejaVu Sans}◇} \textit{the manager is hoping for victory to compensate for the team's dismal league campaign} \colorBulletS{SYN} make amends, make up, make restitution, make reparation, make recompense, recompense, atone, requite, pay}{}{}{ \colorBullet{ORIGIN} Mid 17th century (in the sense ‘counterbalance’): from Latin compensat{-} ‘weighed against’, from the verb compensare, from com{-} ‘together’ + pensare (frequentative of pendere ‘weigh’).}%
\par%
\entry{compensation}{/kɒmpɛnˈseɪʃ(ə)n/}{ক্ষতিপূরণ}{ \textsf{\textit{noun}}\ \textbf{1} Something, typically money, awarded to someone in recognition of loss, suffering, or injury. {\fontspec{DejaVu Sans}◇} \textit{he is seeking compensation for injuries suffered at work} \colorBulletS{SYN} recompense, repayment, payment, reimbursement, remuneration, requital, indemnification, indemnity, redress, satisfaction \textbf{2} The process of concealing or offsetting a psychological difficulty by developing in another direction. {\fontspec{DejaVu Sans}◇} \textit{} \textbf{3} The money received by an employee from an employer as a salary or wages. {\fontspec{DejaVu Sans}◇} \textit{send your CV and current compensation to Executive Search Consultant} \colorBulletS{SYN} salary, wages, wage, pay, earnings, fee, fees, remuneration, take{-}home pay, gross pay, net pay}{}{}{ \colorBullet{ORIGIN} Late Middle English via Old French from Latin compensatio(n{-}), from the verb compensare ‘weigh against’ (see compensate).}%
\par%
\entry{complainant}{/kəmˈpleɪnənt/}{বাদী}{ \textsf{\textit{noun}}\ \textbf{1} A plaintiff in certain lawsuits. {\fontspec{DejaVu Sans}◇} \textit{} \colorBulletS{SYN} litigator, opponent in law, opponent, contestant, contender, disputant, plaintiff, claimant, complainant, petitioner, appellant, respondent, party, interest, defendant, accused}{}{}{ \colorBullet{ORIGIN} Late Middle English from French complaignant, present participle of complaindre ‘to lament’ (see complain).}%
\par%
\entry{complaint}{/kəmˈpleɪnt/}{অভিযোগ}{ \textsf{\textit{noun}}\ \textbf{1} A statement that something is unsatisfactory or unacceptable. {\fontspec{DejaVu Sans}◇} \textit{I intend to make an official complaint} \colorBulletS{SYN} protest, protestation, objection, remonstrance, statement of dissatisfaction, grievance, charge, accusation, criticism \textbf{2} An illness or medical condition, especially a relatively minor one. {\fontspec{DejaVu Sans}◇} \textit{she is receiving treatment for her skin complaint} \colorBulletS{SYN} disorder, disease, infection, affliction, illness, ailment, sickness, malady, malaise, infirmity, indisposition, weakness, condition, problem, upset}{}{}{ \colorBullet{ORIGIN} Late Middle English from Old French complainte, feminine past participle of complaindre ‘to lament’ (see complain).}%
\par%
\entry{complement}{/ˈkɒmplɪm(ə)nt/}{পূরক}{\small{\textsf{\textit{noun, verb}}} \\{\fontspec{DejaVu Sans}▪ }\textsf{\textit{noun}}\\ \textbf{1} A thing that contributes extra features to something else in such a way as to improve or emphasize its quality. {\fontspec{DejaVu Sans}◇} \textit{local ales provide the perfect complement to fine food} \colorBulletS{SYN} accompaniment, companion, addition, supplement, accessory, adjunct, trimming, finishing touch, final touch \textbf{2} A number or quantity of something, especially that required to make a group complete. {\fontspec{DejaVu Sans}◇} \textit{at the moment we have a full complement of staff} \colorBulletS{SYN} amount, total, aggregate, contingent, company \textbf{3} One or more words, phrases, or clauses governed by a verb (or by a nominalization or a predicative adjective) that complete the meaning of the predicate. In generative grammar, all the constituents of a sentence that are governed by a verb form the complement. {\fontspec{DejaVu Sans}◇} \textit{} \textbf{4} A group of proteins present in blood plasma and tissue fluid which combine with an antigen–antibody complex to bring about the lysis of foreign cells. {\fontspec{DejaVu Sans}◇} \textit{} \\{\fontspec{DejaVu Sans}▪ }\textsf{\textit{verb}}\\ \textbf{1} Contribute extra features to (someone or something) in such a way as to improve or emphasize their qualities. {\fontspec{DejaVu Sans}◇} \textit{a classic blazer complements a look that's smart or casual} \colorBulletS{SYN} accompany, go with, round off, set off, suit, harmonize with, be the perfect companion to, be the perfect addition to, add the finishing touch to, add the final touch to, add to, supplement, augment, enhance, complete}{}{}{ \colorBullet{ORIGIN} Late Middle English (in the sense ‘completion’): from Latin complementum, from complere ‘fill up’ (see complete). Compare with compliment.}%
\par%
\entry{complementary}{/kɒmplɪˈmɛnt(ə)ri/}{পরিপূরক}{ \textsf{\textit{adjective}}\ \textbf{1} Combining in such a way as to enhance or emphasize the qualities of each other or another. {\fontspec{DejaVu Sans}◇} \textit{they had different but complementary skills} \colorBulletS{SYN} harmonizing, harmonious, complementing, supportive, supporting, reciprocal, interdependent, interrelated, compatible, corresponding, matching, twin \textbf{2} Relating to complementary medicine. {\fontspec{DejaVu Sans}◇} \textit{complementary therapies such as aromatherapy}}{}{}{}%
\par%
\entry{compliant}{/kəmˈplʌɪənt/}{অনুবর্তী}{ \textsf{\textit{adjective}}\ \textbf{1} Disposed to agree with others or obey rules, especially to an excessive degree; acquiescent. {\fontspec{DejaVu Sans}◇} \textit{a compliant labour force} \colorBulletS{SYN} acquiescent, amenable, biddable, tractable, complaisant, accommodating, cooperative, adaptable \textbf{2} Meeting or in accordance with rules or standards. {\fontspec{DejaVu Sans}◇} \textit{food that is compliant with safety regulations} \textbf{3} Having the property of compliance. {\fontspec{DejaVu Sans}◇} \textit{the conversion of the gel to a much less compliant, rigid glass}}{ \colorBullet{OTHER} compliant to}{}{}%
\par%
\entry{comply}{/kəmˈplʌɪ/}{মেনে চলতে}{ \textsf{\textit{verb}}\ \textbf{1} Act in accordance with a wish or command. {\fontspec{DejaVu Sans}◇} \textit{we are unable to comply with your request} \colorBulletS{SYN} abide by, act in accordance with, observe, obey, adhere to, conform to, follow, respect}{}{}{ \colorBullet{ORIGIN} Late 16th century from Italian complire, Catalan complir, Spanish cumplir, from Latin complere ‘fill up, fulfil’ (see complete). The original sense was ‘fulfil, accomplish’, later ‘fulfil the requirements of courtesy’, hence ‘to be agreeable, to oblige or obey’. Compare with compliment.}%
\par%
\entry{comprehensible}{/kɒmprɪˈhɛnsɪb(ə)l/}{বোধগম্য}{ \textsf{\textit{adjective}}\ \textbf{1} Able to be understood; intelligible. {\fontspec{DejaVu Sans}◇} \textit{clear and comprehensible English} \colorBulletS{SYN} intelligible, understandable, easy to understand, digestible, user{-}friendly, accessible}{}{}{ \colorBullet{ORIGIN} Late 15th century from French compréhensible or Latin comprehensibilis, from comprehens{-} ‘seized, comprised’, from the verb comprehendere (see comprehend).}%
\par%
\entry{compromise}{/ˈkɒmprəmʌɪz/}{আপস}{\small{\textsf{\textit{noun, verb}}} \\{\fontspec{DejaVu Sans}▪ }\textsf{\textit{noun}}\\ \textbf{1} An agreement or settlement of a dispute that is reached by each side making concessions. {\fontspec{DejaVu Sans}◇} \textit{eventually they reached a compromise} \colorBulletS{SYN} agreement, understanding, settlement, terms, accommodation \textbf{2} The expedient acceptance of standards that are lower than is desirable. {\fontspec{DejaVu Sans}◇} \textit{sexism should be tackled without compromise} \\{\fontspec{DejaVu Sans}▪ }\textsf{\textit{verb}}\\ \textbf{1} Settle a dispute by mutual concession. {\fontspec{DejaVu Sans}◇} \textit{in the end we compromised and deferred the issue} \colorBulletS{SYN} meet each other halfway, find the middle ground, come to terms, come to an understanding, make a deal, make concessions, find a happy medium, strike a balance \textbf{2} Expediently accept standards that are lower than is desirable. {\fontspec{DejaVu Sans}◇} \textit{we were not prepared to compromise on safety} \colorBulletS{SYN} change one's mind, give way, give in, yield, acquiesce, compromise, adapt, retract, do a U{-}turn, eat one's words \textbf{3} Bring into disrepute or danger by indiscreet, foolish, or reckless behaviour. {\fontspec{DejaVu Sans}◇} \textit{situations in which his troops could be compromised}}{}{Concentration camp: রাজনৈতিক বন্দীশিবির}{ \colorBullet{ORIGIN} Late Middle English (denoting mutual consent to arbitration): from Old French compromis, from late Latin compromissum ‘a consent to arbitration’, neuter past participle of compromittere, from com{-} ‘together’ + promittere (see promise).}%
\par%
\entry{conception}{/kənˈsɛpʃ(ə)n/}{ধারণা}{ \textsf{\textit{noun}}\ \textbf{1} The action of conceiving a child or of one being conceived. {\fontspec{DejaVu Sans}◇} \textit{an unfertilized egg before conception} \colorBulletS{SYN} inception of pregnancy, conceiving, fertilization, impregnation, insemination \textbf{2} The forming or devising of a plan or idea. {\fontspec{DejaVu Sans}◇} \textit{the time between a product's conception and its launch} \colorBulletS{SYN} inception, genesis, origination, creation, formation, formulation, invention}{}{}{ \colorBullet{ORIGIN} Middle English via Old French from Latin conceptio(n{-}), from the verb concipere (see conceive).}%
\par%
\entry{conclude}{/kənˈkluːd/}{শেষ করা}{ \textsf{\textit{verb}}\ \textbf{1} Bring or come to an end. {\fontspec{DejaVu Sans}◇} \textit{they conclude their study with these words} \colorBulletS{SYN} finish, end, come to an end, draw to a close, wind up, be over, stop, terminate, close, cease \textbf{2} Arrive at a judgement or opinion by reasoning. {\fontspec{DejaVu Sans}◇} \textit{the doctors concluded that Esther had suffered a stroke} \colorBulletS{SYN} come to the conclusion, deduce, infer, draw the inference, gather, judge, decide}{}{}{ \colorBullet{ORIGIN} Middle English (in the sense ‘convince’): from Latin concludere, from con{-} ‘completely’ + claudere ‘to shut’.}%
\par%
\entry{conclusive}{/kənˈkluːsɪv/}{চূড়ান্ত}{ \textsf{\textit{adjective}}\ \textbf{1} (of evidence or argument) having or likely to have the effect of proving a case; decisive. {\fontspec{DejaVu Sans}◇} \textit{conclusive evidence} \colorBulletS{SYN} incontrovertible, incontestable, irrefutable, unquestionable, undeniable, indisputable, unassailable, beyond dispute, beyond question, beyond doubt, beyond a shadow of a doubt, certain, decisive, convincing, clinching, definitive, definite, positive, final, ultimate, categorical, demonstrative, unequivocal, unarguable, unanswerable, uncontroversial}{}{}{ \colorBullet{ORIGIN} Late 16th century (in the sense ‘summing up’): from late Latin conclusivus, from Latin conclus{-} ‘closed up’, from the verb concludere (see conclusion).}%
\par%
\entry{concussion}{/kənˈkʌʃ(ə)n/}{আলোড়ন}{ \textsf{\textit{noun}}\ \textbf{1} Temporary unconsciousness or confusion and other symptoms caused by a blow on the head. {\fontspec{DejaVu Sans}◇} \textit{he was carried off the pitch with concussion} \colorBulletS{SYN} temporary unconsciousness, temporary loss of consciousness, bang on the head \textbf{2} A violent shock as from a heavy blow. {\fontspec{DejaVu Sans}◇} \textit{the ground shuddered with the concussion of the blast} \colorBulletS{SYN} force, impact, shock}{}{}{ \colorBullet{ORIGIN} Late Middle English from Latin concussio(n{-}), from the verb concutere ‘dash together, shake’ (see concuss).}%
\par%
\entry{condemn}{/kənˈdɛm/}{নিন্দা করা}{ \textsf{\textit{verb}}\ \textbf{1} Express complete disapproval of; censure. {\fontspec{DejaVu Sans}◇} \textit{most leaders roundly condemned the attack} \colorBulletS{SYN} censure, criticize, castigate, attack, denounce, deplore, decry, revile, inveigh against, blame, chastise, berate, upbraid, reprimand, rebuke, reprove, reprehend, take to task, find fault with, give someone a bad press, give something a bad press \textbf{2} Sentence (someone) to a particular punishment, especially death. {\fontspec{DejaVu Sans}◇} \textit{the rebels had been condemned to death} \colorBulletS{SYN} sentence, pass sentence on}{}{}{ \colorBullet{ORIGIN} Middle English (in condemn (sense 2)): from Old French condemner, from Latin condemnare, from con{-} (expressing intensive force) + damnare ‘inflict loss on’ (see damn).}%
\par%
\entry{condescension}{/ˌkɒndɪˈsɛnʃn/}{অনুকম্পা}{ \textsf{\textit{noun}}\ \textbf{1} An attitude of patronizing superiority; disdain. {\fontspec{DejaVu Sans}◇} \textit{a tone of condescension}}{}{}{}%
\par%
\entry{conduct}{/ˈkɒndʌkt/}{আবহ}{\small{\textsf{\textit{noun, verb}}} \\{\fontspec{DejaVu Sans}▪ }\textsf{\textit{noun}}\\ \textbf{1} The manner in which a person behaves, especially in a particular place or situation. {\fontspec{DejaVu Sans}◇} \textit{they were arrested for disorderly conduct} \colorBulletS{SYN} behaviour, way of behaving, performance, comportment, demeanour, bearing, deportment \textbf{2} The manner in which an organization or activity is managed or directed. {\fontspec{DejaVu Sans}◇} \textit{the conduct of the elections} \colorBulletS{SYN} management, managing, running, direction, control, controlling, overseeing, supervision, regulation, leadership, masterminding, administration, organization, coordination, orchestration, handling, guidance, carrying out, carrying on \\{\fontspec{DejaVu Sans}▪ }\textsf{\textit{verb}}\\ \textbf{1} Organize and carry out. {\fontspec{DejaVu Sans}◇} \textit{in the second trial he conducted his own defence} \colorBulletS{SYN} manage, direct, run, be in control of, control, oversee, supervise, be in charge of, preside over, regulate, mastermind, administer, organize, coordinate, orchestrate, handle, guide, govern, lead, carry out, carry on \textbf{2} Lead or guide (someone) to or around a particular place. {\fontspec{DejaVu Sans}◇} \textit{he conducted us through his personal gallery of the Civil War} \colorBulletS{SYN} escort, guide, lead, usher, pilot, accompany, show, show someone the way \textbf{3} Transmit (a form of energy such as heat or electricity) by conduction. {\fontspec{DejaVu Sans}◇} \textit{heat is conducted to the surface} \colorBulletS{SYN} transmit, convey, carry, transfer, pass on, hand on, communicate, impart, channel, bear, relay, dispatch, mediate \textbf{4} Direct the performance of (a piece of music or an orchestra, choir, etc.) {\fontspec{DejaVu Sans}◇} \textit{the concert is to be conducted by Sir Simon Rattle} \textbf{5} Behave in a specified way. {\fontspec{DejaVu Sans}◇} \textit{he conducted himself with the utmost propriety} \colorBulletS{SYN} behave, perform, act, acquit oneself, bear oneself, carry oneself}{}{}{ \colorBullet{ORIGIN} Middle English from Old French, from Latin conduct{-} ‘brought together’, from the verb conducere. The term originally denoted a provision for safe passage, surviving in safe conduct; later the verb sense ‘lead, guide’ arose, hence ‘manage’ and ‘management’ (late Middle English), later ‘management of oneself, behaviour’ (mid 16th century). The original form of the word was conduit, which was preserved only in the sense ‘channel’ (see conduit); in other uses the spelling was influenced by Latin.}%
\par%
\entry{confer}{/kənˈfəː/}{প্রদায়ক}{ \textsf{\textit{verb}}\ \textbf{1} Grant (a title, degree, benefit, or right) {\fontspec{DejaVu Sans}◇} \textit{the Minister may have exceeded the powers conferred on him by Parliament} \colorBulletS{SYN} bestow on, present to, present with, grant to, award to, decorate with, honour with, give to, give out to, gift with, endow with, vest in, hand out to, extend to, vouchsafe to, accord to \textbf{2} Have discussions; exchange opinions. {\fontspec{DejaVu Sans}◇} \textit{the officials were conferring with allies} \colorBulletS{SYN} consult, have discussions, discuss things, exchange views, talk, have a talk, speak, converse, communicate, have a chat, have a tête{-}à{-}tête}{}{}{ \colorBullet{ORIGIN} Late Middle English (in the general sense ‘bring together’, also in confer (sense 2)): from Latin conferre, from con{-} ‘together’ + ferre ‘bring’.}%
\par%
\entry{confess}{/kənˈfɛs/}{স্বীকার করা}{ \textsf{\textit{verb}}\ \textbf{1} Admit that one has committed a crime or done something wrong. {\fontspec{DejaVu Sans}◇} \textit{he confessed that he had attacked the old man} \colorBulletS{SYN} admit, acknowledge, reveal, make known, disclose, divulge, make public, avow, declare, blurt out, profess, own up to, tell all about, bring into the open, bring to light}{}{}{ \colorBullet{ORIGIN} Late Middle English from Old French confesser, from Latin confessus, past participle of confiteri ‘acknowledge’, from con{-} (expressing intensive force) + fateri ‘declare, avow’.}%
\par%
\entry{confessional}{/kənˈfɛʃ(ə)n(ə)l/}{স্বীকারোক্তিমূলক}{\small{\textsf{\textit{adjective, noun}}} \\{\fontspec{DejaVu Sans}▪ }\textsf{\textit{adjective}}\\ \textbf{1} (of speech or writing) in which a person reveals private thoughts or admits to past incidents, especially ones about which they feel ashamed or embarrassed. {\fontspec{DejaVu Sans}◇} \textit{the autobiography is remarkably confessional} \textbf{2} Relating to confessions of faith or doctrinal systems. {\fontspec{DejaVu Sans}◇} \textit{the confessional approach to religious education} \\{\fontspec{DejaVu Sans}▪ }\textsf{\textit{noun}}\\ \textbf{1} An enclosed stall in a church divided by a screen or curtain in which a priest sits to hear confessions. {\fontspec{DejaVu Sans}◇} \textit{the secrets of the confessional} \textbf{2} An acknowledgement that one has done something shameful or embarrassing; a confession. {\fontspec{DejaVu Sans}◇} \textit{tabloid confessionals}}{}{Confessional statement: স্বীকারোক্তিমূলক বিবৃতি }{ \colorBullet{ORIGIN} Late Middle English (as an adjective): the adjective from confession+ {-}al; the noun via French from Italian confessionale, from medieval Latin, neuter of confessionalis, from Latin confessio(n{-}), from confiteri ‘acknowledge’ (see confess).}%
\par%
\entry{confidence}{/ˈkɒnfɪd(ə)ns/}{বিশ্বাস}{ \textsf{\textit{noun}}\ \textbf{1} The feeling or belief that one can have faith in or rely on someone or something. {\fontspec{DejaVu Sans}◇} \textit{we had every confidence in the staff} \colorBulletS{SYN} trust, belief, faith, credence, conviction \textbf{2} The telling of private matters or secrets with mutual trust. {\fontspec{DejaVu Sans}◇} \textit{someone with whom you may raise your suspicions in confidence}}{}{}{ \colorBullet{ORIGIN} Late Middle English from Latin confidentia, from confidere ‘have full trust’ (see confident).}%
\par%
\entry{confidential}{/kɒnfɪˈdɛnʃ(ə)l/}{গোপনীয়}{ \textsf{\textit{adjective}}\ \textbf{1} Intended to be kept secret. {\fontspec{DejaVu Sans}◇} \textit{confidential information} \colorBulletS{SYN} private, personal, intimate, privileged, quiet}{}{}{}%
\par%
\entry{confine}{/kənˈfʌɪn/}{পুরা}{\small{\textsf{\textit{noun, verb}}} \\{\fontspec{DejaVu Sans}▪ }\textsf{\textit{noun}}\\ \textbf{1} The borders or boundaries of a place, especially with regard to their restricting freedom of movement. {\fontspec{DejaVu Sans}◇} \textit{within the confines of the hall escape was difficult} \colorBulletS{SYN} limits, outer limits, borders, boundaries, margins, extremities, edges, fringes, marches \\{\fontspec{DejaVu Sans}▪ }\textsf{\textit{verb}}\\ \textbf{1} Keep or restrict someone or something within certain limits of (space, scope, or time) {\fontspec{DejaVu Sans}◇} \textit{he does not confine his message to high politics} \colorBulletS{SYN} enclose, incarcerate, imprison, intern, impound, hold captive, trap}{}{}{ \colorBullet{ORIGIN} Late Middle English (as a noun): from French confins (plural noun), from Latin confinia, from confinis ‘bordering’, from con{-} ‘together’ + finis ‘end, limit’ (plural fines ‘territory’). The verb senses are from French confiner, based on Latin confinis.}%
\par%
\entry{confiscate}{/ˈkɒnfɪskeɪt/}{বাজেয়াপ্ত করা}{ \textsf{\textit{verb}}\ \textbf{1} Take or seize (someone's property) with authority. {\fontspec{DejaVu Sans}◇} \textit{the guards confiscated his camera} \colorBulletS{SYN} impound, seize, commandeer, requisition, appropriate, expropriate, take possession of, sequester, sequestrate, take away, take over, take, annex}{}{}{ \colorBullet{ORIGIN} Mid 16th century from Latin confiscat{-} ‘put away in a chest, consigned to the public treasury’, from the verb confiscare, based on con{-} ‘together’ + fiscus ‘chest, treasury’.}%
\par%
\entry{confrontation}{/ˌkɒnfrʌnˈteɪʃn/}{মুকাবিলা}{ \textsf{\textit{noun}}\ \textbf{1} A hostile or argumentative situation or meeting between opposing parties. {\fontspec{DejaVu Sans}◇} \textit{a confrontation with the legislature} \colorBulletS{SYN} conflict, clash, brush, fight, battle, contest, encounter, head{-}to{-}head, face{-}off, engagement, tangle, skirmish, collision, meeting, duel, incident, high noon}{ \colorBullet{OTHER} confrontation over}{}{}%
\par%
\entry{congestion}{/kənˈdʒɛstʃ(ə)n/}{পূর্ণতা}{ \textsf{\textit{noun}}\ \textbf{1} The state of being congested. {\fontspec{DejaVu Sans}◇} \textit{the new bridge should ease congestion in the area} \colorBulletS{SYN} crowding, overcrowding}{}{}{ \colorBullet{ORIGIN} Late Middle English via Old French from Latin congestio(n{-}), from congere ‘heap up’, from con{-} ‘together’ + gerere ‘bring’.}%
\par%
\entry{conquest}{/ˈkɒŋkwɛst/}{বিজয়}{ \textsf{\textit{noun}}\ \textbf{1} The subjugation and assumption of control of a place or people by military force. {\fontspec{DejaVu Sans}◇} \textit{the conquest of the Aztecs by the Spanish} \colorBulletS{SYN} defeat, beating, conquering, vanquishment, vanquishing, trouncing, annihilation, overpowering, overthrow, subduing, subjugation, rout, mastery, crushing}{}{}{ \colorBullet{ORIGIN} Middle English from Old French conquest(e), based on Latin conquirere (see conquer).}%
\par%
\entry{consent}{/kənˈsɛnt/}{সম্মতি}{\small{\textsf{\textit{noun, verb}}} \\{\fontspec{DejaVu Sans}▪ }\textsf{\textit{noun}}\\ \textbf{1} Permission for something to happen or agreement to do something. {\fontspec{DejaVu Sans}◇} \textit{no change may be made without the consent of all the partners} \colorBulletS{SYN} agreement, assent, concurrence, accord \\{\fontspec{DejaVu Sans}▪ }\textsf{\textit{verb}}\\ \textbf{1} Give permission for something to happen. {\fontspec{DejaVu Sans}◇} \textit{he consented to a search by a detective} \colorBulletS{SYN} agree to, assent to, allow, give permission for, sanction, accept, approve, acquiesce in, go along with, accede to, concede to, yield to, give in to, submit to, comply with, abide by, concur with, conform to}{}{}{ \colorBullet{ORIGIN} Middle English from Old French consente (noun), consentir (verb), from Latin consentire, from con{-} ‘together’ + sentire ‘feel’.}%
\par%
\entry{consequence}{/ˈkɒnsɪkw(ə)ns/}{ফল; পরিণতি}{ \textsf{\textit{noun}}\ \textbf{1} A result or effect, typically one that is unwelcome or unpleasant. {\fontspec{DejaVu Sans}◇} \textit{abrupt withdrawal of drug treatment can have serious consequences} \colorBulletS{SYN} result, upshot, outcome, out{-}turn, sequel, effect, reaction, repercussion, reverberations, ramification, end, end result, conclusion, termination, culmination, denouement, corollary, concomitant, aftermath, fruit, fruits, product, produce, by{-}product \textbf{2} Importance or relevance. {\fontspec{DejaVu Sans}◇} \textit{the past is of no consequence} \colorBulletS{SYN} importance, import, significance, account, moment, momentousness, substance, note, mark, prominence, value, weightiness, weight, concern, interest, gravity, seriousness \textbf{3} A game in which a narrative is made up by the players in turn, each ignorant of what has already been contributed. {\fontspec{DejaVu Sans}◇} \textit{}}{}{}{ \colorBullet{ORIGIN} Late Middle English via Old French from Latin consequentia, from consequent{-} ‘following closely’, from the verb consequi.}%
\par%
\entry{considerable}{/kənˈsɪd(ə)rəb(ə)l/}{গণ্যমান্য}{ \textsf{\textit{adjective}}\ \textbf{1} Notably large in size, amount, or extent. {\fontspec{DejaVu Sans}◇} \textit{a position of considerable influence} \colorBulletS{SYN} sizeable, substantial, appreciable, significant}{}{}{ \colorBullet{ORIGIN} Late Middle English (in the sense ‘capable of being considered’): from medieval Latin considerabilis ‘worthy of consideration’, from Latin considerare (see consider).}%
\par%
\entry{considerably}{/kənˈsɪd(ə)rəbli/}{অনেক}{ \textsf{\textit{adverb}}\ \textbf{1} By a notably large amount or to a notably large extent; greatly. {\fontspec{DejaVu Sans}◇} \textit{things have improved considerably over the last few years} \colorBulletS{SYN} greatly, much, very much, a great deal, a lot, lots, a fair amount}{}{}{}%
\par%
\entry{consigliere}{/ˌkɒnsɪˈljɛːreɪ/}{আপ্তসহায়ক}{ \textsf{\textit{noun}}\ \textbf{1} A member of a Mafia family who serves as an adviser to the leader and resolves disputes within the family. {\fontspec{DejaVu Sans}◇} \textit{} \colorBulletS{SYN} counsellor, mentor, guide, consultant, consultee, confidant, confidante, guide, right hand man, right hand woman, aide, helper}{}{}{ \colorBullet{ORIGIN} Italian, literally ‘a member of a council’.}%
\par%
\entry{consignment}{/kənˈsʌɪnm(ə)nt/}{চালান}{ \textsf{\textit{noun}}\ \textbf{1} A batch of goods destined for or delivered to someone. {\fontspec{DejaVu Sans}◇} \textit{a consignment of drugs} \colorBulletS{SYN} delivery, shipment, load, containerload, shipload, boatload, lorryload, truckload, cargo}{}{}{}%
\par%
\entry{conspiracy}{/kənˈspɪrəsi/}{চক্রান্ত}{ \textsf{\textit{noun}}\ \textbf{1} A secret plan by a group to do something unlawful or harmful. {\fontspec{DejaVu Sans}◇} \textit{a conspiracy to destroy the government} \colorBulletS{SYN} plot, scheme, stratagem, plan, machination, cabal, intrigue, palace intrigue}{}{}{ \colorBullet{ORIGIN} Late Middle English from Anglo{-}Norman French conspiracie, alteration of Old French conspiration, based on Latin conspirare ‘agree, plot’ (see conspire).}%
\par%
\entry{conspire}{/kənˈspʌɪə/}{চক্রান্ত}{ \textsf{\textit{verb}}\ \textbf{1} Make secret plans jointly to commit an unlawful or harmful act. {\fontspec{DejaVu Sans}◇} \textit{they conspired against him} \colorBulletS{SYN} plot, hatch a plot, form a conspiracy, scheme, plan, lay plans, intrigue, collude, connive, collaborate, consort, machinate, manoeuvre, be hand in glove, work hand in glove}{}{}{ \colorBullet{ORIGIN} Late Middle English from Old French conspirer, from Latin conspirare ‘agree, plot’, from con{-} ‘together with’ + spirare ‘breathe’.}%
\par%
\entry{constipated}{/ˈkɒnstɪpeɪtɪd/}{কোষ্ঠকাঠিন্য}{ \textsf{\textit{adjective}}\ \textbf{1} Affected with constipation. {\fontspec{DejaVu Sans}◇} \textit{regular heroin users can become constipated}}{}{}{ \colorBullet{ORIGIN} Mid 16th century from Latin constipat{-} ‘crowded or pressed together’, from the verb constipare, from con{-} ‘together’ + stipare ‘press, cram’.}%
\par%
\entry{contain}{/kənˈteɪn/}{অন্তর্ভুক্ত}{ \textsf{\textit{verb}}\ \textbf{1} Have or hold (someone or something) within. {\fontspec{DejaVu Sans}◇} \textit{the cigarettes were thought to contain cannabis} \colorBulletS{SYN} hold, have capacity for, have room for, have seating for, have space for, carry, accommodate, seat \textbf{2} Control or restrain (oneself or a feeling) {\fontspec{DejaVu Sans}◇} \textit{he must contain his hatred} \colorBulletS{SYN} restrain, curb, rein in, suppress, repress, stifle, subdue, quell, limit, swallow, bottle up, keep under control, keep back, hold in, keep in check}{}{}{ \colorBullet{ORIGIN} Middle English from Old French contenir, from Latin continere, from con{-} ‘altogether’ + tenere ‘to hold’.}%
\par%
\entry{contaminated}{/kənˈtamɪneɪtɪd/}{কলুষিত}{ \textsf{\textit{adjective}}\ \textbf{1} Having been made impure by exposure to or addition of a poisonous or polluting substance. {\fontspec{DejaVu Sans}◇} \textit{contaminated blood products}}{}{}{}%
\par%
\entry{contemplate}{/ˈkɒntɛmpleɪt/}{চিন্তা}{ \textsf{\textit{verb}}\ \textbf{1} Look thoughtfully for a long time at. {\fontspec{DejaVu Sans}◇} \textit{he contemplated his image in the mirrors} \colorBulletS{SYN} look at, view, regard, examine, inspect, observe, survey, study, scrutinize, scan, stare at, gaze at, eye, take a good look at}{}{}{ \colorBullet{ORIGIN} Late 16th century from Latin contemplat{-} ‘surveyed, observed, contemplated’, from the verb contemplari, based on templum ‘place for observation’.}%
\par%
\entry{contemporary}{/kənˈtɛmp(ə)r(ər)i/}{সমসাময়িক}{\small{\textsf{\textit{adjective, noun}}} \\{\fontspec{DejaVu Sans}▪ }\textsf{\textit{adjective}}\\ \textbf{1} Living or occurring at the same time. {\fontspec{DejaVu Sans}◇} \textit{the event was recorded by a contemporary historian} \textbf{2} Belonging to or occurring in the present. {\fontspec{DejaVu Sans}◇} \textit{the tension and complexities of our contemporary society} \colorBulletS{SYN} modern, present{-}day, present, current, present{-}time, immediate, extant \\{\fontspec{DejaVu Sans}▪ }\textsf{\textit{noun}}\\ \textbf{1} A person or thing living or existing at the same time as another. {\fontspec{DejaVu Sans}◇} \textit{he was a contemporary of Darwin} \colorBulletS{SYN} peer, fellow}{}{}{ \colorBullet{ORIGIN} Mid 17th century from medieval Latin contemporarius, from con{-} ‘together with’ + tempus, tempor{-} ‘time’ (on the pattern of Latin contemporaneus and late Latin contemporalis).}%
\par%
\entry{contempt}{/kənˈtɛm(p)t/}{অবজ্ঞা}{ \textsf{\textit{noun}}\ \textbf{1} The feeling that a person or a thing is worthless or beneath consideration. {\fontspec{DejaVu Sans}◇} \textit{Pam stared at the girl with total contempt} \colorBulletS{SYN} scorn, disdain, disrespect, deprecation, disparagement, denigration, opprobrium, odium, obloquy, scornfulness}{}{}{ \colorBullet{ORIGIN} Late Middle English from Latin contemptus, from contemnere (see contemn).}%
\par%
\entry{contemptible}{/kənˈtɛm(p)tɪb(ə)l/}{নীচ}{ \textsf{\textit{adjective}}\ \textbf{1} Deserving contempt; despicable. {\fontspec{DejaVu Sans}◇} \textit{a display of contemptible cowardice} \colorBulletS{SYN} despicable, detestable, hateful, reprehensible, deplorable, loathsome, odious, revolting, execrable, unspeakable, heinous, shocking, offensive}{}{}{ \colorBullet{ORIGIN} Late Middle English from Old French, or from late Latin contemptibilis, from Latin contemnere (see contemn).}%
\par%
\entry{content}{/kənˈtɛnt/}{সন্তুষ্ট}{\small{\textsf{\textit{adjective, noun, verb}}} \\{\fontspec{DejaVu Sans}▪ }\textsf{\textit{adjective}}\\ \textbf{1} In a state of peaceful happiness. {\fontspec{DejaVu Sans}◇} \textit{he seemed more content, less bitter} \colorBulletS{SYN} contented, satisfied, pleased \\{\fontspec{DejaVu Sans}▪ }\textsf{\textit{noun}}\\ \textbf{1} A state of satisfaction. {\fontspec{DejaVu Sans}◇} \textit{the greater part of the century was a time of content} \colorBulletS{SYN} contentedness, content, satisfaction, fulfilment \textbf{2} A member of the British House of Lords who votes for a particular motion. {\fontspec{DejaVu Sans}◇} \textit{The chairman of the committee said he was disappointed at the prospect of the contents of the house leaving the country.} \\{\fontspec{DejaVu Sans}▪ }\textsf{\textit{verb}}\\ \textbf{1} Satisfy (someone) {\fontspec{DejaVu Sans}◇} \textit{nothing would content her apart from going off to Barcelona} \colorBulletS{SYN} soothe, pacify, placate, appease, please, mollify, make happy, satisfy, still, quieten, silence}{}{}{ \colorBullet{ORIGIN} Late Middle English via Old French from Latin contentus ‘satisfied’, past participle of continere (see contain).}%
\par%
\entry{content}{/ˈkɒntɛnt/}{সন্তুষ্ট}{ \textsf{\textit{noun}}\ \textbf{1} The things that are held or included in something. {\fontspec{DejaVu Sans}◇} \textit{she unscrewed the top of the flask and drank the contents} \colorBulletS{SYN} things inside, content, load}{}{}{ \colorBullet{ORIGIN} Late Middle English from medieval Latin contentum (plural contenta ‘things contained’), neuter past participle of continere (see contain).}%
\par%
\entry{contention}{/kənˈtɛnʃ(ə)n/}{তর্ক}{ \textsf{\textit{noun}}\ \textbf{1} Heated disagreement. {\fontspec{DejaVu Sans}◇} \textit{the captured territory was the main area of contention between the two countries} \colorBulletS{SYN} disagreement, dispute, disputation, argument, variance \textbf{2} An assertion, especially one maintained in argument. {\fontspec{DejaVu Sans}◇} \textit{Freud's contention that all dreams were wish fulfilment} \colorBulletS{SYN} argument, claim, plea, submission, allegation}{}{}{ \colorBullet{ORIGIN} Late Middle English from Latin contentio(n{-}), from contendere ‘strive with’ (see contend).}%
\par%
\entry{contest}{/ˈkɒntɛst/}{প্রতিযোগিতা}{\small{\textsf{\textit{noun, verb}}} \\{\fontspec{DejaVu Sans}▪ }\textsf{\textit{noun}}\\ \textbf{1} An event in which people compete for supremacy in a sport or other activity, or in a quality. {\fontspec{DejaVu Sans}◇} \textit{a tennis contest} \colorBulletS{SYN} competition, match, tournament, game, meet \\{\fontspec{DejaVu Sans}▪ }\textsf{\textit{verb}}\\ \textbf{1} Engage in competition to attain (a position of power) {\fontspec{DejaVu Sans}◇} \textit{she declared her intention to contest the presidency} \colorBulletS{SYN} compete for, contend for, vie for, challenge for, fight for, fight over, battle for, struggle for, tussle for \textbf{2} Oppose (an action or theory) as mistaken or wrong. {\fontspec{DejaVu Sans}◇} \textit{the former chairman contests his dismissal} \colorBulletS{SYN} oppose, object to, challenge, dispute, take a stand against, resist, defy, strive against, struggle against, take issue with}{}{}{ \colorBullet{ORIGIN} Late 16th century (as a verb in the sense ‘swear to, attest’): from Latin contestari ‘call upon to witness, initiate (by calling witnesses)’, from con{-} ‘together’ + testare ‘to witness’. The senses ‘wrangle, struggle for’ arose in the early 17th century, whence the current noun and verb senses.}%
\par%
\entry{context}{/ˈkɒntɛkst/}{প্রসঙ্গ}{ \textsf{\textit{noun}}\ \textbf{1} The circumstances that form the setting for an event, statement, or idea, and in terms of which it can be fully understood. {\fontspec{DejaVu Sans}◇} \textit{the proposals need to be considered in the context of new European directives} \colorBulletS{SYN} circumstances, conditions, surroundings, factors, state of affairs}{}{To put this in context: এ প্রসঙ্গে উল্লেখ্য}{ \colorBullet{ORIGIN} Late Middle English (denoting the construction of a text): from Latin contextus, from con{-} ‘together’ + texere ‘to weave’.}%
\par%
\entry{contextual}{/kənˈtɛkstʃʊəl/}{বর্ণনাপ্রাসঙ্গিক}{ \textsf{\textit{adjective}}\ \textbf{1} Depending on or relating to the circumstances that form the setting for an event, statement, or idea. {\fontspec{DejaVu Sans}◇} \textit{he included contextual information in footnotes}}{}{}{}%
\par%
\entry{contraband}{/ˈkɒntrəband/}{নিষিদ্ধ}{\small{\textsf{\textit{adjective, noun}}} \\{\fontspec{DejaVu Sans}▪ }\textsf{\textit{adjective}}\\ \textbf{1} Imported or exported illegally, either in defiance of a total ban or without payment of duty. {\fontspec{DejaVu Sans}◇} \textit{contraband brandy} \colorBulletS{SYN} smuggled, black{-}market, bootleg, bootlegged, under the counter, illegal, illicit, unlawful \\{\fontspec{DejaVu Sans}▪ }\textsf{\textit{noun}}\\ \textbf{1} Goods that have been imported or exported illegally. {\fontspec{DejaVu Sans}◇} \textit{customs men had searched the carriages for contraband} \colorBulletS{SYN} booty, spoils, plunder, stolen goods, contraband, pillage}{}{}{ \colorBullet{ORIGIN} Late 16th century from Spanish contrabanda, from Italian contrabando, from contra{-} ‘against’ + bando ‘proclamation, ban’.}%
\par%
\entry{contrast}{/ˈkɒntrɑːst/}{বিপরীত হত্তয়া}{\small{\textsf{\textit{noun, verb}}} \\{\fontspec{DejaVu Sans}▪ }\textsf{\textit{noun}}\\ \textbf{1} The state of being strikingly different from something else in juxtaposition or close association. {\fontspec{DejaVu Sans}◇} \textit{the day began cold and blustery, in contrast to almost two weeks of uninterrupted sunshine} \colorBulletS{SYN} difference, dissimilarity, disparity, dissimilitude, distinction, contradistinction, divergence, variance, variation, differentiation \\{\fontspec{DejaVu Sans}▪ }\textsf{\textit{verb}}\\ \textbf{1} Differ strikingly. {\fontspec{DejaVu Sans}◇} \textit{his friend's success contrasted with his own failure} \colorBulletS{SYN} differ from, be at variance with, be contrary to, conflict with, go against, be at odds with, be in opposition to, disagree with, clash with}{}{}{ \colorBullet{ORIGIN} Late 17th century (as a term in fine art, in the sense ‘juxtapose so as to bring out differences in form and colour’): from French contraste (noun), contraster (verb), via Italian from medieval Latin contrastare, from Latin contra{-} ‘against’ + stare ‘stand’.}%
\par%
\entry{controversial}{/kɒntrəˈvəːʃ(ə)l/}{বিতর্কমূলক}{ \textsf{\textit{adjective}}\ \textbf{1} Giving rise or likely to give rise to controversy or public disagreement. {\fontspec{DejaVu Sans}◇} \textit{years of wrangling over a controversial bypass} \colorBulletS{SYN} contentious, disputed, contended, at issue, moot, disputable, debatable, arguable, vexed, open to discussion, open to question, under discussion}{}{}{ \colorBullet{ORIGIN} Late 16th century from late Latin controversialis, from controversia (see controversy).}%
\par%
\entry{controversy}{/ˈkɒntrəvəːsi/}{বিতর্ক}{ \textsf{\textit{noun}}\ \textbf{1} Prolonged public disagreement or heated discussion. {\fontspec{DejaVu Sans}◇} \textit{the design of the building has caused controversy} \colorBulletS{SYN} disagreement, dispute, argument, debate, dissension, contention, disputation, altercation, wrangle, quarrel, squabble, war of words, storm}{}{}{ \colorBullet{ORIGIN} Late Middle English from Latin controversia, from controversus ‘turned against, disputed’, from contro{-} (variant of contra{-} ‘against’) + versus, past participle of vertere ‘to turn’.}%
\par%
\entry{convenience}{/kənˈviːnɪəns/}{সুবিধা}{ \textsf{\textit{noun}}\ \textbf{1} The state of being able to proceed with something without difficulty. {\fontspec{DejaVu Sans}◇} \textit{services should be run to suit the convenience of customers, not of staff} \colorBulletS{SYN} benefit, use, good, comfort, ease, enjoyment, satisfaction \textbf{2} A public toilet. {\fontspec{DejaVu Sans}◇} \textit{the large council car park next to the public conveniences}}{}{}{ \colorBullet{ORIGIN} Late Middle English from Latin convenientia, from convenient{-} ‘assembling, agreeing’, from the verb convenire (see convene).}%
\par%
\entry{conventional}{/kənˈvɛnʃ(ə)n(ə)l/}{প্রচলিত}{ \textsf{\textit{adjective}}\ \textbf{1} Based on or in accordance with what is generally done or believed. {\fontspec{DejaVu Sans}◇} \textit{a conventional morality had dictated behaviour} \colorBulletS{SYN} normal, standard, regular, ordinary, usual, traditional, typical, common \textbf{2} (of a bid) intended to convey a particular meaning according to an agreed convention. {\fontspec{DejaVu Sans}◇} \textit{West made a conventional bid showing a hand with at least 5 spades}}{}{}{ \colorBullet{ORIGIN} Late 15th century (in the sense ‘relating to a formal agreement or convention’): from French conventionnel or late Latin conventionalis, from Latin conventio(n{-}) ‘meeting, covenant’, from the verb convenire (see convene).}%
\par%
\entry{convict}{/kənˈvɪkt/}{আসামি}{\small{\textsf{\textit{noun, verb}}} \\{\fontspec{DejaVu Sans}▪ }\textsf{\textit{noun}}\\ \textbf{1} A person found guilty of a criminal offence and serving a sentence of imprisonment. {\fontspec{DejaVu Sans}◇} \textit{two escaped convicts kidnapped them at gunpoint} \colorBulletS{SYN} prisoner, inmate \\{\fontspec{DejaVu Sans}▪ }\textsf{\textit{verb}}\\ \textbf{1} Declare (someone) to be guilty of a criminal offence by the verdict of a jury or the decision of a judge in a court of law. {\fontspec{DejaVu Sans}◇} \textit{the theives were convicted of the robbery} \colorBulletS{SYN} declare guilty, find guilty, pronounce guilty}{}{}{ \colorBullet{ORIGIN} Middle English from Latin convict{-} ‘demonstrated, refuted, convicted’, from the verb convincere (see convince). The noun is from obsolete convict ‘convicted’.}%
\par%
\entry{conviction}{/kənˈvɪkʃ(ə)n/}{দণ্ডাজ্ঞা}{ \textsf{\textit{noun}}\ \textbf{1} A formal declaration by the verdict of a jury or the decision of a judge in a court of law that someone is guilty of a criminal offence. {\fontspec{DejaVu Sans}◇} \textit{she had a previous conviction for a similar offence} \colorBulletS{SYN} declaration of guilt, pronouncement of guilt, sentence, judgement \textbf{2} A firmly held belief or opinion. {\fontspec{DejaVu Sans}◇} \textit{she takes pride in stating her political convictions} \colorBulletS{SYN} belief, opinion, view, thought, persuasion, idea, position, stance}{}{}{ \colorBullet{ORIGIN} Late Middle English from Latin convictio(n{-}), from the verb convincere (see convince).}%
\par%
\entry{convince}{/kənˈvɪns/}{সন্তুষ্ট}{ \textsf{\textit{verb}}\ \textbf{1} Cause (someone) to believe firmly in the truth of something. {\fontspec{DejaVu Sans}◇} \textit{Robert's expression had obviously convinced her of his innocence} \colorBulletS{SYN} persuade, satisfy, prove to, cause to feel certain}{}{}{ \colorBullet{ORIGIN} Mid 16th century (in the sense ‘overcome, defeat in argument’): from Latin convincere, from con{-} ‘with’ + vincere ‘conquer’. Compare with convict.}%
\par%
\entry{convincing}{/kənˈvɪnsɪŋ/}{বিশ্বাসী}{ \textsf{\textit{adjective}}\ \textbf{1} Capable of causing someone to believe that something is true or real. {\fontspec{DejaVu Sans}◇} \textit{there is no convincing evidence that advertising influences total alcohol consumption} \colorBulletS{SYN} cogent, persuasive, powerful, potent, strong, forceful, compelling, irresistible, telling, conclusive, incontrovertible, unanswerable, incontestable, unassailable}{}{}{}%
\par%
\entry{cop}{/kɒp/}{পুলিশ}{\small{\textsf{\textit{noun, verb}}} \\{\fontspec{DejaVu Sans}▪ }\textsf{\textit{noun}}\\ \textbf{1} A police officer. {\fontspec{DejaVu Sans}◇} \textit{a cop in a patrol car gave chase} \colorBulletS{SYN} policeman, policewoman, officer of the law, law enforcement agent, law enforcement officer, officer \textbf{2} Shrewdness; practical intelligence. {\fontspec{DejaVu Sans}◇} \textit{he had the cop{-}on to stay clear of Hugh Thornley} \\{\fontspec{DejaVu Sans}▪ }\textsf{\textit{verb}}\\ \textbf{1} Catch or arrest (an offender) {\fontspec{DejaVu Sans}◇} \textit{he was copped for speeding} \textbf{2} Receive or attain (something welcome) {\fontspec{DejaVu Sans}◇} \textit{she copped an award for her role in the film} \textbf{3} Strike (an attitude or pose) {\fontspec{DejaVu Sans}◇} \textit{I copped an attitude—I acted real tough}}{}{}{ \colorBullet{ORIGIN} Early 18th century (as a verb): perhaps from obsolete cap ‘arrest’, from Old French caper ‘seize’, from Latin capere. The noun is from copper.}%
\par%
\entry{cop}{/kɒp/}{পুলিশ}{ \textsf{\textit{noun}}\ \textbf{1} A conical mass of thread wound on to a spindle. {\fontspec{DejaVu Sans}◇} \textit{}}{}{}{ \colorBullet{ORIGIN} Late 18th century possibly from Old English cop ‘summit, top’.}%
\par%
\entry{COP}{}{পুলিশ}{ \textsf{\textit{abbreviation}}\ \textbf{1} Colombian peso(s). {\fontspec{DejaVu Sans}◇} \textit{}}{}{}{}%
\par%
\entry{copper}{/ˈkɒpə/}{তামা}{\small{\textsf{\textit{noun, verb}}} \\{\fontspec{DejaVu Sans}▪ }\textsf{\textit{noun}}\\ \textbf{1}  {\fontspec{DejaVu Sans}◇} \textit{} \textbf{2} Brown coins of low value made of copper or bronze. {\fontspec{DejaVu Sans}◇} \textit{} \textbf{3} A large copper or iron container for boiling laundry. {\fontspec{DejaVu Sans}◇} \textit{} \textbf{4} A reddish{-}brown colour like that of copper. {\fontspec{DejaVu Sans}◇} \textit{she had copper{-}coloured hair} \textbf{5} A small butterfly with bright reddish{-}brown wings. {\fontspec{DejaVu Sans}◇} \textit{} \\{\fontspec{DejaVu Sans}▪ }\textsf{\textit{verb}}\\ \textbf{1} Cover or coat (something) with copper. {\fontspec{DejaVu Sans}◇} \textit{some iron hulls were sheathed with wood and then coppered}}{}{}{ \colorBullet{ORIGIN} Old English copor, coper (related to Dutch koper and German Kupfer), based on late Latin cuprum, from Latin cyprium aes ‘Cyprus metal’ (so named because Cyprus was the chief source).}%
\par%
\entry{copper}{/ˈkɒpə/}{তামা}{ \textsf{\textit{noun}}\ \textbf{1} A police officer. {\fontspec{DejaVu Sans}◇} \textit{}}{}{}{ \colorBullet{ORIGIN} Mid 19th century from cop+ {-}er.}%
\par%
\entry{cordon}{/ˈkɔːd(ə)n/}{বেষ্টনী}{\small{\textsf{\textit{noun, verb}}} \\{\fontspec{DejaVu Sans}▪ }\textsf{\textit{noun}}\\ \textbf{1} A line or circle of police, soldiers, or guards preventing access to or from an area or building. {\fontspec{DejaVu Sans}◇} \textit{the crowd was halted in front of the police cordon} \colorBulletS{SYN} barrier, line, column, row, file, ranks, chain, ring, circle \textbf{2} A fruit tree trained to grow as a single stem. {\fontspec{DejaVu Sans}◇} \textit{} \textbf{3} A projecting course of brick or stone on the face of a wall. {\fontspec{DejaVu Sans}◇} \textit{} \\{\fontspec{DejaVu Sans}▪ }\textsf{\textit{verb}}\\ \textbf{1} Prevent access to or from an area or building by surrounding it with police or other guards. {\fontspec{DejaVu Sans}◇} \textit{the city centre was cordoned off after fires were discovered in two stores} \colorBulletS{SYN} close off, seal off, tape off, fence off, rope off, screen off, curtain off, shut off, partition off, separate off, isolate, segregate, quarantine}{}{}{ \colorBullet{ORIGIN} Late Middle English (denoting an ornamental braid): from Italian cordone, augmentative of corda, and French cordon, diminutive of corde, both from Latin chorda ‘string, rope’ (see cord). cordon (sense 3 of the noun), the earliest of the current noun senses, dates from the early 18th century.}%
\par%
\entry{corporal}{/ˈkɔːp(ə)r(ə)l/}{শারীরিক}{ \textsf{\textit{noun}}\ \textbf{1} A rank of non{-}commissioned officer in the army, above lance corporal or private first class and below sergeant. {\fontspec{DejaVu Sans}◇} \textit{} \textbf{2}  {\fontspec{DejaVu Sans}◇} \textit{} \textbf{3} another term for fallfish {\fontspec{DejaVu Sans}◇} \textit{}}{}{Corporal punishment: শারীরিক শাস্তি}{ \colorBullet{ORIGIN} Mid 16th century from French, obsolete variant of caporal, from Italian caporale, probably based on Latin corpus, corpor{-} ‘body (of troops)’, with a change of spelling in Italian due to association with capo ‘head’.}%
\par%
\entry{corporal}{/ˈkɔːp(ə)r(ə)l/}{শারীরিক}{ \textsf{\textit{adjective}}\ \textbf{1} Relating to the human body. {\fontspec{DejaVu Sans}◇} \textit{} \colorBulletS{SYN} bodily, fleshly, corporeal, carnal, mortal, earthly, worldly, physical, material, real, actual, tangible, substantial}{}{Corporal punishment: শারীরিক শাস্তি}{ \colorBullet{ORIGIN} Late Middle English via Old French from Latin corporalis, from corpus, corpor{-} ‘body’.}%
\par%
\entry{corporal}{/ˈkɔːp(ə)r(ə)l/}{শারীরিক}{ \textsf{\textit{noun}}\ \textbf{1} A cloth on which the chalice and paten are placed during the celebration of the Eucharist. {\fontspec{DejaVu Sans}◇} \textit{}}{}{Corporal punishment: শারীরিক শাস্তি}{ \colorBullet{ORIGIN} Middle English from medieval Latin corporale (pallium) ‘body (cloth)’, from Latin corpus, corpor{-} ‘body’.}%
\par%
\entry{corpse}{/kɔːps/}{মড়া}{\small{\textsf{\textit{noun, verb}}} \\{\fontspec{DejaVu Sans}▪ }\textsf{\textit{noun}}\\ \textbf{1} A dead body, especially of a human being rather than an animal. {\fontspec{DejaVu Sans}◇} \textit{the corpse of a man lay there} \colorBulletS{SYN} dead body, body, cadaver, carcass, skeleton \\{\fontspec{DejaVu Sans}▪ }\textsf{\textit{verb}}\\ \textbf{1} Spoil a piece of acting by forgetting one's lines or laughing uncontrollably. {\fontspec{DejaVu Sans}◇} \textit{Peter just can't stop himself corpsing when he is on stage}}{}{}{ \colorBullet{ORIGIN} Middle English (denoting the living body of a person or animal): alteration of corse by association with Latin corpus, a change which also took place in French (Old French cors becoming corps). The p was originally silent, as in French; the final e was rare before the 19th century, but now distinguishes corpse from corps.}%
\par%
\entry{correspondent}{/kɒrɪˈspɒnd(ə)nt/}{সংবাদদাতা}{\small{\textsf{\textit{adjective, noun}}} \\{\fontspec{DejaVu Sans}▪ }\textsf{\textit{adjective}}\\ \textbf{1} Corresponding. {\fontspec{DejaVu Sans}◇} \textit{However, correspondent payment can involve payment between two banks in the same jurisdiction, if payment is to be in foreign currency.} \colorBulletS{SYN} corresponding, equivalent, comparable, parallel, matching, related, similar, analogous, commensurate \\{\fontspec{DejaVu Sans}▪ }\textsf{\textit{noun}}\\ \textbf{1} A person who writes letters on a regular basis. {\fontspec{DejaVu Sans}◇} \textit{she wasn't much of a correspondent} \colorBulletS{SYN} letter writer, penfriend, pen pal \textbf{2} A person employed to report for a newspaper or broadcasting organization. {\fontspec{DejaVu Sans}◇} \textit{a cricket correspondent} \colorBulletS{SYN} reporter, journalist, columnist, writer, contributor, newspaperman, newspaperwoman, newsman, newswoman, commentator, chronicler}{}{}{ \colorBullet{ORIGIN} Late Middle English (as an adjective): from Old French correspondant or medieval Latin correspondent{-} ‘corresponding’, from the verb correspondere (see correspond).}%
\par%
\entry{counsel}{/ˈkaʊns(ə)l/}{পরামর্শ}{\small{\textsf{\textit{noun, verb}}} \\{\fontspec{DejaVu Sans}▪ }\textsf{\textit{noun}}\\ \textbf{1} Advice, especially that given formally. {\fontspec{DejaVu Sans}◇} \textit{with wise counsel a couple can buy a home that will be appreciating in value} \colorBulletS{SYN} advice, guidance, direction, instruction, information, enlightenment \textbf{2} A barrister or other legal adviser conducting a case. {\fontspec{DejaVu Sans}◇} \textit{the counsel for the defence} \colorBulletS{SYN} barrister, lawyer, counsellor, legal practitioner \\{\fontspec{DejaVu Sans}▪ }\textsf{\textit{verb}}\\ \textbf{1} Give advice to (someone) {\fontspec{DejaVu Sans}◇} \textit{careers officers should counsel young people in making their career decisions}}{}{}{ \colorBullet{ORIGIN} Middle English via Old French counseil (noun), conseiller (verb), from Latin consilium ‘consultation, advice’, related to consulere (see consult). Compare with council.}%
\par%
\entry{counterpart}{/ˈkaʊntəpɑːt/}{প্রতিরুপ}{ \textsf{\textit{noun}}\ \textbf{1} A person or thing that corresponds to or has the same function as another person or thing in a different place or situation. {\fontspec{DejaVu Sans}◇} \textit{the minister held talks with his French counterpart} \colorBulletS{SYN} equivalent, opposite number, peer, equal, parallel, complement, match, twin, mate, fellow, brother, sister, analogue, correlative \textbf{2} One of two copies of a legal document. {\fontspec{DejaVu Sans}◇} \textit{} \colorBulletS{SYN} copy, carbon copy, carbon, photocopy, facsimile, mimeo, mimeograph, reprint}{}{}{}%
\par%
\entry{coup}{/kuː/}{ঘা}{ \textsf{\textit{noun}}\ \textbf{1}  {\fontspec{DejaVu Sans}◇} \textit{he was overthrown in an army coup} \colorBulletS{SYN} seizure of power, overthrow, takeover, ousting, deposition, regime change \textbf{2} An instance of successfully achieving something difficult. {\fontspec{DejaVu Sans}◇} \textit{it was a major coup to get such a prestigious contract} \colorBulletS{SYN} success, triumph, feat, successful manoeuvre, stunt, accomplishment, achievement, attainment, stroke, master stroke, stroke of genius \textbf{3} A direct pocketing of the cue ball, which is a foul stroke. {\fontspec{DejaVu Sans}◇} \textit{} \textbf{4} (among some North American Indian peoples) an act of touching an armed enemy in battle as a deed of bravery, or an act of first touching an item of the enemy's in order to claim it. {\fontspec{DejaVu Sans}◇} \textit{}}{}{}{ \colorBullet{ORIGIN} Late 18th century from French, from medieval Latin colpus ‘blow’ (see cope).}%
\par%
\entry{courage}{/ˈkʌrɪdʒ/}{সাহস}{ \textsf{\textit{noun}}\ \textbf{1} The ability to do something that frightens one; bravery. {\fontspec{DejaVu Sans}◇} \textit{she called on all her courage to face the ordeal}}{}{}{ \colorBullet{ORIGIN} Middle English (denoting the heart, as the seat of feelings): from Old French corage, from Latin cor ‘heart’.}%
\par%
\entry{court}{/kɔːt/}{আদালত; অনুগ্রহ প্রার্থনা করা}{\small{\textsf{\textit{noun, verb}}} \\{\fontspec{DejaVu Sans}▪ }\textsf{\textit{noun}}\\ \textbf{1}  {\fontspec{DejaVu Sans}◇} \textit{she will take the matter to court} \colorBulletS{SYN} court of law, law court, bench, bar, court of justice, judicature, tribunal, forum, chancery, assizes \textbf{2} A quadrangular area, either open or covered, marked out for ball games such as tennis or squash. {\fontspec{DejaVu Sans}◇} \textit{a squash court} \colorBulletS{SYN} playing area, enclosure, field, ground, ring, rink, green, alley, stadium, track, arena \textbf{3} The courtiers, retinue, and household of a sovereign. {\fontspec{DejaVu Sans}◇} \textit{the emperor is shown with his court} \colorBulletS{SYN} royal household, establishment, retinue, entourage, train, suite, escort, company, attendant company, staff, personnel, cortège, following, bodyguard \textbf{4} The qualified members of a company or a corporation. {\fontspec{DejaVu Sans}◇} \textit{The decision on the succession rests with the nomination committee of the court of directors.} \\{\fontspec{DejaVu Sans}▪ }\textsf{\textit{verb}}\\ \textbf{1} Be involved with (someone) romantically, with the intention of marrying. {\fontspec{DejaVu Sans}◇} \textit{he was courting a girl from the neighbouring farm} \colorBulletS{SYN} woo, go out with, be involved with, be romantically linked with, pursue, run after, chase, seek the company of, make advances to, make up to, flirt with \textbf{2} Pay special attention to (someone) in an attempt to win their support or favour. {\fontspec{DejaVu Sans}◇} \textit{Western politicians courted the leaders of the newly independent states} \colorBulletS{SYN} curry favour with, make up to, play up to}{}{}{ \colorBullet{ORIGIN} Middle English from Old French cort, from Latin cohors, cohort{-} ‘yard or retinue’. The verb is influenced by Old Italian corteare, Old French courtoyer. Compare with cohort.}%
\par%
\entry{courtesy}{/ˈkəːtɪsi/}{শ্লীলতা}{ \textsf{\textit{noun}}\ \textbf{1} The showing of politeness in one's attitude and behaviour towards others. {\fontspec{DejaVu Sans}◇} \textit{he treated the players with courtesy and good humour} \colorBulletS{SYN} politeness, courteousness, good manners, civility, respect, respectfulness, deference, chivalry, gallantry, good breeding, gentility, graciousness, kindness, consideration, thought, thoughtfulness, cordiality, geniality, affability, urbanity, polish, refinement, courtliness, decorousness, tact, discretion, diplomacy \textbf{2} A curtsy. {\fontspec{DejaVu Sans}◇} \textit{}}{}{}{ \colorBullet{ORIGIN} Middle English from Old French cortesie, from corteis (see courteous).}%
\par%
\entry{cradle}{/ˈkreɪd(ə)l/}{শৈশবাবস্থা}{\small{\textsf{\textit{noun, verb}}} \\{\fontspec{DejaVu Sans}▪ }\textsf{\textit{noun}}\\ \textbf{1} A baby's bed or cot, typically one mounted on rockers. {\fontspec{DejaVu Sans}◇} \textit{the baby slept peacefully in its cradle} \colorBulletS{SYN} crib, bassinet, Moses basket, cot, carrycot \textbf{2} A framework on which a ship or boat rests during construction or repairs. {\fontspec{DejaVu Sans}◇} \textit{} \colorBulletS{SYN} framework, rack, holder, stand, base, support, mounting, mount, platform, prop, horse, rest, chock, plinth, bottom, trivet, bracket, frame, subframe, structure, substructure, chassis \\{\fontspec{DejaVu Sans}▪ }\textsf{\textit{verb}}\\ \textbf{1} Hold gently and protectively. {\fontspec{DejaVu Sans}◇} \textit{she cradled his head in her arms} \colorBulletS{SYN} hold, support, prop up, rest, pillow, bolster, cushion, shelter, protect \textbf{2} Place (a telephone receiver) in its cradle. {\fontspec{DejaVu Sans}◇} \textit{she cradled the receiver gently}}{}{}{ \colorBullet{ORIGIN} Old English cradol, of uncertain origin; perhaps related to German Kratte ‘basket’.}%
\par%
\entry{craft}{/krɑːft/}{নৈপুণ্য}{\small{\textsf{\textit{noun, verb}}} \\{\fontspec{DejaVu Sans}▪ }\textsf{\textit{noun}}\\ \textbf{1} An activity involving skill in making things by hand. {\fontspec{DejaVu Sans}◇} \textit{the craft of cobbling} \colorBulletS{SYN} activity, pursuit, occupation, work, line, line of work, profession, job, business, line of business, trade, employment, position, post, situation, career, métier, vocation, calling, skill, field, province, walk of life \textbf{2} Skill used in deceiving others. {\fontspec{DejaVu Sans}◇} \textit{her cousin was not her equal in guile and evasive craft} \colorBulletS{SYN} cunning, craftiness, guile, wiliness, artfulness, deviousness, slyness, trickery, trickiness \textbf{3} A boat or ship. {\fontspec{DejaVu Sans}◇} \textit{sailing craft} \colorBulletS{SYN} boat, sailing boat, ship, yacht, craft, watercraft \\{\fontspec{DejaVu Sans}▪ }\textsf{\textit{verb}}\\ \textbf{1} Exercise skill in making (an object), typically by hand. {\fontspec{DejaVu Sans}◇} \textit{he crafted the chair lovingly}}{}{}{ \colorBullet{ORIGIN} Old English cræft ‘strength, skill’, of Germanic origin; related to Dutch kracht, German Kraft, and Swedish kraft ‘strength’. craft (sense 3 of the noun), originally in the expression small craft ‘small trading vessels’, may be elliptical, referring to vessels requiring a small amount of ‘craft’ or skill to handle, as opposed to large ocean{-}going ships.}%
\par%
\entry{cram}{/kram/}{ঠাসা}{ \textsf{\textit{verb}}\ \textbf{1} Completely fill (a place or container) to the point of overflowing. {\fontspec{DejaVu Sans}◇} \textit{the ashtray by the bed was crammed with cigarette butts} \colorBulletS{SYN} stuff, pack, jam, fill, crowd, throng \textbf{2} Study intensively over a short period of time just before an examination. {\fontspec{DejaVu Sans}◇} \textit{lectures were called off so students could cram for the semester finals} \colorBulletS{SYN} study intensively, revise}{}{}{ \colorBullet{ORIGIN} Old English crammian, of Germanic origin; related to Dutch krammen ‘to cramp or clamp’.}%
\par%
\entry{crap}{/krap/}{বিষ্ঠা}{\small{\textsf{\textit{adjective, noun, verb}}} \\{\fontspec{DejaVu Sans}▪ }\textsf{\textit{adjective}}\\ \textbf{1} Extremely poor in quality. {\fontspec{DejaVu Sans}◇} \textit{} \colorBulletS{SYN} substandard, poor, inferior, second{-}rate, second{-}class, unsatisfactory, inadequate, unacceptable, not up to scratch, not up to par, deficient, imperfect, defective, faulty, shoddy, amateurish, careless, negligent \\{\fontspec{DejaVu Sans}▪ }\textsf{\textit{noun}}\\ \textbf{1} Something of extremely poor quality. {\fontspec{DejaVu Sans}◇} \textit{} \textbf{2} Excrement. {\fontspec{DejaVu Sans}◇} \textit{} \\{\fontspec{DejaVu Sans}▪ }\textsf{\textit{verb}}\\ \textbf{1} Defecate. {\fontspec{DejaVu Sans}◇} \textit{} \textbf{2} Talk at length in a foolish or boring way. {\fontspec{DejaVu Sans}◇} \textit{}}{}{}{ \colorBullet{ORIGIN} Late Middle English related to Dutch krappe, from krappen ‘pluck or cut off’, and perhaps also to Old French crappe ‘siftings’, Anglo{-}Latin crappa ‘chaff’. The original sense was ‘chaff’, later ‘residue from rendering fat’, also ‘dregs of beer’. Current senses date from the late 19th century.}%
\par%
\entry{crap}{/krap/}{বিষ্ঠা}{\small{\textsf{\textit{noun, verb}}} \\{\fontspec{DejaVu Sans}▪ }\textsf{\textit{noun}}\\ \textbf{1} A losing throw of 2, 3, or 12 in craps. {\fontspec{DejaVu Sans}◇} \textit{} \\{\fontspec{DejaVu Sans}▪ }\textsf{\textit{verb}}\\ \textbf{1} Make a losing throw at craps. {\fontspec{DejaVu Sans}◇} \textit{he put all his chips on the table and rolled the dice—sooner or later he had to crap out}}{}{}{ \colorBullet{ORIGIN} Early 20th century from craps.}%
\par%
\entry{crawl}{/krɔːl/}{হামাগুড়ি}{\small{\textsf{\textit{noun, verb}}} \\{\fontspec{DejaVu Sans}▪ }\textsf{\textit{noun}}\\ \textbf{1} An act of moving on one's hands and knees or dragging one's body along the ground. {\fontspec{DejaVu Sans}◇} \textit{they began the crawl back to their own lines} \textbf{2} A swimming stroke involving alternate overarm movements and rapid kicks of the legs. {\fontspec{DejaVu Sans}◇} \textit{she could do the crawl and so many other strokes} \\{\fontspec{DejaVu Sans}▪ }\textsf{\textit{verb}}\\ \textbf{1} Move forward on the hands and knees or by dragging the body close to the ground. {\fontspec{DejaVu Sans}◇} \textit{they crawled from under the table} \colorBulletS{SYN} creep, go on all fours, move on hands and knees, inch, drag oneself along, pull oneself along, drag, trail, slither, slink, squirm, wriggle, writhe, scrabble, worm one's way, advance slowly, advance stealthily, sneak \textbf{2} Behave obsequiously or ingratiatingly in the hope of gaining someone's favour. {\fontspec{DejaVu Sans}◇} \textit{a reporter's job can involve crawling to objectionable people} \colorBulletS{SYN} grovel to, be obsequious towards, ingratiate oneself with, be servile towards, be sycophantic towards, kowtow to, pander to, abase oneself to, demean oneself to, bow and scrape to, prostrate oneself before, toady to, truckle to, dance attendance on, fawn on, fawn over, curry favour with, cultivate, seek the favour of, try to win over, try to get on the good side of, make up to, play up to \textbf{3} Be covered or crowded with (insects or people), to an extent that is objectionable. {\fontspec{DejaVu Sans}◇} \textit{the floor was dirty and crawling with bugs} \colorBulletS{SYN} be full of, overflow with, teem with, abound in, abound with, be packed with, be crowded with, be thronged with, be jammed with, be alive with, be overrun with, swarm with, be bristling with, be infested with, be thick with \textbf{4} (of a program) systematically visit (a number of web pages) in order to create an index of data. {\fontspec{DejaVu Sans}◇} \textit{its automated software robots crawl websites, grabbing copies of pages to index}}{}{}{ \colorBullet{ORIGIN} Middle English of unknown origin; possibly related to Swedish kravla and Danish kravle.}%
\par%
\entry{credible}{/ˈkrɛdɪb(ə)l/}{বিশ্বাসযোগ্য}{ \textsf{\textit{adjective}}\ \textbf{1} Able to be believed; convincing. {\fontspec{DejaVu Sans}◇} \textit{few people found his story credible} \colorBulletS{SYN} acceptable, trustworthy, reliable, dependable, sure, good, valid}{}{}{ \colorBullet{ORIGIN} Late Middle English from Latin credibilis, from credere ‘believe’.}%
\par%
\entry{creep}{/kriːp/}{হামাগুড়ি}{\small{\textsf{\textit{noun, verb}}} \\{\fontspec{DejaVu Sans}▪ }\textsf{\textit{noun}}\\ \textbf{1} A detestable person. {\fontspec{DejaVu Sans}◇} \textit{I thought he was a nasty little creep} \colorBulletS{SYN} rogue, villain, wretch, reprobate \textbf{2} Slow steady movement, especially when imperceptible. {\fontspec{DejaVu Sans}◇} \textit{an attempt to prevent this slow creep of costs} \textbf{3} An opening in a hedge or wall for an animal to pass through. {\fontspec{DejaVu Sans}◇} \textit{low in the wall are creeps, through which ewes gain access to grazing from the pastures behind} \textbf{4} Solid food given to young farm animals in order to wean them. {\fontspec{DejaVu Sans}◇} \textit{we've started to wean the lambs earlier and to keep them on creep} \\{\fontspec{DejaVu Sans}▪ }\textsf{\textit{verb}}\\ \textbf{1} Move slowly and carefully in order to avoid being heard or noticed. {\fontspec{DejaVu Sans}◇} \textit{he crept downstairs, hardly making any noise} \colorBulletS{SYN} crawl, move on all fours, move on hands and knees, pull oneself, inch, edge, slither, slide, squirm, wriggle, writhe, worm, worm one's way, insinuate oneself \textbf{2} (of a negative characteristic or fact) occur or develop gradually and almost imperceptibly. {\fontspec{DejaVu Sans}◇} \textit{errors crept into his game} \colorBulletS{SYN} penetrate, invade, intrude on, insinuate oneself into, worm one's way into, sneak into, slip into, creep into, impinge on, trespass on, butt into}{}{}{ \colorBullet{ORIGIN} Old English crēopan ‘move with the body close to the ground’, of Germanic origin; related to Dutch kruipen. Sense 1 of the verb dates from Middle English.}%
\par%
\entry{creepy}{/ˈkriːpi/}{ছম্ছমে}{ \textsf{\textit{adjective}}\ \textbf{1} Causing an unpleasant feeling of fear or unease. {\fontspec{DejaVu Sans}◇} \textit{the creepy feelings one often gets in a strange house} \colorBulletS{SYN} frightening, scaring, terrifying, hair{-}raising, spine{-}chilling, blood{-}curdling, chilling, petrifying, alarming, shocking, harrowing, horrifying, horrific, horrible, awful, nightmarish, macabre, ghostly}{}{}{}%
\par%
\entry{crimson}{/ˈkrɪmz(ə)n/}{আরক্ত}{\small{\textsf{\textit{adjective, noun, verb}}} \\{\fontspec{DejaVu Sans}▪ }\textsf{\textit{adjective}}\\ \textbf{1} Of a rich deep red colour inclining to purple. {\fontspec{DejaVu Sans}◇} \textit{she blushed crimson with embarrassment} \colorBulletS{SYN} red, reddish, scarlet, vermilion, crimson, blood red, rose red, pink, roseate \\{\fontspec{DejaVu Sans}▪ }\textsf{\textit{noun}}\\ \textbf{1} A rich deep red colour inclining to purple. {\fontspec{DejaVu Sans}◇} \textit{a pair of corduroy trousers in livid crimson, they were horrid to behold} \colorBulletS{SYN} flush, blush, rosiness, pinkness, redness, crimson, scarlet, reddening, ruddiness, high colour \\{\fontspec{DejaVu Sans}▪ }\textsf{\textit{verb}}\\ \textbf{1} (of a person's face) become flushed, especially through embarrassment. {\fontspec{DejaVu Sans}◇} \textit{my face crimsoned and my hands began to shake} \colorBulletS{SYN} flush, blush, redden, go red, colour, colour up, go pink, crimson, go scarlet, be suffused with colour}{}{}{ \colorBullet{ORIGIN} Late Middle English from obsolete French cramoisin or Old Spanish cremesin, based on Arabic qirmizī, from qirmiz (see kermes). Compare with carmine.}%
\par%
\entry{cruel}{/krʊəl/}{নিষ্ঠুর}{\small{\textsf{\textit{adjective, verb}}} \\{\fontspec{DejaVu Sans}▪ }\textsf{\textit{adjective}}\\ \textbf{1} Wilfully causing pain or suffering to others, or feeling no concern about it. {\fontspec{DejaVu Sans}◇} \textit{people who are cruel to animals} \colorBulletS{SYN} brutal, savage, inhuman, barbaric, barbarous, brutish, bloodthirsty, murderous, homicidal, cut{-}throat, vicious, ferocious, fierce \\{\fontspec{DejaVu Sans}▪ }\textsf{\textit{verb}}\\ \textbf{1} Spoil or ruin (an opportunity or a chance of success) {\fontspec{DejaVu Sans}◇} \textit{Ernie nearly cruelled the whole thing by laughing} \colorBulletS{SYN} wreck, ruin, spoil, disrupt, undo, upset, play havoc with, make a mess of, put an end to, end, bring to an end, put a stop to, terminate, prevent, frustrate, blight, crush, quell, quash, dash, scotch, shatter, vitiate, blast, devastate, demolish, sabotage, torpedo}{}{}{ \colorBullet{ORIGIN} Middle English via Old French from Latin crudelis, related to crudus (see crude).}%
\par%
\entry{cruise}{/kruːz/}{সমুদ্রভ্রমণ}{\small{\textsf{\textit{noun, verb}}} \\{\fontspec{DejaVu Sans}▪ }\textsf{\textit{noun}}\\ \textbf{1} A voyage on a ship or boat taken for pleasure or as a holiday and usually calling in at several places. {\fontspec{DejaVu Sans}◇} \textit{a cruise down the Nile} \colorBulletS{SYN} boat trip \\{\fontspec{DejaVu Sans}▪ }\textsf{\textit{verb}}\\ \textbf{1} Sail about in an area without a precise destination, especially for pleasure. {\fontspec{DejaVu Sans}◇} \textit{they were cruising off the California coast} \colorBulletS{SYN} sail, steam, voyage, journey \textbf{2} (of a motor vehicle or aircraft) travel smoothly at a moderate or economical speed. {\fontspec{DejaVu Sans}◇} \textit{we sit in a jet, cruising at some 30,000 ft} \colorBulletS{SYN} coast, drift, meander, drive slowly, travel slowly, travel aimlessly \textbf{3} (of a young child) walk while holding on to furniture or other structures, prior to learning to walk without support. {\fontspec{DejaVu Sans}◇} \textit{my daughter cruised at seven months and didn't walk until just after her first birthday}}{}{}{ \colorBullet{ORIGIN} Mid 17th century (as a verb): probably from Dutch kruisen ‘to cross’, from kruis ‘cross’, from Latin crux.}%
\par%
\entry{crunch}{/krʌn(t)ʃ/}{কড়্কড়্ শব্দ}{\small{\textsf{\textit{noun, verb}}} \\{\fontspec{DejaVu Sans}▪ }\textsf{\textit{noun}}\\ \textbf{1} A loud muffled grinding sound like that of something hard or brittle being crushed. {\fontspec{DejaVu Sans}◇} \textit{Marco's fist struck Brian's nose with a crunch} \textbf{2} A crucial point or situation, typically one at which a decision with important consequences must be made. {\fontspec{DejaVu Sans}◇} \textit{when it comes to the crunch you chicken out} \colorBulletS{SYN} moment of truth, critical point, crux, crisis, decision time, zero hour, point of no return \textbf{3} A physical exercise designed to strengthen the abdominal muscles; a sit{-}up. {\fontspec{DejaVu Sans}◇} \textit{} \\{\fontspec{DejaVu Sans}▪ }\textsf{\textit{verb}}\\ \textbf{1} Crush (a hard or brittle foodstuff) with the teeth, making a loud but muffled grinding sound. {\fontspec{DejaVu Sans}◇} \textit{she paused to crunch a ginger biscuit} \colorBulletS{SYN} munch, chew noisily, chomp, champ, bite, gnaw, masticate \textbf{2} (especially of a computer) process (large quantities of information) {\fontspec{DejaVu Sans}◇} \textit{the program crunches data from 14,000 sensors to decipher evolving patterns}}{}{}{ \colorBullet{ORIGIN} Early 19th century (as a verb): variant of 17th{-}century cranch (probably imitative), by association with crush and munch.}%
\par%
\entry{cucumber}{/ˈkjuːkʌmbə/}{শসা}{ \textsf{\textit{noun}}\ \textbf{1} A long, green{-}skinned fruit with watery flesh, usually eaten raw in salads or pickled. {\fontspec{DejaVu Sans}◇} \textit{} \textbf{2} The climbing plant of the gourd family that yields cucumbers, native to the Chinese Himalayan region. It is widely cultivated but very rare in the wild. {\fontspec{DejaVu Sans}◇} \textit{}}{}{}{ \colorBullet{ORIGIN} Late Middle English from Old French cocombre, coucombre, from Latin cucumis, cucumer{-}.}%
\par%
\entry{cue}{/kjuː/}{সূত্র}{\small{\textsf{\textit{noun, verb}}} \\{\fontspec{DejaVu Sans}▪ }\textsf{\textit{noun}}\\ \textbf{1} A thing said or done that serves as a signal to an actor or other performer to enter or to begin their speech or performance. {\fontspec{DejaVu Sans}◇} \textit{she had not yet been given her cue to come out on to the dais} \colorBulletS{SYN} signal, sign, indication, prompt, reminder, prompting \textbf{2} A facility for playing through an audio or video recording very rapidly until a desired starting point is reached. {\fontspec{DejaVu Sans}◇} \textit{} \\{\fontspec{DejaVu Sans}▪ }\textsf{\textit{verb}}\\ \textbf{1} Give a cue to or for. {\fontspec{DejaVu Sans}◇} \textit{Ros and Guil, cued by Hamlet, also bow deeply} \textbf{2} Set a piece of audio or video equipment in readiness to play (a particular part of the recorded material) {\fontspec{DejaVu Sans}◇} \textit{there was a pause while she cued up the next tape}}{}{}{ \colorBullet{ORIGIN} Mid 16th century of unknown origin.}%
\par%
\entry{cue}{/kjuː/}{সূত্র}{\small{\textsf{\textit{noun, verb}}} \\{\fontspec{DejaVu Sans}▪ }\textsf{\textit{noun}}\\ \textbf{1} A long straight tapering wooden rod for striking the ball in snooker, billiards, etc. {\fontspec{DejaVu Sans}◇} \textit{} \\{\fontspec{DejaVu Sans}▪ }\textsf{\textit{verb}}\\ \textbf{1} Use a cue to strike the ball. {\fontspec{DejaVu Sans}◇} \textit{Mark cued well early on}}{}{}{ \colorBullet{ORIGIN} Mid 18th century (denoting a long plait or pigtail): variant of queue.}%
\par%
\entry{cuisine}{/kwɪˈziːn/}{রন্ধনপ্রণালী}{ \textsf{\textit{noun}}\ \textbf{1} A style or method of cooking, especially as characteristic of a particular country, region, or establishment. {\fontspec{DejaVu Sans}◇} \textit{much Venetian cuisine is based on seafood} \colorBulletS{SYN} cooking, cookery, fare, food}{}{}{ \colorBullet{ORIGIN} Late 18th century French, literally ‘kitchen’, from Latin coquina, from coquere ‘to cook’.}%
\par%
\entry{culpability}{/ˌkʌlpəˈbɪlɪti/}{নিন্দনীয়তা}{ \textsf{\textit{noun}}\ \textbf{1} Responsibility for a fault or wrong; blame. {\fontspec{DejaVu Sans}◇} \textit{a level of moral culpability} \colorBulletS{SYN} guilt, blame, fault, responsibility, accountability, liability, answerability}{}{}{}%
\par%
\entry{culprit}{/ˈkʌlprɪt/}{অভিযুক্ত ব্যক্তি}{ \textsf{\textit{noun}}\ \textbf{1} A person who is responsible for a crime or other misdeed. {\fontspec{DejaVu Sans}◇} \textit{the car's front nearside door had been smashed in but the culprits had fled} \colorBulletS{SYN} guilty party, offender, wrongdoer, person responsible}{}{}{ \colorBullet{ORIGIN} Late 17th century (originally in the formula Culprit, how will you be tried?, said by the Clerk of the Crown to a prisoner pleading not guilty): perhaps from a misinterpretation of the written abbreviation cul. prist for Anglo{-}Norman French Culpable: prest d'averrer notre bille ‘(You are) guilty: (We are) ready to prove our indictment’; in later use influenced by Latin culpa ‘fault, blame’.}%
\par%
\entry{cumin}{/ˈkʌmɪn/}{জিরা}{ \textsf{\textit{noun}}\ \textbf{1} The aromatic seeds of a plant of the parsley family, used as a spice, especially ground and used in curry powder. {\fontspec{DejaVu Sans}◇} \textit{add a pinch of cumin} \textbf{2} The small, slender plant which bears cumin seeds, occurring from the Mediterranean to central Asia. {\fontspec{DejaVu Sans}◇} \textit{Yarrow, alyssum, fennel, cumin, \& coriander all attract beneficial insects as well.}}{}{}{ \colorBullet{ORIGIN} Old English cymen, from Latin cuminum, from Greek kuminon, probably of Semitic origin and related to Hebrew kammōn and Arabic kammūn; superseded in Middle English by forms from Old French cumon, comin, also from Latin.}%
\par%
\entry{curb}{/kəːb/}{প্রতিবন্ধক}{\small{\textsf{\textit{noun, verb}}} \\{\fontspec{DejaVu Sans}▪ }\textsf{\textit{noun}}\\ \textbf{1} A check or restraint on something. {\fontspec{DejaVu Sans}◇} \textit{plans to introduce tougher curbs on insider dealing} \colorBulletS{SYN} restraint, restriction, check, brake, rein, control, limitation, limit, constraint, stricture \textbf{2}  {\fontspec{DejaVu Sans}◇} \textit{} \textbf{3} variant spelling of kerb {\fontspec{DejaVu Sans}◇} \textit{} \textbf{4} A swelling on the back of a horse's hock, caused by spraining a ligament. {\fontspec{DejaVu Sans}◇} \textit{} \\{\fontspec{DejaVu Sans}▪ }\textsf{\textit{verb}}\\ \textbf{1} Restrain or keep in check. {\fontspec{DejaVu Sans}◇} \textit{she promised she would curb her temper} \colorBulletS{SYN} restrain, hold back, keep back, hold in, repress, suppress, fight back, bite back, keep in check, check, control, keep under control, rein in, keep a tight rein on, contain, discipline, govern, bridle, tame, subdue, stifle, smother, swallow, choke back, muzzle, silence, muffle, strangle, gag \textbf{2} Lead (a dog being walked) near the curb to urinate or defecate, in order to avoid soiling buildings, pavements, etc. {\fontspec{DejaVu Sans}◇} \textit{}}{}{}{ \colorBullet{ORIGIN} Late 15th century (denoting a strap fastened to the bit): from Old French courber ‘bend, bow’, from Latin curvare (see curve).}%
\par%
\entry{curse}{/kəːs/}{অভিশাপ}{\small{\textsf{\textit{noun, verb}}} \\{\fontspec{DejaVu Sans}▪ }\textsf{\textit{noun}}\\ \textbf{1} A solemn utterance intended to invoke a supernatural power to inflict harm or punishment on someone or something. {\fontspec{DejaVu Sans}◇} \textit{she'd put a curse on him} \colorBulletS{SYN} malediction, the evil eye, imprecation, execration, voodoo, hoodoo \textbf{2} An offensive word or phrase used to express anger or annoyance. {\fontspec{DejaVu Sans}◇} \textit{at every blow there was a curse} \colorBulletS{SYN} swear word, expletive, oath, profanity, four{-}letter word, dirty word, obscenity, imprecation, blasphemy, vulgarism, vulgarity \\{\fontspec{DejaVu Sans}▪ }\textsf{\textit{verb}}\\ \textbf{1} Invoke or use a curse against. {\fontspec{DejaVu Sans}◇} \textit{it often seemed as if the family had been cursed} \colorBulletS{SYN} put a curse on, put the evil eye on, execrate, imprecate, hoodoo \textbf{2} Utter offensive words in anger or annoyance. {\fontspec{DejaVu Sans}◇} \textit{he cursed loudly as he burned his hand} \colorBulletS{SYN} swear, utter profanities, utter oaths, use bad language, use foul language, be foul{-}mouthed, blaspheme, be blasphemous, take the Lord's name in vain, swear like a trooper, damn}{}{}{ \colorBullet{ORIGIN} Old English, of unknown origin.}%
\par%
\entry{cursory}{/ˈkəːs(ə)ri/}{দ্রুত}{ \textsf{\textit{adjective}}\ \textbf{1} Hasty and therefore not thorough or detailed. {\fontspec{DejaVu Sans}◇} \textit{a cursory glance at the figures} \colorBulletS{SYN} perfunctory, desultory, casual, superficial, token, uninterested, half{-}hearted, inattentive, unthinking, offhand, mechanical, automatic, routine}{}{}{ \colorBullet{ORIGIN} Early 17th century from Latin cursorius ‘of a runner’, from cursor (see cursor).}%
\par%
\entry{custody}{/ˈkʌstədi/}{হেফাজত}{ \textsf{\textit{noun}}\ \textbf{1} The protective care or guardianship of someone or something. {\fontspec{DejaVu Sans}◇} \textit{the property was placed in the custody of a trustee} \colorBulletS{SYN} care, guardianship, charge, keeping, safe keeping, wardship, ward, responsibility, protection, guidance, tutelage \textbf{2} Imprisonment. {\fontspec{DejaVu Sans}◇} \textit{my father was being taken into custody} \colorBulletS{SYN} imprisonment, detention, confinement, incarceration, internment, captivity}{}{}{ \colorBullet{ORIGIN} Late Middle English from Latin custodia, from custos ‘guardian’.}%
\par%
\entry{cutie pie}{}{Someone who is pretty and makes you laugh and pokes you every once in a while}{\small{\textsf{\textit{}}}}{}{The girl I like is my cutie pie.}{}%
\par%
\entry{cynical}{/ˈsɪnɪk(ə)l/}{কঠোর; মানববিদ্বেষী}{ \textsf{\textit{adjective}}\ \textbf{1} Believing that people are motivated purely by self{-}interest; distrustful of human sincerity or integrity. {\fontspec{DejaVu Sans}◇} \textit{he was brutally cynical and hardened to every sob story under the sun} \colorBulletS{SYN} bitter, resentful, cynical, soured, distorted, disenchanted, disillusioned, disappointed, pessimistic, sceptical, distrustful, suspicious, misanthropic \textbf{2} Concerned only with one's own interests and typically disregarding accepted standards in order to achieve them. {\fontspec{DejaVu Sans}◇} \textit{a cynical manipulation of public opinion}}{}{"I have to say from the experience of the last 10 to 12 days, the russian engagement in the minsk process is rather cynical," british foreign secretary philip hammond said in the estonian capital tallinn.}{}%
\par%
\end{multicols}%
\pagebreak%
\section*{D}%
\begin{multicols}{2}%
\entry{daunt}{/dɔːnt/}{ভীত করা}{ \textsf{\textit{verb}}\ \textbf{1} Make (someone) feel intimidated or apprehensive. {\fontspec{DejaVu Sans}◇} \textit{some people are daunted by technology} \colorBulletS{SYN} intimidate, abash, take aback, shake, ruffle, throw, demoralize, discourage}{}{}{ \colorBullet{ORIGIN} Middle English from Old French danter, from Latin domitare, frequentative of domare ‘to tame’.}%
\par%
\entry{daunting}{/ˈdɔːntɪŋ/}{কঠিন}{ \textsf{\textit{adjective}}\ \textbf{1} Seeming difficult to deal with in prospect; intimidating. {\fontspec{DejaVu Sans}◇} \textit{a daunting task} \colorBulletS{SYN} intimidating, formidable, disconcerting, unnerving, unsettling, dismaying}{}{}{}%
\par%
\entry{deadlock}{/ˈdɛdlɒk/}{অচল অবস্থা}{\small{\textsf{\textit{noun, verb}}} \\{\fontspec{DejaVu Sans}▪ }\textsf{\textit{noun}}\\ \textbf{1} A situation, typically one involving opposing parties, in which no progress can be made. {\fontspec{DejaVu Sans}◇} \textit{an attempt to break the deadlock} \colorBulletS{SYN} stalemate, impasse, checkmate, stand{-}off \textbf{2} A type of lock requiring a key to open and close it, as distinct from a spring lock. {\fontspec{DejaVu Sans}◇} \textit{} \colorBulletS{SYN} bolt, lock, latch, catch, fastening, fastener \\{\fontspec{DejaVu Sans}▪ }\textsf{\textit{verb}}\\ \textbf{1} Cause (a situation or opposing parties) to come to a point where no progress can be made because of fundamental disagreement. {\fontspec{DejaVu Sans}◇} \textit{the meeting is deadlocked} \colorBulletS{SYN} tie, draw, dead heat \textbf{2} Secure (a door) with a deadlock. {\fontspec{DejaVu Sans}◇} \textit{you can deadlock any exit door from the outside} \colorBulletS{SYN} bolt, lock, fasten, padlock, secure, latch, deadlock, block, barricade, obstruct}{}{}{}%
\par%
\entry{deaf}{/dɛf/}{বধির}{ \textsf{\textit{adjective}}\ \textbf{1} Lacking the power of hearing or having impaired hearing. {\fontspec{DejaVu Sans}◇} \textit{I'm a bit deaf so you'll have to speak up} \colorBulletS{SYN} hard of hearing, hearing{-}impaired, with impaired hearing, unhearing, stone deaf, deafened, profoundly deaf}{}{}{ \colorBullet{ORIGIN} Old English dēaf, of Germanic origin; related to Dutch doof and German taub, from an Indo{-}European root shared by Greek tuphlos ‘blind’.}%
\par%
\entry{debt}{/dɛt/}{ঋণ}{ \textsf{\textit{noun}}\ \textbf{1} A sum of money that is owed or due. {\fontspec{DejaVu Sans}◇} \textit{I paid off my debts} \colorBulletS{SYN} bill, account, tally, financial obligation, outstanding payment, amount due, money owing}{}{}{ \colorBullet{ORIGIN} Middle English dette from Old French, based on Latin debitum ‘something owed’, past participle of debere ‘owe’. The spelling change in French and English was by association with the Latin word.}%
\par%
\entry{deceased}{/dɪˈsiːst/}{মৃত}{\small{\textsf{\textit{adjective, noun}}} \\{\fontspec{DejaVu Sans}▪ }\textsf{\textit{adjective}}\\ \textbf{1} Recently dead. {\fontspec{DejaVu Sans}◇} \textit{the deceased man's family} \colorBulletS{SYN} dead, expired, departed, gone, no more, passed on, passed away \\{\fontspec{DejaVu Sans}▪ }\textsf{\textit{noun}}\\ \textbf{1} The recently dead person in question. {\fontspec{DejaVu Sans}◇} \textit{the judge inferred that the deceased was confused as to the extent of his assets}}{}{}{}%
\par%
\entry{deceive}{/dɪˈsiːv/}{ছলা}{ \textsf{\textit{verb}}\ \textbf{1} Deliberately cause (someone) to believe something that is not true, especially for personal gain. {\fontspec{DejaVu Sans}◇} \textit{I didn't intend to deceive people into thinking it was French champagne} \colorBulletS{SYN} swindle, defraud, cheat, trick, hoodwink, hoax, dupe, take in, mislead, delude, fool, outwit, misguide, lead on, inveigle, seduce, ensnare, entrap, beguile, double{-}cross, gull}{}{}{ \colorBullet{ORIGIN} Middle English from Old French deceivre, from Latin decipere ‘catch, ensnare, cheat’.}%
\par%
\entry{decent}{/ˈdiːs(ə)nt/}{শালীন}{ \textsf{\textit{adjective}}\ \textbf{1} Conforming with generally accepted standards of respectable or moral behaviour. {\fontspec{DejaVu Sans}◇} \textit{a decent clean{-}living individual} \colorBulletS{SYN} respectable, upright, upstanding, honourable, honest, on the level, decent, right{-}minded, law{-}abiding \textbf{2} Of an acceptable standard; satisfactory. {\fontspec{DejaVu Sans}◇} \textit{people need decent homes} \colorBulletS{SYN} satisfactory, reasonable, fair, acceptable, adequate, sufficient, sufficiently good, good enough, ample, up to scratch, up to the mark, up to standard, up to par, competent, not bad, all right, average, tolerable, passable, suitable}{}{}{ \colorBullet{ORIGIN} Mid 16th century (in the sense ‘suitable, appropriate’): from Latin decent{-} ‘being fitting’, from the verb decere.}%
\par%
\entry{decline}{/dɪˈklʌɪn/}{পতন}{\small{\textsf{\textit{noun, verb}}} \\{\fontspec{DejaVu Sans}▪ }\textsf{\textit{noun}}\\ \textbf{1} A gradual and continuous loss of strength, numbers, quality, or value. {\fontspec{DejaVu Sans}◇} \textit{a serious decline in bird numbers} \colorBulletS{SYN} reduction, decrease, downturn, downswing, lowering, devaluation, depreciation, lessening, diminishing, diminution, slackening, waning, dwindling, fading, ebb, falling off, abatement, drop, slump, plunge, tumble \\{\fontspec{DejaVu Sans}▪ }\textsf{\textit{verb}}\\ \textbf{1} (typically of something regarded as good) become smaller, fewer, or less; decrease. {\fontspec{DejaVu Sans}◇} \textit{the birth rate continued to decline} \colorBulletS{SYN} decrease, reduce, get smaller, grow smaller, lessen, get less, diminish, wane, dwindle, contract, shrink, fall off, taper off, tail off, peter out \textbf{2} Politely refuse (an invitation or offer) {\fontspec{DejaVu Sans}◇} \textit{Caroline declined the coffee} \colorBulletS{SYN} turn down, reject, brush aside, refuse, rebuff, spurn, disdain, look down one's nose at, repulse, repudiate, dismiss, forgo, deny oneself, pass up, refuse to take advantage of, turn one's back on \textbf{3} (especially of the sun) move downwards. {\fontspec{DejaVu Sans}◇} \textit{the sun began to creep round to the west and to decline} \colorBulletS{SYN} go down, sink, decline, descend, drop, subside \textbf{4} (in the grammar of Latin, Greek, and certain other languages) state the forms of (a noun, pronoun, or adjective) corresponding to case, number, and gender. {\fontspec{DejaVu Sans}◇} \textit{}}{}{}{ \colorBullet{ORIGIN} Late Middle English from Old French decliner, from Latin declinare ‘bend down, turn aside’, from de{-} ‘down’ + clinare ‘to bend’.}%
\par%
\entry{declining}{/dɪˈklʌɪnɪŋ/}{পড়ন্ত}{ \textsf{\textit{adjective}}\ \textbf{1} Becoming smaller, fewer, or less; decreasing. {\fontspec{DejaVu Sans}◇} \textit{declining budgets}}{}{}{}%
\par%
\entry{deem}{/diːm/}{বিবেচনা করা}{ \textsf{\textit{verb}}\ \textbf{1} Regard or consider in a specified way. {\fontspec{DejaVu Sans}◇} \textit{the event was deemed a great success} \colorBulletS{SYN} regard as, consider, judge, adjudge, hold to be, look on as, view as, see as, take to be, take for, class as, estimate as, count, rate, find, esteem, calculate to be, gauge, suppose, reckon, account, interpret as}{}{}{ \colorBullet{ORIGIN} Old English dēman (also in the sense ‘act as judge’), of Germanic origin; related to Dutch doeman, also to doom.}%
\par%
\entry{defamation}{/ˌdɛfəˈmeɪʃ(ə)n/}{মানহানি}{ \textsf{\textit{noun}}\ \textbf{1} The action of damaging the good reputation of someone; slander or libel. {\fontspec{DejaVu Sans}◇} \textit{she sued him for defamation} \colorBulletS{SYN} libel, slander, character assassination, defamation of character, calumny, vilification, traducement, obloquy, scandal, scandalmongering, malicious gossip, tittle{-}tattle, backbiting, aspersions, muckraking, abuse, malediction}{}{The defamation case filed against the barguna uno}{}%
\par%
\entry{defamatory}{/dɪˈfamət(ə)ri/}{মানহানিকর}{ \textsf{\textit{adjective}}\ \textbf{1} (of remarks, writing, etc.) damaging the good reputation of someone; slanderous or libellous. {\fontspec{DejaVu Sans}◇} \textit{a defamatory allegation} \colorBulletS{SYN} libellous, slanderous, defaming, calumnious, calumniatory, vilifying, traducing, scandalous, scandalmongering, malicious, vicious, backbiting, muckraking, abusive, maledictory, maledictive}{}{}{}%
\par%
\entry{default}{/dɪˈfɔːlt/}{ডিফল্ট}{\small{\textsf{\textit{noun, verb}}} \\{\fontspec{DejaVu Sans}▪ }\textsf{\textit{noun}}\\ \textbf{1} Failure to fulfil an obligation, especially to repay a loan or appear in a law court. {\fontspec{DejaVu Sans}◇} \textit{the company will have to restructure its debts to avoid default} \colorBulletS{SYN} non{-}payment, failure to pay, non{-}remittance \textbf{2} A preselected option adopted by a computer program or other mechanism when no alternative is specified by the user or programmer. {\fontspec{DejaVu Sans}◇} \textit{the default is fifty lines} \\{\fontspec{DejaVu Sans}▪ }\textsf{\textit{verb}}\\ \textbf{1} Fail to fulfil an obligation, especially to repay a loan or to appear in a law court. {\fontspec{DejaVu Sans}◇} \textit{the dealer could repossess the goods if the customer defaulted} \colorBulletS{SYN} fail to pay, not pay, renege, fail to honour, back out, backtrack, backslide \textbf{2} (of a computer program or other mechanism) revert automatically to (a preselected option) {\fontspec{DejaVu Sans}◇} \textit{when you start a fresh letter the system will default to its own style} \colorBulletS{SYN} revert}{}{}{ \colorBullet{ORIGIN} Middle English from Old French defaut, from defaillir ‘to fail’, based on Latin fallere ‘disappoint, deceive’.}%
\par%
\entry{defeat}{/dɪˈfiːt/}{পরাজয়}{\small{\textsf{\textit{noun, verb}}} \\{\fontspec{DejaVu Sans}▪ }\textsf{\textit{noun}}\\ \textbf{1} An instance of defeating or being defeated. {\fontspec{DejaVu Sans}◇} \textit{a 1–0 defeat by Grimsby} \colorBulletS{SYN} loss, beating, conquest, conquering, besting, worsting, vanquishing, vanquishment, game, set, and match \\{\fontspec{DejaVu Sans}▪ }\textsf{\textit{verb}}\\ \textbf{1} Win a victory over (someone) in a battle or other contest; overcome or beat. {\fontspec{DejaVu Sans}◇} \textit{Garibaldi defeated the Neapolitan army} \colorBulletS{SYN} beat, conquer, win against, win a victory over, triumph over, prevail over, get the better of, best, worst, vanquish}{}{}{ \colorBullet{ORIGIN} Late Middle English (in the sense ‘undo, destroy, annul’): from Old French desfait ‘undone’, past participle of desfaire, from medieval Latin disfacere ‘undo’.}%
\par%
\entry{defecate}{/ˈdɛfɪkeɪt/}{মলত্যাগ করা}{ \textsf{\textit{verb}}\ \textbf{1} Discharge faeces from the body. {\fontspec{DejaVu Sans}◇} \textit{} \colorBulletS{SYN} excrete, discharge faeces, excrete faeces, pass faeces, have a bowel movement, have a BM, evacuate one's bowels, open one's bowels, void excrement, relieve oneself, go to the lavatory}{}{}{ \colorBullet{ORIGIN} Late Middle English (in the sense ‘clear of dregs, purify’): from Latin defaecat{-} ‘cleared of dregs’, from the verb defaecare, from de{-} (expressing removal) + faex, faec{-} ‘dregs’. The current sense dates from the mid 19th century.}%
\par%
\entry{deficiency}{/dɪˈfɪʃ(ə)nsi/}{অভাব}{ \textsf{\textit{noun}}\ \textbf{1} A lack or shortage. {\fontspec{DejaVu Sans}◇} \textit{deficiencies in material resources} \colorBulletS{SYN} insufficiency, lack, shortage, want, dearth, inadequacy, deficit, shortfall}{}{}{}%
\par%
\entry{defile}{/dɪˈfʌɪl/}{গিরিসঙ্কট}{ \textsf{\textit{verb}}\ \textbf{1} Damage the purity or appearance of; mar or spoil. {\fontspec{DejaVu Sans}◇} \textit{the land was defiled by a previous owner} \colorBulletS{SYN} spoil, sully, mar, impair, debase, degrade}{}{}{ \colorBullet{ORIGIN} Late Middle English alteration of obsolete defoul, from Old French defouler ‘trample down’, influenced by obsolete befile ‘befoul, defile’.}%
\par%
\entry{defile}{/dɪˈfʌɪl/}{গিরিসঙ্কট}{\small{\textsf{\textit{noun, verb}}} \\{\fontspec{DejaVu Sans}▪ }\textsf{\textit{noun}}\\ \textbf{1} A steep{-}sided narrow gorge or passage (originally one requiring troops to march in single file) {\fontspec{DejaVu Sans}◇} \textit{the twisting track wormed its way up a defile to level ground} \\{\fontspec{DejaVu Sans}▪ }\textsf{\textit{verb}}\\ \textbf{1} (of troops) march in single file. {\fontspec{DejaVu Sans}◇} \textit{we emerged after defiling through the mountainsides}}{}{}{ \colorBullet{ORIGIN} Late 17th century from French défilé (noun), défiler (verb), from dé ‘away from’ + file ‘column, file’.}%
\par%
\entry{deflection}{/dɪˈflɛkʃ(ə)n/}{বিনিময়তা}{ \textsf{\textit{noun}}\ \textbf{1} The action or process of deflecting or being deflected. {\fontspec{DejaVu Sans}◇} \textit{the deflection of the light beam} \colorBulletS{SYN} turning aside, turning away, turning, diversion, drawing away}{}{}{ \colorBullet{ORIGIN} Early 17th century from late Latin deflexio(n{-}), from deflectere ‘bend away’ (see deflect).}%
\par%
\entry{delegate}{/ˈdɛlɪɡət/}{প্রতিনিধি}{\small{\textsf{\textit{noun, verb}}} \\{\fontspec{DejaVu Sans}▪ }\textsf{\textit{noun}}\\ \textbf{1} A person sent or authorized to represent others, in particular an elected representative sent to a conference. {\fontspec{DejaVu Sans}◇} \textit{congress delegates rejected the proposals} \colorBulletS{SYN} representative, envoy, emissary, commissioner, agent, deputy, commissary \\{\fontspec{DejaVu Sans}▪ }\textsf{\textit{verb}}\\ \textbf{1} Entrust (a task or responsibility) to another person, typically one who is less senior than oneself. {\fontspec{DejaVu Sans}◇} \textit{she must delegate duties so as to free herself for more important tasks} \colorBulletS{SYN} assign, entrust, give, pass on, hand on, hand over, turn over, consign, devolve, depute, transfer}{}{}{ \colorBullet{ORIGIN} Late Middle English from Latin delegatus ‘sent on a commission’, from the verb delegare, from de{-} ‘down’ + legare ‘depute’.}%
\par%
\entry{delegation}{/dɛlɪˈɡeɪʃ(ə)n/}{প্রতিনিধিদল}{ \textsf{\textit{noun}}\ \textbf{1} A body of delegates or representatives; a deputation. {\fontspec{DejaVu Sans}◇} \textit{a delegation of teachers} \colorBulletS{SYN} deputation, delegacy, legation, mission, diplomatic mission, commission \textbf{2} The action or process of delegating or being delegated. {\fontspec{DejaVu Sans}◇} \textit{the delegation of power to the district councils} \colorBulletS{SYN} assignment, entrusting, giving, committal, devolution, deputation, transference}{}{}{ \colorBullet{ORIGIN} Early 17th century (denoting the action or process of delegating; also in the sense ‘delegated power’): from Latin delegatio(n{-}), from delegare ‘send on a commission’ (see delegate).}%
\par%
\entry{delight}{/dɪˈlʌɪt/}{আমোদ}{\small{\textsf{\textit{noun, verb}}} \\{\fontspec{DejaVu Sans}▪ }\textsf{\textit{noun}}\\ \textbf{1} Great pleasure. {\fontspec{DejaVu Sans}◇} \textit{the little girls squealed with delight} \colorBulletS{SYN} pleasure, happiness, joy, joyfulness, glee, gladness, gratification, relish, excitement, amusement \\{\fontspec{DejaVu Sans}▪ }\textsf{\textit{verb}}\\ \textbf{1} Please (someone) greatly. {\fontspec{DejaVu Sans}◇} \textit{an experience guaranteed to delight both young and old} \colorBulletS{SYN} please greatly, charm, enchant, captivate, entrance, bewitch, thrill, excite, take someone's breath away}{}{}{ \colorBullet{ORIGIN} Middle English from Old French delitier (verb), delit (noun), from Latin delectare ‘to charm’, frequentative of delicere. The {-}gh{-} was added in the 16th century by association with light.}%
\par%
\entry{delinquency}{/dɪˈlɪŋkw(ə)nsi/}{কর্তব্যে অবহেলা}{ \textsf{\textit{noun}}\ \textbf{1} Minor crime, especially that committed by young people. {\fontspec{DejaVu Sans}◇} \textit{social causes of crime and delinquency} \colorBulletS{SYN} crime, wrongdoing, criminality, lawbreaking, lawlessness, misconduct, misbehaviour \textbf{2} Neglect of one's duty. {\fontspec{DejaVu Sans}◇} \textit{he relayed this in such a manner as to imply grave delinquency on the host's part} \colorBulletS{SYN} negligence, dereliction of duty, remissness, neglectfulness, irresponsibility}{}{}{ \colorBullet{ORIGIN} Mid 17th century from ecclesiastical Latin delinquentia, from Latin delinquent{-} ‘offending’ (see delinquent).}%
\par%
\entry{deluge}{/ˈdɛljuːdʒ/}{মহাপ্লাবন}{\small{\textsf{\textit{noun, verb}}} \\{\fontspec{DejaVu Sans}▪ }\textsf{\textit{noun}}\\ \textbf{1} A severe flood. {\fontspec{DejaVu Sans}◇} \textit{this may be the worst deluge in living memory} \colorBulletS{SYN} flood, flash flood, torrent \\{\fontspec{DejaVu Sans}▪ }\textsf{\textit{verb}}\\ \textbf{1} Overwhelm with a flood. {\fontspec{DejaVu Sans}◇} \textit{caravans were deluged by the heavy rains} \colorBulletS{SYN} flood, inundate, engulf, submerge, swamp, drown}{}{}{ \colorBullet{ORIGIN} Late Middle English from Old French, variant of diluve, from Latin diluvium, from diluere ‘wash away’.}%
\par%
\entry{demo}{/ˈdɛməʊ/}{ডেমো}{\small{\textsf{\textit{noun, verb}}} \\{\fontspec{DejaVu Sans}▪ }\textsf{\textit{noun}}\\ \textbf{1} A demonstration of a product or technique. {\fontspec{DejaVu Sans}◇} \textit{a cookery demo} \colorBulletS{SYN} exhibition, presentation, display, illustration, exposition, teach{-}in \textbf{2} A public meeting or march protesting against something or expressing views on a political issue. {\fontspec{DejaVu Sans}◇} \textit{a peace demo} \colorBulletS{SYN} protest, protest march, march, parade, rally, lobby, sit{-}in, sit{-}down, sleep{-}in, stoppage, strike, walkout, picket, picket line, blockade \\{\fontspec{DejaVu Sans}▪ }\textsf{\textit{verb}}\\ \textbf{1} Record (a song or piece of music) to demonstrate the capabilities of a musical group or performer or as preparation for a full recording. {\fontspec{DejaVu Sans}◇} \textit{they've already demoed twelve new songs} \textbf{2} Demonstrate the capabilities of (software or another product) {\fontspec{DejaVu Sans}◇} \textit{Apple is expected to demo the newest version of its mobile operating system at the conference next week}}{}{}{ \colorBullet{ORIGIN} Early 20th century abbreviation of demonstration and demonstrate.}%
\par%
\entry{demo}{/ˈdɛməʊ/}{ডেমো}{ \textsf{\textit{noun}}\ \textbf{1} short for demographic {\fontspec{DejaVu Sans}◇} \textit{both channels managed to maintain ratings among young male demos}}{}{}{}%
\par%
\entry{demography}{/dɪˈmɒɡrəfi/}{জনসংখ্যা}{ \textsf{\textit{noun}}\ \textbf{1} The study of statistics such as births, deaths, income, or the incidence of disease, which illustrate the changing structure of human populations. {\fontspec{DejaVu Sans}◇} \textit{}}{}{}{ \colorBullet{ORIGIN} Mid 19th century from Greek dēmos ‘the people’ + {-}graphy.}%
\par%
\entry{demonstration}{/dɛmənˈstreɪʃ(ə)n/}{প্রদর্শন; বিক্ষোভ}{ \textsf{\textit{noun}}\ \textbf{1} An act of showing that something exists or is true by giving proof or evidence. {\fontspec{DejaVu Sans}◇} \textit{his demonstration of the need for computer corpora in language study is convincing} \colorBulletS{SYN} proof, substantiation, confirmation, affirmation, corroboration, verification, validation \textbf{2} A practical exhibition and explanation of how something works or is performed. {\fontspec{DejaVu Sans}◇} \textit{a microwave cookery demonstration} \colorBulletS{SYN} exhibition, presentation, display, illustration, exposition, teach{-}in \textbf{3} A public meeting or march protesting against something or expressing views on a political issue. {\fontspec{DejaVu Sans}◇} \textit{a pro{-}democracy demonstration} \colorBulletS{SYN} protest, protest march, march, parade, rally, lobby, sit{-}in, sit{-}down, sleep{-}in, stoppage, strike, walkout, picket, picket line, blockade}{}{}{ \colorBullet{ORIGIN} Late Middle English (also in the senses ‘proof provided by logic’ and ‘sign, indication’): from Latin demonstratio(n{-}), from demonstrare ‘point out’ (see demonstrate). demonstration (sense 3) dates from the mid 19th century.}%
\par%
\entry{denial}{/dɪˈnʌɪ(ə)l/}{অস্বীকার}{ \textsf{\textit{noun}}\ \textbf{1} The action of denying something. {\fontspec{DejaVu Sans}◇} \textit{she shook her head in denial} \colorBulletS{SYN} contradiction, counterstatement, refutation, rebuttal, repudiation, disclaimer, retraction, abjuration}{}{}{}%
\par%
\entry{deny}{/dɪˈnʌɪ/}{অস্বীকার করা}{ \textsf{\textit{verb}}\ \textbf{1} State that one refuses to admit the truth or existence of. {\fontspec{DejaVu Sans}◇} \textit{both firms deny any responsibility for the tragedy} \colorBulletS{SYN} contradict, repudiate, gainsay, declare untrue, dissent from, disagree with, challenge, contest, oppose \textbf{2} Refuse to give (something requested or desired) to (someone) {\fontspec{DejaVu Sans}◇} \textit{the inquiry was denied access to intelligence sources} \colorBulletS{SYN} refuse, turn down, reject, rebuff, repulse, decline, veto, dismiss}{}{}{ \colorBullet{ORIGIN} Middle English from Old French deni{-}, stressed stem of deneier, from Latin denegare, from de{-} ‘formally’ + negare ‘say no’.}%
\par%
\entry{depart}{/dɪˈpɑːt/}{চরা}{ \textsf{\textit{verb}}\ \textbf{1} Leave, especially in order to start a journey. {\fontspec{DejaVu Sans}◇} \textit{they departed for Germany} \colorBulletS{SYN} leave, go, go away, go off, take one's leave, take oneself off, withdraw, absent oneself, say one's goodbyes, quit, make an exit, exit, break camp, decamp, retreat, beat a retreat, retire}{}{}{ \colorBullet{ORIGIN} Middle English from Old French departir, based on Latin dispertire ‘to divide’. The original sense was ‘separate’, also ‘take leave of each other’, hence ‘go away’.}%
\par%
\entry{deportation}{/diːpɔːˈteɪʃ(ə)n/}{বিতাড়িততা}{ \textsf{\textit{noun}}\ \textbf{1} The action of deporting a foreigner from a country. {\fontspec{DejaVu Sans}◇} \textit{asylum seekers facing deportation} \colorBulletS{SYN} expulsion, expelling, banishment, banishing, exile, exiling, transportation, transporting, extradition, extraditing, expatriation, expatriating, repatriation, repatriating, refoulement}{}{}{}%
\par%
\entry{depose}{/dɪˈpəʊz/}{প্রত্যায়ন করা}{ \textsf{\textit{verb}}\ \textbf{1} Remove from office suddenly and forcefully. {\fontspec{DejaVu Sans}◇} \textit{he had been deposed by a military coup} \colorBulletS{SYN} overthrow, overturn, topple, bring down, remove from office, remove, unseat, dethrone, supplant, displace \textbf{2} Testify to or give (evidence) under oath, typically in a written statement. {\fontspec{DejaVu Sans}◇} \textit{every affidavit shall state which of the facts deposed to are within the deponent's knowledge} \colorBulletS{SYN} swear, testify, attest, undertake, assert, declare, profess, aver, submit, claim}{}{}{ \colorBullet{ORIGIN} Middle English from Old French deposer, from Latin deponere (see deponent), but influenced by Latin depositus and Old French poser ‘to place’.}%
\par%
\entry{deprecate}{/ˈdɛprɪkeɪt/}{গম্ভীর করা}{ \textsf{\textit{verb}}\ \textbf{1} Express disapproval of. {\fontspec{DejaVu Sans}◇} \textit{what I deprecate is persistent indulgence} \colorBulletS{SYN} disapprove of, deplore, abhor, find unacceptable, be against, frown on, take a dim view of, look askance at, take exception to, detest, despise, execrate \textbf{2} another term for depreciate (sense 2) {\fontspec{DejaVu Sans}◇} \textit{he deprecates the value of children's television} \colorBulletS{SYN} belittle, disparage, denigrate, run down, discredit, decry, cry down, play down, make little of, trivialize, underrate, undervalue, underestimate, diminish, depreciate, deflate}{}{}{ \colorBullet{ORIGIN} Early 17th century (in the sense ‘pray against’): from Latin deprecat{-} ‘prayed against (as being evil)’, from the verb deprecari, from de{-} (expressing reversal) + precari ‘pray’.}%
\par%
\entry{depression}{/dɪˈprɛʃ(ə)n/}{}{ \textsf{\textit{noun}}\ \textbf{1} Feelings of severe despondency and dejection. {\fontspec{DejaVu Sans}◇} \textit{self{-}doubt creeps in and that swiftly turns to depression} \colorBulletS{SYN} melancholy, misery, sadness, unhappiness, sorrow, woe, gloom, gloominess, dejection, downheartedness, despondency, dispiritedness, low spirits, heavy{-}heartedness, moroseness, discouragement, despair, desolation, dolefulness, moodiness, pessimism, hopelessness \textbf{2} A long and severe recession in an economy or market. {\fontspec{DejaVu Sans}◇} \textit{the depression in the housing market} \colorBulletS{SYN} recession, slump, decline, downturn, slowdown, standstill \textbf{3} The action of lowering something or pressing something down. {\fontspec{DejaVu Sans}◇} \textit{depression of the plunger delivers two units of insulin} \textbf{4} A region of lower atmospheric pressure, especially a cyclonic weather system. {\fontspec{DejaVu Sans}◇} \textit{hurricanes start off as loose regions of bad weather known as tropical depressions} \textbf{5} The angular distance of an object below the horizon or a horizontal plane. {\fontspec{DejaVu Sans}◇} \textit{}}{}{Land depression}{ \colorBullet{ORIGIN} Late Middle English from Latin depressio(n{-}), from deprimere ‘press down’ (see depress).}%
\par%
\entry{deprive}{/dɪˈprʌɪv/}{বঞ্চিত}{ \textsf{\textit{verb}}\ \textbf{1} Prevent (a person or place) from having or using something. {\fontspec{DejaVu Sans}◇} \textit{the city was deprived of its water supplies} \colorBulletS{SYN} dispossess, strip, divest, relieve, bereave}{}{}{ \colorBullet{ORIGIN} Middle English (in the sense ‘depose from office’): from Old French depriver, from medieval Latin deprivare, from de{-} ‘away, completely’ + privare (see private).}%
\par%
\entry{derision}{/dɪˈrɪʒ(ə)n/}{উপহাস}{ \textsf{\textit{noun}}\ \textbf{1} Contemptuous ridicule or mockery. {\fontspec{DejaVu Sans}◇} \textit{my stories were greeted with derision and disbelief} \colorBulletS{SYN} mockery, ridicule, jeering, jeers, sneers, scoffing, jibing, taunts}{}{}{ \colorBullet{ORIGIN} Late Middle English via Old French from late Latin derisio(n{-}), from deridere ‘scoff at’.}%
\par%
\entry{derive}{/dɪˈrʌɪv/}{উদ্ভূত}{ \textsf{\textit{verb}}\ \textbf{1} Obtain something from (a specified source) {\fontspec{DejaVu Sans}◇} \textit{they derived great comfort from this assurance} \colorBulletS{SYN} obtain, get, take, gain, acquire, procure, extract, attain, glean}{}{}{ \colorBullet{ORIGIN} Late Middle English (in the sense ‘draw a fluid through or into a channel’): from Old French deriver or Latin derivare, from de{-} ‘down, away’ + rivus ‘brook, stream’.}%
\par%
\entry{derogatory}{/dɪˈrɒɡət(ə)ri/}{হানিকর}{ \textsf{\textit{adjective}}\ \textbf{1} Showing a critical or disrespectful attitude. {\fontspec{DejaVu Sans}◇} \textit{she tells me I'm fat and is always making derogatory remarks} \colorBulletS{SYN} disparaging, denigratory, belittling, diminishing, slighting, deprecatory, depreciatory, depreciative, detracting, deflating}{}{}{ \colorBullet{ORIGIN} Early 16th century (in the sense ‘impairing in force or effect’): from late Latin derogatorius, from derogat{-} ‘abrogated’, from the verb derogare (see derogate).}%
\par%
\entry{descend}{/dɪˈsɛnd/}{নামা}{ \textsf{\textit{verb}}\ \textbf{1} Move or fall downwards. {\fontspec{DejaVu Sans}◇} \textit{the aircraft began to descend} \colorBulletS{SYN} go down, come down \textbf{2} (of a road, path, or flight of steps) slope or lead downwards. {\fontspec{DejaVu Sans}◇} \textit{a side road descended into the forest} \colorBulletS{SYN} slope, dip, slant, decline, go down, sink, fall away \textbf{3} Make a sudden attack on. {\fontspec{DejaVu Sans}◇} \textit{the militia descended on Rye} \colorBulletS{SYN} attack, make a raid on, assault, set upon, descend on, swoop on, harass, harry, blitz, make inroads on, assail, storm, rush, charge \textbf{4} Be a blood relative of (a specified ancestor) {\fontspec{DejaVu Sans}◇} \textit{John Dalrymple was descended from an ancient Ayrshire family} \colorBulletS{SYN} be a descendant of, originate from, issue from, spring from, have as an ancestor, derive from}{}{Flood water has already started descending in many districts}{ \colorBullet{ORIGIN} Middle English from Old French descendre, from Latin descendere, from de{-} ‘down’ + scandere ‘to climb’.}%
\par%
\entry{designate}{/ˈdɛzɪɡneɪt/}{নামকরণ করা; মনোনীত করা}{\small{\textsf{\textit{adjective, verb}}} \\{\fontspec{DejaVu Sans}▪ }\textsf{\textit{adjective}}\\ \textbf{1} Appointed to an office or post but not yet installed. {\fontspec{DejaVu Sans}◇} \textit{the Director designate} \\{\fontspec{DejaVu Sans}▪ }\textsf{\textit{verb}}\\ \textbf{1} Appoint (someone) to a specified office or post. {\fontspec{DejaVu Sans}◇} \textit{he was designated as prime minister} \colorBulletS{SYN} appoint, nominate, depute, delegate}{}{}{ \colorBullet{ORIGIN} Mid 17th century (as an adjective): from Latin designatus ‘designated’, past participle of designare, based on signum ‘a mark’.}%
\par%
\entry{desire}{/dɪˈzʌɪə/}{ইচ্ছা}{\small{\textsf{\textit{noun, verb}}} \\{\fontspec{DejaVu Sans}▪ }\textsf{\textit{noun}}\\ \textbf{1} A strong feeling of wanting to have something or wishing for something to happen. {\fontspec{DejaVu Sans}◇} \textit{he resisted public desires for choice in education} \colorBulletS{SYN} wish, want \\{\fontspec{DejaVu Sans}▪ }\textsf{\textit{verb}}\\ \textbf{1} Strongly wish for or want (something) {\fontspec{DejaVu Sans}◇} \textit{he never achieved the status he so desired} \colorBulletS{SYN} wish for, want, long for, yearn for, crave, set one's heart on, hanker after, hanker for, pine after, pine for, thirst for, itch for, be desperate for, be bent on, have a need for, covet, aspire to}{}{}{ \colorBullet{ORIGIN} Middle English from Old French desir (noun), desirer (verb), from Latin desiderare (see desiderate).}%
\par%
\entry{desperate}{/ˈdɛsp(ə)rət/}{মরিয়া}{ \textsf{\textit{adjective}}\ \textbf{1} Feeling or showing a hopeless sense that a situation is so bad as to be impossible to deal with. {\fontspec{DejaVu Sans}◇} \textit{a desperate sadness enveloped Ruth} \colorBulletS{SYN} despairing, hopeless \textbf{2} (of a person) having a great need or desire for something. {\fontspec{DejaVu Sans}◇} \textit{I am desperate for a cigarette} \colorBulletS{SYN} in great need of, urgently requiring, craving, in want of, lacking, wanting}{}{}{ \colorBullet{ORIGIN} Late Middle English (in the sense ‘in despair’): from Latin desperatus ‘deprived of hope’, past participle of desperare (see despair).}%
\par%
\entry{desperation}{/dɛspəˈreɪʃn/}{হতাশা}{ \textsf{\textit{noun}}\ \textbf{1} A state of despair, typically one which results in rash or extreme behaviour. {\fontspec{DejaVu Sans}◇} \textit{she wrote to him in desperation} \colorBulletS{SYN} hopelessness, despair, distress}{}{}{ \colorBullet{ORIGIN} Late Middle English from Old French, from Latin desperatio(n{-}), from the verb desperare (see despair).}%
\par%
\entry{despicable}{/dɪˈspɪkəb(ə)l/}{ঘৃণ্য}{ \textsf{\textit{adjective}}\ \textbf{1} Deserving hatred and contempt. {\fontspec{DejaVu Sans}◇} \textit{a despicable crime} \colorBulletS{SYN} contemptible, loathsome, hateful, detestable, reprehensible, abhorrent, abominable, awful, heinous, beyond the pale}{}{}{ \colorBullet{ORIGIN} Mid 16th century from late Latin despicabilis, from despicari ‘look down on’.}%
\par%
\entry{despite}{/dɪˈspʌɪt/}{সত্ত্বেও}{\small{\textsf{\textit{noun, preposition}}} \\{\fontspec{DejaVu Sans}▪ }\textsf{\textit{noun}}\\ \textbf{1} Contemptuous treatment or behaviour; outrage. {\fontspec{DejaVu Sans}◇} \textit{the despite done by him to the holy relics} \textbf{2} Contempt; disdain. {\fontspec{DejaVu Sans}◇} \textit{the theatre only earns my despite} \colorBulletS{SYN} contempt, scorn, scornfulness, contemptuousness, derision, disrespect \\{\fontspec{DejaVu Sans}▪ }\textsf{\textit{preposition}}\\ \textbf{1} Without being affected by; in spite of. {\fontspec{DejaVu Sans}◇} \textit{he remains a great leader despite age and infirmity} \colorBulletS{SYN} in spite of, notwithstanding, regardless of, in defiance of, without being affected by, in the face of, for all, even with, undeterred by}{}{}{ \colorBullet{ORIGIN} Middle English (originally used as a noun meaning ‘contempt, scorn’ in the phrase in despite of): from Old French despit, from Latin despectus ‘looking down on’, past participle (used as a noun) of despicere (see despise).}%
\par%
\entry{destabilize}{/diːˈsteɪb(ə)lʌɪz/}{অস্থিতিশীল}{ \textsf{\textit{verb}}\ \textbf{1} Upset the stability of (a region or system); cause unrest or instability in. {\fontspec{DejaVu Sans}◇} \textit{the accused were charged with conspiracy to destabilize the country} \colorBulletS{SYN} undermine, weaken, impair, damage, subvert, sabotage, unsettle, upset, disrupt, wreck, ruin}{}{}{}%
\par%
\entry{detain}{/dɪˈteɪn/}{আটক করা}{ \textsf{\textit{verb}}\ \textbf{1} Keep (someone) from proceeding by holding them back or making claims on their attention. {\fontspec{DejaVu Sans}◇} \textit{she made to open the door, but he detained her} \colorBulletS{SYN} delay, hold up, make late, retard, keep, keep back, slow up, slow down, set back, get bogged down}{}{}{ \colorBullet{ORIGIN} Late Middle English (in the sense ‘be afflicted with sickness or infirmity’): from Old French detenir, from a variant of Latin detinere, from de{-} ‘away, aside’ + tenere ‘to hold’.}%
\par%
\entry{detention}{/dɪˈtɛnʃ(ə)n/}{আটক}{ \textsf{\textit{noun}}\ \textbf{1} The action of detaining someone or the state of being detained in official custody. {\fontspec{DejaVu Sans}◇} \textit{the fifteen people arrested were still in police detention} \colorBulletS{SYN} custody, imprisonment, confinement, incarceration, internment, captivity, restraint, arrest, house arrest, remand, committal}{}{}{ \colorBullet{ORIGIN} Late Middle English (in the sense ‘withholding of what is claimed or due’): from late Latin detentio(n{-}), from Latin detinere ‘hold back’ (see detain).}%
\par%
\entry{deteriorate}{/dɪˈtɪərɪəreɪt/}{ধসা; অবনতি}{ \textsf{\textit{verb}}\ \textbf{1} Become progressively worse. {\fontspec{DejaVu Sans}◇} \textit{relations between the countries had deteriorated sharply} \colorBulletS{SYN} worsen, get worse, decline, be in decline, degenerate, decay}{}{}{ \colorBullet{ORIGIN} Late 16th century (in the sense ‘make worse’): from late Latin deteriorat{-} ‘worsened’, from the verb deteriorare, from Latin deterior ‘worse’.}%
\par%
\entry{detonate}{/ˈdɛtəneɪt/}{বিস্ফোরিত হত্তয়া}{ \textsf{\textit{verb}}\ \textbf{1} Explode or cause to explode. {\fontspec{DejaVu Sans}◇} \textit{two other bombs failed to detonate} \colorBulletS{SYN} explode, go off, be set off, blow up, burst apart, shatter, erupt}{}{}{ \colorBullet{ORIGIN} Early 18th century from Latin detonat{-} ‘thundered down or forth’, from the verb detonare, from de{-} ‘down’ + tonare ‘to thunder’.}%
\par%
\entry{devastating}{/ˈdɛvəsteɪtɪŋ/}{বিধ্বংসী}{ \textsf{\textit{adjective}}\ \textbf{1} Highly destructive or damaging. {\fontspec{DejaVu Sans}◇} \textit{a devastating cyclone} \colorBulletS{SYN} destructive, ruinous, disastrous, catastrophic, calamitous, cataclysmic}{}{}{}%
\par%
\entry{devise}{/dɪˈvʌɪz/}{উইল}{\small{\textsf{\textit{noun, verb}}} \\{\fontspec{DejaVu Sans}▪ }\textsf{\textit{noun}}\\ \textbf{1} A clause in a will leaving something, especially real estate, to someone. {\fontspec{DejaVu Sans}◇} \textit{The issue, however, is whether the language of the devise of the Somerset Estate can fairly be interpreted so as to include the rights under the s. 2 reverter.} \\{\fontspec{DejaVu Sans}▪ }\textsf{\textit{verb}}\\ \textbf{1} Plan or invent (a complex procedure, system, or mechanism) by careful thought. {\fontspec{DejaVu Sans}◇} \textit{a training programme should be devised} \colorBulletS{SYN} conceive, think up, come up with, dream up, draw up, work out, form, formulate, concoct, design, frame, invent, coin, originate, compose, construct, fabricate, create, produce, put together, make up, develop, evolve \textbf{2} Leave (something, especially real estate) to someone by the terms of a will. {\fontspec{DejaVu Sans}◇} \textit{All the residue of my estate, including real and personal property, I give, devise, and bequeath to Earlham College.} \colorBulletS{SYN} leave, leave in one's will, will, make over, pass on, hand on, hand down, cede, consign, commit, entrust, grant, transfer, convey}{}{}{ \colorBullet{ORIGIN} Middle English the verb from Old French deviser, from Latin divis{-} ‘divided’, from the verb dividere (this sense being reflected in the original English sense of the verb); the noun is a variant of device (in the early sense ‘will, desire’).}%
\par%
\entry{devour}{/dɪˈvaʊə/}{গ্রাস করা}{ \textsf{\textit{verb}}\ \textbf{1} Eat (food or prey) hungrily or quickly. {\fontspec{DejaVu Sans}◇} \textit{he devoured half of his burger in one bite} \colorBulletS{SYN} eat hungrily, eat quickly, eat greedily, eat heartily, eat up, swallow, gobble, gobble down, gobble up, guzzle, guzzle down, gulp, gulp down, bolt, bolt down, cram down, gorge oneself on, wolf, wolf down, feast on, consume}{}{Brahmaputra continues devouring houses, land}{ \colorBullet{ORIGIN} Middle English from Old French devorer, from Latin devorare, from de{-} ‘down’ + vorare ‘to swallow’.}%
\par%
\entry{dictate}{/dɪkˈteɪt/}{নির্দেশ}{\small{\textsf{\textit{noun, verb}}} \\{\fontspec{DejaVu Sans}▪ }\textsf{\textit{noun}}\\ \textbf{1} An order or principle that must be obeyed. {\fontspec{DejaVu Sans}◇} \textit{the dictates of fashion} \colorBulletS{SYN} order, command, decree, edict, rule, ruling, ordinance, dictum, directive, direction, instruction, pronouncement, mandate, requirement, stipulation, injunction, ultimatum, demand, exhortation \\{\fontspec{DejaVu Sans}▪ }\textsf{\textit{verb}}\\ \textbf{1} State or order authoritatively. {\fontspec{DejaVu Sans}◇} \textit{the tsar's attempts to dictate policy} \colorBulletS{SYN} give orders to, order about, order around, boss, boss about, boss around, impose one's will on, lord it over, bully, domineer, dominate, tyrannize, oppress, ride roughshod over, control, pressurize, browbeat \textbf{2} Say or read aloud (words to be typed, written down, or recorded on tape) {\fontspec{DejaVu Sans}◇} \textit{I have four letters to dictate} \colorBulletS{SYN} say aloud, utter, speak, read out, read aloud, recite}{}{}{ \colorBullet{ORIGIN} Late 16th century (in dictate (sense 2 of the verb)): from Latin dictat{-} ‘dictated’, from the verb dictare.}%
\par%
\entry{dietary}{/ˈdʌɪət(ə)ri/}{খাদ্যতালিকাগত}{\small{\textsf{\textit{adjective, noun}}} \\{\fontspec{DejaVu Sans}▪ }\textsf{\textit{adjective}}\\ \textbf{1} Relating to or provided by diet. {\fontspec{DejaVu Sans}◇} \textit{dietary advice for healthy skin and hair} \\{\fontspec{DejaVu Sans}▪ }\textsf{\textit{noun}}\\ \textbf{1} A regulated or restricted diet. {\fontspec{DejaVu Sans}◇} \textit{}}{}{Dietary fiber}{ \colorBullet{ORIGIN} Late Middle English (as a noun): from medieval Latin dietarium, from Latin diaeta (see diet).}%
\par%
\entry{dignity}{/ˈdɪɡnɪti/}{সম্মান}{ \textsf{\textit{noun}}\ \textbf{1} The state or quality of being worthy of honour or respect. {\fontspec{DejaVu Sans}◇} \textit{the dignity of labour} \textbf{2} A composed or serious manner or style. {\fontspec{DejaVu Sans}◇} \textit{he bowed with great dignity} \colorBulletS{SYN} stateliness, nobleness, nobility, majesty, regalness, regality, royalness, courtliness, augustness, loftiness, exaltedness, lordliness, impressiveness, grandeur, magnificence}{}{}{ \colorBullet{ORIGIN} Middle English from Old French dignete, from Latin dignitas, from dignus ‘worthy’.}%
\par%
\entry{dilapidated}{/dɪˈlapɪdeɪtɪd/}{জীর্ণ}{ \textsf{\textit{adjective}}\ \textbf{1} (of a building or object) in a state of disrepair or ruin as a result of age or neglect. {\fontspec{DejaVu Sans}◇} \textit{old, dilapidated buildings} \colorBulletS{SYN} run down, tumbledown, ramshackle, broken{-}down, in disrepair, shabby, battered, rickety, shaky, unsound, crumbling, in ruins, ruined, decayed, decaying, deteriorating, deteriorated, decrepit, worn out}{}{}{}%
\par%
\entry{dilemma}{/dɪˈlɛmə/}{উভয়সঙ্কট}{ \textsf{\textit{noun}}\ \textbf{1} A situation in which a difficult choice has to be made between two or more alternatives, especially ones that are equally undesirable. {\fontspec{DejaVu Sans}◇} \textit{he wants to make money, but he also disapproves of it: Den's dilemma in a nutshell} \colorBulletS{SYN} quandary, predicament, difficulty, problem, puzzle, conundrum, awkward situation, tricky situation, difficult situation, difficult choice, catch{-}22, vicious circle, plight, mess, muddle}{}{}{ \colorBullet{ORIGIN} Early 16th century (denoting a form of argument involving a choice between equally unfavourable alternatives): via Latin from Greek dilēmma, from di{-} ‘twice’ + lēmma ‘premise’.}%
\par%
\entry{dilettante}{/ˌdɪlɪˈtanteɪ/}{অপটু কর্মী}{ \textsf{\textit{noun}}\ \textbf{1} A person who cultivates an area of interest, such as the arts, without real commitment or knowledge. {\fontspec{DejaVu Sans}◇} \textit{a wealthy literary dilettante} \colorBulletS{SYN} dabbler, potterer, tinkerer, trifler, dallier}{}{}{ \colorBullet{ORIGIN} Mid 18th century from Italian, ‘person loving the arts’, from dilettare ‘to delight’, from Latin delectare.}%
\par%
\entry{dire}{/ˈdʌɪə/}{ভয়ানক}{ \textsf{\textit{adjective}}\ \textbf{1} Extremely serious or urgent. {\fontspec{DejaVu Sans}◇} \textit{misuse of drugs can have dire consequences} \colorBulletS{SYN} terrible, dreadful, appalling, frightful, awful, horrible, atrocious, grim, unspeakable, distressing, harrowing, alarming, shocking, outrageous \textbf{2} Of a very poor quality. {\fontspec{DejaVu Sans}◇} \textit{the concert was dire} \colorBulletS{SYN} substandard, below standard, below par, bad, deficient, defective, faulty, imperfect, inferior, mediocre}{}{}{ \colorBullet{ORIGIN} Mid 16th century from Latin dirus ‘fearful, threatening’.}%
\par%
\entry{direct}{/dɪˈrɛkt/}{সরাসরি}{\small{\textsf{\textit{adjective, adverb, verb}}} \\{\fontspec{DejaVu Sans}▪ }\textsf{\textit{adjective}}\\ \textbf{1} Extending or moving from one place to another without changing direction or stopping. {\fontspec{DejaVu Sans}◇} \textit{there was no direct flight that day} \colorBulletS{SYN} straight, undeviating, unswerving \textbf{2} Without intervening factors or intermediaries. {\fontspec{DejaVu Sans}◇} \textit{the complications are a direct result of bacteria spreading} \colorBulletS{SYN} face to face, personal, unmediated, head{-}on, immediate, first{-}hand \textbf{3} (of a person or their behaviour) going straight to the point; frank. {\fontspec{DejaVu Sans}◇} \textit{he is very direct and honest} \colorBulletS{SYN} frank, straightforward, honest, candid, open, sincere, straight, straight to the point, blunt, plain{-}spoken, outspoken, forthright, downright, uninhibited, unreserved, point blank, no{-}nonsense, matter{-}of{-}fact, bluff, undiplomatic, tactless \textbf{4} Perpendicular to a surface; not oblique. {\fontspec{DejaVu Sans}◇} \textit{a direct butt joint between surfaces of steel} \\{\fontspec{DejaVu Sans}▪ }\textsf{\textit{adverb}}\\ \textbf{1} With no one or nothing in between. {\fontspec{DejaVu Sans}◇} \textit{they seem reluctant to deal with me direct} \colorBulletS{SYN} directly, straight, in person, without an intermediary \\{\fontspec{DejaVu Sans}▪ }\textsf{\textit{verb}}\\ \textbf{1} Control the operations of; manage or govern. {\fontspec{DejaVu Sans}◇} \textit{an economic elite directed the nation's affairs} \colorBulletS{SYN} administer, manage, run, control, govern, conduct, handle \textbf{2} Aim (something) in a particular direction or at a particular person. {\fontspec{DejaVu Sans}◇} \textit{heating ducts to direct warm air to rear{-}seat passengers} \colorBulletS{SYN} aim, point, level \textbf{3} Give (someone) an official order or authoritative instruction. {\fontspec{DejaVu Sans}◇} \textit{the judge directed him to perform community service} \colorBulletS{SYN} instruct, tell, command, order, give orders to, charge, call on, require, dictate}{}{}{ \colorBullet{ORIGIN} Late Middle English from Latin directus, past participle of dirigere, from di{-} ‘distinctly’ or de{-} ‘down’ + regere ‘put straight’.}%
\par%
\entry{directorate}{/dɪˈrɛkt(ə)rət/}{পরিচালকের দপ্তর}{ \textsf{\textit{noun}}\ \textbf{1} The board of directors of a company. {\fontspec{DejaVu Sans}◇} \textit{} \colorBulletS{SYN} committee, council, panel, directorate, commission, group, delegation, delegates, trustees, panel of trustees, convocation \textbf{2} A section of a government department in charge of a particular activity. {\fontspec{DejaVu Sans}◇} \textit{the Food Safety Directorate} \colorBulletS{SYN} administration, executive, regime, authority, powers that be, directorate, council, leadership, management}{}{}{}%
\par%
\entry{disappear}{/dɪsəˈpɪə/}{অদৃশ্য}{ \textsf{\textit{verb}}\ \textbf{1} Cease to be visible. {\fontspec{DejaVu Sans}◇} \textit{he disappeared into the trees} \colorBulletS{SYN} vanish, pass from sight, cease to be visible, vanish from sight, recede from view, be lost to sight, be lost to view, fade, fade away, melt away}{}{}{ \colorBullet{ORIGIN} Late Middle English from dis{-} (expressing reversal) + appear, on the pattern of French disparaître.}%
\par%
\entry{disburse}{/dɪsˈbəːs/}{নির্বাহ করা}{ \textsf{\textit{verb}}\ \textbf{1} Pay out (money from a fund) {\fontspec{DejaVu Sans}◇} \textit{\$67 million of the pledged aid had already been disbursed} \colorBulletS{SYN} pay out, lay out, spend, expend, dole out, hand out, part with, donate, give}{}{}{ \colorBullet{ORIGIN} Mid 16th century from Old French desbourser, from des{-} (expressing removal) + bourse ‘purse’.}%
\par%
\entry{disbursement}{/dɪsˈbəːsm(ə)nt/}{বিতরণ}{ \textsf{\textit{noun}}\ \textbf{1} The payment of money from a fund. {\fontspec{DejaVu Sans}◇} \textit{they established a committee to supervise the disbursement of aid} \colorBulletS{SYN} payment, disbursal, paying out, laying out, spending, expending, expenditure, disposal, outlay, doling out, handing out, parting with, donation, giving}{}{}{}%
\par%
\entry{discard}{/dɪˈskɑːd/}{বাতিল}{\small{\textsf{\textit{noun, verb}}} \\{\fontspec{DejaVu Sans}▪ }\textsf{\textit{noun}}\\ \textbf{1} A thing rejected as no longer useful or desirable. {\fontspec{DejaVu Sans}◇} \textit{} \colorBulletS{SYN} substandard article, discard, second \\{\fontspec{DejaVu Sans}▪ }\textsf{\textit{verb}}\\ \textbf{1} Get rid of (someone or something) as no longer useful or desirable. {\fontspec{DejaVu Sans}◇} \textit{Hilary bundled up the clothes she had discarded} \colorBulletS{SYN} dispose of, throw away, throw out, get rid of, toss out}{}{}{ \colorBullet{ORIGIN} Late 16th century (originally in the sense ‘reject (a playing card’)): from dis{-} (expressing removal) + the noun card.}%
\par%
\entry{disclose}{/dɪsˈkləʊz/}{প্রকাশ করা}{ \textsf{\textit{verb}}\ \textbf{1} Make (secret or new information) known. {\fontspec{DejaVu Sans}◇} \textit{they disclosed her name to the press} \colorBulletS{SYN} reveal, make known, divulge, tell, impart, communicate, pass on, vouchsafe, unfold}{}{}{ \colorBullet{ORIGIN} Late Middle English from Old French desclos{-}, stem of desclore, based on Latin claudere ‘to close’.}%
\par%
\entry{disclosure}{/dɪsˈkləʊʒə/}{প্রকাশ}{ \textsf{\textit{noun}}\ \textbf{1} The action of making new or secret information known. {\fontspec{DejaVu Sans}◇} \textit{a judge ordered the disclosure of the government documents} \colorBulletS{SYN} revelation, surprising fact, divulgence, declaration, announcement, news, report}{}{She said there should be a standardised information disclosure policy}{ \colorBullet{ORIGIN} Late 16th century from disclose, on the pattern of closure.}%
\par%
\entry{discontent}{/dɪskənˈtɛnt/}{অসন্তোষ}{\small{\textsf{\textit{adjective, noun}}} \\{\fontspec{DejaVu Sans}▪ }\textsf{\textit{adjective}}\\ \textbf{1} Dissatisfied. {\fontspec{DejaVu Sans}◇} \textit{he was discontent with his wages} \colorBulletS{SYN} dissatisfied, disgruntled, fed up, disaffected, discontent, malcontent, unhappy, aggrieved, displeased, resentful, envious \\{\fontspec{DejaVu Sans}▪ }\textsf{\textit{noun}}\\ \textbf{1} Dissatisfaction with one's circumstances; lack of contentment. {\fontspec{DejaVu Sans}◇} \textit{voters voiced discontent with both parties} \colorBulletS{SYN} dissatisfaction, disaffection, discontentment, discontentedness, disgruntlement, grievances, unhappiness, displeasure, bad feelings, resentment, envy}{}{}{}%
\par%
\entry{discreet}{/dɪˈskriːt/}{বিচক্ষণ}{ \textsf{\textit{adjective}}\ \textbf{1} Careful and prudent in one's speech or actions, especially in order to keep something confidential or to avoid embarrassment. {\fontspec{DejaVu Sans}◇} \textit{we made some discreet inquiries} \colorBulletS{SYN} careful, circumspect, cautious, wary, chary, guarded, close{-}lipped, close{-}mouthed}{}{}{ \colorBullet{ORIGIN} Middle English from Old French discret, from Latin discretus ‘separate’, past participle of discernere ‘discern’, the sense arising from late Latin discretio (see discretion). Compare with discrete.}%
\par%
\entry{discriminate}{/dɪˈskrɪmɪneɪt/}{ভেদ করা}{ \textsf{\textit{verb}}\ \textbf{1} Recognize a distinction; differentiate. {\fontspec{DejaVu Sans}◇} \textit{babies can discriminate between different facial expressions} \colorBulletS{SYN} differentiate, distinguish, draw a distinction, recognize a distinction, tell the difference, discern a difference \textbf{2} Make an unjust or prejudicial distinction in the treatment of different categories of people, especially on the grounds of race, sex, or age. {\fontspec{DejaVu Sans}◇} \textit{existing employment policies discriminate against women} \colorBulletS{SYN} be biased, show prejudice, be prejudiced}{}{}{ \colorBullet{ORIGIN} Early 17th century from Latin discriminat{-} ‘distinguished between’, from the verb discriminare, from discrimen ‘distinction’, from the verb discernere (see discern).}%
\par%
\entry{discriminatory}{/dɪˈskrɪmɪnɪˌt(ə)ri/}{পক্ষপাতমূলক}{ \textsf{\textit{adjective}}\ \textbf{1} Making or showing an unfair or prejudicial distinction between different categories of people or things, especially on the grounds of race, age, or sex. {\fontspec{DejaVu Sans}◇} \textit{discriminatory employment practices} \colorBulletS{SYN} prejudicial, biased, prejudiced, preferential, unfair, unjust, invidious, inequitable, weighted, one{-}sided, partisan}{}{}{}%
\par%
\entry{disenchant}{/dɪsɪnˈtʃɑːnt/}{মোহমুক্তি করা}{ \textsf{\textit{verb}}\ \textbf{1} Cause (someone) to be disappointed. {\fontspec{DejaVu Sans}◇} \textit{he may have been disenchanted by the loss of his huge following} \colorBulletS{SYN} disillusioned, disappointed, let down, fed up, dissatisfied, discontented, disabused, undeceived, set straight}{}{}{ \colorBullet{ORIGIN} Late 16th century from French désenchanter, from dés{-} (expressing reversal) + enchanter (see enchant).}%
\par%
\entry{disheveled}{/dəˈSHevəld/}{অপরিচ্ছন্ন}{ \textsf{\textit{adjective}}\ \textbf{1} (of a person's hair, clothes, or appearance) untidy; disordered. {\fontspec{DejaVu Sans}◇} \textit{a man with long disheveled hair} \colorBulletS{SYN} untidy, unkempt, scruffy, messy, in a mess, disordered, disarranged, rumpled, bedraggled}{}{}{ \colorBullet{ORIGIN} Late Middle English from obsolete dishevely, from Old French deschevele, past participle of descheveler (based on chevel ‘hair’, from Latin capillus). The original sense was ‘having the hair uncovered’; later, referring to the hair itself, ‘hanging loose’, hence ‘disordered, untidy’. Compare with unkempt.}%
\par%
\entry{dismal}{/ˈdɪzm(ə)l/}{অন্ধকারময়}{ \textsf{\textit{adjective}}\ \textbf{1} Causing a mood of gloom or depression. {\fontspec{DejaVu Sans}◇} \textit{the dismal weather made the late afternoon seem like evening} \colorBulletS{SYN} dingy, dim, dark, gloomy, sombre, dreary, drab, dull, desolate, bleak, cheerless, comfortless, depressing, grim, funereal, inhospitable, uninviting, unwelcoming}{}{The bangladeshi athletes put up dismal performances in the 5th youth commonwealth games}{ \colorBullet{ORIGIN} Late Middle English from earlier dismal (noun), denoting the two days in each month which in medieval times were believed to be unlucky, from Anglo{-}Norman French dis mal, from medieval Latin dies mali ‘evil days’.}%
\par%
\entry{dismantle}{/dɪsˈmant(ə)l/}{ইতি টেনে}{ \textsf{\textit{verb}}\ \textbf{1} Take (a machine or structure) to pieces. {\fontspec{DejaVu Sans}◇} \textit{the engines were dismantled and the bits piled into a heap} \colorBulletS{SYN} take apart, take to pieces, take to bits, pull apart, pull to pieces, deconstruct, disassemble, break up, strip, strip down}{}{}{ \colorBullet{ORIGIN} Late 16th century (in the sense ‘destroy the defensive capability of a fortification’): from Old French desmanteler, from des{-} (expressing reversal) + manteler ‘fortify’ (from Latin mantellum ‘cloak’).}%
\par%
\entry{dispel}{/dɪˈspɛl/}{দুরীভূত করা}{ \textsf{\textit{verb}}\ \textbf{1} Make (a doubt, feeling, or belief) disappear. {\fontspec{DejaVu Sans}◇} \textit{the brightness of the day did nothing to dispel Elaine's dejection} \colorBulletS{SYN} banish, eliminate, dismiss, chase away, drive away, drive off, get rid of, dissipate, disperse, scatter, disseminate}{}{}{ \colorBullet{ORIGIN} Late Middle English from Latin dispellere, from dis{-} ‘apart’ + pellere ‘to drive’.}%
\par%
\entry{disperse}{/dɪˈspəːs/}{অদৃশ্য করা}{\small{\textsf{\textit{adjective, verb}}} \\{\fontspec{DejaVu Sans}▪ }\textsf{\textit{adjective}}\\ \textbf{1} Denoting a phase dispersed in another phase, as in a colloid. {\fontspec{DejaVu Sans}◇} \textit{emulsions should be examined after storage for droplet size of the disperse phase} \\{\fontspec{DejaVu Sans}▪ }\textsf{\textit{verb}}\\ \textbf{1} Distribute or spread over a wide area. {\fontspec{DejaVu Sans}◇} \textit{storms can disperse seeds via high altitudes} \colorBulletS{SYN} scatter, disseminate, distribute, spread, broadcast, diffuse, strew, sow, sprinkle, pepper}{}{}{ \colorBullet{ORIGIN} Late Middle English from Latin dispers{-} ‘scattered’, from the verb dispergere, from dis{-} ‘widely’ + spargere ‘scatter, strew’.}%
\par%
\entry{displeasure}{/dɪsˈplɛʒə/}{অপ্রসন্নতা}{\small{\textsf{\textit{noun, verb}}} \\{\fontspec{DejaVu Sans}▪ }\textsf{\textit{noun}}\\ \textbf{1} A feeling of annoyance or disapproval. {\fontspec{DejaVu Sans}◇} \textit{he started hanging around the local pubs, much to the displeasure of his mother} \colorBulletS{SYN} annoyance, irritation, crossness, infuriation, anger, vexation, wrath, pique, chagrin, rancour, resentment, indignation, exasperation \\{\fontspec{DejaVu Sans}▪ }\textsf{\textit{verb}}\\ \textbf{1} Annoy; displease. {\fontspec{DejaVu Sans}◇} \textit{not for worlds would I do aught that might displeasure thee}}{}{}{ \colorBullet{ORIGIN} Late Middle English from Old French desplaisir (see displease), influenced by pleasure.}%
\par%
\entry{disposal}{/dɪˈspəʊz(ə)l/}{নিষ্পত্তি}{ \textsf{\textit{noun}}\ \textbf{1} The action or process of getting rid of something. {\fontspec{DejaVu Sans}◇} \textit{the disposal of radioactive waste} \colorBulletS{SYN} throwing away, getting rid of, discarding, jettisoning, ejection, scrapping, destruction \textbf{2} The sale of shares, property, or other assets. {\fontspec{DejaVu Sans}◇} \textit{the disposal of his shares in the company} \colorBulletS{SYN} distribution, handing out, giving out, giving away, allotment, allocation, donation, transfer, transference, making over, conveyance, bestowal, bequest \textbf{3} The arrangement of something. {\fontspec{DejaVu Sans}◇} \textit{she brushed her hair carefully, as if her success lay in the sleek disposal of each gleaming black thread} \colorBulletS{SYN} arrangement, arranging, ordering, positioning, placement, lining up, setting up, organization, disposition}{}{}{ \colorBullet{ORIGIN} Available for one to use whenever or however one wishes.Ready to assist the person concerned in any way they wish.}%
\par%
\entry{dispose}{/dɪˈspəʊz/}{মীমাংসা করা}{ \textsf{\textit{verb}}\ \textbf{1} Get rid of by throwing away or giving or selling to someone else. {\fontspec{DejaVu Sans}◇} \textit{the waste is disposed of in the North Sea} \colorBulletS{SYN} throw away, throw out, cast out, get rid of, do away with, discard, jettison, abandon, eject, unload \textbf{2} Incline (someone) towards a particular activity or mood. {\fontspec{DejaVu Sans}◇} \textit{prolactin, a calming hormone, is released, disposing you towards sleep} \colorBulletS{SYN} incline, encourage, persuade, predispose, make willing, make, move, prompt, lead, induce, inspire, tempt, motivate, actuate \textbf{3} Arrange in a particular position. {\fontspec{DejaVu Sans}◇} \textit{the chief disposed his attendants in a circle} \colorBulletS{SYN} arrange, order, place, put, position, orient, array, spread out, range, set up, form, organize, seat, stand}{}{}{ \colorBullet{ORIGIN} Late Middle English from Old French disposer, from Latin disponere ‘arrange’, influenced by dispositus ‘arranged’ and Old French poser ‘to place’.}%
\par%
\entry{disproportionate}{/ˌdɪsprəˈpɔːʃ(ə)nət/}{অনুপাতহীন}{ \textsf{\textit{adjective}}\ \textbf{1} Too large or too small in comparison with something else. {\fontspec{DejaVu Sans}◇} \textit{people on lower incomes spend a disproportionate amount of their income on fuel} \colorBulletS{SYN} out of proportion to, not in proportion to, not appropriate to, not commensurate with, relatively too large for, relatively too small for}{}{}{ \colorBullet{ORIGIN} Mid 16th century from dis{-} (expressing absence) + proportionate, on the pattern of French disproportionné.}%
\par%
\entry{disproportionate}{/ˌdɪsprəˈpɔːʃ(ə)neɪt/}{অনুপাতহীন}{ \textsf{\textit{verb}}\ \textbf{1} Undergo disproportionation. {\fontspec{DejaVu Sans}◇} \textit{water disproportionates to oxygen and hydrogen}}{}{}{}%
\par%
\entry{dispute}{/dɪˈspjuːt/}{বিতর্ক}{\small{\textsf{\textit{noun, verb}}} \\{\fontspec{DejaVu Sans}▪ }\textsf{\textit{noun}}\\ \textbf{1} A disagreement or argument. {\fontspec{DejaVu Sans}◇} \textit{a territorial dispute between the two countries} \colorBulletS{SYN} debate, discussion, discourse, disputation, argument, controversy, contention, disagreement, altercation, falling{-}out, quarrelling, variance, dissension, conflict, friction, strife, discord, antagonism \\{\fontspec{DejaVu Sans}▪ }\textsf{\textit{verb}}\\ \textbf{1} Argue about (something) {\fontspec{DejaVu Sans}◇} \textit{the point has been much disputed} \colorBulletS{SYN} debate, discuss, exchange views \textbf{2} Compete for; strive to win. {\fontspec{DejaVu Sans}◇} \textit{the two drivers crashed while disputing the lead}}{}{}{ \colorBullet{ORIGIN} Middle English via Old French from Latin disputare ‘to estimate’ (in late Latin ‘to dispute’), from dis{-} ‘apart’ + putare ‘reckon’.}%
\par%
\entry{disrepute}{/ˌdɪsrɪˈpjuːt/}{দুর্নাম}{ \textsf{\textit{noun}}\ \textbf{1} The state of being held in low esteem by the public. {\fontspec{DejaVu Sans}◇} \textit{one of the top clubs in the country is bringing the game into disrepute} \colorBulletS{SYN} disgrace, shame, dishonour, infamy, notoriety, ignominy, stigma, scandal, bad reputation, lack of respectability}{}{}{}%
\par%
\entry{disrupt}{/dɪsˈrʌpt/}{চূর্ণবিচূর্ণ করা}{ \textsf{\textit{verb}}\ \textbf{1} Interrupt (an event, activity, or process) by causing a disturbance or problem. {\fontspec{DejaVu Sans}◇} \textit{flooding disrupted rail services} \colorBulletS{SYN} throw into confusion, throw into disorder, throw into disarray, cause confusion in, cause turmoil in, play havoc with, derange, turn upside{-}down, make a mess of}{}{}{ \colorBullet{ORIGIN} Late Middle English from Latin disrupt{-} ‘broken apart’, from the verb disrumpere.}%
\par%
\entry{distinction}{/dɪˈstɪŋ(k)ʃ(ə)n/}{পার্থক্য}{ \textsf{\textit{noun}}\ \textbf{1} A difference or contrast between similar things or people. {\fontspec{DejaVu Sans}◇} \textit{there is a sharp distinction between domestic politics and international politics} \colorBulletS{SYN} difference, contrast, dissimilarity, dissimilitude, divergence, variance, variation \textbf{2} Excellence that sets someone or something apart from others. {\fontspec{DejaVu Sans}◇} \textit{a novelist of distinction} \colorBulletS{SYN} importance, significance, note, consequence, account}{}{}{ \colorBullet{ORIGIN} Middle English (in the sense ‘subdivision, category’): via Old French from Latin distinctio(n{-}), from the verb distinguere (see distinguish).}%
\par%
\entry{distract}{/dɪˈstrakt/}{বিভ্রান্ত করা}{ \textsf{\textit{verb}}\ \textbf{1} Prevent (someone) from concentrating on something. {\fontspec{DejaVu Sans}◇} \textit{don't allow noise to distract you from your work} \colorBulletS{SYN} disturbing, unsettling, intrusive, disconcerting, bothersome, confusing}{}{}{ \colorBullet{ORIGIN} Late Middle English (also in the sense ‘pull in different directions’): from Latin distract{-} ‘drawn apart’, from the verb distrahere, from dis{-} ‘apart’ + trahere ‘to draw, drag’.}%
\par%
\entry{distress}{/dɪˈstrɛs/}{মর্মপীড়া}{\small{\textsf{\textit{noun, verb}}} \\{\fontspec{DejaVu Sans}▪ }\textsf{\textit{noun}}\\ \textbf{1} Extreme anxiety, sorrow, or pain. {\fontspec{DejaVu Sans}◇} \textit{to his distress he saw that she was trembling} \colorBulletS{SYN} anguish, suffering, pain, agony, ache, affliction, torment, torture, discomfort, heartache, heartbreak \textbf{2} another term for distraint {\fontspec{DejaVu Sans}◇} \textit{} \\{\fontspec{DejaVu Sans}▪ }\textsf{\textit{verb}}\\ \textbf{1} Cause (someone) anxiety, sorrow, or pain. {\fontspec{DejaVu Sans}◇} \textit{I didn't mean to distress you} \colorBulletS{SYN} cause anguish to, cause suffering to, pain, upset, make miserable, make wretched \textbf{2} Give (furniture or clothing) simulated marks of age and wear. {\fontspec{DejaVu Sans}◇} \textit{the manner in which leather jackets are industrially distressed} \colorBulletS{SYN} age, season, condition, mellow, weather, simulate age in}{}{}{ \colorBullet{ORIGIN} Middle English from Old French destresce (noun), destrecier (verb), based on Latin distringere ‘stretch apart’.}%
\par%
\entry{distressing}{/dɪˈstrɛsɪŋ/}{পীড়াদায়ক}{ \textsf{\textit{adjective}}\ \textbf{1} Causing anxiety, sorrow or pain; upsetting. {\fontspec{DejaVu Sans}◇} \textit{some very distressing news} \colorBulletS{SYN} upsetting, worrying, affecting, painful, traumatic, agonizing, harrowing, tormenting}{}{}{}%
\par%
\entry{divergent}{/dʌɪˈvəːdʒ(ə)nt/}{বিপথগামী}{ \textsf{\textit{adjective}}\ \textbf{1} Tending to be different or develop in different directions. {\fontspec{DejaVu Sans}◇} \textit{divergent interpretations} \colorBulletS{SYN} differing, varying, different, dissimilar, unlike, unalike, disparate, contrasting, contrastive, antithetical \textbf{2} (of a series) increasing indefinitely as more of its terms are added. {\fontspec{DejaVu Sans}◇} \textit{}}{}{}{}%
\par%
\entry{diversity}{/dʌɪˈvəːsɪti/}{বৈচিত্র্য}{ \textsf{\textit{noun}}\ \textbf{1} The state of being diverse. {\fontspec{DejaVu Sans}◇} \textit{there was considerable diversity in the style of the reports}}{}{}{ \colorBullet{ORIGIN} Middle English from Old French diversite, from Latin diversitas, from diversus ‘diverse’, past participle of divertere ‘turn aside’ (see divert).}%
\par%
\entry{divert}{/dʌɪˈvəːt/}{সরাইয়া করা}{ \textsf{\textit{verb}}\ \textbf{1} Cause (someone or something) to change course or turn from one direction to another. {\fontspec{DejaVu Sans}◇} \textit{a scheme to divert water from the river to irrigate agricultural land} \colorBulletS{SYN} reroute, redirect, change the course of, draw away, turn aside, head off, deflect, avert, transfer, channel \textbf{2} Draw (the attention) of someone from something. {\fontspec{DejaVu Sans}◇} \textit{public relations policies are sometimes intended to divert attention away from criticism} \colorBulletS{SYN} distract, detract, sidetrack, lead away, draw away, be a distraction, put off, disturb someone's concentration}{}{}{ \colorBullet{ORIGIN} Late Middle English via French from Latin divertere, from di{-} ‘aside’ + vertere ‘to turn’.}%
\par%
\entry{divine}{/dɪˈvʌɪn/}{ঐশ্বরিক}{\small{\textsf{\textit{adjective, noun}}} \\{\fontspec{DejaVu Sans}▪ }\textsf{\textit{adjective}}\\ \textbf{1} Of or like God or a god. {\fontspec{DejaVu Sans}◇} \textit{heroes with divine powers} \colorBulletS{SYN} godly, godlike, angelic, seraphic, saintly, beatific \textbf{2} Very pleasing; delightful. {\fontspec{DejaVu Sans}◇} \textit{he had the most divine smile} \colorBulletS{SYN} lovely, handsome, beautiful, good{-}looking, prepossessing, charming, delightful, appealing, engaging, winsome, ravishing, gorgeous, bewitching, beguiling \\{\fontspec{DejaVu Sans}▪ }\textsf{\textit{noun}}\\ \textbf{1} A cleric or theologian. {\fontspec{DejaVu Sans}◇} \textit{} \colorBulletS{SYN} theologian, clergyman, member of the clergy, churchman, churchwoman, cleric, ecclesiastic, man of the cloth, man of God, holy man, holy woman, preacher, priest \textbf{2} Providence or God. {\fontspec{DejaVu Sans}◇} \textit{}}{}{}{ \colorBullet{ORIGIN} Late Middle English via Old French from Latin divinus, from divus ‘godlike’ (related to deus ‘god’).}%
\par%
\entry{divine}{/dɪˈvʌɪn/}{ঐশ্বরিক}{ \textsf{\textit{verb}}\ \textbf{1} Discover (something) by guesswork or intuition. {\fontspec{DejaVu Sans}◇} \textit{mum had divined my state of mind} \colorBulletS{SYN} guess, surmise, conjecture, suspect, suppose, assume, presume, deduce, infer, work out, theorize, hypothesize \textbf{2} Have supernatural or magical insight into (future events) {\fontspec{DejaVu Sans}◇} \textit{frauds who claimed to divine the future in chickens' entrails} \colorBulletS{SYN} foretell, predict, prophesy, forecast, foresee, prognosticate}{}{}{ \colorBullet{ORIGIN} Late Middle English from Old French deviner ‘predict’, from Latin divinare, from divinus (see divine).}%
\par%
\entry{dizzy}{/ˈdɪzi/}{হতবুদ্ধি}{\small{\textsf{\textit{adjective, verb}}} \\{\fontspec{DejaVu Sans}▪ }\textsf{\textit{adjective}}\\ \textbf{1} Having or involving a sensation of spinning around and losing one's balance. {\fontspec{DejaVu Sans}◇} \textit{Jonathan had begun to suffer dizzy spells} \colorBulletS{SYN} giddy, light{-}headed, faint, weak, weak at the knees, unsteady, shaky, wobbly, off{-}balance \\{\fontspec{DejaVu Sans}▪ }\textsf{\textit{verb}}\\ \textbf{1} Make (someone) feel unsteady, confused, or amazed. {\fontspec{DejaVu Sans}◇} \textit{the dizzying rate of change}}{}{}{ \colorBullet{ORIGIN} Old English dysig ‘foolish’, of West Germanic origin; related to Low German dusig, dösig ‘giddy’ and Old High German tusic ‘foolish, weak’.}%
\par%
\entry{do}{/duː/}{করা}{\small{\textsf{\textit{auxiliary verb, noun, verb}}} \\{\fontspec{DejaVu Sans}▪ }\textsf{\textit{auxiliary verb}}\\ \textbf{1} Used before a verb (except be, can, may, ought, shall, will) in questions and negative statements. {\fontspec{DejaVu Sans}◇} \textit{do you have any pets?} \textbf{2} Used to refer back to a verb already mentioned. {\fontspec{DejaVu Sans}◇} \textit{he looks better than he did before} \textbf{3} Used to give emphasis to a positive verb. {\fontspec{DejaVu Sans}◇} \textit{I do want to act on this} \textbf{4} Used with inversion of a subject and verb when an adverbial phrase begins a clause for emphasis. {\fontspec{DejaVu Sans}◇} \textit{only rarely did they succumb} \\{\fontspec{DejaVu Sans}▪ }\textsf{\textit{noun}}\\ \textbf{1} A party or other social event. {\fontspec{DejaVu Sans}◇} \textit{the soccer club Christmas do} \colorBulletS{SYN} party, reception, gathering, celebration, function, affair, event, social event, social occasion, social function, social \textbf{2} short for hairdo {\fontspec{DejaVu Sans}◇} \textit{a bowl{-}shaped do of perfect silky hair} \textbf{3}  {\fontspec{DejaVu Sans}◇} \textit{the air was rancid with the smell of donkey doo} \textbf{4} A swindle or hoax. {\fontspec{DejaVu Sans}◇} \textit{} \\{\fontspec{DejaVu Sans}▪ }\textsf{\textit{verb}}\\ \textbf{1} Perform (an action, the precise nature of which is often unspecified) {\fontspec{DejaVu Sans}◇} \textit{something must be done about the city's traffic} \colorBulletS{SYN} carry out, undertake, discharge, execute, perpetrate, perform, accomplish, implement, achieve, complete, finish, conclude \textbf{2} Achieve or complete. {\fontspec{DejaVu Sans}◇} \textit{} \textbf{3} Act or behave in a specified way. {\fontspec{DejaVu Sans}◇} \textit{they are free to do as they please} \colorBulletS{SYN} act, behave, conduct oneself, acquit oneself \textbf{4} Be suitable or acceptable. {\fontspec{DejaVu Sans}◇} \textit{if he's anything like you, he'll do} \colorBulletS{SYN} suffice, be adequate, be satisfactory, be acceptable, be good enough, be of use, fill the bill, fit the bill, answer the purpose, serve the purpose, meet one's needs, pass muster \textbf{5} Beat up or kill. {\fontspec{DejaVu Sans}◇} \textit{one day I'll do him} \textbf{6} Prosecute or convict. {\fontspec{DejaVu Sans}◇} \textit{we got done for conspiracy to cause GBH}}{}{Do so: তাই করো }{ \colorBullet{ORIGIN} Old English dōn, of Germanic origin; related to Dutch doen and German tun, from an Indo{-}European root shared by Greek tithēmi ‘I place’ and Latin facere ‘make, do’.}%
\par%
\entry{do}{/dəʊ/}{করা}{\small{\textsf{\textit{}}}}{}{Do so: তাই করো }{}%
\par%
\entry{dodge}{/dɒdʒ/}{লুকাচুরি}{\small{\textsf{\textit{noun, verb}}} \\{\fontspec{DejaVu Sans}▪ }\textsf{\textit{noun}}\\ \textbf{1} A sudden quick movement to avoid someone or something. {\fontspec{DejaVu Sans}◇} \textit{} \colorBulletS{SYN} dart, bolt, duck, dive, swerve, jump, leap, spring \textbf{2} The dodging of a bell in change{-}ringing. {\fontspec{DejaVu Sans}◇} \textit{} \\{\fontspec{DejaVu Sans}▪ }\textsf{\textit{verb}}\\ \textbf{1} Avoid (someone or something) by a sudden quick movement. {\fontspec{DejaVu Sans}◇} \textit{marchers had to dodge missiles thrown by loyalists} \colorBulletS{SYN} dart, bolt, duck, dive, swerve, body{-}swerve, sidestep, veer, lunge, jump, leap, spring \textbf{2} Expose (one area of a print) less than the rest during processing or enlarging. {\fontspec{DejaVu Sans}◇} \textit{} \textbf{3} (of a bell in change{-}ringing) move one place contrary to the normal sequence, and then back again in the following round. {\fontspec{DejaVu Sans}◇} \textit{}}{}{}{ \colorBullet{ORIGIN} Mid 16th century (in the senses ‘dither’ and ‘haggle’): of unknown origin.}%
\par%
\entry{doom}{/duːm/}{নিয়তি}{\small{\textsf{\textit{noun, verb}}} \\{\fontspec{DejaVu Sans}▪ }\textsf{\textit{noun}}\\ \textbf{1} Death, destruction, or some other terrible fate. {\fontspec{DejaVu Sans}◇} \textit{the aircraft was sent crashing to its doom in the water} \colorBulletS{SYN} destruction, downfall, grim fate, terrible fate, ruin, ruination, rack and ruin, catastrophe, disaster \\{\fontspec{DejaVu Sans}▪ }\textsf{\textit{verb}}\\ \textbf{1} Condemn to certain death or destruction. {\fontspec{DejaVu Sans}◇} \textit{fuel was spilling out of the damaged wing and the aircraft was doomed}}{}{}{ \colorBullet{ORIGIN} Old English dōm ‘statute, judgement’, of Germanic origin, from a base meaning ‘to put in place’; related to do.}%
\par%
\entry{douche}{/duːʃ/}{}{\small{\textsf{\textit{noun, verb}}} \\{\fontspec{DejaVu Sans}▪ }\textsf{\textit{noun}}\\ \textbf{1} A shower of water. {\fontspec{DejaVu Sans}◇} \textit{I felt better for taking a daily douche} \colorBulletS{SYN} wash, soak, dip, shower, douche, soaping, sponging, toilet \textbf{2} An obnoxious or contemptible person (typically used of a man) {\fontspec{DejaVu Sans}◇} \textit{that guy is such a douche} \\{\fontspec{DejaVu Sans}▪ }\textsf{\textit{verb}}\\ \textbf{1} Spray or shower with water. {\fontspec{DejaVu Sans}◇} \textit{she did not douche herself and the smell, at times, was off{-}putting} \colorBulletS{SYN} sprinkle, shower, spritz, spread in droplets, spatter}{}{}{ \colorBullet{ORIGIN} Mid 18th century (as a noun): via French from Italian doccia ‘conduit pipe’, from docciare ‘pour by drops’, based on Latin ductus ‘leading’ (see duct).}%
\par%
\entry{downfall}{/ˈdaʊnfɔːl/}{সম্পূর্ণ বিনাশ}{ \textsf{\textit{noun}}\ \textbf{1} A loss of power, prosperity, or status. {\fontspec{DejaVu Sans}◇} \textit{the crisis led to the downfall of the government} \colorBulletS{SYN} undoing, ruin, ruination, loss of power, loss of prosperity, loss of status \textbf{2} A heavy fall of rain or snow. {\fontspec{DejaVu Sans}◇} \textit{the wind was whipping up the downfall into deep drifts on the moor}}{}{}{}%
\par%
\entry{downpour}{/ˈdaʊnpɔː/}{প্রবল বর্ষণ}{ \textsf{\textit{noun}}\ \textbf{1} A heavy fall of rain. {\fontspec{DejaVu Sans}◇} \textit{a sudden downpour had filled the gutters and drains} \colorBulletS{SYN} rainstorm, cloudburst, torrent of rain, deluge}{}{}{}%
\par%
\entry{downward}{/ˈdaʊnwəd/}{নিম্নাভিমুখ}{\small{\textsf{\textit{adjective, adverb}}} \\{\fontspec{DejaVu Sans}▪ }\textsf{\textit{adjective}}\\ \textbf{1} Moving or leading towards a lower place or level. {\fontspec{DejaVu Sans}◇} \textit{a downward trend in inflation} \colorBulletS{SYN} descending, downhill, falling, sinking, going down, moving down, sliding, slipping, dipping, earthbound, earthward \\{\fontspec{DejaVu Sans}▪ }\textsf{\textit{adverb}}\\ \textbf{1} Towards a lower place, point, or level. {\fontspec{DejaVu Sans}◇} \textit{he was lying face downward}}{}{}{ \colorBullet{ORIGIN} Middle English shortening of Old English adūnweard.}%
\par%
\entry{dowry}{/ˈdaʊ(ə)ri/}{যৌতুক}{ \textsf{\textit{noun}}\ \textbf{1} An amount of property or money brought by a bride to her husband on their marriage. {\fontspec{DejaVu Sans}◇} \textit{Elizabeth's dowry was to be £45,000 in diamonds} \colorBulletS{SYN} marriage settlement, portion, marriage portion}{}{}{ \colorBullet{ORIGIN} Middle English (denoting a widow's life interest in her husband's estate): from Anglo{-}Norman French dowarie, from medieval Latin dotarium (see dower).}%
\par%
\entry{draft}{/drɑːft/}{খসড়া}{\small{\textsf{\textit{noun, verb}}} \\{\fontspec{DejaVu Sans}▪ }\textsf{\textit{noun}}\\ \textbf{1} A preliminary version of a piece of writing. {\fontspec{DejaVu Sans}◇} \textit{the first draft of the party's manifesto} \colorBulletS{SYN} version, edition, issue, model, mark, draft, form, impression, publication \textbf{2} A written order to pay a specified sum. {\fontspec{DejaVu Sans}◇} \textit{} \colorBulletS{SYN} cheque, order, banker's order, money order, bill of exchange, postal order \textbf{3} Compulsory recruitment for military service. {\fontspec{DejaVu Sans}◇} \textit{25 million men were subject to the draft} \textbf{4} US spelling of draught (noun) {\fontspec{DejaVu Sans}◇} \textit{} \\{\fontspec{DejaVu Sans}▪ }\textsf{\textit{verb}}\\ \textbf{1} Prepare a preliminary version of (a document) {\fontspec{DejaVu Sans}◇} \textit{I drafted a letter of resignation} \textbf{2} Select (a person or group of people) and bring them somewhere for a certain purpose. {\fontspec{DejaVu Sans}◇} \textit{riot police were drafted in to break up the blockade}}{}{}{ \colorBullet{ORIGIN} Mid 16th century phonetic spelling of draught.}%
\par%
\entry{drastic}{/ˈdrastɪk/}{প্রচণ্ড}{ \textsf{\textit{adjective}}\ \textbf{1} Likely to have a strong or far{-}reaching effect; radical and extreme. {\fontspec{DejaVu Sans}◇} \textit{a drastic reduction of staffing levels} \colorBulletS{SYN} extreme, serious, forceful, desperate, dire, radical, far{-}reaching, momentous, substantial}{}{}{ \colorBullet{ORIGIN} Late 17th century (originally applied to the effect of medicine): from Greek drastikos, from dran ‘do’.}%
\par%
\entry{drawing}{/ˈdrɔː(r)ɪŋ/}{অঙ্কন}{ \textsf{\textit{noun}}\ \textbf{1} A picture or diagram made with a pencil, pen, or crayon rather than paint. {\fontspec{DejaVu Sans}◇} \textit{a series of charcoal drawings on white paper} \colorBulletS{SYN} sketch, picture, illustration, representation, portrayal, delineation, depiction, composition, study, diagram, outline, design, plan, artist's impression \textbf{2} An instance of selecting the winner or winners in a lottery or raffle. {\fontspec{DejaVu Sans}◇} \textit{entrants need not be present at the drawing}}{}{}{}%
\par%
\entry{drawn}{/drɔːn/}{টানা}{\small{\textsf{\textit{adjective, verb}}} \\{\fontspec{DejaVu Sans}▪ }\textsf{\textit{adjective}}\\ \textbf{1} (of a person) looking strained from illness, exhaustion, anxiety, or pain. {\fontspec{DejaVu Sans}◇} \textit{Cathy was pale and drawn and she looked tired out} \colorBulletS{SYN} worn, pinched, haggard, gaunt, drained, wan, hollow{-}cheeked \\{\fontspec{DejaVu Sans}▪ }\textsf{\textit{verb}}\\ \textbf{1} past participle of draw {\fontspec{DejaVu Sans}◇} \textit{}}{}{}{}%
\par%
\entry{drool}{/druːl/}{আবোলতাবোল বকা}{\small{\textsf{\textit{noun, verb}}} \\{\fontspec{DejaVu Sans}▪ }\textsf{\textit{noun}}\\ \textbf{1} Saliva falling from the mouth. {\fontspec{DejaVu Sans}◇} \textit{a fine trickle of drool leaked from the corner of his mouth} \colorBulletS{SYN} saliva, spit, spittle, dribble, slaver, slobber \\{\fontspec{DejaVu Sans}▪ }\textsf{\textit{verb}}\\ \textbf{1} Drop saliva uncontrollably from the mouth. {\fontspec{DejaVu Sans}◇} \textit{the baby begins to drool, then to cough} \colorBulletS{SYN} salivate, dribble, slaver, slobber, drivel, water at the mouth}{}{}{ \colorBullet{ORIGIN} Early 19th century contraction of drivel.}%
\par%
\entry{dubito, ergo cogito, ergo sum}{}{I doubt, therefore i think, therefore i am}{\small{\textsf{\textit{}}}}{}{}{}%
\par%
\entry{ducks and drakes}{}{The pastime of skimming flat stones or shells along the surface of calm water}{\small{\textsf{\textit{}}}}{}{Play ducks and drakes with or make ducks and drakes of: played ducks and drakes with his money.\newline%
}{}%
\par%
\entry{due}{/djuː/}{কারণে}{\small{\textsf{\textit{adjective, adverb, noun}}} \\{\fontspec{DejaVu Sans}▪ }\textsf{\textit{adjective}}\\ \textbf{1} Expected at or planned for at a certain time. {\fontspec{DejaVu Sans}◇} \textit{the baby's due in August} \colorBulletS{SYN} expected, required, awaited, anticipated, scheduled for \textbf{2} Of the proper quality or extent. {\fontspec{DejaVu Sans}◇} \textit{driving without due care and attention} \colorBulletS{SYN} proper, right and proper, correct, rightful, fitting, suitable, appropriate, apt, adequate, sufficient, enough, ample, satisfactory, requisite \\{\fontspec{DejaVu Sans}▪ }\textsf{\textit{adverb}}\\ \textbf{1} (with reference to a point of the compass) exactly; directly. {\fontspec{DejaVu Sans}◇} \textit{we'll head due south again on the same road} \colorBulletS{SYN} directly, straight, exactly, precisely, without deviating, undeviatingly, dead, plumb, squarely \\{\fontspec{DejaVu Sans}▪ }\textsf{\textit{noun}}\\ \textbf{1} One's right; what is owed to one. {\fontspec{DejaVu Sans}◇} \textit{he thought it was his due} \colorBulletS{SYN} rightful treatment, fair treatment, deserved fate, just punishment \textbf{2} An obligatory payment; a fee. {\fontspec{DejaVu Sans}◇} \textit{he had paid trade union dues for years} \colorBulletS{SYN} fee, membership fee, subscription, charge, toll, levy}{}{}{ \colorBullet{ORIGIN} Middle English (in the sense ‘payable’): from Old French deu ‘owed’, based on Latin debitus ‘owed’, from debere ‘owe’.}%
\par%
\entry{dummy}{/ˈdʌmi/}{পুতুল}{\small{\textsf{\textit{noun, verb}}} \\{\fontspec{DejaVu Sans}▪ }\textsf{\textit{noun}}\\ \textbf{1} A model or replica of a human being. {\fontspec{DejaVu Sans}◇} \textit{a waxwork dummy} \textbf{2} An object designed to resemble and serve as a substitute for the real or usual one. {\fontspec{DejaVu Sans}◇} \textit{tests using stuffed owls and wooden dummies} \textbf{3} (chiefly in rugby and soccer) a feigned pass or kick intended to deceive an opponent. {\fontspec{DejaVu Sans}◇} \textit{} \textbf{4} A stupid person. {\fontspec{DejaVu Sans}◇} \textit{} \colorBulletS{SYN} idiot, fool, ass, halfwit, nincompoop, dunce, dolt, ignoramus, cretin, imbecile, dullard, moron, simpleton, clod \textbf{5} The declarer's partner, whose cards are exposed on the table after the opening lead and played by the declarer. {\fontspec{DejaVu Sans}◇} \textit{} \\{\fontspec{DejaVu Sans}▪ }\textsf{\textit{verb}}\\ \textbf{1} (chiefly in rugby and soccer) feign a pass or kick in order to deceive an opponent. {\fontspec{DejaVu Sans}◇} \textit{Blanco dummied past a static defence} \textbf{2} Create a mock{-}up of (a book, document, etc.) {\fontspec{DejaVu Sans}◇} \textit{officials dummied up a set of photos}}{}{}{ \colorBullet{ORIGIN} Late 16th century from dumb+ {-}y. The original sense was ‘a person who cannot speak’, then ‘an imaginary fourth player in whist’ (mid 18th century), whence ‘a substitute for the real thing’ and ‘a model of a human being’ (mid 19th century).}%
\par%
\entry{dupe}{/djuːp/}{প্রতারিত ব্যক্তি}{\small{\textsf{\textit{noun, verb}}} \\{\fontspec{DejaVu Sans}▪ }\textsf{\textit{noun}}\\ \textbf{1} A victim of deception. {\fontspec{DejaVu Sans}◇} \textit{men who were simply the dupes of their unscrupulous leaders} \colorBulletS{SYN} victim, gull, pawn, puppet, instrument \\{\fontspec{DejaVu Sans}▪ }\textsf{\textit{verb}}\\ \textbf{1} Deceive; trick. {\fontspec{DejaVu Sans}◇} \textit{the newspaper was duped into publishing an untrue story} \colorBulletS{SYN} deceive, trick, hoodwink, hoax, swindle, defraud, cheat, double{-}cross, gull, mislead, take in, fool, delude, misguide, lead on, inveigle, seduce, ensnare, entrap, beguile}{}{}{ \colorBullet{ORIGIN} Late 17th century from dialect French dupe ‘hoopoe’, from the bird's supposedly stupid appearance.}%
\par%
\entry{dupe}{/djuːp/}{প্রতারিত ব্যক্তি}{\small{\textsf{\textit{}}}}{}{}{}%
\par%
\entry{duress}{/djʊ(ə)ˈrɛs/}{জবরদস্তি}{ \textsf{\textit{noun}}\ \textbf{1} Threats, violence, constraints, or other action used to coerce someone into doing something against their will or better judgement. {\fontspec{DejaVu Sans}◇} \textit{confessions extracted under duress} \colorBulletS{SYN} coercion, compulsion, force, pressure, pressurization, intimidation, threats, constraint, enforcement, exaction}{}{}{ \colorBullet{ORIGIN} Middle English (in the sense ‘harshness, severity, cruel treatment’): via Old French from Latin duritia, from durus ‘hard’.}%
\par%
\entry{dust}{/dʌst/}{ঝাড়া}{\small{\textsf{\textit{noun, verb}}} \\{\fontspec{DejaVu Sans}▪ }\textsf{\textit{noun}}\\ \textbf{1} Fine, dry powder consisting of tiny particles of earth or waste matter lying on the ground or on surfaces or carried in the air. {\fontspec{DejaVu Sans}◇} \textit{the car sent up clouds of dust} \colorBulletS{SYN} fine powder, fine particles \textbf{2} An act of dusting. {\fontspec{DejaVu Sans}◇} \textit{a quick dust, to get rid of the cobwebs} \colorBulletS{SYN} clean, sweep, wipe, dust, mop \\{\fontspec{DejaVu Sans}▪ }\textsf{\textit{verb}}\\ \textbf{1} Remove the dust or dirt from the surface of (something) by wiping or brushing it. {\fontspec{DejaVu Sans}◇} \textit{I broke the vase I had been dusting} \colorBulletS{SYN} wipe, clean, buff, brush, sweep, mop \textbf{2} Cover lightly with a powdered substance. {\fontspec{DejaVu Sans}◇} \textit{roll out on a surface dusted with icing sugar} \colorBulletS{SYN} sprinkle, scatter, powder, dredge, sift, spray, cover, spread, strew \textbf{3} Beat up or kill someone. {\fontspec{DejaVu Sans}◇} \textit{the officers dusted him up a little bit}}{}{}{ \colorBullet{ORIGIN} Old English dūst, of Germanic origin; related to Dutch duist ‘chaff’.}%
\par%
\entry{duty{-}free}{/ˌdjuːtɪˈfriː/}{শুল্কমুক্ত}{\small{\textsf{\textit{adjective \& adverb, noun}}} \\{\fontspec{DejaVu Sans}▪ }\textsf{\textit{adjective \& adverb}}\\ \textbf{1} Exempt from payment of duty. {\fontspec{DejaVu Sans}◇} \textit{the permitted number of duty{-}free goods} \\{\fontspec{DejaVu Sans}▪ }\textsf{\textit{noun}}\\ \textbf{1} Goods that are exempt from payment of duty. {\fontspec{DejaVu Sans}◇} \textit{a bag of duty{-}free}}{}{}{}%
\par%
\entry{dweller}{/ˈdwɛlə/}{অধিবাসী}{ \textsf{\textit{noun}}\ \textbf{1} A person or animal that lives in or at a specified place. {\fontspec{DejaVu Sans}◇} \textit{city dwellers}}{}{}{}%
\par%
\entry{dwindle}{/ˈdwɪnd(ə)l/}{ক্ষীণ হত্তয়া}{ \textsf{\textit{verb}}\ \textbf{1} Diminish gradually in size, amount, or strength. {\fontspec{DejaVu Sans}◇} \textit{traffic has dwindled to a trickle} \colorBulletS{SYN} diminish, decrease, reduce, get smaller, become smaller, grow smaller, become less, grow less, lessen, wane, contract, shrink, fall off, taper off, tail off, drop, fall, go down, sink, slump, plummet}{}{}{ \colorBullet{ORIGIN} Late 16th century frequentative of Scots and dialect dwine ‘fade away’, from Old English dwīnan, of Germanic origin; related to Middle Dutch dwīnen and Old Norse dvína.}%
\par%
\entry{dysfunctional}{/dɪsˈfʌŋkʃənl/}{ক্রিয়াহীন}{ \textsf{\textit{adjective}}\ \textbf{1} Not operating normally or properly. {\fontspec{DejaVu Sans}◇} \textit{the telephones are dysfunctional} \colorBulletS{SYN} troubled, distressed, unsettled, upset, distraught}{}{}{}%
\par%
\end{multicols}%
\pagebreak%
\section*{E}%
\begin{multicols}{2}%
\entry{earmark}{/ˈɪəmɑːk/}{পরিচায়ক চিহ্ন; নির্দিষ্ট}{\small{\textsf{\textit{noun, verb}}} \\{\fontspec{DejaVu Sans}▪ }\textsf{\textit{noun}}\\ \textbf{1} A characteristic or identifying feature. {\fontspec{DejaVu Sans}◇} \textit{this car has all the earmarks of a classic} \colorBulletS{SYN} characteristic, attribute, feature, quality, essential quality, property, mark, trademark, hallmark \textbf{2} A congressional directive that funds should be spent on a specific project. {\fontspec{DejaVu Sans}◇} \textit{} \textbf{3} A mark on the ear of a domesticated animal indicating ownership or identity. {\fontspec{DejaVu Sans}◇} \textit{} \\{\fontspec{DejaVu Sans}▪ }\textsf{\textit{verb}}\\ \textbf{1} Designate (funds or resources) for a particular purpose. {\fontspec{DejaVu Sans}◇} \textit{the cash had been earmarked for a big expansion of the programme} \colorBulletS{SYN} set aside, lay aside, set apart, keep back, appropriate, reserve, keep \textbf{2} Mark the ear of (a domesticated animal) as a sign of ownership or identity. {\fontspec{DejaVu Sans}◇} \textit{Condition scoring is also useful for earmarking cattle as they come close to finish as sometimes farmers who are looking at the same cattle each day can be unaware of the degree of finish achieved.}}{}{}{}%
\par%
\entry{earthy}{/ˈəːθi/}{পার্থিব}{ \textsf{\textit{adjective}}\ \textbf{1} Resembling or suggestive of earth or soil. {\fontspec{DejaVu Sans}◇} \textit{an earthy smell} \colorBulletS{SYN} soil{-}like, dirtlike \textbf{2} (of a person or their language) direct and uninhibited, especially about sexual subjects or bodily functions. {\fontspec{DejaVu Sans}◇} \textit{their good{-}natured vulgarity and earthy humour} \colorBulletS{SYN} bawdy, ribald, off colour, racy, rude, vulgar, lewd, crude, foul, coarse, uncouth, rough, dirty, filthy, smutty, unseemly, indelicate, indecent, indecorous, obscene}{}{}{}%
\par%
\entry{ease}{/iːz/}{আরাম}{\small{\textsf{\textit{noun, verb}}} \\{\fontspec{DejaVu Sans}▪ }\textsf{\textit{noun}}\\ \textbf{1} Absence of difficulty or effort. {\fontspec{DejaVu Sans}◇} \textit{she gave up smoking with ease} \colorBulletS{SYN} effortlessness, no difficulty, no trouble, no bother, facility, facileness, simplicity \\{\fontspec{DejaVu Sans}▪ }\textsf{\textit{verb}}\\ \textbf{1} Make (something unpleasant or intense) less serious or severe. {\fontspec{DejaVu Sans}◇} \textit{a huge road{-}building programme to ease congestion} \colorBulletS{SYN} relieve, alleviate, mitigate, assuage, allay, soothe, soften, palliate, ameliorate, mollify, moderate, tone down, blunt, dull, deaden, numb, take the edge off \textbf{2} Move carefully or gradually. {\fontspec{DejaVu Sans}◇} \textit{I eased down the slope with care} \colorBulletS{SYN} move slowly, ease, inch, edge, move, manoeuvre, steer, slip, squeeze, slide}{}{}{ \colorBullet{ORIGIN} Middle English from Old French aise, based on Latin adjacens ‘lying close by’, present participle of adjacere. The verb is originally from Old French aisier, from the phrase a aise ‘at ease’; in later use from the noun.}%
\par%
\entry{eatery}{/ˈiːtəri/}{খাবারের দোকান}{ \textsf{\textit{noun}}\ \textbf{1} A restaurant or cafe. {\fontspec{DejaVu Sans}◇} \textit{}}{}{}{}%
\par%
\entry{eavesdrop}{/ˈiːvzdrɒp/}{আড়ি}{ \textsf{\textit{verb}}\ \textbf{1} Secretly listen to a conversation. {\fontspec{DejaVu Sans}◇} \textit{my father eavesdropped on my phone calls} \colorBulletS{SYN} listen in, spy, intrude}{}{}{ \colorBullet{ORIGIN} Early 17th century back{-}formation from eavesdropper(late Middle English)‘a person who listens from under the eaves’, from the obsolete noun eavesdrop ‘the ground on to which water drips from the eaves’, probably from Old Norse upsardropi, from ups ‘eaves’ + dropi ‘a drop’.}%
\par%
\entry{echo}{/ˈɛkəʊ/}{প্রতিধ্বনি}{\small{\textsf{\textit{noun, verb}}} \\{\fontspec{DejaVu Sans}▪ }\textsf{\textit{noun}}\\ \textbf{1} A sound or sounds caused by the reflection of sound waves from a surface back to the listener. {\fontspec{DejaVu Sans}◇} \textit{the walls threw back the echoes of his footsteps} \colorBulletS{SYN} reverberation, reverberating, reflection, resounding, ringing, repetition, repeat, reiteration, answer \textbf{2} A close parallel to an idea, feeling, or event. {\fontspec{DejaVu Sans}◇} \textit{his love for her found an echo in her own feelings} \colorBulletS{SYN} duplicate, copy, replica, facsimile, reproduction, imitation, close likeness, exact likeness, mirror image, twin, double, clone, match, mate, fellow, counterpart, parallel \textbf{3} A person who slavishly repeats the words or opinions of another. {\fontspec{DejaVu Sans}◇} \textit{Clarendon, whom they reckoned the faithful echo of their master's intentions} \textbf{4} A play by a defender of a higher card in a suit followed by a lower one in a subsequent trick, used as a signal to request a further lead of that suit by their partner. {\fontspec{DejaVu Sans}◇} \textit{} \textbf{5} A code word representing the letter E, used in radio communication. {\fontspec{DejaVu Sans}◇} \textit{} \textbf{6} Used in names of newspapers. {\fontspec{DejaVu Sans}◇} \textit{the South Wales Echo} \\{\fontspec{DejaVu Sans}▪ }\textsf{\textit{verb}}\\ \textbf{1} (of a sound) be repeated or reverberate after the original sound has stopped. {\fontspec{DejaVu Sans}◇} \textit{their footsteps echoed on the metal catwalks} \textbf{2} (of an object or event) be reminiscent of or have shared characteristics with. {\fontspec{DejaVu Sans}◇} \textit{a blue suit that echoed the colour of her eyes} \textbf{3} Send a copy of (an input signal or character) back to its source or to a screen for display. {\fontspec{DejaVu Sans}◇} \textit{for security reasons, the password will not be echoed to the screen} \textbf{4} (of a defender) play a higher card followed by a lower one in the same suit, as a signal to request one's partner to lead that suit. {\fontspec{DejaVu Sans}◇} \textit{}}{}{}{ \colorBullet{ORIGIN} Middle English from Old French or Latin, from Greek ēkhō, related to ēkhē ‘a sound’.}%
\par%
\entry{Echo}{/ˈɛkəʊ/}{প্রতিধ্বনি}{ \textsf{\textit{proper noun}}\ \textbf{1} A nymph deprived of speech by Hera in order to stop her chatter, and left able only to repeat what others had said. {\fontspec{DejaVu Sans}◇} \textit{}}{}{}{}%
\par%
\entry{effect}{/ɪˈfɛkt/}{প্রভাব}{\small{\textsf{\textit{noun, verb}}} \\{\fontspec{DejaVu Sans}▪ }\textsf{\textit{noun}}\\ \textbf{1} A change which is a result or consequence of an action or other cause. {\fontspec{DejaVu Sans}◇} \textit{the lethal effects of hard drugs} \colorBulletS{SYN} affect, influence, exert influence on, act on, work on, condition, touch, interact with, have an impact on, impact on, take hold of, attack, infect, strike, strike at, hit \textbf{2} The lighting, sound, or scenery used in a play, film, or broadcast. {\fontspec{DejaVu Sans}◇} \textit{the production relied too much on spectacular effects} \textbf{3} Personal belongings. {\fontspec{DejaVu Sans}◇} \textit{the insurance covers personal effects} \colorBulletS{SYN} belongings, possessions, personal possessions, personal effects, goods, worldly goods, chattels, goods and chattels, accoutrements, appurtenances \\{\fontspec{DejaVu Sans}▪ }\textsf{\textit{verb}}\\ \textbf{1} Cause (something) to happen; bring about. {\fontspec{DejaVu Sans}◇} \textit{the prime minister effected many policy changes} \colorBulletS{SYN} achieve, accomplish, carry out, succeed in, realize, attain, manage, bring off, carry off, carry through, execute, conduct, fix, engineer, perform, do, perpetrate, discharge, fulfil, complete, finish, consummate, conclude}{}{}{ \colorBullet{ORIGIN} Late Middle English from Old French, or from Latin effectus, from efficere ‘accomplish’, from ex{-} ‘out, thoroughly’ + facere ‘do, make’. effect (sense 3 of the noun), ‘personal belongings’, arose from the obsolete sense ‘something acquired on completion of an action’.}%
\par%
\entry{efficiency}{/ɪˈfɪʃ(ə)nsi/}{দক্ষতা}{ \textsf{\textit{noun}}\ \textbf{1} The state or quality of being efficient. {\fontspec{DejaVu Sans}◇} \textit{greater energy efficiency} \colorBulletS{SYN} organization, order, orderliness, planning, regulation, logicality, coherence, productivity, effectiveness, cost{-}effectiveness}{}{}{ \colorBullet{ORIGIN} Late 16th century (in the sense ‘the fact of being an efficient cause’): from Latin efficientia, from efficere ‘accomplish’ (see effect).}%
\par%
\entry{effigy}{/ˈɛfɪdʒi/}{প্রতিকৃতি}{ \textsf{\textit{noun}}\ \textbf{1} A sculpture or model of a person. {\fontspec{DejaVu Sans}◇} \textit{a tomb effigy of Eleanor of Aquitaine} \colorBulletS{SYN} statue, statuette, carving, sculpture, graven image, model, dummy, figure, figurine, guy}{}{}{ \colorBullet{ORIGIN} Mid 16th century from Latin effigies, from effingere ‘to fashion (artistically)’, from ex{-} ‘out’ + fingere ‘to shape’.}%
\par%
\entry{electrocution}{/ɪlɛktrəˈkjuːʃ(ə)n/}{বিদ্যুৎপৃষ্ট; তড়িতাহত}{ \textsf{\textit{noun}}\ \textbf{1} The injury or killing of someone by electric shock. {\fontspec{DejaVu Sans}◇} \textit{they switched off the power supply to avoid any risk of electrocution}}{}{Five die from electrocution in panchagarh}{}%
\par%
\entry{elude}{/ɪˈl(j)uːd/}{কৌশলে এড়ান; পালান }{ \textsf{\textit{verb}}\ \textbf{1} Escape from or avoid (a danger, enemy, or pursuer), typically in a skilful or cunning way. {\fontspec{DejaVu Sans}◇} \textit{he tried to elude the security men by sneaking through a back door} \colorBulletS{SYN} evade, avoid, get away from, dodge, flee, escape, escape from, run from, run away from \textbf{2} (of an achievement or something desired) fail to be attained by (someone) {\fontspec{DejaVu Sans}◇} \textit{sleep still eluded her}}{}{}{ \colorBullet{ORIGIN} Mid 16th century (in the sense ‘delude, baffle’): from Latin eludere, from e{-} (variant of ex{-}) ‘out, away from’ + ludere ‘to play’.}%
\par%
\entry{elusive}{/ɪˈluːsɪv/}{অধরা}{ \textsf{\textit{adjective}}\ \textbf{1} Difficult to find, catch, or achieve. {\fontspec{DejaVu Sans}◇} \textit{success will become ever more elusive} \colorBulletS{SYN} difficult to catch, difficult to find, difficult to track down}{}{}{ \colorBullet{ORIGIN} Early 18th century from Latin elus{-} ‘eluded’ (from the verb eludere) + {-}ive.}%
\par%
\entry{embankment}{/ɪmˈbaŋkm(ə)nt/}{বাঁধ}{ \textsf{\textit{noun}}\ \textbf{1} A wall or bank of earth or stone built to prevent a river flooding an area. {\fontspec{DejaVu Sans}◇} \textit{Chelsea Embankment}}{}{}{}%
\par%
\entry{embarrassing}{/ɪmˈbarəsɪŋ/}{হতবুদ্ধিকর}{ \textsf{\textit{adjective}}\ \textbf{1} Causing embarrassment. {\fontspec{DejaVu Sans}◇} \textit{an embarrassing muddle} \colorBulletS{SYN} shaming, shameful, humiliating, mortifying, demeaning, degrading, ignominious}{}{}{}%
\par%
\entry{emerge}{/ɪˈməːdʒ/}{উত্থান করা}{ \textsf{\textit{verb}}\ \textbf{1} Move out of or away from something and become visible. {\fontspec{DejaVu Sans}◇} \textit{black ravens emerged from the fog} \colorBulletS{SYN} come out, appear, come into view, become visible, make an appearance \textbf{2} Become apparent or prominent. {\fontspec{DejaVu Sans}◇} \textit{United have emerged as the bookies' clear favourite} \colorBulletS{SYN} become known, become apparent, become evident, be revealed, come to light, come out, transpire, come to the fore, enter the picture, unfold, turn out, prove to be the case \textbf{3} Recover from or survive a difficult situation. {\fontspec{DejaVu Sans}◇} \textit{the economy has started to emerge from recession}}{}{}{ \colorBullet{ORIGIN} Late 16th century (in the sense ‘become known, come to light’): from Latin emergere, from e{-} (variant of ex{-}) ‘out, forth’ + mergere ‘to dip’.}%
\par%
\entry{emeritus}{/ɪˈmɛrɪtəs/}{এমেরিটাস}{ \textsf{\textit{adjective}}\ \textbf{1} (of the former holder of an office, especially a university professor) having retired but allowed to retain their title as an honour. {\fontspec{DejaVu Sans}◇} \textit{emeritus professor of microbiology} \colorBulletS{SYN} former, ex{-}, emeritus, past, in retirement, pensioned, pensioned off}{}{}{ \colorBullet{ORIGIN} Mid 18th century from Latin, past participle of emereri ‘earn one's discharge by service’, from e{-} (variant of ex{-}) ‘out of, from’ + mereri ‘earn’.}%
\par%
\entry{eminent}{/ˈɛmɪnənt/}{বিশিষ্ট}{ \textsf{\textit{adjective}}\ \textbf{1} (of a person) famous and respected within a particular sphere. {\fontspec{DejaVu Sans}◇} \textit{one of the world's most eminent statisticians} \colorBulletS{SYN} illustrious, distinguished, renowned, esteemed, pre{-}eminent, notable, noteworthy, great, prestigious, important, significant, influential, outstanding, noted, of note \textbf{2} (of a positive quality) present to a notable degree. {\fontspec{DejaVu Sans}◇} \textit{the book's scholarship and eminent readability} \colorBulletS{SYN} obvious, clear, conspicuous, marked, singular, signal, outstanding}{}{}{ \colorBullet{ORIGIN} Late Middle English from Latin eminent{-} ‘jutting, projecting’, from the verb eminere.}%
\par%
\entry{emphasis}{/ˈɛmfəsɪs/}{জোর}{ \textsf{\textit{noun}}\ \textbf{1} Special importance, value, or prominence given to something. {\fontspec{DejaVu Sans}◇} \textit{they placed great emphasis on the individual's freedom} \colorBulletS{SYN} prominence, importance, significance \textbf{2} Stress given to a word or words when speaking to indicate particular importance. {\fontspec{DejaVu Sans}◇} \textit{inflection and emphasis can change the meaning of what is said} \colorBulletS{SYN} stress, accent, accentuation, weight, force, prominence}{}{}{ \colorBullet{ORIGIN} Late 16th century via Latin from Greek, originally ‘appearance, show’, later denoting a figure of speech in which more is implied than is said (the original sense in English), from emphainein ‘exhibit’, from em{-} ‘in, within’ + phainein ‘to show’.}%
\par%
\entry{emphatically}{/ɪmˈfatɪkli/}{সজোরে}{ \textsf{\textit{adverb}}\ \textbf{1} In a forceful way. {\fontspec{DejaVu Sans}◇} \textit{she closed the door behind her emphatically} \colorBulletS{SYN} vehemently, emphatically, fiercely, forcefully, sharply, bitterly, severely}{}{}{}%
\par%
\entry{en route}{}{On or along the way }{\small{\textsf{\textit{}}}}{}{1. He reads en route 2. Arrived early despite en route delays}{}%
\par%
\entry{enchanting}{/ɪnˈtʃɑːntɪŋ/}{আকর্ষণীয়}{ \textsf{\textit{adjective}}\ \textbf{1} Delightfully charming or attractive. {\fontspec{DejaVu Sans}◇} \textit{enchanting views} \colorBulletS{SYN} captivating, charming, delightful, attractive, appealing, engaging, winning, dazzling, bewitching, beguiling, alluring, tantalizing, seductive, ravishing, disarming, irresistible, spellbinding, entrancing, enthralling, fetching, dreamy}{}{}{}%
\par%
\entry{encompass}{/ɪnˈkʌmpəs/}{পরিবেষ্টন করা}{ \textsf{\textit{verb}}\ \textbf{1} Surround and have or hold within. {\fontspec{DejaVu Sans}◇} \textit{this area of London encompasses Piccadilly to the north and St James's Park to the south} \colorBulletS{SYN} surround, enclose, ring, encircle, circumscribe, skirt, bound, border, fringe \textbf{2} Cause to take place. {\fontspec{DejaVu Sans}◇} \textit{an act designed to encompass the death of the king}}{}{}{}%
\par%
\entry{endeavour}{/ɪnˈdɛvə/}{চেষ্টা}{\small{\textsf{\textit{noun, verb}}} \\{\fontspec{DejaVu Sans}▪ }\textsf{\textit{noun}}\\ \textbf{1} An attempt to achieve a goal. {\fontspec{DejaVu Sans}◇} \textit{an endeavour to reduce serious injury} \colorBulletS{SYN} attempt, try, bid, effort, trial, venture \\{\fontspec{DejaVu Sans}▪ }\textsf{\textit{verb}}\\ \textbf{1} Try hard to do or achieve something. {\fontspec{DejaVu Sans}◇} \textit{he is endeavouring to help the Third World} \colorBulletS{SYN} try, attempt, venture, undertake, aspire, aim, seek, set out}{}{}{ \colorBullet{ORIGIN} Late Middle English (in the sense ‘exert oneself’): from the phrase put oneself in devoir ‘do one's utmost’ (see devoir).}%
\par%
\entry{enhance}{/ɪnˈhɑːns/}{বাড়ান}{ \textsf{\textit{verb}}\ \textbf{1} Intensify, increase, or further improve the quality, value, or extent of. {\fontspec{DejaVu Sans}◇} \textit{his refusal does nothing to enhance his reputation} \colorBulletS{SYN} increase, add to, intensify, magnify, amplify, inflate, strengthen, build up, supplement, augment, boost, upgrade, raise, lift, escalate, elevate, exalt, aggrandize, swell}{}{}{ \colorBullet{ORIGIN} Middle English (formerly also as inhance): from Anglo{-}Norman French enhauncer, based on Latin in{-} (expressing intensive force) + altus ‘high’. The word originally meant ‘elevate’ (literally and figuratively), later ‘exaggerate, make appear greater’, also ‘raise the value or price of something’. Current senses date from the early 16th century.}%
\par%
\entry{enormous}{/ɪˈnɔːməs/}{প্রচুর}{ \textsf{\textit{adjective}}\ \textbf{1} Very large in size, quantity, or extent. {\fontspec{DejaVu Sans}◇} \textit{enormous sums of money} \colorBulletS{SYN} huge, vast, extensive, expansive, broad, wide}{}{}{ \colorBullet{ORIGIN} Mid 16th century from Latin enormis ‘unusual, huge’ (see enormity) + {-}ous.}%
\par%
\entry{enquiry}{}{অনুসন্ধান}{\small{\textsf{\textit{}}}}{}{}{}%
\par%
\entry{enroll}{/inˈrōl/}{নথিভুক্ত করা}{ \textsf{\textit{intransitive verb}}\ \textbf{1} Officially register as a member of an institution or a student on a course. {\fontspec{DejaVu Sans}◇} \textit{he enrolled in drama school} \colorBulletS{SYN} register, sign on, sign up, apply, volunteer, put one's name down, matriculate}{}{}{ \colorBullet{ORIGIN} Late Middle English (formerly also as inroll): from Old French enroller, from en{-} ‘in’ + rolle ‘a roll’ (names being originally written on a roll of parchment).}%
\par%
\entry{enrollment}{/inˈrōlmənt/}{নিয়োগ; ভর্তি}{ \textsf{\textit{noun}}\ \textbf{1} The action of enrolling or being enrolled. {\fontspec{DejaVu Sans}◇} \textit{the amount due must be paid on enrollment in October} \colorBulletS{SYN} employment, appointment, work, job, day job, post, situation}{}{1. The public universities in the country enroll students once a year. 2. The gross enrollment rose to nearly 10 per cent during the last 11 years against the backdrop of stagnant primary school enrollment for almost 30 years, reports bss.}{}%
\par%
\entry{enthusiast}{/ɪnˈθjuːzɪast/}{কৌতূহলী ব্যক্তি}{ \textsf{\textit{noun}}\ \textbf{1} A person who is very interested in a particular activity or subject. {\fontspec{DejaVu Sans}◇} \textit{a sports car enthusiast} \colorBulletS{SYN} fan, fanatic, devotee, aficionado, addict, lover, admirer, supporter, follower \textbf{2} A person of intense and visionary Christian views. {\fontspec{DejaVu Sans}◇} \textit{}}{}{}{ \colorBullet{ORIGIN} Early 17th century (denoting a person believing that he or she is divinely inspired): from French enthousiaste or ecclesiastical Latin enthusiastes ‘member of a heretical sect’, from Greek enthousiastēs ‘person inspired by a god’, from the adjective enthous (see enthusiasm).}%
\par%
\entry{envoy}{/ˈɛnvɔɪ/}{দূত}{ \textsf{\textit{noun}}\ \textbf{1} A messenger or representative, especially one on a diplomatic mission. {\fontspec{DejaVu Sans}◇} \textit{the UN special envoy to Yugoslavia} \colorBulletS{SYN} representative, delegate, deputy, agent, intermediary, mediator, negotiator, proxy, surrogate, liaison, broker, accredited messenger, courier, spokesperson, spokesman, spokeswoman, mouthpiece, stand{-}in \textbf{2} A minister plenipotentiary, ranking below ambassador and above chargé d'affaires. {\fontspec{DejaVu Sans}◇} \textit{} \colorBulletS{SYN} ambassador, emissary, diplomat, legate, consul, attaché, chargé d'affaires, plenipotentiary}{}{}{ \colorBullet{ORIGIN} Mid 17th century from French envoyé, past participle of envoyer ‘send’, from en voie ‘on the way’, based on Latin via ‘way’.}%
\par%
\entry{envy}{/ˈɛnvi/}{দ্বেষ; ঈর্ষা}{\small{\textsf{\textit{noun, verb}}} \\{\fontspec{DejaVu Sans}▪ }\textsf{\textit{noun}}\\ \textbf{1} A feeling of discontented or resentful longing aroused by someone else's possessions, qualities, or luck. {\fontspec{DejaVu Sans}◇} \textit{she felt a twinge of envy for the people on board} \colorBulletS{SYN} jealousy, enviousness, covetousness, desire \\{\fontspec{DejaVu Sans}▪ }\textsf{\textit{verb}}\\ \textbf{1} Desire to have a quality, possession, or other desirable thing belonging to (someone else) {\fontspec{DejaVu Sans}◇} \textit{he envied people who did not have to work at the weekends} \colorBulletS{SYN} be envious of, be jealous of}{}{}{ \colorBullet{ORIGIN} Middle English (also in the sense ‘hostility, enmity’): from Old French envie (noun), envier (verb), from Latin invidia, from invidere ‘regard maliciously, grudge’, from in{-} ‘into’ + videre ‘to see’.}%
\par%
\entry{epidemic}{/ɛpɪˈdɛmɪk/}{মহামারী}{\small{\textsf{\textit{adjective, noun}}} \\{\fontspec{DejaVu Sans}▪ }\textsf{\textit{adjective}}\\ \textbf{1} Of the nature of an epidemic. {\fontspec{DejaVu Sans}◇} \textit{shoplifting has reached epidemic proportions} \colorBulletS{SYN} rife, rampant, widespread, wide{-}ranging, extensive, sweeping, penetrating, pervading \\{\fontspec{DejaVu Sans}▪ }\textsf{\textit{noun}}\\ \textbf{1} A widespread occurrence of an infectious disease in a community at a particular time. {\fontspec{DejaVu Sans}◇} \textit{a flu epidemic} \colorBulletS{SYN} outbreak, plague, scourge, infestation}{}{}{ \colorBullet{ORIGIN} Early 17th century (as an adjective): from French épidémique, from épidémie, via late Latin from Greek epidēmia ‘prevalence of disease’, from epidēmios ‘prevalent’, from epi ‘upon’ + dēmos ‘the people’.}%
\par%
\entry{eradication}{/ɪˌradɪˈkeɪʃ(ə)n/}{নির্মূল; উচ্ছেদ}{ \textsf{\textit{noun}}\ \textbf{1} The complete destruction of something. {\fontspec{DejaVu Sans}◇} \textit{the eradication of poverty} \colorBulletS{SYN} elimination, removal, suppression}{}{Mosquito eradication programme: মশা নির্মূল প্রোগ্রাম}{}%
\par%
\entry{ergo}{/ˈəːɡəʊ/}{অতএব}{ \textsf{\textit{adverb}}\ \textbf{1} Therefore. {\fontspec{DejaVu Sans}◇} \textit{she was the sole beneficiary of the will, ergo the prime suspect} \colorBulletS{SYN} therefore, consequently, so, as a result, as a consequence, hence, thus, accordingly, for that reason, that being so, this being so, that being the case, this being the case, on that account, on this account}{}{}{ \colorBullet{ORIGIN} Latin.}%
\par%
\entry{erode}{/ɪˈrəʊd/}{ক্ষয় করা}{ \textsf{\textit{verb}}\ \textbf{1} (of wind, water, or other natural agents) gradually wear away (soil, rock, or land) {\fontspec{DejaVu Sans}◇} \textit{the cliffs on this coast have been eroded by the sea} \colorBulletS{SYN} erode, abrade, scour, scratch, scrape, rasp, rub away, rub down, grind away, fret, waste away, wash away, crumble, crumble away, wear down}{}{}{ \colorBullet{ORIGIN} Early 17th century from French éroder or Latin erodere, from e{-} (variant of ex{-}) ‘out, away’ + rodere ‘gnaw’.}%
\par%
\entry{erosion}{/ɪˈrəʊʒ(ə)n/}{ক্ষয়}{ \textsf{\textit{noun}}\ \textbf{1} The process of eroding or being eroded by wind, water, or other natural agents. {\fontspec{DejaVu Sans}◇} \textit{the problem of soil erosion}}{}{River erosion: নদী ভাঙন}{ \colorBullet{ORIGIN} Mid 16th century via French from Latin erosio(n{-}), from erodere ‘wear or gnaw away’ (see erode).}%
\par%
\entry{errant}{/ˈɛr(ə)nt/}{ভ্রমণরত}{ \textsf{\textit{adjective}}\ \textbf{1} Erring or straying from the accepted course or standards. {\fontspec{DejaVu Sans}◇} \textit{an errant husband coming back from a night on the tiles} \colorBulletS{SYN} offending, guilty, culpable, misbehaving, delinquent, lawless, lawbreaking, criminal, transgressing, aberrant, deviant, erring, sinning \textbf{2} Travelling in search of adventure. {\fontspec{DejaVu Sans}◇} \textit{that same lady errant} \colorBulletS{SYN} travelling, wandering, itinerant, journeying, rambling, roaming, roving, drifting, floating, wayfaring, voyaging, touring}{}{}{ \colorBullet{ORIGIN} Middle English (in errant (sense 2)): errant (sense 1) from Latin errant{-} ‘erring’, from the verb errare; errant (sense 2) from Old French errant ‘travelling’, present participle of errer, from late Latin iterare ‘go on a journey’, from iter ‘journey’. Compare with arrant.}%
\par%
\entry{escalate}{/ˈɛskəleɪt/}{ধাপে ধাপে বৃদ্ধি করা}{ \textsf{\textit{verb}}\ \textbf{1} Increase rapidly. {\fontspec{DejaVu Sans}◇} \textit{the price of tickets escalated} \colorBulletS{SYN} increase rapidly, soar, rocket, shoot up, mount, surge, spiral, grow rapidly, rise rapidly, climb, go up}{}{}{ \colorBullet{ORIGIN} 1920s (in the sense ‘travel on an escalator’): back{-}formation from escalator.}%
\par%
\entry{escalation}{/ɛskəˈleɪʃ(ə)n/}{তীব্রতাবৃদ্ধি}{ \textsf{\textit{noun}}\ \textbf{1} A rapid increase; a rise. {\fontspec{DejaVu Sans}◇} \textit{cost escalations} \colorBulletS{SYN} rapid increase, rise, hike, advance, growth, leap, upsurge, upturn, upswing, climb, jump, spiralling}{}{}{}%
\par%
\entry{essence}{/ˈɛs(ə)ns/}{সারাংশ}{ \textsf{\textit{noun}}\ \textbf{1} The intrinsic nature or indispensable quality of something, especially something abstract, which determines its character. {\fontspec{DejaVu Sans}◇} \textit{conflict is the essence of drama} \colorBulletS{SYN} quintessence, soul, spirit, ethos, nature, life, lifeblood, core, heart, centre, crux, nub, nucleus, kernel, marrow, meat, pith, gist, substance, principle, central part, fundamental quality, basic quality, essential part, intrinsic nature, sum and substance, reality, actuality \textbf{2} An extract or concentrate obtained from a plant or other matter and used for flavouring or scent. {\fontspec{DejaVu Sans}◇} \textit{vanilla essence} \colorBulletS{SYN} extract, concentrate, concentration, quintessence, distillate, elixir, abstraction, decoction, juice, tincture, solution, suspension, dilution}{}{}{ \colorBullet{ORIGIN} Late Middle English via Old French from Latin essentia, from esse ‘be’.}%
\par%
\entry{ethos}{/ˈiːθɒs/}{তত্ত্ব}{ \textsf{\textit{noun}}\ \textbf{1} The characteristic spirit of a culture, era, or community as manifested in its attitudes and aspirations. {\fontspec{DejaVu Sans}◇} \textit{a challenge to the ethos of the 1960s} \colorBulletS{SYN} spirit, character, atmosphere, climate, prevailing tendency, mood, feeling, temper, tenor, flavour, essence, quintessence}{}{}{ \colorBullet{ORIGIN} Mid 19th century from modern Latin, from Greek ēthos ‘nature, disposition’, (plural) ‘customs’.}%
\par%
\entry{eunuch}{/ˈjuːnək/}{নপুংসক}{ \textsf{\textit{noun}}\ \textbf{1} A man who has been castrated, especially (in the past) one employed to guard the women's living areas at an oriental court. {\fontspec{DejaVu Sans}◇} \textit{}}{}{}{ \colorBullet{ORIGIN} Old English, via Latin eunuchus from Greek eunoukhos, literally ‘bedroom guard’, from eunē ‘bed’ + a second element related to ekhein ‘to hold’.}%
\par%
\entry{evacuate}{/ɪˈvakjʊeɪt/}{উদ্বাসিত}{ \textsf{\textit{verb}}\ \textbf{1} Remove (someone) from a place of danger to a safer place. {\fontspec{DejaVu Sans}◇} \textit{several families were evacuated from their homes} \colorBulletS{SYN} remove, clear, move out, shift, take away, turn out, expel, evict \textbf{2} Remove air, water, or other contents from (a container) {\fontspec{DejaVu Sans}◇} \textit{when it springs a leak, evacuate the pond}}{}{}{ \colorBullet{ORIGIN} Late Middle English (in the sense ‘clear the contents of’): from Latin evacuat{-} ‘(of the bowels) emptied’, from the verb evacuare, from e{-} (variant of ex{-}) ‘out of’ + vacuus ‘empty’.}%
\par%
\entry{evade}{/ɪˈveɪd/}{টালা}{ \textsf{\textit{verb}}\ \textbf{1} Escape or avoid (someone or something), especially by guile or trickery. {\fontspec{DejaVu Sans}◇} \textit{friends helped him to evade capture for a time} \colorBulletS{SYN} elude, avoid, dodge, escape, escape from, stay away from, steer clear of, run away from, break away from, lose, leave behind, shake, shake off, keep at arm's length, keep out of someone's way, give someone a wide berth, sidestep, keep one's distance from}{}{}{ \colorBullet{ORIGIN} Late 15th century from French évader, from Latin evadere from e{-} (variant of ex{-}) ‘out of’ + vadere ‘go’.}%
\par%
\entry{evaluate}{/ɪˈvaljʊeɪt/}{মূল্যায়ন}{ \textsf{\textit{verb}}\ \textbf{1} Form an idea of the amount, number, or value of; assess. {\fontspec{DejaVu Sans}◇} \textit{the study will assist in evaluating the impact of recent changes} \colorBulletS{SYN} assess, assess the worth of, put a price on, put a value on \textbf{2} Find a numerical expression or equivalent for (an equation, formula, or function) {\fontspec{DejaVu Sans}◇} \textit{substitute numbers in a simple formula and evaluate the answer}}{}{}{ \colorBullet{ORIGIN} Mid 19th century (earlier (mid 18th century) as evaluation): from French évaluer, from es{-} (from Latin ex{-}) ‘out, from’ + Old French value ‘value’.}%
\par%
\entry{evasion}{/ɪˈveɪʒ(ə)n/}{ছল}{ \textsf{\textit{noun}}\ \textbf{1} The action of evading something. {\fontspec{DejaVu Sans}◇} \textit{their adroit evasion of almost all questions} \colorBulletS{SYN} avoidance, dodging, eluding, elusion, sidestepping, bypassing, circumvention, shunning, shirking}{}{}{ \colorBullet{ORIGIN} Late Middle English (in the sense ‘prevaricating excuse’): via Old French from Latin evasio(n{-}), from evadere (see evade).}%
\par%
\entry{evidence}{/ˈɛvɪd(ə)ns/}{প্রমাণ}{\small{\textsf{\textit{noun, verb}}} \\{\fontspec{DejaVu Sans}▪ }\textsf{\textit{noun}}\\ \textbf{1} The available body of facts or information indicating whether a belief or proposition is true or valid. {\fontspec{DejaVu Sans}◇} \textit{the study finds little evidence of overt discrimination} \colorBulletS{SYN} proof, confirmation, verification, substantiation, corroboration, affirmation, authentication, attestation, documentation \\{\fontspec{DejaVu Sans}▪ }\textsf{\textit{verb}}\\ \textbf{1} Be or show evidence of. {\fontspec{DejaVu Sans}◇} \textit{the quality of the bracelet, as evidenced by the workmanship, is exceptional} \colorBulletS{SYN} indicate, show, reveal, be evidence of, display, exhibit, manifest, denote, evince, signify}{}{}{ \colorBullet{ORIGIN} Middle English via Old French from Latin evidentia, from evident{-} ‘obvious to the eye or mind’ (see evident).}%
\par%
\entry{excavation}{/ɛkskəˈveɪʃ(ə)n/}{খনন}{ \textsf{\textit{noun}}\ \textbf{1} The action of excavating something, especially an archaeological site. {\fontspec{DejaVu Sans}◇} \textit{the methods of excavation have to be extremely rigorous} \colorBulletS{SYN} unearthing, digging up, uncovering, revealing}{}{}{}%
\par%
\entry{excessive}{/ɪkˈsɛsɪv/}{অত্যধিক}{ \textsf{\textit{adjective}}\ \textbf{1} More than is necessary, normal, or desirable; immoderate. {\fontspec{DejaVu Sans}◇} \textit{he was drinking excessive amounts of brandy} \colorBulletS{SYN} immoderate, intemperate, imprudent, overindulgent, unrestrained, unrestricted, uncontrolled, uncurbed, unbridled, lavish, extravagant}{}{}{ \colorBullet{ORIGIN} Late Middle English from Old French excessif, {-}ive, from medieval Latin excessivus, from Latin excedere ‘surpass’ (see exceed).}%
\par%
\entry{excrement}{/ˈɛkskrɪm(ə)nt/}{মল}{ \textsf{\textit{noun}}\ \textbf{1} Waste matter discharged from the bowels; faeces. {\fontspec{DejaVu Sans}◇} \textit{} \colorBulletS{SYN} faeces, excreta, stools, droppings}{}{}{ \colorBullet{ORIGIN} Mid 16th century from French excrément or Latin excrementum, from excernere ‘to sift out’ (see excrete).}%
\par%
\entry{exemption}{/ɪɡˈzɛmpʃn/}{অব্যাহতি}{ \textsf{\textit{noun}}\ \textbf{1} The action of freeing or state of being free from an obligation or liability imposed on others. {\fontspec{DejaVu Sans}◇} \textit{vehicles that may qualify for exemption from tax} \colorBulletS{SYN} immunity, exception, dispensation, indemnity, exclusion, freedom, release, relief, absolution, exoneration}{}{}{ \colorBullet{ORIGIN} Late Middle English from Old French, or from Latin exemptio(n{-}), from eximere ‘take out, free’.}%
\par%
\entry{exile}{/ˈɛksʌɪl/}{নির্বাসন}{\small{\textsf{\textit{noun, verb}}} \\{\fontspec{DejaVu Sans}▪ }\textsf{\textit{noun}}\\ \textbf{1} The state of being barred from one's native country, typically for political or punitive reasons. {\fontspec{DejaVu Sans}◇} \textit{he knew now that he would die in exile} \colorBulletS{SYN} banishment, expulsion, expatriation, deportation, eviction \\{\fontspec{DejaVu Sans}▪ }\textsf{\textit{verb}}\\ \textbf{1} Expel and bar (someone) from their native country, typically for political or punitive reasons. {\fontspec{DejaVu Sans}◇} \textit{a corrupt dictator who had been exiled from his country} \colorBulletS{SYN} expel, banish, expatriate, deport, ban, bar}{}{A year in exile ends tomorrow}{ \colorBullet{ORIGIN} Middle English the noun partly from Old French exil ‘banishment’ and partly from Old French exile ‘banished person’; the verb from Old French exiler; all based on Latin exilium ‘banishment’, from exul ‘banished person’.}%
\par%
\entry{exotic}{/ɪɡˈzɒtɪk/}{বহিরাগত}{\small{\textsf{\textit{adjective, noun}}} \\{\fontspec{DejaVu Sans}▪ }\textsf{\textit{adjective}}\\ \textbf{1} Originating in or characteristic of a distant foreign country. {\fontspec{DejaVu Sans}◇} \textit{exotic birds} \colorBulletS{SYN} foreign, non{-}native, tropical \\{\fontspec{DejaVu Sans}▪ }\textsf{\textit{noun}}\\ \textbf{1} An exotic plant or animal. {\fontspec{DejaVu Sans}◇} \textit{he planted exotics in the sheltered garden}}{}{}{ \colorBullet{ORIGIN} Late 16th century via Latin from Greek exōtikos ‘foreign’, from exō ‘outside’.}%
\par%
\entry{expatriate}{/ɪksˈpatrɪət/}{প্রবাসীদের}{\small{\textsf{\textit{adjective, noun, verb}}} \\{\fontspec{DejaVu Sans}▪ }\textsf{\textit{adjective}}\\ \textbf{1} Denoting or relating to a person living outside their native country. {\fontspec{DejaVu Sans}◇} \textit{expatriate workers} \colorBulletS{SYN} emigrant, living abroad, working abroad, non{-}native, émigré \\{\fontspec{DejaVu Sans}▪ }\textsf{\textit{noun}}\\ \textbf{1} A person who lives outside their native country. {\fontspec{DejaVu Sans}◇} \textit{American expatriates in London} \colorBulletS{SYN} newcomer, settler, incomer, new arrival, migrant, emigrant \\{\fontspec{DejaVu Sans}▪ }\textsf{\textit{verb}}\\ \textbf{1} Send (a person or money) abroad. {\fontspec{DejaVu Sans}◇} \textit{we expatriated the prisoners of war immediately after the end of the war} \colorBulletS{SYN} settle abroad, live abroad, relocate abroad}{}{}{ \colorBullet{ORIGIN} Mid 18th century (as a verb): from medieval Latin expatriat{-} ‘gone out from one's country’, from the verb expatriare, from ex{-} ‘out’ + patria ‘native country’.}%
\par%
\entry{expedite}{/ˈɛkspɪdʌɪt/}{সুবিধাযুক্ত}{ \textsf{\textit{verb}}\ \textbf{1} Make (an action or process) happen sooner or be accomplished more quickly. {\fontspec{DejaVu Sans}◇} \textit{he promised to expedite economic reforms} \colorBulletS{SYN} speed up, accelerate, hurry, hasten, step up, quicken, precipitate, rush}{}{}{ \colorBullet{ORIGIN} Late 15th century (in the sense ‘perform quickly’): from Latin expedire ‘extricate (originally by freeing the feet), put in order’, from ex{-} ‘out’ + pes, ped{-} ‘foot’.}%
\par%
\entry{expel}{/ɪkˈspɛl/}{বহিষ্কৃত}{ \textsf{\textit{verb}}\ \textbf{1} Officially make (someone) leave a school or other organization. {\fontspec{DejaVu Sans}◇} \textit{she was expelled from school} \colorBulletS{SYN} throw out, bar, ban, debar, drum out, thrust out, push out, turn out, oust, remove, get rid of}{}{}{ \colorBullet{ORIGIN} Late Middle English from Latin expellere, from ex{-} ‘out’ + pellere ‘to drive’.}%
\par%
\entry{exploitation}{/ɛksplɔɪˈteɪʃ(ə)n/}{শোষণ}{ \textsf{\textit{noun}}\ \textbf{1} The action or fact of treating someone unfairly in order to benefit from their work. {\fontspec{DejaVu Sans}◇} \textit{the exploitation of migrant workers} \colorBulletS{SYN} taking advantage, making use, abuse of, misuse, ill treatment, unfair treatment, bleeding dry, sucking dry, squeezing, wringing \textbf{2} The action of making use of and benefiting from resources. {\fontspec{DejaVu Sans}◇} \textit{the Bronze Age saw exploitation of gold deposits} \colorBulletS{SYN} utilization, utilizing, use, making use of, putting to use, making the most of, capitalization on}{}{}{}%
\par%
\entry{expulsion}{/ɪkˈspʌlʃ(ə)n/}{বিতাড়ন}{ \textsf{\textit{noun}}\ \textbf{1} The action of forcing someone to leave an organization. {\fontspec{DejaVu Sans}◇} \textit{his expulsion from the union} \colorBulletS{SYN} removal, debarment, dismissal, exclusion, discharge, ejection, rejection, blackballing, blacklisting}{}{}{ \colorBullet{ORIGIN} Late Middle English from Latin expulsio(n{-}), from expellere ‘drive out’ (see expel).}%
\par%
\entry{extortion}{/ɪkˈstɔːʃ(ə)n/}{চাঁদাবাজি}{ \textsf{\textit{noun}}\ \textbf{1} The practice of obtaining something, especially money, through force or threats. {\fontspec{DejaVu Sans}◇} \textit{he used bribery and extortion to build himself a huge, art{-}stuffed mansion} \colorBulletS{SYN} demanding money with menaces, exaction, extraction, blackmail}{}{}{ \colorBullet{ORIGIN} Middle English from late Latin extortio(n{-}), from Latin extorquere ‘wrest’ (see extort).}%
\par%
\entry{exude}{/ɪɡˈzjuːd/}{}{ \textsf{\textit{verb}}\ \textbf{1} (with reference to moisture or a smell) discharge or be discharged slowly and steadily. {\fontspec{DejaVu Sans}◇} \textit{the beetle exudes a caustic liquid} \colorBulletS{SYN} give off, give out, discharge, release, send out, send forth, emit, issue, emanate \textbf{2} (of a person) display (an emotion or quality) strongly and openly. {\fontspec{DejaVu Sans}◇} \textit{Sir Thomas exuded goodwill} \colorBulletS{SYN} emanate, radiate, ooze, give out, give forth, send out, issue, emit}{}{Probably because exude confidence.}{ \colorBullet{ORIGIN} Late 16th century from Latin exsudare, from ex{-} ‘out’ + sudare ‘to sweat’.}%
\par%
\end{multicols}%
\pagebreak%
\section*{F}%
\begin{multicols}{2}%
\entry{fabulous}{/ˈfabjʊləs/}{কল্পিত}{ \textsf{\textit{adjective}}\ \textbf{1} Extraordinary, especially extraordinarily large. {\fontspec{DejaVu Sans}◇} \textit{fabulous riches} \colorBulletS{SYN} tremendous, stupendous, prodigious, phenomenal \textbf{2} Having no basis in reality; mythical. {\fontspec{DejaVu Sans}◇} \textit{fabulous creatures} \colorBulletS{SYN} mythical, legendary, mythic, mythological, fabled, folkloric, fairy{-}tale, heroic, traditional}{}{}{ \colorBullet{ORIGIN} Late Middle English (in the sense ‘known through fable’): from French fabuleux or Latin fabulosus ‘celebrated in fable’, from fabula (see fable).}%
\par%
\entry{factoid}{/ˈfaktɔɪd/}{}{ \textsf{\textit{noun}}\ \textbf{1} An item of unreliable information that is reported and repeated so often that it becomes accepted as fact. {\fontspec{DejaVu Sans}◇} \textit{he addresses the facts and factoids which have buttressed the film's legend}}{}{}{}%
\par%
\entry{faint}{/feɪnt/}{ভীরু}{\small{\textsf{\textit{adjective, noun, verb}}} \\{\fontspec{DejaVu Sans}▪ }\textsf{\textit{adjective}}\\ \textbf{1} (of a sight, smell, or sound) barely perceptible. {\fontspec{DejaVu Sans}◇} \textit{the faint murmur of voices} \colorBulletS{SYN} indistinct, vague, unclear, indefinite, ill{-}defined, obscure, imperceptible, hardly noticeable, hardly detectable, unobtrusive \textbf{2} Feeling weak and dizzy and close to losing consciousness. {\fontspec{DejaVu Sans}◇} \textit{the heat made him feel faint} \colorBulletS{SYN} dizzy, giddy, light{-}headed, muzzy, weak, weak at the knees, unsteady, shaky, wobbly, off{-}balance, reeling \\{\fontspec{DejaVu Sans}▪ }\textsf{\textit{noun}}\\ \textbf{1} A sudden loss of consciousness. {\fontspec{DejaVu Sans}◇} \textit{she hit the floor in a dead faint} \colorBulletS{SYN} blackout, fainting fit, loss of consciousness, collapse \\{\fontspec{DejaVu Sans}▪ }\textsf{\textit{verb}}\\ \textbf{1} Lose consciousness for a short time because of a temporarily insufficient supply of oxygen to the brain. {\fontspec{DejaVu Sans}◇} \textit{I fainted from loss of blood} \colorBulletS{SYN} pass out, lose consciousness, fall unconscious, black out, collapse}{}{}{ \colorBullet{ORIGIN} Middle English (in the sense ‘feigned’, also ‘feeble, cowardly’, surviving in faint heart): from Old French faint, past participle of faindre (see feign). Compare with feint.}%
\par%
\entry{faltering}{/ˈfɔːltərɪŋ/}{অস্বচ্ছন্দ}{ \textsf{\textit{adjective}}\ \textbf{1} Losing strength or momentum. {\fontspec{DejaVu Sans}◇} \textit{his faltering career}}{}{}{}%
\par%
\entry{famine}{/ˈfamɪn/}{দুর্ভিক্ষ}{ \textsf{\textit{noun}}\ \textbf{1} Extreme scarcity of food. {\fontspec{DejaVu Sans}◇} \textit{drought resulted in famine throughout the region} \colorBulletS{SYN} scarcity of food, food shortages}{}{}{ \colorBullet{ORIGIN} Late Middle English from Old French, from faim ‘hunger’, from Latin fames.}%
\par%
\entry{fancy}{/ˈfansi/}{অভিনব; কাল্পনিক}{\small{\textsf{\textit{adjective, noun, verb}}} \\{\fontspec{DejaVu Sans}▪ }\textsf{\textit{adjective}}\\ \textbf{1} Elaborate in structure or decoration. {\fontspec{DejaVu Sans}◇} \textit{the furniture was very fancy} \colorBulletS{SYN} ornate, decorated, embellished, adorned, ornamented, fancy, over{-}elaborate, fussy, busy, ostentatious, extravagant, showy, baroque, rococo, florid, wedding{-}cake, gingerbread \textbf{2} (of a drawing, painting, or sculpture) created from the imagination rather than from life. {\fontspec{DejaVu Sans}◇} \textit{I used to take a seat and busy myself in sketching fancy vignettes} \\{\fontspec{DejaVu Sans}▪ }\textsf{\textit{noun}}\\ \textbf{1} A superficial or transient feeling of liking or attraction. {\fontspec{DejaVu Sans}◇} \textit{this was no passing fancy, but a feeling he would live by} \colorBulletS{SYN} desire, urge, wish, want \textbf{2} The faculty of imagination. {\fontspec{DejaVu Sans}◇} \textit{he is prone to flights of fancy} \colorBulletS{SYN} imagination, imaginative faculty, imaginative power, creativity, creative faculty, creative power, conception, fancifulness, inventiveness, invention, originality, ingenuity, cleverness, wit, artistry \textbf{3}  {\fontspec{DejaVu Sans}◇} \textit{chocolate fancies} \textbf{4} (in 16th and 17th century music) a composition for keyboard or strings in free or variation form. {\fontspec{DejaVu Sans}◇} \textit{Division technique...penetrated nearly all 17th century English instrumental forms, including the venerable polyphonic fancy.} \\{\fontspec{DejaVu Sans}▪ }\textsf{\textit{verb}}\\ \textbf{1} Feel a desire or liking for. {\fontspec{DejaVu Sans}◇} \textit{do you fancy a drink?} \colorBulletS{SYN} wish for, want, desire \textbf{2} Regard (a horse, team, or player) as a likely winner. {\fontspec{DejaVu Sans}◇} \textit{I fancy him to win the tournament} \textbf{3} Imagine; think. {\fontspec{DejaVu Sans}◇} \textit{he fancied he could smell the perfume of roses} \colorBulletS{SYN} think, imagine, guess, believe, have an idea, suppose}{}{}{ \colorBullet{ORIGIN} Late Middle English contraction of fantasy.}%
\par%
\entry{farcical}{/ˈfɑːsɪk(ə)l/}{হাস্যকর}{ \textsf{\textit{adjective}}\ \textbf{1} Relating to or resembling farce, especially because of absurd or ridiculous aspects. {\fontspec{DejaVu Sans}◇} \textit{he considered the whole idea farcical} \colorBulletS{SYN} ridiculous, preposterous, ludicrous, absurd, laughable, risible, nonsensical}{}{}{}%
\par%
\entry{fare}{/fɛː/}{ভাড়া}{\small{\textsf{\textit{noun, verb}}} \\{\fontspec{DejaVu Sans}▪ }\textsf{\textit{noun}}\\ \textbf{1} The money paid for a journey on public transport. {\fontspec{DejaVu Sans}◇} \textit{we should go to Seville, but we cannot afford the air fare} \colorBulletS{SYN} ticket price, transport cost, price, cost, charge, fee, payment, toll, tariff, levy \textbf{2} A range of food of a particular type. {\fontspec{DejaVu Sans}◇} \textit{traditional Scottish fare} \colorBulletS{SYN} food, meals, board, sustenance, nourishment, nutriment, foodstuffs, refreshments, eatables, provisions, daily bread \\{\fontspec{DejaVu Sans}▪ }\textsf{\textit{verb}}\\ \textbf{1} Perform in a specified way in a particular situation or over a particular period. {\fontspec{DejaVu Sans}◇} \textit{the party fared badly in the elections} \colorBulletS{SYN} get on, proceed, get along, progress, make out, do, manage, muddle along, muddle through, cope, survive \textbf{2} Travel. {\fontspec{DejaVu Sans}◇} \textit{a knight fares forth}}{}{}{ \colorBullet{ORIGIN} Old English fær, faru ‘travelling, a journey or expedition’, faran‘to travel’, also ‘get on (well or badly’), of Germanic origin; related to Dutch varen and German fahren ‘to travel’, Old Norse ferja ‘ferry boat’, also to ford. Sense 1 of the noun stems from an earlier meaning ‘a journey for which a price is paid’. Noun sense 2 was originally used with reference to the quality or quantity of food provided, probably from the idea of faring well or badly.}%
\par%
\entry{fatal}{/ˈfeɪt(ə)l/}{মারাত্মক}{ \textsf{\textit{adjective}}\ \textbf{1} Causing death. {\fontspec{DejaVu Sans}◇} \textit{a fatal accident} \colorBulletS{SYN} deadly, lethal, mortal, causing death, death dealing, killing}{}{}{ \colorBullet{ORIGIN} Late Middle English (in the senses ‘destined by fate’ and ‘ominous’): from Old French, or from Latin fatalis, from fatum (see fate).}%
\par%
\entry{fatality}{/fəˈtalɪti/}{নশ্বরতা}{ \textsf{\textit{noun}}\ \textbf{1} An occurrence of death by accident, in war, or from disease. {\fontspec{DejaVu Sans}◇} \textit{80 per cent of pedestrian fatalities occur in built{-}up areas} \colorBulletS{SYN} death, casualty, mortality, victim, loss, dead person \textbf{2} Helplessness in the face of fate. {\fontspec{DejaVu Sans}◇} \textit{a sense of fatality gripped her}}{}{Fatality rate: মৃত্যুর হার}{ \colorBullet{ORIGIN} Late 15th century (denoting the quality of causing death or disaster): from French fatalité or late Latin fatalitas, from Latin fatalis ‘decreed by fate’, from fatum (see fate). fatality (sense 1) dates from the mid 19th century.}%
\par%
\entry{fault}{/fɔːlt/}{দোষ}{\small{\textsf{\textit{noun, verb}}} \\{\fontspec{DejaVu Sans}▪ }\textsf{\textit{noun}}\\ \textbf{1} An unattractive or unsatisfactory feature, especially in a piece of work or in a person's character. {\fontspec{DejaVu Sans}◇} \textit{my worst fault is impatience} \colorBulletS{SYN} flaw, fault, failing, deficiency, weakness, weak point, weak spot, shortcoming, fallibility, frailty, infirmity, foible, inadequacy, limitation \textbf{2} Responsibility for an accident or misfortune. {\fontspec{DejaVu Sans}◇} \textit{if books were not selling, it wasn't the fault of the publishers} \colorBulletS{SYN} responsibility, liability, culpability, blameworthiness, guilt \textbf{3} An extended break in a rock formation, marked by the relative displacement and discontinuity of strata on either side of a particular plane. {\fontspec{DejaVu Sans}◇} \textit{a landscape broken by numerous faults} \\{\fontspec{DejaVu Sans}▪ }\textsf{\textit{verb}}\\ \textbf{1} Criticize for inadequacy or mistakes. {\fontspec{DejaVu Sans}◇} \textit{her superiors could not fault her dedication to the job} \colorBulletS{SYN} find fault with, find lacking \textbf{2} (of a rock formation) be broken by a fault or faults. {\fontspec{DejaVu Sans}◇} \textit{the continental crust has been thinned and faulted as a result of geological processes}}{}{}{ \colorBullet{ORIGIN} Middle English faut(e) ‘lack, failing’, from Old French, based on Latin fallere ‘deceive’. The {-}l{-} was added (in French and English) in the 15th century to conform with the Latin word, but did not become standard in English until the 17th century, remaining silent in pronunciation until well into the 18th.}%
\par%
\entry{fauna}{/ˈfɔːnə/}{প্রাণিকুল}{ \textsf{\textit{noun}}\ \textbf{1} The animals of a particular region, habitat, or geological period. {\fontspec{DejaVu Sans}◇} \textit{the flora and fauna of Siberia} \colorBulletS{SYN} living things, living beings, living creatures, the living}{}{}{ \colorBullet{ORIGIN} Late 18th century modern Latin application of Fauna, the name of a rural goddess, sister of Faunus.}%
\par%
\entry{faux}{/fəʊ/}{ভুল}{ \textsf{\textit{adjective}}\ \textbf{1} Made in imitation; artificial. {\fontspec{DejaVu Sans}◇} \textit{a rope of faux pearls} \colorBulletS{SYN} imaginary, imagined, pretended, make{-}believe, made{-}up, fantasy, fantasized, fancied, dream, dreamed{-}up, unreal, fanciful, invented, fictitious, fictive, mythical, feigned, fake, mock, imitative, sham, simulated, artificial, ersatz, dummy, false, faux, spurious, bogus, counterfeit, fraudulent, forged, pseudo}{}{}{ \colorBullet{ORIGIN} French, ‘false’.}%
\par%
\entry{fear}{/fɪə/}{ভয়}{\small{\textsf{\textit{noun, verb}}} \\{\fontspec{DejaVu Sans}▪ }\textsf{\textit{noun}}\\ \textbf{1} An unpleasant emotion caused by the threat of danger, pain, or harm. {\fontspec{DejaVu Sans}◇} \textit{I cowered in fear as bullets whizzed past} \colorBulletS{SYN} terror, fright, fearfulness, horror, alarm, panic, agitation, trepidation, dread, consternation, dismay, distress \\{\fontspec{DejaVu Sans}▪ }\textsf{\textit{verb}}\\ \textbf{1} Be afraid of (someone or something) as likely to be dangerous, painful, or harmful. {\fontspec{DejaVu Sans}◇} \textit{I hated him but didn't fear him any more} \colorBulletS{SYN} be afraid of, be fearful of, be scared of, be apprehensive of, dread, live in fear of, go in terror of, be terrified of, be terrified by, cower before, tremble before, cringe from, shrink from, flinch from}{}{Feared dead: মৃত্যুর আশঙ্কা করা}{ \colorBullet{ORIGIN} Old English fǣr ‘calamity, danger’, fǣran ‘frighten’, also ‘revere’, of Germanic origin; related to Dutch gevaar and German Gefahr ‘danger’.}%
\par%
\entry{feasible}{/ˈfiːzɪb(ə)l/}{সাধ্য}{ \textsf{\textit{adjective}}\ \textbf{1} Possible to do easily or conveniently. {\fontspec{DejaVu Sans}◇} \textit{it is not feasible to put most finds from excavations on public display} \colorBulletS{SYN} practicable, practical, workable, achievable, attainable, realizable, viable, realistic, sensible, reasonable, within reason, within the bounds of possibility}{}{}{ \colorBullet{ORIGIN} Late Middle English from Old French faisible, from fais{-}, stem of faire ‘do, make’, from Latin facere.}%
\par%
\entry{feat}{/fiːt/}{কৃতিত্ব}{ \textsf{\textit{noun}}\ \textbf{1} An achievement that requires great courage, skill, or strength. {\fontspec{DejaVu Sans}◇} \textit{the new printing presses were considerable feats of engineering} \colorBulletS{SYN} achievement, accomplishment, attainment, coup, master stroke, triumph}{}{}{ \colorBullet{ORIGIN} Late Middle English (in the general sense ‘action or deed’): from Old French fait, from Latin factum (see fact).}%
\par%
\entry{fella}{/ˈfɛlə/}{বন্ধুরা}{ \textsf{\textit{noun}}\ \textbf{1} non{-}standard spelling of fellow, used in representing speech in various dialects {\fontspec{DejaVu Sans}◇} \textit{you can't blame the wee fella} \textbf{2} A person's boyfriend or lover. {\fontspec{DejaVu Sans}◇} \textit{she took a fancy to her best friend's fella} \colorBulletS{SYN} boyfriend, girlfriend, man friend, woman friend, lady friend, lady{-}love, beau, loved one, beloved, love, darling, sweetheart}{}{}{}%
\par%
\entry{fellow}{/ˈfɛləʊ/}{সহকর্মী}{\small{\textsf{\textit{adjective, noun}}} \\{\fontspec{DejaVu Sans}▪ }\textsf{\textit{adjective}}\\ \textbf{1} Sharing a particular activity, quality, or condition with someone or something. {\fontspec{DejaVu Sans}◇} \textit{they urged the troops not to fire on their fellow citizens} \\{\fontspec{DejaVu Sans}▪ }\textsf{\textit{noun}}\\ \textbf{1} A man or boy. {\fontspec{DejaVu Sans}◇} \textit{he was an extremely obliging fellow} \colorBulletS{SYN} man, boy \textbf{2} A person in the same position, involved in the same activity, or otherwise associated with another. {\fontspec{DejaVu Sans}◇} \textit{he was learning with a rapidity unique among his fellows} \colorBulletS{SYN} companion, friend, crony, comrade, partner, associate, co{-}worker, colleague \textbf{3} A member of a learned society. {\fontspec{DejaVu Sans}◇} \textit{a fellow of the Geological Society} \colorBulletS{SYN} subscriber, associate, representative, attender, insider, fellow, comrade, adherent, life member, founder member, card{-}carrying member}{}{}{ \colorBullet{ORIGIN} Late Old English fēolaga ‘a partner or colleague’ (literally ‘one who lays down money in a joint enterprise’), from Old Norse félagi, from fé ‘cattle, property, money’ + the Germanic base of lay.}%
\par%
\entry{fellowship}{/ˈfɛlə(ʊ)ʃɪp/}{সহকারিতা}{ \textsf{\textit{noun}}\ \textbf{1} Friendly association, especially with people who share one's interests. {\fontspec{DejaVu Sans}◇} \textit{they valued fun and good fellowship as the cement of the community} \colorBulletS{SYN} companionship, companionability, sociability, comradeship, fraternization, camaraderie, friendship, mutual support, mutual respect, mutual liking \textbf{2} The status of a fellow of a college or society. {\fontspec{DejaVu Sans}◇} \textit{a fellowship in mathematics}}{}{}{}%
\par%
\entry{ferocity}{/fəˈrɒsɪti/}{হিংস্রতা}{ \textsf{\textit{noun}}\ \textbf{1} The state or quality of being ferocious. {\fontspec{DejaVu Sans}◇} \textit{the ferocity of the storm caught them by surprise} \colorBulletS{SYN} savagery, brutality, brutishness, barbarity, fierceness, violence, aggression, bloodthirstiness, murderousness}{}{}{ \colorBullet{ORIGIN} Mid 16th century from French, or from Latin ferocitas, from ferox, feroc{-} ‘fierce’.}%
\par%
\entry{ferrite}{/ˈfɛrʌɪt/}{চুম্বক}{ \textsf{\textit{noun}}\ \textbf{1} A ceramic compound consisting of a mixed oxide of iron and one or more other metals which has ferrimagnetic properties and is used in high{-}frequency electrical components such as aerials. {\fontspec{DejaVu Sans}◇} \textit{} \textbf{2} A form of pure iron with a body{-}centred cubic crystal structure, occurring in low{-}carbon steel. {\fontspec{DejaVu Sans}◇} \textit{}}{}{}{ \colorBullet{ORIGIN} Mid 19th century from Latin ferrum ‘iron’ + {-}ite.}%
\par%
\entry{fetish}{/ˈfɛtɪʃ/}{ফেটিশ}{ \textsf{\textit{noun}}\ \textbf{1} A form of sexual desire in which gratification is linked to an abnormal degree to a particular object, item of clothing, part of the body, etc. {\fontspec{DejaVu Sans}◇} \textit{a man with a fetish for surgical masks} \colorBulletS{SYN} fixation, sexual fixation, obsession, compulsion, mania \textbf{2} An inanimate object worshipped for its supposed magical powers or because it is considered to be inhabited by a spirit. {\fontspec{DejaVu Sans}◇} \textit{} \colorBulletS{SYN} juju, talisman, charm, amulet}{}{}{ \colorBullet{ORIGIN} Early 17th century (originally denoting an object used by the peoples of West Africa as an amulet or charm): from French fétiche, from Portuguese feitiço ‘charm, sorcery’ (originally an adjective meaning ‘made by art’), from Latin facticius (see factitious).}%
\par%
\entry{fiancée}{/fɪˈɒnseɪ/}{বাগ্দত্তা}{ \textsf{\textit{noun}}\ \textbf{1} A woman to whom someone is engaged to be married. {\fontspec{DejaVu Sans}◇} \textit{he went back to the valley to marry his fiancée} \colorBulletS{SYN} betrothed, wife{-}to{-}be, bride{-}to{-}be, future wife, prospective wife, prospective spouse}{}{}{ \colorBullet{ORIGIN} Mid 19th century from French (see fiancé).}%
\par%
\entry{fidelity}{/fɪˈdɛlɪti/}{বিশ্বস্ততা}{ \textsf{\textit{noun}}\ \textbf{1} Faithfulness to a person, cause, or belief, demonstrated by continuing loyalty and support. {\fontspec{DejaVu Sans}◇} \textit{his fidelity to liberal ideals} \colorBulletS{SYN} loyalty, allegiance, obedience, constancy, fealty, homage \textbf{2} The degree of exactness with which something is copied or reproduced. {\fontspec{DejaVu Sans}◇} \textit{the 1949 recording provides reasonable fidelity} \colorBulletS{SYN} accuracy, exactness, exactitude, precision, preciseness, correctness, scrupulousness}{}{}{ \colorBullet{ORIGIN} Late Middle English from Old French fidelite or Latin fidelitas, from fidelis ‘faithful’, from fides ‘faith’. Compare with fealty.}%
\par%
\entry{filthy}{/ˈfɪlθi/}{অকথ্য}{\small{\textsf{\textit{adjective, adverb}}} \\{\fontspec{DejaVu Sans}▪ }\textsf{\textit{adjective}}\\ \textbf{1} Disgustingly dirty. {\fontspec{DejaVu Sans}◇} \textit{a filthy hospital with no sanitation} \colorBulletS{SYN} dirty, mucky, grimy, muddy, murky, slimy, unclean \\{\fontspec{DejaVu Sans}▪ }\textsf{\textit{adverb}}\\ \textbf{1} To an extreme extent. {\fontspec{DejaVu Sans}◇} \textit{he has become filthy rich} \colorBulletS{SYN} very, extremely, tremendously, immensely, vastly, hugely, remarkably}{}{}{}%
\par%
\entry{fiscal}{/ˈfɪsk(ə)l/}{রাজকোষ}{\small{\textsf{\textit{adjective, noun}}} \\{\fontspec{DejaVu Sans}▪ }\textsf{\textit{adjective}}\\ \textbf{1} Relating to government revenue, especially taxes. {\fontspec{DejaVu Sans}◇} \textit{monetary and fiscal policy} \colorBulletS{SYN} tax, budgetary, revenue \\{\fontspec{DejaVu Sans}▪ }\textsf{\textit{noun}}\\ \textbf{1} A legal or treasury official in some countries. {\fontspec{DejaVu Sans}◇} \textit{As early as 1711, an Oberfiscal was appointed aided by a staff of fiscals who had to be secret appointments as they had the task of checking the honesty and integrity of government officials.} \textbf{2}  {\fontspec{DejaVu Sans}◇} \textit{}}{}{}{ \colorBullet{ORIGIN} Mid 16th century from French, or from Latin fiscalis, from fiscus ‘purse, treasury’ (see fisc).}%
\par%
\entry{flagging}{/ˈflaɡɪŋ/}{ঝিমুনি}{ \textsf{\textit{adjective}}\ \textbf{1} Becoming tired or less dynamic; declining in strength. {\fontspec{DejaVu Sans}◇} \textit{she wants to revive her flagging career}}{}{}{}%
\par%
\entry{flagship}{/ˈflaɡʃɪp/}{পোত{-}নায়কের জাহাজ}{ \textsf{\textit{noun}}\ \textbf{1} The ship in a fleet which carries the commanding admiral. {\fontspec{DejaVu Sans}◇} \textit{}}{}{}{}%
\par%
\entry{flamboyance}{/flamˈbɔɪəns/}{ধুমধাম}{ \textsf{\textit{noun}}\ \textbf{1} The tendency to attract attention because of one's exuberance, confidence, and stylishness. {\fontspec{DejaVu Sans}◇} \textit{he had a reputation for flair and flamboyance}}{}{}{}%
\par%
\entry{flared}{/flɛːd/}{উদ্দীপ্ত}{ \textsf{\textit{adjective}}\ \textbf{1} (especially of an item of clothing) having a shape that widens progressively towards the end or bottom. {\fontspec{DejaVu Sans}◇} \textit{a flared skirt} \textbf{2} (of the nostrils) dilated. {\fontspec{DejaVu Sans}◇} \textit{horses snorted impatiently through flared nostrils}}{ \colorBullet{OTHER} flared up}{}{}%
\par%
\entry{flawed}{/flɔːd/}{দ্বিধান্বিত}{ \textsf{\textit{adjective}}\ \textbf{1} Having or characterized by a fundamental weakness or imperfection. {\fontspec{DejaVu Sans}◇} \textit{a fatally flawed strategy} \colorBulletS{SYN} unsound, defective, faulty, distorted, inaccurate, incorrect, erroneous, imprecise, fallacious, wrong}{}{}{}%
\par%
\entry{flawless}{/ˈflɔːləs/}{নিশ্ছিদ্র}{ \textsf{\textit{adjective}}\ \textbf{1} Without any imperfections or defects; perfect. {\fontspec{DejaVu Sans}◇} \textit{her smooth flawless skin} \colorBulletS{SYN} perfect, without blemish, unblemished, unmarked, unimpaired}{}{}{}%
\par%
\entry{flee}{/fliː/}{ভাগা}{ \textsf{\textit{verb}}\ \textbf{1} Run away from a place or situation of danger. {\fontspec{DejaVu Sans}◇} \textit{to escape the fighting, his family fled from their village} \colorBulletS{SYN} run, run away, run off, make a run for it, run for it, take flight, be gone, make off, take off, take to one's heels, make a break for it, bolt, beat a retreat, beat a hasty retreat, make a quick exit, make one's getaway, escape, absent oneself, make oneself scarce, abscond, head for the hills, do a disappearing act}{}{}{ \colorBullet{ORIGIN} Old English flēon, of Germanic origin; related to Dutch vlieden and German fliehen.}%
\par%
\entry{fleet}{/fliːt/}{বহর}{ \textsf{\textit{noun}}\ \textbf{1} A group of ships sailing together, engaged in the same activity, or under the same ownership. {\fontspec{DejaVu Sans}◇} \textit{the small port supports a fishing fleet}}{}{}{ \colorBullet{ORIGIN} Old English flēot ‘ship, shipping’, from flēotan ‘float, swim’ (see fleet).}%
\par%
\entry{fleet}{/fliːt/}{বহর}{ \textsf{\textit{adjective}}\ \textbf{1} Fast and nimble in movement. {\fontspec{DejaVu Sans}◇} \textit{a man of advancing years, but fleet of foot} \colorBulletS{SYN} nimble, agile, deft, lithe, limber, lissom, acrobatic, supple, light{-}footed, nimble{-}footed, light, light of foot, light on one's feet, spry, sprightly, lively, active}{}{}{ \colorBullet{ORIGIN} Early 16th century probably from Old Norse fljótr, of Germanic origin and related to fleet.}%
\par%
\entry{fleet}{/fliːt/}{বহর}{ \textsf{\textit{noun}}\ \textbf{1} A marshland creek, channel, or ditch. {\fontspec{DejaVu Sans}◇} \textit{Sam explained that the 3,000 acres of the Nature Reserve is the largest in the English lowlands, the main area being grazing marsh divided by a network of ditches and fleets.} \textbf{2} A stream, now wholly underground, running into the Thames east of Fleet Street. {\fontspec{DejaVu Sans}◇} \textit{}}{}{}{ \colorBullet{ORIGIN} Old English flēot, of Germanic origin; related to Dutch vliet, also to fleet.}%
\par%
\entry{fleet}{/fliːt/}{বহর}{ \textsf{\textit{verb}}\ \textbf{1} Move or pass quickly. {\fontspec{DejaVu Sans}◇} \textit{a variety of expressions fleeted across his face}}{}{}{ \colorBullet{ORIGIN} Old English flēotan ‘float, swim’, of Germanic origin; related to Dutch vlieten and German fliessen, also to flit and float.}%
\par%
\entry{fleet}{/fliːt/}{বহর}{\small{\textsf{\textit{adjective, adverb}}} \\{\fontspec{DejaVu Sans}▪ }\textsf{\textit{adjective}}\\ \textbf{1} (of water) shallow. {\fontspec{DejaVu Sans}◇} \textit{} \\{\fontspec{DejaVu Sans}▪ }\textsf{\textit{adverb}}\\ \textbf{1} At or to a small depth. {\fontspec{DejaVu Sans}◇} \textit{}}{}{}{ \colorBullet{ORIGIN} Early 17th century perhaps based on an Old English cognate of Dutch vloot ‘shallow’ and related to fleet.}%
\par%
\entry{flirtation}{/fləːˈteɪʃn/}{}{ \textsf{\textit{noun}}\ \textbf{1} Behaviour that demonstrates a playful sexual attraction to someone. {\fontspec{DejaVu Sans}◇} \textit{Fabia was in no mood for his light{-}hearted flirtation} \colorBulletS{SYN} coquetry, teasing, trifling, toying, dalliance, philandering, romantic advances}{}{}{}%
\par%
\entry{flock}{/flɒk/}{পাল}{\small{\textsf{\textit{noun, verb}}} \\{\fontspec{DejaVu Sans}▪ }\textsf{\textit{noun}}\\ \textbf{1} A number of birds of one kind feeding, resting, or travelling together. {\fontspec{DejaVu Sans}◇} \textit{a flock of gulls} \colorBulletS{SYN} group, flight, congregation \\{\fontspec{DejaVu Sans}▪ }\textsf{\textit{verb}}\\ \textbf{1} (of birds) congregate in a flock. {\fontspec{DejaVu Sans}◇} \textit{sandgrouse are liable to flock with other species}}{}{}{ \colorBullet{ORIGIN} Old English flocc, of unknown origin. The original sense was ‘a band or body of people’: this became obsolete, but has been reintroduced as a transferred use of the sense ‘a number of animals kept together’.}%
\par%
\entry{flock}{/flɒk/}{পাল}{ \textsf{\textit{noun}}\ \textbf{1} A soft material for stuffing cushions, quilts, and other soft furnishings, made of wool refuse or torn{-}up cloth. {\fontspec{DejaVu Sans}◇} \textit{flock mattresses}}{}{}{ \colorBullet{ORIGIN} Middle English from Old French floc, from Latin floccus (see floccus).}%
\par%
\entry{flora}{/ˈflɔːrə/}{উদ্ভিদকুল; Flora and fauna: The flora and fauna of a place are its plants and animals.}{ \textsf{\textit{noun}}\ \textbf{1} The plants of a particular region, habitat, or geological period. {\fontspec{DejaVu Sans}◇} \textit{Britain's native flora}}{}{1. The flora and fauna of santal life have been clearly indicated in his work and captured with passion.}{ \colorBullet{ORIGIN} Late 18th century from Latin flos, flor{-} ‘flower’.}%
\par%
\entry{Flora}{/ˈflɔːrə/}{উদ্ভিদকুল; Flora and fauna: The flora and fauna of a place are its plants and animals.}{ \textsf{\textit{proper noun}}\ \textbf{1} The goddess of flowering plants. {\fontspec{DejaVu Sans}◇} \textit{}}{}{1. The flora and fauna of santal life have been clearly indicated in his work and captured with passion.}{}%
\par%
\entry{fluctuate}{/ˈflʌktʃʊeɪt/}{অস্থির হত্তয়া}{ \textsf{\textit{verb}}\ \textbf{1} Rise and fall irregularly in number or amount. {\fontspec{DejaVu Sans}◇} \textit{trade with other countries tends to fluctuate from year to year} \colorBulletS{SYN} vary, differ, shift, change, alter, waver, swing, oscillate, alternate, rise and fall, go up and down, see{-}saw, yo{-}yo, be unstable, be unsteady}{}{}{ \colorBullet{ORIGIN} Mid 17th century (earlier (late Middle English) as fluctuation): from Latin fluctuat{-} ‘undulated’, from the verb fluctuare, from fluctus ‘flow, current, wave’, from fluere ‘to flow’.}%
\par%
\entry{flummox}{/ˈflʌməks/}{বিহ্বল করা}{ \textsf{\textit{verb}}\ \textbf{1} Perplex (someone) greatly; bewilder. {\fontspec{DejaVu Sans}◇} \textit{I was completely flummoxed by the whole thing} \colorBulletS{SYN} baffle, bewilder, mystify, bemuse, perplex, puzzle, confuse, confound, nonplus, disconcert, throw, throw off balance, disorientate, take aback, set thinking}{}{}{ \colorBullet{ORIGIN} Mid 19th century probably of dialect origin; flummock ‘to make untidy, confuse’ is recorded in western counties and the north Midlands.}%
\par%
\entry{flunk}{/flʌŋk/}{কার্যবিপত্তি}{ \textsf{\textit{verb}}\ \textbf{1} Fail to reach the required standard in (an examination, test, or course of study) {\fontspec{DejaVu Sans}◇} \textit{I flunked biology in the tenth grade} \colorBulletS{SYN} be unsuccessful in, not pass}{}{}{ \colorBullet{ORIGIN} Early 19th century (in the general sense ‘back down, fail utterly’; originally US): perhaps related to funk or to US flink ‘be a coward’, perhaps a variant of flinch.}%
\par%
\entry{foil}{/fɔɪl/}{পাত}{\small{\textsf{\textit{noun, verb}}} \\{\fontspec{DejaVu Sans}▪ }\textsf{\textit{noun}}\\ \textbf{1} The track or scent of a hunted animal. {\fontspec{DejaVu Sans}◇} \textit{} \textbf{2} A setback in an enterprise; a defeat. {\fontspec{DejaVu Sans}◇} \textit{} \\{\fontspec{DejaVu Sans}▪ }\textsf{\textit{verb}}\\ \textbf{1} Prevent (something considered wrong or undesirable) from succeeding. {\fontspec{DejaVu Sans}◇} \textit{a brave policewoman foiled the armed robbery} \colorBulletS{SYN} thwart, frustrate, counter, oppose, balk, disappoint, impede, obstruct, hamper, hinder, snooker, cripple, scotch, derail, smash, dash}{}{}{ \colorBullet{ORIGIN} Middle English (in the sense ‘trample down’): perhaps from Old French fouler ‘to full cloth, trample’, based on Latin fullo ‘fuller’. Compare with full.}%
\par%
\entry{foil}{/fɔɪl/}{পাত}{ \textsf{\textit{noun}}\ \textbf{1} Metal hammered or rolled into a thin flexible sheet, used chiefly for covering or wrapping food. {\fontspec{DejaVu Sans}◇} \textit{aluminium foil} \textbf{2} A person or thing that contrasts with and so emphasizes and enhances the qualities of another. {\fontspec{DejaVu Sans}◇} \textit{his white cravat was a perfect foil for his bronzed features} \colorBulletS{SYN} contrast, background, setting, relief, antithesis \textbf{3} A leaf{-}shaped curve formed by the cusping of an arch or circle. {\fontspec{DejaVu Sans}◇} \textit{}}{}{}{ \colorBullet{ORIGIN} Middle English via Old French from Latin folium ‘leaf’.}%
\par%
\entry{foil}{/fɔɪl/}{পাত}{ \textsf{\textit{noun}}\ \textbf{1} A light, blunt{-}edged fencing sword with a button on its point. {\fontspec{DejaVu Sans}◇} \textit{}}{}{}{ \colorBullet{ORIGIN} Late 16th century of unknown origin.}%
\par%
\entry{foil}{/fɔɪl/}{পাত}{ \textsf{\textit{noun}}\ \textbf{1} Each of the structures fitted to a hydrofoil's hull to lift it clear of the water at speed. {\fontspec{DejaVu Sans}◇} \textit{}}{}{}{ \colorBullet{ORIGIN} Abbreviation of hydrofoil.}%
\par%
\entry{folk}{/fəʊk/}{লোক}{\small{\textsf{\textit{adjective, noun}}} \\{\fontspec{DejaVu Sans}▪ }\textsf{\textit{adjective}}\\ \textbf{1} Relating to the traditional art or culture of a community or nation. {\fontspec{DejaVu Sans}◇} \textit{a revival of interest in folk customs} \colorBulletS{SYN} racial, race{-}related, ethnological, genetic, inherited \textbf{2} Relating to folk music. {\fontspec{DejaVu Sans}◇} \textit{a folk club} \\{\fontspec{DejaVu Sans}▪ }\textsf{\textit{noun}}\\ \textbf{1}  {\fontspec{DejaVu Sans}◇} \textit{some folk will do anything for money} \colorBulletS{SYN} people, humans, persons, individuals, souls, living souls, mortals \textbf{2} Folk music. {\fontspec{DejaVu Sans}◇} \textit{a mixture of folk and reggae}}{}{}{ \colorBullet{ORIGIN} Old English folc, of Germanic origin; related to Dutch volk and German Volk.}%
\par%
\entry{fondle}{/ˈfɒnd(ə)l/}{নেহ}{\small{\textsf{\textit{noun, verb}}} \\{\fontspec{DejaVu Sans}▪ }\textsf{\textit{noun}}\\ \textbf{1} An act of fondling. {\fontspec{DejaVu Sans}◇} \textit{} \colorBulletS{SYN} stroke, stroking, touch, touching, fondle, fondling, skim, pat, nuzzle, nuzzling, kiss \\{\fontspec{DejaVu Sans}▪ }\textsf{\textit{verb}}\\ \textbf{1} Stroke or caress lovingly or erotically. {\fontspec{DejaVu Sans}◇} \textit{he kissed and fondled her} \colorBulletS{SYN} caress, stroke, pat, pet, pull, finger, touch, tickle, twiddle, play with, massage, knead}{}{}{ \colorBullet{ORIGIN} Late 17th century (in the sense ‘pamper’): back{-}formation from obsolete fondling ‘much{-}loved or petted person’, from fond+ {-}ling.}%
\par%
\entry{forged}{/fɔːdʒd/}{নকল}{ \textsf{\textit{adjective}}\ \textbf{1} Copied fraudulently; fake. {\fontspec{DejaVu Sans}◇} \textit{they have illegally entered the UK using forged travel documents}}{}{}{}%
\par%
\entry{formidable}{/ˈfɔːmɪdəb(ə)l/}{দুর্দান্ত}{ \textsf{\textit{adjective}}\ \textbf{1} Inspiring fear or respect through being impressively large, powerful, intense, or capable. {\fontspec{DejaVu Sans}◇} \textit{a formidable opponent} \colorBulletS{SYN} intimidating, forbidding, redoubtable, daunting, alarming, frightening, terrifying, petrifying, horrifying, chilling, disturbing, disquieting, dreadful, brooding, awesome, fearsome, ominous, foreboding, sinister, menacing, mean{-}looking, threatening, dangerous}{}{}{ \colorBullet{ORIGIN} Late Middle English from French, or from Latin formidabilis, from formidare ‘to fear’.}%
\par%
\entry{fountainhead}{/ˈfaʊntɪnhɛd/}{উৎস}{ \textsf{\textit{noun}}\ \textbf{1} An original source of something. {\fontspec{DejaVu Sans}◇} \textit{he was the sole fountainhead of advice} \colorBulletS{SYN} source, fount, fountainhead, well head, wellspring, well}{}{}{}%
\par%
\entry{fowl}{/faʊl/}{পাখি}{ \textsf{\textit{noun}}\ \textbf{1}  {\fontspec{DejaVu Sans}◇} \textit{} \colorBulletS{SYN} poultry}{}{}{ \colorBullet{ORIGIN} Old English fugol ‘bird’, of Germanic origin; related to Dutch vogel and German Vogel, also to fly.}%
\par%
\entry{fragile}{/ˈfradʒʌɪl/}{ভঙ্গুর}{ \textsf{\textit{adjective}}\ \textbf{1} (of an object) easily broken or damaged. {\fontspec{DejaVu Sans}◇} \textit{fragile items such as glass and china} \colorBulletS{SYN} breakable, easily broken, brittle, frangible, smashable, splintery, flimsy, weak, frail, insubstantial, delicate, dainty, fine}{}{}{ \colorBullet{ORIGIN} Late 15th century (in the sense ‘morally weak’): from Latin fragilis, from frangere ‘to break’. The sense ‘liable to break’ dates from the mid 16th century.}%
\par%
\entry{frame}{/freɪm/}{ফ্রেম}{\small{\textsf{\textit{noun, verb}}} \\{\fontspec{DejaVu Sans}▪ }\textsf{\textit{noun}}\\ \textbf{1} A rigid structure that surrounds something such as a picture, door, or windowpane. {\fontspec{DejaVu Sans}◇} \textit{} \colorBulletS{SYN} setting, mount, mounting, surround, fixture, support, stand \textbf{2} A person's body with reference to its size or build. {\fontspec{DejaVu Sans}◇} \textit{a shiver shook her slim frame} \colorBulletS{SYN} body, figure, form, shape, physique, build, size, proportions \textbf{3} A basic structure that underlies or supports a system, concept, or text. {\fontspec{DejaVu Sans}◇} \textit{the establishment of conditions provides a frame for interpretation} \colorBulletS{SYN} structure, framework, context \textbf{4} A structural environment within which a class of words or other linguistic units can be correctly used. For example I — him is a frame for a large class of transitive verbs. {\fontspec{DejaVu Sans}◇} \textit{} \textbf{5} A single complete picture in a series forming a cinema, television, or video film. {\fontspec{DejaVu Sans}◇} \textit{video footage slowed down to 20 frames a second} \textbf{6} The triangular structure for positioning the red balls in snooker. {\fontspec{DejaVu Sans}◇} \textit{} \\{\fontspec{DejaVu Sans}▪ }\textsf{\textit{verb}}\\ \textbf{1} Place (a picture or photograph) in a frame. {\fontspec{DejaVu Sans}◇} \textit{he had had the photo framed} \colorBulletS{SYN} mount, set in a frame \textbf{2} Formulate (a concept, plan, or system) {\fontspec{DejaVu Sans}◇} \textit{staff have proved invaluable in framing the proposals} \colorBulletS{SYN} formulate, draw up, plan, draft, map out, sketch out, work out, shape, compose, put together, arrange, form, devise, create, establish, conceive, think up, hatch, originate, orchestrate, engineer, organize, coordinate \textbf{3} Produce false evidence against (an innocent person) so that they appear guilty. {\fontspec{DejaVu Sans}◇} \textit{he claims he was framed} \colorBulletS{SYN} falsely incriminate, fabricate charges against, fabricate evidence against, entrap}{}{}{ \colorBullet{ORIGIN} Old English framian ‘be useful’, of Germanic origin and related to from. The general sense in Middle English, ‘make ready for use’, probably led to frame (sense 2 of the verb); it also gave rise to the specific meaning ‘prepare timber for use in building’, later ‘make the wooden parts (framework) of a building’, hence the noun sense ‘structure’ (late Middle English).}%
\par%
\entry{freak}{/friːk/}{খামখেয়াল}{\small{\textsf{\textit{noun, verb}}} \\{\fontspec{DejaVu Sans}▪ }\textsf{\textit{noun}}\\ \textbf{1} A very unusual and unexpected event or situation. {\fontspec{DejaVu Sans}◇} \textit{the teacher says the accident was a total freak} \colorBulletS{SYN} fluke, anomaly, aberration, rogue, rarity, quirk, oddity, unusual occurrence, peculiar turn of events, twist of fate \textbf{2}  {\fontspec{DejaVu Sans}◇} \textit{a few freaks have been discovered, one amazing cat tipping the scales at no less than 43 lbs} \colorBulletS{SYN} aberration, abnormality, irregularity, oddity, monster, monstrosity, malformation, mutant \textbf{3} A person who is obsessed with a particular activity or interest. {\fontspec{DejaVu Sans}◇} \textit{a fitness freak} \colorBulletS{SYN} enthusiast, fan, fanatic, addict, devotee, lover \textbf{4} A sudden arbitrary change of mind; a whim. {\fontspec{DejaVu Sans}◇} \textit{follow this way or that, as the freak takes you} \colorBulletS{SYN} whim, whimsy, fancy, fad, vagary, notion, conceit, caprice, kink, twist, freak, fetish, passion, bent, foible, quirk, eccentricity, idiosyncrasy \\{\fontspec{DejaVu Sans}▪ }\textsf{\textit{verb}}\\ \textbf{1} Behave or cause to behave in a wild and irrational way, typically because of the effects of extreme emotion or drugs. {\fontspec{DejaVu Sans}◇} \textit{he freaked out and smashed the place up} \colorBulletS{SYN} go crazy, go mad, go out of one's mind, go to pieces, crack, snap, lose control, lose one's self{-}control, lose control of the situation, act wildly \textbf{2} Fleck or streak randomly. {\fontspec{DejaVu Sans}◇} \textit{the white pink and the pansy freaked with jet} \colorBulletS{SYN} stripe, band, bar, fleck}{}{}{ \colorBullet{ORIGIN} Mid 16th century (in freak (sense 4 of the noun)): probably from a dialect word.}%
\par%
\entry{frighten}{/ˈfrʌɪt(ə)n/}{আতঙ্কিত}{ \textsf{\textit{verb}}\ \textbf{1} Make (someone) afraid or anxious. {\fontspec{DejaVu Sans}◇} \textit{the savagery of his thoughts frightened him} \colorBulletS{SYN} scare, startle, alarm, terrify, petrify, shock, chill, appal, agitate, panic, throw into panic, fluster, ruffle, shake, disturb, disconcert, unnerve, unman, intimidate, terrorize, cow, daunt, dismay}{}{}{}%
\par%
\entry{frontier}{/ˈfrʌntɪə/}{সীমান্ত}{ \textsf{\textit{noun}}\ \textbf{1} A line or border separating two countries. {\fontspec{DejaVu Sans}◇} \textit{international crime knows no frontiers} \colorBulletS{SYN} border, boundary, partition, borderline, dividing line, bounding line, demarcation line}{}{}{ \colorBullet{ORIGIN} Late Middle English from Old French frontiere, based on Latin frons, front{-} ‘front’.}%
\par%
\entry{frustrate}{/frʌˈstreɪt/}{হতাশ}{\small{\textsf{\textit{adjective, verb}}} \\{\fontspec{DejaVu Sans}▪ }\textsf{\textit{adjective}}\\ \textbf{1} Frustrated. {\fontspec{DejaVu Sans}◇} \textit{} \\{\fontspec{DejaVu Sans}▪ }\textsf{\textit{verb}}\\ \textbf{1} Prevent (a plan or attempted action) from progressing, succeeding, or being fulfilled. {\fontspec{DejaVu Sans}◇} \textit{the rescue attempt was frustrated by bad weather} \colorBulletS{SYN} thwart, defeat, foil, block, stop, put a stop to, counter, spoil, check, balk, circumvent, disappoint, forestall, bar, dash, scotch, quash, crush, derail, nip in the bud, baffle, nullify, snooker \textbf{2} Cause (someone) to feel upset or annoyed as a result of being unable to change or achieve something. {\fontspec{DejaVu Sans}◇} \textit{it frustrated me that more couldn't be done for her} \colorBulletS{SYN} exasperate, infuriate, annoy, anger, madden, vex, irritate, irk, embitter, sour, get someone's back up, try someone's patience}{}{}{ \colorBullet{ORIGIN} Late Middle English from Latin frustrat{-} ‘disappointed’, from the verb frustrare, from frustra ‘in vain’.}%
\par%
\entry{frustration}{/frʌˈstreɪʃn/}{পরাজয়}{ \textsf{\textit{noun}}\ \textbf{1} The feeling of being upset or annoyed as a result of being unable to change or achieve something. {\fontspec{DejaVu Sans}◇} \textit{tears of frustration rolled down her cheeks} \colorBulletS{SYN} exasperation, annoyance, anger, vexation, irritation, bitterness, resentment \textbf{2} The prevention of the progress, success, or fulfilment of something. {\fontspec{DejaVu Sans}◇} \textit{the frustration of their wishes} \colorBulletS{SYN} thwarting, defeat, foiling, blocking, stopping, countering, spoiling, checking, balking, circumvention, forestalling, dashing, scotching, quashing, crushing}{}{}{ \colorBullet{ORIGIN} Mid 16th century from Latin frustratio(n{-}), from frustrare ‘disappoint’ (see frustrate).}%
\par%
\entry{fugitive}{/ˈfjuːdʒɪtɪv/}{পলাতক}{\small{\textsf{\textit{adjective, noun}}} \\{\fontspec{DejaVu Sans}▪ }\textsf{\textit{adjective}}\\ \textbf{1} Quick to disappear; fleeting. {\fontspec{DejaVu Sans}◇} \textit{the fugitive effects of light} \colorBulletS{SYN} fleeting, transient, transitory, ephemeral, evanescent, flitting, flying, fading, momentary, short{-}lived, short, brief, passing, impermanent, fly{-}by{-}night, here today and gone tomorrow \\{\fontspec{DejaVu Sans}▪ }\textsf{\textit{noun}}\\ \textbf{1} A person who has escaped from captivity or is in hiding. {\fontspec{DejaVu Sans}◇} \textit{fugitives from justice} \colorBulletS{SYN} escapee, escaper, runaway, deserter, refugee, renegade, absconder}{}{Fugitive abu borhan chowdhury, chairman of everest holding and technologies ltd, was convicted in a graft case filed over misappropriation of tk 15 core from rupali bank…}{ \colorBullet{ORIGIN} Late Middle English from Old French fugitif, {-}ive, from Latin fugitivus, from fugere ‘flee’.}%
\par%
\entry{fume}{/fjuːm/}{ধূম্র}{\small{\textsf{\textit{noun, verb}}} \\{\fontspec{DejaVu Sans}▪ }\textsf{\textit{noun}}\\ \textbf{1} An amount of gas or vapour that smells strongly or is dangerous to inhale. {\fontspec{DejaVu Sans}◇} \textit{clouds of exhaust fumes spewed by cars} \colorBulletS{SYN} smoke, vapour, gas, exhalation, exhaust, effluvium, pollution \\{\fontspec{DejaVu Sans}▪ }\textsf{\textit{verb}}\\ \textbf{1} Feel, show, or express great anger. {\fontspec{DejaVu Sans}◇} \textit{‘We simply cannot have this’, she fumed} \colorBulletS{SYN} be furious, be enraged, be angry, seethe, smoulder, simmer, boil, be livid, be incensed, bristle, be beside oneself, spit, chafe \textbf{2} Emit gas or vapour. {\fontspec{DejaVu Sans}◇} \textit{fragments of lava hit the ground, fuming and sizzling} \colorBulletS{SYN} emit smoke, emit gas, smoke}{}{}{ \colorBullet{ORIGIN} Late Middle English from Old French fumer (verb), from Latin fumare ‘to smoke’.}%
\par%
\entry{furthermore}{/fəːðəˈmɔː/}{তদ্ব্যতীত}{ \textsf{\textit{adverb}}\ \textbf{1} In addition; besides (used to introduce a fresh consideration in an argument) {\fontspec{DejaVu Sans}◇} \textit{It was also a highly desirable political end. Furthermore, it gave the English a door into France} \colorBulletS{SYN} moreover, further, what's more, also, additionally, in addition, besides, as well, too, to boot, on top of that, over and above that, into the bargain, by the same token}{}{}{}%
\par%
\entry{furtive}{/ˈfəːtɪv/}{অলক্ষিত}{ \textsf{\textit{adjective}}\ \textbf{1} Attempting to avoid notice or attention, typically because of guilt or a belief that discovery would lead to trouble; secretive. {\fontspec{DejaVu Sans}◇} \textit{they spent a furtive day together} \colorBulletS{SYN} secretive, secret, surreptitious}{}{}{ \colorBullet{ORIGIN} Early 17th century from French furtif, {-}ive or Latin furtivus, from furtum ‘theft’.}%
\par%
\entry{fury}{/ˈfjʊəri/}{উন্মত্ততা}{ \textsf{\textit{noun}}\ \textbf{1} Wild or violent anger. {\fontspec{DejaVu Sans}◇} \textit{tears of fury and frustration} \colorBulletS{SYN} rage, anger, wrath, passion, outrage, spleen, temper, savagery, frenzy, madness \textbf{2} Extreme strength or violence in an action or a natural phenomenon. {\fontspec{DejaVu Sans}◇} \textit{the fury of a gathering storm} \colorBulletS{SYN} fierceness, ferocity, violence, turbulence, tempestuousness, savagery \textbf{3}  {\fontspec{DejaVu Sans}◇} \textit{}}{}{}{ \colorBullet{ORIGIN} Late Middle English from Old French furie, from Latin furia, from furiosus ‘furious’, from furere ‘be mad, rage’.}%
\par%
\entry{fuselage}{/ˈfjuːzəlɑːʒ/}{বিমানপোতের কাঠাম}{ \textsf{\textit{noun}}\ \textbf{1} The main body of an aircraft. {\fontspec{DejaVu Sans}◇} \textit{} \colorBulletS{SYN} framework, frame, skeleton, shell, casing, structure, substructure, bodywork, body}{}{}{ \colorBullet{ORIGIN} Early 20th century from French, from fuseler ‘shape into a spindle’, from fuseau ‘spindle’.}%
\par%
\end{multicols}%
\pagebreak%
\section*{G}%
\begin{multicols}{2}%
\entry{galloping}{/ˈɡaləpɪŋ/}{দ্রুতগতিতে বৃদ্ধি পায় এমন}{ \textsf{\textit{adjective}}\ \textbf{1} (of a horse) going at the pace of a gallop. {\fontspec{DejaVu Sans}◇} \textit{the sound of galloping hooves} \textbf{2} (of a process or event) progressing in a rapid and seemingly uncontrollable manner. {\fontspec{DejaVu Sans}◇} \textit{galloping inflation}}{}{}{}%
\par%
\entry{gamble}{/ˈɡamb(ə)l/}{জুয়া}{\small{\textsf{\textit{noun, verb}}} \\{\fontspec{DejaVu Sans}▪ }\textsf{\textit{noun}}\\ \textbf{1} An act of gambling. {\fontspec{DejaVu Sans}◇} \textit{Dad likes a bit of a gamble} \colorBulletS{SYN} bet, wager, speculation \textbf{2} A risky action undertaken with the hope of success. {\fontspec{DejaVu Sans}◇} \textit{we decided to take a gamble and offer him a place on our staff} \colorBulletS{SYN} risk, chance, hazard, speculation, venture, random shot, leap in the dark \\{\fontspec{DejaVu Sans}▪ }\textsf{\textit{verb}}\\ \textbf{1} Play games of chance for money; bet. {\fontspec{DejaVu Sans}◇} \textit{he gambles on football} \colorBulletS{SYN} bet, wager, place a bet, lay a bet, stake money on something, back the horses, try one's luck on the horses \textbf{2} Take risky action in the hope of a desired result. {\fontspec{DejaVu Sans}◇} \textit{he was gambling on the success of his satellite TV channel} \colorBulletS{SYN} take a chance, take a risk, take a leap in the dark, leave things to chance, speculate, venture, buy a pig in a poke}{}{}{ \colorBullet{ORIGIN} Early 18th century from obsolete gamel ‘play games’, or from the verb game.}%
\par%
\entry{garbled}{/ˈɡɑːb(ə)ld/}{বিকৃত হয়ে}{ \textsf{\textit{adjective}}\ \textbf{1} (of a message, sound, or transmission) confused and distorted; unclear. {\fontspec{DejaVu Sans}◇} \textit{I got a garbled set of directions}}{}{}{}%
\par%
\entry{garlic}{/ˈɡɑːlɪk/}{রসুন}{ \textsf{\textit{noun}}\ \textbf{1} A strong{-}smelling pungent{-}tasting bulb, used as a flavouring in cooking and in herbal medicine. {\fontspec{DejaVu Sans}◇} \textit{garlic butter} \textbf{2} The central Asian plant, closely related to the onion, which produces garlic. {\fontspec{DejaVu Sans}◇} \textit{}}{}{}{ \colorBullet{ORIGIN} Old English gārlēac, from gār ‘spear’ (because the shape of a clove resembles the head of a spear) + lēac ‘leek’.}%
\par%
\entry{gauge}{/ɡeɪdʒ/}{হিসাব করার নিয়ম}{\small{\textsf{\textit{noun, verb}}} \\{\fontspec{DejaVu Sans}▪ }\textsf{\textit{noun}}\\ \textbf{1} An instrument that measures and gives a visual display of the amount, level, or contents of something. {\fontspec{DejaVu Sans}◇} \textit{a fuel gauge} \colorBulletS{SYN} measuring instrument, measuring device, meter, measure \textbf{2} The thickness, size, or capacity of something, especially as a standard measure. {\fontspec{DejaVu Sans}◇} \textit{} \colorBulletS{SYN} size, measure, extent, degree, scope, capacity, magnitude \textbf{3} The position of a sailing ship to windward (the weather gage) or leeward (the lee gage) of another. {\fontspec{DejaVu Sans}◇} \textit{the French fleet was heavily outnumbered but had the weather gage} \\{\fontspec{DejaVu Sans}▪ }\textsf{\textit{verb}}\\ \textbf{1} Estimate or determine the amount, level, or volume of. {\fontspec{DejaVu Sans}◇} \textit{astronomers can gauge the star's intrinsic brightness} \colorBulletS{SYN} compute, calculate, work out \textbf{2} Measure the dimensions of (an object) with a gauge. {\fontspec{DejaVu Sans}◇} \textit{when dry the assemblies can be gauged exactly} \colorBulletS{SYN} measure, calculate, compute, work out, determine, ascertain}{}{}{ \colorBullet{ORIGIN} Middle English (denoting a standard measure): from Old French gauge (noun), gauger (verb), variant of Old Northern French jauge (noun), jauger (verb), of unknown origin.}%
\par%
\entry{generosity}{/dʒɛnəˈrɒsəti/}{দাক্ষিণ্য}{ \textsf{\textit{noun}}\ \textbf{1} The quality of being kind and generous. {\fontspec{DejaVu Sans}◇} \textit{I was overwhelmed by the generosity of friends and neighbours} \colorBulletS{SYN} liberality, lavishness, magnanimity, magnanimousness, munificence, open{-}handedness, free{-}handedness, bounty, unselfishness, indulgence, prodigality, princeliness \textbf{2} The quality or fact of being plentiful or large. {\fontspec{DejaVu Sans}◇} \textit{diners certainly cannot complain about the generosity of portions} \colorBulletS{SYN} abundance, plentifulness, copiousness, amplitude, profuseness, richness, lavishness, liberality, munificence, largeness, superabundance, infinity, inexhaustibility, opulence}{}{}{ \colorBullet{ORIGIN} Late Middle English (denoting nobility of birth): from Latin generositas, from generosus ‘magnanimous’ (see generous). Current senses date from the 17th century.}%
\par%
\entry{genital}{/ˈdʒɛnɪt(ə)l/}{জনন সম্বন্ধীয়}{\small{\textsf{\textit{adjective, noun}}} \\{\fontspec{DejaVu Sans}▪ }\textsf{\textit{adjective}}\\ \textbf{1} Relating to the human or animal reproductive organs. {\fontspec{DejaVu Sans}◇} \textit{the genital area} \colorBulletS{SYN} generative, procreative, propagative \\{\fontspec{DejaVu Sans}▪ }\textsf{\textit{noun}}\\ \textbf{1} A person's or animal's external organs of reproduction. {\fontspec{DejaVu Sans}◇} \textit{} \colorBulletS{SYN} private parts, genitalia, sexual organs, reproductive organs, pudenda, nether regions, crotch, groin}{}{}{ \colorBullet{ORIGIN} Late Middle English from Old French, or from Latin genitalis, from genitus, past participle of gignere ‘beget’.}%
\par%
\entry{gesture}{/ˈdʒɛstʃə/}{অঙ্গভঙ্গি}{\small{\textsf{\textit{noun, verb}}} \\{\fontspec{DejaVu Sans}▪ }\textsf{\textit{noun}}\\ \textbf{1} A movement of part of the body, especially a hand or the head, to express an idea or meaning. {\fontspec{DejaVu Sans}◇} \textit{Alex made a gesture of apology} \colorBulletS{SYN} signal, signalling, sign, signing, motion, motioning, wave, indication, gesticulation \\{\fontspec{DejaVu Sans}▪ }\textsf{\textit{verb}}\\ \textbf{1} Make a gesture. {\fontspec{DejaVu Sans}◇} \textit{she gestured meaningfully with the pistol}}{}{}{ \colorBullet{ORIGIN} Late Middle English from medieval Latin gestura, from Latin gerere ‘bear, wield, perform’. The original sense was ‘bearing, deportment’, hence ‘the use of posture and bodily movements for effect in oratory’.}%
\par%
\entry{ginger}{/ˈdʒɪndʒə/}{আদা}{\small{\textsf{\textit{adjective, noun, verb}}} \\{\fontspec{DejaVu Sans}▪ }\textsf{\textit{adjective}}\\ \textbf{1} (chiefly of hair or fur) of a light reddish{-}yellow or orange{-}brown colour. {\fontspec{DejaVu Sans}◇} \textit{} \colorBulletS{SYN} reddish brown, tawny, chestnut, russet, coppery, copper, auburn, Titian, reddish, ginger, gingery, rusty, rufous \\{\fontspec{DejaVu Sans}▪ }\textsf{\textit{noun}}\\ \textbf{1} A hot, fragrant spice made from the rhizome of a plant, which may be chopped or powdered for cooking, preserved in syrup, or candied. {\fontspec{DejaVu Sans}◇} \textit{} \colorBulletS{SYN} flavour, taste, savour \textbf{2} A SE Asian plant, which resembles bamboo in appearance, from which ginger is taken. {\fontspec{DejaVu Sans}◇} \textit{} \textbf{3} A light reddish{-}yellow or orange{-}brown colour. {\fontspec{DejaVu Sans}◇} \textit{} \textbf{4} A quality of energy or spiritedness. {\fontspec{DejaVu Sans}◇} \textit{the ginger had gone out of the men} \\{\fontspec{DejaVu Sans}▪ }\textsf{\textit{verb}}\\ \textbf{1} Flavour with ginger. {\fontspec{DejaVu Sans}◇} \textit{gingered chicken wings} \textbf{2} Make someone or something more lively. {\fontspec{DejaVu Sans}◇} \textit{she slapped his hand lightly to ginger him up} \colorBulletS{SYN} encourage, act as a fillip to, act as a impetus to, act as a incentive to, act as a spur to, act as a stimulus to, prompt, prod, move, motivate, trigger, spark, spur on, galvanize, activate, kindle, fire, fire with enthusiasm, fuel, whet, nourish}{}{}{ \colorBullet{ORIGIN} Late Old English gingifer, conflated in Middle English with Old French gingimbre, from medieval Latin gingiber, from Greek zingiberis, from Pali siṅgivera, of Dravidian origin.}%
\par%
\entry{gladiator}{/ˈɡladɪeɪtə/}{প্রাচীন রোমের মল্লযোদ্ধা}{ \textsf{\textit{noun}}\ \textbf{1} (in ancient Rome) a man trained to fight with weapons against other men or wild animals in an arena. {\fontspec{DejaVu Sans}◇} \textit{}}{}{}{ \colorBullet{ORIGIN} Late Middle English from Latin, from gladius ‘sword’.}%
\par%
\entry{glimpse}{/ɡlɪm(p)s/}{আভাস}{\small{\textsf{\textit{noun, verb}}} \\{\fontspec{DejaVu Sans}▪ }\textsf{\textit{noun}}\\ \textbf{1} A momentary or partial view. {\fontspec{DejaVu Sans}◇} \textit{she caught a glimpse of the ocean} \colorBulletS{SYN} brief look, quick look \\{\fontspec{DejaVu Sans}▪ }\textsf{\textit{verb}}\\ \textbf{1} See or perceive briefly or partially. {\fontspec{DejaVu Sans}◇} \textit{he glimpsed a figure standing in the shade} \colorBulletS{SYN} catch sight of, catch a glimpse of, get a glimpse of, see briefly, get a sight of, notice, discern, spot, spy, sight, note, pick out, make out}{ \colorBullet{OTHER} glimpse into}{1. Europe's top central bankers who met their global peers in japan this weekend may have caught a glimpse of their own future. 2. A glimpse into the pre{-}modern islamic culture in bengal.}{ \colorBullet{ORIGIN} Middle English (in the sense ‘shine faintly’): probably of Germanic origin; related to Middle High German glimsen, also to glimmer.}%
\par%
\entry{gloom}{/ɡluːm/}{বিষাদ}{\small{\textsf{\textit{noun, verb}}} \\{\fontspec{DejaVu Sans}▪ }\textsf{\textit{noun}}\\ \textbf{1} Partial or total darkness. {\fontspec{DejaVu Sans}◇} \textit{he strained his eyes peering into the gloom} \colorBulletS{SYN} darkness, semi{-}darkness, dark, gloominess, dimness, blackness, murkiness, murk, shadows, shade, shadiness, obscurity \textbf{2} A state of depression or despondency. {\fontspec{DejaVu Sans}◇} \textit{a year of economic gloom for the car industry} \colorBulletS{SYN} despondency, depression, dejection, downheartedness, dispiritedness, heavy{-}heartedness, melancholy, melancholia, unhappiness, sadness, glumness, gloominess, low spirits, dolefulness, misery, sorrow, sorrowfulness, forlornness, woefulness, woe, wretchedness, lugubriousness, moroseness, mirthlessness, cheerlessness \\{\fontspec{DejaVu Sans}▪ }\textsf{\textit{verb}}\\ \textbf{1} Have a dark or sombre appearance. {\fontspec{DejaVu Sans}◇} \textit{the black gibbet glooms beside the way} \textbf{2} Be or look depressed or despondent. {\fontspec{DejaVu Sans}◇} \textit{Charles was always glooming about money}}{}{}{ \colorBullet{ORIGIN} Late Middle English (as a verb): of unknown origin.}%
\par%
\entry{glum}{/ɡlʌm/}{বিষাদগ্রস্ত}{ \textsf{\textit{adjective}}\ \textbf{1} Looking or feeling dejected; morose. {\fontspec{DejaVu Sans}◇} \textit{the princess looked glum but later cheered up} \colorBulletS{SYN} gloomy, downcast, downhearted, dejected, disconsolate, dispirited, despondent, crestfallen, cast down, depressed, disappointed, disheartened, discouraged, demoralized, desolate, heavy{-}hearted, in low spirits, low{-}spirited, sad, unhappy, doleful, melancholy, miserable, woebegone, mournful, forlorn, long{-}faced, fed up, in the doldrums, wretched, lugubrious, morose, sepulchral, saturnine, dour, mirthless}{}{}{ \colorBullet{ORIGIN} Mid 16th century related to dialect glum ‘to frown’, variant of gloom.}%
\par%
\entry{gobble}{/ˈɡɒb(ə)l/}{গরগর শব্দ}{ \textsf{\textit{verb}}\ \textbf{1} Eat (something) hurriedly and noisily. {\fontspec{DejaVu Sans}◇} \textit{he gobbled up the rest of his sandwich} \colorBulletS{SYN} eat greedily, eat hungrily, guzzle, bolt, gulp, swallow hurriedly, devour, wolf, cram, gorge on, gorge oneself on, gorge oneself}{}{}{ \colorBullet{ORIGIN} Early 17th century probably from gob.}%
\par%
\entry{gobble}{/ˈɡɒb(ə)l/}{গরগর শব্দ}{ \textsf{\textit{verb}}\ \textbf{1} (of a turkeycock) make a characteristic swallowing sound in the throat. {\fontspec{DejaVu Sans}◇} \textit{}}{}{}{ \colorBullet{ORIGIN} Late 17th century imitative, perhaps influenced by gobble.}%
\par%
\entry{godson}{/ˈɡɒdsʌn/}{ধর্মপুত্র}{ \textsf{\textit{noun}}\ \textbf{1} A male godchild. {\fontspec{DejaVu Sans}◇} \textit{Freddie was a godson of his father's closest friend}}{}{}{}%
\par%
\entry{goombah}{/ɡuːmˈbɑː/}{}{ \textsf{\textit{noun}}\ \textbf{1} An associate or accomplice, especially a senior member of a criminal gang. {\fontspec{DejaVu Sans}◇} \textit{}}{}{}{ \colorBullet{ORIGIN} 1960s probably a dialect alteration of Italian compare ‘godfather, friend, accomplice’.}%
\par%
\entry{goon}{/ɡuːn/}{গণ্ডমূর্খ}{ \textsf{\textit{noun}}\ \textbf{1} A silly, foolish, or eccentric person. {\fontspec{DejaVu Sans}◇} \textit{} \colorBulletS{SYN} idiot, ass, halfwit, nincompoop, blockhead, buffoon, dunce, dolt, ignoramus, cretin, imbecile, dullard, moron, simpleton, clod \textbf{2} A bully or thug, especially a member of an armed or security force. {\fontspec{DejaVu Sans}◇} \textit{a squad of goons waving pistols} \colorBulletS{SYN} thug, roughneck, scoundrel, villain, rogue, rascal, lout, hooligan, hoodlum, vandal, delinquent, rowdy, bully boy, bully, brute \textbf{3} A guard in a German prisoner{-}of{-}war camp during the Second World War. {\fontspec{DejaVu Sans}◇} \textit{}}{}{}{ \colorBullet{ORIGIN} Mid 19th century perhaps from dialect gooney ‘booby’; influenced by the subhuman cartoon character ‘Alice the Goon’, created by E. C. Segar (1894–1938), American cartoonist.}%
\par%
\entry{goon}{/ɡuːn/}{গণ্ডমূর্খ}{ \textsf{\textit{noun}}\ \textbf{1} Cheap wine, especially when sold in large cartons. {\fontspec{DejaVu Sans}◇} \textit{we sat in the kitchen drinking the rest of the goon}}{}{}{ \colorBullet{ORIGIN} 1980s probably an alteration of flagon, possibly influenced by goom.}%
\par%
\entry{grab}{/ɡrab/}{দখল}{\small{\textsf{\textit{noun, verb}}} \\{\fontspec{DejaVu Sans}▪ }\textsf{\textit{noun}}\\ \textbf{1} A quick sudden clutch or attempt to seize. {\fontspec{DejaVu Sans}◇} \textit{he made a grab at the pistol} \colorBulletS{SYN} lunge for, attempt to grab \textbf{2} A mechanical device for clutching, lifting, and moving things, especially materials in bulk. {\fontspec{DejaVu Sans}◇} \textit{The lessons had still not been learned by November the following year, when the mechanical grab ripped up part of a late medieval barge near Trig Stairs.} \\{\fontspec{DejaVu Sans}▪ }\textsf{\textit{verb}}\\ \textbf{1} Grasp or seize suddenly and roughly. {\fontspec{DejaVu Sans}◇} \textit{she grabbed him by the shirt collar} \colorBulletS{SYN} seize, grasp, snatch, seize hold of, grab hold of, take hold of, catch hold of, lay hold of, lay hands on, lay one's hands on, get one's hands on, take a grip of, fasten round, grapple, grip, clasp, clutch \textbf{2} Attract the attention of; make an impression on. {\fontspec{DejaVu Sans}◇} \textit{how does that grab you?} \colorBulletS{SYN} make an impression on, have an impact on, influence, affect, leave a mark on, move, stir, rouse, excite, inspire, galvanize}{}{}{ \colorBullet{ORIGIN} Late 16th century from Middle Low German and Middle Dutch grabben; perhaps related to grip, gripe, and grope.}%
\par%
\entry{grace}{/ɡreɪs/}{অনুগ্রহ}{\small{\textsf{\textit{noun, verb}}} \\{\fontspec{DejaVu Sans}▪ }\textsf{\textit{noun}}\\ \textbf{1} Smoothness and elegance of movement. {\fontspec{DejaVu Sans}◇} \textit{she moved through the water with effortless grace} \colorBulletS{SYN} elegance, stylishness, poise, finesse, charm \textbf{2} Courteous good will. {\fontspec{DejaVu Sans}◇} \textit{he had the good grace to apologize to her afterwards} \colorBulletS{SYN} courtesy, courteousness, politeness, manners, good manners, mannerliness, civility, decorum, decency, propriety, breeding, respect, respectfulness \textbf{3} (in Christian belief) the free and unmerited favour of God, as manifested in the salvation of sinners and the bestowal of blessings. {\fontspec{DejaVu Sans}◇} \textit{} \colorBulletS{SYN} favour, good will, generosity, kindness, benefaction, beneficence, indulgence \textbf{4} A period officially allowed for payment of a sum due or for compliance with a law or condition, especially an extended period granted as a special favour. {\fontspec{DejaVu Sans}◇} \textit{we'll give them 30 days' grace and then we'll be doing checks} \colorBulletS{SYN} deferment, deferral, postponement, suspension, putting back, putting off, adjournment, delay, shelving, rescheduling, interruption, arrest, pause \textbf{5} A short prayer of thanks said before or after a meal. {\fontspec{DejaVu Sans}◇} \textit{} \colorBulletS{SYN} prayer of thanks, thanksgiving, blessing, benediction \textbf{6} Used as forms of description or address for a duke, duchess, or archbishop. {\fontspec{DejaVu Sans}◇} \textit{His Grace, the Duke of Atholl} \textbf{7} (in Greek mythology) three beautiful goddesses (Aglaia, Thalia, and Euphrosyne) believed to personify and bestow charm, grace, and beauty. {\fontspec{DejaVu Sans}◇} \textit{} \\{\fontspec{DejaVu Sans}▪ }\textsf{\textit{verb}}\\ \textbf{1} Bring honour or credit to (someone or something) by one's attendance or participation. {\fontspec{DejaVu Sans}◇} \textit{he is one of the best players ever to have graced the game} \colorBulletS{SYN} dignify, distinguish, add distinction to, add dignity to, honour, bestow honour on, favour, enhance, add lustre to, magnify, ennoble, glorify, elevate, make lofty, aggrandize, upgrade}{}{}{ \colorBullet{ORIGIN} Middle English via Old French from Latin gratia, from gratus ‘pleasing, thankful’; related to grateful.}%
\par%
\entry{graft}{/ɡrɑːft/}{ঘুস}{\small{\textsf{\textit{noun, verb}}} \\{\fontspec{DejaVu Sans}▪ }\textsf{\textit{noun}}\\ \textbf{1} A shoot or twig inserted into a slit on the trunk or stem of a living plant, from which it receives sap. {\fontspec{DejaVu Sans}◇} \textit{} \colorBulletS{SYN} scion, cutting, shoot, offshoot, bud, slip, new growth, sprout, sprig \textbf{2} A piece of living tissue that is transplanted surgically. {\fontspec{DejaVu Sans}◇} \textit{} \colorBulletS{SYN} transplant, implant, implantation \\{\fontspec{DejaVu Sans}▪ }\textsf{\textit{verb}}\\ \textbf{1} Insert (a shoot or twig) as a graft. {\fontspec{DejaVu Sans}◇} \textit{it was common to graft different varieties on to a single tree trunk} \textbf{2} Transplant (living tissue) as a graft. {\fontspec{DejaVu Sans}◇} \textit{they can graft a new hand on to the nerve ends} \colorBulletS{SYN} transplant, implant, transfer \textbf{3} Combine or integrate (an idea, system, etc.) with another, typically in a way considered inappropriate. {\fontspec{DejaVu Sans}◇} \textit{old values have been grafted on to a new economic class} \colorBulletS{SYN} fasten, attach, add, fix, join, insert}{}{}{ \colorBullet{ORIGIN} Late Middle English graff, from Old French grafe, via Latin from Greek graphion ‘stylus, writing implement’ (with reference to the tapered tip of the scion), from graphein ‘write’. The final {-}t is typical of phonetic confusion between {-}f and {-}ft at the end of words; compare with tuft.}%
\par%
\entry{graft}{/ɡrɑːft/}{ঘুস}{\small{\textsf{\textit{noun, verb}}} \\{\fontspec{DejaVu Sans}▪ }\textsf{\textit{noun}}\\ \textbf{1} Bribery and other corrupt practices used to secure illicit advantages or gains in politics or business. {\fontspec{DejaVu Sans}◇} \textit{sweeping measures to curb official graft} \colorBulletS{SYN} corruption, bribery, bribing, dishonesty, deceit, fraud, fraudulence, subornation, unlawful practices, illegal means, underhand means \\{\fontspec{DejaVu Sans}▪ }\textsf{\textit{verb}}\\ \textbf{1} Make money by shady or dishonest means. {\fontspec{DejaVu Sans}◇} \textit{}}{}{}{ \colorBullet{ORIGIN} Mid 19th century of unknown origin.}%
\par%
\entry{graft}{/ɡrɑːft/}{ঘুস}{\small{\textsf{\textit{noun, verb}}} \\{\fontspec{DejaVu Sans}▪ }\textsf{\textit{noun}}\\ \textbf{1} Hard work. {\fontspec{DejaVu Sans}◇} \textit{success came after years of hard graft} \colorBulletS{SYN} work, effort, endeavour, toil, labour, exertion, the sweat of one's brow, drudgery, donkey work \\{\fontspec{DejaVu Sans}▪ }\textsf{\textit{verb}}\\ \textbf{1} Work hard. {\fontspec{DejaVu Sans}◇} \textit{I need people prepared to go out and graft} \colorBulletS{SYN} work hard, exert oneself, toil, labour, hammer away, grind away, sweat}{}{}{ \colorBullet{ORIGIN} Mid 19th century perhaps related to the phrase spade's graft ‘the amount of earth that one stroke of a spade will move’, based on Old Norse grǫftr ‘digging’.}%
\par%
\entry{grapevine}{/ˈɡreɪpvʌɪn/}{দ্রাক্ষালতা}{ \textsf{\textit{noun}}\ \textbf{1} A vine native to both Eurasia and North America, especially one bearing grapes used for eating or winemaking. {\fontspec{DejaVu Sans}◇} \textit{} \textbf{2} Used to refer to the circulation of rumours and unofficial information. {\fontspec{DejaVu Sans}◇} \textit{I'd heard on the grapevine that the business was nearly settled} \colorBulletS{SYN} system, complex, interconnected structure, interconnected system, complex arrangement, complex system, nexus, web}{}{}{}%
\par%
\entry{grapple}{/ˈɡrap(ə)l/}{কুস্তি করা}{\small{\textsf{\textit{noun, verb}}} \\{\fontspec{DejaVu Sans}▪ }\textsf{\textit{noun}}\\ \textbf{1} An act of grappling. {\fontspec{DejaVu Sans}◇} \textit{} \textbf{2} An instrument for seizing hold of something; a grappling hook. {\fontspec{DejaVu Sans}◇} \textit{} \\{\fontspec{DejaVu Sans}▪ }\textsf{\textit{verb}}\\ \textbf{1} Engage in a close fight or struggle without weapons; wrestle. {\fontspec{DejaVu Sans}◇} \textit{passers{-}by grappled with the man after the knife attack} \colorBulletS{SYN} wrestle, struggle, tussle \textbf{2} Seize or hold with a grappling hook. {\fontspec{DejaVu Sans}◇} \textit{This said, they grappled him with more than hundred hooks.}}{}{}{ \colorBullet{ORIGIN} Middle English (as a noun denoting a grappling hook): from Old French grapil, from Provençal, diminutive of grapa ‘hook’, of Germanic origin; related to grape. The verb dates from the mid 16th century.}%
\par%
\entry{grasp}{/ɡrɑːsp/}{উপলব্ধি}{\small{\textsf{\textit{noun, verb}}} \\{\fontspec{DejaVu Sans}▪ }\textsf{\textit{noun}}\\ \textbf{1} A firm hold or grip. {\fontspec{DejaVu Sans}◇} \textit{the child slipped from her grasp} \colorBulletS{SYN} grip, hold \\{\fontspec{DejaVu Sans}▪ }\textsf{\textit{verb}}\\ \textbf{1} Seize and hold firmly. {\fontspec{DejaVu Sans}◇} \textit{she grasped the bottle} \colorBulletS{SYN} grip, clutch, clasp, hold, clench, lay hold of}{}{}{ \colorBullet{ORIGIN} Late Middle English perhaps related to grope.}%
\par%
\entry{gratitude}{/ˈɡratɪtjuːd/}{কৃতজ্ঞতা}{ \textsf{\textit{noun}}\ \textbf{1} The quality of being thankful; readiness to show appreciation for and to return kindness. {\fontspec{DejaVu Sans}◇} \textit{she expressed her gratitude to the committee for their support} \colorBulletS{SYN} gratefulness, thankfulness, thanks, appreciation, recognition, acknowledgement, hat tip, credit, regard, respect}{}{}{ \colorBullet{ORIGIN} Late Middle English from Old French, or from medieval Latin gratitudo, from Latin gratus ‘pleasing, thankful’.}%
\par%
\entry{grave}{/ɡreɪv/}{}{ \textsf{\textit{noun}}\ \textbf{1} A hole dug in the ground to receive a coffin or dead body, typically marked by a stone or mound. {\fontspec{DejaVu Sans}◇} \textit{the coffin was lowered into the grave} \colorBulletS{SYN} burying place, tomb, sepulchre, vault, burial chamber, burial pit, mausoleum, crypt, catacomb}{}{The gravest ethno{-}religious cleansing of recent times}{ \colorBullet{ORIGIN} Old English græf, of Germanic origin; related to Dutch graf and German Grab.}%
\par%
\entry{grave}{/ɡreɪv/}{}{ \textsf{\textit{adjective}}\ \textbf{1} Giving cause for alarm; serious. {\fontspec{DejaVu Sans}◇} \textit{a matter of grave concern} \colorBulletS{SYN} serious, important, all{-}important, profound, significant, momentous, weighty, of great consequence \textbf{2} Serious or solemn in manner or appearance. {\fontspec{DejaVu Sans}◇} \textit{his face was grave} \colorBulletS{SYN} solemn, earnest, serious, sombre, sober, severe}{}{The gravest ethno{-}religious cleansing of recent times}{ \colorBullet{ORIGIN} Late 15th century (originally of a wound in the sense ‘severe, serious’): from Old French grave or Latin gravis ‘heavy, serious’.}%
\par%
\entry{grave}{/ɡreɪv/}{}{ \textsf{\textit{verb}}\ \textbf{1} Engrave (an inscription or image) on a surface. {\fontspec{DejaVu Sans}◇} \textit{marble graved with exquisite flower, human and animal forms}}{}{The gravest ethno{-}religious cleansing of recent times}{ \colorBullet{ORIGIN} Old English grafan ‘dig’, of Germanic origin; related to German graben, Dutch graven ‘dig’ and German begraben ‘bury’, also to grave and groove.}%
\par%
\entry{grave}{/ɡreɪv/}{}{ \textsf{\textit{verb}}\ \textbf{1} Clean (a ship's bottom) by burning off the accretions and then tarring it. {\fontspec{DejaVu Sans}◇} \textit{they graved the ship there and remained 26 days}}{}{The gravest ethno{-}religious cleansing of recent times}{ \colorBullet{ORIGIN} Late Middle English perhaps from French dialect grave, variant of Old French greve ‘shore’ (because originally the ship would have been run aground).}%
\par%
\entry{grave}{/ɡrɑːˈveɪ/}{}{ \textsf{\textit{adverb \& adjective}}\ \textbf{1} (as a direction) slowly; with solemnity. {\fontspec{DejaVu Sans}◇} \textit{}}{}{The gravest ethno{-}religious cleansing of recent times}{ \colorBullet{ORIGIN} Italian, ‘slow’.}%
\par%
\entry{graze}{/ɡreɪz/}{আচড়}{ \textsf{\textit{verb}}\ \textbf{1} (of cattle, sheep, etc.) eat grass in a field. {\fontspec{DejaVu Sans}◇} \textit{cattle graze on the open meadows} \colorBulletS{SYN} feed, eat, crop, browse, ruminate, pasture, nibble, take nourishment}{}{}{ \colorBullet{ORIGIN} Old English grasian, from græs ‘grass’.}%
\par%
\entry{graze}{/ɡreɪz/}{আচড়}{\small{\textsf{\textit{noun, verb}}} \\{\fontspec{DejaVu Sans}▪ }\textsf{\textit{noun}}\\ \textbf{1} A slight injury where the skin is scraped. {\fontspec{DejaVu Sans}◇} \textit{cuts and grazes on the skin} \colorBulletS{SYN} scratch, scrape, abrasion, cut, injury, sore \\{\fontspec{DejaVu Sans}▪ }\textsf{\textit{verb}}\\ \textbf{1} Scrape and break the surface of the skin of (a part of the body) {\fontspec{DejaVu Sans}◇} \textit{she fell down and grazed her knees} \colorBulletS{SYN} scrape, abrade, skin, scratch, chafe, bark, scuff, rasp, break the skin of, cut, nick, snick}{}{}{ \colorBullet{ORIGIN} Late 16th century perhaps a specific use of graze.}%
\par%
\entry{grief}{/ɡriːf/}{বিষাদ}{ \textsf{\textit{noun}}\ \textbf{1} Intense sorrow, especially caused by someone's death. {\fontspec{DejaVu Sans}◇} \textit{she was overcome with grief} \colorBulletS{SYN} sorrow, misery, sadness, anguish, pain, distress, agony, torment, affliction, suffering, heartache, heartbreak, broken{-}heartedness, heaviness of heart, woe, desolation, despondency, dejection, despair, angst, mortification \textbf{2} Trouble or annoyance. {\fontspec{DejaVu Sans}◇} \textit{we were too tired to cause any grief} \colorBulletS{SYN} trouble, annoyance, bother, irritation, vexation, harassment, nuisance}{}{}{ \colorBullet{ORIGIN} Middle English from Old French grief, from grever ‘to burden’ (see grieve).}%
\par%
\entry{grieve}{/ɡriːv/}{খিদ্যমান}{ \textsf{\textit{verb}}\ \textbf{1} Feel intense sorrow. {\fontspec{DejaVu Sans}◇} \textit{she grieved for her father} \colorBulletS{SYN} mourn, lament, be mournful, be sorrowful, sorrow, be sad, be miserable}{}{}{ \colorBullet{ORIGIN} Middle English (also in the sense ‘harm, oppress’): from Old French grever ‘burden, encumber’, based on Latin gravare, from gravis ‘heavy, grave’ (see grave).}%
\par%
\entry{grieve}{/ɡriːv/}{খিদ্যমান}{ \textsf{\textit{noun}}\ \textbf{1} An overseer, manager, or bailiff on a farm. {\fontspec{DejaVu Sans}◇} \textit{}}{}{}{ \colorBullet{ORIGIN} Late 15th century related to reeve.}%
\par%
\entry{groan}{/ɡrəʊn/}{গভীর আর্তনাদ}{\small{\textsf{\textit{noun, verb}}} \\{\fontspec{DejaVu Sans}▪ }\textsf{\textit{noun}}\\ \textbf{1} A deep inarticulate sound conveying pain, despair, pleasure, etc. {\fontspec{DejaVu Sans}◇} \textit{she lay back with a groan} \colorBulletS{SYN} moan, murmur, whine, whimper, mewl, bleat, sigh \textbf{2} A low creaking sound made by an object under pressure. {\fontspec{DejaVu Sans}◇} \textit{the protesting groan of timbers} \colorBulletS{SYN} creaking, creak, grating, grinding, jarring \\{\fontspec{DejaVu Sans}▪ }\textsf{\textit{verb}}\\ \textbf{1} Make a deep inarticulate sound conveying pain, despair, pleasure, etc. {\fontspec{DejaVu Sans}◇} \textit{Marty groaned and pulled the blanket over his head} \colorBulletS{SYN} moan, murmur, whine, whimper, mewl, bleat, sigh \textbf{2} (of an object) make a low creaking sound when pressure or weight is applied. {\fontspec{DejaVu Sans}◇} \textit{James slumped back into his chair, making it groan} \colorBulletS{SYN} creak, grate, grind, jar}{}{}{ \colorBullet{ORIGIN} Old English grānian, of Germanic origin; related to German greinen ‘grizzle, whine’, grinsen ‘grin’, also probably to grin.}%
\par%
\entry{grudge}{/ɡrʌdʒ/}{দ্বেষ; গাত্রদাহ}{\small{\textsf{\textit{noun, verb}}} \\{\fontspec{DejaVu Sans}▪ }\textsf{\textit{noun}}\\ \textbf{1} A persistent feeling of ill will or resentment resulting from a past insult or injury. {\fontspec{DejaVu Sans}◇} \textit{I've never been one to hold a grudge} \colorBulletS{SYN} grievance \\{\fontspec{DejaVu Sans}▪ }\textsf{\textit{verb}}\\ \textbf{1} Be resentfully unwilling to give or allow (something) {\fontspec{DejaVu Sans}◇} \textit{he grudged the work and time that the meeting involved} \colorBulletS{SYN} begrudge, resent, feel aggrieved about, feel bitter about, be annoyed about, be angry about, be displeased about, be resentful of, mind, object to, take exception to, regret}{}{}{ \colorBullet{ORIGIN} Late Middle English variant of obsolete grutch ‘complain, murmur, grumble’, from Old French grouchier, of unknown origin. Compare with grouch.}%
\par%
\entry{grunt}{/ɡrʌnt/}{ঘোঁৎ ঘোঁৎ}{\small{\textsf{\textit{noun, verb}}} \\{\fontspec{DejaVu Sans}▪ }\textsf{\textit{noun}}\\ \textbf{1} A low, short guttural sound made by an animal or a person. {\fontspec{DejaVu Sans}◇} \textit{with snorts and grunts the animals were coaxed down the ramp} \textbf{2} A low{-}ranking soldier or unskilled worker. {\fontspec{DejaVu Sans}◇} \textit{he went from grunt to senior executive vice president in five years} \colorBulletS{SYN} private soldier, common soldier \textbf{3} Mechanical power, especially in a motor vehicle. {\fontspec{DejaVu Sans}◇} \textit{what the big wagon needs is grunt, and the turbo does the business} \colorBulletS{SYN} driving force, horsepower, hp, acceleration \textbf{4} An edible shoaling fish of tropical coasts and coral reefs, able to make a loud noise by grinding its teeth and amplifying the sound in the swim bladder. {\fontspec{DejaVu Sans}◇} \textit{} \\{\fontspec{DejaVu Sans}▪ }\textsf{\textit{verb}}\\ \textbf{1} (of an animal, especially a pig) make a low, short guttural sound. {\fontspec{DejaVu Sans}◇} \textit{an enormous pig grunted and shuffled in a sty outside}}{}{}{ \colorBullet{ORIGIN} Old English grunnettan, of Germanic origin and related to German grunzen; probably originally imitative.}%
\par%
\entry{guava}{/ˈɡwɑːvə/}{পেয়ারা}{ \textsf{\textit{noun}}\ \textbf{1} An edible, pale orange tropical fruit with pink juicy flesh and a strong sweet aroma. {\fontspec{DejaVu Sans}◇} \textit{} \textbf{2} The small tropical American tree which bears guavas. {\fontspec{DejaVu Sans}◇} \textit{}}{}{}{ \colorBullet{ORIGIN} Mid 16th century from Spanish guayaba, probably from Taino.}%
\par%
\entry{guilt}{/ɡɪlt/}{দোষ}{\small{\textsf{\textit{noun, verb}}} \\{\fontspec{DejaVu Sans}▪ }\textsf{\textit{noun}}\\ \textbf{1} The fact of having committed a specified or implied offence or crime. {\fontspec{DejaVu Sans}◇} \textit{it is the duty of the prosecution to prove the prisoner's guilt} \colorBulletS{SYN} culpability, guiltiness, blameworthiness, wrongdoing, wrong, wrongfulness, criminality, unlawfulness, misconduct, delinquency, sin, sinfulness, iniquity \\{\fontspec{DejaVu Sans}▪ }\textsf{\textit{verb}}\\ \textbf{1} Make (someone) feel guilty, especially in order to induce them to do something. {\fontspec{DejaVu Sans}◇} \textit{Celeste had been guilted into going by her parents}}{}{}{ \colorBullet{ORIGIN} Old English gylt, of unknown origin.}%
\par%
\entry{gust}{/ɡʌst/}{ঝড়ো}{\small{\textsf{\textit{noun, verb}}} \\{\fontspec{DejaVu Sans}▪ }\textsf{\textit{noun}}\\ \textbf{1} A sudden strong rush of wind. {\fontspec{DejaVu Sans}◇} \textit{} \colorBulletS{SYN} flurry, blast, puff, blow, rush, squall \\{\fontspec{DejaVu Sans}▪ }\textsf{\textit{verb}}\\ \textbf{1} (of the wind) blow in gusts. {\fontspec{DejaVu Sans}◇} \textit{the wind was gusting through the branches of the tree} \colorBulletS{SYN} bluster, flurry, blow, blast, roar}{}{Gusting wind: ঝড়ো বাতাস}{ \colorBullet{ORIGIN} Late 16th century from Old Norse gustr, related to gjósa ‘to gush’.}%
\par%
\end{multicols}%
\pagebreak%
\section*{H}%
\begin{multicols}{2}%
\entry{hack}{/hak/}{টাট্টু ঘোড়া}{\small{\textsf{\textit{noun, verb}}} \\{\fontspec{DejaVu Sans}▪ }\textsf{\textit{noun}}\\ \textbf{1} A rough cut, blow, or stroke. {\fontspec{DejaVu Sans}◇} \textit{he was sure one of us was going to take a hack at him} \textbf{2} An act of computer hacking. {\fontspec{DejaVu Sans}◇} \textit{the challenge of the hack itself} \\{\fontspec{DejaVu Sans}▪ }\textsf{\textit{verb}}\\ \textbf{1} Cut with rough or heavy blows. {\fontspec{DejaVu Sans}◇} \textit{I watched them hack the branches} \colorBulletS{SYN} cut, chop, hew, lop, saw \textbf{2} Gain unauthorized access to data in a system or computer. {\fontspec{DejaVu Sans}◇} \textit{they hacked into the bank's computer} \textbf{3} Cough persistently. {\fontspec{DejaVu Sans}◇} \textit{I was waking up in the middle of the night and coughing and hacking for hours} \textbf{4} Manage; cope. {\fontspec{DejaVu Sans}◇} \textit{lots of people leave because they can't hack it} \colorBulletS{SYN} cope, manage, get on, get along, get by, carry on, muddle through, muddle along, come through, stand on one's own two feet, weather the storm}{}{Hack to death; one hacked to death by group of men}{ \colorBullet{ORIGIN} Old English haccian ‘cut in pieces’, of West Germanic origin; related to Dutch hakken and German hacken.}%
\par%
\entry{hack}{/hak/}{টাট্টু ঘোড়া}{\small{\textsf{\textit{noun, verb}}} \\{\fontspec{DejaVu Sans}▪ }\textsf{\textit{noun}}\\ \textbf{1} A writer or journalist producing dull, unoriginal work. {\fontspec{DejaVu Sans}◇} \textit{Sunday newspaper hacks earn their livings on such gullibilities} \colorBulletS{SYN} journalist, reporter, correspondent, newspaperman, newspaperwoman, newsman, newswoman, writer, feature writer, contributor, columnist, Grub Street writer \textbf{2} A horse for ordinary riding. {\fontspec{DejaVu Sans}◇} \textit{} \colorBulletS{SYN} nag, inferior horse, tired{-}out horse, worn{-}out horse, Rosinante \textbf{3} A taxi. {\fontspec{DejaVu Sans}◇} \textit{You're going to have to take me or I'll turn you in and you'll lose your hack license.} \colorBulletS{SYN} taxi, cab, taxi cab, minicab, hackney cab \\{\fontspec{DejaVu Sans}▪ }\textsf{\textit{verb}}\\ \textbf{1} Ride a horse for pleasure or exercise. {\fontspec{DejaVu Sans}◇} \textit{some gentle hacking in a scenic setting}}{}{Hack to death; one hacked to death by group of men}{ \colorBullet{ORIGIN} Middle English (in hack (sense 2 of the noun)): abbreviation of hackney. hack (sense 1 of the noun) dates from the late 17th century.}%
\par%
\entry{hack}{/hak/}{টাট্টু ঘোড়া}{ \textsf{\textit{noun}}\ \textbf{1} A board on which a hawk's meat is laid. {\fontspec{DejaVu Sans}◇} \textit{‘Take up’ is sometimes used to mean to withdraw a hawk from the mews or from hack with a view to preparing her for hunting.} \textbf{2} A wooden frame for drying bricks, cheeses, etc. {\fontspec{DejaVu Sans}◇} \textit{}}{}{Hack to death; one hacked to death by group of men}{ \colorBullet{ORIGIN} Late Middle English (denoting the lower half of a divided door): variant of hatch.}%
\par%
\entry{hand}{/hand/}{হাত}{\small{\textsf{\textit{noun, verb}}} \\{\fontspec{DejaVu Sans}▪ }\textsf{\textit{noun}}\\ \textbf{1} The end part of a person's arm beyond the wrist, including the palm, fingers, and thumb. {\fontspec{DejaVu Sans}◇} \textit{the palm of her hand} \colorBulletS{SYN} fist, palm \textbf{2} A pointer on a clock or watch indicating the passing of units of time. {\fontspec{DejaVu Sans}◇} \textit{the second hand} \colorBulletS{SYN} pointer, indicator, needle, arrow, marker, index \textbf{3} Used in reference to the power to direct something. {\fontspec{DejaVu Sans}◇} \textit{the day{-}to{-}day running of the house was in her hands} \colorBulletS{SYN} control, power, charge, authority \textbf{4} A person's workmanship, especially in artistic work. {\fontspec{DejaVu Sans}◇} \textit{his idiosyncratic hand} \textbf{5} A person who engages in manual labour, especially in a factory, on a farm, or on board a ship. {\fontspec{DejaVu Sans}◇} \textit{a factory hand} \colorBulletS{SYN} worker, factory worker, manual worker, unskilled worker, blue{-}collar worker, workman, workwoman, workperson, working man, labourer, operative, hired hand, hireling, roustabout, employee, artisan \textbf{6} The set of cards dealt to a player in a card game. {\fontspec{DejaVu Sans}◇} \textit{he's got a good hand} \textbf{7} A unit of measurement of a horse's height, equal to 4 inches (10.16 cm). {\fontspec{DejaVu Sans}◇} \textit{Direct Access is no pony himself and at 17 hands is the biggest horse in Lungo's yard.} \textbf{8} A bunch of bananas. {\fontspec{DejaVu Sans}◇} \textit{mottled hands of bananas} \\{\fontspec{DejaVu Sans}▪ }\textsf{\textit{verb}}\\ \textbf{1} Pick (something) up and give it to (someone) {\fontspec{DejaVu Sans}◇} \textit{he handed each man a glass} \colorBulletS{SYN} pass, give, reach, let someone have, throw, toss \textbf{2} Hold the hand of (someone) in order to guide them in a specified direction. {\fontspec{DejaVu Sans}◇} \textit{he handed them into the carriage} \colorBulletS{SYN} assist, help, aid, give someone a hand, give someone a helping hand, give someone assistance \textbf{3} Take in or furl (a sail) {\fontspec{DejaVu Sans}◇} \textit{hand in the main!}}{ \colorBullet{OTHER} hands off}{Hands off my sister}{ \colorBullet{ORIGIN} Old English hand, hond, of Germanic origin; related to Dutch hand and German Hand.}%
\par%
\entry{handloom}{/ˈhandluːm/}{তন্তু}{ \textsf{\textit{noun}}\ \textbf{1} A manually operated loom. {\fontspec{DejaVu Sans}◇} \textit{}}{}{}{}%
\par%
\entry{handwoven}{/ˈhandwəʊvn/}{হাতে বুননকৃত}{ \textsf{\textit{adjective}}\ \textbf{1} (of fabric) woven by hand or on an unpowered loom. {\fontspec{DejaVu Sans}◇} \textit{}}{}{}{}%
\par%
\entry{handy}{/ˈhandi/}{কুশলী}{\small{\textsf{\textit{adjective, noun}}} \\{\fontspec{DejaVu Sans}▪ }\textsf{\textit{adjective}}\\ \textbf{1} Convenient to handle or use; useful. {\fontspec{DejaVu Sans}◇} \textit{a handy desktop encyclopedia} \colorBulletS{SYN} useful, convenient, practical, easy{-}to{-}use, well designed, user{-}friendly, user{-}oriented, helpful, functional, serviceable, utilitarian \textbf{2} Ready to hand. {\fontspec{DejaVu Sans}◇} \textit{keep credit cards handy} \colorBulletS{SYN} readily available, available, at hand, to hand, near at hand, within reach, accessible, ready, close, close by, near, nearby, at the ready, at one's fingertips, at one's disposal, convenient \textbf{3} Skilful. {\fontspec{DejaVu Sans}◇} \textit{he's handy with a needle} \colorBulletS{SYN} skilful, skilled, dexterous, deft, nimble{-}fingered, adroit, practical, able, adept, proficient, capable \\{\fontspec{DejaVu Sans}▪ }\textsf{\textit{noun}}\\ \textbf{1} (in Europe) a mobile phone. {\fontspec{DejaVu Sans}◇} \textit{But if you can manage not to ask questions about why they don't just use their mobiles/cell phones / handies, or why they don't just use email, this doesn't matter.}}{}{}{ \colorBullet{ORIGIN} Turn out to be useful.}%
\par%
\entry{hang}{/haŋ/}{লেগে থাকা}{\small{\textsf{\textit{exclamation, noun, verb}}} \\{\fontspec{DejaVu Sans}▪ }\textsf{\textit{exclamation}}\\ \textbf{1} Used to express a range of strong emotions from enthusiasm to anger. {\fontspec{DejaVu Sans}◇} \textit{hang, but I loved those soldiers!} \\{\fontspec{DejaVu Sans}▪ }\textsf{\textit{noun}}\\ \textbf{1} A downward droop or bend. {\fontspec{DejaVu Sans}◇} \textit{the bullish hang of his head} \\{\fontspec{DejaVu Sans}▪ }\textsf{\textit{verb}}\\ \textbf{1} Suspend or be suspended from above with the lower part dangling free. {\fontspec{DejaVu Sans}◇} \textit{that's where people are supposed to hang their washing} \colorBulletS{SYN} be suspended, hang down, be pendent, dangle, swing, sway \textbf{2} Kill (someone) by tying a rope attached from above around their neck and removing the support from beneath them (often used as a form of capital punishment) {\fontspec{DejaVu Sans}◇} \textit{he was hanged for murder} \colorBulletS{SYN} execute by hanging, hang by the neck, send to the gallows, send to the gibbet, send to the scaffold, gibbet, put to death \textbf{3} Remain static in the air. {\fontspec{DejaVu Sans}◇} \textit{a black pall of smoke hung over Valletta} \colorBulletS{SYN} hover, float, drift, linger, remain static, be suspended, be poised \textbf{4} Come or cause to come unexpectedly to a state in which no further operations can be carried out. {\fontspec{DejaVu Sans}◇} \textit{the machine has hung} \textbf{5} Spend time relaxing or enjoying oneself. {\fontspec{DejaVu Sans}◇} \textit{I guess I wasn't cool enough to hang with them anymore} \textbf{6} Deliver (a pitch) which does not change direction and is easily hit by a batter. {\fontspec{DejaVu Sans}◇} \textit{this leads to hanging a breaking ball}}{ \colorBullet{OTHER} hang on; hang out}{Hang out with friends}{ \colorBullet{ORIGIN} Old English hangian (intransitive verb), of West Germanic origin, related to Dutch and German hangen, reinforced by the Old Norse transitive verb hanga.}%
\par%
\entry{hardship}{/ˈhɑːdʃɪp/}{কষ্ট}{ \textsf{\textit{noun}}\ \textbf{1} Severe suffering or privation. {\fontspec{DejaVu Sans}◇} \textit{intolerable levels of hardship} \colorBulletS{SYN} privation, deprivation, destitution, poverty, austerity, penury, want, need, neediness, beggary, impecuniousness, impecuniosity, financial distress}{}{}{}%
\par%
\entry{harvest}{/ˈhɑːvɪst/}{ফসল}{\small{\textsf{\textit{noun, verb}}} \\{\fontspec{DejaVu Sans}▪ }\textsf{\textit{noun}}\\ \textbf{1} The process or period of gathering in crops. {\fontspec{DejaVu Sans}◇} \textit{farmers work longer hours during the harvest} \colorBulletS{SYN} gathering in of the crops, harvesting, harvest time, harvest home \\{\fontspec{DejaVu Sans}▪ }\textsf{\textit{verb}}\\ \textbf{1} Gather (a crop) as a harvest. {\fontspec{DejaVu Sans}◇} \textit{after harvesting, most of the crop is stored in large buildings} \colorBulletS{SYN} gather in, gather, bring in, take in}{}{}{ \colorBullet{ORIGIN} Old English hærfest ‘autumn’, of Germanic origin; related to Dutch herfst and German Herbst, from an Indo{-}European root shared by Latin carpere ‘pluck’ and Greek karpos ‘fruit’.}%
\par%
\entry{hastily}{/ˈheɪstɪli/}{ব্যস্তভাবে; ঝটিকা}{ \textsf{\textit{adverb}}\ \textbf{1} With excessive speed or urgency; hurriedly. {\fontspec{DejaVu Sans}◇} \textit{maybe I acted too hastily} \colorBulletS{SYN} quickly, hurriedly, in a hurry, fast, swiftly, rapidly, speedily, briskly, expeditiously, without delay, post{-}haste, at high speed, at full speed, with all speed, at full tilt, at the speed of light, as fast as possible, with all possible haste, like a whirlwind, like an arrow from a bow, at breakneck speed, as fast as one's legs can carry one, at a run, at a gallop, hotfoot, on the double}{}{}{}%
\par%
\entry{hatch}{/hatʃ/}{ডিম পাড়া}{ \textsf{\textit{noun}}\ \textbf{1} A door in an aircraft, spacecraft, or submarine. {\fontspec{DejaVu Sans}◇} \textit{} \textbf{2} The rear door of a hatchback car. {\fontspec{DejaVu Sans}◇} \textit{a spare wheel mounted on the rear hatch} \textbf{3} short for hatchback {\fontspec{DejaVu Sans}◇} \textit{}}{}{}{ \colorBullet{ORIGIN} Old English hæcc (denoting the lower half of a divided door), of Germanic origin; related to Dutch hek ‘paling, screen’.}%
\par%
\entry{hatch}{/hatʃ/}{ডিম পাড়া}{\small{\textsf{\textit{noun, verb}}} \\{\fontspec{DejaVu Sans}▪ }\textsf{\textit{noun}}\\ \textbf{1} A newly hatched brood. {\fontspec{DejaVu Sans}◇} \textit{a hatch of mayflies} \\{\fontspec{DejaVu Sans}▪ }\textsf{\textit{verb}}\\ \textbf{1} (of an egg) open and produce a young animal. {\fontspec{DejaVu Sans}◇} \textit{eggs need to be put in a warm place to hatch} \textbf{2} Conspire to devise (a plot or plan) {\fontspec{DejaVu Sans}◇} \textit{the little plot that you and Sylvia hatched up last night} \colorBulletS{SYN} devise, conceive, contrive, concoct, brew, invent, plan, design, formulate}{}{}{ \colorBullet{ORIGIN} Middle English hacche; related to Swedish häcka and Danish hække.}%
\par%
\entry{hatch}{/hatʃ/}{ডিম পাড়া}{ \textsf{\textit{verb}}\ \textbf{1} (in fine art and technical drawing) shade (an area) with closely drawn parallel lines. {\fontspec{DejaVu Sans}◇} \textit{the unused space has been hatched with lines}}{}{}{ \colorBullet{ORIGIN} Late 15th century (in the sense ‘inlay with strips of metal’): from Old French hacher, from hache (see hatchet).}%
\par%
\entry{haughty}{/ˈhɔːti/}{উদ্ধত}{ \textsf{\textit{adjective}}\ \textbf{1} Arrogantly superior and disdainful. {\fontspec{DejaVu Sans}◇} \textit{a look of haughty disdain} \colorBulletS{SYN} proud, vain, arrogant, conceited, snobbish, stuck{-}up, pompous, self{-}important, superior, egotistical, supercilious, condescending, lofty, patronizing, smug, scornful, contemptuous, disdainful, overweening, overbearing, imperious, lordly, cavalier, high{-}handed, full of oneself, above oneself}{}{}{ \colorBullet{ORIGIN} Mid 16th century extended form of obsolete haught, earlier haut, from Old French, from Latin altus ‘high’.}%
\par%
\entry{haul}{/hɔːl/}{টান}{\small{\textsf{\textit{noun, verb}}} \\{\fontspec{DejaVu Sans}▪ }\textsf{\textit{noun}}\\ \textbf{1} A quantity of something that has been stolen or is possessed illegally. {\fontspec{DejaVu Sans}◇} \textit{they escaped with a haul of antiques} \colorBulletS{SYN} booty, loot, plunder \textbf{2} A distance to be covered in a journey. {\fontspec{DejaVu Sans}◇} \textit{the thirty{-}mile haul to Boston} \\{\fontspec{DejaVu Sans}▪ }\textsf{\textit{verb}}\\ \textbf{1} (of a person) pull or drag with effort or force. {\fontspec{DejaVu Sans}◇} \textit{he hauled his bike out of the shed} \colorBulletS{SYN} drag, pull, tug, heave, hump, trail, draw, tow, manhandle \textbf{2} (of a vehicle) pull (an attached trailer or carriage) behind it. {\fontspec{DejaVu Sans}◇} \textit{the engine hauls the overnight sleeper from London Euston} \textbf{3} (especially of a sailing ship) make an abrupt change of course. {\fontspec{DejaVu Sans}◇} \textit{my plan was to haul offshore, well clear of the land}}{}{}{ \colorBullet{ORIGIN} Mid 16th century (originally in the nautical sense ‘trim sails for sailing closer to the wind’): variant of hale.}%
\par%
\entry{havoc}{/ˈhavək/}{ব্যাপক ধ্বংস}{\small{\textsf{\textit{noun, verb}}} \\{\fontspec{DejaVu Sans}▪ }\textsf{\textit{noun}}\\ \textbf{1} Widespread destruction. {\fontspec{DejaVu Sans}◇} \textit{the hurricane ripped through Florida causing havoc} \colorBulletS{SYN} devastation, destruction, damage, desolation, depredation, despoliation, ruination, ruin, disaster, ravagement, waste, catastrophe \\{\fontspec{DejaVu Sans}▪ }\textsf{\textit{verb}}\\ \textbf{1} Lay waste to; devastate. {\fontspec{DejaVu Sans}◇} \textit{The lack of participants is associated to a large storm that havocked Latvia in January 2005 and uprooted and destroyed large forest areas.} \colorBulletS{SYN} lay waste, devastate, ruin, leave in ruins, destroy, wreak havoc on, leave desolate, level, raze, demolish, wipe out, wreck, damage}{}{Flood wreaks havoc on croplands}{ \colorBullet{ORIGIN} Late Middle English from Anglo{-}Norman French havok, alteration of Old French havot, of unknown origin. The word was originally used in the phrase cry havoc (Old French crier havot) ‘to give an army the order havoc’, which was the signal for plundering.}%
\par%
\entry{heading}{/ˈhɛdɪŋ/}{শিরোনাম, অগ্রগতি}{ \textsf{\textit{noun}}\ \textbf{1} A title at the head of a page or section of a book. {\fontspec{DejaVu Sans}◇} \textit{chapter headings} \colorBulletS{SYN} title, caption, legend, subtitle, subheading, wording, rubric, inscription, name, headline, banner headline \textbf{2} A direction or bearing. {\fontspec{DejaVu Sans}◇} \textit{he crawled on a heading of 90 degrees until he came to the track} \textbf{3} A horizontal passage made in preparation for building a tunnel. {\fontspec{DejaVu Sans}◇} \textit{A top heading is first excavated, and then a bench that is sometimes split further into bench and invert sections is constructed.} \textbf{4} A strip of cloth at the top of a curtain above the hooks or wire by which it is suspended. {\fontspec{DejaVu Sans}◇} \textit{}}{ \colorBullet{OTHER} heading to: অগ্রসর হওয়া}{Floods heading to heartland}{}%
\par%
\entry{heartland}{/ˈhɑːtland/}{ভূখন্ড}{ \textsf{\textit{noun}}\ \textbf{1} The central or most important part of a country, area, or field of activity. {\fontspec{DejaVu Sans}◇} \textit{wildlife sites in the heartland of Russia}}{}{Floods heading to heartland}{}%
\par%
\entry{held}{/hɛld/}{দখলী}{\small{\textsf{\textit{}}}}{}{}{}%
\par%
\entry{hello}{/həˈləʊ/}{হ্যালো}{\small{\textsf{\textit{exclamation, noun, verb}}} \\{\fontspec{DejaVu Sans}▪ }\textsf{\textit{exclamation}}\\ \textbf{1} Used as a greeting or to begin a telephone conversation. {\fontspec{DejaVu Sans}◇} \textit{hello there, Katie!} \\{\fontspec{DejaVu Sans}▪ }\textsf{\textit{noun}}\\ \textbf{1} An utterance of ‘hello’; a greeting. {\fontspec{DejaVu Sans}◇} \textit{she was getting polite nods and hellos from people} \colorBulletS{SYN} greeting, welcome, salutation, saluting, hailing, address, hello, hallo \\{\fontspec{DejaVu Sans}▪ }\textsf{\textit{verb}}\\ \textbf{1} Say or shout ‘hello’ {\fontspec{DejaVu Sans}◇} \textit{I pressed the phone button and helloed}}{}{}{ \colorBullet{ORIGIN} Early 19th century variant of earlier hollo; related to holla.}%
\par%
\entry{helm}{/hɛlm/}{হাল}{\small{\textsf{\textit{noun, verb}}} \\{\fontspec{DejaVu Sans}▪ }\textsf{\textit{noun}}\\ \textbf{1} A tiller or wheel for steering a ship or boat. {\fontspec{DejaVu Sans}◇} \textit{she stayed at the helm, alert for tankers} \colorBulletS{SYN} tiller, wheel \\{\fontspec{DejaVu Sans}▪ }\textsf{\textit{verb}}\\ \textbf{1} Steer (a boat or ship) {\fontspec{DejaVu Sans}◇} \textit{he helmed a sailing vessel} \colorBulletS{SYN} steer, captain, pilot, skipper, navigate, con, helm}{}{}{ \colorBullet{ORIGIN} Old English helma; probably related to helve.}%
\par%
\entry{helm}{/hɛlm/}{হাল}{ \textsf{\textit{noun}}\ \textbf{1} A helmet. {\fontspec{DejaVu Sans}◇} \textit{}}{}{}{ \colorBullet{ORIGIN} Old English, of Germanic origin; related to Dutch helm and German Helm, also to helmet, from an Indo{-}European root meaning ‘to cover or hide’.}%
\par%
\entry{herb}{/həːb/}{ঔষধি}{ \textsf{\textit{noun}}\ \textbf{1} Any plant with leaves, seeds, or flowers used for flavouring, food, medicine, or perfume. {\fontspec{DejaVu Sans}◇} \textit{bundles of dried herbs} \colorBulletS{SYN} flavouring, salt and pepper, herbs, spices, condiments, dressing, relish \textbf{2} Any seed{-}bearing plant which does not have a woody stem and dies down to the ground after flowering. {\fontspec{DejaVu Sans}◇} \textit{the banana plant is the world's largest herb}}{}{}{ \colorBullet{ORIGIN} Middle English via Old French from Latin herba ‘grass, green crops, herb’. Although herb has always been spelled with an h, pronunciation without it was usual until the 19th century and is still standard in the US.}%
\par%
\entry{hike}{/hʌɪk/}{ভাড়ায়}{\small{\textsf{\textit{noun, verb}}} \\{\fontspec{DejaVu Sans}▪ }\textsf{\textit{noun}}\\ \textbf{1} A long walk or walking tour. {\fontspec{DejaVu Sans}◇} \textit{a five{-}mile hike across rough terrain} \colorBulletS{SYN} walk, trek, tramp, trudge, traipse, slog, footslog, plod, march, journey on foot \textbf{2} A sharp increase, especially in price. {\fontspec{DejaVu Sans}◇} \textit{a price hike} \colorBulletS{SYN} growth, rise, enlargement, expansion, extension, multiplication, elevation, swelling, inflation \\{\fontspec{DejaVu Sans}▪ }\textsf{\textit{verb}}\\ \textbf{1} Walk for a long distance, especially across country. {\fontspec{DejaVu Sans}◇} \textit{they hiked across the moors} \colorBulletS{SYN} walk, go on foot, trek, tramp, trudge, traipse, slog, footslog, plod, march \textbf{2} Pull or lift up (something, especially clothing) {\fontspec{DejaVu Sans}◇} \textit{Roy hiked up his trousers to reveal his socks} \colorBulletS{SYN} hitch up, pull up, jerk up, lift, raise, hoist}{ \colorBullet{OTHER} hike in: বৃদ্ধি}{Gas price hike in Bangladesh remains effective}{ \colorBullet{ORIGIN} Early 19th century (originally dialect, as a verb): of unknown origin.}%
\par%
\entry{hilarious}{/hɪˈlɛːrɪəs/}{অত্যধিক হাসিখুশি}{ \textsf{\textit{adjective}}\ \textbf{1} Extremely amusing. {\fontspec{DejaVu Sans}◇} \textit{her hilarious novel} \colorBulletS{SYN} very funny, extremely amusing, hysterically funny, hysterical, uproarious, riotous, farcical, side{-}splitting, rib{-}tickling, too funny for words}{}{}{ \colorBullet{ORIGIN} Early 19th century from Latin hilaris (from Greek hilaros ‘cheerful’) + {-}ous. The sense ‘exceedingly amusing’ dates from the 1920s.}%
\par%
\entry{hindsight}{/ˈhʌɪn(d)sʌɪt/}{সংঘটনের পরে বোধোদয়}{ \textsf{\textit{noun}}\ \textbf{1} Understanding of a situation or event only after it has happened or developed. {\fontspec{DejaVu Sans}◇} \textit{with hindsight, I should never have gone}}{}{}{}%
\par%
\entry{hippie{-}dippie}{}{1. of, relating to, or reflecting the far{-}out styles and values of hippies 2. Uncool, due to lack of forethought.}{\small{\textsf{\textit{}}}}{}{Your hippie{-}dippie ideas lack a thorough understanding of reality. }{}%
\par%
\entry{hitherto}{/hɪðəˈtuː/}{এ যাবৎ; এই সময় পর্যন্ত}{ \textsf{\textit{adverb}}\ \textbf{1} Until now or until the point in time under discussion. {\fontspec{DejaVu Sans}◇} \textit{hitherto part of French West Africa, Benin achieved independence in 1960} \colorBulletS{SYN} previously, formerly, earlier, so far, thus far, before, beforehand, to date, as yet}{}{}{}%
\par%
\entry{homicide}{/ˈhɒmɪsʌɪd/}{নরহত্যা}{ \textsf{\textit{noun}}\ \textbf{1} The killing of one person by another. {\fontspec{DejaVu Sans}◇} \textit{he was charged with homicide} \colorBulletS{SYN} murder, killing, assassination, liquidation, extermination, execution, slaughter, butchery, massacre}{}{}{ \colorBullet{ORIGIN} Middle English from Old French, from Latin homicidium, from homo, homin{-} ‘man’.}%
\par%
\entry{hooligan}{/ˈhuːlɪɡ(ə)n/}{গুণ্ডা}{ \textsf{\textit{noun}}\ \textbf{1} A violent young troublemaker, typically one of a gang. {\fontspec{DejaVu Sans}◇} \textit{a drunken hooligan} \colorBulletS{SYN} hoodlum, thug, lout, delinquent, tearaway, vandal, ruffian, rowdy, troublemaker}{}{}{ \colorBullet{ORIGIN} Late 19th century perhaps from Hooligan, the surname of a fictional rowdy Irish family in a music{-}hall song of the 1890s, also of a cartoon character.}%
\par%
\entry{hoop}{/huːp/}{পতর}{\small{\textsf{\textit{noun, verb}}} \\{\fontspec{DejaVu Sans}▪ }\textsf{\textit{noun}}\\ \textbf{1} A circular band of metal, wood, or similar material, especially one used for binding the staves of barrels or forming part of a framework. {\fontspec{DejaVu Sans}◇} \textit{} \colorBulletS{SYN} ring, band, circle, circlet, loop, wheel, round, girdle \textbf{2} A horizontal band of a contrasting colour on a sports shirt or jockey's cap. {\fontspec{DejaVu Sans}◇} \textit{} \\{\fontspec{DejaVu Sans}▪ }\textsf{\textit{verb}}\\ \textbf{1} Bind or encircle with or as with hoops. {\fontspec{DejaVu Sans}◇} \textit{a man was hooping a barrel}}{}{}{ \colorBullet{ORIGIN} Late Old English hōp, of West Germanic origin; related to Dutch hoep.}%
\par%
\entry{horrible}{/ˈhɒrɪb(ə)l/}{ভয়ঙ্কর}{ \textsf{\textit{adjective}}\ \textbf{1} Causing or likely to cause horror; shocking. {\fontspec{DejaVu Sans}◇} \textit{a horrible massacre} \colorBulletS{SYN} dreadful, horrifying, horrific, horrendous, frightful, fearful, awful, terrible, shocking, appalling, hideous, grim, grisly, ghastly, harrowing, gruesome, heinous, vile, nightmarish, macabre, unspeakable, hair{-}raising, spine{-}chilling}{}{}{ \colorBullet{ORIGIN} Middle English via Old French from Latin horribilis, from horrere ‘tremble, shudder’ (see horrid).}%
\par%
\entry{hosiery}{/ˈhəʊzɪəri/}{হোসিয়ারি}{ \textsf{\textit{noun}}\ \textbf{1} Stockings, socks, and tights collectively. {\fontspec{DejaVu Sans}◇} \textit{} \colorBulletS{SYN} stockings, tights, stay{-}ups, nylons}{}{}{}%
\par%
\entry{hostage}{/ˈhɒstɪdʒ/}{জিম্মি}{ \textsf{\textit{noun}}\ \textbf{1} A person seized or held as security for the fulfilment of a condition. {\fontspec{DejaVu Sans}◇} \textit{they were held hostage by armed rebels} \colorBulletS{SYN} captive, prisoner, detainee, internee}{}{}{ \colorBullet{ORIGIN} Middle English from Old French, based on late Latin obsidatus ‘the state of being a hostage’ (the earliest sense in English), from Latin obses, obsid{-} ‘hostage’.}%
\par%
\entry{hostile}{/ˈhɒstʌɪl/}{প্রতিকূল}{ \textsf{\textit{adjective}}\ \textbf{1} Showing or feeling opposition or dislike; unfriendly. {\fontspec{DejaVu Sans}◇} \textit{a hostile audience} \colorBulletS{SYN} antagonistic, aggressive, confrontational, belligerent, bellicose, pugnacious, militant, truculent, combative, warlike}{}{}{ \colorBullet{ORIGIN} Late 16th century from French, or from Latin hostilis, from hostis ‘stranger, enemy’.}%
\par%
\entry{hover}{/ˈhɒvə/}{বাতাসে ভাসিতে থাকা}{\small{\textsf{\textit{noun, verb}}} \\{\fontspec{DejaVu Sans}▪ }\textsf{\textit{noun}}\\ \textbf{1} An act of remaining in the air in one place. {\fontspec{DejaVu Sans}◇} \textit{keep the model in a stable hover} \\{\fontspec{DejaVu Sans}▪ }\textsf{\textit{verb}}\\ \textbf{1} Remain in one place in the air. {\fontspec{DejaVu Sans}◇} \textit{Army helicopters hovered overhead} \colorBulletS{SYN} be suspended, be poised, hang, float, levitate, drift, fly, flutter}{}{}{ \colorBullet{ORIGIN} Late Middle English from archaic hove ‘hover, linger’, of unknown origin.}%
\par%
\entry{hue}{/hjuː/}{রঙ}{ \textsf{\textit{noun}}\ \textbf{1} A colour or shade. {\fontspec{DejaVu Sans}◇} \textit{the water is the deepest hue of aquamarine} \colorBulletS{SYN} colour, tone, shade, tint, tinge, cast, tincture}{}{}{ \colorBullet{ORIGIN} Old English hīw, hēow (also ‘form, appearance’, obsolete except in Scots), of Germanic origin; related to Swedish hy ‘skin, complexion’. The sense ‘colour, shade’ dates from the mid 19th century.}%
\par%
\entry{Hué}{/hweɪ/}{রঙ}{ \textsf{\textit{proper noun}}\ \textbf{1} A city in central Vietnam; population 233,800 (est. 2009). {\fontspec{DejaVu Sans}◇} \textit{}}{}{}{}%
\par%
\entry{hum}{/hʌm/}{গুণ গুণ শব্দ}{\small{\textsf{\textit{noun, verb}}} \\{\fontspec{DejaVu Sans}▪ }\textsf{\textit{noun}}\\ \textbf{1} A low, steady continuous sound. {\fontspec{DejaVu Sans}◇} \textit{the hum of insects} \colorBulletS{SYN} murmur, murmuring, drone, droning, vibration, purr, purring, buzz, buzzing, whir, whirring, throb, throbbing, thrum, thrumming \\{\fontspec{DejaVu Sans}▪ }\textsf{\textit{verb}}\\ \textbf{1} Make a low, steady continuous sound like that of a bee. {\fontspec{DejaVu Sans}◇} \textit{the computers hummed} \colorBulletS{SYN} purr, whir, throb, vibrate, murmur, buzz, thrum, drone \textbf{2} Be in a state of great activity. {\fontspec{DejaVu Sans}◇} \textit{the house was humming with preparations for the dance} \colorBulletS{SYN} be busy, be active, be lively, buzz, bustle, be bustling, be a hive of activity, throb, vibrate, pulsate, pulse \textbf{3} Smell unpleasant. {\fontspec{DejaVu Sans}◇} \textit{when the wind drops this stuff really hums} \colorBulletS{SYN} smell, stink, stink to high heaven, reek, have a bad smell, be malodorous}{}{}{ \colorBullet{ORIGIN} Late Middle English imitative.}%
\par%
\entry{hum}{/hʌm/}{গুণ গুণ শব্দ}{ \textsf{\textit{exclamation}}\ \textbf{1} Used to express hesitation or dissent. {\fontspec{DejaVu Sans}◇} \textit{‘Ah, hum, Elsie, isn't it?’}}{}{}{ \colorBullet{ORIGIN} Mid 16th century imitative; related to the verb hum.}%
\par%
\entry{humanitarian}{/hjʊˌmanɪˈtɛːrɪən/}{মানবিক}{\small{\textsf{\textit{adjective, noun}}} \\{\fontspec{DejaVu Sans}▪ }\textsf{\textit{adjective}}\\ \textbf{1} Concerned with or seeking to promote human welfare. {\fontspec{DejaVu Sans}◇} \textit{groups sending humanitarian aid} \colorBulletS{SYN} compassionate, humane \\{\fontspec{DejaVu Sans}▪ }\textsf{\textit{noun}}\\ \textbf{1} A person who seeks to promote human welfare. {\fontspec{DejaVu Sans}◇} \textit{} \colorBulletS{SYN} philanthropist, altruist, benefactor, social reformer, do{-}gooder, good Samaritan}{}{}{ \colorBullet{ORIGIN} The primary sense of humanitarian is 'concerned with or seeking to promote human welfare’. Since the 1930s a new sense, exemplified by phrases such as the worst humanitarian disaster this country has seen, has been gaining currency, and is now broadly established, especially in journalism, although it is not considered good style by all. In the Oxford English Corpus the second most common collocation of humanitarian is crisis}%
\par%
\entry{humiliating}{/hjʊˈmɪlɪeɪtɪŋ/}{অপমানকর}{ \textsf{\textit{adjective}}\ \textbf{1} Making someone feel ashamed and foolish by injuring their dignity and pride. {\fontspec{DejaVu Sans}◇} \textit{a humiliating defeat}}{}{}{}%
\par%
\entry{hurl}{/həːl/}{সজোরে নিক্ষেপ}{\small{\textsf{\textit{noun, verb}}} \\{\fontspec{DejaVu Sans}▪ }\textsf{\textit{noun}}\\ \textbf{1} A ride in a vehicle; a lift. {\fontspec{DejaVu Sans}◇} \textit{hey pal, any chance of a hurl?} \\{\fontspec{DejaVu Sans}▪ }\textsf{\textit{verb}}\\ \textbf{1} Throw or impel (someone or something) with great force. {\fontspec{DejaVu Sans}◇} \textit{rioters hurled a brick through the windscreen} \colorBulletS{SYN} throw, toss, fling, pitch, cast, lob, launch, flip, catapult, shy, dash, send, bowl, aim, direct, project, propel, fire, let fly}{}{}{ \colorBullet{ORIGIN} Middle English probably imitative, but corresponding in form and partly in sense with Low German hurreln.}%
\par%
\end{multicols}%
\pagebreak%
\section*{I}%
\begin{multicols}{2}%
\entry{IED}{}{আইইডি}{ \textsf{\textit{noun}}\ \textbf{1} A simple bomb made and used by unofficial or unauthorized forces. {\fontspec{DejaVu Sans}◇} \textit{}}{}{}{}%
\par%
\entry{if i could, i would, but i can't, so i shan't.}{}{}{\small{\textsf{\textit{}}}}{}{}{}%
\par%
\entry{immense}{/ɪˈmɛns/}{অপরিমেয়}{ \textsf{\textit{adjective}}\ \textbf{1} Extremely large or great, especially in scale or degree. {\fontspec{DejaVu Sans}◇} \textit{the cost of restoration has been immense} \colorBulletS{SYN} huge, vast, massive, enormous, gigantic, colossal, cosmic, great, very large, very big, extensive, expansive, monumental, towering, mountainous, tremendous, prodigious, substantial}{}{}{ \colorBullet{ORIGIN} Late Middle English via French from Latin immensus ‘immeasurable’, from in{-} ‘not’ + mensus ‘measured’ (past participle of metiri).}%
\par%
\entry{impaired}{/ɪmˈpɛːd/}{হত}{ \textsf{\textit{adjective}}\ \textbf{1} Weakened or damaged. {\fontspec{DejaVu Sans}◇} \textit{an impaired banking system} \textbf{2} Having a disability of a specified kind. {\fontspec{DejaVu Sans}◇} \textit{sight{-}impaired children}}{}{}{}%
\par%
\entry{impasse}{/amˈpɑːs/}{কানাগলি}{ \textsf{\textit{noun}}\ \textbf{1} A situation in which no progress is possible, especially because of disagreement; a deadlock. {\fontspec{DejaVu Sans}◇} \textit{the current political impasse} \colorBulletS{SYN} deadlock, dead end, stalemate, checkmate, stand{-}off}{}{}{ \colorBullet{ORIGIN} Mid 19th century from French, from im{-} (expressing negation) + the stem of passer ‘to pass’.}%
\par%
\entry{impeccable}{/ɪmˈpɛkəb(ə)l/}{অনবদ্য}{ \textsf{\textit{adjective}}\ \textbf{1} In accordance with the highest standards; faultless. {\fontspec{DejaVu Sans}◇} \textit{he had impeccable manners} \colorBulletS{SYN} flawless, faultless, unblemished, spotless, stainless, untarnished, perfect, exemplary, ideal, model}{}{}{ \colorBullet{ORIGIN} Mid 16th century (in the theological sense): from Latin impeccabilis, from in{-} ‘not’ + peccare ‘to sin’.}%
\par%
\entry{impede}{/ɪmˈpiːd/}{ঠেকান}{ \textsf{\textit{verb}}\ \textbf{1} Delay or prevent (someone or something) by obstructing them; hinder. {\fontspec{DejaVu Sans}◇} \textit{the sap causes swelling which can impede breathing} \colorBulletS{SYN} hinder, obstruct, hamper, handicap, hold back, hold up, delay, interfere with, disrupt, retard, slow, slow down, brake, put a brake on, restrain, fetter, shackle, hamstring, cramp, cripple}{}{}{ \colorBullet{ORIGIN} Late 16th century from Latin impedire ‘shackle the feet of’, based on pes, ped{-} ‘foot’. Compare with impeach.}%
\par%
\entry{imperative}{/ɪmˈpɛrətɪv/}{অনুজ্ঞাসূচক}{\small{\textsf{\textit{adjective, noun}}} \\{\fontspec{DejaVu Sans}▪ }\textsf{\textit{adjective}}\\ \textbf{1} Of vital importance; crucial. {\fontspec{DejaVu Sans}◇} \textit{immediate action was imperative} \colorBulletS{SYN} vitally important, of vital importance, all{-}important, vital, crucial, critical, essential, of the essence, a matter of life and death, of great consequence, necessary, indispensable, exigent, pressing, urgent \textbf{2} Giving an authoritative command; peremptory. {\fontspec{DejaVu Sans}◇} \textit{the bell pealed again, a final imperative call} \colorBulletS{SYN} peremptory, commanding, imperious, authoritative, masterful, lordly, magisterial, autocratic, dictatorial, domineering, overbearing, assertive, firm, insistent, bossy, high{-}handed, overweening \\{\fontspec{DejaVu Sans}▪ }\textsf{\textit{noun}}\\ \textbf{1} An essential or urgent thing. {\fontspec{DejaVu Sans}◇} \textit{free movement of labour was an economic imperative} \colorBulletS{SYN} necessary condition, precondition, condition, essential, requirement, requisite, necessity, proviso, qualification, imperative, basic, rudiment, obligation, duty \textbf{2} A verb or phrase in the imperative mood. {\fontspec{DejaVu Sans}◇} \textit{}}{}{}{ \colorBullet{ORIGIN} Late Middle English (as a grammatical term): from late Latin imperativus (literally ‘specially ordered’, translating Greek prostatikē enklisis ‘imperative mood’), from imperare ‘to command’, from in{-} ‘towards’ + parare ‘make ready’.}%
\par%
\entry{impersonate}{/ɪmˈpəːs(ə)neɪt/}{ছদ্মবেশ ধারণ}{ \textsf{\textit{verb}}\ \textbf{1} Pretend to be (another person) for entertainment or fraud. {\fontspec{DejaVu Sans}◇} \textit{it's a very serious offence to impersonate a police officer} \colorBulletS{SYN} imitate, mimic, do an impression of, ape}{}{}{ \colorBullet{ORIGIN} Early 17th century (in the sense ‘personify’): from in{-}‘into’ + Latin persona ‘person’, on the pattern of incorporate.}%
\par%
\entry{implausible}{/ɪmˈplɔːzɪb(ə)l/}{অকল্পনীয়}{ \textsf{\textit{adjective}}\ \textbf{1} (of an argument or statement) not seeming reasonable or probable; failing to convince. {\fontspec{DejaVu Sans}◇} \textit{this is a blatantly implausible claim} \colorBulletS{SYN} unlikely, not likely, improbable, questionable, doubtful, debatable}{}{}{}%
\par%
\entry{implement}{/ˈɪmplɪm(ə)nt/}{বাস্তবায়ন}{\small{\textsf{\textit{noun, verb}}} \\{\fontspec{DejaVu Sans}▪ }\textsf{\textit{noun}}\\ \textbf{1} A tool, utensil, or other piece of equipment that is used for a particular purpose. {\fontspec{DejaVu Sans}◇} \textit{garden implements} \colorBulletS{SYN} tool, utensil, instrument, device, apparatus, contrivance, gadget, contraption, appliance, machine, labour{-}saving device \textbf{2} Performance of an obligation. {\fontspec{DejaVu Sans}◇} \textit{} \\{\fontspec{DejaVu Sans}▪ }\textsf{\textit{verb}}\\ \textbf{1} Put (a decision, plan, agreement, etc.) into effect. {\fontspec{DejaVu Sans}◇} \textit{the scheme to implement student loans} \colorBulletS{SYN} execute, apply, put into action, put into effect, put into practice, carry out, carry through, perform, enact, administer}{}{}{ \colorBullet{ORIGIN} Late Middle English (in the sense ‘article of furniture, equipment, or dress’): partly from medieval Latin implementa (plural), partly from late Latin implementum ‘filling up, fulfilment’, both from Latin implere ‘fill up’ (later ‘employ’), from in{-} ‘in’ + Latin plere ‘fill’. The verb dates from the early 18th century.}%
\par%
\entry{implicate}{/ˈɪmplɪkeɪt/}{জড়িয়ে}{\small{\textsf{\textit{noun, verb}}} \\{\fontspec{DejaVu Sans}▪ }\textsf{\textit{noun}}\\ \textbf{1} A thing implied. {\fontspec{DejaVu Sans}◇} \textit{The dual nature of the Heart represents the meeting of the changeless and the changing, the inevitable and the contingent, the implicate and the manifest.} \\{\fontspec{DejaVu Sans}▪ }\textsf{\textit{verb}}\\ \textbf{1} Show (someone) to be involved in a crime. {\fontspec{DejaVu Sans}◇} \textit{he implicated his government in the murders of three judges} \colorBulletS{SYN} incriminate, compromise \textbf{2} Convey (a meaning) indirectly through what one says, rather than stating it explicitly. {\fontspec{DejaVu Sans}◇} \textit{by saying that coffee would keep her awake, Mary implicated that she didn't want any} \colorBulletS{SYN} imply, suggest, hint, intimate, say indirectly, indicate, insinuate, give someone to understand, give someone to believe, convey the impression, signal}{}{}{ \colorBullet{ORIGIN} Late Middle English from Latin implicatus ‘folded in’, past participle of implicare (see imply). The original sense was ‘entwine’; compare with employ and imply. The earliest modern (implicate (sense 2 of the verb)), dates from the early 17th century.}%
\par%
\entry{implode}{/ɪmˈpləʊd/}{কেন্দ্রীভূত করা}{ \textsf{\textit{verb}}\ \textbf{1} Collapse or cause to collapse violently inwards. {\fontspec{DejaVu Sans}◇} \textit{both the windows had imploded} \colorBulletS{SYN} break up, break, break into pieces, crack apart, crack open, shatter, splinter, fracture, burst apart, explode, blow apart, implode}{}{}{ \colorBullet{ORIGIN} Late 19th century from in{-}‘within’ + Latin plodere, plaudere ‘to clap’, on the pattern of explode.}%
\par%
\entry{imply}{/ɪmˈplʌɪ/}{পরোক্ষভাবে প্রকাশ করা}{ \textsf{\textit{verb}}\ \textbf{1} Indicate the truth or existence of (something) by suggestion rather than explicit reference. {\fontspec{DejaVu Sans}◇} \textit{salesmen who use jargon to imply superior knowledge} \colorBulletS{SYN} insinuate, suggest, hint, intimate, implicate, say indirectly, indicate, give someone to understand, give someone to believe, convey the impression, signal}{}{}{ \colorBullet{ORIGIN} Late Middle English from Old French emplier, from Latin implicare, from in{-} ‘in’ + plicare ‘to fold’. The original sense was ‘entwine’; in the 16th and 17th centuries the word also meant ‘employ’. Compare with employ and implicate.}%
\par%
\entry{impose}{/ɪmˈpəʊz/}{আরোপ করা}{ \textsf{\textit{verb}}\ \textbf{1} Force (an unwelcome decision or ruling) on someone. {\fontspec{DejaVu Sans}◇} \textit{the decision was theirs and was not imposed on them by others} \colorBulletS{SYN} foist, force, thrust, inflict, obtrude, press, urge \textbf{2} Take advantage of someone by demanding their attention or commitment. {\fontspec{DejaVu Sans}◇} \textit{she realized that she had imposed on Mark's kindness} \colorBulletS{SYN} take advantage of, abuse, exploit, take liberties with, misuse, ill{-}treat, treat unfairly, manipulate \textbf{3} Arrange (pages of type) so as to be in the correct order after printing and folding. {\fontspec{DejaVu Sans}◇} \textit{}}{}{}{ \colorBullet{ORIGIN} Late 15th century (in the sense ‘impute’): from French imposer, from Latin imponere ‘inflict, deceive’ (from in{-} ‘in, upon’ + ponere ‘put’), but influenced by impositus ‘inflicted’ and Old French poser ‘to place’.}%
\par%
\entry{imposing}{/ɪmˈpəʊzɪŋ/}{মনোরম}{ \textsf{\textit{adjective}}\ \textbf{1} Grand and impressive in appearance. {\fontspec{DejaVu Sans}◇} \textit{an imposing 17th{-}century manor house} \colorBulletS{SYN} impressive, striking, arresting, eye{-}catching, dramatic, spectacular, staggering, stunning, awesome, awe{-}inspiring, remarkable, formidable}{}{}{}%
\par%
\entry{improbable}{/ɪmˈprɒbəb(ə)l/}{অভাবনীয়}{ \textsf{\textit{adjective}}\ \textbf{1} Not likely to be true or to happen. {\fontspec{DejaVu Sans}◇} \textit{this account of events was seen by the jury as most improbable} \colorBulletS{SYN} unlikely, not likely, doubtful, dubious, debatable, questionable, uncertain}{}{}{ \colorBullet{ORIGIN} Late 16th century from French, or from Latin improbabilis ‘hard to prove’, from in{-} ‘not’ + probabilis (see probable).}%
\par%
\entry{improvisation}{/ɪmprəvʌɪˈzeɪʃn/}{অচিন্তিত রচনা}{ \textsf{\textit{noun}}\ \textbf{1} The action of improvising. {\fontspec{DejaVu Sans}◇} \textit{she specializes in improvisation on the piano} \colorBulletS{SYN} extemporization, ad{-}libbing, spontaneity, lack of premeditation}{}{}{}%
\par%
\entry{improvise}{/ˈɪmprəvʌɪz/}{আশুরচনা করা}{ \textsf{\textit{verb}}\ \textbf{1} Create and perform (music, drama, or verse) spontaneously or without preparation. {\fontspec{DejaVu Sans}◇} \textit{he invited actors to improvise dialogue} \colorBulletS{SYN} extemporize, ad lib, speak impromptu, make it up as one goes along, think on one's feet, take it as it comes}{}{}{ \colorBullet{ORIGIN} Early 19th century (earlier (late 18th century) as improvisation): from French improviser or its source, Italian improvvisare, from improvviso ‘extempore’, from Latin improvisus ‘unforeseen’, based on provisus, past participle of providere ‘make preparation for’.}%
\par%
\entry{inadequate}{/ɪnˈadɪkwət/}{অপর্যাপ্ত}{ \textsf{\textit{adjective}}\ \textbf{1} Lacking the quality or quantity required; insufficient for a purpose. {\fontspec{DejaVu Sans}◇} \textit{these labels prove to be wholly inadequate} \colorBulletS{SYN} insufficient, not enough, deficient, poor, scant, scanty, scarce, sparse, too little, too few, short, in short supply}{}{}{}%
\par%
\entry{inadvertently}{/ˌɪnədˈvəːt(ə)ntli/}{অসাবধানতাবসত}{ \textsf{\textit{adverb}}\ \textbf{1} Without intention; accidentally. {\fontspec{DejaVu Sans}◇} \textit{his name had been inadvertently omitted from the list} \colorBulletS{SYN} accidentally, by accident, unintentionally, unwittingly}{}{}{}%
\par%
\entry{inauguration}{/ɪˌnɔːɡjʊˈreɪʃ(ə)n/}{উদ্বোধন}{ \textsf{\textit{noun}}\ \textbf{1} The beginning or introduction of a system, policy, or period. {\fontspec{DejaVu Sans}◇} \textit{the inauguration of an independent prosecution service} \colorBulletS{SYN} initiation, institution, setting up, launch, establishment, foundation, founding, origination, formation}{}{}{}%
\par%
\entry{incapable}{/ɪnˈkeɪpəb(ə)l/}{অসমর্থ}{ \textsf{\textit{adjective}}\ \textbf{1} Unable to do or achieve (something) {\fontspec{DejaVu Sans}◇} \textit{Wilson blushed and was incapable of speech} \colorBulletS{SYN} unable to, not capable of, lacking the ability to, not equipped to, lacking the experience to \textbf{2} Unable to behave rationally or manage one's affairs. {\fontspec{DejaVu Sans}◇} \textit{the pilot may become incapable from the lack of oxygen} \colorBulletS{SYN} incapacitated, helpless, powerless, impotent}{}{}{ \colorBullet{ORIGIN} Late 16th century from French, or from late Latin incapabilis, from in{-} ‘not’ + capabilis (see capable).}%
\par%
\entry{incisive}{/ɪnˈsʌɪsɪv/}{ব্যঙ্গকারী}{ \textsf{\textit{adjective}}\ \textbf{1} (of a person or mental process) intelligently analytical and clear{-}thinking. {\fontspec{DejaVu Sans}◇} \textit{she was an incisive critic} \colorBulletS{SYN} penetrating, acute, sharp, sharp{-}witted, razor{-}sharp, keen, rapier{-}like, astute, shrewd, trenchant, piercing, perceptive, insightful, percipient, perspicacious, discerning, analytical, intelligent, canny, clever, smart, quick \textbf{2} (of an action) quick and direct. {\fontspec{DejaVu Sans}◇} \textit{the most incisive move of a tight match}}{}{}{ \colorBullet{ORIGIN} Late Middle English (in the sense ‘cutting, penetrating’): from medieval Latin incisivus, from Latin incidere ‘cut into’ (see incise).}%
\par%
\entry{inclement}{/ɪnˈklɛm(ə)nt/}{ঝড়ো}{ \textsf{\textit{adjective}}\ \textbf{1} (of the weather) unpleasantly cold or wet. {\fontspec{DejaVu Sans}◇} \textit{walkers should be prepared for inclement weather} \colorBulletS{SYN} cold, chilly, bitter, bleak, raw, wintry, freezing, snowy, icy}{}{}{ \colorBullet{ORIGIN} Early 17th century from French inclément or Latin inclement{-}, from in{-} ‘not’ + clement{-} ‘clement’.}%
\par%
\entry{incline}{/ɪnˈklʌɪn/}{ঢলা}{\small{\textsf{\textit{noun, verb}}} \\{\fontspec{DejaVu Sans}▪ }\textsf{\textit{noun}}\\ \textbf{1} An inclined surface or plane; a slope, especially on a road or railway. {\fontspec{DejaVu Sans}◇} \textit{the road climbs a long incline through a forest} \colorBulletS{SYN} slope, gradient, pitch, ramp, bank, ascent, rise, acclivity, upslope, dip, descent, declivity, downslope \\{\fontspec{DejaVu Sans}▪ }\textsf{\textit{verb}}\\ \textbf{1} Be favourably disposed towards or willing to do something. {\fontspec{DejaVu Sans}◇} \textit{he was inclined to accept the offer} \colorBulletS{SYN} disposed, minded, of a mind, willing, ready, prepared \textbf{2} Have a tendency to do something. {\fontspec{DejaVu Sans}◇} \textit{she's inclined to gossip with complete strangers} \colorBulletS{SYN} liable, likely, prone, disposed, given, apt, wont, with a tendency \textbf{3} Lean or turn away from a given plane or direction, especially the vertical or horizontal. {\fontspec{DejaVu Sans}◇} \textit{the bunker doors incline outwards} \colorBulletS{SYN} lean, tilt, angle, tip, slope, slant, bend, curve, bank, cant, bevel}{}{}{ \colorBullet{ORIGIN} Middle English (originally in the sense ‘bend (the head or body) towards something’; formerly also as encline): from Old French encliner, from Latin inclinare, from in{-} ‘towards’ + clinare ‘to bend’.}%
\par%
\entry{incur}{/ɪnˈkəː/}{ভারাক্রান্ত করা}{ \textsf{\textit{verb}}\ \textbf{1} Become subject to (something unwelcome or unpleasant) as a result of one's own behaviour or actions. {\fontspec{DejaVu Sans}◇} \textit{I will pay any expenses incurred} \colorBulletS{SYN} suffer, sustain, experience, bring upon oneself, expose oneself to, lay oneself open to}{}{}{ \colorBullet{ORIGIN} Late Middle English from Latin incurrere, from in{-} ‘towards’ + currere ‘run’.}%
\par%
\entry{indecency}{/ɪnˈdiːsnsi/}{অশ্লীলতা}{ \textsf{\textit{noun}}\ \textbf{1} Indecent behaviour. {\fontspec{DejaVu Sans}◇} \textit{seven offences of rape and indecency} \colorBulletS{SYN} indecent behaviour, gross indecency, pornography}{}{}{}%
\par%
\entry{indictment}{/ɪnˈdʌɪtm(ə)nt/}{অভিযোগ}{ \textsf{\textit{noun}}\ \textbf{1} A formal charge or accusation of a serious crime. {\fontspec{DejaVu Sans}◇} \textit{an indictment for conspiracy} \colorBulletS{SYN} charge, accusation, arraignment, citation, summons \textbf{2} A thing that serves to illustrate that a system or situation is bad and deserves to be condemned. {\fontspec{DejaVu Sans}◇} \textit{these rapidly escalating crime figures are an indictment of our society}}{}{}{ \colorBullet{ORIGIN} Middle English enditement, inditement, from Anglo{-}Norman French enditement, from enditer (see indict).}%
\par%
\entry{indifferent}{/ɪnˈdɪf(ə)r(ə)nt/}{উদাসীন}{ \textsf{\textit{adjective}}\ \textbf{1} Having no particular interest or sympathy; unconcerned. {\fontspec{DejaVu Sans}◇} \textit{he gave an indifferent shrug} \colorBulletS{SYN} unconcerned about, apathetic about, apathetic towards, uncaring about, casual about, nonchalant about, offhand about, uninterested in, uninvolved in, uninvolved with \textbf{2} Neither good nor bad; mediocre. {\fontspec{DejaVu Sans}◇} \textit{a pair of indifferent watercolours} \colorBulletS{SYN} mediocre, ordinary, commonplace, average, middle{-}of{-}the{-}road, middling, medium, moderate, everyday, workaday, tolerable, passable, adequate, fair}{}{}{ \colorBullet{ORIGIN} Late Middle English (in the sense ‘having no partiality for or against’): via Old French from Latin indifferent{-} ‘not making any difference’, from in{-} ‘not’ + different{-} ‘differing’ (see different).}%
\par%
\entry{indiscriminate}{/ˌɪndɪˈskrɪmɪnət/}{বাছবিচারহীন}{ \textsf{\textit{adjective}}\ \textbf{1} Done at random or without careful judgement. {\fontspec{DejaVu Sans}◇} \textit{the indiscriminate use of antibiotics can cause problems} \colorBulletS{SYN} non{-}selective, unselective, undiscriminating, uncritical, aimless, hit{-}or{-}miss, haphazard, random, unsystematic, unmethodical}{}{}{ \colorBullet{ORIGIN} Late 16th century (in the sense ‘haphazard, not selective’): from in{-}‘not’ + Latin discriminatus, past participle of discriminare (see discriminate).}%
\par%
\entry{inevitable}{/ɪnˈɛvɪtəb(ə)l/}{অনিবার্য}{\small{\textsf{\textit{adjective, noun}}} \\{\fontspec{DejaVu Sans}▪ }\textsf{\textit{adjective}}\\ \textbf{1} Certain to happen; unavoidable. {\fontspec{DejaVu Sans}◇} \textit{war was inevitable} \colorBulletS{SYN} unavoidable, inescapable, bound to happen, sure to happen, inexorable, unpreventable, assured, certain, for sure, sure, fated, predestined, predetermined, preordained, ineluctable \\{\fontspec{DejaVu Sans}▪ }\textsf{\textit{noun}}\\ \textbf{1} A situation that is unavoidable. {\fontspec{DejaVu Sans}◇} \textit{by the morning he had accepted the inevitable}}{}{}{ \colorBullet{ORIGIN} Late Middle English from Latin inevitabilis, from in{-} ‘not’ + evitabilis ‘avoidable’ (from evitare ‘avoid’).}%
\par%
\entry{infatuation}{/ɪnˌfatʃʊˈeɪʃ(ə)n/}{মায়া}{ \textsf{\textit{noun}}\ \textbf{1} An intense but short{-}lived passion or admiration for someone or something. {\fontspec{DejaVu Sans}◇} \textit{he had developed an infatuation with the girl} \colorBulletS{SYN} passion for, love for, adoration of, desire for, fondness for, feeling for, regard for, devotion to, penchant for, preoccupation with, obsession with, fixation with, craze for, mania for, addiction to}{}{}{}%
\par%
\entry{infer}{/ɪnˈfəː/}{}{ \textsf{\textit{verb}}\ \textbf{1} Deduce or conclude (something) from evidence and reasoning rather than from explicit statements. {\fontspec{DejaVu Sans}◇} \textit{from these facts we can infer that crime has been increasing} \colorBulletS{SYN} deduce, reason, work out, conclude, come to the conclusion, draw the inference, conjecture, surmise, theorize, hypothesize}{}{}{ \colorBullet{ORIGIN} Late 15th century (in the sense ‘bring about, inflict’): from Latin inferre ‘bring in, bring about’ (in medieval Latin ‘deduce’), from in{-} ‘into’ + ferre ‘bring’.}%
\par%
\entry{inferior}{/ɪnˈfɪərɪə/}{নিকৃষ্ট}{\small{\textsf{\textit{adjective, noun}}} \\{\fontspec{DejaVu Sans}▪ }\textsf{\textit{adjective}}\\ \textbf{1} Lower in rank, status, or quality. {\fontspec{DejaVu Sans}◇} \textit{schooling in inner{-}city areas was inferior to that in the rest of the country} \colorBulletS{SYN} lower in status, lesser, second{-}class, second{-}fiddle, minor, subservient, lowly, humble, menial, not very important, not so important, below someone, beneath someone, under someone's heel \textbf{2} Low or lower in position. {\fontspec{DejaVu Sans}◇} \textit{ulcers located in the inferior and posterior wall of the duodenum} \textbf{3} (of a letter, figure, or symbol) written or printed below the line. {\fontspec{DejaVu Sans}◇} \textit{} \\{\fontspec{DejaVu Sans}▪ }\textsf{\textit{noun}}\\ \textbf{1} A person lower than another in rank, status, or ability. {\fontspec{DejaVu Sans}◇} \textit{her social and intellectual inferiors} \colorBulletS{SYN} subordinate, junior, underling, minion, menial \textbf{2} An inferior letter, figure, or symbol. {\fontspec{DejaVu Sans}◇} \textit{This mark indicates that the letter is superior to an inferior.}}{}{}{ \colorBullet{ORIGIN} Late Middle English (in inferior (sense 2 of the adjective)): from Latin, comparative of inferus ‘low’.}%
\par%
\entry{infiltrator}{/ˈɪnfɪltreɪtə/}{অনুপ্রবেশকারী}{\small{\textsf{\textit{}}}}{}{}{}%
\par%
\entry{inflatable}{/ɪnˈfleɪtəb(ə)l/}{বাজে}{\small{\textsf{\textit{adjective, noun}}} \\{\fontspec{DejaVu Sans}▪ }\textsf{\textit{adjective}}\\ \textbf{1} Capable of being filled with air. {\fontspec{DejaVu Sans}◇} \textit{an inflatable mattress} \\{\fontspec{DejaVu Sans}▪ }\textsf{\textit{noun}}\\ \textbf{1} A plastic or rubber object that must be filled with air before use. {\fontspec{DejaVu Sans}◇} \textit{three sailors manned the inflatable}}{}{}{}%
\par%
\entry{influx}{/ˈɪnflʌks/}{কোনো স্থানে লোকজনের ক্রমাগত আগমন}{ \textsf{\textit{noun}}\ \textbf{1} An arrival or entry of large numbers of people or things. {\fontspec{DejaVu Sans}◇} \textit{a massive influx of tourists} \colorBulletS{SYN} inundation, inrush, rush, stream, flood, incursion, ingress \textbf{2} An inflow of water into a river, lake, or the sea. {\fontspec{DejaVu Sans}◇} \textit{the lakes are fed by influxes of meltwater} \colorBulletS{SYN} inflow, inrush, flood, inundation}{}{The rohingya influx has caused 14.3 percent wage reduction of all labourers among the host community in teknaf}{ \colorBullet{ORIGIN} Late 16th century (denoting an inflow of liquid, gas, or light): from late Latin influxus, from influere ‘flow in’ (see influence).}%
\par%
\entry{infraction}{/ɪnˈfrakʃ(ə)n/}{ব্যত্যয়}{ \textsf{\textit{noun}}\ \textbf{1} A violation or infringement of a law or agreement. {\fontspec{DejaVu Sans}◇} \textit{} \colorBulletS{SYN} infringement, contravention, breach, violation, transgression, breaking}{}{}{ \colorBullet{ORIGIN} Late Middle English from Latin infractio(n{-}), from the verb infringere (see infringe).}%
\par%
\entry{infuriate}{/ɪnˈfjʊərɪeɪt/}{প্রকুপিত}{ \textsf{\textit{verb}}\ \textbf{1} Make (someone) extremely angry and impatient. {\fontspec{DejaVu Sans}◇} \textit{I was infuriated by your article} \colorBulletS{SYN} enrage, incense, anger, madden, inflame, send into a rage, make someone's blood boil, stir up, fire up}{}{}{ \colorBullet{ORIGIN} Mid 17th century from medieval Latin infuriat{-} ‘made angry’, from the verb infuriare, from in{-} ‘into’ + Latin furia ‘fury’.}%
\par%
\entry{infusion}{/ɪnˈfjuːʒ(ə)n/}{আধান}{ \textsf{\textit{noun}}\ \textbf{1} A drink, remedy, or extract prepared by soaking tea leaves or herbs in liquid. {\fontspec{DejaVu Sans}◇} \textit{a strong rosemary infusion} \colorBulletS{SYN} stock, broth, bouillon, juice, gravy, liquid, infusion, extract, concentrate, decoction \textbf{2} The introduction of a new element or quality into something. {\fontspec{DejaVu Sans}◇} \textit{the infusion of \$6.3 million for improvements} \colorBulletS{SYN} introduction, instilling, infusion, imbuing, inculcation \textbf{3} The slow injection of a substance into a vein or tissue. {\fontspec{DejaVu Sans}◇} \textit{a four{-}hour intravenous infusion}}{}{}{ \colorBullet{ORIGIN} Late Middle English (denoting the pouring in of a liquid): from Latin infusio(n{-}), from the verb infundere (see infuse).}%
\par%
\entry{ingratiate}{/ɪnˈɡreɪʃɪeɪt/}{অনুগ্রহ ভাজন করান}{ \textsf{\textit{verb}}\ \textbf{1} Bring oneself into favour with someone by flattering or trying to please them. {\fontspec{DejaVu Sans}◇} \textit{a sycophantic attempt to ingratiate herself with the local aristocracy} \colorBulletS{SYN} curry favour with, find the favour of, cultivate, win over, get on the good side of, get in someone's good books}{}{}{ \colorBullet{ORIGIN} Early 17th century from Latin in gratiam ‘into favour’, on the pattern of obsolete Italian ingratiare, earlier form of ingraziare.}%
\par%
\entry{injury}{/ˈɪn(d)ʒ(ə)ri/}{আঘাত}{ \textsf{\textit{noun}}\ \textbf{1} An instance of being injured. {\fontspec{DejaVu Sans}◇} \textit{she suffered an injury to her back} \colorBulletS{SYN} wound, bruise, cut, gash, tear, rent, slash, gouge, scratch, graze, laceration, abrasion, contusion, lesion, sore \textbf{2} Damage to a person's feelings. {\fontspec{DejaVu Sans}◇} \textit{compensation for injury to feelings} \colorBulletS{SYN} offence, abuse}{}{}{ \colorBullet{ORIGIN} Late Middle English from Anglo{-}Norman French injurie, from Latin injuria ‘a wrong’, from in{-} (expressing negation) + jus, jur{-} ‘right’.}%
\par%
\entry{innocence}{/ˈɪnəsəns/}{নিরীহতা}{ \textsf{\textit{noun}}\ \textbf{1} The state, quality, or fact of being innocent of a crime or offence. {\fontspec{DejaVu Sans}◇} \textit{they must prove their innocence} \colorBulletS{SYN} guiltlessness, blamelessness, freedom from guilt, freedom from blame, irreproachability, clean hands}{}{}{ \colorBullet{ORIGIN} Middle English from Old French, from Latin innocentia, from innocent{-} ‘not harming’ (based on nocere ‘injure’).}%
\par%
\entry{inquiry}{/ɪnˈkwʌɪri/}{অনুসন্ধান}{ \textsf{\textit{noun}}\ \textbf{1} An act of asking for information. {\fontspec{DejaVu Sans}◇} \textit{} \colorBulletS{SYN} question, query}{}{}{ \colorBullet{ORIGIN} Late Middle English (as enquery): from inquire + {-}y.}%
\par%
\entry{insane}{/ɪnˈseɪn/}{উন্মাদ}{ \textsf{\textit{adjective}}\ \textbf{1} In a state of mind which prevents normal perception, behaviour, or social interaction; seriously mentally ill. {\fontspec{DejaVu Sans}◇} \textit{he had gone insane} \colorBulletS{SYN} mentally ill, severely mentally disordered, of unsound mind, certifiable, psychotic, schizophrenic \textbf{2} Shocking; outrageous. {\fontspec{DejaVu Sans}◇} \textit{they were making insane amounts of money}}{}{}{ \colorBullet{ORIGIN} Mid 16th century from Latin insanus, from in{-} ‘not’ + sanus ‘healthy’.}%
\par%
\entry{insanity}{/ɪnˈsanəti/}{বাতুলতা}{ \textsf{\textit{noun}}\ \textbf{1} The state of being seriously mentally ill; madness. {\fontspec{DejaVu Sans}◇} \textit{he suffered from bouts of insanity} \colorBulletS{SYN} mental illness, mental disorder, mental derangement, madness, insaneness, dementia, dementedness, lunacy, instability, unsoundness of mind, loss of reason}{}{}{ \colorBullet{ORIGIN} Late 16th century from Latin insanitas, from insanus (see insane).}%
\par%
\entry{inscrutable}{/ɪnˈskruːtəb(ə)l/}{অবর্ণনীয়}{ \textsf{\textit{adjective}}\ \textbf{1} Impossible to understand or interpret. {\fontspec{DejaVu Sans}◇} \textit{Guy looked blankly inscrutable} \colorBulletS{SYN} enigmatic, unreadable, impenetrable, mysterious, impossible to interpret, cryptic}{}{}{ \colorBullet{ORIGIN} Late Middle English from ecclesiastical Latin inscrutabilis, from in{-} ‘not’ + scrutari ‘to search’ (see scrutiny).}%
\par%
\entry{insight}{/ˈɪnsʌɪt/}{সূক্ষ্মদৃষ্টি}{ \textsf{\textit{noun}}\ \textbf{1} The capacity to gain an accurate and deep understanding of someone or something. {\fontspec{DejaVu Sans}◇} \textit{his mind soared to previously unattainable heights of insight} \colorBulletS{SYN} intuition, perception, awareness, discernment, understanding, comprehension, apprehension, appreciation, cognizance, penetration, acumen, astuteness, perspicacity, perspicaciousness, sagacity, sageness, discrimination, judgement, shrewdness, sharpness, sharp{-}wittedness, acuity, acuteness, flair, breadth of view, vision, far{-}sightedness, prescience, imagination}{}{}{ \colorBullet{ORIGIN} Middle English (in the sense ‘inner sight, wisdom’): probably of Scandinavian and Low German origin and related to Swedish insikt, Danish indsigt, Dutch inzicht, and German Einsicht.}%
\par%
\entry{insist}{/ɪnˈsɪst/}{}{ \textsf{\textit{verb}}\ \textbf{1} Demand something forcefully, not accepting refusal. {\fontspec{DejaVu Sans}◇} \textit{she insisted on carrying her own bag} \colorBulletS{SYN} stand firm, be firm, stand one's ground, make a stand, stand up for oneself, be resolute, be determined, show determination, hold on, hold out, be emphatic, not take no for an answer, brook no refusal}{ \colorBullet{OTHER} insist on: জিদ}{}{ \colorBullet{ORIGIN} Late 16th century (in the sense ‘persist, persevere’): from Latin insistere ‘persist’, from in{-} ‘upon’ + sistere ‘stand’.}%
\par%
\entry{insistence}{/ɪnˈsɪst(ə)ns/}{গোঁ; জেদ}{ \textsf{\textit{noun}}\ \textbf{1} The fact or quality of insisting that something is the case or should be done. {\fontspec{DejaVu Sans}◇} \textit{Alison's insistence on doing the washing{-}up straight after the meal} \colorBulletS{SYN} demand, bidding, command, dictate, instruction, requirement, request, entreaty, urging, exhortation, importuning}{}{}{}%
\par%
\entry{instance}{/ˈɪnst(ə)ns/}{এই ক্ষেত্রে}{\small{\textsf{\textit{noun, verb}}} \\{\fontspec{DejaVu Sans}▪ }\textsf{\textit{noun}}\\ \textbf{1} An example or single occurrence of something. {\fontspec{DejaVu Sans}◇} \textit{a serious instance of corruption} \colorBulletS{SYN} example, occasion, occurrence, case, representative case, typical case, case in point, illustration, specimen, sample, exemplar, exemplification \\{\fontspec{DejaVu Sans}▪ }\textsf{\textit{verb}}\\ \textbf{1} Cite (a fact, case, etc.) as an example. {\fontspec{DejaVu Sans}◇} \textit{I instanced Bob as someone whose commitment had certainly got things done} \colorBulletS{SYN} cite, quote, refer to, make reference to, mention, allude to, adduce, give, give as an example, point to, point out}{ \colorBullet{OTHER} for instance: যেমন; উদাহরণস্বরূপ}{}{ \colorBullet{ORIGIN} Middle English via Old French from Latin instantia ‘presence, urgency’, from instare ‘be present, press upon’, from in{-} ‘upon’ + stare ‘to stand’. The original sense was ‘urgency, urgent entreaty’, surviving in at the instance of. In the late 16th century the word denoted a particular case cited to disprove a general assertion, derived from medieval Latin instantia ‘example to the contrary’ (translating Greek enstasis ‘objection’); hence the meaning ‘single occurrence’.}%
\par%
\entry{institution}{/ɪnstɪˈtjuːʃ(ə)n/}{প্রতিষ্ঠান}{ \textsf{\textit{noun}}\ \textbf{1} An organization founded for a religious, educational, professional, or social purpose. {\fontspec{DejaVu Sans}◇} \textit{an academic institution} \colorBulletS{SYN} organization, establishment, institute, foundation, centre \textbf{2} An established law or practice. {\fontspec{DejaVu Sans}◇} \textit{the institution of marriage} \colorBulletS{SYN} practice, custom, phenomenon, fact, procedure, convention, usage, tradition, rite, ritual, fashion, use, habit, wont \textbf{3} The action of instituting something. {\fontspec{DejaVu Sans}◇} \textit{a delay in the institution of proceedings} \colorBulletS{SYN} installation, instatement, induction, investiture, inauguration, introduction, swearing in, initiation}{}{}{ \colorBullet{ORIGIN} Late Middle English (in institution (sense 2, institution sense 3)): via Old French from Latin institutio(n{-}), from the verb instituere (see institute). institution (sense 1) dates from the early 18th century.}%
\par%
\entry{instrument}{/ˈɪnstrʊm(ə)nt/}{যন্ত্র}{\small{\textsf{\textit{noun, verb}}} \\{\fontspec{DejaVu Sans}▪ }\textsf{\textit{noun}}\\ \textbf{1} A tool or implement, especially one for precision work. {\fontspec{DejaVu Sans}◇} \textit{a surgical instrument} \colorBulletS{SYN} implement, tool, utensil, device, apparatus, contrivance, gadget, contraption, appliance, mechanism \textbf{2} A measuring device used to gauge the level, position, speed, etc. of something, especially a motor vehicle or aircraft. {\fontspec{DejaVu Sans}◇} \textit{a new instrument for measuring ozone levels} \colorBulletS{SYN} measuring device, gauge, meter, measure \textbf{3}  {\fontspec{DejaVu Sans}◇} \textit{the value of learning to play a musical instrument} \textbf{4} A formal or legal document. {\fontspec{DejaVu Sans}◇} \textit{execution involves signature and unconditional delivery of the instrument} \\{\fontspec{DejaVu Sans}▪ }\textsf{\textit{verb}}\\ \textbf{1} Equip (something) with measuring instruments. {\fontspec{DejaVu Sans}◇} \textit{engineers have instrumented rockets to study the upper atmosphere}}{}{}{ \colorBullet{ORIGIN} Middle English from Old French, or from Latin instrumentum ‘equipment, implement’, from the verb instruere ‘construct, equip’.}%
\par%
\entry{insufferable}{/ɪnˈsʌf(ə)rəb(ə)l/}{অসহনীয়}{ \textsf{\textit{adjective}}\ \textbf{1} Too extreme to bear; intolerable. {\fontspec{DejaVu Sans}◇} \textit{the heat would be insufferable by July} \colorBulletS{SYN} intolerable, unbearable, unendurable, insupportable, unacceptable, oppressive, overwhelming, overpowering, impossible, not to be borne, past bearing, too much to bear, more than one can stand, more than flesh and blood can stand, enough to tax the patience of a saint, enough to test the patience of a saint, enough to try the patience of a saint}{}{}{ \colorBullet{ORIGIN} Late Middle English perhaps via French (now dialect) insouffrable, based on Latin sufferre ‘endure’ (see suffer).}%
\par%
\entry{integrity}{/ɪnˈtɛɡrɪti/}{অখণ্ডতা}{ \textsf{\textit{noun}}\ \textbf{1} The quality of being honest and having strong moral principles. {\fontspec{DejaVu Sans}◇} \textit{a gentleman of complete integrity} \colorBulletS{SYN} honesty, uprightness, probity, rectitude, honour, honourableness, upstandingness, good character, principle, principles, ethics, morals, righteousness, morality, nobility, high{-}mindedness, right{-}mindedness, noble{-}mindedness, virtue, decency, fairness, scrupulousness, sincerity, truthfulness, trustworthiness \textbf{2} The state of being whole and undivided. {\fontspec{DejaVu Sans}◇} \textit{upholding territorial integrity and national sovereignty} \colorBulletS{SYN} unity, unification, wholeness, coherence, cohesion, undividedness, togetherness, solidarity, coalition}{}{}{ \colorBullet{ORIGIN} Late Middle English (in integrity (sense 2)): from French intégrité or Latin integritas, from integer ‘intact’ (see integer). Compare with entirety, integral, and integrate.}%
\par%
\entry{intended}{/ɪnˈtɛndɪd/}{অভিপ্রেত}{\small{\textsf{\textit{adjective, noun}}} \\{\fontspec{DejaVu Sans}▪ }\textsf{\textit{adjective}}\\ \textbf{1} Planned or meant. {\fontspec{DejaVu Sans}◇} \textit{the intended victim escaped} \colorBulletS{SYN} deliberate, intentional, calculated, conscious, done on purpose, planned, considered, studied, knowing, wilful, wanton, purposeful, purposive, premeditated, pre{-}planned, thought out in advance, prearranged, preconceived, predetermined \\{\fontspec{DejaVu Sans}▪ }\textsf{\textit{noun}}\\ \textbf{1} The person one intends to marry; one's fiancé or fiancée. {\fontspec{DejaVu Sans}◇} \textit{she used to be my intended} \colorBulletS{SYN} fiancée, fiancé, wife{-}to{-}be, husband{-}to{-}be, bride{-}to{-}be, future husband, future wife, prospective husband, prospective wife, prospective spouse}{}{}{}%
\par%
\entry{intense}{/ɪnˈtɛns/}{তীব্র}{ \textsf{\textit{adjective}}\ \textbf{1} Of extreme force, degree, or strength. {\fontspec{DejaVu Sans}◇} \textit{the job demands intense concentration} \colorBulletS{SYN} great, acute, enormous, fierce, severe, extreme, high, exceptional, extraordinary, harsh, strong, powerful, potent, vigorous \textbf{2} Having or showing strong feelings or opinions; extremely earnest or serious. {\fontspec{DejaVu Sans}◇} \textit{an intense young woman, passionate about her art} \colorBulletS{SYN} passionate, impassioned, ardent, earnest, fervent, fervid, hot{-}blooded, zealous, vehement, fiery, heated, feverish, emotional, heartfelt, eager, keen, enthusiastic, excited, animated, spirited, vigorous, strong, energetic, messianic, fanatical, committed}{}{}{ \colorBullet{ORIGIN} Late Middle English from Old French, or from Latin intensus ‘stretched tightly, strained’, past participle of intendere (see intend).}%
\par%
\entry{intensify}{/ɪnˈtɛnsɪfʌɪ/}{প্রবল বা তীব্র করে}{ \textsf{\textit{verb}}\ \textbf{1} Become or make more intense. {\fontspec{DejaVu Sans}◇} \textit{the dispute began to intensify} \colorBulletS{SYN} escalate, step up, boost, increase, raise, sharpen, strengthen, augment, add to, concentrate, reinforce \textbf{2} Increase the opacity of (a negative) using a chemical. {\fontspec{DejaVu Sans}◇} \textit{the negative may be intensified with bichloride}}{}{}{ \colorBullet{ORIGIN} Early 19th century coined by Coleridge.}%
\par%
\entry{intercept}{/ˌɪntəˈsɛpt/}{পথিমধ্যে রোধ করা}{\small{\textsf{\textit{noun, verb}}} \\{\fontspec{DejaVu Sans}▪ }\textsf{\textit{noun}}\\ \textbf{1} An act or instance of intercepting something. {\fontspec{DejaVu Sans}◇} \textit{he read the file of radio intercepts} \\{\fontspec{DejaVu Sans}▪ }\textsf{\textit{verb}}\\ \textbf{1} Obstruct (someone or something) so as to prevent them from continuing to a destination. {\fontspec{DejaVu Sans}◇} \textit{intelligence agencies intercepted a series of telephone calls} \colorBulletS{SYN} stop, head off, cut off}{}{}{ \colorBullet{ORIGIN} Late Middle English (in the senses ‘contain between limits’ and ‘halt (an effect’)): from Latin intercept{-} ‘caught between’, from the verb intercipere, from inter{-} ‘between’ + capere ‘take’.}%
\par%
\entry{intermittent}{/ɪntəˈmɪt(ə)nt/}{সবিরাম}{ \textsf{\textit{adjective}}\ \textbf{1} Occurring at irregular intervals; not continuous or steady. {\fontspec{DejaVu Sans}◇} \textit{intermittent rain} \colorBulletS{SYN} sporadic, irregular, fitful, spasmodic, broken, fragmentary, discontinuous, disconnected, isolated, odd, random, patchy, scattered}{}{Intermittent rain}{ \colorBullet{ORIGIN} Mid 16th century from Latin intermittent{-} ‘ceasing’, from the verb intermittere (see intermit).}%
\par%
\entry{intern}{/ˈɪntəːn/}{অন্তরীণ}{\small{\textsf{\textit{noun, verb}}} \\{\fontspec{DejaVu Sans}▪ }\textsf{\textit{noun}}\\ \textbf{1} A student or trainee who works, sometimes without pay, in order to gain work experience or satisfy requirements for a qualification. {\fontspec{DejaVu Sans}◇} \textit{} \colorBulletS{SYN} trainee, apprentice, probationer, student, novice, learner, beginner \\{\fontspec{DejaVu Sans}▪ }\textsf{\textit{verb}}\\ \textbf{1} Confine (someone) as a prisoner, especially for political or military reasons. {\fontspec{DejaVu Sans}◇} \textit{the family were interned for the duration of the war as enemy aliens} \colorBulletS{SYN} imprison, incarcerate, impound, jail, put in jail, put behind bars, detain, take into custody, hold in custody, hold captive, hold, lock up, keep under lock and key, confine \textbf{2} Serve as an intern. {\fontspec{DejaVu Sans}◇} \textit{}}{}{}{ \colorBullet{ORIGIN} Early 16th century (as an adjective in the sense ‘internal’): from French interne (adjective), interner (verb), from Latin internus ‘inward, internal’. Current senses date from the 19th century.}%
\par%
\entry{interrogation}{/ɪnˌtɛrəˈɡeɪʃ(ə)n/}{জিজ্ঞাসাবাদ}{ \textsf{\textit{noun}}\ \textbf{1} The action of interrogating or the process of being interrogated. {\fontspec{DejaVu Sans}◇} \textit{would he keep his mouth shut under interrogation?} \colorBulletS{SYN} questioning, cross{-}questioning, cross{-}examination, quizzing, probing, inquisition, catechism}{}{}{}%
\par%
\entry{intervene}{/ɪntəˈviːn/}{হস্তক্ষেপ করা}{ \textsf{\textit{verb}}\ \textbf{1} Take part in something so as to prevent or alter a result or course of events. {\fontspec{DejaVu Sans}◇} \textit{he acted outside his authority when he intervened in the dispute} \colorBulletS{SYN} intercede, involve oneself, get involved, interpose oneself, insinuate oneself, step in, cut in \textbf{2} Occur in the time between events. {\fontspec{DejaVu Sans}◇} \textit{to occupy the intervening months she took a job in a hospital}}{}{}{ \colorBullet{ORIGIN} Late 16th century (in the sense ‘come in as an extraneous factor or thing’): from Latin intervenire, from inter{-} ‘between’ + venire ‘come’.}%
\par%
\entry{intervention}{/ɪntəˈvɛnʃ(ə)n/}{হস্তক্ষেপ}{ \textsf{\textit{noun}}\ \textbf{1} The action or process of intervening. {\fontspec{DejaVu Sans}◇} \textit{a high degree of state intervention in the economy}}{}{}{ \colorBullet{ORIGIN} Late Middle English from Latin interventio(n{-}), from the verb intervenire (see intervene).}%
\par%
\entry{intestine}{/ɪnˈtɛstɪn/}{অন্ত্র}{ \textsf{\textit{noun}}\ \textbf{1} (in vertebrates) the lower part of the alimentary canal from the end of the stomach to the anus. {\fontspec{DejaVu Sans}◇} \textit{the contents of the intestine} \colorBulletS{SYN} gut, guts, entrails, viscera}{}{Rotavirus causes gastroenteritis, an inflammation of the stomach and intestines.}{ \colorBullet{ORIGIN} Late Middle English from Latin intestinum, neuter of intestinus, from intus ‘within’.}%
\par%
\entry{intimacy}{/ˈɪntɪməsi/}{অন্তরঙ্গতা}{ \textsf{\textit{noun}}\ \textbf{1} Close familiarity or friendship. {\fontspec{DejaVu Sans}◇} \textit{the intimacy between a husband and wife} \colorBulletS{SYN} closeness, togetherness, affinity, rapport, attachment, familiarity, confidentiality, close association, close relationship, close attachment, close friendship, friendliness, comradeship, companionship, amity, affection, mutual affection, warmth, warm feelings, understanding, fellow feeling}{}{}{}%
\par%
\entry{intimate}{/ˈɪntɪmət/}{অন্তরঙ্গ}{\small{\textsf{\textit{adjective, noun}}} \\{\fontspec{DejaVu Sans}▪ }\textsf{\textit{adjective}}\\ \textbf{1} Closely acquainted; familiar. {\fontspec{DejaVu Sans}◇} \textit{intimate friends} \colorBulletS{SYN} close, bosom, boon, dear, cherished, familiar, confidential, faithful, constant, devoted, fast, firm, favourite, special \textbf{2} Private and personal. {\fontspec{DejaVu Sans}◇} \textit{intimate details of his sexual encounters} \colorBulletS{SYN} personal, private, confidential, secret \\{\fontspec{DejaVu Sans}▪ }\textsf{\textit{noun}}\\ \textbf{1} A very close friend. {\fontspec{DejaVu Sans}◇} \textit{his circle of intimates} \colorBulletS{SYN} close friend, best friend, bosom friend, constant companion, alter ego, confidant, confidante, close associate}{}{}{ \colorBullet{ORIGIN} Early 17th century (as a noun): from late Latin intimatus, past participle of Latin intimare ‘impress, make familiar’, from intimus ‘inmost’.}%
\par%
\entry{intimate}{/ˈɪntɪmeɪt/}{অন্তরঙ্গ}{ \textsf{\textit{verb}}\ \textbf{1} State or make known. {\fontspec{DejaVu Sans}◇} \textit{Mr Hutchison has intimated his decision to retire} \colorBulletS{SYN} announce, state, proclaim, set forth, make known, make public, make plain, impart, disclose, reveal, divulge}{}{}{ \colorBullet{ORIGIN} Early 16th century (earlier (late Middle English) as intimation) from late Latin intimat{-} ‘made known’, from the verb intimare (see intimate).}%
\par%
\entry{intimidate}{/ɪnˈtɪmɪdeɪt/}{ভয় দেখান}{ \textsf{\textit{verb}}\ \textbf{1} Frighten or overawe (someone), especially in order to make them do what one wants. {\fontspec{DejaVu Sans}◇} \textit{the forts are designed to intimidate the nationalist population} \colorBulletS{SYN} frighten, menace, terrify, scare, alarm, terrorize, overawe, awe, cow, subdue, discourage, daunt, unnerve}{}{}{ \colorBullet{ORIGIN} Mid 17th century from medieval Latin intimidat{-} ‘made timid’, from the verb intimidare (based on timidus ‘timid’).}%
\par%
\entry{intimidation}{/ɪnˌtɪmɪˈdeɪʃn/}{হুমকি}{ \textsf{\textit{noun}}\ \textbf{1} The action of intimidating someone, or the state of being intimidated. {\fontspec{DejaVu Sans}◇} \textit{the intimidation of witnesses and jurors} \colorBulletS{SYN} frightening, menacing, terrifying, scaring, alarming, terrorization, terrorizing, cowing, subduing, daunting, unnerving}{}{}{}%
\par%
\entry{intoxicate}{/ɪnˈtɒksɪkeɪt/}{প্রমত্ত করা}{ \textsf{\textit{verb}}\ \textbf{1} (of alcoholic drink or a drug) cause (someone) to lose control of their faculties or behaviour. {\fontspec{DejaVu Sans}◇} \textit{he was charged with operating a vehicle while intoxicated} \colorBulletS{SYN} drunk, inebriated, inebriate, drunken, tipsy, the worse for drink, under the influence \textbf{2} Poison (someone). {\fontspec{DejaVu Sans}◇} \textit{}}{}{}{ \colorBullet{ORIGIN} Late Middle English (in the sense ‘poison’): from medieval Latin intoxicare, from in{-} ‘into’ + toxicare ‘to poison’, from Latin toxicum (see toxic).}%
\par%
\entry{intravenous}{/ˌɪntrəˈviːnəs/}{শিরায় প্রদানের জন্য}{ \textsf{\textit{adjective}}\ \textbf{1} Existing or taking place within, or administered into, a vein or veins. {\fontspec{DejaVu Sans}◇} \textit{an intravenous drip}}{}{}{}%
\par%
\entry{intricate}{/ˈɪntrɪkət/}{জটিল}{ \textsf{\textit{adjective}}\ \textbf{1} Very complicated or detailed. {\fontspec{DejaVu Sans}◇} \textit{an intricate network of canals} \colorBulletS{SYN} complex, complicated, convoluted, tangled, entangled, ravelled, twisted, knotty, maze{-}like, labyrinthine, winding, serpentine, circuitous, sinuous}{}{}{ \colorBullet{ORIGIN} Late Middle English from Latin intricat{-} ‘entangled’, from the verb intricare, from in{-} ‘into’ + tricae ‘tricks, perplexities’.}%
\par%
\entry{intriguing}{/ɪnˈtriːɡɪŋ/}{কুচুটে}{ \textsf{\textit{adjective}}\ \textbf{1} Arousing one's curiosity or interest; fascinating. {\fontspec{DejaVu Sans}◇} \textit{an intriguing story}}{}{}{}%
\par%
\entry{intriguingly}{/ɪnˈtriːɡɪŋli/}{}{ \textsf{\textit{adverb}}\ \textbf{1} In a manner that arouses one's curiosity or interest; fascinatingly. {\fontspec{DejaVu Sans}◇} \textit{}}{}{}{}%
\par%
\entry{inundate}{/ˈɪnʌndeɪt/}{প্রবাহিত করা}{ \textsf{\textit{verb}}\ \textbf{1} Overwhelm (someone) with things or people to be dealt with. {\fontspec{DejaVu Sans}◇} \textit{we've been inundated with complaints from listeners} \colorBulletS{SYN} overwhelm, overpower, overburden, overrun, overload, swamp, bog down, besiege, snow under, bury, bombard, glut \textbf{2} Flood. {\fontspec{DejaVu Sans}◇} \textit{the islands may be the first to be inundated as sea levels rise} \colorBulletS{SYN} flood, deluge, overflow, overrun, swamp, submerge, engulf, drown, immerse, cover}{}{}{ \colorBullet{ORIGIN} Late 16th century (earlier (late Middle English) as inundation) from Latin inundat{-} ‘flooded’, from the verb inundare, from in{-} ‘into, upon’ + undare ‘to flow’ (from unda ‘a wave’).}%
\par%
\entry{invariably}{/ɪnˈvɛːrɪəbli/}{অপরিবর্তনীয়ভাবে}{ \textsf{\textit{adverb}}\ \textbf{1} In every case or on every occasion; always. {\fontspec{DejaVu Sans}◇} \textit{ranch meals are invariably big and hearty} \colorBulletS{SYN} always, every time, each time, on every occasion, at all times, without fail, without exception, whatever happens, universally}{}{}{}%
\par%
\entry{inveigle}{/ɪnˈviːɡ(ə)l/}{মুগ্ধ করা}{ \textsf{\textit{verb}}\ \textbf{1} Persuade (someone) to do something by means of deception or flattery. {\fontspec{DejaVu Sans}◇} \textit{we cannot inveigle him into putting pen to paper} \colorBulletS{SYN} cajole, wheedle, coax, persuade, convince, talk}{}{}{ \colorBullet{ORIGIN} Late 15th century (in the sense ‘beguile, deceive’; formerly also as enveigle): from Anglo{-}Norman French envegler, alteration of Old French aveugler ‘to blind’, from aveugle ‘blind’.}%
\par%
\entry{invoke}{/ɪnˈvəʊk/}{ডাকা}{ \textsf{\textit{verb}}\ \textbf{1} Call on (a deity or spirit) in prayer, as a witness, or for inspiration. {\fontspec{DejaVu Sans}◇} \textit{} \colorBulletS{SYN} pray to, call on, appeal to, plead with, supplicate, entreat, solicit, beseech, beg, implore, importune, petition \textbf{2} Cite or appeal to (someone or something) as an authority for an action or in support of an argument. {\fontspec{DejaVu Sans}◇} \textit{the antiquated defence of insanity is rarely invoked in England} \colorBulletS{SYN} cite, refer to, adduce, instance \textbf{3} Cause (a procedure) to be carried out. {\fontspec{DejaVu Sans}◇} \textit{}}{}{}{ \colorBullet{ORIGIN} Late 15th century from French invoquer, from Latin invocare, from in{-} ‘upon’ + vocare ‘to call’.}%
\par%
\entry{involuntarily}{/ɪnˈvɒlənt(ə)rɪli/}{অনিচ্ছাজনিত}{ \textsf{\textit{adverb}}\ \textbf{1} Without will or conscious control. {\fontspec{DejaVu Sans}◇} \textit{she shuddered involuntarily at the memory} \textbf{2} Against someone's will; without someone's cooperation. {\fontspec{DejaVu Sans}◇} \textit{Alicia had her husband involuntarily hospitalized}}{}{}{}%
\par%
\entry{ire}{/ˈʌɪə/}{ক্রোধ}{ \textsf{\textit{noun}}\ \textbf{1} Anger. {\fontspec{DejaVu Sans}◇} \textit{the plans provoked the ire of conservationists} \colorBulletS{SYN} anger, rage, fury, wrath, hot temper, outrage, temper, crossness, spleen}{}{}{ \colorBullet{ORIGIN} Middle English via Old French from Latin ira.}%
\par%
\entry{irk}{/əːk/}{ক্লান্ত করে তোলা}{ \textsf{\textit{verb}}\ \textbf{1} Irritate; annoy. {\fontspec{DejaVu Sans}◇} \textit{it irks her to think of the runaround she received} \colorBulletS{SYN} irritate, annoy, vex, gall, rattle, pique, rub up the wrong way, exasperate, try someone's patience, put out, displease}{}{}{ \colorBullet{ORIGIN} Middle English (in the sense ‘be annoyed or disgusted’): perhaps from Old Norse yrkja ‘to work’.}%
\par%
\entry{irrelevant}{/ɪˈrɛlɪv(ə)nt/}{অপ্রাসঙ্গিক}{ \textsf{\textit{adjective}}\ \textbf{1} Not connected with or relevant to something. {\fontspec{DejaVu Sans}◇} \textit{an irrelevant comment} \colorBulletS{SYN} beside the point, not to the point, immaterial, not pertinent, not germane, off the subject, neither here nor there, unconnected, unrelated, peripheral, tangential, extraneous, inapposite, inapt, inapplicable}{}{}{}%
\par%
\entry{it's not what it looks like.}{}{এটি দেখতে যেমন দেখাচ্ছে তেমন নয়।}{\small{\textsf{\textit{}}}}{}{}{}%
\par%
\end{multicols}%
\pagebreak%
\section*{J}%
\begin{multicols}{2}%
\entry{jack}{/dʒak/}{নাবিক}{ \textsf{\textit{noun}}\ \textbf{1} A device for lifting heavy objects, especially one for raising the axle of a motor vehicle off the ground so that a wheel can be changed or the underside inspected. {\fontspec{DejaVu Sans}◇} \textit{} \textbf{2} A playing card bearing a representation of a soldier, page, or knave, normally ranking next below a queen. {\fontspec{DejaVu Sans}◇} \textit{} \textbf{3}  {\fontspec{DejaVu Sans}◇} \textit{} \textbf{4} A small white ball in bowls, at which the players aim. {\fontspec{DejaVu Sans}◇} \textit{} \textbf{5} A game played by tossing and catching small round pebbles or star{-}shaped pieces of metal or plastic. {\fontspec{DejaVu Sans}◇} \textit{} \textbf{6}  {\fontspec{DejaVu Sans}◇} \textit{he had that world{-}weary look of the working Jack who'd seen everything} \textbf{7} A small version of a national flag flown at the bow of a vessel in harbour to indicate its nationality. {\fontspec{DejaVu Sans}◇} \textit{At daylight we hoisted the jack for a pilot and a Delaware pilot came off, Boat C, but couldn't take us to New York.} \textbf{8} Money. {\fontspec{DejaVu Sans}◇} \textit{} \textbf{9} A device for turning a spit. {\fontspec{DejaVu Sans}◇} \textit{When running a spit from a weight driven clockwork jack, it is essential to ensure that the joint or bird is properly centred, or the spit may stop running.} \textbf{10} A part of the mechanism in a spinet or harpsichord that connects a key to its corresponding string and causes the string to be plucked when the key is pressed down. {\fontspec{DejaVu Sans}◇} \textit{} \textbf{11} A marine fish that is typically laterally compressed with a row of large spiky scales along each side, important in many places as food or game fish. {\fontspec{DejaVu Sans}◇} \textit{} \textbf{12} The male of various animals, especially a merlin or (US) an ass. {\fontspec{DejaVu Sans}◇} \textit{A mule results from a cross between a female horse, or mare, and a male donkey, or jack.} \textbf{13} Used in names of animals that are smaller than similar kinds, e.g. jack snipe. {\fontspec{DejaVu Sans}◇} \textit{The Jack Snipe is an extremely difficult bird to see, partly because they are not very common but mostly because they are so well{-}camouflaged they will often sit unnoticed and let you walk past them.} \textbf{14} short for jack shit {\fontspec{DejaVu Sans}◇} \textit{}}{}{}{ \colorBullet{ORIGIN} Late Middle English from Jack, pet form of the given name John. The term was used originally to denote an ordinary man (jack (sense 6)), also a youth (mid 16th century), hence the ‘knave’ in cards and ‘male animal’. The word also denoted various devices saving human labour, as though one had a helper (jack (sense 1, jack sense 3, jack sense 9, jack sense 10), and in compounds such as jackhammer and jackknife); the general sense ‘labourer’ arose in the early 18th century and survives in cheapjack, lumberjack, steeplejack, etc. Since the mid 16th century a notion of ‘smallness’ has arisen, hence jack (sense 4, jack sense 5, jack sense 7, jack sense 13).}%
\par%
\entry{jack}{/dʒak/}{নাবিক}{ \textsf{\textit{noun}}\ \textbf{1} another term for blackjack (sense 5) {\fontspec{DejaVu Sans}◇} \textit{} \textbf{2} A sleeveless padded tunic worn by foot soldiers. {\fontspec{DejaVu Sans}◇} \textit{}}{}{}{}%
\par%
\entry{jack}{/dʒak/}{নাবিক}{ \textsf{\textit{verb}}\ \textbf{1} Take (something) illicitly; steal. {\fontspec{DejaVu Sans}◇} \textit{what's wrong is to jack somebody's lyrics and not acknowledge the fact}}{}{}{ \colorBullet{ORIGIN} 1990s from hijack.}%
\par%
\entry{jack}{/dʒak/}{নাবিক}{ \textsf{\textit{adjective}}\ \textbf{1} Tired of or bored with someone or something. {\fontspec{DejaVu Sans}◇} \textit{people are getting jack of strikes}}{}{}{ \colorBullet{ORIGIN} Late 19th century from jack up ‘give up’ (see jack up).}%
\par%
\entry{jackal}{/ˈdʒakəl/}{শৃগাল}{ \textsf{\textit{noun}}\ \textbf{1} A slender long{-}legged wild dog that feeds on carrion, game, and fruit and often hunts cooperatively, found in Africa and southern Asia. {\fontspec{DejaVu Sans}◇} \textit{}}{}{}{ \colorBullet{ORIGIN} Early 17th century from Turkish çakal, from Persian šagāl. The change in the first syllable was due to association with jack.}%
\par%
\entry{jealous}{/ˈdʒɛləs/}{ঈর্ষান্বিত}{ \textsf{\textit{adjective}}\ \textbf{1} Feeling or showing an envious resentment of someone or their achievements, possessions, or perceived advantages. {\fontspec{DejaVu Sans}◇} \textit{she was always jealous of me} \colorBulletS{SYN} envious, covetous, desirous}{}{}{ \colorBullet{ORIGIN} Middle English from Old French gelos, from medieval Latin zelosus (see zealous).}%
\par%
\entry{jeopardize}{/ˈdʒɛpədʌɪz/}{বিপন্ন}{ \textsf{\textit{verb}}\ \textbf{1} Put (someone or something) into a situation in which there is a danger of loss, harm, or failure. {\fontspec{DejaVu Sans}◇} \textit{a devaluation of the dollar would jeopardize New York's position as a financial centre} \colorBulletS{SYN} threaten, endanger, imperil, menace, risk, put at risk, expose to risk, put in danger, expose to danger, put in jeopardy, put on the line}{}{}{}%
\par%
\entry{jibber{-}jabber}{/ˈdʒɪbədʒabə/}{Jibber jabber is Incoherent and unintelligible rapid speech often in slang or patois. Used in the UK as a disparaging name for other languages such as French, Spanish or American.}{\small{\textsf{\textit{noun, verb}}} \\{\fontspec{DejaVu Sans}▪ }\textsf{\textit{noun}}\\ \textbf{1} Rapid and excited speech that is difficult to understand. {\fontspec{DejaVu Sans}◇} \textit{enough jibber{-}jabber from me; let's get on with the story!} \\{\fontspec{DejaVu Sans}▪ }\textsf{\textit{verb}}\\ \textbf{1} Talk in a rapid and excited way that is difficult to understand. {\fontspec{DejaVu Sans}◇} \textit{he was jibber{-}jabbering with his wife through the entire first piece}}{}{“It was all jibber jabber. Couldn’t understand a dam’ word the wretched feller was sayin’, Jeeves.”\newline%
“But, Milord, he was an American.”\newline%
“Yes?”}{ \colorBullet{ORIGIN} Early 19th century related to gibber, jabber.}%
\par%
\entry{jolly}{/ˈdʒɒli/}{বলিষ্ঠ}{\small{\textsf{\textit{adjective, adverb, noun, verb}}} \\{\fontspec{DejaVu Sans}▪ }\textsf{\textit{adjective}}\\ \textbf{1} Happy and cheerful. {\fontspec{DejaVu Sans}◇} \textit{he was a jolly man full of jokes} \colorBulletS{SYN} cheerful, happy, cheery, good{-}humoured, jovial, merry, sunny, bright, joyful, light{-}hearted, in high spirits, in good spirits, sparkling, bubbly, exuberant, effervescent, ebullient, breezy, airy, lively, vivacious, full of life, sprightly, jaunty \\{\fontspec{DejaVu Sans}▪ }\textsf{\textit{adverb}}\\ \textbf{1} Very; extremely. {\fontspec{DejaVu Sans}◇} \textit{he is jolly busy} \colorBulletS{SYN} very, extremely, exceedingly, exceptionally, especially, tremendously, immensely, vastly, hugely \\{\fontspec{DejaVu Sans}▪ }\textsf{\textit{noun}}\\ \textbf{1} A party or celebration. {\fontspec{DejaVu Sans}◇} \textit{these events were jollies} \\{\fontspec{DejaVu Sans}▪ }\textsf{\textit{verb}}\\ \textbf{1} Encourage (someone) in a friendly way. {\fontspec{DejaVu Sans}◇} \textit{he jollied people along} \colorBulletS{SYN} encourage, urge, coax, cajole, persuade, wheedle}{}{}{ \colorBullet{ORIGIN} Middle English from Old French jolif, an earlier form of joli ‘pretty’, perhaps from Old Norse jól (see Yule).}%
\par%
\entry{jolly}{/ˈdʒɒli/}{বলিষ্ঠ}{ \textsf{\textit{noun}}\ \textbf{1} A clinker{-}built ship's boat that is smaller than a cutter, typically hoisted at the stern of the ship. {\fontspec{DejaVu Sans}◇} \textit{}}{}{}{ \colorBullet{ORIGIN} Early 18th century perhaps related to yawl.}%
\par%
\entry{jolt}{/dʒəʊlt/}{অস্পষ্ট}{\small{\textsf{\textit{noun, verb}}} \\{\fontspec{DejaVu Sans}▪ }\textsf{\textit{noun}}\\ \textbf{1} An abrupt rough or violent movement. {\fontspec{DejaVu Sans}◇} \textit{he felt a jolt when the plane started to climb} \colorBulletS{SYN} bump, bounce, shake, jerk, lurch, vibration \\{\fontspec{DejaVu Sans}▪ }\textsf{\textit{verb}}\\ \textbf{1} Push or shake (someone or something) abruptly and roughly. {\fontspec{DejaVu Sans}◇} \textit{a surge in the crowd behind him jolted him forwards} \colorBulletS{SYN} push, thrust}{}{}{ \colorBullet{ORIGIN} Late 16th century of unknown origin.}%
\par%
\entry{jubilee}{/ˈdʒuːbɪliː/}{জয়ন্তী}{ \textsf{\textit{noun}}\ \textbf{1} A special anniversary of an event, especially one celebrating twenty{-}five or fifty years of a reign or activity. {\fontspec{DejaVu Sans}◇} \textit{to celebrate its jubilee, the club is holding a tournament} \colorBulletS{SYN} anniversary, commemoration \textbf{2} A year of emancipation and restoration, kept every fifty years. {\fontspec{DejaVu Sans}◇} \textit{} \textbf{3} A period of remission from the penal consequences of sin, granted by the Roman Catholic Church under certain conditions for a year, usually at intervals of twenty{-}five years. {\fontspec{DejaVu Sans}◇} \textit{}}{}{}{ \colorBullet{ORIGIN} Late Middle English from Old French jubile, from late Latin jubilaeus (annus) ‘(year) of jubilee’, based on Hebrew yōḇēl, originally ‘ram's{-}horn trumpet’, with which the jubilee year was proclaimed.}%
\par%
\end{multicols}%
\pagebreak%
\section*{K}%
\begin{multicols}{2}%
\entry{keen}{/kiːn/}{উত্সাহী}{ \textsf{\textit{adjective}}\ \textbf{1} Having or showing eagerness or enthusiasm. {\fontspec{DejaVu Sans}◇} \textit{a keen gardener} \colorBulletS{SYN} eager, anxious, impatient, determined, desirous, longing, wishing, itching, dying, yearning, ambitious, ready \textbf{2} (of a sense) highly developed. {\fontspec{DejaVu Sans}◇} \textit{I have keen eyesight} \colorBulletS{SYN} acute, sharp, penetrating, discerning, sensitive, perceptive, piercing, clear, observant \textbf{3} (of the edge or point of a blade) sharp. {\fontspec{DejaVu Sans}◇} \textit{the keen blade went through the weeds} \colorBulletS{SYN} sharp, sharp{-}edged, sharpened, honed, razor{-}like, razor{-}sharp, whetted, fine{-}edged \textbf{4} (of activity or feeling) intense. {\fontspec{DejaVu Sans}◇} \textit{there could be keen competition to provide the service} \colorBulletS{SYN} intense, acute, extreme, fierce, violent, passionate, consuming, burning, fervent, fervid, ardent \textbf{5} Excellent. {\fontspec{DejaVu Sans}◇} \textit{I would soon fly to distant stars—how keen!}}{}{}{ \colorBullet{ORIGIN} Old English cēne ‘wise, clever’, also ‘brave, daring’, of Germanic origin; related to Dutch koen and German kühn ‘bold, brave’. Current senses date from Middle English.}%
\par%
\entry{keen}{/kiːn/}{উত্সাহী}{\small{\textsf{\textit{noun, verb}}} \\{\fontspec{DejaVu Sans}▪ }\textsf{\textit{noun}}\\ \textbf{1} An Irish funeral song accompanied by wailing in lamentation for the dead. {\fontspec{DejaVu Sans}◇} \textit{} \\{\fontspec{DejaVu Sans}▪ }\textsf{\textit{verb}}\\ \textbf{1} Wail in grief for a dead person. {\fontspec{DejaVu Sans}◇} \textit{the body of Johnny was taken by his own people who keened over him} \colorBulletS{SYN} lament, mourn, weep, cry, sob, sorrow, grieve}{}{}{ \colorBullet{ORIGIN} Mid 19th century from Irish caoinim ‘I wail’.}%
\par%
\entry{kid}{/kɪd/}{বাচ্চা}{\small{\textsf{\textit{noun, verb}}} \\{\fontspec{DejaVu Sans}▪ }\textsf{\textit{noun}}\\ \textbf{1} A child or young person. {\fontspec{DejaVu Sans}◇} \textit{she collected the kids from school} \colorBulletS{SYN} child, youngster, little one, young one, baby, toddler, infant, boy, girl, young person, minor, juvenile, adolescent, teenager, youth, stripling \textbf{2} A young goat. {\fontspec{DejaVu Sans}◇} \textit{} \\{\fontspec{DejaVu Sans}▪ }\textsf{\textit{verb}}\\ \textbf{1} (of a goat) give birth. {\fontspec{DejaVu Sans}◇} \textit{milk fever usually occurs in heavy milkers shortly after kidding}}{}{1. I kid you not: আমি আপনার সাথে ঠাট্টা করছি না2. I kid, of course. }{ \colorBullet{ORIGIN} Middle English (in kid (sense 2 of the noun)): from Old Norse kith, of Germanic origin; related to German Kitze.}%
\par%
\entry{kid}{/kɪd/}{বাচ্চা}{ \textsf{\textit{verb}}\ \textbf{1} Deceive (someone) in a playful way; tease. {\fontspec{DejaVu Sans}◇} \textit{you're kidding me!} \colorBulletS{SYN} joke, tease, jest, chaff, be facetious}{}{1. I kid you not: আমি আপনার সাথে ঠাট্টা করছি না2. I kid, of course. }{ \colorBullet{ORIGIN} Early 19th century perhaps from kid, expressing the notion ‘make a child or goat of’.}%
\par%
\entry{kid}{/kɪd/}{বাচ্চা}{ \textsf{\textit{noun}}\ \textbf{1} A small wooden tub, especially a sailor's mess tub for grog or rations. {\fontspec{DejaVu Sans}◇} \textit{}}{}{1. I kid you not: আমি আপনার সাথে ঠাট্টা করছি না2. I kid, of course. }{ \colorBullet{ORIGIN} Mid 18th century perhaps a variant of kit.}%
\par%
\entry{kiln}{/kɪln/}{ভাটা}{ \textsf{\textit{noun}}\ \textbf{1} A furnace or oven for burning, baking, or drying, especially one for calcining lime or firing pottery. {\fontspec{DejaVu Sans}◇} \textit{}}{}{}{ \colorBullet{ORIGIN} Old English cylene, from Latin culina ‘kitchen, cooking stove’.}%
\par%
\entry{kitty}{/ˈkɪti/}{বিড়ালছানা}{ \textsf{\textit{noun}}\ \textbf{1} A fund of money for communal use, made up of contributions from a group of people. {\fontspec{DejaVu Sans}◇} \textit{} \colorBulletS{SYN} fund, funds, reserves, resources, money, finances, wealth, cash, wherewithal, capital, assets, deep pockets, purse, kitty, pool, bank, treasury, exchequer \textbf{2} (in bowls) the jack. {\fontspec{DejaVu Sans}◇} \textit{}}{}{}{ \colorBullet{ORIGIN} Early 19th century (denoting a jail): of unknown origin.}%
\par%
\entry{kitty}{/ˈkɪti/}{বিড়ালছানা}{ \textsf{\textit{noun}}\ \textbf{1} A pet name or a child's name for a kitten or cat. {\fontspec{DejaVu Sans}◇} \textit{}}{}{}{}%
\par%
\entry{kmn}{}{Abbreviation for "kill me now"}{\small{\textsf{\textit{}}}}{}{Person 1: when will all these election ads be done?\newline%
Person 2: november.\newline%
Person 1: kmn}{}%
\par%
\entry{knee}{/niː/}{হাঁটু}{\small{\textsf{\textit{noun, verb}}} \\{\fontspec{DejaVu Sans}▪ }\textsf{\textit{noun}}\\ \textbf{1} The joint between the thigh and the lower leg in humans. {\fontspec{DejaVu Sans}◇} \textit{} \textbf{2} An angled piece of wood or metal frame used to connect and support the beams and timbers of a wooden ship. {\fontspec{DejaVu Sans}◇} \textit{The deck and hull are through bolted on an inward flange and structural knees and bulkheads are securely attached.} \textbf{3} An abrupt obtuse or approximately right{-}angled bend in a graph between parts where the slope varies smoothly. {\fontspec{DejaVu Sans}◇} \textit{} \\{\fontspec{DejaVu Sans}▪ }\textsf{\textit{verb}}\\ \textbf{1} Hit (someone) with one's knee. {\fontspec{DejaVu Sans}◇} \textit{she kneed him in the groin}}{}{}{ \colorBullet{ORIGIN} Old English cnēow, cnēo, of Germanic origin; related to Dutch knie and German Knie, from an Indo{-}European root shared by Latin genu and Greek gonu.}%
\par%
\entry{knock (one) off (one's) feet}{}{To thoroughly impress, overwhelm, or excite one.}{\small{\textsf{\textit{}}}}{}{1. you knocked me off my feet 2. The final 30 minutes of the film completely knocked me off my feet.}{}%
\par%
\entry{knowingly}{/ˈnəʊɪŋli/}{জ্ঞাতসারে}{ \textsf{\textit{adverb}}\ \textbf{1} In a way that suggests one has secret knowledge or awareness. {\fontspec{DejaVu Sans}◇} \textit{Amy looked at me knowingly} \colorBulletS{SYN} deliberately, intentionally, consciously, wittingly, with full knowledge, in full awareness, with one's eyes open, on purpose, by design, calculatedly, premeditatedly, studiedly, wilfully, purposefully, willingly \textbf{2} In full awareness or consciousness; deliberately. {\fontspec{DejaVu Sans}◇} \textit{when a journalist knowingly misleads their readers}}{}{}{}%
\par%
\end{multicols}%
\pagebreak%
\section*{L}%
\begin{multicols}{2}%
\entry{laden}{/ˈleɪd(ə)n/}{ভারাক্রান্ত}{ \textsf{\textit{adjective}}\ \textbf{1} Heavily loaded or weighed down. {\fontspec{DejaVu Sans}◇} \textit{a tree laden with apples} \colorBulletS{SYN} loaded, burdened, weighed down, overloaded, weighted, piled high, fully charged, encumbered, hampered, oppressed, taxed}{}{}{ \colorBullet{ORIGIN} Late 16th century past participle of lade.}%
\par%
\entry{lag}{/laɡ/}{পিছনে ধীরে ধীরে চলা}{\small{\textsf{\textit{noun, verb}}} \\{\fontspec{DejaVu Sans}▪ }\textsf{\textit{noun}}\\ \textbf{1}  {\fontspec{DejaVu Sans}◇} \textit{a time lag between infection and symptoms} \textbf{2} A retardation in an electric current or movement. {\fontspec{DejaVu Sans}◇} \textit{With a longitudinal bias field, there was a lag of about 3.5 ns as the magnetization responded to the switching pulse.} \\{\fontspec{DejaVu Sans}▪ }\textsf{\textit{verb}}\\ \textbf{1} Fail to keep up with another or others in movement or development. {\fontspec{DejaVu Sans}◇} \textit{they waited for Tim who was lagging behind} \colorBulletS{SYN} fall behind, straggle, fall back, trail, trail behind, linger, dally, dawdle, hang back, delay, move slowly, loiter, drag one's feet, take one's time, not keep pace, idle, dither, saunter, bring up the rear \textbf{2} another term for string (sense 6 of the verb) {\fontspec{DejaVu Sans}◇} \textit{}}{}{Lag behind}{ \colorBullet{ORIGIN} Early 16th century (as a noun in the sense ‘hindmost person in a game, race, etc.’, also ‘dregs’): related to the dialect adjective lag(perhaps from a fanciful distortion of last, or of Scandinavian origin: compare with Norwegian dialect lagga ‘go slowly’).}%
\par%
\entry{lag}{/laɡ/}{পিছনে ধীরে ধীরে চলা}{ \textsf{\textit{verb}}\ \textbf{1} Enclose or cover (a boiler, pipes, etc.) with material that provides heat insulation. {\fontspec{DejaVu Sans}◇} \textit{all pipes and tanks in the attic should be lagged}}{}{Lag behind}{ \colorBullet{ORIGIN} Late 19th century from earlier lag ‘piece of insulating cover’.}%
\par%
\entry{lag}{/laɡ/}{পিছনে ধীরে ধীরে চলা}{\small{\textsf{\textit{noun, verb}}} \\{\fontspec{DejaVu Sans}▪ }\textsf{\textit{noun}}\\ \textbf{1} A person who has been frequently convicted and sent to prison. {\fontspec{DejaVu Sans}◇} \textit{both old lags were sentenced to ten years' imprisonment} \\{\fontspec{DejaVu Sans}▪ }\textsf{\textit{verb}}\\ \textbf{1} Arrest or send to prison. {\fontspec{DejaVu Sans}◇} \textit{they were nearly lagged by the constables}}{}{Lag behind}{ \colorBullet{ORIGIN} Late 16th century (as a verb in the sense ‘carry off, steal’): of unknown origin. Current senses date from the 19th century.}%
\par%
\entry{landslide}{/ˈlan(d)slʌɪd/}{ভূমিস্থলন}{ \textsf{\textit{noun}}\ \textbf{1} A collapse of a mass of earth or rock from a mountain or cliff. {\fontspec{DejaVu Sans}◇} \textit{the road was blocked by a landslide} \colorBulletS{SYN} landslip, rockfall, mudslide, earthslip, earthfall \textbf{2} An overwhelming majority of votes for one party or candidate in an election. {\fontspec{DejaVu Sans}◇} \textit{they won by a landslide} \colorBulletS{SYN} decisive victory, runaway victory, overwhelming majority, grand slam, triumph, walkover, game, set, and match}{}{}{}%
\par%
\entry{languish}{/ˈlaŋɡwɪʃ/}{শক্তিহীনতা}{ \textsf{\textit{verb}}\ \textbf{1} (of a person, animal, or plant) lose or lack vitality; grow weak. {\fontspec{DejaVu Sans}◇} \textit{plants may appear to be languishing simply because they are dormant} \colorBulletS{SYN} weaken, grow weak, deteriorate, decline, go into a decline \textbf{2} Be forced to remain in an unpleasant place or situation. {\fontspec{DejaVu Sans}◇} \textit{he has been languishing in jail since 1974} \colorBulletS{SYN} waste away, rot, decay, wither away, moulder, be abandoned, be neglected, be forgotten, suffer}{}{}{ \colorBullet{ORIGIN} Middle English (in the sense ‘become faint, feeble, or ill’): from Old French languiss{-}, lengthened stem of languir ‘languish’, from a variant of Latin languere, related to laxus ‘loose, lax’.}%
\par%
\entry{larva}{/ˈlɑːvə/}{শুককীট}{ \textsf{\textit{noun}}\ \textbf{1} The active immature form of an insect, especially one that differs greatly from the adult and forms the stage between egg and pupa, e.g. a caterpillar or grub. {\fontspec{DejaVu Sans}◇} \textit{}}{}{}{ \colorBullet{ORIGIN} Mid 17th century (denoting a disembodied spirit or ghost): from Latin, literally ‘ghost, mask’.}%
\par%
\entry{lash}{/laʃ/}{কশাঘাত}{\small{\textsf{\textit{noun, verb}}} \\{\fontspec{DejaVu Sans}▪ }\textsf{\textit{noun}}\\ \textbf{1} A sharp blow or stroke with a whip or rope. {\fontspec{DejaVu Sans}◇} \textit{he was sentenced to fifty lashes for his crime} \colorBulletS{SYN} stroke, blow, hit, strike, welt, bang, thwack, thump \textbf{2} An eyelash. {\fontspec{DejaVu Sans}◇} \textit{she fluttered her long dark lashes} \\{\fontspec{DejaVu Sans}▪ }\textsf{\textit{verb}}\\ \textbf{1} Strike or beat with a whip or stick. {\fontspec{DejaVu Sans}◇} \textit{they lashed him repeatedly about the head} \colorBulletS{SYN} whip, flog, beat, thrash, horsewhip, scourge, birch, switch, flay, belt, strap, cane, leather \textbf{2} (of an animal) move (a part of the body, especially the tail) quickly and violently. {\fontspec{DejaVu Sans}◇} \textit{the cat was lashing its tail back and forth} \colorBulletS{SYN} swish, flick, twitch, switch, whip, wave, wag \textbf{3} Fasten (something) securely with a cord or rope. {\fontspec{DejaVu Sans}◇} \textit{the hatch was securely lashed down} \colorBulletS{SYN} fasten, bind, tie, tie up, tether, hitch, attach, knot, rope, strap, leash, truss, fetter, make fast, secure}{}{}{ \colorBullet{ORIGIN} Middle English (in the sense ‘make a sudden movement’): probably imitative.}%
\par%
\entry{lath}{/lɑːθ/}{ছিলকা}{\small{\textsf{\textit{noun, verb}}} \\{\fontspec{DejaVu Sans}▪ }\textsf{\textit{noun}}\\ \textbf{1} A thin flat strip of wood, especially one of a series forming a foundation for the plaster of a wall. {\fontspec{DejaVu Sans}◇} \textit{} \colorBulletS{SYN} joist, purlin, girder, spar, support, strut, stay, brace, scantling, batten, transom, lintel, stringer, balk, board, timber, plank, lath, rafter \\{\fontspec{DejaVu Sans}▪ }\textsf{\textit{verb}}\\ \textbf{1} Cover with laths. {\fontspec{DejaVu Sans}◇} \textit{}}{}{}{ \colorBullet{ORIGIN} Old English lætt, of Germanic origin; related to Dutch lat and German Latte, also to lattice.}%
\par%
\entry{laud}{/lɔːd/}{প্রশংসা}{\small{\textsf{\textit{noun, verb}}} \\{\fontspec{DejaVu Sans}▪ }\textsf{\textit{noun}}\\ \textbf{1} Praise. {\fontspec{DejaVu Sans}◇} \textit{all glory, laud, and honour to Thee Redeemer King} \\{\fontspec{DejaVu Sans}▪ }\textsf{\textit{verb}}\\ \textbf{1} Praise (a person or their achievements) highly. {\fontspec{DejaVu Sans}◇} \textit{the obituary lauded him as a great statesman and soldier} \colorBulletS{SYN} praise, extol, hail, applaud, acclaim, commend, admire, approve of, make much of, sing the praises of, lionize, speak highly of, pay homage to, pay tribute to, eulogize, sing paeans to}{}{}{ \colorBullet{ORIGIN} Late Middle English the noun from Old French laude, the verb from Latin laudare, both from Latin laus, laud{-} ‘praise’ (see also lauds).}%
\par%
\entry{lax}{/laks/}{শিথিল}{ \textsf{\textit{adjective}}\ \textbf{1} Not sufficiently strict, severe, or careful. {\fontspec{DejaVu Sans}◇} \textit{lax security arrangements at the airport} \colorBulletS{SYN} slack, slipshod, negligent, neglectful, remiss, careless, heedless, unmindful, inattentive, slapdash, offhand, casual \textbf{2} (of the limbs or muscles) relaxed. {\fontspec{DejaVu Sans}◇} \textit{muscles have more potential energy when they are stretched than when they are lax}}{}{}{ \colorBullet{ORIGIN} Late Middle English (in the sense ‘loose’, said of the bowels): from Latin laxus.}%
\par%
\entry{lax}{/laks/}{শিথিল}{ \textsf{\textit{noun}}\ \textbf{1} Lacrosse. {\fontspec{DejaVu Sans}◇} \textit{I wore pads and a helmet whenever I played lax}}{}{}{ \colorBullet{ORIGIN} 1950s abbreviation of lacrosse, with x representing crosse (by association with cross).}%
\par%
\entry{led}{/lɛd/}{চালিত}{\small{\textsf{\textit{}}}}{ \colorBullet{OTHER} led by}{}{}%
\par%
\entry{LED}{/ɛliːˈdiː/}{চালিত}{ \textsf{\textit{noun}}\ \textbf{1} A light{-}emitting diode (a semiconductor diode which glows when a voltage is applied) {\fontspec{DejaVu Sans}◇} \textit{light sources can be fluorescent tubes, optical fibres, or LEDs}}{ \colorBullet{OTHER} led by}{}{ \colorBullet{ORIGIN} 1960s abbreviation.}%
\par%
\entry{leer}{/lɪə/}{অপাঙ্গদৃষ্টি}{\small{\textsf{\textit{noun, verb}}} \\{\fontspec{DejaVu Sans}▪ }\textsf{\textit{noun}}\\ \textbf{1} A lascivious or unpleasant look. {\fontspec{DejaVu Sans}◇} \textit{he gave me a sly leer} \colorBulletS{SYN} lecherous look, lascivious look, suggestive look, ogle, sly glance, stare \\{\fontspec{DejaVu Sans}▪ }\textsf{\textit{verb}}\\ \textbf{1} Look or gaze in a lascivious or unpleasant way. {\fontspec{DejaVu Sans}◇} \textit{bystanders were leering at the nude painting} \colorBulletS{SYN} ogle, look lasciviously, look suggestively, give sly looks to, eye, watch, stare, goggle}{}{}{ \colorBullet{ORIGIN} Mid 16th century (in the general sense ‘look sideways or askance’): perhaps from obsolete leer ‘cheek’, from Old English hlēor, as though the sense were ‘to glance over one's cheek’.}%
\par%
\entry{leer}{}{অপাঙ্গদৃষ্টি}{\small{\textsf{\textit{}}}}{}{}{}%
\par%
\entry{legacy}{/ˈlɛɡəsi/}{উত্তরাধিকার}{\small{\textsf{\textit{adjective, noun}}} \\{\fontspec{DejaVu Sans}▪ }\textsf{\textit{adjective}}\\ \textbf{1} Denoting or relating to software or hardware that has been superseded but is difficult to replace because of its wide use. {\fontspec{DejaVu Sans}◇} \textit{} \\{\fontspec{DejaVu Sans}▪ }\textsf{\textit{noun}}\\ \textbf{1} An amount of money or property left to someone in a will. {\fontspec{DejaVu Sans}◇} \textit{my grandmother died and unexpectedly left me a small legacy} \colorBulletS{SYN} bequest, inheritance, heritage, bequeathal, bestowal, benefaction, endowment, gift, patrimony, heirloom, settlement, birthright, provision \textbf{2} An applicant to a particular college or university who is regarded preferentially because a parent or other relative attended the same institution. {\fontspec{DejaVu Sans}◇} \textit{being a legacy increased a student's chance of being accepted to a highly selective college by up to 45 per cent}}{}{}{ \colorBullet{ORIGIN} Late Middle English (also denoting the function or office of a deputy, especially a papal legate): from Old French legacie, from medieval Latin legatia ‘legateship’, from legatus ‘person delegated’ (see legate).}%
\par%
\entry{legitimate}{/lɪˈdʒɪtɪmət/}{বৈধ}{\small{\textsf{\textit{adjective, verb}}} \\{\fontspec{DejaVu Sans}▪ }\textsf{\textit{adjective}}\\ \textbf{1} Conforming to the law or to rules. {\fontspec{DejaVu Sans}◇} \textit{his claims to legitimate authority} \colorBulletS{SYN} legal, lawful, licit, legalized, authorized, permitted, permissible, allowable, allowed, admissible, recognized, sanctioned, approved, licensed, statutory, constitutional, within the law, going by the rules, above board, valid, honest, upright \textbf{2} Able to be defended with logic or justification; valid. {\fontspec{DejaVu Sans}◇} \textit{a legitimate excuse for being late} \colorBulletS{SYN} valid, sound, admissible, acceptable, well founded, justifiable, reasonable, sensible, tenable, defensible, supportable, just, warrantable, fair, bona fide, proper, genuine, plausible, credible, believable, reliable, understandable, logical, rational \textbf{3} Constituting or relating to serious drama as distinct from musical comedy, revue, etc. {\fontspec{DejaVu Sans}◇} \textit{the legitimate theatre} \\{\fontspec{DejaVu Sans}▪ }\textsf{\textit{verb}}\\ \textbf{1} Make lawful or justify. {\fontspec{DejaVu Sans}◇} \textit{the regime was not legitimated by popular support}}{}{}{ \colorBullet{ORIGIN} Late Middle English (in the sense ‘born of parents lawfully married to each other’): from medieval Latin legitimatus ‘made legal’, from the verb legitimare, from Latin legitimus ‘lawful’, from lex, leg{-} ‘law’.}%
\par%
\entry{lending}{/ˈlɛndɪŋ/}{ঋণদান}{ \textsf{\textit{noun}}\ \textbf{1} The action of allowing a person or organization the use of a sum of money under an agreement to pay it back later. {\fontspec{DejaVu Sans}◇} \textit{balance sheets weakened by unwise lending}}{}{}{}%
\par%
\entry{lentil}{/ˈlɛnt(ə)l/}{মসুর}{ \textsf{\textit{noun}}\ \textbf{1} A high{-}protein pulse which is dried and then soaked and cooked prior to eating. {\fontspec{DejaVu Sans}◇} \textit{} \textbf{2} The plant which yields lentils, native to the Mediterranean and Africa and grown also for fodder. {\fontspec{DejaVu Sans}◇} \textit{Settlements began to encourage the growth of plants such as barley and lentils and the domestication of pigs, sheep and goats.}}{}{}{ \colorBullet{ORIGIN} Middle English from Old French lentille, from Latin lenticula, diminutive of lens, lent{-} ‘lentil’.}%
\par%
\entry{lest}{/lɛst/}{পাছে}{ \textsf{\textit{conjunction}}\ \textbf{1} With the intention of preventing (something undesirable); to avoid the risk of. {\fontspec{DejaVu Sans}◇} \textit{he spent whole days in his room, wearing headphones lest he disturb anyone}}{}{}{ \colorBullet{ORIGIN} Old English thȳ lǣs the ‘whereby less that’, later the læste.}%
\par%
\entry{levy}{/ˈlɛvi/}{ধার্য}{\small{\textsf{\textit{noun, verb}}} \\{\fontspec{DejaVu Sans}▪ }\textsf{\textit{noun}}\\ \textbf{1} An act of levying a tax, fee, or fine. {\fontspec{DejaVu Sans}◇} \textit{police forces receive 49 per cent of their funding via a levy on the rates} \colorBulletS{SYN} tax, tariff, toll, excise, duty, fee, imposition, impost, exaction, assessment, tithe, payment \textbf{2} An act of enlisting troops. {\fontspec{DejaVu Sans}◇} \textit{Edward I and Edward II had made substantial use of the feudal levy for raising an army} \\{\fontspec{DejaVu Sans}▪ }\textsf{\textit{verb}}\\ \textbf{1} Impose (a tax, fee, or fine) {\fontspec{DejaVu Sans}◇} \textit{a tax of two per cent was levied on all cargoes} \colorBulletS{SYN} impose, charge, exact, demand, raise, collect, gather \textbf{2} Enlist (someone) for military service. {\fontspec{DejaVu Sans}◇} \textit{he sought to levy one man from each vill for service} \colorBulletS{SYN} conscript, call up, enlist, mobilize, rally, muster, marshal, press, recruit, raise, assemble, round up}{}{}{ \colorBullet{ORIGIN} Middle English (as a noun): from Old French levee, feminine past participle of lever ‘raise’, from Latin levare, from levis ‘light’.}%
\par%
\entry{liable}{/ˈlʌɪəb(ə)l/}{দায়ী}{ \textsf{\textit{adjective}}\ \textbf{1} Responsible by law; legally answerable. {\fontspec{DejaVu Sans}◇} \textit{the credit{-}card company is liable for any breach of contract} \colorBulletS{SYN} responsible, legally responsible, accountable, answerable, chargeable, blameworthy, at fault, culpable, subject, guilty, faulty, censurable \textbf{2} Likely to do or to be something. {\fontspec{DejaVu Sans}◇} \textit{patients were liable to faint if they stood up too suddenly} \colorBulletS{SYN} likely, inclined, tending, disposed, apt, predisposed, prone, given}{}{}{ \colorBullet{ORIGIN} Late Middle English perhaps from Anglo{-}Norman French, from French lier ‘to bind’, from Latin ligare.}%
\par%
\entry{libido}{/lɪˈbiːdəʊ/}{কামশক্তি}{ \textsf{\textit{noun}}\ \textbf{1} Sexual desire. {\fontspec{DejaVu Sans}◇} \textit{loss of libido} \colorBulletS{SYN} sex drive, sexual appetite, sexual passion, sexual urge, sexual longing}{}{}{ \colorBullet{ORIGIN} Early 20th century from Latin, literally ‘desire, lust’.}%
\par%
\entry{lice}{/lʌɪs/}{উকুন}{\small{\textsf{\textit{}}}}{}{}{}%
\par%
\entry{lift}{/lɪft/}{উত্তোলন}{\small{\textsf{\textit{noun, verb}}} \\{\fontspec{DejaVu Sans}▪ }\textsf{\textit{noun}}\\ \textbf{1} A platform or compartment housed in a shaft for raising and lowering people or things to different levels. {\fontspec{DejaVu Sans}◇} \textit{Alice went up to the second floor in the lift} \colorBulletS{SYN} elevator, hoist \textbf{2} An act of lifting. {\fontspec{DejaVu Sans}◇} \textit{weightlifters attempting a particularly heavy lift} \colorBulletS{SYN} push, hoist, heave, thrust, shove, uplift, a helping hand \textbf{3} A free ride in another person's vehicle. {\fontspec{DejaVu Sans}◇} \textit{Miss Green is giving me a lift to school} \colorBulletS{SYN} car ride, ride, run, drive, transportation, journey \textbf{4} A feeling of confidence or cheerfulness. {\fontspec{DejaVu Sans}◇} \textit{winning this match has given everyone a lift} \colorBulletS{SYN} boost, fillip, pick{-}me{-}up, stimulus, impetus, encouragement, spur, reassurance, aid, help, push \\{\fontspec{DejaVu Sans}▪ }\textsf{\textit{verb}}\\ \textbf{1} Raise to a higher position or level. {\fontspec{DejaVu Sans}◇} \textit{he lifted his trophy over his head} \colorBulletS{SYN} raise, hoist, heave, haul up, uplift, heft, boost, raise aloft, raise up, upraise, elevate, thrust, hold high, bear aloft \textbf{2} Pick up and move to a different position. {\fontspec{DejaVu Sans}◇} \textit{he lifted her down from the pony's back} \colorBulletS{SYN} pick up, grab, scoop up, gather up, snatch up, swoop up \textbf{3} Raise (a person's spirits or confidence) {\fontspec{DejaVu Sans}◇} \textit{we heard inspiring talks which lifted our spirits} \colorBulletS{SYN} boost, raise, buoy up, elevate, give a lift to, cheer up, perk up, enliven, uplift, brighten up, lighten, ginger up, gladden, encourage, stimulate, arouse, revive, restore \textbf{4} Formally remove or end (a legal restriction, decision, or ban) {\fontspec{DejaVu Sans}◇} \textit{the European Community lifted its oil embargo against South Africa} \colorBulletS{SYN} cancel, raise, remove, withdraw, revoke, rescind, annul, void, discontinue, countermand, relax, end, stop, terminate \textbf{5} Carry off or win (a prize or event) {\fontspec{DejaVu Sans}◇} \textit{she staged a magnificent comeback to lift the British Open title}}{}{}{ \colorBullet{ORIGIN} Middle English from Old Norse lypta, of Germanic origin; related to loft.}%
\par%
\entry{light{-}headed}{}{1. লঘুচিত্ত 2 : mentally disoriented : dizzy 3 : lacking in maturity or seriousness : frivolous}{ \textsf{\textit{adjective}}\ \textbf{1} Dizzy and slightly faint. {\fontspec{DejaVu Sans}◇} \textit{she felt light{-}headed with relief} \colorBulletS{SYN} dizzy, giddy, faint, unsteady, light in the head, weak{-}headed, muzzy}{}{}{}%
\par%
\entry{liquor}{/ˈlɪkə/}{পানীয়}{\small{\textsf{\textit{noun, verb}}} \\{\fontspec{DejaVu Sans}▪ }\textsf{\textit{noun}}\\ \textbf{1} Alcoholic drink, especially distilled spirits. {\fontspec{DejaVu Sans}◇} \textit{} \colorBulletS{SYN} alcohol, spirits, alcoholic drink, strong drink, drink, intoxicating liquor, intoxicant \textbf{2} Liquid in which something has been steeped or cooked. {\fontspec{DejaVu Sans}◇} \textit{These had been slightly glazed with concentrated poaching liquor and dusted with what tasted like ground{-}down, caramelised peach crisps.} \\{\fontspec{DejaVu Sans}▪ }\textsf{\textit{verb}}\\ \textbf{1} Dress (leather) with grease or oil. {\fontspec{DejaVu Sans}◇} \textit{} \textbf{2} Steep (something, especially malt) in water. {\fontspec{DejaVu Sans}◇} \textit{}}{}{}{ \colorBullet{ORIGIN} Middle English (denoting liquid or something to drink): from Old French lic(o)ur, from Latin liquor; related to liquare ‘liquefy’, liquere ‘be fluid’.}%
\par%
\entry{livelihood}{/ˈlʌɪvlɪhʊd/}{জীবিকা}{ \textsf{\textit{noun}}\ \textbf{1} A means of securing the necessities of life. {\fontspec{DejaVu Sans}◇} \textit{people whose livelihoods depend on the rainforest} \colorBulletS{SYN} income, source of income, means of support, means, living, subsistence, keep, maintenance, sustenance, nourishment, daily bread, upkeep}{}{}{ \colorBullet{ORIGIN} Old English līflād ‘way of life’, from līf ‘life’ + lād ‘course’ (see lode). The change in the word's form in the 16th century was due to association with lively and {-}hood.}%
\par%
\entry{livestock}{/ˈlʌɪvstɒk/}{পশুসম্পত্তি}{ \textsf{\textit{noun}}\ \textbf{1} Farm animals regarded as an asset. {\fontspec{DejaVu Sans}◇} \textit{markets for the trading of livestock} \colorBulletS{SYN} livestock, farm animals, cattle, beasts}{}{}{}%
\par%
\entry{loathe}{/ləʊð/}{অতিশয় অপছন্দ করা}{ \textsf{\textit{verb}}\ \textbf{1} Feel intense dislike or disgust for. {\fontspec{DejaVu Sans}◇} \textit{she loathed him on sight} \colorBulletS{SYN} hate, detest, abhor, despise, abominate, dislike greatly, execrate}{}{}{ \colorBullet{ORIGIN} Old English lāthian, of Germanic origin; related to loath.}%
\par%
\entry{lob}{/lɒb/}{ডেলা}{\small{\textsf{\textit{noun, verb}}} \\{\fontspec{DejaVu Sans}▪ }\textsf{\textit{noun}}\\ \textbf{1} (in sport) a ball lobbed over an opponent or a stroke producing this result. {\fontspec{DejaVu Sans}◇} \textit{Federer played a lob and Nadal's high volley was in the net} \colorBulletS{SYN} stroke, hit, strike \\{\fontspec{DejaVu Sans}▪ }\textsf{\textit{verb}}\\ \textbf{1} Throw or hit (a ball or missile) in a high arc. {\fontspec{DejaVu Sans}◇} \textit{he lobbed the ball over their heads} \colorBulletS{SYN} throw, toss, fling, pitch, shy, hurl, pelt, sling, loft, cast, let fly with, flip}{}{}{ \colorBullet{ORIGIN} Late 16th century (in the senses ‘cause or allow to hang heavily’ and ‘behave like a lout’): from the archaic noun lob ‘lout’, ‘pendulous object’, probably from Low German or Dutch (compare with modern Dutch lubbe ‘hanging lip’). The current sense dates from the mid 19th century.}%
\par%
\entry{lobby}{/ˈlɒbi/}{লবি}{\small{\textsf{\textit{noun, verb}}} \\{\fontspec{DejaVu Sans}▪ }\textsf{\textit{noun}}\\ \textbf{1} A room providing a space out of which one or more other rooms or corridors lead, typically one near the entrance of a public building. {\fontspec{DejaVu Sans}◇} \textit{they went into the hotel lobby} \colorBulletS{SYN} entrance hall, hallway, hall, entrance, vestibule, foyer, reception area, outer room, waiting room, anteroom, antechamber, porch \textbf{2} (in the UK) any of several large halls in the Houses of Parliament in which MPs may meet members of the public. {\fontspec{DejaVu Sans}◇} \textit{} \textbf{3} A group of people seeking to influence legislators on a particular issue. {\fontspec{DejaVu Sans}◇} \textit{members of the anti{-}abortion lobby} \colorBulletS{SYN} pressure group, interest group, interest, movement, campaign, crusade, lobbyists, supporters \\{\fontspec{DejaVu Sans}▪ }\textsf{\textit{verb}}\\ \textbf{1} Seek to influence (a legislator) on an issue. {\fontspec{DejaVu Sans}◇} \textit{they insist on their right to lobby Congress} \colorBulletS{SYN} seek to influence, try to persuade, bring pressure to bear on, importune, persuade, influence, sway}{}{}{ \colorBullet{ORIGIN} Mid 16th century (in the sense ‘monastic cloister’): from medieval Latin lobia, lobium ‘covered walk, portico’. The verb sense (originally US) derives from the practice of frequenting the lobby of a house of legislature to influence its members into supporting a cause.}%
\par%
\entry{lobster}{/ˈlɒbstə/}{গলদা চিংড়ি}{\small{\textsf{\textit{noun, verb}}} \\{\fontspec{DejaVu Sans}▪ }\textsf{\textit{noun}}\\ \textbf{1} A large marine crustacean with a cylindrical body, stalked eyes, and the first of its five pairs of limbs modified as pincers. {\fontspec{DejaVu Sans}◇} \textit{} \\{\fontspec{DejaVu Sans}▪ }\textsf{\textit{verb}}\\ \textbf{1} Catch lobsters. {\fontspec{DejaVu Sans}◇} \textit{he has been lobstering in Maine for 50 years}}{}{}{ \colorBullet{ORIGIN} Old English lopustre, alteration of Latin locusta ‘crustacean, locust’.}%
\par%
\entry{lodged}{/lɒdʒd/}{দায়ের}{ \textsf{\textit{adjective}}\ \textbf{1} (of a crop) flattened by wind or rain. {\fontspec{DejaVu Sans}◇} \textit{in lodged crops there is rapid leaf decay}}{}{}{}%
\par%
\entry{lofty}{/ˈlɒfti/}{অহংকারী}{ \textsf{\textit{adjective}}\ \textbf{1} Of imposing height. {\fontspec{DejaVu Sans}◇} \textit{the elegant square was shaded by lofty palms} \colorBulletS{SYN} tall, high, giant, towering, soaring, sky{-}high, sky{-}scraping \textbf{2} (of wool and other textiles) thick and resilient. {\fontspec{DejaVu Sans}◇} \textit{Because fleece is such a lofty, stretchy fabric, use a 3 mm or 3.5 mm stitch length.}}{}{}{ \colorBullet{ORIGIN} Middle English from loft, influenced by aloft.}%
\par%
\entry{loggerhead}{/ˈlɒɡəhɛd/}{বিবদমান}{ \textsf{\textit{noun}}\ \textbf{1}  {\fontspec{DejaVu Sans}◇} \textit{} \textbf{2}  {\fontspec{DejaVu Sans}◇} \textit{} \textbf{3} A foolish person. {\fontspec{DejaVu Sans}◇} \textit{} \colorBulletS{SYN} idiot, ass, halfwit, nincompoop, blockhead, buffoon, dunce, dolt, ignoramus, cretin, imbecile, dullard, moron, simpleton, clod}{}{Two groups of recruiting agencies are at loggerheads over the saudi embassy move to start visa service centres in dhaka under two leading recruiting agents.}{ \colorBullet{ORIGIN} Late 16th century (in loggerhead (sense 3)): from dialect logger ‘block of wood for hobbling a horse’ + head.}%
\par%
\entry{logistics}{/ləˈdʒɪstɪks/}{}{ \textsf{\textit{plural noun}}\ \textbf{1} The detailed organization and implementation of a complex operation. {\fontspec{DejaVu Sans}◇} \textit{the logistics of a large{-}scale rock show demand certain necessities} \colorBulletS{SYN} organization, planning, plans, management, arrangement, administration, masterminding, direction, orchestration, regimentation, engineering, coordination, execution, handling, running}{}{}{ \colorBullet{ORIGIN} Late 19th century from French logistique, from loger ‘to lodge’.}%
\par%
\entry{loin}{/lɔɪn/}{কোমর}{ \textsf{\textit{noun}}\ \textbf{1} The part of the body on both sides of the spine between the lowest (false) ribs and the hip bones. {\fontspec{DejaVu Sans}◇} \textit{}}{}{}{ \colorBullet{ORIGIN} Middle English from Old French loigne, based on Latin lumbus.}%
\par%
\entry{loiter}{/ˈlɔɪtə/}{ঘুরাফেরা করিতে}{ \textsf{\textit{verb}}\ \textbf{1} Stand or wait around without apparent purpose. {\fontspec{DejaVu Sans}◇} \textit{she saw Mary loitering near the cloakrooms} \colorBulletS{SYN} linger, potter, wait, skulk}{}{}{ \colorBullet{ORIGIN} Late Middle English perhaps from Middle Dutch loteren ‘wag about’.}%
\par%
\entry{lone}{/ləʊn/}{নির্জন}{ \textsf{\textit{adjective}}\ \textbf{1} Having no companions; solitary or single. {\fontspec{DejaVu Sans}◇} \textit{I approached a lone drinker across the bar} \colorBulletS{SYN} solitary, single, solo, unaccompanied, unescorted, alone, all alone, by itself, by oneself, sole, without companions, companionless \textbf{2} (of a place) unfrequented and remote. {\fontspec{DejaVu Sans}◇} \textit{houses in lone rural settings} \colorBulletS{SYN} deserted, uninhabited, unfrequented, lonely, unpopulated, desolate, barren, isolated, remote, marooned, out of the way, secluded, sequestered, off the beaten track, in the back of beyond, in the middle of nowhere, godforsaken}{}{}{ \colorBullet{ORIGIN} Late Middle English shortening of alone.}%
\par%
\entry{long{-}drawn}{}{অযথা প্রলম্বিত}{ \textsf{\textit{adjective}}\ \textbf{1} Continuing for a long time, especially for longer than is necessary. {\fontspec{DejaVu Sans}◇} \textit{long{-}drawn{-}out negotiations} \colorBulletS{SYN} prolonged, protracted, lengthy, lasting, long{-}lasting, marathon, overlong, extended, drawn{-}out, spun{-}out, dragged{-}out, dragging, time{-}consuming, seemingly endless, lingering, interminable}{ \colorBullet{OTHER} long{-}drawn{-}out}{}{}%
\par%
\entry{loo}{/luː/}{পায়খানা}{ \textsf{\textit{noun}}\ \textbf{1} A toilet. {\fontspec{DejaVu Sans}◇} \textit{loo paper} \colorBulletS{SYN} lavatory, WC, water closet, convenience, public convenience, facilities, urinal, privy, latrine, outhouse, earth closet, jakes}{}{}{ \colorBullet{ORIGIN} 1940s many theories have been put forward about the word's origin: one suggests the source is Waterloo, a trade name for iron cisterns in the early part of the century; the evidence remains inconclusive.}%
\par%
\entry{loo}{/luː/}{পায়খানা}{ \textsf{\textit{noun}}\ \textbf{1} A gambling card game, popular from the 17th to the 19th centuries, in which a player who fails to win a trick must pay a sum to a pool. {\fontspec{DejaVu Sans}◇} \textit{}}{}{}{ \colorBullet{ORIGIN} Late 17th century abbreviation of obsolete lanterloo from French lanturlu, a meaningless song refrain.}%
\par%
\entry{loom}{/luːm/}{তাঁত}{ \textsf{\textit{noun}}\ \textbf{1} An apparatus for making fabric by weaving yarn or thread. {\fontspec{DejaVu Sans}◇} \textit{}}{}{}{ \colorBullet{ORIGIN} Old English gelōma ‘tool’, shortened to lome in Middle English.}%
\par%
\entry{loom}{/luːm/}{তাঁত}{\small{\textsf{\textit{noun, verb}}} \\{\fontspec{DejaVu Sans}▪ }\textsf{\textit{noun}}\\ \textbf{1} A vague and often exaggerated first appearance of an object seen in darkness or fog, especially at sea. {\fontspec{DejaVu Sans}◇} \textit{the loom of the land} \\{\fontspec{DejaVu Sans}▪ }\textsf{\textit{verb}}\\ \textbf{1} Appear as a vague form, especially one that is large or threatening. {\fontspec{DejaVu Sans}◇} \textit{vehicles loomed out of the darkness} \colorBulletS{SYN} emerge, appear, become visible, come into view, take shape, materialize, reveal itself, appear indistinctly, come to light, take on a threatening shape}{}{}{ \colorBullet{ORIGIN} Mid 16th century probably from Low German or Dutch; compare with East Frisian lōmen ‘move slowly’, Middle High German lüemen ‘be weary’.}%
\par%
\entry{lube}{/luːb/}{পিচ্ছিলকারক পদার্থ}{\small{\textsf{\textit{noun, verb}}} \\{\fontspec{DejaVu Sans}▪ }\textsf{\textit{noun}}\\ \textbf{1} A lubricant. {\fontspec{DejaVu Sans}◇} \textit{a wide variety of lubes and waxes} \colorBulletS{SYN} lubricant, lubrication, grease \\{\fontspec{DejaVu Sans}▪ }\textsf{\textit{verb}}\\ \textbf{1} Lubricate (something) {\fontspec{DejaVu Sans}◇} \textit{lube the hinge with some oil} \colorBulletS{SYN} lubricate, grease}{}{}{ \colorBullet{ORIGIN} 1930s abbreviation.}%
\par%
\entry{lucky duck}{}{An incredibly lucky person; one who falls into good fortune.}{\small{\textsf{\textit{}}}}{}{A: "I won another bet in the basketball tournament—that's three in a row now!" B: "Wow, you lucky duck!"}{}%
\par%
\entry{lucrative}{/ˈluːkrətɪv/}{লাভজনক}{ \textsf{\textit{adjective}}\ \textbf{1} Producing a great deal of profit. {\fontspec{DejaVu Sans}◇} \textit{a lucrative career as a stand{-}up comedian} \colorBulletS{SYN} profitable, profit{-}making, gainful, remunerative, moneymaking, paying, high{-}income, well paid, high{-}paying, bankable, cost{-}effective}{}{}{ \colorBullet{ORIGIN} Late Middle English from Latin lucrativus, from lucrat{-} ‘gained’, from the verb lucrari, from lucrum (see lucre).}%
\par%
\entry{lump sum}{}{একটি একক সমষ্টিগত অর্থ}{ \textsf{\textit{noun}}\ \textbf{1} A single payment made at a particular time, as opposed to a number of smaller payments or instalments. {\fontspec{DejaVu Sans}◇} \textit{your pension plan can provide a cash lump sum at retirement as well as a regular income}}{}{}{}%
\par%
\entry{lurch}{/ləːtʃ/}{সহসা জাহাজের কাৎ}{\small{\textsf{\textit{noun, verb}}} \\{\fontspec{DejaVu Sans}▪ }\textsf{\textit{noun}}\\ \textbf{1} An abrupt uncontrolled movement, especially an unsteady tilt or roll. {\fontspec{DejaVu Sans}◇} \textit{the boat gave a violent lurch and he missed his footing} \\{\fontspec{DejaVu Sans}▪ }\textsf{\textit{verb}}\\ \textbf{1} Make an abrupt, unsteady, uncontrolled movement or series of movements; stagger. {\fontspec{DejaVu Sans}◇} \textit{the car lurched forward} \colorBulletS{SYN} stagger, stumble, sway, reel, roll, weave, totter, flounder, falter, wobble, slip, move clumsily}{}{}{ \colorBullet{ORIGIN} Late 17th century (as a noun denoting the sudden leaning of a ship to one side): of unknown origin.}%
\par%
\entry{lurch}{/ləːtʃ/}{সহসা জাহাজের কাৎ}{ \textsf{\textit{noun}}\ \textbf{1} Leave an associate or friend abruptly and without assistance or support when they are in a difficult situation. {\fontspec{DejaVu Sans}◇} \textit{he left you in the lurch when you needed him most} \colorBulletS{SYN} leave in trouble, let down, leave helpless, leave stranded, leave high and dry, abandon, desert, betray}{}{}{ \colorBullet{ORIGIN} Mid 16th century (denoting a state of discomfiture): from French lourche, the name of a game resembling backgammon, used in the phrase demeurer lourche ‘be discomfited’.}%
\par%
\entry{lynch}{/lɪn(t)ʃ/}{}{ \textsf{\textit{verb}}\ \textbf{1} (of a group of people) kill (someone) for an alleged offence without a legal trial, especially by hanging. {\fontspec{DejaVu Sans}◇} \textit{her father had been lynched for a crime he didn't commit} \colorBulletS{SYN} hang, hang by the neck}{}{It is abhorrent and disguting to see people falling victims of public lynching in several parts of the country over a period of several days.}{ \colorBullet{ORIGIN} Mid 19th century from Lynch's law, named after Capt. William Lynch, head of a self{-}constituted judicial tribunal in Virginia c1780.}%
\par%
\end{multicols}%
\pagebreak%
\section*{M}%
\begin{multicols}{2}%
\entry{macaque}{/məˈkɑːk/}{একজাতের ছোটো লেজওয়ালা বাঁদর}{ \textsf{\textit{noun}}\ \textbf{1} A medium{-}sized, chiefly forest{-}dwelling Old World monkey which has a long face and cheek pouches for holding food. {\fontspec{DejaVu Sans}◇} \textit{}}{}{}{ \colorBullet{ORIGIN} Late 17th century via French and Portuguese; based on the Bantu morpheme ma (denoting a plural) + kaku ‘monkey’.}%
\par%
\entry{machete}{/məˈtʃɛti/}{চাপাতি}{ \textsf{\textit{noun}}\ \textbf{1} A broad, heavy knife used as an implement or weapon, originating in Central America and the Caribbean. {\fontspec{DejaVu Sans}◇} \textit{}}{}{}{ \colorBullet{ORIGIN} Late 16th century from Spanish, from macho ‘hammer’.}%
\par%
\entry{macho}{/ˈmatʃəʊ/}{পৌরুষপূর্ণ ব্যক্তি}{\small{\textsf{\textit{adjective, noun}}} \\{\fontspec{DejaVu Sans}▪ }\textsf{\textit{adjective}}\\ \textbf{1} Masculine in an overly assertive or aggressive way. {\fontspec{DejaVu Sans}◇} \textit{the big macho tough guy} \colorBulletS{SYN} male, aggressively male, masculine, unpleasantly masculine \\{\fontspec{DejaVu Sans}▪ }\textsf{\textit{noun}}\\ \textbf{1} A man who is aggressively proud of his masculinity. {\fontspec{DejaVu Sans}◇} \textit{I realized just what a macho I was at heart} \colorBulletS{SYN} red{-}blooded male, macho man, muscleman}{}{}{ \colorBullet{ORIGIN} 1920s from Mexican Spanish, ‘masculine or vigorous’.}%
\par%
\entry{MACHO}{/ˈmatʃəʊ/}{পৌরুষপূর্ণ ব্যক্তি}{ \textsf{\textit{noun}}\ \textbf{1} A relatively dark, dense object, such as a brown dwarf, a low{-}mass star, or a black hole, of a kind believed to occur in a halo around a galaxy and to contain a significant proportion of the galaxy's mass. {\fontspec{DejaVu Sans}◇} \textit{}}{}{}{ \colorBullet{ORIGIN} 1990s acronym from Massive (Astrophysical) Compact Halo Object.}%
\par%
\entry{magistrate}{/ˈmadʒɪstrət/}{হাকিম}{ \textsf{\textit{noun}}\ \textbf{1} A civil officer who administers the law, especially one who conducts a court that deals with minor offences and holds preliminary hearings for more serious ones. {\fontspec{DejaVu Sans}◇} \textit{} \colorBulletS{SYN} judge, magistrate, Her Honour, His Honour, Your Honour}{}{}{ \colorBullet{ORIGIN} Late Middle English from Latin magistratus ‘administrator’, from magister ‘master’.}%
\par%
\entry{magnetite}{/ˈmaɡnɪtʌɪt/}{ম্যাগনেটাইট}{ \textsf{\textit{noun}}\ \textbf{1} A grey{-}black magnetic mineral which consists of an oxide of iron and is an important form of iron ore. {\fontspec{DejaVu Sans}◇} \textit{} \colorBulletS{SYN} lodestone, magnetite}{}{}{ \colorBullet{ORIGIN} Mid 19th century from magnet+ {-}ite.}%
\par%
\entry{maim}{/meɪm/}{পঙ্গু করা}{ \textsf{\textit{verb}}\ \textbf{1} Wound or injure (a person or animal) so that part of the body is permanently damaged. {\fontspec{DejaVu Sans}◇} \textit{100,000 soldiers were killed or maimed} \colorBulletS{SYN} injure, wound, hurt, disable, put out of action, incapacitate, impair, mar, mutilate, lacerate, disfigure, deform, mangle}{}{}{ \colorBullet{ORIGIN} Middle English from Old French mahaignier, of unknown origin.}%
\par%
\entry{makeover}{/ˈmeɪkəʊvə/}{পরিবর্তন}{ \textsf{\textit{noun}}\ \textbf{1} A complete transformation of the appearance of someone or something. {\fontspec{DejaVu Sans}◇} \textit{win one of our special pampering makeovers} \colorBulletS{SYN} improvement, betterment, amelioration, refinement, rectification, correction, rehabilitation}{}{}{}%
\par%
\entry{makeshift}{/ˈmeɪkʃɪft/}{অস্থায়ী}{\small{\textsf{\textit{adjective, noun}}} \\{\fontspec{DejaVu Sans}▪ }\textsf{\textit{adjective}}\\ \textbf{1} Acting as an interim and temporary measure. {\fontspec{DejaVu Sans}◇} \textit{arranging a row of chairs to form a makeshift bed} \colorBulletS{SYN} temporary, make{-}do, provisional, stopgap, standby, rough and ready, substitute, emergency, improvised, ad hoc, impromptu, extemporary, extempore, thrown together, cobbled together \\{\fontspec{DejaVu Sans}▪ }\textsf{\textit{noun}}\\ \textbf{1} A temporary substitute or device. {\fontspec{DejaVu Sans}◇} \textit{}}{}{}{}%
\par%
\entry{malice}{/ˈmalɪs/}{আক্রোশ}{ \textsf{\textit{noun}}\ \textbf{1} The desire to harm someone; ill will. {\fontspec{DejaVu Sans}◇} \textit{I bear no malice towards anybody} \colorBulletS{SYN} spitefulness, spite, malevolence, maliciousness, animosity, hostility, ill will, ill feeling, hatred, hate, bitterness, venom, vindictiveness, vengefulness, revenge, malignity, malignance, evil intentions, animus, enmity, devilment, devilry, bad blood, backbiting, gall, rancour, spleen, grudge}{}{}{ \colorBullet{ORIGIN} Middle English via Old French from Latin malitia, from malus ‘bad’.}%
\par%
\entry{malign}{/məˈlʌɪn/}{অপবাদ}{\small{\textsf{\textit{adjective, verb}}} \\{\fontspec{DejaVu Sans}▪ }\textsf{\textit{adjective}}\\ \textbf{1} Evil in nature or effect. {\fontspec{DejaVu Sans}◇} \textit{she had a strong and malign influence} \colorBulletS{SYN} harmful, evil, bad, baleful, hostile, inimical, destructive, malevolent, evil{-}intentioned, malignant, injurious, spiteful, malicious, vicious \\{\fontspec{DejaVu Sans}▪ }\textsf{\textit{verb}}\\ \textbf{1} Speak about (someone) in a spitefully critical manner. {\fontspec{DejaVu Sans}◇} \textit{don't you dare malign her in my presence} \colorBulletS{SYN} defame, slander, libel, blacken someone's character, blacken someone's name, smear, run a smear campaign against, vilify, speak ill of, spread lies about, accuse falsely, cast aspersions on, run down, misrepresent, calumniate, traduce, denigrate, disparage, slur, derogate, abuse, revile}{}{}{ \colorBullet{ORIGIN} Middle English via Old French maligne (adjective), malignier (verb), based on Latin malignus ‘tending to evil’, from malus ‘bad’.}%
\par%
\entry{mangle}{/ˈmaŋɡ(ə)l/}{ম্যাঙ্গলেড}{ \textsf{\textit{verb}}\ \textbf{1} Destroy or severely damage by tearing or crushing. {\fontspec{DejaVu Sans}◇} \textit{the car was mangled almost beyond recognition} \colorBulletS{SYN} mutilate, maim, disfigure, damage, injure, crush, crumple}{}{}{ \colorBullet{ORIGIN} Late Middle English from Anglo{-}Norman French mahangler, perhaps a frequentative of mahaignier ‘maim’.}%
\par%
\entry{mangle}{/ˈmaŋɡ(ə)l/}{ম্যাঙ্গলেড}{\small{\textsf{\textit{noun, verb}}} \\{\fontspec{DejaVu Sans}▪ }\textsf{\textit{noun}}\\ \textbf{1} A machine having two or more rollers turned by a handle, between which wet laundry is squeezed to remove excess moisture. {\fontspec{DejaVu Sans}◇} \textit{} \\{\fontspec{DejaVu Sans}▪ }\textsf{\textit{verb}}\\ \textbf{1} Press or squeeze with a mangle. {\fontspec{DejaVu Sans}◇} \textit{the hard household labour often involved pounding clothes in a dolly tub and mangling them with a hand wringer}}{}{}{ \colorBullet{ORIGIN} Late 17th century from Dutch mangel, from mangelen ‘to mangle’, from medieval Latin mango, manga, from Greek manganon ‘axis, engine of war’.}%
\par%
\entry{mannequin}{/ˈmanɪkɪn/}{মানবমূর্তি}{ \textsf{\textit{noun}}\ \textbf{1} A dummy used to display clothes in a shop window. {\fontspec{DejaVu Sans}◇} \textit{} \colorBulletS{SYN} dummy, model, figure}{}{}{ \colorBullet{ORIGIN} Mid 18th century from French (see manikin).}%
\par%
\entry{mare}{/mɛː/}{ঘোটকী}{ \textsf{\textit{noun}}\ \textbf{1} The female of a horse or other equine animal. {\fontspec{DejaVu Sans}◇} \textit{}}{}{}{ \colorBullet{ORIGIN} Old English mearh ‘horse’, mere ‘mare’, from a Germanic base with cognates in Celtic languages meaning ‘stallion’.}%
\par%
\entry{mare}{/mɛː/}{ঘোটকী}{ \textsf{\textit{noun}}\ \textbf{1} A very unpleasant or frustrating experience. {\fontspec{DejaVu Sans}◇} \textit{this week is going to be a bit of a mare but at least the end is in sight} \colorBulletS{SYN} ordeal, horror, torment, trial}{}{}{ \colorBullet{ORIGIN} 1990s abbreviation of nightmare.}%
\par%
\entry{mare}{/ˈmɑːreɪ/}{ঘোটকী}{ \textsf{\textit{noun}}\ \textbf{1} A large, level basalt plain on the surface of the moon, appearing dark by contrast with highland areas. {\fontspec{DejaVu Sans}◇} \textit{the maria are largely confined to the near side of the moon}}{}{}{ \colorBullet{ORIGIN} Mid 19th century special use of Latin mare ‘sea’; these areas were once thought to be seas.}%
\par%
\entry{marijuana}{/ˌmarɪˈhwɑːnə/}{গাঁজা}{ \textsf{\textit{noun}}\ \textbf{1} Cannabis, especially as smoked or consumed as a psychoactive (mind{-}altering) drug. {\fontspec{DejaVu Sans}◇} \textit{the cops told us that he had been smoking marijuana} \colorBulletS{SYN} cannabis, hashish, bhang, hemp, kef, kif, charas, ganja, sinsemilla}{}{}{ \colorBullet{ORIGIN} Late 19th century from Latin American Spanish.}%
\par%
\entry{maroon}{/məˈruːn/}{পানিবন্দি}{\small{\textsf{\textit{adjective, noun}}} \\{\fontspec{DejaVu Sans}▪ }\textsf{\textit{adjective}}\\ \textbf{1} Of a brownish{-}red colour. {\fontspec{DejaVu Sans}◇} \textit{ornate maroon and gold wallpaper} \\{\fontspec{DejaVu Sans}▪ }\textsf{\textit{noun}}\\ \textbf{1} A brownish{-}red colour. {\fontspec{DejaVu Sans}◇} \textit{the hat is available in either white or maroon} \textbf{2} A firework that makes a loud bang, used as a signal or warning. {\fontspec{DejaVu Sans}◇} \textit{}}{}{}{ \colorBullet{ORIGIN} Late 17th century (in the sense ‘chestnut’): from French marron ‘chestnut’, via Italian from medieval Greek maraon. The sense relating to colour dates from the late 18th century.}%
\par%
\entry{maroon}{/məˈruːn/}{পানিবন্দি}{ \textsf{\textit{verb}}\ \textbf{1} Leave (someone) trapped and alone in an inaccessible place, especially an island. {\fontspec{DejaVu Sans}◇} \textit{a novel about schoolboys marooned on a desert island} \colorBulletS{SYN} strand, leave stranded, cast away, cast ashore, abandon, leave behind, leave, leave in the lurch, desert, turn one's back on, leave isolated}{}{}{ \colorBullet{ORIGIN} Early 18th century from Maroon, originally in the form marooned ‘lost in the wilds’.}%
\par%
\entry{Maroon}{/məˈruːn/}{পানিবন্দি}{ \textsf{\textit{noun}}\ \textbf{1} A member of any of various communities in parts of the Caribbean who were originally descended from escaped slaves. In the 18th century Jamaican Maroons fought two wars against the British, both of which ended with treaties affirming the independence of the Maroons. {\fontspec{DejaVu Sans}◇} \textit{}}{}{}{ \colorBullet{ORIGIN} Mid 17th century from French marron ‘feral’, from Spanish cimarrón ‘wild’, (as a noun) ‘runaway slave’.}%
\par%
\entry{maternity}{/məˈtəːnɪti/}{মাতৃত্ব}{ \textsf{\textit{noun}}\ \textbf{1} Motherhood. {\fontspec{DejaVu Sans}◇} \textit{she is not a woman with an interest in maternity} \colorBulletS{SYN} motherhood, parenthood}{}{}{ \colorBullet{ORIGIN} Early 17th century from French maternité, from Latin maternus, from mater ‘mother’.}%
\par%
\entry{mean}{/miːn/}{}{ \textsf{\textit{verb}}\ \textbf{1} Intend to convey or refer to (a particular thing); signify. {\fontspec{DejaVu Sans}◇} \textit{I don't know what you mean} \colorBulletS{SYN} signify, convey, denote, designate, indicate, connote, show, express, spell out, stand for, represent, symbolize, imply, purport, suggest, allude to, intimate, hint at, insinuate, drive at, refer to \textbf{2} Intend (something) to occur or be the case. {\fontspec{DejaVu Sans}◇} \textit{they mean no harm} \colorBulletS{SYN} intend, aim, plan, design, have in mind, have in view, contemplate, think of, purpose, propose, have plans, set out, aspire, desire, want, wish, expect \textbf{3} Have as a consequence or result. {\fontspec{DejaVu Sans}◇} \textit{the proposals are likely to mean another hundred closures} \colorBulletS{SYN} entail, involve, necessitate, lead to, result in, give rise to, bring about, cause, engender, produce, effect}{ \colorBullet{OTHER} by all means: যে কোন উপায়ে }{}{ \colorBullet{ORIGIN} Old English mænan, of West Germanic origin; related to Dutch meenen and German meinen, from an Indo{-}European root shared by mind.}%
\par%
\entry{mean}{/miːn/}{}{ \textsf{\textit{adjective}}\ \textbf{1} Unwilling to give or share things, especially money; not generous. {\fontspec{DejaVu Sans}◇} \textit{she felt mean not giving a tip} \colorBulletS{SYN} miserly, niggardly, close{-}fisted, parsimonious, penny{-}pinching, cheese{-}paring, ungenerous, penurious, illiberal, close, grasping, greedy, avaricious, acquisitive, Scrooge{-}like \textbf{2} Unkind, spiteful, or unfair. {\fontspec{DejaVu Sans}◇} \textit{I was mean to them over the festive season} \colorBulletS{SYN} unkind, nasty, spiteful, foul, malicious, malevolent, despicable, contemptible, obnoxious, vile, odious, loathsome, disagreeable, unpleasant, unfriendly, uncharitable, shabby, unfair, callous, cruel, vicious, base, low \textbf{3} (especially of a place) poor in quality and appearance; shabby. {\fontspec{DejaVu Sans}◇} \textit{her home was mean and small} \colorBulletS{SYN} squalid, shabby, dilapidated, sordid, seedy, slummy, sleazy, insalubrious, poor, sorry, wretched, dismal, dingy, miserable, mangy, broken{-}down, run down, down at heel \textbf{4} Very skilful or effective; excellent. {\fontspec{DejaVu Sans}◇} \textit{he's a mean cook} \colorBulletS{SYN} excellent, marvellous, magnificent, superb, fine, wonderful, outstanding, exceptional, formidable, first{-}class, first{-}rate, virtuoso, skilful, masterful, masterly}{ \colorBullet{OTHER} by all means: যে কোন উপায়ে }{}{ \colorBullet{ORIGIN} Middle English, shortening of Old English gemǣne, of Germanic origin, from an Indo{-}European root shared by Latin communis ‘common’. The original sense was ‘common to two or more people’, later ‘inferior in rank’, leading to mean (sense 3) and a sense ‘ignoble, small{-}minded’, from which mean (sense 1 and mean sense 2) (which became common in the 19th century) arose.}%
\par%
\entry{mean}{/miːn/}{}{\small{\textsf{\textit{adjective, noun}}} \\{\fontspec{DejaVu Sans}▪ }\textsf{\textit{adjective}}\\ \textbf{1} (of a quantity) calculated as a mean; average. {\fontspec{DejaVu Sans}◇} \textit{participants in the study had a mean age of 35 years} \colorBulletS{SYN} average, median, middle, halfway, centre, central, intermediate, medial, medium, normal, standard, middling \textbf{2} Equally far from two extremes. {\fontspec{DejaVu Sans}◇} \textit{hope is the mean virtue between despair and presumption} \\{\fontspec{DejaVu Sans}▪ }\textsf{\textit{noun}}\\ \textbf{1} The value obtained by dividing the sum of several quantities by their number; an average. {\fontspec{DejaVu Sans}◇} \textit{acid output was calculated by taking the mean of all three samples} \textbf{2} A condition, quality, or course of action equally removed from two opposite extremes. {\fontspec{DejaVu Sans}◇} \textit{the measure expresses a mean between saving and splashing out} \colorBulletS{SYN} middle course, middle way, mid point, central point, middle, happy medium, golden mean, compromise, balance, median, norm, average}{ \colorBullet{OTHER} by all means: যে কোন উপায়ে }{}{ \colorBullet{ORIGIN} Middle English from Old French meien, from Latin medianus ‘middle’ (see median).}%
\par%
\entry{meant}{/mɛnt/}{অভিপ্রেত}{\small{\textsf{\textit{}}}}{}{}{}%
\par%
\entry{measure}{/ˈmɛʒə/}{পরিমাপ}{\small{\textsf{\textit{noun, verb}}} \\{\fontspec{DejaVu Sans}▪ }\textsf{\textit{noun}}\\ \textbf{1} A plan or course of action taken to achieve a particular purpose. {\fontspec{DejaVu Sans}◇} \textit{cost{-}cutting measures} \colorBulletS{SYN} action, act, course, course of action, deed, proceeding, procedure, step, means, expedient \textbf{2} A standard unit used to express the size, amount, or degree of something. {\fontspec{DejaVu Sans}◇} \textit{a furlong is an obsolete measure of length} \colorBulletS{SYN} system, standard, units, scale \textbf{3} A certain quantity or degree of something. {\fontspec{DejaVu Sans}◇} \textit{the states retain a large measure of independence} \colorBulletS{SYN} certain amount, amount, degree, quantity \textbf{4} The rhythm of a piece of poetry or a piece of music. {\fontspec{DejaVu Sans}◇} \textit{The golden measure of poetry does not yet exist, only the rhythm of the maracas, the exact sound of the kettledrum.} \colorBulletS{SYN} metre, cadence, rhythm, foot \textbf{5} A group of rock strata. {\fontspec{DejaVu Sans}◇} \textit{} \\{\fontspec{DejaVu Sans}▪ }\textsf{\textit{verb}}\\ \textbf{1} Ascertain the size, amount, or degree of (something) by using an instrument or device marked in standard units. {\fontspec{DejaVu Sans}◇} \textit{the amount of water collected is measured in pints} \colorBulletS{SYN} take the measurements of, calculate, compute, estimate, count, meter, quantify, weigh, size, evaluate, rate, assess, appraise, gauge, plumb, measure out, determine, judge, survey \textbf{2} Assess the importance, effect, or value of (something) {\fontspec{DejaVu Sans}◇} \textit{it is hard to measure teaching ability} \colorBulletS{SYN} choose carefully, select with care, consider, think carefully about, plan, calculate \textbf{3} Travel over (a certain distance or area) {\fontspec{DejaVu Sans}◇} \textit{we must measure twenty miles today}}{}{}{ \colorBullet{ORIGIN} Middle English (as a noun in the senses ‘moderation’, ‘instrument for measuring’, ‘unit of capacity’): from Old French mesure, from Latin mensura, from mens{-} ‘measured’, from the verb metiri.}%
\par%
\entry{mediate}{/ˈmiːdɪeɪt/}{মধ্যস্থতার}{\small{\textsf{\textit{adjective, verb}}} \\{\fontspec{DejaVu Sans}▪ }\textsf{\textit{adjective}}\\ \textbf{1} Connected indirectly through another person or thing; involving an intermediate agency. {\fontspec{DejaVu Sans}◇} \textit{public law institutions are a type of mediate state administration} \\{\fontspec{DejaVu Sans}▪ }\textsf{\textit{verb}}\\ \textbf{1} Intervene in a dispute in order to bring about an agreement or reconciliation. {\fontspec{DejaVu Sans}◇} \textit{Wilson attempted to mediate between the powers to end the war} \colorBulletS{SYN} arbitrate, conciliate, moderate, umpire, referee, act as peacemaker, reconcile differences, restore harmony, make peace, bring to terms, liaise \textbf{2} Bring about (a result such as a physiological effect) {\fontspec{DejaVu Sans}◇} \textit{the right hemisphere plays an important role in mediating tactile perception of direction} \colorBulletS{SYN} arbitrate, conciliate, moderate, umpire, referee, act as peacemaker, reconcile differences, restore harmony, make peace, bring to terms, liaise}{}{}{ \colorBullet{ORIGIN} Late Middle English (as an adjective in the sense ‘interposed’): from late Latin mediatus ‘placed in the middle’, past participle of the verb mediare, from Latin medius ‘middle’.}%
\par%
\entry{mediterranean}{/ˌmɛdɪtəˈreɪnɪən/}{ভূমধ্য}{\small{\textsf{\textit{adjective, noun}}} \\{\fontspec{DejaVu Sans}▪ }\textsf{\textit{adjective}}\\ \textbf{1} Of or characteristic of the Mediterranean Sea, the countries bordering it, or their inhabitants. {\fontspec{DejaVu Sans}◇} \textit{a leisurely Mediterranean cruise} \\{\fontspec{DejaVu Sans}▪ }\textsf{\textit{noun}}\\ \textbf{1} The Mediterranean Sea or the countries bordering it. {\fontspec{DejaVu Sans}◇} \textit{a permanent American naval presence in the Mediterranean} \textbf{2} A native of a Mediterranean country. {\fontspec{DejaVu Sans}◇} \textit{an admiring audience of Mediterraneans}}{}{}{ \colorBullet{ORIGIN} Mid 16th century from Latin mediterraneus ‘inland’ (from medius ‘middle’ + terra ‘land’) + {-}an.}%
\par%
\entry{menace}{/ˈmɛnəs/}{ভীতিপ্রদর্শন}{\small{\textsf{\textit{noun, verb}}} \\{\fontspec{DejaVu Sans}▪ }\textsf{\textit{noun}}\\ \textbf{1} A person or thing that is likely to cause harm; a threat or danger. {\fontspec{DejaVu Sans}◇} \textit{a new initiative aimed at beating the menace of drugs} \colorBulletS{SYN} danger, peril, risk, hazard, threat \\{\fontspec{DejaVu Sans}▪ }\textsf{\textit{verb}}\\ \textbf{1} Be a threat or possible danger to. {\fontspec{DejaVu Sans}◇} \textit{Africa's elephants are still menaced by poaching} \colorBulletS{SYN} threatening, ominous, black, thunderous, glowering, brooding, sinister, intimidating, frightening, terrifying, fearsome, mean{-}looking, alarming, forbidding, baleful, warning}{}{}{ \colorBullet{ORIGIN} Middle English via Old French from late Latin minacia, from Latin minax, minac{-} ‘threatening’, from minae ‘threats’.}%
\par%
\entry{mend}{/mɛnd/}{মেরামত করা}{\small{\textsf{\textit{noun, verb}}} \\{\fontspec{DejaVu Sans}▪ }\textsf{\textit{noun}}\\ \textbf{1} A repair in a material. {\fontspec{DejaVu Sans}◇} \textit{the mend was barely visible} \\{\fontspec{DejaVu Sans}▪ }\textsf{\textit{verb}}\\ \textbf{1} Repair (something that is broken or damaged) {\fontspec{DejaVu Sans}◇} \textit{workmen were mending faulty cabling} \colorBulletS{SYN} repair, fix, put back together, piece together, patch up, restore, sew, sew up, stitch, darn, patch, cobble, botch, vamp, vamp up \textbf{2} Add fuel to (a fire) {\fontspec{DejaVu Sans}◇} \textit{he mended the fire and turned the meat on the greenwood racks} \colorBulletS{SYN} stoke, stoke up, make up, charge, fuel}{}{}{ \colorBullet{ORIGIN} Middle English shortening of amend.}%
\par%
\entry{merely}{/ˈmɪəli/}{নিছক}{ \textsf{\textit{adverb}}\ \textbf{1} Just; only. {\fontspec{DejaVu Sans}◇} \textit{Gary, a silent boy, merely nodded} \colorBulletS{SYN} only, purely, solely, simply, entirely, just, but}{}{}{}%
\par%
\entry{meteorologist}{/ˌmiːtɪəˈrɒlədʒɪst/}{আবহাওয়াবিদ}{ \textsf{\textit{noun}}\ \textbf{1} An expert in or student of meteorology; a weather forecaster. {\fontspec{DejaVu Sans}◇} \textit{meteorologists predict rain for the rest of the week} \colorBulletS{SYN} weather forecaster, met officer, weatherman, weatherwoman, nowcaster, weather prophet}{}{}{}%
\par%
\entry{mild}{/mʌɪld/}{হালকা}{\small{\textsf{\textit{adjective, noun}}} \\{\fontspec{DejaVu Sans}▪ }\textsf{\textit{adjective}}\\ \textbf{1} Not severe, serious, or harsh. {\fontspec{DejaVu Sans}◇} \textit{mild criticism} \colorBulletS{SYN} lenient, clement, light \textbf{2} Gentle and not easily provoked. {\fontspec{DejaVu Sans}◇} \textit{she was implacable, despite her mild exterior} \colorBulletS{SYN} gentle, tender, soft, soft{-}hearted, tender{-}hearted, sensitive, sympathetic, warm, warm{-}hearted, unassuming, conciliatory, placid, meek, modest, docile, calm, tranquil, serene, peaceful, peaceable, pacific, good{-}natured, amiable, affable, genial, easy, easy{-}going, mellow \\{\fontspec{DejaVu Sans}▪ }\textsf{\textit{noun}}\\ \textbf{1} A kind of dark beer not strongly flavoured with hops. {\fontspec{DejaVu Sans}◇} \textit{They still brew a delicious dark mild which is one of my favourite drinks.}}{}{}{ \colorBullet{ORIGIN} Old English milde (originally in the sense ‘gracious, not severe in command’), of Germanic origin; related to Dutch and German mild, from an Indo{-}European root shared by Latin mollis and Greek malthakos ‘soft’.}%
\par%
\entry{mildly}{/ˈmʌɪldli/}{আস্তে}{ \textsf{\textit{adverb}}\ \textbf{1} In a mild or gentle manner. {\fontspec{DejaVu Sans}◇} \textit{‘Don't be childish,’ he reproved mildly} \colorBulletS{SYN} without severe punishment, easily, leniently, mildly}{}{}{ \colorBullet{ORIGIN} Used to imply that the reality is more extreme, usually worse.}%
\par%
\entry{militant}{/ˈmɪlɪt(ə)nt/}{জঙ্গিদের}{\small{\textsf{\textit{adjective, noun}}} \\{\fontspec{DejaVu Sans}▪ }\textsf{\textit{adjective}}\\ \textbf{1} Favouring confrontational or violent methods in support of a political or social cause. {\fontspec{DejaVu Sans}◇} \textit{the army are in conflict with militant groups} \colorBulletS{SYN} aggressive, violent, belligerent, bellicose, assertive, pushy, vigorous, forceful, active, ultra{-}active, fierce, combative, pugnacious \\{\fontspec{DejaVu Sans}▪ }\textsf{\textit{noun}}\\ \textbf{1} A militant person. {\fontspec{DejaVu Sans}◇} \textit{militants became increasingly impatient of parliamentary manoeuvres} \colorBulletS{SYN} activist, extremist, radical, enthusiast, supporter, follower, devotee, Young Turk, zealot, fanatic, sectarian, partisan}{}{}{ \colorBullet{ORIGIN} Late Middle English (in the sense ‘engaged in warfare’): from Old French, or from Latin militant{-} ‘serving as a soldier’, from the verb militare (see militate). The current sense dates from the early 20th century.}%
\par%
\entry{million}{/ˈmɪljən/}{মিলিয়ন}{ \textsf{\textit{cardinal number}}\ \textbf{1} The number equivalent to the product of a thousand and a thousand; 1,000,000 or 10⁶ {\fontspec{DejaVu Sans}◇} \textit{a million people will benefit}}{}{}{ \colorBullet{ORIGIN} Late Middle English from Old French, probably from Italian milione, from mille ‘thousand’ + the augmentative suffix {-}one.}%
\par%
\entry{mimosa}{/mɪˈməʊzə/}{লজ্জাবতী লতা}{ \textsf{\textit{noun}}\ \textbf{1} An Australian acacia tree with delicate fernlike leaves and yellow flowers. {\fontspec{DejaVu Sans}◇} \textit{} \textbf{2} A plant of a genus that includes the sensitive plant. {\fontspec{DejaVu Sans}◇} \textit{} \textbf{3} A drink of champagne and orange juice. {\fontspec{DejaVu Sans}◇} \textit{}}{}{}{ \colorBullet{ORIGIN} Modern Latin, apparently from Latin mimus ‘mime’ (because the plant seemingly mimics the sensitivity of an animal) + the feminine suffix {-}osa.}%
\par%
\entry{minnow}{/ˈmɪnəʊ/}{দুর্বল}{ \textsf{\textit{noun}}\ \textbf{1} A small freshwater Eurasian fish of the carp family, which typically forms large shoals. {\fontspec{DejaVu Sans}◇} \textit{} \textbf{2} A small or insignificant person or organization. {\fontspec{DejaVu Sans}◇} \textit{the paper is a minnow in the national newspaper mass market}}{}{}{ \colorBullet{ORIGIN} Late Middle English probably related to Dutch meun and German Münne, influenced by Anglo{-}Norman French menu ‘small, minnow’.}%
\par%
\entry{minuscule}{/ˈmɪnəskjuːl/}{অণুমাত্র}{\small{\textsf{\textit{adjective, noun}}} \\{\fontspec{DejaVu Sans}▪ }\textsf{\textit{adjective}}\\ \textbf{1} Extremely small; tiny. {\fontspec{DejaVu Sans}◇} \textit{a minuscule fragment of DNA} \colorBulletS{SYN} tiny, minute, microscopic, nanoscopic, very small, little, micro, diminutive, miniature, baby, toy, midget, dwarf, pygmy, Lilliputian, infinitesimal \textbf{2} Of or in lower{-}case letters, as distinct from capitals or uncials. {\fontspec{DejaVu Sans}◇} \textit{The small (minuscule) letters are earth symbols{-} the (majuscule) capital letter A is a picture of the missing capstone from Khufu's pyramid.} \\{\fontspec{DejaVu Sans}▪ }\textsf{\textit{noun}}\\ \textbf{1} Minuscule script. {\fontspec{DejaVu Sans}◇} \textit{the humanistic hands of the 15th century were based on the Carolingian minuscule}}{}{}{ \colorBullet{ORIGIN} Early 18th century from French, from Latin minuscula (littera) ‘somewhat smaller (letter)’.}%
\par%
\entry{mire}{/mʌɪə/}{কর্দম}{\small{\textsf{\textit{noun, verb}}} \\{\fontspec{DejaVu Sans}▪ }\textsf{\textit{noun}}\\ \textbf{1} A stretch of swampy or boggy ground. {\fontspec{DejaVu Sans}◇} \textit{acres of land had been reduced to a mire} \colorBulletS{SYN} swamp, morass, bog, peat bog, marsh, mire, quag, marshland, fen, slough, quicksand \textbf{2} A complicated or unpleasant situation from which it is difficult to extricate oneself. {\fontspec{DejaVu Sans}◇} \textit{the service is sinking in the mire of its own regulations} \\{\fontspec{DejaVu Sans}▪ }\textsf{\textit{verb}}\\ \textbf{1} Cause to become stuck in mud. {\fontspec{DejaVu Sans}◇} \textit{sometimes a heavy truck gets mired down} \colorBulletS{SYN} get bogged down, sink, sink down, stick in the mud}{}{}{ \colorBullet{ORIGIN} Middle English from Old Norse mýrr, of Germanic origin; related to moss.}%
\par%
\entry{mirth}{/məːθ/}{আনন্দ}{ \textsf{\textit{noun}}\ \textbf{1} Amusement, especially as expressed in laughter. {\fontspec{DejaVu Sans}◇} \textit{his six{-}foot frame shook with mirth} \colorBulletS{SYN} merriment, high spirits, mirthfulness, cheerfulness, cheeriness, cheer, hilarity, glee, laughter, jocularity, levity, gaiety, buoyancy, blitheness, euphoria, exhilaration, elation, light{-}heartedness, joviality, joy, joyfulness, joyousness, fun, enjoyment, amusement, pleasure, merrymaking, jollity, festivity, revelry, frolics, frolicsomeness}{}{}{ \colorBullet{ORIGIN} Old English myrgth, of Germanic origin; related to merry.}%
\par%
\entry{misappropriation}{/ˌmɪsəˌprəʊprɪˈeɪʃn/}{আত্মসাৎ}{ \textsf{\textit{noun}}\ \textbf{1} The action of misappropriating something; embezzlement. {\fontspec{DejaVu Sans}◇} \textit{an alleged misappropriation of funds} \colorBulletS{SYN} embezzlement, expropriation, swindle, stealing, theft, thieving, pilfering, unauthorized removal}{}{}{}%
\par%
\entry{misbegotten}{/mɪsbɪˈɡɒt(ə)n/}{জারজ}{ \textsf{\textit{adjective}}\ \textbf{1} Badly conceived or planned. {\fontspec{DejaVu Sans}◇} \textit{someone's misbegotten idea of an English country house} \colorBulletS{SYN} ill{-}conceived, ill{-}advised, ill{-}made, badly planned, badly thought{-}out, hare{-}brained, abortive}{}{}{}%
\par%
\entry{misconduct}{/mɪsˈkɒndʌkt/}{অসদাচরণ}{\small{\textsf{\textit{noun, verb}}} \\{\fontspec{DejaVu Sans}▪ }\textsf{\textit{noun}}\\ \textbf{1} Unacceptable or improper behaviour, especially by an employee or professional person. {\fontspec{DejaVu Sans}◇} \textit{she was found guilty of professional misconduct by a disciplinary tribunal and dismissed} \colorBulletS{SYN} wrongdoing, delinquency, unlawfulness, lawlessness, crime, felony, criminality, sin, sinfulness, evil, evil{-}doing \textbf{2} Mismanagement, especially culpable neglect of duties. {\fontspec{DejaVu Sans}◇} \textit{the general was pardoned for misconduct of the war} \colorBulletS{SYN} negligence, neglect, neglectfulness, delinquency, failure, non{-}performance \\{\fontspec{DejaVu Sans}▪ }\textsf{\textit{verb}}\\ \textbf{1} Behave in an improper manner. {\fontspec{DejaVu Sans}◇} \textit{the committee reprimanded two members who were found to have misconducted themselves} \colorBulletS{SYN} misbehave, do wrong, go wrong, behave badly, misconduct oneself, be bad, be naughty, get up to mischief, get up to no good, act up, act badly, give someone trouble, cause someone trouble \textbf{2} Mismanage (an activity) {\fontspec{DejaVu Sans}◇} \textit{there is no evidence that the premises were being misconducted} \colorBulletS{SYN} botch, bungle, fluff, fumble, make a mess of, mishandle, misdirect, misgovern, misconduct, mar, spoil, ruin, mangle, wreck}{}{}{}%
\par%
\entry{miscreant}{/ˈmɪskrɪənt/}{দুর্বৃত্ত}{\small{\textsf{\textit{adjective, noun}}} \\{\fontspec{DejaVu Sans}▪ }\textsf{\textit{adjective}}\\ \textbf{1} (of a person) behaving badly or unlawfully. {\fontspec{DejaVu Sans}◇} \textit{her miscreant husband} \colorBulletS{SYN} unethical, bad, morally wrong, wrongful, wicked, evil, unprincipled, unscrupulous, dishonourable, dishonest, unconscionable, iniquitous, disreputable, fraudulent, corrupt, depraved, vile, villainous, nefarious, base, unfair, underhand, devious \\{\fontspec{DejaVu Sans}▪ }\textsf{\textit{noun}}\\ \textbf{1} A person who has done something wrong or unlawful. {\fontspec{DejaVu Sans}◇} \textit{the police are straining every nerve to bring the miscreants to justice} \colorBulletS{SYN} criminal, culprit, wrongdoer, malefactor, offender, villain, black hat, lawbreaker, evil{-}doer, convict, delinquent, sinner, transgressor, outlaw, trespasser, scoundrel, wretch, reprobate, rogue, rascal}{}{}{ \colorBullet{ORIGIN} Middle English (as an adjective in the sense ‘disbelieving’): from Old French mescreant, present participle of mescreire ‘disbelieve’, from mes{-} ‘mis{-}’ + creire ‘believe’ (from Latin credere).}%
\par%
\entry{miscue1}{}{}{\small{\textsf{\textit{}}}}{}{}{}%
\par%
\entry{mislead}{/mɪsˈliːd/}{ভুল পথে চালিত করা}{ \textsf{\textit{verb}}\ \textbf{1} Cause (someone) to have a wrong idea or impression. {\fontspec{DejaVu Sans}◇} \textit{the government misled the public about the road's environmental impact} \colorBulletS{SYN} deceive, delude, take in, lie to, fool, hoodwink, lead astray, throw off the scent, send on a wild goose chase, put on the wrong track, pull the wool over someone's eyes, pull someone's leg, misguide, misdirect, misinform, give wrong information to}{}{}{}%
\par%
\entry{misnomer}{/mɪsˈnəʊmə/}{অসার্থক নাম}{ \textsf{\textit{noun}}\ \textbf{1} A wrong or inaccurate name or designation. {\fontspec{DejaVu Sans}◇} \textit{morning sickness is a misnomer for many women, since the nausea can occur any time during the day}}{}{}{ \colorBullet{ORIGIN} Late Middle English from Anglo{-}Norman French, from the Old French verb mesnommer, from mes{-} ‘wrongly’ + nommer ‘to name’ (based on Latin nomen ‘name’).}%
\par%
\entry{mitigation}{/mɪtɪˈɡeɪʃ(ə)n/}{প্রশমন}{ \textsf{\textit{noun}}\ \textbf{1} The action of reducing the severity, seriousness, or painfulness of something. {\fontspec{DejaVu Sans}◇} \textit{the identification and mitigation of pollution} \colorBulletS{SYN} alleviation, reduction, diminution, lessening, easing, weakening, lightening, assuagement, palliation, cushioning, dulling, deadening}{}{}{ \colorBullet{ORIGIN} Late Middle English from Old French, or from Latin mitigatio(n{-}), from the verb mitigare ‘alleviate’ (see mitigate).}%
\par%
\entry{mob}{/mɒb/}{উচ্ছৃঙ্খল জনতা}{\small{\textsf{\textit{noun, verb}}} \\{\fontspec{DejaVu Sans}▪ }\textsf{\textit{noun}}\\ \textbf{1} A large crowd of people, especially one that is disorderly and intent on causing trouble or violence. {\fontspec{DejaVu Sans}◇} \textit{a mob of protesters} \colorBulletS{SYN} crowd, horde, multitude, rabble, mass, body, throng \textbf{2} The Mafia or a similar criminal organization. {\fontspec{DejaVu Sans}◇} \textit{he gambled at a time when the Mob ran gaming} \textbf{3} A flock or herd of animals. {\fontspec{DejaVu Sans}◇} \textit{a mob of cattle} \\{\fontspec{DejaVu Sans}▪ }\textsf{\textit{verb}}\\ \textbf{1} Crowd round (someone) or into (a place) in an unruly way. {\fontspec{DejaVu Sans}◇} \textit{he was mobbed by autograph hunters} \colorBulletS{SYN} surround, swarm around, besiege, jostle}{}{Mobs beat to death five people, including two women, and injured 10 others on suspicion of being child kidnappers.}{ \colorBullet{ORIGIN} Late 17th century abbreviation of archaic mobile, short for Latin mobile vulgus ‘excitable crowd’.}%
\par%
\entry{mock}{/mɒk/}{উপহাস}{\small{\textsf{\textit{adjective, noun, verb}}} \\{\fontspec{DejaVu Sans}▪ }\textsf{\textit{adjective}}\\ \textbf{1} Not authentic or real, but without the intention to deceive. {\fontspec{DejaVu Sans}◇} \textit{a mock{-}Georgian red brick house} \colorBulletS{SYN} imitation, artificial, man{-}made, manufactured, simulated, synthetic, ersatz, plastic, so{-}called, fake, false, faux, reproduction, replica, facsimile, dummy, model, toy, make{-}believe, sham, spurious, bogus, counterfeit, fraudulent, forged, pseudo, pretended \\{\fontspec{DejaVu Sans}▪ }\textsf{\textit{noun}}\\ \textbf{1} Mock examinations. {\fontspec{DejaVu Sans}◇} \textit{obtaining Grade A in mocks} \textbf{2} An object of derision. {\fontspec{DejaVu Sans}◇} \textit{he has become the mock of all his contemporaries} \\{\fontspec{DejaVu Sans}▪ }\textsf{\textit{verb}}\\ \textbf{1} Tease or laugh at in a scornful or contemptuous manner. {\fontspec{DejaVu Sans}◇} \textit{opposition MPs mocked the government's decision} \colorBulletS{SYN} ridicule, jeer at, sneer at, deride, treat with contempt, treat contemptuously, scorn, make fun of, poke fun at, laugh at, make jokes about, laugh to scorn, scoff at, pillory, be sarcastic about, tease, taunt, make a monkey of, rag, chaff, jibe at \textbf{2} Make a replica or imitation of something. {\fontspec{DejaVu Sans}◇} \textit{}}{}{}{ \colorBullet{ORIGIN} Late Middle English from Old French mocquer ‘deride’.}%
\par%
\entry{mockery}{/ˈmɒk(ə)ri/}{উপহাস}{ \textsf{\textit{noun}}\ \textbf{1} Teasing and contemptuous language or behaviour directed at a particular person or thing. {\fontspec{DejaVu Sans}◇} \textit{stung by her mockery, Frankie hung his head} \colorBulletS{SYN} ridicule, derision, jeering, sneering, contempt, scorn, scoffing, joking, teasing, taunting, sarcasm, ragging, chaffing, jibing}{}{}{ \colorBullet{ORIGIN} Late Middle English from Old French moquerie, from mocquer ‘to deride’.}%
\par%
\entry{mocking}{/ˈmɒkɪŋ/}{বিদ্রূপকারী}{ \textsf{\textit{adjective}}\ \textbf{1} Making fun of someone or something in a cruel way; derisive. {\fontspec{DejaVu Sans}◇} \textit{he got jeers and mocking laughter as he addressed the marchers}}{}{}{}%
\par%
\entry{modest}{/ˈmɒdɪst/}{বিনয়ী}{ \textsf{\textit{adjective}}\ \textbf{1} Unassuming in the estimation of one's abilities or achievements. {\fontspec{DejaVu Sans}◇} \textit{he was a very modest man, refusing to take any credit for the enterprise} \colorBulletS{SYN} self{-}effacing, self{-}deprecating, humble, unpretentious, unassuming, unpresuming, unostentatious, low{-}key, free from vanity, keeping one's light under a bushel \textbf{2} (of an amount, rate, or level) relatively moderate, limited, or small. {\fontspec{DejaVu Sans}◇} \textit{drink modest amounts of alcohol} \colorBulletS{SYN} moderate, fair, tolerable, passable, adequate, satisfactory, acceptable, unexceptional, small \textbf{3} (of a woman) dressing or behaving so as to avoid impropriety or indecency, especially to avoid attracting sexual attention. {\fontspec{DejaVu Sans}◇} \textit{the modest women wear long{-}sleeved dresses and all but cover their faces}}{}{}{ \colorBullet{ORIGIN} Mid 16th century from French modeste, from Latin modestus ‘keeping due measure’, related to modus ‘measure’.}%
\par%
\entry{modesty}{/ˈmɒdɪsti/}{বিনয়}{ \textsf{\textit{noun}}\ \textbf{1} The quality or state of being unassuming in the estimation of one's abilities. {\fontspec{DejaVu Sans}◇} \textit{with typical modesty he insisted on sharing the credit with others} \colorBulletS{SYN} self{-}effacement, humility, lack of vanity, lack of pretension, unpretentiousness \textbf{2} The quality of being relatively moderate, limited, or small in amount, rate, or level. {\fontspec{DejaVu Sans}◇} \textit{the modesty of his political aspirations} \colorBulletS{SYN} limited scope, moderation, fairness, acceptability, smallness \textbf{3} Behaviour, manner, or appearance intended to avoid impropriety or indecency. {\fontspec{DejaVu Sans}◇} \textit{modesty forbade her to undress in front of so many people} \colorBulletS{SYN} unpretentiousness, simplicity, plainness, lack of pretension, inexpensiveness, lack of extravagance}{}{}{}%
\par%
\entry{mole}{/məʊl/}{আঁচিল}{ \textsf{\textit{noun}}\ \textbf{1} A small burrowing mammal with dark velvety fur, a long muzzle, and very small eyes, feeding mainly on worms, grubs, and other invertebrates. {\fontspec{DejaVu Sans}◇} \textit{} \colorBulletS{SYN} mouldwarp, mouldywarp \textbf{2} A spy who gradually achieves an important position within the security defences of a country. {\fontspec{DejaVu Sans}◇} \textit{a well{-}placed mole was feeding them the names of operatives} \colorBulletS{SYN} spy, agent, secret agent, double agent, undercover agent, operative, plant, infiltrator}{}{}{ \colorBullet{ORIGIN} Late Middle English from the Germanic base of Middle Dutch and Middle Low German mol.}%
\par%
\entry{mole}{/məʊl/}{আঁচিল}{ \textsf{\textit{noun}}\ \textbf{1} A small, often slightly raised blemish on the skin made dark by a high concentration of melanin. {\fontspec{DejaVu Sans}◇} \textit{a mole on her arm had not been there at the beginning of the summer} \colorBulletS{SYN} mark, freckle, blotch, discoloration, spot, blemish}{}{}{ \colorBullet{ORIGIN} Old English māl ‘discoloured spot’, of Germanic origin.}%
\par%
\entry{mole}{/məʊl/}{আঁচিল}{ \textsf{\textit{noun}}\ \textbf{1} A large solid structure on a shore serving as a pier, breakwater, or causeway. {\fontspec{DejaVu Sans}◇} \textit{} \colorBulletS{SYN} breakwater, groyne, dyke, pier, jetty, sea wall, embankment, causeway}{}{}{ \colorBullet{ORIGIN} Mid 16th century from French môle, from Latin moles ‘mass’.}%
\par%
\entry{mole}{/məʊl/}{আঁচিল}{ \textsf{\textit{noun}}\ \textbf{1} The SI unit of amount of substance, equal to the quantity containing as many elementary units as there are atoms in 0.012 kg of carbon{-}12. {\fontspec{DejaVu Sans}◇} \textit{}}{}{}{ \colorBullet{ORIGIN} Early 20th century from German Mol, from Molekul, from Latin (see molecule).}%
\par%
\entry{mole}{/məʊl/}{আঁচিল}{ \textsf{\textit{noun}}\ \textbf{1} An abnormal mass of tissue in the uterus. {\fontspec{DejaVu Sans}◇} \textit{}}{}{}{ \colorBullet{ORIGIN} Late Middle English from French môle, from Latin mola in the sense ‘false conception’.}%
\par%
\entry{mole}{/ˈməʊleɪ/}{আঁচিল}{ \textsf{\textit{noun}}\ \textbf{1} A highly spiced Mexican sauce made chiefly from chilli peppers and chocolate, served with meat. {\fontspec{DejaVu Sans}◇} \textit{}}{}{}{ \colorBullet{ORIGIN} Mexican Spanish, from Nahuatl molli ‘sauce, stew’.}%
\par%
\entry{monkey}{/ˈmʌŋki/}{}{\small{\textsf{\textit{noun, verb}}} \\{\fontspec{DejaVu Sans}▪ }\textsf{\textit{noun}}\\ \textbf{1} A small to medium{-}sized primate that typically has a long tail, most kinds of which live in trees in tropical countries. {\fontspec{DejaVu Sans}◇} \textit{} \textbf{2} A sum of £500. {\fontspec{DejaVu Sans}◇} \textit{} \textbf{3}  {\fontspec{DejaVu Sans}◇} \textit{} \\{\fontspec{DejaVu Sans}▪ }\textsf{\textit{verb}}\\ \textbf{1} Behave in a silly or playful way. {\fontspec{DejaVu Sans}◇} \textit{I saw them monkeying about by the shop} \colorBulletS{SYN} fool about, fool around, play about, play around, clown about, clown around, fiddle{-}faddle, footle about, footle around \textbf{2} Ape; mimic. {\fontspec{DejaVu Sans}◇} \textit{then marched the Three who monkeyed our Great and Dead} \colorBulletS{SYN} imitate, copy, impersonate, do an impression of, take off, do an impersonation of, do, ape, caricature, mock, make fun of, parody, satirize, lampoon, burlesque, travesty}{}{Monkey business: বাঁদরামি }{ \colorBullet{ORIGIN} Mid 16th century of unknown origin, perhaps from Low German.}%
\par%
\entry{mooch}{/muːtʃ/}{ছিঁচকে চুরি করা}{\small{\textsf{\textit{noun, verb}}} \\{\fontspec{DejaVu Sans}▪ }\textsf{\textit{noun}}\\ \textbf{1} An instance of loitering in a bored or listless manner. {\fontspec{DejaVu Sans}◇} \textit{} \textbf{2} A beggar or scrounger. {\fontspec{DejaVu Sans}◇} \textit{} \colorBulletS{SYN} tramp, beggarman, beggarwoman, vagrant, vagabond, down{-}and{-}out, homeless person, derelict, mendicant \\{\fontspec{DejaVu Sans}▪ }\textsf{\textit{verb}}\\ \textbf{1} Loiter in a bored or listless manner. {\fontspec{DejaVu Sans}◇} \textit{he just mooched about his bedsit} \colorBulletS{SYN} loiter, linger, potter, skulk \textbf{2} Ask for or obtain (something) without paying for it. {\fontspec{DejaVu Sans}◇} \textit{a bunch of your friends will show up, mooching food} \colorBulletS{SYN} beg, ask for, ask for money, borrow}{}{}{ \colorBullet{ORIGIN} Late Middle English (in the sense ‘to hoard’): probably from Old French muchier (Anglo{-}Norman muscher) ‘hide, skulk’ compare with mitch. Current senses date from the mid 19th century.}%
\par%
\entry{mooring}{/ˈmɔːrɪŋ/}{আঘাট}{ \textsf{\textit{noun}}\ \textbf{1} A place where a boat or ship is moored. {\fontspec{DejaVu Sans}◇} \textit{they tied up at Water Gypsy's permanent moorings} \colorBulletS{SYN} port, dock, haven, marina, dockyard, boatyard, mooring, anchorage, roads, waterfront}{}{}{}%
\par%
\entry{mope}{/məʊp/}{}{\small{\textsf{\textit{noun, verb}}} \\{\fontspec{DejaVu Sans}▪ }\textsf{\textit{noun}}\\ \textbf{1} A person given to prolonged spells of low spirits. {\fontspec{DejaVu Sans}◇} \textit{a bunch of totally depressed mopes} \colorBulletS{SYN} melancholic, depressive, pessimist, prophet of doom, killjoy, moaner \\{\fontspec{DejaVu Sans}▪ }\textsf{\textit{verb}}\\ \textbf{1} Feel dejected and apathetic. {\fontspec{DejaVu Sans}◇} \textit{no use moping—things could be worse} \colorBulletS{SYN} brood, sulk, be miserable, be gloomy, be sad, be despondent, pine, eat one's heart out, fret, grieve, despair}{}{}{ \colorBullet{ORIGIN} Mid 16th century (the early noun sense being ‘fool or simpleton’): perhaps of Scandinavian origin; compare with Swedish dialect mopa ‘to sulk’.}%
\par%
\entry{moron}{/ˈmɔːrɒn/}{গাধা, বোকা}{ \textsf{\textit{noun}}\ \textbf{1} A stupid person. {\fontspec{DejaVu Sans}◇} \textit{we can't let these thoughtless morons get away with mindless vandalism every weekend} \colorBulletS{SYN} idiot, ass, halfwit, nincompoop, blockhead, buffoon, dunce, dolt, ignoramus, cretin, imbecile, dullard, moron, simpleton, clod}{}{}{ \colorBullet{ORIGIN} Early 20th century (as a medical term denoting an adult with a mental age of about 8–12): from Greek mōron, neuter of mōros ‘foolish’.}%
\par%
\entry{mourn}{/mɔːn/}{শোক প্রকাশ করা}{ \textsf{\textit{verb}}\ \textbf{1} Feel or show sorrow for the death of (someone), typically by following conventions such as the wearing of black clothes. {\fontspec{DejaVu Sans}◇} \textit{Isobel mourned her husband} \colorBulletS{SYN} grieve for, sorrow over, lament for, weep for, shed tears for, shed tears over, keen over, wail over}{}{}{ \colorBullet{ORIGIN} Old English murnan, of Germanic origin.}%
\par%
\entry{mugging}{/ˈmʌɡɪŋ/}{বোকা}{ \textsf{\textit{noun}}\ \textbf{1} An act of attacking and robbing someone in a public place. {\fontspec{DejaVu Sans}◇} \textit{he was the victim of a brutal mugging} \colorBulletS{SYN} theft, robbery, raid, ram raid, burglary, larceny, thievery, break{-}in, hold{-}up}{}{}{}%
\par%
\entry{mule}{/mjuːl/}{অশ্বতর}{ \textsf{\textit{noun}}\ \textbf{1} The offspring of a donkey and a horse (strictly, a male donkey and a female horse), typically sterile and used as a beast of burden. {\fontspec{DejaVu Sans}◇} \textit{} \colorBulletS{SYN} ass \textbf{2} A hybrid plant or animal, especially a sterile one. {\fontspec{DejaVu Sans}◇} \textit{} \textbf{3}  {\fontspec{DejaVu Sans}◇} \textit{} \textbf{4} A small tractor or locomotive, typically one that is electrically powered. {\fontspec{DejaVu Sans}◇} \textit{The trolley pole is mounted on a cylindrical bearing member secured to the side of an electric mule or locomotive for pivotal movement about a vertical axis.} \textbf{5} A coin with the obverse and reverse of designs not originally intended to be used together. {\fontspec{DejaVu Sans}◇} \textit{There are three recognised mule coins from the Republic of India.}}{}{}{ \colorBullet{ORIGIN} Old English mūl, probably of Germanic origin, from Latin mulus, mula; reinforced in Middle English by Old French mule.}%
\par%
\entry{mule}{/mjuːl/}{অশ্বতর}{ \textsf{\textit{noun}}\ \textbf{1} A woman's slipper or light shoe without a back. {\fontspec{DejaVu Sans}◇} \textit{} \colorBulletS{SYN} mule, moccasin, house shoe}{}{}{ \colorBullet{ORIGIN} Mid 16th century from French, ‘slipper’.}%
\par%
\entry{mull}{/mʌl/}{তালগোল পাকান অবস্থা}{ \textsf{\textit{verb}}\ \textbf{1} Think about (a fact, proposal, or request) deeply and at length. {\fontspec{DejaVu Sans}◇} \textit{she began to mull over the various possibilities} \colorBulletS{SYN} ponder, consider, think about, think over, reflect on, contemplate, deliberate, turn over in one's mind, chew over, weigh up, consider the pros and cons of, cogitate on, meditate on, muse on, ruminate on, ruminate over, brood on, have one's mind on, give some thought to, evaluate, examine, study, review, revolve}{}{}{ \colorBullet{ORIGIN} Mid 19th century of uncertain origin.}%
\par%
\entry{mull}{/mʌl/}{তালগোল পাকান অবস্থা}{ \textsf{\textit{verb}}\ \textbf{1} Warm (an alcoholic drink, especially wine or beer) and add sugar and spices to it. {\fontspec{DejaVu Sans}◇} \textit{a glass of mulled wine}}{}{}{ \colorBullet{ORIGIN} Early 17th century of unknown origin.}%
\par%
\entry{mull}{/mʌl/}{তালগোল পাকান অবস্থা}{ \textsf{\textit{noun}}\ \textbf{1} Humus formed under non{-}acid conditions. {\fontspec{DejaVu Sans}◇} \textit{Humus should be of the mull type {-} ranging from acidic to calcareous, or moder in podsol.}}{}{}{ \colorBullet{ORIGIN} 1920s from Danish muld ‘soil’.}%
\par%
\entry{mull}{/mʌl/}{তালগোল পাকান অবস্থা}{ \textsf{\textit{noun}}\ \textbf{1} A promontory. {\fontspec{DejaVu Sans}◇} \textit{the Mull of Kintyre}}{}{}{ \colorBullet{ORIGIN} Middle English compare with Scottish Gaelic maol and Icelandic múli.}%
\par%
\entry{mull}{/mʌl/}{তালগোল পাকান অবস্থা}{ \textsf{\textit{noun}}\ \textbf{1} Thin, soft, plain muslin, used in bookbinding for joining the spine of a book to its cover. {\fontspec{DejaVu Sans}◇} \textit{}}{}{}{ \colorBullet{ORIGIN} Late 17th century abbreviation, from Hindi malmal.}%
\par%
\entry{Mull}{/mʌl/}{তালগোল পাকান অবস্থা}{ \textsf{\textit{proper noun}}\ \textbf{1} A large island of the Inner Hebrides; chief town, Tobermory. It is separated from the coast of Scotland near Oban by the Sound of Mull. {\fontspec{DejaVu Sans}◇} \textit{}}{}{}{}%
\par%
\entry{mutiny}{/ˈmjuːtɪni/}{বিদ্রোহ}{\small{\textsf{\textit{noun, verb}}} \\{\fontspec{DejaVu Sans}▪ }\textsf{\textit{noun}}\\ \textbf{1} An open rebellion against the proper authorities, especially by soldiers or sailors against their officers. {\fontspec{DejaVu Sans}◇} \textit{a mutiny by those manning the weapons could trigger a global war} \colorBulletS{SYN} insurrection, rebellion, revolt, riot, revolution, uprising, rising, coup, coup d'état, putsch, protest, strike \\{\fontspec{DejaVu Sans}▪ }\textsf{\textit{verb}}\\ \textbf{1} Refuse to obey the orders of a person in authority. {\fontspec{DejaVu Sans}◇} \textit{thousands of the soldiers mutinied over the non{-}payment of wages} \colorBulletS{SYN} rise up, rebel, revolt, riot, take part in an insurrection, take part in an uprising, oppose authority, resist authority, defy authority, disobey authority, refuse to obey orders}{}{}{ \colorBullet{ORIGIN} Mid 16th century from obsolete mutine ‘rebellion’, from French mutin ‘mutineer’, based on Latin movere ‘to move’.}%
\par%
\entry{mystery}{/ˈmɪst(ə)ri/}{রহস্য}{ \textsf{\textit{noun}}\ \textbf{1} Something that is difficult or impossible to understand or explain. {\fontspec{DejaVu Sans}◇} \textit{the mysteries of outer space} \colorBulletS{SYN} puzzle, enigma, conundrum, riddle, secret, unsolved problem, problem, question, question mark, closed book \textbf{2} A novel, play, or film dealing with a puzzling crime, especially a murder. {\fontspec{DejaVu Sans}◇} \textit{the 1920s murder mystery, The Ghost Train} \colorBulletS{SYN} thriller, detective novel, detective story, murder story \textbf{3} The secret rites of Greek and Roman pagan religion, or of any ancient or tribal religion, to which only initiates are admitted. {\fontspec{DejaVu Sans}◇} \textit{} \textbf{4} A religious belief based on divine revelation, especially one regarded as beyond human understanding. {\fontspec{DejaVu Sans}◇} \textit{the mystery of Christ}}{}{}{ \colorBullet{ORIGIN} Middle English (in the sense ‘mystic presence, hidden religious symbolism’): from Old French mistere or Latin mysterium, from Greek mustērion; related to mystic.}%
\par%
\entry{mystery}{/ˈmɪst(ə)ri/}{রহস্য}{ \textsf{\textit{noun}}\ \textbf{1} A handicraft or trade, especially when referred to in indentures. {\fontspec{DejaVu Sans}◇} \textit{}}{}{}{ \colorBullet{ORIGIN} Late Middle English from medieval Latin misterium, contraction of ministerium ‘ministry’, by association with mysterium (see mystery).}%
\par%
\end{multicols}%
\pagebreak%
\section*{N}%
\begin{multicols}{2}%
\entry{nachos}{/ˈnatʃəʊz/}{}{ \textsf{\textit{plural noun}}\ \textbf{1} A dish of tortilla chips topped with melted cheese and often also with other savoury toppings. {\fontspec{DejaVu Sans}◇} \textit{he made us nachos and chicken fajitas}}{}{}{ \colorBullet{ORIGIN} 1940s perhaps from Mexican SpanishNacho, pet form of Ignacio, the first name of the chef credited with creating the dish. An alternative derivation is from Spanish nacho ‘flat{-}nosed’.}%
\par%
\entry{nagging}{/ˈnaɡɪŋ/}{বিরক্তি}{ \textsf{\textit{adjective}}\ \textbf{1} (of a person) constantly harassing someone to do something. {\fontspec{DejaVu Sans}◇} \textit{jokes about nagging wives and tyrannous mothers{-}in{-}law} \colorBulletS{SYN} shrewish, complaining, grumbling, fault{-}finding, scolding, carping, cavilling, criticizing \textbf{2} Persistently painful or worrying. {\fontspec{DejaVu Sans}◇} \textit{a nagging pain} \colorBulletS{SYN} persistent, continuous, lingering, niggling, troublesome, unrelenting, unremitting, unabating}{}{}{}%
\par%
\entry{nail}{/neɪl/}{পেরেক}{\small{\textsf{\textit{noun, verb}}} \\{\fontspec{DejaVu Sans}▪ }\textsf{\textit{noun}}\\ \textbf{1} A small metal spike with a broadened flat head, driven into wood to join things together or to serve as a hook. {\fontspec{DejaVu Sans}◇} \textit{don't try and hammer nails into the ceiling joists} \colorBulletS{SYN} pin, spike, tack, rivet \textbf{2} A horny covering on the upper surface of the tip of the finger and toe in humans and other primates. {\fontspec{DejaVu Sans}◇} \textit{she began to bite her nails} \colorBulletS{SYN} fingernail, thumbnail, toenail \textbf{3} A medieval measure of length for cloth, equal to 21/4 inches. {\fontspec{DejaVu Sans}◇} \textit{} \textbf{4} A medieval measure of wool, beef, or other commodity, roughly equal to 7 or 8 pounds. {\fontspec{DejaVu Sans}◇} \textit{} \\{\fontspec{DejaVu Sans}▪ }\textsf{\textit{verb}}\\ \textbf{1} Fasten with a nail or nails. {\fontspec{DejaVu Sans}◇} \textit{the strips are simply nailed to the roof} \colorBulletS{SYN} fasten, attach, fix, affix, secure, tack, hammer, pin, post \textbf{2} Detect or catch (someone, especially a suspected criminal) {\fontspec{DejaVu Sans}◇} \textit{have you nailed the killer?} \colorBulletS{SYN} catch, capture, apprehend, arrest, take into custody, seize, take in, bring in \textbf{3} (of a player) strike (a ball) forcefully and successfully. {\fontspec{DejaVu Sans}◇} \textit{she was stretched to the limit and failed to nail the smash} \textbf{4} Perform (an action or task) perfectly. {\fontspec{DejaVu Sans}◇} \textit{she absolutely nailed the high notes} \textbf{5} (of a man) have sexual intercourse with. {\fontspec{DejaVu Sans}◇} \textit{}}{}{}{ \colorBullet{ORIGIN} Old English nægel (noun), næglan (verb), of Germanic origin; related to Dutch nagel and German Nagel, from an Indo{-}European root shared by Latin unguis and Greek onux.}%
\par%
\entry{narco{-}state}{/ˈnɑːkəʊsteɪt/}{}{ \textsf{\textit{noun}}\ \textbf{1} A country whose economy is dependent on the trade in illegal drugs. {\fontspec{DejaVu Sans}◇} \textit{he turned his nation into a narco{-}state by giving drug cartels free rein to produce and ship cocaine}}{}{How a tiny west african country became the world's first narco state – the guardian headline}{ \colorBullet{ORIGIN} 1970s from narco{-} + state.}%
\par%
\entry{narcotic}{/nɑːˈkɒtɪk/}{মাদক}{\small{\textsf{\textit{adjective, noun}}} \\{\fontspec{DejaVu Sans}▪ }\textsf{\textit{adjective}}\\ \textbf{1} Relating to or denoting narcotics or their effects or use. {\fontspec{DejaVu Sans}◇} \textit{the substance has a mild narcotic effect} \colorBulletS{SYN} soporific, sleep{-}inducing, opiate, hypnotic \\{\fontspec{DejaVu Sans}▪ }\textsf{\textit{noun}}\\ \textbf{1} An addictive drug affecting mood or behaviour, especially an illegal one. {\fontspec{DejaVu Sans}◇} \textit{cultivation of a plant used to make a popular local narcotic} \colorBulletS{SYN} drugs, narcotics, addictive drugs, recreational drugs, illegal drugs}{}{}{ \colorBullet{ORIGIN} Late Middle English from Old French narcotique, via medieval Latin from Greek narkōtikos, from narkoun ‘make numb’.}%
\par%
\entry{nasty}{/ˈnɑːsti/}{কদর্য}{\small{\textsf{\textit{adjective, noun}}} \\{\fontspec{DejaVu Sans}▪ }\textsf{\textit{adjective}}\\ \textbf{1} Very bad or unpleasant. {\fontspec{DejaVu Sans}◇} \textit{plastic bags burn with a nasty, acrid smell} \colorBulletS{SYN} unpleasant, disagreeable, disgusting, distasteful, awful, dreadful, horrible, terrible, vile, foul, abominable, frightful, loathsome, revolting, repulsive, odious, sickening, nauseating, nauseous, repellent, repugnant, horrendous, hideous, appalling, atrocious, offensive, objectionable, obnoxious, unpalatable, unsavoury, unappetizing, off{-}putting, uninviting, dirty, filthy, squalid \textbf{2} Behaving in an unpleasant or spiteful way. {\fontspec{DejaVu Sans}◇} \textit{Harry was a nasty, foul{-}mouthed old devil} \colorBulletS{SYN} unkind, unpleasant, unfriendly, disagreeable, inconsiderate, uncharitable, rude, churlish, spiteful, malicious, mean, mean{-}spirited, ill{-}tempered, ill{-}natured, ill{-}humoured, bad{-}tempered, hostile, vicious, malevolent, evil{-}minded, surly, obnoxious, poisonous, venomous, vindictive, malign, malignant, cantankerous, hateful, hurtful, cruel, wounding, abusive \textbf{3} Damaging or harmful. {\fontspec{DejaVu Sans}◇} \textit{a nasty, vicious{-}looking hatchet} \colorBulletS{SYN} poisonous, toxic, deadly, virulent \\{\fontspec{DejaVu Sans}▪ }\textsf{\textit{noun}}\\ \textbf{1} An unpleasant or harmful person or thing. {\fontspec{DejaVu Sans}◇} \textit{a water conditioner to neutralize chlorine and other nasties}}{}{}{ \colorBullet{ORIGIN} Late Middle English of unknown origin.}%
\par%
\entry{neat}{/niːt/}{ঝরঝরে}{ \textsf{\textit{adjective}}\ \textbf{1} Arranged in a tidy way; in good order. {\fontspec{DejaVu Sans}◇} \textit{the books had been stacked up in neat piles} \colorBulletS{SYN} tidy, neat and tidy, as neat as a new pin, orderly, well ordered, in order, in good order, well kept, shipshape, shipshape and Bristol fashion, in apple{-}pie order, immaculate, spick and span, uncluttered, straight, trim, spruce \textbf{2} Done with or demonstrating skill or efficiency. {\fontspec{DejaVu Sans}◇} \textit{a neat bit of deduction} \colorBulletS{SYN} skilful, deft, dexterous, adroit, adept, expert, practised, accurate, precise, nimble, agile, graceful, stylish \textbf{3} (of liquid, especially spirits) not diluted or mixed with anything else. {\fontspec{DejaVu Sans}◇} \textit{he drank neat Scotch} \colorBulletS{SYN} undiluted, straight, unmixed, unadulterated, unblended, pure, uncut \textbf{4} Very good; excellent. {\fontspec{DejaVu Sans}◇} \textit{it was really neat seeing the city} \colorBulletS{SYN} excellent, very good, superb, outstanding, magnificent, of high quality, of the highest quality, of the highest standard, exceptional, marvellous, wonderful, sublime, perfect, eminent, pre{-}eminent, matchless, peerless, supreme, first{-}rate, first{-}class, superior, superlative, splendid, admirable, worthy, sterling, fine}{}{}{ \colorBullet{ORIGIN} Late 15th century (in the sense ‘clean, free from impurities’): from French net, from Latin nitidus ‘shining’, from nitere ‘to shine’; related to net. The sense ‘bright’ (now obsolete) was recorded in English in the late 16th century.}%
\par%
\entry{neat}{/niːt/}{ঝরঝরে}{ \textsf{\textit{noun}}\ \textbf{1} A bovine animal. {\fontspec{DejaVu Sans}◇} \textit{I had a pretty dinner for them, viz. a brace of stewed carp, six roast chickens and a jowl of hot salmon for the first course; a tanzy and two neats' tongues and cheese second.}}{}{}{ \colorBullet{ORIGIN} Old English, of Germanic origin; related to Dutch noot, also to the base of dialect nait meaning ‘companion’.}%
\par%
\entry{NEAT}{/niːt/}{ঝরঝরে}{ \textsf{\textit{abbreviation}}\ \textbf{1} Non{-}exercise activity thermogenesis (the energy dissipated as heat by a person during minor physical activity, such as fidgeting or shivering, that does not involve a large expenditure of energy and is not perceived as exercise). {\fontspec{DejaVu Sans}◇} \textit{}}{}{}{}%
\par%
\entry{needful}{/ˈniːdfʊl/}{প্রয়োজনীয়}{\small{\textsf{\textit{adjective, noun}}} \\{\fontspec{DejaVu Sans}▪ }\textsf{\textit{adjective}}\\ \textbf{1} Necessary; requisite. {\fontspec{DejaVu Sans}◇} \textit{a further word was needful} \colorBulletS{SYN} obligatory, requisite, required, compulsory, mandatory, imperative, demanded, needed, called for, needful \textbf{2} Needy. {\fontspec{DejaVu Sans}◇} \textit{she gave her money away to needful people} \colorBulletS{SYN} poor, deprived, disadvantaged, underprivileged, in want, needful, badly off, hard up, in reduced circumstances, in straitened circumstances, unable to make ends meet, unable to keep the wolf from the door, poverty{-}stricken, indigent, impoverished, on one's beam{-}ends, as poor as a church mouse, dirt poor, destitute, penurious, impecunious, penniless, moneyless \\{\fontspec{DejaVu Sans}▪ }\textsf{\textit{noun}}\\ \textbf{1} What is necessary. {\fontspec{DejaVu Sans}◇} \textit{I call upon the authorities to do the needful}}{}{}{}%
\par%
\entry{needy}{/ˈniːdi/}{অতি দরিদ্র}{ \textsf{\textit{adjective}}\ \textbf{1} (of a person) lacking the necessities of life; very poor. {\fontspec{DejaVu Sans}◇} \textit{needy and elderly people} \colorBulletS{SYN} poor, deprived, disadvantaged, underprivileged, in want, needful, badly off, hard up, in reduced circumstances, in straitened circumstances, unable to make ends meet, unable to keep the wolf from the door, poverty{-}stricken, indigent, impoverished, on one's beam{-}ends, as poor as a church mouse, dirt poor, destitute, penurious, impecunious, penniless, moneyless \textbf{2} (of a person) needing emotional support; insecure. {\fontspec{DejaVu Sans}◇} \textit{}}{}{}{}%
\par%
\entry{negotiate}{/nɪˈɡəʊʃɪeɪt/}{দরাদরি করা}{ \textsf{\textit{verb}}\ \textbf{1} Obtain or bring about by discussion. {\fontspec{DejaVu Sans}◇} \textit{he negotiated a new contract with the sellers} \colorBulletS{SYN} arrange, work out, thrash out, hammer out, reach an agreement on, agree on, come to terms about, reach terms on, broker \textbf{2} Find a way over or through (an obstacle or difficult route) {\fontspec{DejaVu Sans}◇} \textit{she cautiously negotiated the hairpin bend} \colorBulletS{SYN} get over, get past, get round, make one's way over, make one's way past, make one's way round, make it over, make it past, make it round, clear, cross, pass over \textbf{3} Transfer (a cheque, bill, or other document) to the legal ownership of another person, who thus becomes entitled to any benefit. {\fontspec{DejaVu Sans}◇} \textit{}}{}{}{ \colorBullet{ORIGIN} Early 17th century from Latin negotiat{-} ‘done in the course of business’, from the verb negotiari, from negotium ‘business’, from neg{-} ‘not’ + otium ‘leisure’.}%
\par%
\entry{negotiation}{/nɪɡəʊʃɪˈeɪʃ(ə)n/}{আলাপালোচনা}{ \textsf{\textit{noun}}\ \textbf{1} Discussion aimed at reaching an agreement. {\fontspec{DejaVu Sans}◇} \textit{a worldwide ban is currently under negotiation} \colorBulletS{SYN} discussion, discussions, talks, consultation, consultations, parleying, deliberation, deliberations, conference, debate, dialogue \textbf{2} The action or process of transferring legal ownership of a document. {\fontspec{DejaVu Sans}◇} \textit{}}{}{}{ \colorBullet{ORIGIN} Late 15th century (denoting an act of dealing with another person): from Latin negotiatio(n{-}), from the verb negotiari (see negotiate).}%
\par%
\entry{negotiator}{/nɪˈɡəʊʃɪeɪtə/}{আলাপালোচনকারী}{ \textsf{\textit{noun}}\ \textbf{1} A person who conducts negotiations. {\fontspec{DejaVu Sans}◇} \textit{US trade negotiators} \colorBulletS{SYN} mediator, arbitrator, arbiter, moderator, go{-}between, middleman, intermediary, intercessor, interceder, intervener, conciliator}{}{}{}%
\par%
\entry{nerd}{/nəːd/}{}{ \textsf{\textit{noun}}\ \textbf{1} A foolish or contemptible person who lacks social skills or is boringly studious. {\fontspec{DejaVu Sans}◇} \textit{I was a serious nerd until I discovered girls and cars} \colorBulletS{SYN} bore, dull person}{}{}{ \colorBullet{ORIGIN} 1950s (originally US): of unknown origin.}%
\par%
\entry{newly{-}wed}{}{নবদম্পতি}{ \textsf{\textit{noun}}\ \textbf{1} A recently married person. {\fontspec{DejaVu Sans}◇} \textit{the newly{-}weds shared a kiss} \colorBulletS{SYN} husband and wife, twosome}{}{}{}%
\par%
\entry{nibble}{/ˈnɪb(ə)l/}{মৃদু কামড়}{\small{\textsf{\textit{noun, verb}}} \\{\fontspec{DejaVu Sans}▪ }\textsf{\textit{noun}}\\ \textbf{1} An act or instance of nibbling. {\fontspec{DejaVu Sans}◇} \textit{I'm distracted by a nibble on my line} \colorBulletS{SYN} bite, gnaw, peck, taste \textbf{2} A show of interest in a commercial opportunity. {\fontspec{DejaVu Sans}◇} \textit{I had been trying to unload my apartment for weeks without even a nibble} \\{\fontspec{DejaVu Sans}▪ }\textsf{\textit{verb}}\\ \textbf{1} Take small bites out of. {\fontspec{DejaVu Sans}◇} \textit{he nibbled a biscuit} \colorBulletS{SYN} take small bites, take small bites from, pick, pick at, gnaw, gnaw at, peck at, pick over, eat listlessly, toy with, eat like a bird \textbf{2} Show cautious interest in a commercial opportunity. {\fontspec{DejaVu Sans}◇} \textit{there's an American agent nibbling}}{}{}{ \colorBullet{ORIGIN} Late 15th century probably of Low German or Dutch origin; compare with Low German nibbeln ‘gnaw’.}%
\par%
\entry{niche}{/niːʃ/}{কুলুঙ্গি}{\small{\textsf{\textit{adjective, noun, verb}}} \\{\fontspec{DejaVu Sans}▪ }\textsf{\textit{adjective}}\\ \textbf{1} Denoting or relating to products, services, or interests that appeal to a small, specialized section of the population. {\fontspec{DejaVu Sans}◇} \textit{other companies in this space had to adapt to being niche players} \\{\fontspec{DejaVu Sans}▪ }\textsf{\textit{noun}}\\ \textbf{1} A shallow recess, especially one in a wall to display a statue or other ornament. {\fontspec{DejaVu Sans}◇} \textit{each niche holding a shepherdess in Dresden china} \colorBulletS{SYN} recess, alcove, nook, cranny, slot, slit, hollow, bay, cavity, cubbyhole, pigeonhole, opening, aperture \textbf{2} A comfortable or suitable position in life or employment. {\fontspec{DejaVu Sans}◇} \textit{he is now head chef at a leading law firm and feels he has found his niche} \colorBulletS{SYN} ideal position, calling, vocation, métier, place, function, job, slot, opportunity \textbf{3} A specialized segment of the market for a particular kind of product or service. {\fontspec{DejaVu Sans}◇} \textit{he believes he has found a niche in the market} \\{\fontspec{DejaVu Sans}▪ }\textsf{\textit{verb}}\\ \textbf{1} Place (something) in a niche. {\fontspec{DejaVu Sans}◇} \textit{these elements were niched within the shadowy reaches}}{}{}{ \colorBullet{ORIGIN} Early 17th century from French, literally ‘recess’, from nicher ‘make a nest’, based on Latin nidus ‘nest’.}%
\par%
\entry{nightmare}{/ˈnʌɪtmɛː/}{দুঃস্বপ্ন}{ \textsf{\textit{noun}}\ \textbf{1} A frightening or unpleasant dream. {\fontspec{DejaVu Sans}◇} \textit{I had nightmares after watching the horror movie} \colorBulletS{SYN} bad dream, night terrors \textbf{2} A very unpleasant or frightening experience or prospect. {\fontspec{DejaVu Sans}◇} \textit{the nightmare of racial hatred} \colorBulletS{SYN} ordeal, horror, torment, trial}{}{}{ \colorBullet{ORIGIN} Middle English (denoting a female evil spirit thought to lie upon and suffocate sleepers): from night+ Old English mære ‘incubus’.}%
\par%
\entry{nowhere}{/ˈnəʊwɛː/}{কোথাও}{\small{\textsf{\textit{adjective, adverb, pronoun}}} \\{\fontspec{DejaVu Sans}▪ }\textsf{\textit{adjective}}\\ \textbf{1} Having no prospect of progress or success. {\fontspec{DejaVu Sans}◇} \textit{a nowhere job} \\{\fontspec{DejaVu Sans}▪ }\textsf{\textit{adverb}}\\ \textbf{1} Not in or to any place; not anywhere. {\fontspec{DejaVu Sans}◇} \textit{plants and animals found nowhere else in the world} \\{\fontspec{DejaVu Sans}▪ }\textsf{\textit{pronoun}}\\ \textbf{1} No place. {\fontspec{DejaVu Sans}◇} \textit{there was nowhere for her to sit} \textbf{2} A place that is remote, uninteresting, or nondescript. {\fontspec{DejaVu Sans}◇} \textit{a stretch of road between nowhere and nowhere}}{}{}{ \colorBullet{ORIGIN} Old English nāhwǣr(see no, where).}%
\par%
\entry{nuke}{/njuːk/}{পারমাণবিক অস্ত্র}{\small{\textsf{\textit{noun, verb}}} \\{\fontspec{DejaVu Sans}▪ }\textsf{\textit{noun}}\\ \textbf{1} A nuclear weapon. {\fontspec{DejaVu Sans}◇} \textit{} \\{\fontspec{DejaVu Sans}▪ }\textsf{\textit{verb}}\\ \textbf{1} Attack or destroy with nuclear weapons. {\fontspec{DejaVu Sans}◇} \textit{}}{}{}{ \colorBullet{ORIGIN} 1950s abbreviation of nuclear.}%
\par%
\entry{nullify}{/ˈnʌlɪfʌɪ/}{বাতিল করা}{ \textsf{\textit{verb}}\ \textbf{1} Make legally null and void; invalidate. {\fontspec{DejaVu Sans}◇} \textit{judges were unwilling to nullify government decisions} \colorBulletS{SYN} annul, declare null and void, render null and void, void, invalidate, render invalid}{}{}{}%
\par%
\end{multicols}%
\pagebreak%
\section*{O}%
\begin{multicols}{2}%
\entry{oath}{/əʊθ/}{শপথ}{ \textsf{\textit{noun}}\ \textbf{1} A solemn promise, often invoking a divine witness, regarding one's future action or behaviour. {\fontspec{DejaVu Sans}◇} \textit{they took an oath of allegiance to the king} \colorBulletS{SYN} vow, sworn statement, promise, pledge, avowal, affirmation, attestation, word of honour, word, bond, guarantee, guaranty \textbf{2} A profane or offensive expression used to express anger or other strong emotions. {\fontspec{DejaVu Sans}◇} \textit{he exploded with a mouthful of oaths} \colorBulletS{SYN} swear word, profanity, expletive, four{-}letter word, dirty word, obscenity, imprecation, curse, malediction, blasphemy}{}{}{ \colorBullet{ORIGIN} Old English āth, of Germanic origin; related to Dutch eed and German Eid.}%
\par%
\entry{object}{/ˈɒbdʒɛkt/}{উদ্দেশ্য}{\small{\textsf{\textit{noun, verb}}} \\{\fontspec{DejaVu Sans}▪ }\textsf{\textit{noun}}\\ \textbf{1} A material thing that can be seen and touched. {\fontspec{DejaVu Sans}◇} \textit{he was dragging a large object} \colorBulletS{SYN} thing, article, item, piece, device, gadget, entity, body \textbf{2} A person or thing to which a specified action or feeling is directed. {\fontspec{DejaVu Sans}◇} \textit{disease became the object of investigation} \colorBulletS{SYN} target, butt, focus, recipient, victim \textbf{3} A noun or noun phrase governed by an active transitive verb or by a preposition. {\fontspec{DejaVu Sans}◇} \textit{in Gaelic the word order is verb, subject, object} \textbf{4} A data construct that provides a description of anything known to a computer (such as a processor or a piece of code) and defines its method of operation. {\fontspec{DejaVu Sans}◇} \textit{the interface treats most items, including cells, graphs, and buttons, as objects} \\{\fontspec{DejaVu Sans}▪ }\textsf{\textit{verb}}\\ \textbf{1} Say something to express one's opposition to or disagreement with something. {\fontspec{DejaVu Sans}◇} \textit{residents object to the volume of traffic} \colorBulletS{SYN} protest, protest against, lodge a protest, lodge a protest against, express objections, raise objections, express objections to, raise objections to, express disapproval, express disapproval of, express disagreement, express disagreement with, oppose, be in opposition, be in opposition to, take exception, take exception to, take issue, take issue with, take a stand against, have a problem, have a problem with, argue, argue against, remonstrate, remonstrate against, make a fuss, make a fuss about, quarrel with, disapprove, disapprove of, condemn, draw the line, draw the line at, demur, mind, complain, complain about, moan, moan about, grumble, grumble about, grouse, grouse about, cavil, cavil at, quibble, quibble about}{}{I object: আমি আপত্তি জানাচ্ছি}{ \colorBullet{ORIGIN} Late Middle English from medieval Latin objectum ‘thing presented to the mind’, neuter past participle (used as a noun) of Latin obicere, from ob{-} ‘in the way of’ + jacere ‘to throw’; the verb may also partly represent the Latin frequentative objectare.}%
\par%
\entry{objectionable}{/əbˈdʒɛkʃ(ə)nəb(ə)l/}{আপত্তিকর}{ \textsf{\textit{adjective}}\ \textbf{1} Arousing distaste or opposition; unpleasant or offensive. {\fontspec{DejaVu Sans}◇} \textit{I find his theory objectionable in its racist undertones} \colorBulletS{SYN} offensive, unpleasant, disagreeable, distasteful, displeasing, unacceptable, off{-}putting, undesirable, obnoxious}{}{1. Objectionable remarks 2. India’s supreme court today granted bail to journalist prashant kanojia who was arrested for allegedly making objectionable comments against uttar pradesh chief minister yogi adityanath on social media.}{}%
\par%
\entry{obnoxious}{/əbˈnɒkʃəs/}{আপত্তিকর}{ \textsf{\textit{adjective}}\ \textbf{1} Extremely unpleasant. {\fontspec{DejaVu Sans}◇} \textit{obnoxious odours} \colorBulletS{SYN} disagreeable, irksome, troublesome, annoying, irritating, vexatious, displeasing, uncomfortable, distressing, nasty, horrible, appalling, terrible, awful, dreadful, hateful, detestable, miserable, abominable, execrable, odious, invidious, objectionable, offensive, obnoxious, repugnant, repulsive, repellent, revolting, disgusting, distasteful, nauseating, unsavoury, unpalatable, ugly}{}{}{ \colorBullet{ORIGIN} Late 16th century (in the sense ‘vulnerable to harm’): from Latin obnoxiosus, from obnoxius ‘exposed to harm’, from ob{-} ‘towards’ + noxa ‘harm’. The current sense, influenced by noxious, dates from the late 17th century.}%
\par%
\entry{obscene}{/əbˈsiːn/}{অশ্লীল}{ \textsf{\textit{adjective}}\ \textbf{1} (of the portrayal or description of sexual matters) offensive or disgusting by accepted standards of morality and decency. {\fontspec{DejaVu Sans}◇} \textit{obscene jokes} \colorBulletS{SYN} pornographic, indecent, salacious, smutty, X{-}rated, lewd, rude, dirty, filthy, vulgar, foul, coarse, crude, gross, vile, nasty, disgusting, offensive, shameless, immoral, improper, immodest, impure, indecorous, indelicate, unwholesome, scabrous, off colour, lubricious, risqué, ribald, bawdy, suggestive, titillating, racy, erotic, carnal, sensual, sexy, lascivious, lecherous, licentious, libidinous, goatish, degenerate, depraved, amoral, debauched, dissolute, prurient}{}{}{ \colorBullet{ORIGIN} Late 16th century from French obscène or Latin obscaenus ‘ill{-}omened or abominable’.}%
\par%
\entry{obscure}{/əbˈskjʊə/}{অস্পষ্ট}{\small{\textsf{\textit{adjective, verb}}} \\{\fontspec{DejaVu Sans}▪ }\textsf{\textit{adjective}}\\ \textbf{1} Not discovered or known about; uncertain. {\fontspec{DejaVu Sans}◇} \textit{his origins and parentage are obscure} \colorBulletS{SYN} unclear, uncertain, unknown, in doubt, doubtful, dubious, mysterious, hazy, vague, indeterminate, concealed, hidden \textbf{2} Not clearly expressed or easily understood. {\fontspec{DejaVu Sans}◇} \textit{obscure references to Proust} \colorBulletS{SYN} abstruse, recondite, arcane, esoteric, recherché, occult \\{\fontspec{DejaVu Sans}▪ }\textsf{\textit{verb}}\\ \textbf{1} Keep from being seen; conceal. {\fontspec{DejaVu Sans}◇} \textit{grey clouds obscure the sun} \colorBulletS{SYN} hide, conceal, cover, veil, shroud, screen, mask, cloak, cast a shadow over, shadow, envelop, mantle, block, block out, blank out, obliterate, eclipse, overshadow}{}{}{ \colorBullet{ORIGIN} Late Middle English from Old French obscur, from Latin obscurus ‘dark’, from an Indo{-}European root meaning ‘cover’.}%
\par%
\entry{observe}{/əbˈzəːv/}{পালন করা; মান্য করা}{ \textsf{\textit{verb}}\ \textbf{1} Notice or perceive (something) and register it as being significant. {\fontspec{DejaVu Sans}◇} \textit{she observed that all the chairs were already occupied} \colorBulletS{SYN} notice, see, note, perceive, discern, remark, spot, detect, discover, distinguish, make out \textbf{2} Make a remark. {\fontspec{DejaVu Sans}◇} \textit{‘It's chilly,’ she observed} \colorBulletS{SYN} comment, remark, say, mention, note, declare, announce, state, utter, pronounce, interpose, interject \textbf{3} Fulfil or comply with (a social, legal, ethical, or religious obligation) {\fontspec{DejaVu Sans}◇} \textit{a tribunal must observe the principles of natural justice} \colorBulletS{SYN} comply with, abide by, keep, obey, adhere to, conform to, heed, honour, respect, be heedful of, pay attention to, follow, acquiesce in, consent to, accept, defer to, fulfil, stand by}{}{}{ \colorBullet{ORIGIN} Late Middle English (in observe (sense 3)): from Old French observer, from Latin observare ‘to watch’, from ob{-} ‘towards’ + servare ‘attend to, look at’.}%
\par%
\entry{observer}{/əbˈzəːvə/}{পর্যবেক্ষক}{ \textsf{\textit{noun}}\ \textbf{1} A person who watches or notices something. {\fontspec{DejaVu Sans}◇} \textit{to a casual observer, he was at peace} \colorBulletS{SYN} spectator, onlooker, watcher, looker{-}on, fly on the wall, viewer, witness, eyewitness, bystander, sightseer}{}{}{}%
\par%
\entry{obsess}{/əbˈsɛs/}{ভাববে}{ \textsf{\textit{verb}}\ \textbf{1} Preoccupy or fill the mind of (someone) continually and to a troubling extent. {\fontspec{DejaVu Sans}◇} \textit{he was obsessed with the idea of revenge} \colorBulletS{SYN} preoccupy, be uppermost in someone's mind, prey on someone's mind, prey on, possess, haunt, consume, plague, torment, hound, bedevil, take control of, take over, become an obsession with, have a hold on, engross, eat up, have a grip on, grip, dominate, rule, control, beset, monopolize}{}{}{ \colorBullet{ORIGIN} Late Middle English (in the sense ‘haunt, possess’, referring to an evil spirit): from Latin obsess{-} ‘besieged’, from the verb obsidere, from ob{-} ‘opposite’ + sedere ‘sit’. The current sense dates from the late 19th century.}%
\par%
\entry{obsession}{/əbˈsɛʃ(ə)n/}{আবেশ}{ \textsf{\textit{noun}}\ \textbf{1} The state of being obsessed with someone or something. {\fontspec{DejaVu Sans}◇} \textit{she cared for him with a devotion bordering on obsession}}{}{}{ \colorBullet{ORIGIN} Early 16th century (in the sense ‘siege’): from Latin obsessio(n{-}), from the verb obsidere (see obsess).}%
\par%
\entry{obvious}{/ˈɒbvɪəs/}{সুস্পষ্ট}{ \textsf{\textit{adjective}}\ \textbf{1} Easily perceived or understood; clear, self{-}evident, or apparent. {\fontspec{DejaVu Sans}◇} \textit{unemployment has been the most obvious cost of the recession} \colorBulletS{SYN} clear, plain, plain to see, crystal clear, evident, apparent, manifest, patent, conspicuous, pronounced, transparent, clear{-}cut, palpable, prominent, marked, decided, salient, striking, distinct, bold, noticeable, perceptible, perceivable, visible, discernible, detectable, observable, tangible, recognizable}{}{}{ \colorBullet{ORIGIN} Late 16th century (in the sense ‘frequently encountered’): from Latin obvius (from the phrase ob viam ‘in the way’) + {-}ous.}%
\par%
\entry{occupy}{/ˈɒkjʊpʌɪ/}{দখল করা}{ \textsf{\textit{verb}}\ \textbf{1} Reside or have one's place of business in (a building) {\fontspec{DejaVu Sans}◇} \textit{the rented flat she occupies in Hampstead} \colorBulletS{SYN} inhabited, lived{-}in, tenanted, settled \textbf{2} Fill or preoccupy (the mind) {\fontspec{DejaVu Sans}◇} \textit{her mind was occupied with alarming questions} \colorBulletS{SYN} engage, busy, employ, distract, absorb, engross, preoccupy, hold, hold the attention of, immerse, interest, involve, entertain, divert, amuse, beguile \textbf{3} Take control of (a place, especially a country) by military conquest or settlement. {\fontspec{DejaVu Sans}◇} \textit{Syria was occupied by France under a League of Nations mandate} \colorBulletS{SYN} capture, seize, take possession of, conquer, invade, overrun, take over, colonize, garrison, annex, dominate, subjugate, hegemonize, hold, commandeer, requisition}{}{}{ \colorBullet{ORIGIN} Middle English formed irregularly from Old French occuper, from Latin occupare ‘seize’. A now obsolete vulgar sense ‘have sexual relations with’ seems to have led to the general avoidance of the word in the 17th and most of the 18th century.}%
\par%
\entry{one}{/wʌn/}{এক}{\small{\textsf{\textit{cardinal number, pronoun}}} \\{\fontspec{DejaVu Sans}▪ }\textsf{\textit{cardinal number}}\\ \textbf{1}  {\fontspec{DejaVu Sans}◇} \textit{there's only room for one person} \colorBulletS{SYN} a single, a solitary, a sole, a lone \textbf{2} The same; identical. {\fontspec{DejaVu Sans}◇} \textit{all types of training meet one common standard} \colorBulletS{SYN} only, single, solitary, sole \textbf{3} A joke or story. {\fontspec{DejaVu Sans}◇} \textit{the one about the Englishman, the Irishman, and the Yank} \textbf{4} An alcoholic drink. {\fontspec{DejaVu Sans}◇} \textit{a cool one after a day on the water} \textbf{5} Alone. {\fontspec{DejaVu Sans}◇} \textit{the time when you one tackled a field of cane and finished before the others had even started} \\{\fontspec{DejaVu Sans}▪ }\textsf{\textit{pronoun}}\\ \textbf{1} Referring to a person or thing previously mentioned or easily identified. {\fontspec{DejaVu Sans}◇} \textit{her mood changed from one of moroseness to one of joy} \textbf{2} A person of a specified kind. {\fontspec{DejaVu Sans}◇} \textit{you're the one who ruined her life} \textbf{3} Used to refer to the speaker, or any person, as representing people in general. {\fontspec{DejaVu Sans}◇} \textit{one must admire him for his willingness}}{}{}{ \colorBullet{ORIGIN} Old English ān, of Germanic origin; related to Dutch een and German ein, from an Indo{-}European root shared by Latin unus. The initial w sound developed before the 15th century and was occasionally represented in the spelling; it was not accepted into standard English until the late 17th century.}%
\par%
\entry{onrush}{/ˈɒnrʌʃ/}{আমদানি}{ \textsf{\textit{noun}}\ \textbf{1} A surging rush forward. {\fontspec{DejaVu Sans}◇} \textit{the mesmerizing onrush of the sea} \colorBulletS{SYN} assault, attack, offensive, aggression, advance, charge, onrush, rush, storming, sortie, sally, raid, descent, incursion, invasion, foray, push, thrust, drive, blitz, bombardment, barrage, salvo, storm, volley, shower, torrent, broadside}{}{}{}%
\par%
\entry{onshore}{/ˈɒnʃɔː/}{ডাঙার দিকে}{\small{\textsf{\textit{adjective \& adverb, verb}}} \\{\fontspec{DejaVu Sans}▪ }\textsf{\textit{adjective \& adverb}}\\ \textbf{1} Situated or occurring on land (often used in relation to the oil and gas industry) {\fontspec{DejaVu Sans}◇} \textit{an onshore oilfield} \\{\fontspec{DejaVu Sans}▪ }\textsf{\textit{verb}}\\ \textbf{1} (of a company) transfer (a business operation that was moved overseas) back to the country from which it was originally relocated. {\fontspec{DejaVu Sans}◇} \textit{the case study showed improvement in many key areas once the company decided to onshore its call centre activity}}{}{}{}%
\par%
\entry{onus}{/ˈəʊnəs/}{ভার}{ \textsf{\textit{noun}}\ \textbf{1} Something that is one's duty or responsibility. {\fontspec{DejaVu Sans}◇} \textit{the onus is on you to show that you have suffered loss} \colorBulletS{SYN} burden, responsibility, liability, obligation, duty, weight, load, charge, mantle, encumbrance}{}{}{ \colorBullet{ORIGIN} Mid 17th century from Latin, literally ‘load or burden’.}%
\par%
\entry{optimistic}{/ɒptɪˈmɪstɪk/}{আশাবাদী}{ \textsf{\textit{adjective}}\ \textbf{1} Hopeful and confident about the future. {\fontspec{DejaVu Sans}◇} \textit{the optimistic mood of the Sixties} \colorBulletS{SYN} cheerful, cheery, positive, confident, hopeful, sanguine, bullish, buoyant, bright}{}{}{}%
\par%
\entry{ore}{/ɔː/}{আকরিক}{ \textsf{\textit{noun}}\ \textbf{1} A naturally occurring solid material from which a metal or valuable mineral can be extracted profitably. {\fontspec{DejaVu Sans}◇} \textit{a good deposit of lead{-}bearing ores}}{}{}{ \colorBullet{ORIGIN} Old English ōra ‘unwrought metal’, of West Germanic origin; influenced in form by Old English ār ‘bronze’ (related to Latin aes ‘crude metal, bronze’).}%
\par%
\entry{öre}{/ˈəːrə/}{আকরিক}{ \textsf{\textit{noun}}\ \textbf{1} A monetary unit of Sweden, equal to one hundredth of a krona. {\fontspec{DejaVu Sans}◇} \textit{Today we use coins with the value 10 krona, 5 krona, 1 krona and 50 öre.}}{}{}{ \colorBullet{ORIGIN} Swedish.}%
\par%
\entry{ornate}{/ɔːˈneɪt/}{অলঙ্কৃত}{ \textsf{\textit{adjective}}\ \textbf{1} Elaborately or highly decorated. {\fontspec{DejaVu Sans}◇} \textit{an ornate wrought{-}iron railing} \colorBulletS{SYN} elaborate, decorated, embellished, adorned, ornamented, fancy, over{-}elaborate, fussy, busy, ostentatious, showy, baroque, rococo, florid, wedding{-}cake, gingerbread}{}{Ornate flying snake: সাপ বিশেষ}{ \colorBullet{ORIGIN} Late Middle English from Latin ornatus ‘adorned’, past participle of ornare.}%
\par%
\entry{orthodox}{/ˈɔːθədɒks/}{গোঁড়া}{ \textsf{\textit{adjective}}\ \textbf{1} Following or conforming to the traditional or generally accepted rules or beliefs of a religion, philosophy, or practice. {\fontspec{DejaVu Sans}◇} \textit{Burke's views were orthodox in his time} \colorBulletS{SYN} conservative, traditional, observant, conformist, devout, strict, true, true blue, of the faith, of the true faith \textbf{2} Of the ordinary or usual type; normal. {\fontspec{DejaVu Sans}◇} \textit{they avoided orthodox jazz venues} \colorBulletS{SYN} normal, average, ordinary, standard, regular, routine, run{-}of{-}the{-}mill, stock, orthodox, conventional, predictable, unsurprising, unremarkable, unexceptional \textbf{3} Relating to Orthodox Judaism. {\fontspec{DejaVu Sans}◇} \textit{Orthodox Jewish boys} \textbf{4} Relating to the Orthodox Church. {\fontspec{DejaVu Sans}◇} \textit{}}{}{}{ \colorBullet{ORIGIN} Late Middle English from Greek orthodoxos (probably via ecclesiastical Latin), from orthos ‘straight or right’ + doxa ‘opinion’.}%
\par%
\entry{ostracize}{/ˈɒstrəsʌɪz/}{বহিষ্কৃত করা}{ \textsf{\textit{verb}}\ \textbf{1} Exclude from a society or group. {\fontspec{DejaVu Sans}◇} \textit{she was declared a witch and ostracized by the villagers} \colorBulletS{SYN} exclude, shun, spurn, cold{-}shoulder, give someone the cold shoulder, reject, repudiate, boycott, blackball, blacklist, cast off, cast out, shut out, avoid, ignore, snub, cut dead, keep at arm's length, leave out in the cold, bar, ban, debar, banish, exile, expel \textbf{2} (in ancient Greece) banish (an unpopular or overly powerful citizen) from a city for five or ten years by popular vote. {\fontspec{DejaVu Sans}◇} \textit{Themistocles was indeed out of favour at Athens by the end of the 470s, when he was ostracized} \colorBulletS{SYN} banish, exile, deport, evict, expatriate, dismiss, displace}{}{}{ \colorBullet{ORIGIN} Mid 17th century from Greek ostrakizein, from ostrakon ‘shell or potsherd’ (on which names were written in voting to banish unpopular citizens).}%
\par%
\entry{ostrich}{/ˈɒstrɪtʃ/}{উটপাখী}{ \textsf{\textit{noun}}\ \textbf{1} A flightless swift{-}running African bird with a long neck, long legs, and two toes on each foot. It is the largest living bird, with males reaching a height of up to 2.75 m. {\fontspec{DejaVu Sans}◇} \textit{} \textbf{2} A person who refuses to face reality or accept facts. {\fontspec{DejaVu Sans}◇} \textit{don't be an ostrich when it comes to security systems}}{}{}{ \colorBullet{ORIGIN} Middle English from Old French ostriche, from Latin avis ‘bird’ + late Latin struthio (from Greek strouthiōn ‘ostrich’, from strouthos ‘sparrow or ostrich’).}%
\par%
\entry{ouster}{/ˈaʊstə/}{বেদখল}{ \textsf{\textit{noun}}\ \textbf{1} Ejection from a property, especially wrongful ejection; deprivation of an inheritance. {\fontspec{DejaVu Sans}◇} \textit{ouster proceedings to remove the husband from the matrimonial home} \textbf{2} Dismissal or expulsion from a position. {\fontspec{DejaVu Sans}◇} \textit{the junta's ouster of the Emperor} \colorBulletS{SYN} overthrow, overturning, toppling, downfall, removal from office, removal, unseating, dethronement, supplanting, displacement, dismissal, discharge, ousting, drumming out, throwing out, forcing out, driving out, expulsion, expelling, ejection, ejecting}{}{}{}%
\par%
\entry{outage}{/ˈaʊtɪdʒ/}{বিভ্রাট}{ \textsf{\textit{noun}}\ \textbf{1} A period when a power supply or other service is not available or when equipment is closed down. {\fontspec{DejaVu Sans}◇} \textit{frequent power outages}}{}{}{}%
\par%
\entry{outbreak}{/ˈaʊtbreɪk/}{প্রাদুর্ভাব}{ \textsf{\textit{noun}}\ \textbf{1} A sudden occurrence of something unwelcome, such as war or disease. {\fontspec{DejaVu Sans}◇} \textit{the outbreak of World War II} \colorBulletS{SYN} eruption, flare{-}up, upsurge, outburst, epidemic, breakout, sudden appearance, rash, wave, spate, flood, explosion, burst, blaze, flurry}{}{Dengue outbreak}{}%
\par%
\entry{outclass}{/aʊtˈklɑːs/}{}{ \textsf{\textit{verb}}\ \textbf{1} Be far superior to. {\fontspec{DejaVu Sans}◇} \textit{Villa totally outclassed us in the first half} \colorBulletS{SYN} surpass, be superior to, be better than, outshine, overshadow, eclipse, outdo, outplay, outmanoeuvre, outdistance, outstrip, outrun, outpace, out{-}think, get the better of, dwarf, put in the shade, upstage, transcend}{}{}{}%
\par%
\entry{outrage}{/ˈaʊtreɪdʒ/}{অত্যাচার}{\small{\textsf{\textit{noun, verb}}} \\{\fontspec{DejaVu Sans}▪ }\textsf{\textit{noun}}\\ \textbf{1} An extremely strong reaction of anger, shock, or indignation. {\fontspec{DejaVu Sans}◇} \textit{her voice trembled with outrage} \colorBulletS{SYN} indignation, fury, anger, rage, disapproval, wrath, shock, resentment, horror, disgust, amazement \\{\fontspec{DejaVu Sans}▪ }\textsf{\textit{verb}}\\ \textbf{1} Arouse fierce anger, shock, or indignation in (someone) {\fontspec{DejaVu Sans}◇} \textit{the public were outraged at the brutality involved} \colorBulletS{SYN} enrage, infuriate, incense, anger, scandalize, offend, give offence to, make indignant, affront, be an affront to, shock, horrify, disgust, revolt, repel, appal, displease}{}{}{ \colorBullet{ORIGIN} Middle English (in the senses ‘lack of moderation’ and ‘violent behaviour’): from Old French ou(l)trage, based on Latin ultra ‘beyond’. Sense development has been affected by the belief that the word is a compound of out and rage.}%
\par%
\entry{outrageous}{/aʊtˈreɪdʒəs/}{ভয়ানক}{ \textsf{\textit{adjective}}\ \textbf{1} Shockingly bad or excessive. {\fontspec{DejaVu Sans}◇} \textit{an outrageous act of bribery} \colorBulletS{SYN} shocking, disgraceful, scandalous, atrocious, appalling, abhorrent, monstrous, heinous \textbf{2} Very bold and unusual and rather shocking. {\fontspec{DejaVu Sans}◇} \textit{her outrageous leotards and sexy routines} \colorBulletS{SYN} eye{-}catching, startling, striking, flamboyant, showy, flashy, gaudy, ostentatious, dazzling}{}{}{ \colorBullet{ORIGIN} Late Middle English from Old French outrageus, from outrage ‘excess’ (see outrage).}%
\par%
\entry{overcast}{/ˈəʊvəkɑːst/}{মেঘাচ্ছন্ন}{\small{\textsf{\textit{adjective, noun, verb}}} \\{\fontspec{DejaVu Sans}▪ }\textsf{\textit{adjective}}\\ \textbf{1} (of the sky or weather) marked by a covering of grey cloud; dull. {\fontspec{DejaVu Sans}◇} \textit{a chilly, overcast day} \colorBulletS{SYN} cloudy, clouded, clouded over, overclouded, sunless, darkened, dark, grey, black, leaden, heavy, dull, murky, dirty, misty, hazy, foggy, louring, threatening, menacing, promising rain, dismal, dreary, cheerless, sombre \textbf{2} (of the edge of a piece of fabric) sewn with long slanting stitches to prevent fraying. {\fontspec{DejaVu Sans}◇} \textit{Make new zipper stops by hand sewing a few overcast stitches on the edge of each tape just above the last tooth.} \\{\fontspec{DejaVu Sans}▪ }\textsf{\textit{noun}}\\ \textbf{1} Cloud covering a large part of the sky. {\fontspec{DejaVu Sans}◇} \textit{the planes found the target obscured by overcast} \\{\fontspec{DejaVu Sans}▪ }\textsf{\textit{verb}}\\ \textbf{1} Cover with clouds or shade. {\fontspec{DejaVu Sans}◇} \textit{the pebbled beach, overcast with the shadows of the high cliffs} \textbf{2} Stitch over (a raw edge) to prevent fraying. {\fontspec{DejaVu Sans}◇} \textit{finish off the raw edge of the hem by overcasting it}}{}{}{}%
\par%
\entry{overhaul}{/əʊvəˈhɔːl/}{পৃষ্ঠা পরিবর্তনের}{\small{\textsf{\textit{noun, verb}}} \\{\fontspec{DejaVu Sans}▪ }\textsf{\textit{noun}}\\ \textbf{1} A thorough examination of machinery or a system, with repairs or changes made if necessary. {\fontspec{DejaVu Sans}◇} \textit{a major overhaul of environmental policies} \\{\fontspec{DejaVu Sans}▪ }\textsf{\textit{verb}}\\ \textbf{1} Take apart (a piece of machinery or equipment) in order to examine it and repair it if necessary. {\fontspec{DejaVu Sans}◇} \textit{the steering box was recently overhauled} \colorBulletS{SYN} service, maintain, repair, mend, fix up, patch up, rebuild, renovate, revamp, recondition, remodel, refit, refurbish, modernize \textbf{2} Overtake (someone), especially in a sporting event. {\fontspec{DejaVu Sans}◇} \textit{Jodami overhauled his chief rival} \colorBulletS{SYN} overtake, pass, get past, go past, go by, go faster than, get ahead of, pull ahead of, outdistance, outstrip}{}{}{ \colorBullet{ORIGIN} Early 17th century (originally in nautical use in the sense ‘release rope tackle by slackening’): from over{-}+ haul.}%
\par%
\entry{overplay}{/əʊvəˈpleɪ/}{বাড়াবাড়ি}{ \textsf{\textit{verb}}\ \textbf{1} Give undue importance to; overemphasize. {\fontspec{DejaVu Sans}◇} \textit{he thinks the idea of a special relationship between sitter and artist is much overplayed} \colorBulletS{SYN} overstate, overemphasize, overstress, overestimate, overvalue, magnify, amplify, aggrandize, inflate}{}{}{ \colorBullet{ORIGIN} 1(in a card game) play or bet on one's hand with a mistaken optimism.2Spoil one's chance of success through excessive confidence in one's position.}%
\par%
\entry{overwhelming}{/əʊvəˈwɛlmɪŋ/}{অভিভূতকারী}{ \textsf{\textit{adjective}}\ \textbf{1} Very great in amount. {\fontspec{DejaVu Sans}◇} \textit{his party won overwhelming support} \colorBulletS{SYN} very large, profuse, enormous, immense, inordinate, massive, huge, formidable, stupendous, prodigious, fantastic, staggering, shattering, devastating, sweeping}{}{}{}%
\par%
\entry{owe}{/əʊ/}{ঋণগ্রস্ত থাকা}{ \textsf{\textit{verb}}\ \textbf{1} Have an obligation to pay or repay (something, especially money) in return for something received. {\fontspec{DejaVu Sans}◇} \textit{they have denied they owe money to the company} \colorBulletS{SYN} be in debt, be in debt to, be indebted, be indebted to, be in arrears, be in arrears to, be under an obligation, be under an obligation to, be obligated, be obligated to, be beholden to}{}{}{ \colorBullet{ORIGIN} Old English āgan ‘own, have it as an obligation’, of Germanic origin; from an Indo{-}European root shared by Sanskrit īs ‘possess, own’. Compare with ought.}%
\par%
\entry{owing}{/ˈəʊɪŋ/}{করিতে হইবে এমন}{ \textsf{\textit{adjective}}\ \textbf{1} (of money) yet to be paid. {\fontspec{DejaVu Sans}◇} \textit{no rent was owing} \colorBulletS{SYN} unpaid, unsettled, to be paid, payable, receivable, due, overdue, undischarged, owed, outstanding, in arrears, in the red}{}{}{ \colorBullet{ORIGIN} Because of or on account of.}%
\par%
\end{multicols}%
\pagebreak%
\section*{P}%
\begin{multicols}{2}%
\entry{pace}{/peɪs/}{গতি}{\small{\textsf{\textit{noun, verb}}} \\{\fontspec{DejaVu Sans}▪ }\textsf{\textit{noun}}\\ \textbf{1} A single step taken when walking or running. {\fontspec{DejaVu Sans}◇} \textit{Kirov stepped back a pace} \colorBulletS{SYN} step, stride, footstep \textbf{2} Speed in walking, running, or moving. {\fontspec{DejaVu Sans}◇} \textit{he's an aggressive player with plenty of pace} \colorBulletS{SYN} speed, rate, swiftness, quickness, rapidity, velocity, tempo, momentum \\{\fontspec{DejaVu Sans}▪ }\textsf{\textit{verb}}\\ \textbf{1} Walk at a steady speed, especially without a particular destination and as an expression of anxiety or annoyance. {\fontspec{DejaVu Sans}◇} \textit{we paced up and down in exasperation} \colorBulletS{SYN} walk, stride, tread, march, pound, patrol, walk up and down, walk back and forth, cross, traverse \textbf{2} Move or develop (something) at a particular rate or speed. {\fontspec{DejaVu Sans}◇} \textit{the action is paced to the beat of a perky march}}{}{}{ \colorBullet{ORIGIN} Middle English from Old French pas, from Latin passus ‘stretch (of the leg)’, from pandere ‘to stretch’.}%
\par%
\entry{pace}{/ˈpɑːtʃeɪ/}{গতি}{ \textsf{\textit{preposition}}\ \textbf{1} With due respect to (someone or their opinion), used to express polite disagreement or contradiction. {\fontspec{DejaVu Sans}◇} \textit{narrative history, pace some theorists, is by no means dead}}{}{}{ \colorBullet{ORIGIN} Latin, literally ‘in peace’, ablative of pax, as in pace tua ‘by your leave’.}%
\par%
\entry{PACE}{/ˈpeɪsi/}{গতি}{ \textsf{\textit{abbreviation}}\ \textbf{1} Police and Criminal Evidence Act. {\fontspec{DejaVu Sans}◇} \textit{}}{}{}{}%
\par%
\entry{pale}{/peɪl/}{ম্লান}{\small{\textsf{\textit{adjective, verb}}} \\{\fontspec{DejaVu Sans}▪ }\textsf{\textit{adjective}}\\ \textbf{1} Light in colour or shade; containing little colour or pigment. {\fontspec{DejaVu Sans}◇} \textit{choose pale floral patterns for walls} \colorBulletS{SYN} light, light{-}coloured, pastel, neutral, light{-}toned, muted, subtle, soft, low{-}key, restrained \textbf{2} Inferior or unimpressive. {\fontspec{DejaVu Sans}◇} \textit{the new cheese is a pale imitation of continental cheeses} \colorBulletS{SYN} inferior, poor, feeble, weak, insipid, wishy{-}washy, vapid, bland, puny, flat, inadequate, ineffectual, ineffective, half{-}hearted \\{\fontspec{DejaVu Sans}▪ }\textsf{\textit{verb}}\\ \textbf{1} Become pale in one's face from shock or fear. {\fontspec{DejaVu Sans}◇} \textit{I paled at the thought of what she might say} \colorBulletS{SYN} go white, turn white, become pale, grow pale, turn pale, blanch, blench, lose colour \textbf{2} Seem or become less important. {\fontspec{DejaVu Sans}◇} \textit{all else pales by comparison} \colorBulletS{SYN} decrease in importance, lose significance, pale into insignificance}{}{}{ \colorBullet{ORIGIN} Middle English from Old French pale, from Latin pallidus; the verb is from Old French palir.}%
\par%
\entry{pale}{/peɪl/}{ম্লান}{ \textsf{\textit{noun}}\ \textbf{1} A wooden stake or post used with others to form a fence. {\fontspec{DejaVu Sans}◇} \textit{} \colorBulletS{SYN} stake, post, pole, paling, picket, upright \textbf{2} An area within determined bounds, or subject to a particular jurisdiction. {\fontspec{DejaVu Sans}◇} \textit{The 3 major English Lords whose estates were within the Pale continued to exist, and formed alliances with the neighbouring Irish and became very powerful.} \textbf{3} A broad vertical stripe down the middle of a shield. {\fontspec{DejaVu Sans}◇} \textit{A narrow pale is more likely if it is uncharged, that is, if it does not have other objects placed on it.}}{}{}{ \colorBullet{ORIGIN} Middle English from Old French pal, from Latin palus ‘stake’.}%
\par%
\entry{pamper}{/ˈpampə/}{লাই দেত্তয়া}{ \textsf{\textit{verb}}\ \textbf{1} Indulge with every attention, comfort, and kindness; spoil. {\fontspec{DejaVu Sans}◇} \textit{famous people just love being pampered} \colorBulletS{SYN} spoil, indulge, overindulge, cosset, mollycoddle, coddle, baby, pet, wait on someone hand and foot, cater to someone's every whim, feather{-}bed, wrap in cotton wool, overparent}{}{}{ \colorBullet{ORIGIN} Late Middle English (in the sense ‘cram with food’): probably of Low German or Dutch origin; compare with German dialect pampfen ‘cram, gorge’; perhaps related to pap.}%
\par%
\entry{panacea}{/ˌpanəˈsiːə/}{সর্বব্যাধিহর ঔষধ}{ \textsf{\textit{noun}}\ \textbf{1} A solution or remedy for all difficulties or diseases. {\fontspec{DejaVu Sans}◇} \textit{the panacea for all corporate ills} \colorBulletS{SYN} universal cure, cure{-}all, cure for all ills, universal remedy, sovereign remedy, heal{-}all, nostrum, elixir, wonder drug, perfect solution, magic formula, magic bullet}{}{}{ \colorBullet{ORIGIN} Mid 16th century via Latin from Greek panakeia, from panakēs ‘all{-}healing’, from pan ‘all’ + akos ‘remedy’.}%
\par%
\entry{pang}{/paŋ/}{আকস্মিক তীব্র বেদনা}{ \textsf{\textit{noun}}\ \textbf{1} A sudden sharp pain or painful emotion. {\fontspec{DejaVu Sans}◇} \textit{Lindsey experienced a sharp pang of guilt} \colorBulletS{SYN} pain, sharp pain, shooting pain, twinge, stab, spasm, ache, cramp}{}{}{ \colorBullet{ORIGIN} Late 15th century perhaps an alteration of prong.}%
\par%
\entry{pang}{/paŋ/}{আকস্মিক তীব্র বেদনা}{ \textsf{\textit{adjective}}\ \textbf{1} Crammed or densely packed. {\fontspec{DejaVu Sans}◇} \textit{pang full of meat and bread}}{}{}{ \colorBullet{ORIGIN} Mid 16th century origin unknown.}%
\par%
\entry{panic}{/ˈpanɪk/}{আতঙ্ক}{\small{\textsf{\textit{noun, verb}}} \\{\fontspec{DejaVu Sans}▪ }\textsf{\textit{noun}}\\ \textbf{1} Sudden uncontrollable fear or anxiety, often causing wildly unthinking behaviour. {\fontspec{DejaVu Sans}◇} \textit{she hit him in panic} \colorBulletS{SYN} alarm, anxiety, nervousness, fear, fright, trepidation, dread, terror, horror, agitation, hysteria, consternation, perturbation, dismay, disquiet, apprehension, apprehensiveness \\{\fontspec{DejaVu Sans}▪ }\textsf{\textit{verb}}\\ \textbf{1} Feel or cause to feel panic. {\fontspec{DejaVu Sans}◇} \textit{the crowd panicked and stampeded for the exit} \colorBulletS{SYN} be alarmed, be scared, be nervous, be afraid, overreact, become panic{-}stricken, take fright, be filled with fear, be terrified, be agitated, be hysterical, lose one's nerve, be perturbed, get overwrought, get worked up, fall to pieces, go to pieces, lose control, fall apart}{}{}{ \colorBullet{ORIGIN} Early 17th century from French panique, from modern Latin panicus, from Greek panikos, from the name of the god Pan, noted for causing terror, to whom woodland noises were attributed.}%
\par%
\entry{panic}{/ˈpanɪk/}{আতঙ্ক}{ \textsf{\textit{noun}}\ \textbf{1} A cereal and fodder grass of a group including millet. {\fontspec{DejaVu Sans}◇} \textit{}}{}{}{ \colorBullet{ORIGIN} Late Middle English from Latin panicum, from panus ‘ear of millet’ (literally ‘thread wound on a bobbin’), based on Greek pēnos ‘web’, pēnion ‘bobbin’.}%
\par%
\entry{pants}{/pan(t)s/}{প্যান্ট}{ \textsf{\textit{plural noun}}\ \textbf{1} Underpants or knickers. {\fontspec{DejaVu Sans}◇} \textit{} \colorBulletS{SYN} underpants, briefs, Y{-}fronts, boxer shorts, boxers, long johns, knickers, French knickers, bikini briefs \textbf{2} Trousers. {\fontspec{DejaVu Sans}◇} \textit{corduroy pants} \colorBulletS{SYN} trousers \textbf{3} Rubbish; nonsense. {\fontspec{DejaVu Sans}◇} \textit{he thought we were going to be absolute pants} \colorBulletS{SYN} substandard, poor, inferior, second{-}rate, second{-}class, unsatisfactory, inadequate, unacceptable, not up to scratch, not up to par, deficient, imperfect, defective, faulty, shoddy, amateurish, careless, negligent}{}{}{ \colorBullet{ORIGIN} Mid 19th century abbreviation of pantaloons (see pantaloon).}%
\par%
\entry{papaya}{/pəˈpʌɪə/}{পেঁপে}{ \textsf{\textit{noun}}\ \textbf{1} A tropical fruit shaped like an elongated melon, with edible orange flesh and small black seeds. {\fontspec{DejaVu Sans}◇} \textit{} \textbf{2}  {\fontspec{DejaVu Sans}◇} \textit{}}{}{}{ \colorBullet{ORIGIN} Late 16th century from Spanish and Portuguese (see pawpaw).}%
\par%
\entry{par}{/pɑː/}{1. the established value of the monetary unit of one country expressed in terms of the monetary unit of another country using the same metal as the standard of value 2. common level}{\small{\textsf{\textit{noun, verb}}} \\{\fontspec{DejaVu Sans}▪ }\textsf{\textit{noun}}\\ \textbf{1} The number of strokes a first{-}class player should normally require for a particular hole or course. {\fontspec{DejaVu Sans}◇} \textit{Woosnam had advanced from his overnight position of three under par} \textbf{2} The face value of a share or other security, as distinct from its market value. {\fontspec{DejaVu Sans}◇} \textit{the 9 per cent unsecured loan stock is redeemable at par} \\{\fontspec{DejaVu Sans}▪ }\textsf{\textit{verb}}\\ \textbf{1} Play (a hole) in par. {\fontspec{DejaVu Sans}◇} \textit{he calmly parred the 17th}}{}{1. Judged the recording to be on a par with previous ones. 2. A partition storyteller par excellence (about kuldip nayar)}{ \colorBullet{ORIGIN} Late 16th century (in the sense ‘equality of value or standing’): from Latin, ‘equal’, also ‘equality’. The golf term dates from the late 19th century.}%
\par%
\entry{par}{/pɑː/}{1. the established value of the monetary unit of one country expressed in terms of the monetary unit of another country using the same metal as the standard of value 2. common level}{ \textsf{\textit{noun}}\ \textbf{1} A paragraph. {\fontspec{DejaVu Sans}◇} \textit{fifteen pars on the front page}}{}{1. Judged the recording to be on a par with previous ones. 2. A partition storyteller par excellence (about kuldip nayar)}{ \colorBullet{ORIGIN} Mid 19th century abbreviation.}%
\par%
\entry{parade}{/pəˈreɪd/}{}{\small{\textsf{\textit{noun, verb}}} \\{\fontspec{DejaVu Sans}▪ }\textsf{\textit{noun}}\\ \textbf{1} A public procession, especially one celebrating a special day or event. {\fontspec{DejaVu Sans}◇} \textit{a St George's Day parade} \colorBulletS{SYN} procession, march, cavalcade, motorcade, carcade, cortège, ceremony, spectacle, display, pageant, concours, file, train, column \textbf{2} A public square or promenade. {\fontspec{DejaVu Sans}◇} \textit{we were walking along South Parade} \colorBulletS{SYN} promenade, walk, walkway, esplanade, mall \textbf{3} A parade ground. {\fontspec{DejaVu Sans}◇} \textit{} \\{\fontspec{DejaVu Sans}▪ }\textsf{\textit{verb}}\\ \textbf{1} (of troops) assemble for a formal inspection or ceremonial occasion. {\fontspec{DejaVu Sans}◇} \textit{the recruits were due to parade that day} \textbf{2} Display (someone or something) while marching or moving around a place. {\fontspec{DejaVu Sans}◇} \textit{they paraded national flags}}{}{}{ \colorBullet{ORIGIN} Mid 17th century from French, literally ‘a showing’, from Spanish parada and Italian parata, based on Latin parare ‘prepare, furnish’.}%
\par%
\entry{paradigm}{/ˈparədʌɪm/}{দৃষ্টান্ত}{ \textsf{\textit{noun}}\ \textbf{1} A typical example or pattern of something; a pattern or model. {\fontspec{DejaVu Sans}◇} \textit{society's paradigm of the ‘ideal woman’} \colorBulletS{SYN} specimen, sample, exemplar, exemplification, instance, case, representative case, typical case, case in point, illustration \textbf{2} A set of linguistic items that form mutually exclusive choices in particular syntactic roles. {\fontspec{DejaVu Sans}◇} \textit{English determiners form a paradigm: we can say ‘a book’ or ‘his book’ but not ‘a his book’} \textbf{3} (in the traditional grammar of Latin, Greek, and other inflected languages) a table of all the inflected forms of a particular verb, noun, or adjective, serving as a model for other words of the same conjugation or declension. {\fontspec{DejaVu Sans}◇} \textit{}}{}{}{ \colorBullet{ORIGIN} Late 15th century via late Latin from Greek paradeigma, from paradeiknunai ‘show side by side’, from para{-} ‘beside’ + deiknunai ‘to show’.}%
\par%
\entry{paraffin}{/ˈparəfɪn/}{}{ \textsf{\textit{noun}}\ \textbf{1} old{-}fashioned term for alkane {\fontspec{DejaVu Sans}◇} \textit{}}{}{}{ \colorBullet{ORIGIN} Mid 19th century from German, from Latin parum ‘little’ + affinis ‘related’ (from its low reactivity).}%
\par%
\entry{paralyzing}{/ˈperəˌlīziNG/}{}{ \textsf{\textit{adjective}}\ \textbf{1} Causing a person or part of the body to become partly or wholly incapable of movement. {\fontspec{DejaVu Sans}◇} \textit{the snake's paralyzing venom}}{}{}{}%
\par%
\entry{particularly}{/pəˈtɪkjʊləli/}{}{ \textsf{\textit{adverb}}\ \textbf{1} To a higher degree than is usual or average. {\fontspec{DejaVu Sans}◇} \textit{I don't particularly want to be reminded of that time} \colorBulletS{SYN} especially, specially, very, extremely, exceptionally, singularly, peculiarly, distinctly, unusually, extraordinarily, extra, uncommonly, uniquely, remarkably, strikingly, outstandingly, amazingly, incredibly, awfully, terribly, really, notably, markedly, decidedly, surprisingly, conspicuously \textbf{2} So as to give special emphasis to a point; specifically. {\fontspec{DejaVu Sans}◇} \textit{he particularly asked that I should help you} \colorBulletS{SYN} specifically, explicitly, expressly, in particular, especially, specially}{}{I think it's safe to say that we've all done some things we are not particularly proud of.}{}%
\par%
\entry{parting}{/ˈpɑːtɪŋ/}{ছাড়াছাড়ি}{ \textsf{\textit{noun}}\ \textbf{1} The action of leaving or being separated from someone. {\fontspec{DejaVu Sans}◇} \textit{they exchanged a few words on parting} \colorBulletS{SYN} farewell, leave{-}taking, goodbye, adieu, departure, leaving, going, going away \textbf{2} The action of dividing something into parts. {\fontspec{DejaVu Sans}◇} \textit{the parting of the Red Sea} \colorBulletS{SYN} division, dividing, separation, separating, splitting, breaking apart, breaking up, severance, disjoining, detachment, partition, partitioning \textbf{3} A line of scalp revealed in a person's hair by combing the hair away in opposite directions on either side. {\fontspec{DejaVu Sans}◇} \textit{his hair was dark, with a side parting}}{}{}{ \colorBullet{ORIGIN} A point at which two people must separate or at which a decision must be taken.}%
\par%
\entry{parting ways}{}{বিভাজন উপায়}{\small{\textsf{\textit{}}}}{}{}{}%
\par%
\entry{patriot}{/ˈpatrɪət/}{দেশভক্ত}{ \textsf{\textit{noun}}\ \textbf{1} A person who vigorously supports their country and is prepared to defend it against enemies or detractors. {\fontspec{DejaVu Sans}◇} \textit{a true patriot} \colorBulletS{SYN} nationalist, loyalist \textbf{2}  {\fontspec{DejaVu Sans}◇} \textit{}}{}{}{ \colorBullet{ORIGIN} Late 16th century from French patriote, from late Latin patriota ‘fellow countryman’, from Greek patriōtēs, from patrios ‘of one's fathers’, from patris ‘fatherland’.}%
\par%
\entry{patriotic}{/patrɪˈɒtɪk/}{স্বদেশপ্রেমী}{ \textsf{\textit{adjective}}\ \textbf{1} Having or expressing devotion to and vigorous support for one's country. {\fontspec{DejaVu Sans}◇} \textit{today's game will be played before a fiercely patriotic crowd} \colorBulletS{SYN} nationalist, nationalistic, loyalist, loyal}{}{}{ \colorBullet{ORIGIN} Mid 17th century via late Latin from Greek patriōtikos ‘relating to a fellow countryman’ (see patriot).}%
\par%
\entry{patronize}{/ˈpatrənʌɪz/}{পিঠ চাপড়ান}{ \textsf{\textit{verb}}\ \textbf{1} Treat in a way that is apparently kind or helpful but that betrays a feeling of superiority. {\fontspec{DejaVu Sans}◇} \textit{she was determined not to be put down or patronized} \colorBulletS{SYN} treat condescendingly, treat with condescension, condescend to, look down on, talk down to, put down, humiliate, treat like a child, treat as inferior, treat with disdain, treat contemptuously, treat scornfully, be snobbish to, look down one's nose at \textbf{2} Frequent (a shop, restaurant, or other establishment) as a customer. {\fontspec{DejaVu Sans}◇} \textit{restaurants and bars regularly patronized by the stars were often crowded with paparazzi} \colorBulletS{SYN} do business with, buy from, shop at, be a customer of, be a client of, bring custom to, bring trade to, deal with, trade with}{}{}{}%
\par%
\entry{paved}{/peɪvd/}{বাঁধানো}{ \textsf{\textit{adjective}}\ \textbf{1} (of a piece of ground) covered with flat stones or bricks; laid with paving. {\fontspec{DejaVu Sans}◇} \textit{a paved courtyard}}{}{}{}%
\par%
\entry{pavement}{/ˈpeɪvm(ə)nt/}{ফুটপাথ}{ \textsf{\textit{noun}}\ \textbf{1} A raised paved or asphalted path for pedestrians at the side of a road. {\fontspec{DejaVu Sans}◇} \textit{he fell and hit his head on the pavement} \colorBulletS{SYN} footpath, paved path, pedestrian way, walkway, footway}{}{}{ \colorBullet{ORIGIN} Middle English from Old French, from Latin pavimentum ‘trodden down floor’, from pavire ‘beat, tread down’.}%
\par%
\entry{paving}{/ˈpeɪvɪŋ/}{পাকা রাস্তা}{ \textsf{\textit{noun}}\ \textbf{1} A surface made up of flat stones laid in a pattern. {\fontspec{DejaVu Sans}◇} \textit{weeds had forced their way up through the cracked paving}}{}{}{}%
\par%
\entry{peculiar}{/pɪˈkjuːlɪə/}{অদ্ভুত}{\small{\textsf{\textit{adjective, noun}}} \\{\fontspec{DejaVu Sans}▪ }\textsf{\textit{adjective}}\\ \textbf{1} Different to what is normal or expected; strange. {\fontspec{DejaVu Sans}◇} \textit{he gave her some very peculiar looks} \colorBulletS{SYN} strange, unusual, odd, funny, curious, bizarre, weird, uncanny, queer, unexpected, unfamiliar, abnormal, atypical, anomalous, untypical, different, out of the ordinary, out of the way \textbf{2} Particular; special. {\fontspec{DejaVu Sans}◇} \textit{any attempt to explicate the theme is bound to run into peculiar difficulties} \colorBulletS{SYN} distinctive, characteristic, distinct, different, individual, individualistic, distinguishing, typical, special, specific, representative, unique, idiosyncratic, personal, private, essential, natural \\{\fontspec{DejaVu Sans}▪ }\textsf{\textit{noun}}\\ \textbf{1} A parish or church exempt from the jurisdiction of the diocese in which it lies, and subject to the direct jurisdiction of the monarch or an archbishop. {\fontspec{DejaVu Sans}◇} \textit{deans and canons of royal peculiars, notably Westminster Abbey and Windsor}}{}{}{ \colorBullet{ORIGIN} Late Middle English (in the sense ‘particular’): from Latin peculiaris ‘of private property’, from peculium ‘property’, from pecu ‘cattle’ (cattle being private property). The sense ‘strange’ dates from the early 17th century.}%
\par%
\entry{peddler}{/ˈpɛdlə/}{হকার; দালাল}{ \textsf{\textit{noun}}\ \textbf{1} A person who sells illegal drugs or stolen goods. {\fontspec{DejaVu Sans}◇} \textit{a drug peddler} \colorBulletS{SYN} trafficker, dealer \textbf{2} variant spelling of pedlar {\fontspec{DejaVu Sans}◇} \textit{}}{}{}{ \colorBullet{ORIGIN} See pedal}%
\par%
\entry{pedigree}{/ˈpɛdɪɡriː/}{বংশতালিকা}{\small{\textsf{\textit{adjective, noun}}} \\{\fontspec{DejaVu Sans}▪ }\textsf{\textit{adjective}}\\ \textbf{1} (of an animal) pure{-}bred. {\fontspec{DejaVu Sans}◇} \textit{pedigree cats} \colorBulletS{SYN} pure{-}bred, thoroughbred, pure, pure{-}blooded, full{-}blooded \\{\fontspec{DejaVu Sans}▪ }\textsf{\textit{noun}}\\ \textbf{1} The record of descent of an animal, showing it to be pure{-}bred. {\fontspec{DejaVu Sans}◇} \textit{they are looking for animals with pedigrees} \textbf{2} The recorded ancestry or lineage of a person or family. {\fontspec{DejaVu Sans}◇} \textit{with a pedigree equal to many of the gentry} \colorBulletS{SYN} ancestry, descent, lineage, line, line of descent, genealogy, family tree, extraction, derivation, origin, heritage, parentage, paternity, birth, family, dynasty, house, race, strain, stock, breed, blood, bloodline, history, background, roots}{}{}{ \colorBullet{ORIGIN} Late Middle English from Anglo{-}Norman French pé de grue ‘crane's foot’, a mark used to denote succession in pedigrees.}%
\par%
\entry{peek}{/piːk/}{উঁকি}{\small{\textsf{\textit{noun, verb}}} \\{\fontspec{DejaVu Sans}▪ }\textsf{\textit{noun}}\\ \textbf{1} A quick or furtive look. {\fontspec{DejaVu Sans}◇} \textit{she sneaked a peek at the map} \colorBulletS{SYN} secret look, sly look, stealthy look, sneaky look, peep, glance, glimpse, brief look, hurried look, quick look, look, peer \\{\fontspec{DejaVu Sans}▪ }\textsf{\textit{verb}}\\ \textbf{1} Look quickly or furtively. {\fontspec{DejaVu Sans}◇} \textit{faces peeked from behind twitched curtains} \colorBulletS{SYN} peep, have a peep, have a peek, take a secret look, spy, take a sly look, take a stealthy look, sneak a look, glance, cast a brief look, look hurriedly, look, peer}{}{}{ \colorBullet{ORIGIN} Late Middle English pike, pyke, of unknown origin.}%
\par%
\entry{peel}{/piːl/}{খোসা}{\small{\textsf{\textit{noun, verb}}} \\{\fontspec{DejaVu Sans}▪ }\textsf{\textit{noun}}\\ \textbf{1} The outer covering or rind of a fruit or vegetable. {\fontspec{DejaVu Sans}◇} \textit{pieces of potato peel} \colorBulletS{SYN} rind, skin, covering, zest \textbf{2} An act of exfoliating dead skin in the cosmetic treatment of microdermabrasion. {\fontspec{DejaVu Sans}◇} \textit{} \\{\fontspec{DejaVu Sans}▪ }\textsf{\textit{verb}}\\ \textbf{1} Remove the outer covering or skin from (a fruit, vegetable, or prawn) {\fontspec{DejaVu Sans}◇} \textit{she watched him peel an apple with deliberate care} \colorBulletS{SYN} pare, skin, take the rind off, take the skin off, strip, shave, trim, flay \textbf{2} Remove a thin outer covering or part. {\fontspec{DejaVu Sans}◇} \textit{I peeled off the tissue paper} \colorBulletS{SYN} trim, trim off, peel off, pare, strip, strip off, shave, shave off, remove, take off, flay \textbf{3} (of a surface or object) lose parts of its outer layer or covering in small strips or pieces. {\fontspec{DejaVu Sans}◇} \textit{the walls are peeling} \colorBulletS{SYN} flake, flake off, peel off, come off in layers, come off in strips}{}{}{ \colorBullet{ORIGIN} Middle English (in the sense ‘to plunder’): variant of dialect pill, from Latin pilare ‘to strip hair from’, from pilus ‘hair’. The differentiation of peel and pill may have been by association with the French verbs peler ‘to peel’ and piller ‘to pillage’.}%
\par%
\entry{peel}{/piːl/}{খোসা}{ \textsf{\textit{noun}}\ \textbf{1} A flat implement like a shovel, especially one used by a baker for carrying loaves or similar items of food into or out of an oven. {\fontspec{DejaVu Sans}◇} \textit{a wooden pizza peel}}{}{}{ \colorBullet{ORIGIN} Late Middle English from Old French pele, from Latin pala, from the base of pangere ‘fasten’.}%
\par%
\entry{peel}{/piːl/}{খোসা}{ \textsf{\textit{noun}}\ \textbf{1} A small square defensive tower of a kind built in the 16th century in the border counties of England and Scotland. {\fontspec{DejaVu Sans}◇} \textit{}}{}{}{ \colorBullet{ORIGIN} Probably short for synonymous peel{-}house peel from Anglo{-}Norman French pel ‘stake, palisade’, from Latin palus ‘stake’.}%
\par%
\entry{peel}{/piːl/}{খোসা}{ \textsf{\textit{verb}}\ \textbf{1} Send (another player's ball) through a hoop. {\fontspec{DejaVu Sans}◇} \textit{the better players are capable of peeling a ball through two or three hoops}}{}{}{ \colorBullet{ORIGIN} Late 19th century from the name of Walter H. Peel, founder of the All England Croquet Association, a leading exponent of the practice.}%
\par%
\entry{peep}{/piːp/}{উঁকি}{\small{\textsf{\textit{noun, verb}}} \\{\fontspec{DejaVu Sans}▪ }\textsf{\textit{noun}}\\ \textbf{1} A quick or furtive look. {\fontspec{DejaVu Sans}◇} \textit{Jonathan took a little peep at his watch} \colorBulletS{SYN} quick look, brief look, sly look, stealthy look, sneaky look, peek, glance, glimpse, look, peer \\{\fontspec{DejaVu Sans}▪ }\textsf{\textit{verb}}\\ \textbf{1} Look quickly and furtively at something, especially through a narrow opening. {\fontspec{DejaVu Sans}◇} \textit{his door was ajar and she couldn't resist peeping in} \colorBulletS{SYN} look quickly, cast a brief look, take a secret look, spy, take a sly look, take a stealthy look, sneak a look, peek, have a peek, glance, peer}{}{}{ \colorBullet{ORIGIN} Late 15th century symbolic; compare with peek.}%
\par%
\entry{peep}{/piːp/}{উঁকি}{\small{\textsf{\textit{noun, verb}}} \\{\fontspec{DejaVu Sans}▪ }\textsf{\textit{noun}}\\ \textbf{1} A feeble, high{-}pitched sound made by a young bird or mammal. {\fontspec{DejaVu Sans}◇} \textit{} \colorBulletS{SYN} cheep, chirp, chirrup, tweet, twitter, chirr, pipe, piping, warble, squeak, chatter \textbf{2} A small sandpiper or similar wading bird. {\fontspec{DejaVu Sans}◇} \textit{the peeps have returned to Fundy} \textbf{3} A group of chickens. {\fontspec{DejaVu Sans}◇} \textit{a peep of chickens pecking and scratching around the gate} \\{\fontspec{DejaVu Sans}▪ }\textsf{\textit{verb}}\\ \textbf{1} Make a brief, high{-}pitched sound. {\fontspec{DejaVu Sans}◇} \textit{Don peeped on his whistle} \colorBulletS{SYN} cheep, chirp, chirrup, tweet, twitter, chirr, squeak}{}{}{ \colorBullet{ORIGIN} Late Middle English imitative; compare with cheep.}%
\par%
\entry{peninsula}{/pɪˈnɪnsjʊlə/}{উপদ্বীপ}{ \textsf{\textit{noun}}\ \textbf{1} A piece of land almost surrounded by water or projecting out into a body of water. {\fontspec{DejaVu Sans}◇} \textit{} \colorBulletS{SYN} cape, promontory, point, head, headland, foreland, ness, horn, bill, bluff, limb}{}{}{ \colorBullet{ORIGIN} Mid 16th century from Latin paeninsula, from paene ‘almost’ + insula ‘island’.}%
\par%
\entry{penis envy}{}{}{\small{\textsf{\textit{}}}}{}{Supposed envy of the male's possession of a penis, postulated by freud to account for some aspects of female behaviour (notably the castration complex) but controversial among modern theorists.}{}%
\par%
\entry{penpal}{}{কলম বন্ধু}{\small{\textsf{\textit{}}}}{}{}{}%
\par%
\entry{perimeter}{/pəˈrɪmɪtə/}{ঘের}{ \textsf{\textit{noun}}\ \textbf{1} The continuous line forming the boundary of a closed geometrical figure. {\fontspec{DejaVu Sans}◇} \textit{the perimeter of a rectangle} \colorBulletS{SYN} circumference, outside, outer edge \textbf{2} An instrument for measuring the extent and characteristics of a person's field of vision. {\fontspec{DejaVu Sans}◇} \textit{}}{}{}{ \colorBullet{ORIGIN} Late Middle English via Latin from Greek perimetros, based on peri{-} ‘around’ + metron ‘measure’.}%
\par%
\entry{perish}{/ˈpɛrɪʃ/}{বিনষ্ট করা}{ \textsf{\textit{verb}}\ \textbf{1} Die, especially in a violent or sudden way. {\fontspec{DejaVu Sans}◇} \textit{a great part of his army perished of hunger and disease} \colorBulletS{SYN} die, lose one's life, be killed, fall, expire, meet one's death, be lost, lay down one's life, breathe one's last, draw one's last breath, pass away, go the way of all flesh, give up the ghost, go to glory, meet one's maker, go to one's last resting place, cross the great divide \textbf{2} (of rubber, food, etc.) lose its normal qualities; rot or decay. {\fontspec{DejaVu Sans}◇} \textit{an abandoned tyre whose rubber had perished} \colorBulletS{SYN} go bad, go off, spoil, rot, go mouldy, moulder, putrefy, decay, decompose \textbf{3} Be suffering from extreme cold. {\fontspec{DejaVu Sans}◇} \textit{I was perished with cold before the end of the day}}{}{}{ \colorBullet{ORIGIN} Middle English from Old French periss{-}, lengthened stem of perir, from Latin perire ‘pass away’, from per{-} ‘through, completely’ + ire ‘go’.}%
\par%
\entry{perpetrate}{/ˈpəːpɪtreɪt/}{}{ \textsf{\textit{verb}}\ \textbf{1} Carry out or commit (a harmful, illegal, or immoral action) {\fontspec{DejaVu Sans}◇} \textit{a crime has been perpetrated against a sovereign state} \colorBulletS{SYN} commit, carry out, perform, execute, do, effect, bring about, be guilty of, be to blame for, be responsible for, accomplish, inflict, wreak}{}{}{ \colorBullet{ORIGIN} Mid 16th century from Latin perpetrat{-} ‘performed’, from the verb perpetrare, from per{-} ‘to completion’ + patrare ‘bring about’. In Latin the act perpetrated might be good or bad; in English the verb was first used in the statutes referring to crime, hence the negative association.}%
\par%
\entry{perpetrator}{/ˈpəːpətreɪtə/}{অপরাধী}{ \textsf{\textit{noun}}\ \textbf{1} A person who carries out a harmful, illegal, or immoral act. {\fontspec{DejaVu Sans}◇} \textit{the perpetrators of this horrific crime must be brought to justice}}{}{Malaysia wants the perpetrators of atrocities against the rohingyas be tried immediately at the international criminal court (icc).}{}%
\par%
\entry{persist}{/pəˈsɪst/}{জিদ করা}{ \textsf{\textit{verb}}\ \textbf{1} Continue in an opinion or course of action in spite of difficulty or opposition. {\fontspec{DejaVu Sans}◇} \textit{the minority of drivers who persist in drinking} \colorBulletS{SYN} persevere, continue, carry on, go on, keep at it, keep on, keep going, keep it up, not give up, be persistent, be determined, follow something through, see something through, show determination, press ahead, press on, plod on, plough on, stay with something, not take no for an answer}{}{}{ \colorBullet{ORIGIN} Mid 16th century from Latin persistere, from per{-} ‘through, steadfastly’ + sistere ‘to stand’.}%
\par%
\entry{persistent}{/pəˈsɪst(ə)nt/}{অধ্যবসায়ী}{ \textsf{\textit{adjective}}\ \textbf{1} Continuing firmly or obstinately in an opinion or course of action in spite of difficulty or opposition. {\fontspec{DejaVu Sans}◇} \textit{one of the government's most persistent critics} \colorBulletS{SYN} tenacious, persevering, determined, resolute, purposeful, dogged, single{-}minded, tireless, indefatigable, pertinacious, patient, diligent, assiduous, sedulous, unflagging, untiring, unwavering, insistent, importunate, relentless, unrelenting \textbf{2} Continuing to exist or occur over a prolonged period. {\fontspec{DejaVu Sans}◇} \textit{persistent rain will affect many areas} \colorBulletS{SYN} continuing, constant, continual, continuous, non{-}stop, lasting, never{-}ending, steady, uninterrupted, unbroken, interminable, incessant, unceasing, endless, unending, perpetual, unremitting, unrelenting, relentless, unrelieved, sustained \textbf{3} (of a part of an animal or plant, such as a horn, leaf, etc.) remaining attached instead of falling off in the normal manner. {\fontspec{DejaVu Sans}◇} \textit{}}{}{}{}%
\par%
\entry{personnel}{/pəːsəˈnɛl/}{কর্মিবৃন্দ}{ \textsf{\textit{plural noun}}\ \textbf{1} People employed in an organization or engaged in an organized undertaking such as military service. {\fontspec{DejaVu Sans}◇} \textit{many of the personnel involved require training} \colorBulletS{SYN} staff, employees, workforce, workers, labour force, manpower, human resources, people, men and women}{}{}{ \colorBullet{ORIGIN} Early 19th century from French (adjective used as a noun), contrasted with matériel ‘equipment or materials used in an organization or undertaking’.}%
\par%
\entry{pesticide}{/ˈpɛstɪsʌɪd/}{কীটনাশক}{ \textsf{\textit{noun}}\ \textbf{1} A substance used for destroying insects or other organisms harmful to cultivated plants or to animals. {\fontspec{DejaVu Sans}◇} \textit{}}{}{}{}%
\par%
\entry{petition}{/pɪˈtɪʃ(ə)n/}{আবেদন}{\small{\textsf{\textit{noun, verb}}} \\{\fontspec{DejaVu Sans}▪ }\textsf{\textit{noun}}\\ \textbf{1} A formal written request, typically one signed by many people, appealing to authority in respect of a particular cause. {\fontspec{DejaVu Sans}◇} \textit{she was asked to sign a petition against plans to build on the local playing fields} \colorBulletS{SYN} appeal, round robin, list of protesters, list of signatures \\{\fontspec{DejaVu Sans}▪ }\textsf{\textit{verb}}\\ \textbf{1} Present a petition to (an authority) in respect of a particular cause. {\fontspec{DejaVu Sans}◇} \textit{the organization is petitioning the EU for a moratorium on the patent} \colorBulletS{SYN} appeal to, request, ask, call on, entreat, beg, implore, beseech, plead with, make a plea to, pray, apply to, solicit, press, urge, adjure, present one's suit to, importune}{}{}{ \colorBullet{ORIGIN} Middle English from Latin petitio(n{-}), from petit{-} ‘aimed at, sought, laid claim to’, from the verb petere.}%
\par%
\entry{philanthropy}{/fɪˈlanθrəpi/}{মানবপ্রীতি}{ \textsf{\textit{noun}}\ \textbf{1} The desire to promote the welfare of others, expressed especially by the generous donation of money to good causes. {\fontspec{DejaVu Sans}◇} \textit{he acquired a considerable fortune and was noted for his philanthropy} \colorBulletS{SYN} benevolence, generosity, humanitarianism, public{-}spiritedness, altruism, social conscience, social concern, charity, charitableness, brotherly love, fellow feeling, magnanimity, munificence, liberality, largesse, open{-}handedness, bountifulness, beneficence, benignity, unselfishness, selflessness, humanity, kindness, kind{-}heartedness, big{-}heartedness, compassion, humaneness}{}{}{ \colorBullet{ORIGIN} Early 17th century via late Latin from Greek philanthrōpia, from philanthrōpos ‘man{-}loving’ (see philanthrope).}%
\par%
\entry{pick}{/pɪk/}{গোছগাছ}{\small{\textsf{\textit{noun, verb}}} \\{\fontspec{DejaVu Sans}▪ }\textsf{\textit{noun}}\\ \textbf{1} An act or the right of selecting something from a number of alternatives. {\fontspec{DejaVu Sans}◇} \textit{take your pick from our extensive menu} \colorBulletS{SYN} choice, selection, option, decision \textbf{2} An act of blocking or screening a defensive player from the ball handler. {\fontspec{DejaVu Sans}◇} \textit{} \\{\fontspec{DejaVu Sans}▪ }\textsf{\textit{verb}}\\ \textbf{1} Detach and remove (a flower, fruit, or vegetable) from where it is growing. {\fontspec{DejaVu Sans}◇} \textit{I went to pick some flowers for Jenny's room} \colorBulletS{SYN} harvest, gather, gather in, collect, take in, pluck, pull, dig, crop, reap, bring home \textbf{2} Choose (someone or something) from a number of alternatives. {\fontspec{DejaVu Sans}◇} \textit{maybe I picked the wrong career} \colorBulletS{SYN} choose, select, pick out, single out, include, hand{-}pick, decide on, settle on, fix on \textbf{3} Repeatedly pull at something with one's fingers. {\fontspec{DejaVu Sans}◇} \textit{the old woman was picking at the sheet} \textbf{4} Pluck the strings of (a guitar or banjo) {\fontspec{DejaVu Sans}◇} \textit{people were singing and picking guitars} \colorBulletS{SYN} strum, twang, thrum, pluck, finger}{ \colorBullet{OTHER} picking up}{Picking up steam}{ \colorBullet{ORIGIN} Middle English (earlier as pike, which continues in dialect use): of unknown origin. Compare with Dutch pikken ‘pick, peck’, and German picken ‘peck, puncture’, also with French piquer ‘to prick’.}%
\par%
\entry{pick}{/pɪk/}{গোছগাছ}{ \textsf{\textit{noun}}\ \textbf{1} A tool consisting of a long handle set at right angles in the middle of a curved iron or steel bar with a point at one end and a chisel edge or point at the other, used for breaking up hard ground or rock. {\fontspec{DejaVu Sans}◇} \textit{} \textbf{2} An instrument for picking. {\fontspec{DejaVu Sans}◇} \textit{an ebony hair pick}}{ \colorBullet{OTHER} picking up}{Picking up steam}{ \colorBullet{ORIGIN} Middle English variant of pike.}%
\par%
\entry{pier}{/pɪə/}{জেটি}{ \textsf{\textit{noun}}\ \textbf{1} A platform on pillars projecting from the shore into the sea, typically incorporating entertainment arcades and places to eat. {\fontspec{DejaVu Sans}◇} \textit{} \textbf{2} A long, narrow structure projecting from an airport terminal, giving passengers access to an aircraft. {\fontspec{DejaVu Sans}◇} \textit{} \textbf{3} A solid support designed to sustain vertical pressure. {\fontspec{DejaVu Sans}◇} \textit{}}{}{}{ \colorBullet{ORIGIN} Middle English from medieval Latin pera, of unknown origin.}%
\par%
\entry{pile}{/pʌɪl/}{গাদা}{\small{\textsf{\textit{noun, verb}}} \\{\fontspec{DejaVu Sans}▪ }\textsf{\textit{noun}}\\ \textbf{1} A heap of things laid or lying one on top of another. {\fontspec{DejaVu Sans}◇} \textit{he placed the books in a neat pile} \colorBulletS{SYN} heap, stack, mound, pyramid, mass, quantity, bundle, clump, bunch, jumble \textbf{2} A large imposing building or group of buildings. {\fontspec{DejaVu Sans}◇} \textit{a Victorian Gothic pile} \colorBulletS{SYN} mansion, stately home, hall, manor, big house, manor house, country house, castle, palace \textbf{3} A series of plates of dissimilar metals laid one on another alternately to produce an electric current. {\fontspec{DejaVu Sans}◇} \textit{} \textbf{4} A nuclear reactor. {\fontspec{DejaVu Sans}◇} \textit{} \\{\fontspec{DejaVu Sans}▪ }\textsf{\textit{verb}}\\ \textbf{1} Place (things) one on top of the other. {\fontspec{DejaVu Sans}◇} \textit{she piled all the groceries on the counter} \colorBulletS{SYN} heap, heap up, stack, stack up, make a heap of, make a pile of, make a stack of \textbf{2} (of a group of people) get into or out of (a vehicle) in a disorganized manner. {\fontspec{DejaVu Sans}◇} \textit{ten of us piled into the minibus} \colorBulletS{SYN} crowd, climb, charge, tumble, stream, flock, flood, pack, squeeze, push, shove, jostle, elbow, crush, jam}{}{}{ \colorBullet{ORIGIN} Late Middle English from Old French, from Latin pila ‘pillar, pier’.}%
\par%
\entry{pile}{/pʌɪl/}{গাদা}{\small{\textsf{\textit{noun, verb}}} \\{\fontspec{DejaVu Sans}▪ }\textsf{\textit{noun}}\\ \textbf{1} A heavy stake or post driven vertically into the bed of a river, soft ground, etc., to support the foundations of a superstructure. {\fontspec{DejaVu Sans}◇} \textit{} \colorBulletS{SYN} post, rod, pillar, column, support, foundation, piling \textbf{2} A triangular charge or ordinary formed by two lines meeting at an acute angle, usually pointing down from the top of the shield. {\fontspec{DejaVu Sans}◇} \textit{} \\{\fontspec{DejaVu Sans}▪ }\textsf{\textit{verb}}\\ \textbf{1} Strengthen or support (a structure) with piles. {\fontspec{DejaVu Sans}◇} \textit{an earlier bridge may have been piled}}{}{}{ \colorBullet{ORIGIN} Old English pīl ‘dart, arrow’, also ‘pointed stake’, of Germanic origin; related to Dutch pijl and German Pfeil, from Latin pilum ‘(heavy) javelin’.}%
\par%
\entry{pile}{/pʌɪl/}{গাদা}{ \textsf{\textit{noun}}\ \textbf{1} The soft projecting surface of a carpet or a fabric such as velvet or flannel, consisting of many small threads. {\fontspec{DejaVu Sans}◇} \textit{the thick pile of the new rugs} \colorBulletS{SYN} fibres, threads, loops}{}{}{ \colorBullet{ORIGIN} Middle English (in the sense ‘downy feather’): from Latin pilus ‘hair’. The current sense dates from the mid 16th century.}%
\par%
\entry{pilgrim}{/ˈpɪlɡrɪm/}{নবাগত}{\small{\textsf{\textit{noun, verb}}} \\{\fontspec{DejaVu Sans}▪ }\textsf{\textit{noun}}\\ \textbf{1} A person who journeys to a sacred place for religious reasons. {\fontspec{DejaVu Sans}◇} \textit{} \colorBulletS{SYN} visitor to a shrine, worshipper, devotee, believer, traveller, wayfarer, crusader \textbf{2}  {\fontspec{DejaVu Sans}◇} \textit{This is a monument dedicated in 1910 to commemorate the first landing of the Pilgrims in 1620 at Provincetown, where they wrote and signed the Mayflower Compact.} \\{\fontspec{DejaVu Sans}▪ }\textsf{\textit{verb}}\\ \textbf{1} Travel or wander like a pilgrim. {\fontspec{DejaVu Sans}◇} \textit{he pilgrimed to his old sporting places}}{}{}{ \colorBullet{ORIGIN} Middle English from Provençal pelegrin, from Latin peregrinus ‘foreign’ (see peregrine).}%
\par%
\entry{pilgrimage}{/ˈpɪlɡrɪmɪdʒ/}{তীর্থযাত্রা}{\small{\textsf{\textit{noun, verb}}} \\{\fontspec{DejaVu Sans}▪ }\textsf{\textit{noun}}\\ \textbf{1} A pilgrim's journey. {\fontspec{DejaVu Sans}◇} \textit{he wanted to go on a pilgrimage to Santiago de Compostela} \colorBulletS{SYN} religious journey, holy expedition, crusade, mission, trip, journey, excursion \\{\fontspec{DejaVu Sans}▪ }\textsf{\textit{verb}}\\ \textbf{1} Go on a pilgrimage. {\fontspec{DejaVu Sans}◇} \textit{thousands pilgrimage there every year}}{}{}{ \colorBullet{ORIGIN} Middle English from Provençal pelegrinage, from pelegrin (see pilgrim).}%
\par%
\entry{pimp}{/pɪmp/}{কুটনি; দালাল; ধান্দাবাজ}{\small{\textsf{\textit{noun, verb}}} \\{\fontspec{DejaVu Sans}▪ }\textsf{\textit{noun}}\\ \textbf{1} A man who controls prostitutes and arranges clients for them, taking a percentage of their earnings in return. {\fontspec{DejaVu Sans}◇} \textit{} \colorBulletS{SYN} procurer, procuress \textbf{2} A telltale or informer. {\fontspec{DejaVu Sans}◇} \textit{But he was put in a cell with two Hollywood labour leader pimps.} \colorBulletS{SYN} informant \\{\fontspec{DejaVu Sans}▪ }\textsf{\textit{verb}}\\ \textbf{1} Act as a pimp. {\fontspec{DejaVu Sans}◇} \textit{he was a good{-}looking guy, and some said he pimped on the side} \textbf{2} Make (something) more showy or impressive. {\fontspec{DejaVu Sans}◇} \textit{he pimped up the car with spoilers and twin{-}spoke 18{-}inch alloys} \textbf{3} Inform on. {\fontspec{DejaVu Sans}◇} \textit{they'd pimp on you as soon as look at you} \colorBulletS{SYN} break one's promise to, be disloyal to, be unfaithful to, break faith with, play someone false, fail, let down}{}{}{ \colorBullet{ORIGIN} Late 16th century of unknown origin.}%
\par%
\entry{pine}{/pʌɪn/}{}{ \textsf{\textit{noun}}\ \textbf{1}  {\fontspec{DejaVu Sans}◇} \textit{} \textbf{2} A pineapple. {\fontspec{DejaVu Sans}◇} \textit{}}{ \colorBullet{OTHER} pining for}{}{ \colorBullet{ORIGIN} Old English, from Latin pinus, reinforced in Middle English by Old French pin.}%
\par%
\entry{pine}{/pʌɪn/}{}{ \textsf{\textit{verb}}\ \textbf{1} Suffer a mental and physical decline, especially because of a broken heart. {\fontspec{DejaVu Sans}◇} \textit{she thinks I am pining away from love} \colorBulletS{SYN} languish, decline, go into a decline, lose strength, weaken, waste away, dwindle, wilt, wither, fade, flag, sicken, droop, brood, mope, moon}{ \colorBullet{OTHER} pining for}{}{ \colorBullet{ORIGIN} Old English pīnian ‘(cause to) suffer’, of Germanic origin; related to Dutch pijnen, German peinen ‘experience pain’, also to obsolete pine ‘punishment’; ultimately based on Latin poena ‘punishment’.}%
\par%
\entry{piss}{/pɪs/}{}{\small{\textsf{\textit{noun, verb}}} \\{\fontspec{DejaVu Sans}▪ }\textsf{\textit{noun}}\\ \textbf{1} Urine. {\fontspec{DejaVu Sans}◇} \textit{} \textbf{2} Alcoholic drink, especially beer. {\fontspec{DejaVu Sans}◇} \textit{we'll need 70 cans of piss for the trip} \\{\fontspec{DejaVu Sans}▪ }\textsf{\textit{verb}}\\ \textbf{1} Urinate. {\fontspec{DejaVu Sans}◇} \textit{} \colorBulletS{SYN} pass water, go to the loo, go to the toilet, go to the lavatory, relieve oneself}{ \colorBullet{OTHER} piss off: কেটে পরা}{}{ \colorBullet{ORIGIN} Middle English from Old French pisser, probably of imitative origin.}%
\par%
\entry{pit}{/pɪt/}{কূপ}{\small{\textsf{\textit{noun, verb}}} \\{\fontspec{DejaVu Sans}▪ }\textsf{\textit{noun}}\\ \textbf{1} A large hole in the ground. {\fontspec{DejaVu Sans}◇} \textit{} \colorBulletS{SYN} hole, ditch, trench, trough, hollow, shaft, mineshaft, excavation, cavity, pothole, rut \textbf{2} A hollow or indentation in a surface. {\fontspec{DejaVu Sans}◇} \textit{} \colorBulletS{SYN} pockmark, pock, mark, hollow, indentation, depression, dent, dint, concavity, dimple \textbf{3} An area at the side of a track where racing cars are serviced and refuelled. {\fontspec{DejaVu Sans}◇} \textit{he had a flat tyre when he came into the pits} \textbf{4} An orchestra pit. {\fontspec{DejaVu Sans}◇} \textit{} \textbf{5} A part of the floor of a stock exchange in which a particular stock or commodity is traded. {\fontspec{DejaVu Sans}◇} \textit{pooled commodity funds liquidated positions in the corn and soybean pits} \textbf{6} An enclosure in which animals are made to fight. {\fontspec{DejaVu Sans}◇} \textit{a bear pit} \textbf{7} A person's bed. {\fontspec{DejaVu Sans}◇} \textit{} \textbf{8} A person's armpit. {\fontspec{DejaVu Sans}◇} \textit{} \\{\fontspec{DejaVu Sans}▪ }\textsf{\textit{verb}}\\ \textbf{1} Set someone or something in conflict or competition with. {\fontspec{DejaVu Sans}◇} \textit{you'll get the chance to pit your wits against the world champions} \colorBulletS{SYN} set against, match against, put in opposition to, put in competition with, measure against \textbf{2} Make a hollow or indentation in the surface of. {\fontspec{DejaVu Sans}◇} \textit{rain poured down, pitting the bare earth} \colorBulletS{SYN} make holes in, make hollows in, hole, dent, indent, depress, dint, pothole \textbf{3} Drive a racing car into the pits for fuel or maintenance. {\fontspec{DejaVu Sans}◇} \textit{he pitted on lap 36 with sudden engine trouble}}{}{}{ \colorBullet{ORIGIN} Old English pytt, of West Germanic origin; related to Dutch put and German Pfütze, based on Latin puteus ‘well, shaft’.}%
\par%
\entry{pit}{/pɪt/}{কূপ}{\small{\textsf{\textit{noun, verb}}} \\{\fontspec{DejaVu Sans}▪ }\textsf{\textit{noun}}\\ \textbf{1} The stone of a fruit. {\fontspec{DejaVu Sans}◇} \textit{} \colorBulletS{SYN} stone, pip, seed \\{\fontspec{DejaVu Sans}▪ }\textsf{\textit{verb}}\\ \textbf{1} Remove the pit from (fruit). {\fontspec{DejaVu Sans}◇} \textit{}}{}{}{ \colorBullet{ORIGIN} Mid 19th century apparently from Dutch; related to pith.}%
\par%
\entry{pivotal}{/ˈpɪvətl/}{কেঁদ্রগত}{ \textsf{\textit{adjective}}\ \textbf{1} Of crucial importance in relation to the development or success of something else. {\fontspec{DejaVu Sans}◇} \textit{Japan's pivotal role in the world economy} \colorBulletS{SYN} central, crucial, vital, critical, focal, essential, key, significant, important, determining, decisive, deciding \textbf{2} Fixed on or as if on a pivot. {\fontspec{DejaVu Sans}◇} \textit{a sliding or pivotal motion}}{}{}{}%
\par%
\entry{placate}{/pləˈkeɪt/}{শান্ত করা}{ \textsf{\textit{verb}}\ \textbf{1} Make (someone) less angry or hostile. {\fontspec{DejaVu Sans}◇} \textit{they attempted to placate the students with promises} \colorBulletS{SYN} appease, placate, pacify, mollify, propitiate, assuage, calm down, soothe, humour, reconcile, disarm, win over, make peace with}{}{}{ \colorBullet{ORIGIN} Late 17th century from Latin placat{-} ‘appeased’, from the verb placare.}%
\par%
\entry{plague}{/pleɪɡ/}{জ্বালাতন করা; প্লেগ রোগ}{\small{\textsf{\textit{noun, verb}}} \\{\fontspec{DejaVu Sans}▪ }\textsf{\textit{noun}}\\ \textbf{1} A contagious bacterial disease characterized by fever and delirium, typically with the formation of buboes (bubonic plague) and sometimes infection of the lungs (pneumonic plague). {\fontspec{DejaVu Sans}◇} \textit{} \textbf{2} An unusually large number of insects or animals infesting a place and causing damage. {\fontspec{DejaVu Sans}◇} \textit{a plague of locusts} \colorBulletS{SYN} huge number, infestation, epidemic, invasion, influx, swarm, multitude, host \textbf{3} A thing causing trouble or irritation. {\fontspec{DejaVu Sans}◇} \textit{staff theft is usually the plague of restaurants} \colorBulletS{SYN} bane, curse, scourge, affliction, blight, cancer, canker \\{\fontspec{DejaVu Sans}▪ }\textsf{\textit{verb}}\\ \textbf{1} Cause continual trouble or distress to. {\fontspec{DejaVu Sans}◇} \textit{he has been plagued by ill health} \colorBulletS{SYN} afflict, bedevil, cause suffering to, torture, torment, trouble, beset, dog, curse, rack}{}{}{ \colorBullet{ORIGIN} Late Middle English Latin plaga ‘stroke, wound’, probably from Greek (Doric dialect) plaga, from a base meaning ‘strike’.}%
\par%
\entry{plaintiff}{/ˈpleɪntɪf/}{বাদী}{ \textsf{\textit{noun}}\ \textbf{1} A person who brings a case against another in a court of law. {\fontspec{DejaVu Sans}◇} \textit{the plaintiff commenced an action for damages} \colorBulletS{SYN} litigator, opponent in law, opponent, contestant, contender, disputant, plaintiff, claimant, complainant, petitioner, appellant, respondent, party, interest, defendant, accused}{}{}{ \colorBullet{ORIGIN} Late Middle English from Old French plaintif ‘plaintive’ (used as a noun). The {-}f ending has come down through Law French; the word was originally the same as plaintive.}%
\par%
\entry{plantain}{/ˈplantɪn/}{কলা}{ \textsf{\textit{noun}}\ \textbf{1} A low{-}growing plant that typically has a rosette of leaves and a slender green flower spike, widely growing as a weed in lawns. {\fontspec{DejaVu Sans}◇} \textit{}}{}{}{ \colorBullet{ORIGIN} Late Middle English from Old French, from Latin plantago, plantagin{-}, from planta ‘sole of the foot’ (because of its broad prostrate leaves).}%
\par%
\entry{plantain}{/ˈplanteɪn/}{কলা}{ \textsf{\textit{noun}}\ \textbf{1} A banana containing high levels of starch and little sugar, which is harvested green and widely used as a cooked vegetable in the tropics. {\fontspec{DejaVu Sans}◇} \textit{} \textbf{2} The plant which bears the plantain. {\fontspec{DejaVu Sans}◇} \textit{}}{}{}{ \colorBullet{ORIGIN} Mid 16th century from Spanish plá(n)tano, probably by assimilation of a South American word to the Spanish plá(n)tano ‘plane tree’.}%
\par%
\entry{plausible}{/ˈplɔːzɪb(ə)l/}{বিশ্বাসযোগ্য}{ \textsf{\textit{adjective}}\ \textbf{1} (of an argument or statement) seeming reasonable or probable. {\fontspec{DejaVu Sans}◇} \textit{a plausible explanation} \colorBulletS{SYN} credible, reasonable, believable, likely, feasible, probable, tenable, possible, conceivable, imaginable, within the bounds of possibility, convincing, persuasive, cogent, sound, rational, logical, acceptable, thinkable}{}{}{ \colorBullet{ORIGIN} Mid 16th century (also in the sense ‘deserving applause or approval’): from Latin plausibilis, from plaus{-} ‘applauded’, from the verb plaudere.}%
\par%
\entry{plea}{/pliː/}{অজুহাত}{ \textsf{\textit{noun}}\ \textbf{1} A request made in an urgent and emotional manner. {\fontspec{DejaVu Sans}◇} \textit{he made a dramatic plea for disarmament} \colorBulletS{SYN} appeal, entreaty, supplication, petition, prayer \textbf{2} A formal statement by or on behalf of a defendant or prisoner, stating guilt or innocence in response to a charge, offering an allegation of fact, or claiming that a point of law should apply. {\fontspec{DejaVu Sans}◇} \textit{he changed his plea to not guilty}}{}{}{ \colorBullet{ORIGIN} Middle English (in the sense ‘lawsuit’): from Old French plait, plaid ‘agreement, discussion’, from Latin placitum ‘a decree’, neuter past participle of placere ‘to please’.}%
\par%
\entry{please}{/pliːz/}{অনুগ্রহ}{\small{\textsf{\textit{adverb, verb}}} \\{\fontspec{DejaVu Sans}▪ }\textsf{\textit{adverb}}\\ \textbf{1} Used in polite requests or questions. {\fontspec{DejaVu Sans}◇} \textit{please address letters to the Editor} \colorBulletS{SYN} if you please, if you wouldn't mind, if you would be so good \\{\fontspec{DejaVu Sans}▪ }\textsf{\textit{verb}}\\ \textbf{1} Cause to feel happy and satisfied. {\fontspec{DejaVu Sans}◇} \textit{he arranged a fishing trip to please his son} \colorBulletS{SYN} nice, agreeable, pleasant, pleasurable, satisfying, gratifying, welcome, good, acceptable, to one's liking, enjoyable, entertaining, amusing, delightful, fine \textbf{2} Take only one's own wishes into consideration in deciding how to act or proceed. {\fontspec{DejaVu Sans}◇} \textit{this is the first time in ages that I can just please myself}}{}{}{ \colorBullet{ORIGIN} Middle English from Old French plaisir ‘to please’, from Latin placere.}%
\par%
\entry{pleasing}{/ˈpliːzɪŋ/}{আনন্দদায়ক}{ \textsf{\textit{adjective}}\ \textbf{1} Satisfying or appealing. {\fontspec{DejaVu Sans}◇} \textit{the pleasing austerity of the surroundings}}{}{}{}%
\par%
\entry{pleasure}{/ˈplɛʒə/}{পরিতোষ}{\small{\textsf{\textit{adjective, noun, verb}}} \\{\fontspec{DejaVu Sans}▪ }\textsf{\textit{adjective}}\\ \textbf{1} Used or intended for entertainment rather than business. {\fontspec{DejaVu Sans}◇} \textit{pleasure boats} \\{\fontspec{DejaVu Sans}▪ }\textsf{\textit{noun}}\\ \textbf{1} A feeling of happy satisfaction and enjoyment. {\fontspec{DejaVu Sans}◇} \textit{she smiled with pleasure at being praised} \colorBulletS{SYN} happiness, delight, joy, gladness, rapture, glee, satisfaction, gratification, fulfilment, contentment, contentedness, enjoyment, amusement \\{\fontspec{DejaVu Sans}▪ }\textsf{\textit{verb}}\\ \textbf{1} Give sexual enjoyment or satisfaction to. {\fontspec{DejaVu Sans}◇} \textit{tell me what will pleasure you}}{}{}{ \colorBullet{ORIGIN} Late Middle English from Old French plaisir ‘to please’ (used as a noun). The second syllable was altered under the influence of abstract nouns ending in {-}ure, such as measure.}%
\par%
\entry{pluck}{/plʌk/}{টানিয়া সরাইয়া ফেলা}{\small{\textsf{\textit{noun, verb}}} \\{\fontspec{DejaVu Sans}▪ }\textsf{\textit{noun}}\\ \textbf{1} Spirited and determined courage. {\fontspec{DejaVu Sans}◇} \textit{it must have taken a lot of pluck to walk along a path marked ‘Danger’} \colorBulletS{SYN} courage, bravery, nerve, pluckiness, boldness, courageousness, braveness, backbone, spine, daring, spirit, intrepidness, intrepidity, fearlessness, mettle, determination, fortitude, resolve, resolution, stout{-}heartedness, hardihood, dauntlessness, valour, doughtiness, heroism, audacity \textbf{2} The heart, liver, and lungs of an animal as food. {\fontspec{DejaVu Sans}◇} \textit{Put the pluck into cold salted water, boil, then skim and simmer for 1 hour.} \\{\fontspec{DejaVu Sans}▪ }\textsf{\textit{verb}}\\ \textbf{1} Take hold of (something) and quickly remove it from its place. {\fontspec{DejaVu Sans}◇} \textit{she plucked a blade of grass} \colorBulletS{SYN} remove, pick off, pick, pull, pull off, pull out, extract, take, take off \textbf{2} Quickly or suddenly remove someone from a dangerous or unpleasant situation. {\fontspec{DejaVu Sans}◇} \textit{the baby was plucked from a grim orphanage} \textbf{3} Sound (a musical instrument or its strings) with one's finger or a plectrum. {\fontspec{DejaVu Sans}◇} \textit{she picked up her guitar and plucked it idly} \colorBulletS{SYN} strum, pick, thrum, twang, plunk, finger}{}{}{ \colorBullet{ORIGIN} Late Old English ploccian, pluccian, of Germanic origin; related to Flemish plokken; probably from the base of Old French (es)peluchier ‘to pluck’. Sense 1 of the noun is originally boxers' slang.}%
\par%
\entry{plumber}{/ˈplʌmə/}{সীসক}{ \textsf{\textit{noun}}\ \textbf{1} A person who fits and repairs the pipes, fittings, and other apparatus of water supply, sanitation, or heating systems. {\fontspec{DejaVu Sans}◇} \textit{}}{}{}{ \colorBullet{ORIGIN} Late Middle English (originally denoting a person dealing in and working with lead): from Old French plommier, from Latin plumbarius, from plumbum ‘lead’.}%
\par%
\entry{plunge}{/plʌn(d)ʒ/}{নিমজ্জন}{\small{\textsf{\textit{noun, verb}}} \\{\fontspec{DejaVu Sans}▪ }\textsf{\textit{noun}}\\ \textbf{1} An act of jumping or diving into water. {\fontspec{DejaVu Sans}◇} \textit{fanatics went straight from the hot room to take a cold plunge} \colorBulletS{SYN} jump, dive \\{\fontspec{DejaVu Sans}▪ }\textsf{\textit{verb}}\\ \textbf{1} Jump or dive quickly and energetically. {\fontspec{DejaVu Sans}◇} \textit{our little daughters whooped as they plunged into the sea} \colorBulletS{SYN} jump, dive, hurl oneself, throw oneself, fling oneself, launch oneself, catapult oneself, cast oneself, pitch oneself \textbf{2} Push or thrust quickly. {\fontspec{DejaVu Sans}◇} \textit{he plunged his hands into his pockets} \colorBulletS{SYN} thrust, stick, ram, drive, jab, stab, push, shove, force, sink}{}{}{ \colorBullet{ORIGIN} Late Middle English from Old French plungier ‘thrust down’, based on Latin plumbum ‘lead, plummet’.}%
\par%
\entry{plus{-}size}{}{অতিরিক্ত আকার}{ \textsf{\textit{adjective}}\ \textbf{1} Denoting or relating to clothes of a size larger than those found in standard ranges. {\fontspec{DejaVu Sans}◇} \textit{a new line of plus{-}size bathing suits}}{}{Plus{-}size clothing}{}%
\par%
\entry{ply}{/plʌɪ/}{অটলভাবে কাজ করা}{ \textsf{\textit{noun}}\ \textbf{1} A thickness or layer of a folded or laminated material. {\fontspec{DejaVu Sans}◇} \textit{tiles that have a black PVC ply in the lamination} \colorBulletS{SYN} layer, thickness, strand, sheet, leaf, fold, insertion \textbf{2} short for plywood {\fontspec{DejaVu Sans}◇} \textit{} \textbf{3} (in game theory) the number of levels at which branching occurs in a tree of possible outcomes, typically corresponding to the number of moves ahead (in chess strictly half{-}moves ahead) considered by a computer program. {\fontspec{DejaVu Sans}◇} \textit{This creates a ‘tree’ of analysis with moves branching at each ply.}}{ \colorBullet{OTHER} ply on}{}{ \colorBullet{ORIGIN} Late Middle English (in the sense ‘fold’): from French pli ‘fold’, from the verb plier, from Latin plicare ‘to fold’.}%
\par%
\entry{ply}{/plʌɪ/}{অটলভাবে কাজ করা}{ \textsf{\textit{verb}}\ \textbf{1} Work steadily with (a tool) {\fontspec{DejaVu Sans}◇} \textit{a tailor delicately plying his needle} \colorBulletS{SYN} use, wield, work, work with, employ, operate, utilize, manipulate, handle \textbf{2} (of a vessel or vehicle) travel regularly over a route, typically for commercial purposes. {\fontspec{DejaVu Sans}◇} \textit{ferries ply across a strait to the island} \colorBulletS{SYN} go regularly, travel regularly, make regular journeys, travel, go back and forth, shuttle, commute \textbf{3} Provide someone with (food or drink) in a continuous or insistent way. {\fontspec{DejaVu Sans}◇} \textit{she plied me with tea and scones} \colorBulletS{SYN} provide, supply, keep supplying, lavish, shower, regale, load, heap}{ \colorBullet{OTHER} ply on}{}{ \colorBullet{ORIGIN} Late Middle English shortening of apply.}%
\par%
\entry{poach}{/pəʊtʃ/}{চোরাশিকার}{ \textsf{\textit{verb}}\ \textbf{1} Cook (an egg) without its shell in or over boiling water. {\fontspec{DejaVu Sans}◇} \textit{a breakfast of poached egg and grilled bacon}}{}{}{ \colorBullet{ORIGIN} Late Middle English from Old French pochier, earlier in the sense ‘enclose in a bag’, from poche ‘bag, pocket’.}%
\par%
\entry{poach}{/pəʊtʃ/}{চোরাশিকার}{ \textsf{\textit{verb}}\ \textbf{1} Illegally hunt or catch (game or fish) on land that is not one's own or in contravention of official protection. {\fontspec{DejaVu Sans}◇} \textit{20 tigers are thought to have been poached from national parks} \colorBulletS{SYN} hunt illegally, catch illegally, kill illegally, trap illegally, plunder \textbf{2} (of an animal) trample or cut up (turf) with its hoofs. {\fontspec{DejaVu Sans}◇} \textit{zero{-}grazing saves the fields from poaching}}{}{}{ \colorBullet{ORIGIN} Early 16th century (in the sense ‘push roughly together’): apparently related to poke; sense 1 is perhaps partly from French pocher ‘enclose in a bag’ (see poach).}%
\par%
\entry{ponder}{/ˈpɒndə/}{চিন্তা করা}{ \textsf{\textit{verb}}\ \textbf{1} Think about (something) carefully, especially before making a decision or reaching a conclusion. {\fontspec{DejaVu Sans}◇} \textit{I pondered the question of what clothes to wear for the occasion} \colorBulletS{SYN} think about, give thought to, consider, review, reflect on, mull over, contemplate, study, meditate on, muse on, deliberate about, cogitate on, dwell on, brood on, brood over, ruminate about, ruminate on, chew over, puzzle over, speculate about, weigh up, turn over in one's mind}{}{}{ \colorBullet{ORIGIN} Middle English (in the sense ‘appraise, judge the worth of’): from Old French ponderer ‘consider’, from Latin ponderare ‘weigh, reflect on’, from pondus, ponder{-} ‘weight’.}%
\par%
\entry{porch}{/pɔːtʃ/}{বারান্দা}{ \textsf{\textit{noun}}\ \textbf{1} A covered shelter projecting in front of the entrance of a building. {\fontspec{DejaVu Sans}◇} \textit{the north porch of Hereford Cathedral} \colorBulletS{SYN} vestibule, foyer, entrance, entrance hall, entry, portal, portico, lobby, anteroom}{}{}{ \colorBullet{ORIGIN} Middle English from Old French porche, from Latin porticus ‘colonnade’, from porta ‘passage’.}%
\par%
\entry{porpoise}{/ˈpɔːpəs/}{শুশুক}{\small{\textsf{\textit{noun, verb}}} \\{\fontspec{DejaVu Sans}▪ }\textsf{\textit{noun}}\\ \textbf{1} A small toothed whale with a low triangular dorsal fin and a blunt rounded snout. {\fontspec{DejaVu Sans}◇} \textit{} \\{\fontspec{DejaVu Sans}▪ }\textsf{\textit{verb}}\\ \textbf{1} Move through the water like a porpoise, alternately rising above it and submerging. {\fontspec{DejaVu Sans}◇} \textit{the boat began to porpoise badly}}{}{}{ \colorBullet{ORIGIN} Middle English from Old French porpois, based on Latin porcus ‘pig’ + piscis ‘fish’, rendering earlier porcus marinus ‘sea hog’.}%
\par%
\entry{possession}{/pəˈzɛʃ(ə)n/}{দখল}{ \textsf{\textit{noun}}\ \textbf{1} The state of having, owning, or controlling something. {\fontspec{DejaVu Sans}◇} \textit{she had taken possession of the sofa} \colorBulletS{SYN} ownership, proprietorship, control, hands, keeping, care, custody, charge, hold, title, guardianship \textbf{2} Something that is owned or possessed. {\fontspec{DejaVu Sans}◇} \textit{I had no money or possessions} \colorBulletS{SYN} asset, thing, article, item owned, chattel \textbf{3} The state of being controlled by a demon or spirit. {\fontspec{DejaVu Sans}◇} \textit{they said prayers to protect the people inside the hall from demonic possession}}{}{}{ \colorBullet{ORIGIN} Middle English from Old French, from Latin possessio(n{-}), from the verb possidere (see possess).}%
\par%
\entry{postulate}{/ˈpɒstjʊleɪt/}{স্বীকার্য}{\small{\textsf{\textit{noun, verb}}} \\{\fontspec{DejaVu Sans}▪ }\textsf{\textit{noun}}\\ \textbf{1} A thing suggested or assumed as true as the basis for reasoning, discussion, or belief. {\fontspec{DejaVu Sans}◇} \textit{perhaps the postulate of Babylonian influence on Greek astronomy is incorrect} \colorBulletS{SYN} hypothesis, thesis, conjecture, supposition, speculation, postulation, postulate, proposition, premise, surmise, assumption, presumption, presupposition, notion, guess, hunch, feeling, suspicion \\{\fontspec{DejaVu Sans}▪ }\textsf{\textit{verb}}\\ \textbf{1} Suggest or assume the existence, fact, or truth of (something) as a basis for reasoning, discussion, or belief. {\fontspec{DejaVu Sans}◇} \textit{his theory postulated a rotatory movement for hurricanes} \colorBulletS{SYN} put forward, suggest, advance, posit, hypothesize, take as a hypothesis, propose, assume, presuppose, suppose, presume, predicate, take for granted, theorize \textbf{2} (in ecclesiastical law) nominate or elect (someone) to an ecclesiastical office subject to the sanction of a higher authority. {\fontspec{DejaVu Sans}◇} \textit{the chapter was then allowed to postulate the bishop of Bath}}{}{}{ \colorBullet{ORIGIN} Late Middle English (in postulate (sense 2 of the verb)): from Latin postulat{-} ‘asked’, from the verb postulare.}%
\par%
\entry{pothole}{/ˈpɒthəʊl/}{গর্ত}{\small{\textsf{\textit{noun, verb}}} \\{\fontspec{DejaVu Sans}▪ }\textsf{\textit{noun}}\\ \textbf{1} A deep natural underground cave formed by the erosion of rock, especially by the action of water. {\fontspec{DejaVu Sans}◇} \textit{} \colorBulletS{SYN} cave, cavern, cavity, hollow, recess, alcove \textbf{2} A depression or hollow in a road surface caused by wear or subsidence. {\fontspec{DejaVu Sans}◇} \textit{he drove very cautiously over the potholes in the road} \colorBulletS{SYN} wheel track, furrow, groove, track, trough, ditch, trench, gutter, gouge, crack, hollow, hole, pothole, cavity, crater \\{\fontspec{DejaVu Sans}▪ }\textsf{\textit{verb}}\\ \textbf{1} Explore underground potholes as a pastime. {\fontspec{DejaVu Sans}◇} \textit{they went potholing in the Pennines} \colorBulletS{SYN} make holes in, make hollows in, hole, dent, indent, depress, dint, pothole}{}{}{ \colorBullet{ORIGIN} Early 19th century from Middle English pot ‘pit’ (perhaps of Scandinavian origin) + hole.}%
\par%
\entry{pragmatic}{/praɡˈmatɪk/}{রাষ্ট্রীয়}{ \textsf{\textit{adjective}}\ \textbf{1} Dealing with things sensibly and realistically in a way that is based on practical rather than theoretical considerations. {\fontspec{DejaVu Sans}◇} \textit{a pragmatic approach to politics} \colorBulletS{SYN} empirical, hands{-}on, pragmatic, real, actual, active, applied, experiential, experimental, non{-}theoretical, in the field}{}{}{ \colorBullet{ORIGIN} Late 16th century (in the senses ‘busy, interfering, conceited’): via Latin from Greek pragmatikos ‘relating to fact’, from pragma ‘deed’ (from the stem of prattein ‘do’). The current senses date from the mid 19th century.}%
\par%
\entry{praise}{/preɪz/}{প্রশংসা}{\small{\textsf{\textit{noun, verb}}} \\{\fontspec{DejaVu Sans}▪ }\textsf{\textit{noun}}\\ \textbf{1} The expression of approval or admiration for someone or something. {\fontspec{DejaVu Sans}◇} \textit{the audience was full of praise for the whole production} \colorBulletS{SYN} approval, acclaim, admiration, approbation, acclamation, plaudits, congratulations, commendation, applause, flattery, adulation \textbf{2} The expression of respect and gratitude as an act of worship. {\fontspec{DejaVu Sans}◇} \textit{give praise to God} \colorBulletS{SYN} honour, thanks, glory, glorification, worship, devotion, exaltation, adoration, veneration, reverence, tribute \\{\fontspec{DejaVu Sans}▪ }\textsf{\textit{verb}}\\ \textbf{1} Express warm approval or admiration of. {\fontspec{DejaVu Sans}◇} \textit{we can't praise Chris enough—he did a brilliant job} \colorBulletS{SYN} commend, express approval of, express admiration for, applaud, pay tribute to, speak highly of, eulogize, compliment, congratulate, celebrate, sing the praises of, praise to the skies, rave about, go into raptures about, heap praise on, wax lyrical about, say nice things about, make much of, pat on the back, take one's hat off to, throw bouquets at, lionize, admire, hail, cheer, flatter \textbf{2} Express one's respect and gratitude towards (a deity), especially in song. {\fontspec{DejaVu Sans}◇} \textit{we praise God for past blessings} \colorBulletS{SYN} worship, glorify, honour, exalt, adore, pay tribute to, pay homage to, give thanks to, venerate, reverence, hallow, bless}{}{}{ \colorBullet{ORIGIN} Middle English (also in the sense ‘set a price on, attach value to’): from Old French preisier ‘to prize, praise’, from late Latin pretiare, from Latin pretium ‘price’. Compare with prize.}%
\par%
\entry{precedent}{/ˈprɛsɪd(ə)nt/}{নজির}{\small{\textsf{\textit{adjective, noun}}} \\{\fontspec{DejaVu Sans}▪ }\textsf{\textit{adjective}}\\ \textbf{1} Preceding in time, order, or importance. {\fontspec{DejaVu Sans}◇} \textit{a precedent case} \colorBulletS{SYN} one{-}time, erstwhile, sometime, late, as was \\{\fontspec{DejaVu Sans}▪ }\textsf{\textit{noun}}\\ \textbf{1} An earlier event or action that is regarded as an example or guide to be considered in subsequent similar circumstances. {\fontspec{DejaVu Sans}◇} \textit{there are substantial precedents for using interactive media in training} \colorBulletS{SYN} model, exemplar, example, pattern, previous case, prior case, previous example, previous instance, prior example, prior instance}{}{}{ \colorBullet{ORIGIN} Late Middle English from Old French, literally ‘preceding’.}%
\par%
\entry{precipitation}{/prɪˌsɪpɪˈteɪʃ(ə)n/}{বৃষ্টিপাতের পরিমাণ}{ \textsf{\textit{noun}}\ \textbf{1} The action or process of precipitating a substance from a solution. {\fontspec{DejaVu Sans}◇} \textit{} \textbf{2} Rain, snow, sleet, or hail that falls to or condenses on the ground. {\fontspec{DejaVu Sans}◇} \textit{these convective processes produce cloud and precipitation} \colorBulletS{SYN} frozen rain, hailstones, sleet, precipitation \textbf{3} The fact or quality of acting suddenly and rashly. {\fontspec{DejaVu Sans}◇} \textit{Cora was already regretting her precipitation}}{}{}{ \colorBullet{ORIGIN} Late Middle English (denoting the action of falling or throwing down): from Latin praecipitatio(n{-}), from praecipitare ‘throw down or headlong’ (see precipitate).}%
\par%
\entry{precise}{/prɪˈsʌɪs/}{যথাযথ}{ \textsf{\textit{adjective}}\ \textbf{1} Marked by exactness and accuracy of expression or detail. {\fontspec{DejaVu Sans}◇} \textit{precise directions} \colorBulletS{SYN} exact, accurate, correct, error{-}free, pinpoint, specific, detailed, explicit, clear{-}cut, unambiguous, meticulous, close, strict, definite, particular, express}{}{}{ \colorBullet{ORIGIN} Late Middle English from Old French prescis, from Latin praecis{-} ‘cut short’, from the verb praecidere, from prae ‘in advance’ + caedere ‘to cut’.}%
\par%
\entry{precisely}{/prɪˈsʌɪsli/}{অবিকল}{ \textsf{\textit{adverb}}\ \textbf{1} In exact terms; without vagueness. {\fontspec{DejaVu Sans}◇} \textit{the guidelines are precisely defined} \colorBulletS{SYN} exhaustively, painstakingly, systematically, meticulously, rigorously, scrupulously, punctiliously, in detail}{}{}{}%
\par%
\entry{premise}{/ˈprɛmɪs/}{প্রতিজ্ঞা}{\small{\textsf{\textit{noun, verb}}} \\{\fontspec{DejaVu Sans}▪ }\textsf{\textit{noun}}\\ \textbf{1}  {\fontspec{DejaVu Sans}◇} \textit{if the premise is true, then the conclusion must be true} \colorBulletS{SYN} proposition, assumption, hypothesis, thesis, presupposition, postulation, postulate, supposition, presumption, surmise, conjecture, speculation, datum, argument, assertion, belief, thought \\{\fontspec{DejaVu Sans}▪ }\textsf{\textit{verb}}\\ \textbf{1} Base an argument, theory, or undertaking on. {\fontspec{DejaVu Sans}◇} \textit{the reforms were premised on our findings}}{}{}{ \colorBullet{ORIGIN} Late Middle English from Old French premisse, from medieval Latin praemissa (propositio) ‘(proposition) set in front’, from Latin praemittere, from prae ‘before’ + mittere ‘send’.}%
\par%
\entry{premises}{/ˈprɛmɪsɪz/}{প্রাঙ্গনে}{ \textsf{\textit{plural noun}}\ \textbf{1} A house or building, together with its land and outbuildings, occupied by a business or considered in an official context. {\fontspec{DejaVu Sans}◇} \textit{the company has moved to new premises} \colorBulletS{SYN} building, buildings, property, site, establishment, office, place}{}{}{}%
\par%
\entry{preposterous}{/prɪˈpɒst(ə)rəs/}{ভ্রান্ত}{ \textsf{\textit{adjective}}\ \textbf{1} Contrary to reason or common sense; utterly absurd or ridiculous. {\fontspec{DejaVu Sans}◇} \textit{a preposterous suggestion} \colorBulletS{SYN} absurd, ridiculous, foolish, stupid, ludicrous, farcical, laughable, comical, risible, hare{-}brained, asinine, inane, nonsensical, pointless, senseless, insane, unreasonable, irrational, illogical}{}{}{ \colorBullet{ORIGIN} Mid 16th century from Latin praeposterus ‘reversed, absurd’ (from prae ‘before’ + posterus ‘coming after’) + {-}ous.}%
\par%
\entry{press}{/prɛs/}{প্রেস}{\small{\textsf{\textit{noun, verb}}} \\{\fontspec{DejaVu Sans}▪ }\textsf{\textit{noun}}\\ \textbf{1} A device for applying pressure to something in order to flatten or shape it or to extract juice or oil. {\fontspec{DejaVu Sans}◇} \textit{a flower press} \textbf{2} A printing press. {\fontspec{DejaVu Sans}◇} \textit{} \colorBulletS{SYN} printing press, printing machine \textbf{3} Newspapers or journalists viewed collectively. {\fontspec{DejaVu Sans}◇} \textit{the incident was not reported in the press} \colorBulletS{SYN} the media, the newspapers, the papers, the news media, journalism, the newspaper world, the newspaper business, the print media, the fourth estate \textbf{4} An act of pressing something. {\fontspec{DejaVu Sans}◇} \textit{the system summons medical help at the press of a button} \textbf{5} An act of raising a weight to shoulder height and then gradually pushing it upwards above the head. {\fontspec{DejaVu Sans}◇} \textit{} \textbf{6} A large cupboard. {\fontspec{DejaVu Sans}◇} \textit{} \\{\fontspec{DejaVu Sans}▪ }\textsf{\textit{verb}}\\ \textbf{1} Move or cause to move into a position of contact with something by exerting continuous physical force. {\fontspec{DejaVu Sans}◇} \textit{he pressed his face to the glass} \colorBulletS{SYN} push, push down, press down, thumb, depress, bear down on, lean on, lower, pin, pinion, hold down, force, ram, thrust, cram, squeeze, compress, wedge \textbf{2} Apply pressure to (something) to flatten, shape, or smooth it, typically by ironing. {\fontspec{DejaVu Sans}◇} \textit{she pressed her nicest blouse} \colorBulletS{SYN} smooth, iron, smooth out, remove creases from, put creases in \textbf{3} Forcefully put forward (an opinion, claim, or course of action) {\fontspec{DejaVu Sans}◇} \textit{Rose did not press the point} \colorBulletS{SYN} plead, urge, advance insistently, file, prefer, lodge, tender, present, place, lay, submit, put forward \textbf{4} Raise (a specified weight) by lifting it to shoulder height and then gradually pushing it upwards above the head. {\fontspec{DejaVu Sans}◇} \textit{} \textbf{5} Try too hard to achieve distance with a shot, at the risk of inaccuracy. {\fontspec{DejaVu Sans}◇} \textit{This is not a good golf course to start pressing on.}}{}{}{ \colorBullet{ORIGIN} Middle English from Old French presse (noun), presser (verb), from Latin pressare ‘keep pressing’, frequentative of premere.}%
\par%
\entry{press}{/prɛs/}{প্রেস}{\small{\textsf{\textit{noun, verb}}} \\{\fontspec{DejaVu Sans}▪ }\textsf{\textit{noun}}\\ \textbf{1} A forcible enlistment of men, especially for the navy. {\fontspec{DejaVu Sans}◇} \textit{Any English{-}speaking, able{-}bodied, man on leave in a port might find himself swept up in the press.} \\{\fontspec{DejaVu Sans}▪ }\textsf{\textit{verb}}\\ \textbf{1} Put someone or something to a specified use, especially as a temporary or makeshift measure. {\fontspec{DejaVu Sans}◇} \textit{she was pressed into service as an interpreter} \textbf{2} Force (a man) to enlist in the army or navy. {\fontspec{DejaVu Sans}◇} \textit{At least a third had been pressed into the Navy.}}{}{}{ \colorBullet{ORIGIN} Late 16th century alteration (by association with press) of obsolete prest ‘pay given on enlistment, enlistment by such payment’, from Old French prest ‘loan, advance pay’, based on Latin praestare ‘provide’.}%
\par%
\entry{pretext}{/ˈpriːtɛkst/}{অজুহাত}{ \textsf{\textit{noun}}\ \textbf{1} A reason given in justification of a course of action that is not the real reason. {\fontspec{DejaVu Sans}◇} \textit{the rebels had the perfect pretext for making their move} \colorBulletS{SYN} excuse, false excuse, ostensible reason, alleged reason, plea, supposed grounds}{}{}{ \colorBullet{ORIGIN} Early 16th century from Latin praetextus ‘outward display’, from the verb praetexere ‘to disguise’, from prae ‘before’ + texere ‘weave’.}%
\par%
\entry{prevail}{/prɪˈveɪl/}{বোঝান}{ \textsf{\textit{verb}}\ \textbf{1} Prove more powerful or superior. {\fontspec{DejaVu Sans}◇} \textit{it is hard for logic to prevail over emotion} \colorBulletS{SYN} win, win out, win through, triumph, be victorious, be the victor, gain the victory, carry the day, carry all before one, finish first, come out ahead, come out on top, succeed, prove superior, conquer, overcome, achieve mastery, gain mastery, gain ascendancy \textbf{2} Persuade (someone) to do something. {\fontspec{DejaVu Sans}◇} \textit{she was prevailed upon to give an account of her work} \colorBulletS{SYN} persuade, induce, talk someone into, coax, convince, make, get, press someone into, win someone over, sway, bring someone round, argue someone into, urge, pressure someone into, pressurize someone into, bring pressure to bear on, coerce, influence, prompt}{}{}{ \colorBullet{ORIGIN} Late Middle English from Latin praevalere ‘have greater power’, from prae ‘before’ + valere ‘have power’.}%
\par%
\entry{prevalence}{/ˈprɛv(ə)l(ə)ns/}{প্রাদুর্ভাব}{ \textsf{\textit{noun}}\ \textbf{1} The fact or condition of being prevalent; commonness. {\fontspec{DejaVu Sans}◇} \textit{the prevalence of obesity in adults} \colorBulletS{SYN} commonness, currency, widespread presence, generality, pervasiveness, universality, extensiveness, ubiquity, ubiquitousness}{}{}{}%
\par%
\entry{prevalent}{/ˈprɛv(ə)l(ə)nt/}{প্রভাবশালী}{ \textsf{\textit{adjective}}\ \textbf{1} Widespread in a particular area or at a particular time. {\fontspec{DejaVu Sans}◇} \textit{the social ills prevalent in society today} \colorBulletS{SYN} widespread, prevailing, frequent, usual, common, general, universal, pervasive, extensive, ubiquitous, ordinary}{}{}{ \colorBullet{ORIGIN} Late 16th century from Latin praevalent{-} ‘having greater power’, from the verb praevalere (see prevail).}%
\par%
\entry{price}{/prʌɪs/}{}{\small{\textsf{\textit{noun, verb}}} \\{\fontspec{DejaVu Sans}▪ }\textsf{\textit{noun}}\\ \textbf{1} The amount of money expected, required, or given in payment for something. {\fontspec{DejaVu Sans}◇} \textit{land could be sold for a high price} \colorBulletS{SYN} cost, asking price, selling price, charge, fee, terms, payment, rate, fare, levy, toll, amount, sum, total, figure \textbf{2} An unwelcome experience or action undergone or done as a condition of achieving an objective. {\fontspec{DejaVu Sans}◇} \textit{the price of their success was an entire day spent in discussion} \colorBulletS{SYN} consequence, result, cost, toll, penalty, sacrifice, forfeit, forfeiture \\{\fontspec{DejaVu Sans}▪ }\textsf{\textit{verb}}\\ \textbf{1} Decide the amount required as payment for (something offered for sale) {\fontspec{DejaVu Sans}◇} \textit{the watches are priced at £55} \colorBulletS{SYN} fix the price of, set the price of, put a price on, cost, value, rate, evaluate, assess, estimate, appraise, assay \textbf{2} Discover or establish the price of (something for sale). {\fontspec{DejaVu Sans}◇} \textit{}}{}{Price hike: মূল্যবৃদ্ধি}{ \colorBullet{ORIGIN} Middle English the noun from Old French pris, from Latin pretium ‘value, reward’; the verb, a variant (by assimilation to the noun) of earlier prise ‘estimate the value of’ (see prize). Compare with praise.}%
\par%
\entry{pricey}{/ˈprʌɪsi/}{দামী}{ \textsf{\textit{adjective}}\ \textbf{1} Expensive. {\fontspec{DejaVu Sans}◇} \textit{boutiques selling pricey clothes} \colorBulletS{SYN} expensive, dear, costly, high{-}priced, high{-}cost, high{-}end, big{-}budget}{}{}{}%
\par%
\entry{pride}{/prʌɪd/}{গর্ব}{\small{\textsf{\textit{noun, verb}}} \\{\fontspec{DejaVu Sans}▪ }\textsf{\textit{noun}}\\ \textbf{1} A feeling or deep pleasure or satisfaction derived from one's own achievements, the achievements of those with whom one is closely associated, or from qualities or possessions that are widely admired. {\fontspec{DejaVu Sans}◇} \textit{the faces of the children's parents glowed with pride} \colorBulletS{SYN} pleasure, joy, delight, gratification, fulfilment, satisfaction, sense of achievement \textbf{2} Confidence and self{-}respect as expressed by members of a group, typically one that has been socially marginalized, on the basis of their shared identity, culture, and experience. {\fontspec{DejaVu Sans}◇} \textit{} \textbf{3} Consciousness of one's own dignity. {\fontspec{DejaVu Sans}◇} \textit{he swallowed his pride and asked for help} \colorBulletS{SYN} self{-}esteem, dignity, honour, self{-}respect, ego, self{-}worth, self{-}image, self{-}identity, self{-}regard, pride in oneself, pride in one's abilities, belief in one's worth, faith in oneself \textbf{4} The best state of something; the prime. {\fontspec{DejaVu Sans}◇} \textit{in the pride of youth} \textbf{5} A group of lions forming a social unit. {\fontspec{DejaVu Sans}◇} \textit{the males in the pride are very tolerant towards all the cubs} \\{\fontspec{DejaVu Sans}▪ }\textsf{\textit{verb}}\\ \textbf{1} Be especially proud of (a particular quality or skill) {\fontspec{DejaVu Sans}◇} \textit{he prided himself on his honesty} \colorBulletS{SYN} be proud of, be proud of oneself for, take pride in, take satisfaction in, congratulate oneself on, flatter oneself on, preen oneself on, pat oneself on the back for, revel in, glory in, delight in, exult in, rejoice in, triumph over}{}{}{ \colorBullet{ORIGIN} Late Old English prȳde ‘excessive self{-}esteem’, variant of prȳtu, prȳte, from prūd (see proud).}%
\par%
\entry{probable}{/ˈprɒbəb(ə)l/}{সম্ভাব্য}{\small{\textsf{\textit{adjective, noun}}} \\{\fontspec{DejaVu Sans}▪ }\textsf{\textit{adjective}}\\ \textbf{1} Likely to happen or be the case. {\fontspec{DejaVu Sans}◇} \textit{it is probable that the economic situation will deteriorate further} \colorBulletS{SYN} likely, most likely, odds{-}on, expected, to be expected, anticipated, predictable, foreseeable, ten to one, presumed, potential, credible, quite possible, possible, feasible \\{\fontspec{DejaVu Sans}▪ }\textsf{\textit{noun}}\\ \textbf{1} A person who is likely to become or do something, especially one who is likely to be chosen for a team. {\fontspec{DejaVu Sans}◇} \textit{Merson and Wright are probables}}{}{}{ \colorBullet{ORIGIN} Late Middle English (in the sense ‘worthy of belief’): via Old French from Latin probabilis, from probare ‘to test, demonstrate’.}%
\par%
\entry{probe}{/prəʊb/}{তদন্ত}{\small{\textsf{\textit{noun, verb}}} \\{\fontspec{DejaVu Sans}▪ }\textsf{\textit{noun}}\\ \textbf{1} A blunt{-}ended surgical instrument used for exploring a wound or part of the body. {\fontspec{DejaVu Sans}◇} \textit{} \textbf{2} A thorough investigation into a crime or other matter. {\fontspec{DejaVu Sans}◇} \textit{a probe into city hall corruption} \colorBulletS{SYN} investigation, inquiry, examination, scrutiny, inquest, exploration, study, research, analysis, scrutinization \textbf{3}  {\fontspec{DejaVu Sans}◇} \textit{} \textbf{4} A projecting device for engaging in a drogue, either on an aircraft for use in in{-}flight refuelling or on a spacecraft for use in docking with another craft. {\fontspec{DejaVu Sans}◇} \textit{} \\{\fontspec{DejaVu Sans}▪ }\textsf{\textit{verb}}\\ \textbf{1} Physically explore or examine (something) with the hands or an instrument. {\fontspec{DejaVu Sans}◇} \textit{hands probed his body from top to bottom} \colorBulletS{SYN} examine, feel, feel around, explore, prod, poke, check}{}{Probe committee: তদন্ত কমিটি}{ \colorBullet{ORIGIN} Late Middle English (as a noun): from late Latin proba ‘proof’ (in medieval Latin ‘examination’), from Latin probare ‘to test’. The verb dates from the mid 17th century.}%
\par%
\entry{procreate}{/ˈprəʊkrɪeɪt/}{সন্তান উৎপাদন করা}{ \textsf{\textit{verb}}\ \textbf{1} (of people or animals) produce young; reproduce. {\fontspec{DejaVu Sans}◇} \textit{species that procreate by copulation} \colorBulletS{SYN} produce offspring, reproduce, multiply, propagate, breed}{}{}{ \colorBullet{ORIGIN} Late Middle English from Latin procreat{-} ‘generated, brought forth’, from the verb procreare, from pro{-} ‘forth’ + creare ‘create’.}%
\par%
\entry{procure}{/prəˈkjʊə/}{রাজী করান}{ \textsf{\textit{verb}}\ \textbf{1} Obtain (something), especially with care or effort. {\fontspec{DejaVu Sans}◇} \textit{food procured for the rebels} \colorBulletS{SYN} obtain, acquire, get, find, come by, secure, pick up, get possession of \textbf{2} Persuade or cause (someone) to do something. {\fontspec{DejaVu Sans}◇} \textit{he procured his wife to sign the mandate for the joint account}}{}{}{ \colorBullet{ORIGIN} Middle English from Old French procurer, from Latin procurare ‘take care of, manage’, from pro{-} ‘on behalf of’ + curare ‘see to’.}%
\par%
\entry{procurement}{/prəˈkjʊəmənt/}{আসাদন}{ \textsf{\textit{noun}}\ \textbf{1} The action of obtaining or procuring something. {\fontspec{DejaVu Sans}◇} \textit{financial assistance for the procurement of legal advice} \colorBulletS{SYN} obtaining, acquiring, gaining, gain, procuring, procurement, collecting, collection, attainment, appropriation, amassing}{}{}{}%
\par%
\entry{program}{}{কার্যক্রম}{\small{\textsf{\textit{}}}}{}{}{}%
\par%
\entry{programme}{/ˈprəʊɡram/}{কার্যক্রম}{\small{\textsf{\textit{noun, verb}}} \\{\fontspec{DejaVu Sans}▪ }\textsf{\textit{noun}}\\ \textbf{1} A set of related measures or activities with a particular long{-}term aim. {\fontspec{DejaVu Sans}◇} \textit{an extensive programme of reforms} \colorBulletS{SYN} scheme, plan, plan of action, initiative, series of measures, project, strategy, solution \textbf{2}  {\fontspec{DejaVu Sans}◇} \textit{} \colorBulletS{SYN} program, software, routine, use \textbf{3} A presentation or item on television or radio, especially one broadcast regularly between stated times. {\fontspec{DejaVu Sans}◇} \textit{a nature programme} \colorBulletS{SYN} broadcast, production, show, presentation, transmission, performance, telecast, simulcast, videocast, podcast \textbf{4} A sheet or booklet giving details of items or performers at an event or performance. {\fontspec{DejaVu Sans}◇} \textit{a theatre programme} \colorBulletS{SYN} guide, list of artistes, list of performers, list of players \\{\fontspec{DejaVu Sans}▪ }\textsf{\textit{verb}}\\ \textbf{1}  {\fontspec{DejaVu Sans}◇} \textit{it is a simple matter to program the computer to recognize such symbols} \textbf{2} Arrange according to a plan or schedule. {\fontspec{DejaVu Sans}◇} \textit{we learn how to programme our own lives} \colorBulletS{SYN} arrange, organize, schedule, plan, map out, lay out, timetable, line up, prearrange \textbf{3} Broadcast (an item) {\fontspec{DejaVu Sans}◇} \textit{the station does not program enough contemporary works}}{}{}{ \colorBullet{ORIGIN} Early 17th century (in the sense ‘written notice’): via late Latin from Greek programma, from prographein ‘write publicly’, from pro ‘before’ + graphein ‘write’.}%
\par%
\entry{prolonged}{/prəˈlɒŋd/}{দীর্ঘায়িত}{ \textsf{\textit{adjective}}\ \textbf{1} Continuing for a long time or longer than usual; lengthy. {\fontspec{DejaVu Sans}◇} \textit{the region suffered a prolonged drought} \colorBulletS{SYN} continuous, ongoing, steady, continual, continuing, constant, running, prolonged, persistent, non{-}stop, perpetual, unfaltering, unremitting, unabating, unrelenting, relentless, unrelieved, unbroken, never{-}ending, unending, incessant, unceasing, ceaseless, round the clock}{}{}{}%
\par%
\entry{prom}{/prɒm/}{নাচের}{ \textsf{\textit{noun}}\ \textbf{1} A paved public walk, typically one along the seafront at a resort. {\fontspec{DejaVu Sans}◇} \textit{she took a shortcut along the prom} \textbf{2} A ball or formal dance at a school or college, especially one held at the end of the academic year for students who are in their final year. {\fontspec{DejaVu Sans}◇} \textit{he asked me to the school prom but I turned him down} \colorBulletS{SYN} ball, discotheque \textbf{3}  {\fontspec{DejaVu Sans}◇} \textit{the last night of the Proms}}{}{}{ \colorBullet{ORIGIN} Late 19th century (originally US, in sense ‘formal dance’): short for promenade.}%
\par%
\entry{prominent}{/ˈprɒmɪnənt/}{বিশিষ্ট}{\small{\textsf{\textit{adjective, noun}}} \\{\fontspec{DejaVu Sans}▪ }\textsf{\textit{adjective}}\\ \textbf{1} Important; famous. {\fontspec{DejaVu Sans}◇} \textit{she was a prominent member of the city council} \colorBulletS{SYN} important, well known, leading, eminent, pre{-}eminent, distinguished, notable, noteworthy, noted, public, outstanding, foremost, of mark, illustrious, celebrated, famous, renowned, acclaimed, famed, honoured, esteemed, respected, well thought of, influential, prestigious, big, top, great, chief, main \textbf{2} Projecting from something; protuberant. {\fontspec{DejaVu Sans}◇} \textit{a man with big, prominent eyes like a lobster's} \colorBulletS{SYN} protuberant, protruding, projecting, jutting, jutting out, standing out, sticking out, proud, bulging, bulbous \textbf{3} Situated so as to catch the attention; noticeable. {\fontspec{DejaVu Sans}◇} \textit{the new housing estates are prominent landmarks} \colorBulletS{SYN} conspicuous, noticeable, easily seen, obvious, evident, discernible, recognizable, distinguishable, unmistakable, eye{-}catching, pronounced, salient, striking, outstanding, dominant, predominant \\{\fontspec{DejaVu Sans}▪ }\textsf{\textit{noun}}\\ \textbf{1} A stout drab{-}coloured moth with tufts on the forewings which stick up while at rest, the caterpillars of which typically have fleshy growths on the back. {\fontspec{DejaVu Sans}◇} \textit{}}{}{}{ \colorBullet{ORIGIN} Late Middle English (in the sense ‘projecting’): from Latin prominent{-} ‘jutting out’, from the verb prominere. Compare with eminent.}%
\par%
\entry{prompt}{/prɒm(p)t/}{প্রম্পট}{\small{\textsf{\textit{adjective, adverb, noun, verb}}} \\{\fontspec{DejaVu Sans}▪ }\textsf{\textit{adjective}}\\ \textbf{1} Done without delay; immediate. {\fontspec{DejaVu Sans}◇} \textit{she would have died but for the prompt action of two ambulancemen} \colorBulletS{SYN} quick, swift, rapid, speedy, fast, direct, immediate, instant, instantaneous, expeditious, early, punctual, in good time, on time, timely \\{\fontspec{DejaVu Sans}▪ }\textsf{\textit{adverb}}\\ \textbf{1} Exactly (with reference to a specified time) {\fontspec{DejaVu Sans}◇} \textit{I set off at three{-}thirty prompt} \colorBulletS{SYN} exactly, precisely, sharp, on the dot, dead, dead on, promptly, punctually, on the nail \\{\fontspec{DejaVu Sans}▪ }\textsf{\textit{noun}}\\ \textbf{1} An act of encouraging a hesitating speaker. {\fontspec{DejaVu Sans}◇} \textit{with barely a prompt, Barbara talked on} \textbf{2} The time limit for the payment of an account, stated on a prompt note. {\fontspec{DejaVu Sans}◇} \textit{} \\{\fontspec{DejaVu Sans}▪ }\textsf{\textit{verb}}\\ \textbf{1} (of an event or fact) cause or bring about (an action or feeling) {\fontspec{DejaVu Sans}◇} \textit{the violence prompted a wave of refugees to flee the country} \colorBulletS{SYN} give rise to, bring about, cause, occasion, result in, lead to, elicit, produce, bring on, engender, induce, call forth, evoke, precipitate, trigger, spark off, provoke, instigate \textbf{2} Encourage (a hesitating speaker) to say something. {\fontspec{DejaVu Sans}◇} \textit{‘And the picture?’ he prompted} \colorBulletS{SYN} remind, cue, give someone a cue, help out, coach, feed}{}{}{ \colorBullet{ORIGIN} Middle English (as a verb): based on Old French prompt or Latin promptus ‘brought to light’, also ‘prepared, ready’, past participle of promere ‘to produce’, from pro{-} ‘out, forth’ + emere ‘take’.}%
\par%
\entry{prop}{/prɒp/}{ঠেকনা}{\small{\textsf{\textit{noun, verb}}} \\{\fontspec{DejaVu Sans}▪ }\textsf{\textit{noun}}\\ \textbf{1} A pole or beam used as a temporary support or to keep something in position. {\fontspec{DejaVu Sans}◇} \textit{he looked around for a prop to pin the door open} \colorBulletS{SYN} pole, post, beam, support, upright, brace, buttress, stay, shaft, strut, stanchion, shore, pier, vertical, pillar, pile, piling, bolster, truss, column, rod, stick \textbf{2}  {\fontspec{DejaVu Sans}◇} \textit{} \textbf{3} A sudden stop made by a horse moving at speed. {\fontspec{DejaVu Sans}◇} \textit{} \\{\fontspec{DejaVu Sans}▪ }\textsf{\textit{verb}}\\ \textbf{1} Support or keep in position. {\fontspec{DejaVu Sans}◇} \textit{she propped her chin in the palm of her right hand} \colorBulletS{SYN} hold up, shore up, bolster up, buttress, support, brace, underpin, reinforce, strengthen \textbf{2} (of a horse) come to a dead stop with the forelegs rigid. {\fontspec{DejaVu Sans}◇} \textit{Kalanisi propped while galloping out and unseated exercise rider Wally Lowsby, who held on to the reins.}}{ \colorBullet{OTHER} prop up}{}{ \colorBullet{ORIGIN} Late Middle English probably from Middle Dutch proppe ‘support (for vines)’.}%
\par%
\entry{prop}{/prɒp/}{ঠেকনা}{ \textsf{\textit{noun}}\ \textbf{1} A portable object other than furniture or costumes used on the set of a play or film. {\fontspec{DejaVu Sans}◇} \textit{}}{ \colorBullet{OTHER} prop up}{}{ \colorBullet{ORIGIN} Mid 19th century abbreviation of property.}%
\par%
\entry{prop}{/prɒp/}{ঠেকনা}{ \textsf{\textit{noun}}\ \textbf{1} An aircraft propeller. {\fontspec{DejaVu Sans}◇} \textit{}}{ \colorBullet{OTHER} prop up}{}{ \colorBullet{ORIGIN} Early 20th century abbreviation.}%
\par%
\entry{prophylactic}{/ˌprɒfɪˈlaktɪk/}{প্রতিষেধক}{\small{\textsf{\textit{adjective, noun}}} \\{\fontspec{DejaVu Sans}▪ }\textsf{\textit{adjective}}\\ \textbf{1} Intended to prevent disease. {\fontspec{DejaVu Sans}◇} \textit{prophylactic measures} \colorBulletS{SYN} preventive, preventative, precautionary, protective, disease{-}preventing, pre{-}emptive, counteractive, preclusive, anticipatory, inhibitory, deterrent \\{\fontspec{DejaVu Sans}▪ }\textsf{\textit{noun}}\\ \textbf{1} A medicine or course of action used to prevent disease. {\fontspec{DejaVu Sans}◇} \textit{I took malaria prophylactics} \colorBulletS{SYN} preventive measure, precaution, safeguard, safety measure \textbf{2} A condom. {\fontspec{DejaVu Sans}◇} \textit{} \colorBulletS{SYN} condom, sheath}{}{}{ \colorBullet{ORIGIN} Late 16th century from French prophylactique, from Greek prophulaktikos, from pro ‘before’ + phulassein ‘to guard’.}%
\par%
\entry{proportional}{/prəˈpɔːʃ(ə)n(ə)l/}{সমানুপাতিক}{ \textsf{\textit{adjective}}\ \textbf{1} Corresponding in size or amount to something else. {\fontspec{DejaVu Sans}◇} \textit{the punishment should be proportional to the crime} \colorBulletS{SYN} corresponding, proportionate, comparable, in proportion, pro rata, commensurate, equivalent, consistent, relative, correlated, correlative, analogous, analogical}{}{}{ \colorBullet{ORIGIN} Late Middle English from late Latin proportionalis, from proportio(n{-}) (see proportion).}%
\par%
\entry{proposition}{/prɒpəˈzɪʃ(ə)n/}{প্রতিজ্ঞা}{\small{\textsf{\textit{noun, verb}}} \\{\fontspec{DejaVu Sans}▪ }\textsf{\textit{noun}}\\ \textbf{1} A statement or assertion that expresses a judgement or opinion. {\fontspec{DejaVu Sans}◇} \textit{the proposition that high taxation is undesirable} \colorBulletS{SYN} theory, hypothesis, thesis, argument, premise, postulation, theorem, concept, idea, statement \textbf{2} A suggested scheme or plan of action, especially in a business context. {\fontspec{DejaVu Sans}◇} \textit{a detailed investment proposition} \colorBulletS{SYN} proposal, scheme, plan, project, programme, manifesto, motion, bid, presentation, submission, suggestion, recommendation, approach \textbf{3} A project, task, idea, etc. considered in terms of its likely success or difficulty. {\fontspec{DejaVu Sans}◇} \textit{setting up your own business can seem an attractive proposition} \colorBulletS{SYN} task, job, undertaking, venture, activity, problem, affair \\{\fontspec{DejaVu Sans}▪ }\textsf{\textit{verb}}\\ \textbf{1} Make a suggestion of sexual intercourse to (someone), especially in an unsubtle way. {\fontspec{DejaVu Sans}◇} \textit{she had been propositioned at the party by a subeditor with bad breath} \colorBulletS{SYN} propose sex with, make sexual advances to, make sexual overtures to, make an indecent proposal to, make an improper suggestion to}{}{}{ \colorBullet{ORIGIN} Middle English from Old French, from Latin propositio(n{-}), from the verb proponere (see propound).}%
\par%
\entry{prosecution}{/prɒsɪˈkjuːʃ(ə)n/}{প্রসিকিউশন}{ \textsf{\textit{noun}}\ \textbf{1} The institution and conducting of legal proceedings against someone in respect of a criminal charge. {\fontspec{DejaVu Sans}◇} \textit{the organizers are facing prosecution for noise nuisance} \colorBulletS{SYN} indictment, accusation, denunciation, prosecution, trial, charge, summons, citation \textbf{2} The continuation of a course of action with a view to its completion. {\fontspec{DejaVu Sans}◇} \textit{the BBC's prosecution of its commercial ends} \colorBulletS{SYN} execution, application, carrying out, carrying through, performance, enactment, administration}{}{}{ \colorBullet{ORIGIN} Mid 16th century (in prosecution (sense 2)): from Old French, or from late Latin prosecutio(n{-}), from prosequi ‘pursue, accompany’ (see prosecute).}%
\par%
\entry{prostate}{/ˈprɒsteɪt/}{প্রস্টেট}{ \textsf{\textit{noun}}\ \textbf{1} A gland surrounding the neck of the bladder in male mammals and releasing a fluid component of semen. {\fontspec{DejaVu Sans}◇} \textit{}}{}{}{ \colorBullet{ORIGIN} Mid 17th century via French from modern Latin prostata, from Greek prostatēs ‘one that stands before’, from pro ‘before’ + statos ‘standing’.}%
\par%
\entry{provision}{/prəˈvɪʒ(ə)n/}{বিধান}{\small{\textsf{\textit{noun, verb}}} \\{\fontspec{DejaVu Sans}▪ }\textsf{\textit{noun}}\\ \textbf{1} The action of providing or supplying something for use. {\fontspec{DejaVu Sans}◇} \textit{new contracts for the provision of services} \colorBulletS{SYN} supplying, supply, providing, purveying, delivery, furnishing, equipping, giving, donation, allocation, distribution, presentation \textbf{2} An amount or thing supplied or provided. {\fontspec{DejaVu Sans}◇} \textit{changing levels of transport provision} \colorBulletS{SYN} facilities, services, amenities, resource, resources, equipment, arrangements, solutions \textbf{3} A condition or requirement in a legal document. {\fontspec{DejaVu Sans}◇} \textit{the first private prosecution under the provisions of the 1989 Water Act} \colorBulletS{SYN} term, clause, requirement, specification, stipulation \textbf{4} An appointment to a benefice, especially directly by the Pope rather than by the patron, and originally before it became vacant. {\fontspec{DejaVu Sans}◇} \textit{Let us take another medieval example, the case of papal provisions in medieval England.} \\{\fontspec{DejaVu Sans}▪ }\textsf{\textit{verb}}\\ \textbf{1} Supply with food, drink, or equipment, especially for a journey. {\fontspec{DejaVu Sans}◇} \textit{civilian contractors were responsible for provisioning these armies} \colorBulletS{SYN} supply, provide, furnish, arm, equip, fit out, rig out, kit out, accoutre, outfit, fit up \textbf{2} Set aside an amount in an organization's accounts for a known liability. {\fontspec{DejaVu Sans}◇} \textit{financial institutions have to provision against loan losses}}{}{}{ \colorBullet{ORIGIN} Late Middle English (also in the sense ‘foresight’): via Old French from Latin provisio(n{-}), from providere ‘foresee, attend to’ (see provide). The verb dates from the early 19th century.}%
\par%
\entry{provisionally}{/prəˈvɪʒən(ə)li/}{আপাতত}{ \textsf{\textit{adverb}}\ \textbf{1} Subject to further confirmation; for the time being. {\fontspec{DejaVu Sans}◇} \textit{the film, provisionally entitled Skin, is due to be released next year} \colorBulletS{SYN} subject to confirmation, in an acting capacity, as a fill{-}in, short{-}term, pro tem, temporarily, for the interim, for the present, for the time being, for now, for the nonce}{}{}{}%
\par%
\entry{provocative}{/prəˈvɒkətɪv/}{উত্তেজক}{ \textsf{\textit{adjective}}\ \textbf{1} Causing anger or another strong reaction, especially deliberately. {\fontspec{DejaVu Sans}◇} \textit{a provocative article} \colorBulletS{SYN} annoying, irritating, exasperating, infuriating, provoking, maddening, goading, vexing, galling}{}{}{ \colorBullet{ORIGIN} Late Middle English from Old French provocatif, {-}ive, from late Latin provocativus, from provocat{-} ‘called forth, challenged’, from the verb provocare (see provoke).}%
\par%
\entry{provoking}{/prəˈvəʊkɪŋ/}{উদ্দীপক}{ \textsf{\textit{adjective}}\ \textbf{1} Causing annoyance; irritating. {\fontspec{DejaVu Sans}◇} \textit{there is evidence of provoking conduct and loss of self{-}control} \textbf{2} Giving rise to the specified reaction or emotion. {\fontspec{DejaVu Sans}◇} \textit{fear{-}provoking}}{}{}{}%
\par%
\entry{prowess}{/ˈpraʊɪs/}{পরাক্রম}{ \textsf{\textit{noun}}\ \textbf{1} Skill or expertise in a particular activity or field. {\fontspec{DejaVu Sans}◇} \textit{his prowess as a fisherman} \colorBulletS{SYN} skill, skilfulness, expertise, effectiveness, mastery, facility, ability, capability, capacity, talent, genius, adroitness, adeptness, aptitude, dexterity, deftness, competence, competency, professionalism, excellence, accomplishment, experience, proficiency, expertness, finesse, know{-}how \textbf{2} Bravery in battle. {\fontspec{DejaVu Sans}◇} \textit{the hereditary nobility had no monopoly of skill and prowess in war} \colorBulletS{SYN} courage, bravery, gallantry, valour, heroism, intrepidness, intrepidity, nerve, pluck, pluckiness, doughtiness, hardihood, braveness, courageousness, dauntlessness, gameness, manfulness, boldness, daring, audacity, spirit, fearlessness}{}{}{ \colorBullet{ORIGIN} Middle English (in prowess (sense 2)): from Old French proesce, from prou ‘valiant’. Sense 1 dates from the early 20th century.}%
\par%
\entry{puff}{/pʌf/}{}{\small{\textsf{\textit{noun, verb}}} \\{\fontspec{DejaVu Sans}▪ }\textsf{\textit{noun}}\\ \textbf{1} A short, explosive burst of breath or wind. {\fontspec{DejaVu Sans}◇} \textit{a puff of wind swung the weathercock round} \colorBulletS{SYN} gust, blast, rush, squall, gale, whiff, breath, flurry, draught, waft, breeze, blow \textbf{2} A light pastry case, typically one made of puff pastry, containing a sweet or savoury filling. {\fontspec{DejaVu Sans}◇} \textit{a jam puff} \textbf{3} A review of a work of art, book, or theatrical production, especially an excessively complimentary one. {\fontspec{DejaVu Sans}◇} \textit{the publishers sent him a copy of the book hoping for a puff} \colorBulletS{SYN} favourable mention, piece of publicity, favourable review, advertisement, promotion, recommendation, commendation, mention, good word, commercial \textbf{4} A gathered mass of material in a dress or other garment. {\fontspec{DejaVu Sans}◇} \textit{} \textbf{5} A powder puff. {\fontspec{DejaVu Sans}◇} \textit{she sent her a box of dusting powder with a swansdown puff} \\{\fontspec{DejaVu Sans}▪ }\textsf{\textit{verb}}\\ \textbf{1} Breathe in repeated short gasps. {\fontspec{DejaVu Sans}◇} \textit{exercises that make you puff} \colorBulletS{SYN} breathe heavily, breathe loudly, breathe quickly, breathe rapidly, pant, puff and pant, puff and blow, blow \textbf{2} Swell or become swollen. {\fontspec{DejaVu Sans}◇} \textit{he suddenly sucked his stomach in and puffed his chest out} \colorBulletS{SYN} bulge, swell, swell out, stick out, distend, belly, belly out, balloon, balloon out, balloon up, expand, inflate, enlarge \textbf{3} Advertise with exaggerated or false praise. {\fontspec{DejaVu Sans}◇} \textit{publishers have puffed the book on the grounds that it contains new discoveries} \colorBulletS{SYN} advertise, promote, give publicity to, publicize, push, recommend, commend, endorse, put in a good word for, beat the drum for}{}{Tank engine with real puffing smoke}{ \colorBullet{ORIGIN} Middle English imitative of the sound of a breath, perhaps from Old English pyf (noun), pyffan (verb).}%
\par%
\entry{pull}{/pʊl/}{টান}{\small{\textsf{\textit{noun, verb}}} \\{\fontspec{DejaVu Sans}▪ }\textsf{\textit{noun}}\\ \textbf{1} An act of pulling something. {\fontspec{DejaVu Sans}◇} \textit{give the hair a quick pull and it comes out by the roots} \colorBulletS{SYN} tug, haul, jerk, heave \textbf{2} A force drawing someone or something in a particular direction. {\fontspec{DejaVu Sans}◇} \textit{the pull of the water tore her away} \colorBulletS{SYN} tug, towing, haul, pull, drawing, drag, trailing, trawl \textbf{3} (in sport) a pulling stroke. {\fontspec{DejaVu Sans}◇} \textit{} \textbf{4} A printer's proof. {\fontspec{DejaVu Sans}◇} \textit{Proof ‘pulls’ of World War propaganda posters are quite rare.} \colorBulletS{SYN} page proof, galley proof, galley, pull, slip, trial print \\{\fontspec{DejaVu Sans}▪ }\textsf{\textit{verb}}\\ \textbf{1} Exert force on (someone or something) so as to cause movement towards oneself. {\fontspec{DejaVu Sans}◇} \textit{he pulled them down on to the couch} \colorBulletS{SYN} tug, haul, drag, draw, trail, tow, heave, lug, strain at, jerk, lever, prise, wrench, wrest, twist \textbf{2} Move steadily in a specified direction or manner. {\fontspec{DejaVu Sans}◇} \textit{the bus was about to pull away} \textbf{3} Attract (someone) as a customer; cause to show interest in something. {\fontspec{DejaVu Sans}◇} \textit{anyone can enter the show if they have a good act and the ability to pull a crowd} \colorBulletS{SYN} attract, draw, pull in, bring in, lure, charm, engage, enchant, captivate, bewitch, seduce, catch the eye of, entice, tempt, beckon, interest, fascinate \textbf{4} Cancel or withdraw (an entertainment or advertisement) {\fontspec{DejaVu Sans}◇} \textit{the gig was pulled at the first sign of difficulty} \textbf{5} Play (the ball) round to the leg side from the off. {\fontspec{DejaVu Sans}◇} \textit{} \textbf{6} (of a lineman) withdraw from and cross behind the line of scrimmage to block opposing players and clear the way for a runner. {\fontspec{DejaVu Sans}◇} \textit{he may be their best ever lineman—he can run and pull with the best} \textbf{7} Print (a proof). {\fontspec{DejaVu Sans}◇} \textit{A proof sheet would be pulled, and read against the manuscript.} \colorBulletS{SYN} set in print, send to press, run off, preprint, reprint, pull, proof, copy, reproduce}{}{}{ \colorBullet{ORIGIN} Old English pullian ‘pluck, snatch’; origin uncertain; the sense has developed from expressing a short sharp action to one of sustained force.}%
\par%
\entry{pulsate}{/pʌlˈseɪt/}{স্পন্দিত}{ \textsf{\textit{verb}}\ \textbf{1} Expand and contract with strong regular movements. {\fontspec{DejaVu Sans}◇} \textit{blood vessels throb and pulsate}}{}{}{ \colorBullet{ORIGIN} Late 17th century (earlier (Middle English) as pulsation): from Latin pulsat{-} ‘throbbed, pulsed’, from the verb pulsare, frequentative of pellere ‘to drive, beat’.}%
\par%
\entry{pulse}{/pʌls/}{নাড়ি}{\small{\textsf{\textit{noun, verb}}} \\{\fontspec{DejaVu Sans}▪ }\textsf{\textit{noun}}\\ \textbf{1} A rhythmical throbbing of the arteries as blood is propelled through them, typically as felt in the wrists or neck. {\fontspec{DejaVu Sans}◇} \textit{the doctor found a faint pulse} \colorBulletS{SYN} heartbeat, pulsation, pulsing, throb, throbbing, vibration, pounding, thudding, thud, thumping, thump, drumming \textbf{2} A single vibration or short burst of sound, electric current, light, or other wave. {\fontspec{DejaVu Sans}◇} \textit{a pulse of gamma rays} \colorBulletS{SYN} burst, blast, spurt, eruption, impulse, surge \textbf{3} The central point of energy and organization in an area or activity. {\fontspec{DejaVu Sans}◇} \textit{those close to the financial and economic pulse maintain that there have been fundamental changes} \textbf{4} A measured amount of an isotopic label given to a culture of cells. {\fontspec{DejaVu Sans}◇} \textit{} \\{\fontspec{DejaVu Sans}▪ }\textsf{\textit{verb}}\\ \textbf{1} Throb rhythmically; pulsate. {\fontspec{DejaVu Sans}◇} \textit{a knot of muscles at the side of his jaw pulsed} \colorBulletS{SYN} throb, pulsate, vibrate, palpitate, beat, pound, thud, thump, hammer, drum, thrum, oscillate, reverberate \textbf{2} Modulate (a wave or beam) so that it becomes a series of pulses. {\fontspec{DejaVu Sans}◇} \textit{the current was pulsed}}{}{}{ \colorBullet{ORIGIN} Late Middle English from Latin pulsus ‘beating’, from pellere ‘to drive, beat’.}%
\par%
\entry{pulse}{/pʌls/}{নাড়ি}{ \textsf{\textit{noun}}\ \textbf{1} The edible seed of a leguminous plant, for example a chickpea, lentil, or bean. {\fontspec{DejaVu Sans}◇} \textit{use pulses such as peas and lentils to eke out meat dishes}}{}{}{ \colorBullet{ORIGIN} Middle English from Old French pols, from Latin puls ‘porridge of meal or pulse’; related to pollen.}%
\par%
\entry{pumpkin}{/ˈpʌm(p)kɪn/}{কুমড়া}{ \textsf{\textit{noun}}\ \textbf{1} A large rounded orange{-}yellow fruit with a thick rind, the flesh of which can be used in sweet or savoury dishes. {\fontspec{DejaVu Sans}◇} \textit{} \textbf{2} The plant of the gourd family that produces pumpkins, having tendrils and large lobed leaves and native to warm regions of America. {\fontspec{DejaVu Sans}◇} \textit{}}{}{}{ \colorBullet{ORIGIN} Late 17th century alteration of earlier pumpion, from obsolete French pompon, via Latin from Greek pepōn ‘large melon’ (see pepo).}%
\par%
\entry{pundit}{/ˈpʌndɪt/}{পণ্ডিত}{ \textsf{\textit{noun}}\ \textbf{1} An expert in a particular subject or field who is frequently called upon to give their opinions to the public. {\fontspec{DejaVu Sans}◇} \textit{political pundits were tipping him for promotion} \colorBulletS{SYN} expert, authority, adviser, member of a think tank, member of a policy unit, specialist, consultant, doyen, master, mentor, guru, sage, savant \textbf{2} variant form of pandit {\fontspec{DejaVu Sans}◇} \textit{}}{}{}{ \colorBullet{ORIGIN} Mid 17th century (in pundit (sense 2)): from Sanskrit paṇḍita ‘learned man’, use as noun of paṇḍita ‘learned, skilled’. pundit (sense 1)is first recorded in the early 19th century.}%
\par%
\entry{purr}{/pəː/}{গরগর আওয়াজ}{\small{\textsf{\textit{noun, verb}}} \\{\fontspec{DejaVu Sans}▪ }\textsf{\textit{noun}}\\ \textbf{1} A purring sound. {\fontspec{DejaVu Sans}◇} \textit{a contented purr} \colorBulletS{SYN} murmur, murmuring, drone, droning, vibration, purr, purring, buzz, buzzing, whir, whirring, throb, throbbing, thrum, thrumming \\{\fontspec{DejaVu Sans}▪ }\textsf{\textit{verb}}\\ \textbf{1} (of a cat) make a low continuous vibratory sound expressing contentment. {\fontspec{DejaVu Sans}◇} \textit{the cat purred loudly, rubbing against her legs}}{}{}{ \colorBullet{ORIGIN} Early 17th century imitative.}%
\par%
\entry{purse}{/pəːs/}{টাকার থলি}{\small{\textsf{\textit{noun, verb}}} \\{\fontspec{DejaVu Sans}▪ }\textsf{\textit{noun}}\\ \textbf{1} A small pouch of leather or plastic used for carrying money, typically by a woman. {\fontspec{DejaVu Sans}◇} \textit{she had enough in her purse for bus fare} \colorBulletS{SYN} wallet, pouch, money bag \textbf{2} A handbag. {\fontspec{DejaVu Sans}◇} \textit{a young woman with a purse hanging from her elbow} \colorBulletS{SYN} handbag, bag, clutch bag, shoulder bag, evening bag, pochette \\{\fontspec{DejaVu Sans}▪ }\textsf{\textit{verb}}\\ \textbf{1} (with reference to the lips) pucker or contract, typically to express disapproval or irritation. {\fontspec{DejaVu Sans}◇} \textit{Marianne took a glance at her reflection and pursed her lips disgustedly} \colorBulletS{SYN} press together, compress, contract, tighten, pucker, screw up, wrinkle, pout}{}{}{ \colorBullet{ORIGIN} Late Old English, alteration of late Latin bursa ‘purse’, from Greek bursa ‘hide, leather’. The current verb sense (from the notion of drawing purse strings) dates from the early 17th century.}%
\par%
\entry{pursue}{/pəˈsjuː/}{অন্বেষণ করা}{ \textsf{\textit{verb}}\ \textbf{1} Follow or chase (someone or something) {\fontspec{DejaVu Sans}◇} \textit{the officer pursued the van} \colorBulletS{SYN} go after, run after, follow, chase, give chase to \textbf{2} Continue or proceed along (a path or route) {\fontspec{DejaVu Sans}◇} \textit{the road pursued a straight course over the scrubland}}{}{}{ \colorBullet{ORIGIN} Middle English (originally in the sense ‘follow with enmity’): from Anglo{-}Norman French pursuer, from an alteration of Latin prosequi ‘prosecute’.}%
\par%
\entry{pursuit}{/pəˈsjuːt/}{সাধনা}{ \textsf{\textit{noun}}\ \textbf{1} The action of pursuing someone or something. {\fontspec{DejaVu Sans}◇} \textit{the cat crouched in the grass in pursuit of a bird} \colorBulletS{SYN} chasing, pursuing, stalking, tracking, trailing, shadowing, dogging, hounding \textbf{2} An activity of a specified kind, especially a recreational or sporting one. {\fontspec{DejaVu Sans}◇} \textit{a whole range of leisure pursuits} \colorBulletS{SYN} activity, leisure activity, leisure pursuit, leisure interest, hobby, pastime, diversion, avocation, recreation, relaxation, divertissement, sideline, entertainment, amusement, sport, game}{}{}{ \colorBullet{ORIGIN} Late Middle English from Anglo{-}Norman French purseute ‘following after’, from pursuer (see pursue). Early senses included ‘persecution, annoyance’ and in legal contexts ‘petition, prosecution’.}%
\par%
\end{multicols}%
\pagebreak%
\section*{Q}%
\begin{multicols}{2}%
\entry{quake}{/kweɪk/}{ভূমিকম্প}{\small{\textsf{\textit{noun, verb}}} \\{\fontspec{DejaVu Sans}▪ }\textsf{\textit{noun}}\\ \textbf{1} An earthquake. {\fontspec{DejaVu Sans}◇} \textit{a big quake east of the Rocky Mountains} \colorBulletS{SYN} earth tremor, tremor, convulsion, shock, foreshock, aftershock \\{\fontspec{DejaVu Sans}▪ }\textsf{\textit{verb}}\\ \textbf{1} (especially of the earth) shake or tremble. {\fontspec{DejaVu Sans}◇} \textit{the rumbling vibrations set the whole valley quaking} \colorBulletS{SYN} shake, tremble, quiver, shiver, shudder, sway, rock, wobble, move, heave, convulse}{}{}{ \colorBullet{ORIGIN} Old English cwacian.}%
\par%
\entry{query}{/ˈkwɪəri/}{প্রশ্ন}{\small{\textsf{\textit{noun, verb}}} \\{\fontspec{DejaVu Sans}▪ }\textsf{\textit{noun}}\\ \textbf{1} A question, especially one expressing doubt or requesting information. {\fontspec{DejaVu Sans}◇} \textit{if you have any queries please telephone our office} \colorBulletS{SYN} question, inquiry \\{\fontspec{DejaVu Sans}▪ }\textsf{\textit{verb}}\\ \textbf{1} Ask a question about something, especially in order to express one's doubts about it or to check its validity or accuracy. {\fontspec{DejaVu Sans}◇} \textit{many people queried whether any harm had been done} \colorBulletS{SYN} ask, inquire, question}{}{}{ \colorBullet{ORIGIN} Mid 17th century anglicized form of the Latin imperative quaere!, used in the 16th century in English as a verb in the sense ‘inquire’ and as a noun meaning ‘query’, from Latin quaerere ‘ask, seek’.}%
\par%
\entry{quirky}{/ˈkwəːki/}{বিচিত্র}{ \textsf{\textit{adjective}}\ \textbf{1} Having or characterized by peculiar or unexpected traits or aspects. {\fontspec{DejaVu Sans}◇} \textit{her sense of humour was decidedly quirky} \colorBulletS{SYN} eccentric, idiosyncratic, unconventional, unorthodox, unusual, off{-}centre, strange, bizarre, weird, peculiar, odd, freakish, outlandish, offbeat, out of the ordinary, Bohemian, alternative, zany}{}{}{}%
\par%
\entry{quite a lot}{}{কিছুটা}{\small{\textsf{\textit{}}}}{}{}{}%
\par%
\entry{quizzical}{/ˈkwɪzɪk(ə)l/}{ব্যঙ্গাত্মক}{ \textsf{\textit{adjective}}\ \textbf{1} (of a person's expression or behaviour) indicating mild or amused puzzlement. {\fontspec{DejaVu Sans}◇} \textit{she gave me a quizzical look} \colorBulletS{SYN} puzzled, perplexed, baffled, questioning, inquiring, mystified, curious, sceptical}{}{}{}%
\par%
\end{multicols}%
\pagebreak%
\section*{R}%
\begin{multicols}{2}%
\entry{racist}{/ˈreɪsɪst/}{বর্ণবাদী}{\small{\textsf{\textit{adjective, noun}}} \\{\fontspec{DejaVu Sans}▪ }\textsf{\textit{adjective}}\\ \textbf{1} Showing or feeling discrimination or prejudice against people of other races, or believing that a particular race is superior to another. {\fontspec{DejaVu Sans}◇} \textit{we are investigating complaints about racist abuse at a newsagents} \\{\fontspec{DejaVu Sans}▪ }\textsf{\textit{noun}}\\ \textbf{1} A person who shows or feels discrimination or prejudice against people of other races, or who believes that a particular race is superior to another. {\fontspec{DejaVu Sans}◇} \textit{I had a fear of being called a racist} \colorBulletS{SYN} racial bigot, racialist, xenophobe, chauvinist}{}{}{}%
\par%
\entry{raft}{/rɑːft/}{ভেলা}{\small{\textsf{\textit{noun, verb}}} \\{\fontspec{DejaVu Sans}▪ }\textsf{\textit{noun}}\\ \textbf{1} A flat buoyant structure of timber or other materials fastened together, used as a boat or floating platform. {\fontspec{DejaVu Sans}◇} \textit{} \colorBulletS{SYN} arrangement, assembling, assemblage, line{-}up, formation, ordering, disposition, marshalling, muster, amassing \textbf{2} A layer of reinforced concrete forming the foundation of a building. {\fontspec{DejaVu Sans}◇} \textit{} \\{\fontspec{DejaVu Sans}▪ }\textsf{\textit{verb}}\\ \textbf{1} Travel on or as if on a raft. {\fontspec{DejaVu Sans}◇} \textit{I have rafted along the Rio Grande} \textbf{2} Bring or fasten together (a number of boats or other objects) side by side. {\fontspec{DejaVu Sans}◇} \textit{we rafted the boats together off the shores of Murchison Island}}{}{}{ \colorBullet{ORIGIN} Late Middle English (in the sense ‘beam, rafter’): from Old Norse raptr ‘rafter’. The verb dates from the late 17th century.}%
\par%
\entry{raft}{/rɑːft/}{ভেলা}{ \textsf{\textit{noun}}\ \textbf{1} A large amount of something. {\fontspec{DejaVu Sans}◇} \textit{a raft of government initiatives}}{}{}{ \colorBullet{ORIGIN} Mid 19th century alteration of dialect raff ‘abundance’ (perhaps of Scandinavian origin), by association with raft in the sense ‘floating mass’.}%
\par%
\entry{rafting}{/ˈrɑːftɪŋ/}{ভেলা করিয়া লইয়া যাত্তয়া}{ \textsf{\textit{noun}}\ \textbf{1} The sport or pastime of travelling down a river on a raft. {\fontspec{DejaVu Sans}◇} \textit{activities include rafting and tennis}}{}{}{}%
\par%
\entry{raid}{/reɪd/}{উপদ্রব}{\small{\textsf{\textit{noun, verb}}} \\{\fontspec{DejaVu Sans}▪ }\textsf{\textit{noun}}\\ \textbf{1} A rapid surprise attack on an enemy by troops, aircraft, or other armed forces. {\fontspec{DejaVu Sans}◇} \textit{a bombing raid} \colorBulletS{SYN} surprise attack, hit{-}and{-}run raid, tip{-}and{-}run raid, assault, descent, blitz, incursion, foray, sortie \\{\fontspec{DejaVu Sans}▪ }\textsf{\textit{verb}}\\ \textbf{1} Conduct a raid on. {\fontspec{DejaVu Sans}◇} \textit{officers raided thirty homes yesterday} \colorBulletS{SYN} attack, make a raid on, assault, set upon, descend on, swoop on, harass, harry, blitz, make inroads on, assail, storm, rush, charge}{}{}{ \colorBullet{ORIGIN} Late Middle English (as a noun): Scots variant of road in the early senses ‘journey on horseback’, ‘foray’. The noun became rare from the end of the 16th century but was revived by Sir Walter Scott; the verb dates from the mid 19th century.}%
\par%
\entry{RAID}{/reɪd/}{উপদ্রব}{ \textsf{\textit{abbreviation}}\ \textbf{1} Redundant array of independent (or inexpensive) disks, a system for providing greater capacity, faster access, and security against data corruption by spreading data across several disk drives. {\fontspec{DejaVu Sans}◇} \textit{}}{}{}{}%
\par%
\entry{rally}{/ˈrali/}{দাম বেড়েছে}{\small{\textsf{\textit{noun, verb}}} \\{\fontspec{DejaVu Sans}▪ }\textsf{\textit{noun}}\\ \textbf{1} A mass meeting of people making a political protest or showing support for a cause. {\fontspec{DejaVu Sans}◇} \textit{a banned nationalist rally} \colorBulletS{SYN} meeting, mass meeting, gathering, assembly, tweetup \textbf{2} A long{-}distance race for motor vehicles over public roads or rough terrain, typically in several stages. {\fontspec{DejaVu Sans}◇} \textit{a rally driver} \textbf{3} A quick or marked recovery after a decline. {\fontspec{DejaVu Sans}◇} \textit{the market staged a late rally} \colorBulletS{SYN} recovery, upturn, improvement, revival, comeback, rebound, resurgence, renewal, a turn for the better, reaction \textbf{4} (in tennis and other racket sports) an extended exchange of strokes between players. {\fontspec{DejaVu Sans}◇} \textit{a rally of more than three strokes was a rarity} \\{\fontspec{DejaVu Sans}▪ }\textsf{\textit{verb}}\\ \textbf{1} (of troops) come together again in order to continue fighting after a defeat or dispersion. {\fontspec{DejaVu Sans}◇} \textit{De Montfort's troops rallied and drove back the king's infantry} \colorBulletS{SYN} reassemble, regroup, re{-}form, reunite, gather together again, get together again \textbf{2} Recover or cause to recover in health, spirits, or poise. {\fontspec{DejaVu Sans}◇} \textit{he floundered for a moment, then rallied again} \colorBulletS{SYN} recover, improve, get better, pick up, revive, come back, make a comeback, rebound, bounce back, perk up, look up, take a turn for the better, turn a corner, turn the corner, be given a new lease of life, take on a new lease of life \textbf{3} Drive in a rally. {\fontspec{DejaVu Sans}◇} \textit{we're driving off to Spain to rally}}{}{}{ \colorBullet{ORIGIN} Early 17th century (in the sense ‘bring together again’): from French rallier, from re{-} ‘again’ + allier ‘to ally’.}%
\par%
\entry{rally}{/ˈrali/}{দাম বেড়েছে}{ \textsf{\textit{verb}}\ \textbf{1} Subject (someone) to good{-}humoured ridicule; tease. {\fontspec{DejaVu Sans}◇} \textit{he rallied her on the length of her pigtail}}{}{}{ \colorBullet{ORIGIN} Mid 17th century from French railler ‘to rib, tease’ (see rail).}%
\par%
\entry{ram}{/ram/}{পিটান}{\small{\textsf{\textit{noun, verb}}} \\{\fontspec{DejaVu Sans}▪ }\textsf{\textit{noun}}\\ \textbf{1} An uncastrated male sheep. {\fontspec{DejaVu Sans}◇} \textit{} \textbf{2} A battering ram. {\fontspec{DejaVu Sans}◇} \textit{} \textbf{3} The falling weight of a piledriving machine. {\fontspec{DejaVu Sans}◇} \textit{He says one man with a hoe ram on a Bobcat can break the same amount of concrete that two or three men could do with a jackhammer.} \textbf{4} A hydraulic water{-}raising or lifting machine. {\fontspec{DejaVu Sans}◇} \textit{Burnside Autocyl Ltd, Tullow is a European leader in the manufacture of hydraulic cylinders and rams.} \\{\fontspec{DejaVu Sans}▪ }\textsf{\textit{verb}}\\ \textbf{1} Roughly force (something) into place. {\fontspec{DejaVu Sans}◇} \textit{he rammed his stick into the ground} \colorBulletS{SYN} force, thrust, plunge, stab, push, sink, dig, stick, cram, jam, stuff, pack, compress, squeeze, wedge, press, tamp, pound, drive, hammer, bang \textbf{2} (of a place) be very crowded. {\fontspec{DejaVu Sans}◇} \textit{the club is rammed to the rafters every week}}{}{}{ \colorBullet{ORIGIN} Old English ram(m), of Germanic origin; related to Dutch ram.}%
\par%
\entry{RAM}{/ram/}{পিটান}{ \textsf{\textit{abbreviation}}\ \textbf{1} Random{-}access memory. {\fontspec{DejaVu Sans}◇} \textit{} \colorBulletS{SYN} memory bank, store, cache, disk, RAM, ROM \textbf{2} (in the UK) Royal Academy of Music. {\fontspec{DejaVu Sans}◇} \textit{}}{}{}{}%
\par%
\entry{ramble}{/ˈramb(ə)l/}{ঘুরাঘুরি করা}{\small{\textsf{\textit{noun, verb}}} \\{\fontspec{DejaVu Sans}▪ }\textsf{\textit{noun}}\\ \textbf{1} A walk taken for pleasure in the countryside. {\fontspec{DejaVu Sans}◇} \textit{} \colorBulletS{SYN} walk, hike, trek \\{\fontspec{DejaVu Sans}▪ }\textsf{\textit{verb}}\\ \textbf{1} Walk for pleasure in the countryside. {\fontspec{DejaVu Sans}◇} \textit{I spent most of my spare time rambling and climbing} \colorBulletS{SYN} walk, take a walk, go for a walk, hike, tramp, backpack, trek \textbf{2} Talk or write at length in a confused or inconsequential way. {\fontspec{DejaVu Sans}◇} \textit{Willy rambled on about Norman archways} \colorBulletS{SYN} chatter, babble, prattle, prate, blather, blether, gabble, jabber, twitter, go on, run on, rattle away, rattle on, blither, maunder, drivel \textbf{3} (of a plant) put out long shoots and grow over walls or other plants. {\fontspec{DejaVu Sans}◇} \textit{roses climbed, rambled, hung over walls}}{ \colorBullet{OTHER} ramble on}{}{ \colorBullet{ORIGIN} Late Middle English (in ramble (sense 2 of the verb)): probably related to Middle Dutch rammelen, used of animals in the sense ‘wander about on heat’, also to the noun ram.}%
\par%
\entry{rambling}{/ˈramblɪŋ/}{অসংলগ্ন}{\small{\textsf{\textit{adjective, noun}}} \\{\fontspec{DejaVu Sans}▪ }\textsf{\textit{adjective}}\\ \textbf{1} (of writing or speech) lengthy and confused or inconsequential. {\fontspec{DejaVu Sans}◇} \textit{a rambling six{-}hour speech} \colorBulletS{SYN} long{-}winded, garrulous, verbose, wordy, prolix \textbf{2} (of a plant) putting out long shoots and growing over walls or other plants. {\fontspec{DejaVu Sans}◇} \textit{rambling roses} \colorBulletS{SYN} trailing, creeping, straggling, vining, prostrate \\{\fontspec{DejaVu Sans}▪ }\textsf{\textit{noun}}\\ \textbf{1} The activity of walking in the countryside for pleasure. {\fontspec{DejaVu Sans}◇} \textit{a rambling club}}{}{}{}%
\par%
\entry{rampant}{/ˈramp(ə)nt/}{প্রচণ্ড}{ \textsf{\textit{adjective}}\ \textbf{1} (especially of something unwelcome) flourishing or spreading unchecked. {\fontspec{DejaVu Sans}◇} \textit{political violence was rampant} \colorBulletS{SYN} uncontrolled, unrestrained, unchecked, unbridled, widespread, pandemic, epidemic, pervasive \textbf{2} (of an animal) represented standing on one hind foot with its forefeet in the air (typically in profile, facing the dexter side, with right hind foot and tail raised) {\fontspec{DejaVu Sans}◇} \textit{two gold lions rampant} \colorBulletS{SYN} upright, standing, standing up, erect, rearing, vertical, perpendicular, upended, on end}{}{}{ \colorBullet{ORIGIN} Middle English (as a heraldic term): from Old French, literally ‘crawling’, present participle of ramper (see ramp). From the original use describing a wild animal arose the sense ‘fierce’, whence the current notion of ‘unrestrained’.}%
\par%
\entry{randy}{/ˈrandi/}{কামুক}{ \textsf{\textit{adjective}}\ \textbf{1} Sexually aroused or excited. {\fontspec{DejaVu Sans}◇} \textit{as nervous as a randy adolescent on a hot date} \colorBulletS{SYN} aroused, sexually excited, amorous, lustful, passionate \textbf{2} Having a rude, aggressive manner. {\fontspec{DejaVu Sans}◇} \textit{}}{}{}{ \colorBullet{ORIGIN} Mid 17th century perhaps from obsolete rand ‘rant, rave’, from obsolete Dutch randen ‘to rant’.}%
\par%
\entry{rash}{/raʃ/}{ফুসকুড়ি}{ \textsf{\textit{adjective}}\ \textbf{1} Acting or done without careful consideration of the possible consequences; impetuous. {\fontspec{DejaVu Sans}◇} \textit{it would be extremely rash to make such an assumption} \colorBulletS{SYN} reckless, impetuous, impulsive, hasty, overhasty, foolhardy, incautious, precipitate, precipitous, premature, careless, heedless, thoughtless, imprudent, foolish, headstrong, adventurous, over{-}adventurous, hot{-}headed, daredevil, devil{-}may{-}care, overbold, audacious, indiscreet}{}{}{ \colorBullet{ORIGIN} Late Middle English (also in Scots and northern English in the sense ‘nimble, eager’): of Germanic origin; related to German rasch.}%
\par%
\entry{rash}{/raʃ/}{ফুসকুড়ি}{ \textsf{\textit{noun}}\ \textbf{1} An area of redness and spots on a person's skin, appearing especially as a result of illness. {\fontspec{DejaVu Sans}◇} \textit{a red itchy rash appeared on her legs} \colorBulletS{SYN} spots, skin eruption, breakout \textbf{2} A series of things of the same type, especially when unwelcome, happening within a short space of time. {\fontspec{DejaVu Sans}◇} \textit{a rash of strikes by health service workers} \colorBulletS{SYN} series, succession}{}{}{ \colorBullet{ORIGIN} Early 18th century probably related to Old French rasche ‘eruptive sores, scurf’; compare with Italian raschia ‘itch’.}%
\par%
\entry{rashly}{/ˈraʃli/}{ত্বরায়}{ \textsf{\textit{adverb}}\ \textbf{1} Without careful consideration of the possible consequences; impetuously. {\fontspec{DejaVu Sans}◇} \textit{he rashly promised crime would fall sharply by September}}{}{}{}%
\par%
\entry{raucous}{/ˈrɔːkəs/}{কর্কশ}{ \textsf{\textit{adjective}}\ \textbf{1} Making or constituting a disturbingly harsh and loud noise. {\fontspec{DejaVu Sans}◇} \textit{raucous youths} \colorBulletS{SYN} harsh, strident, screeching, squawky, squawking, sharp, grating, discordant, dissonant, inharmonious, unmelodious, jarring, brassy}{}{}{ \colorBullet{ORIGIN} Mid 18th century from Latin raucus ‘hoarse’ + {-}ous.}%
\par%
\entry{ravage}{/ˈravɪdʒ/}{লুটপাট}{\small{\textsf{\textit{noun, verb}}} \\{\fontspec{DejaVu Sans}▪ }\textsf{\textit{noun}}\\ \textbf{1} The destructive effects of something. {\fontspec{DejaVu Sans}◇} \textit{his face had withstood the ravages of time} \colorBulletS{SYN} damaging effects, ill effects, scars \\{\fontspec{DejaVu Sans}▪ }\textsf{\textit{verb}}\\ \textbf{1} Cause severe and extensive damage to. {\fontspec{DejaVu Sans}◇} \textit{the hurricane ravaged southern Florida} \colorBulletS{SYN} lay waste, devastate, ruin, leave in ruins, destroy, wreak havoc on, leave desolate, level, raze, demolish, wipe out, wreck, damage}{}{}{ \colorBullet{ORIGIN} Early 17th century from French ravager, from earlier ravage, alteration of ravine ‘rush of water’.}%
\par%
\entry{raze}{/reɪz/}{}{ \textsf{\textit{verb}}\ \textbf{1} Completely destroy (a building, town, or other settlement) {\fontspec{DejaVu Sans}◇} \textit{villages were razed to the ground} \colorBulletS{SYN} destroy, demolish, raze to the ground, tear down, pull down, knock down, knock to pieces, level, flatten, bulldoze, fell, wipe out, lay waste, ruin, wreck}{}{}{ \colorBullet{ORIGIN} Middle English (in the sense ‘scratch, incise’): from Old French raser ‘shave closely’, from Latin ras{-} ‘scraped’, from the verb radere.}%
\par%
\entry{rearguard}{/ˈrɪəɡɑːd/}{পশ্চাদ্ভাগরক্ষী সৈনিকগণ}{ \textsf{\textit{noun}}\ \textbf{1} The soldiers at the rear of a body of troops, especially those protecting a retreating army. {\fontspec{DejaVu Sans}◇} \textit{the firing from our rearguard had stopped}}{}{}{ \colorBullet{ORIGIN} Late Middle English (denoting the rear part of an army): from Old French rereguarde.}%
\par%
\entry{reasonable}{/ˈriːz(ə)nəb(ə)l/}{যৌক্তিক}{ \textsf{\textit{adjective}}\ \textbf{1} Having sound judgement; fair and sensible. {\fontspec{DejaVu Sans}◇} \textit{no reasonable person could have objected} \colorBulletS{SYN} sensible, rational, open to reason, full of common sense, logical, fair, fair{-}minded, just, equitable, decent \textbf{2} As much as is appropriate or fair; moderate. {\fontspec{DejaVu Sans}◇} \textit{a police officer may use reasonable force to gain entry} \colorBulletS{SYN} within reason, practicable, sensible}{}{}{ \colorBullet{ORIGIN} Middle English from Old French raisonable, suggested by Latin rationabilis ‘rational’, from ratio (see reason).}%
\par%
\entry{reassure}{/riːəˈʃɔː/}{প্রত্যয় জন্মান}{ \textsf{\textit{verb}}\ \textbf{1} Say or do something to remove the doubts or fears of (someone) {\fontspec{DejaVu Sans}◇} \textit{he understood her feelings and tried to reassure her} \colorBulletS{SYN} put someone's mind at rest, set someone's mind at rest, dispel someone's fears, bolster someone's confidence, restore someone's confidence, raise someone's spirits, put someone at ease, encourage, hearten, buoy up, cheer up}{}{}{}%
\par%
\entry{rebel}{/ˈrɛb(ə)l/}{বিদ্রোহী}{\small{\textsf{\textit{noun, verb}}} \\{\fontspec{DejaVu Sans}▪ }\textsf{\textit{noun}}\\ \textbf{1} A person who rises in opposition or armed resistance against an established government or leader. {\fontspec{DejaVu Sans}◇} \textit{Tory rebels} \colorBulletS{SYN} revolutionary, insurgent, revolutionist, mutineer, agitator, subversive, guerrilla, anarchist, terrorist \\{\fontspec{DejaVu Sans}▪ }\textsf{\textit{verb}}\\ \textbf{1} Rise in opposition or armed resistance to an established government or leader. {\fontspec{DejaVu Sans}◇} \textit{the Earl of Pembroke subsequently rebelled against Henry III} \colorBulletS{SYN} revolt, mutiny, riot, rise up, rise up in arms, take up arms, mount a rebellion, stage a rebellion, take to the streets, defy the authorities, refuse to obey orders, be insubordinate}{}{}{ \colorBullet{ORIGIN} Middle English from Old French rebelle (noun), rebeller (verb), from Latin rebellis (used originally with reference to a fresh declaration of war by the defeated), based on bellum ‘war’.}%
\par%
\entry{rebellious}{/rɪˈbɛljəs/}{বিদ্রোহী}{ \textsf{\textit{adjective}}\ \textbf{1} Showing a desire to resist authority, control, or convention. {\fontspec{DejaVu Sans}◇} \textit{I became very rebellious and opted out} \colorBulletS{SYN} defiant, disobedient, insubordinate, unruly, ungovernable, unmanageable, uncontrollable, turbulent, mutinous, wayward, obstreperous, recalcitrant, refractory, intractable, resistant, dissentient, disaffected, malcontent}{}{}{}%
\par%
\entry{rebound}{/rɪˈbaʊnd/}{প্রতিক্ষেপ}{\small{\textsf{\textit{noun, verb}}} \\{\fontspec{DejaVu Sans}▪ }\textsf{\textit{noun}}\\ \textbf{1} (in sporting contexts) a ball or shot that bounces back after striking a hard surface. {\fontspec{DejaVu Sans}◇} \textit{he blasted the rebound into the net} \textbf{2} An increase in value, amount, or strength after a previous decline. {\fontspec{DejaVu Sans}◇} \textit{they revealed a big rebound in profits for last year} \\{\fontspec{DejaVu Sans}▪ }\textsf{\textit{verb}}\\ \textbf{1} Bounce back through the air after hitting something hard. {\fontspec{DejaVu Sans}◇} \textit{his shot hammered into the post and rebounded across the goal} \colorBulletS{SYN} bounce, bounce back, spring back, ricochet, boomerang, glance, recoil \textbf{2} Recover in value, amount, or strength after a decrease or decline. {\fontspec{DejaVu Sans}◇} \textit{the Share Index rebounded to show a twenty{-}point gain} \colorBulletS{SYN} recover, rally, bounce back, pick up, make a recovery, make a comeback \textbf{3} (of an event or action) have an unexpected adverse consequence for (someone, especially the person responsible for it) {\fontspec{DejaVu Sans}◇} \textit{Nicholas's tricks are rebounding on him} \colorBulletS{SYN} backfire on, misfire on, boomerang on, have an adverse effect on, have unwelcome repercussions for, come back on, be self{-}defeating for, cause one to be hoist with one's own petard}{}{}{ \colorBullet{ORIGIN} Late Middle English from Old French rebondir, from re{-} ‘back’ + bondir ‘bounce up’.}%
\par%
\entry{rebound}{/riːˈbaʊnd/}{প্রতিক্ষেপ}{\small{\textsf{\textit{}}}}{}{}{}%
\par%
\entry{rebuke}{/rɪˈbjuːk/}{তাড়ন}{\small{\textsf{\textit{noun, verb}}} \\{\fontspec{DejaVu Sans}▪ }\textsf{\textit{noun}}\\ \textbf{1} An expression of sharp disapproval or criticism. {\fontspec{DejaVu Sans}◇} \textit{he hadn't meant it as a rebuke, but Neil flinched} \colorBulletS{SYN} reprimand, reproach, reproof, scolding, admonishment, admonition, reproval, remonstration, lecture, upbraiding, castigation, lambasting, criticism, censure \\{\fontspec{DejaVu Sans}▪ }\textsf{\textit{verb}}\\ \textbf{1} Express sharp disapproval or criticism of (someone) because of their behaviour or actions. {\fontspec{DejaVu Sans}◇} \textit{she had rebuked him for drinking too much} \colorBulletS{SYN} reprimand, reproach, scold, admonish, reprove, remonstrate with, chastise, chide, upbraid, berate, take to task, pull up, castigate, lambaste, read someone the Riot Act, give someone a piece of one's mind, haul over the coals, criticize, censure}{}{}{ \colorBullet{ORIGIN} Middle English (originally in the sense ‘force back, repress’): from Anglo{-}Norman French and Old Northern French rebuker, from re{-} ‘back, down’ + bukier ‘to beat’ (originally ‘cut down wood’, from Old French busche ‘log’).}%
\par%
\entry{recall}{/rɪˈkɔːl/}{প্রত্যাহার}{\small{\textsf{\textit{noun, verb}}} \\{\fontspec{DejaVu Sans}▪ }\textsf{\textit{noun}}\\ \textbf{1} The action or faculty of remembering something learned or experienced. {\fontspec{DejaVu Sans}◇} \textit{people's understanding and subsequent recall of stories or events} \colorBulletS{SYN} recollection, memory, remembrance \textbf{2} An act or instance of officially recalling someone or something. {\fontspec{DejaVu Sans}◇} \textit{a recall of Parliament} \colorBulletS{SYN} summoning back, ordering back, calling back \textbf{3} The proportion of the number of relevant documents retrieved from a database in response to an inquiry. {\fontspec{DejaVu Sans}◇} \textit{expert systems can produce solutions with the speed, recall, accuracy, and consistency that only a computer can provide} \\{\fontspec{DejaVu Sans}▪ }\textsf{\textit{verb}}\\ \textbf{1} Bring (a fact, event, or situation) back into one's mind; remember. {\fontspec{DejaVu Sans}◇} \textit{I can still vaguely recall being taken to the hospital} \colorBulletS{SYN} remember, recollect, call to mind, think of \textbf{2} Officially order (someone) to return to a place. {\fontspec{DejaVu Sans}◇} \textit{the Panamanian ambassador was recalled from Peru} \colorBulletS{SYN} summon back, order back, call back, bring back}{}{}{ \colorBullet{ORIGIN} Late 16th century (as a verb): from re{-}‘again’ + call, suggested by Latin revocare or French rappeler ‘call back’.}%
\par%
\entry{recede}{/rɪˈsiːd/}{ফিরিয়া যাত্তয়া}{ \textsf{\textit{verb}}\ \textbf{1} Go or move back or further away from a previous position. {\fontspec{DejaVu Sans}◇} \textit{the floodwaters had receded} \colorBulletS{SYN} retreat, go back, move back, move further off, move away, withdraw \textbf{2} (of a quality, feeling, or possibility) gradually diminish. {\fontspec{DejaVu Sans}◇} \textit{the prospects of an early end to the war receded} \colorBulletS{SYN} diminish, lessen, grow less, decrease, dwindle, fade, abate, subside, ebb, wane, fall off, taper off, peter out, shrink \textbf{3} (of a man's hair) cease to grow at the temples and above the forehead. {\fontspec{DejaVu Sans}◇} \textit{his dark hair was receding a little}}{}{}{ \colorBullet{ORIGIN} Late 15th century (in the sense ‘depart from a usual state or standard’): from Latin recedere, from re{-} ‘back’ + cedere ‘go’.}%
\par%
\entry{reciprocity}{/ˌrɛsɪˈprɒsɪti/}{ক্রিয়া{-}প্রতিক্রিয়া}{ \textsf{\textit{noun}}\ \textbf{1} The practice of exchanging things with others for mutual benefit, especially privileges granted by one country or organization to another. {\fontspec{DejaVu Sans}◇} \textit{the Community intends to start discussions on reciprocity with third countries} \colorBulletS{SYN} exchange, trade, trade{-}off, swap, switch, barter, substitute, substitution, reciprocity, reciprocation, return, payment, remuneration, amends, compensation, indemnity, recompense, restitution, reparation, satisfaction}{}{}{ \colorBullet{ORIGIN} Mid 18th century from French réciprocité, from réciproque, from Latin reciprocus ‘moving backwards and forwards’ (see reciprocate).}%
\par%
\entry{reckon}{/ˈrɛk(ə)n/}{শ্রেণীভুক্ত করা}{ \textsf{\textit{verb}}\ \textbf{1} Establish by calculation. {\fontspec{DejaVu Sans}◇} \textit{his debts were reckoned at £300,000} \colorBulletS{SYN} calculate, compute, work out, put a figure on, figure, number, quantify \textbf{2} Be of the opinion. {\fontspec{DejaVu Sans}◇} \textit{he reckons that the army should pull out entirely} \colorBulletS{SYN} believe, think, be of the opinion, be of the view, be convinced, suspect, dare say, have an idea, have a feeling, imagine, fancy, guess, suppose, assume, surmise, conjecture, consider \textbf{3} Rely on or be sure of. {\fontspec{DejaVu Sans}◇} \textit{they had reckoned on a day or two more of privacy} \colorBulletS{SYN} rely on, depend on, count on, place reliance on, bargain on, plan on, reckon on, calculate on, presume on}{}{}{ \colorBullet{ORIGIN} Old English (ge)recenian ‘recount, relate’, of West Germanic origin; related to Dutch rekenen and German rechnen ‘to count (up)’. Early senses included ‘give an account of items received’ and ‘mention things in order’, which gave rise to the notion of ‘calculation’ and hence of ‘being of an opinion’.}%
\par%
\entry{recognize}{/ˈrɛkəɡnʌɪz/}{চেনা}{ \textsf{\textit{verb}}\ \textbf{1} Identify (someone or something) from having encountered them before; know again. {\fontspec{DejaVu Sans}◇} \textit{I recognized her when her wig fell off} \textbf{2} Acknowledge the existence, validity, or legality of. {\fontspec{DejaVu Sans}◇} \textit{the defence is recognized in British law} \colorBulletS{SYN} acknowledge, accept, admit, concede, allow, grant, confess, own}{}{}{ \colorBullet{ORIGIN} Late Middle English (earliest attested as a term in Scots law): from Old French reconniss{-}, stem of reconnaistre, from Latin recognoscere ‘know again, recall to mind’, from re{-} ‘again’ + cognoscere ‘learn’.}%
\par%
\entry{red tape}{}{আমলাতন্ত্র}{\small{\textsf{\textit{}}}}{}{1. Just because of red tape, a container full of relief materials donated by the indian navy for the victims of cyclone mora has been lying abandoned at the chittagong port for nearly seven months.}{}%
\par%
\entry{redeem}{/rɪˈdiːm/}{খালাস করা; মুক্ত করা}{ \textsf{\textit{verb}}\ \textbf{1} Compensate for the faults or bad aspects of. {\fontspec{DejaVu Sans}◇} \textit{a disappointing debate redeemed only by an outstanding speech} \colorBulletS{SYN} compensating, compensatory, extenuating, offsetting, qualifying, redemptive \textbf{2} Gain or regain possession of (something) in exchange for payment. {\fontspec{DejaVu Sans}◇} \textit{statutes enabled state peasants to redeem their land} \colorBulletS{SYN} retrieve, regain, recover, get back, reclaim, repossess, have something returned, rescue \textbf{3} Fulfil or carry out (a pledge or promise) {\fontspec{DejaVu Sans}◇} \textit{the party prepared to redeem the pledges of the past three years} \colorBulletS{SYN} fulfil, carry out, discharge, make good, execute}{}{1. We will redeem the old promise 2. Bangladesh's footballers will get a chance to redeem themselves after a disappointing show}{ \colorBullet{ORIGIN} Late Middle English (in the sense ‘buy back’): from Old French redimer or Latin redimere, from re{-} ‘back’ + emere ‘buy’.}%
\par%
\entry{redundant}{/rɪˈdʌnd(ə)nt/}{প্রয়োজনাতিরিক্ত; আপৎকালীন}{ \textsf{\textit{adjective}}\ \textbf{1} Not or no longer needed or useful; superfluous. {\fontspec{DejaVu Sans}◇} \textit{an appropriate use for a redundant church} \colorBulletS{SYN} unnecessary, not required, inessential, unessential, needless, unneeded, uncalled for, dispensable, disposable, expendable, unwanted, useless}{}{}{ \colorBullet{ORIGIN} Late 16th century (in the sense ‘abundant’): from Latin redundant{-} ‘surging up’, from the verb redundare (see redound).}%
\par%
\entry{reef}{/riːf/}{প্রবালপ্রাচীর}{ \textsf{\textit{noun}}\ \textbf{1} A ridge of jagged rock, coral, or sand just above or below the surface of the sea. {\fontspec{DejaVu Sans}◇} \textit{} \colorBulletS{SYN} shoal, bar, sandbar, sandbank, spit}{}{}{ \colorBullet{ORIGIN} Late 16th century (earlier as riff): from Middle Low German and Middle Dutch rif, ref, from Old Norse rif, literally ‘rib’, used in the same sense; compare with reef.}%
\par%
\entry{reef}{/riːf/}{প্রবালপ্রাচীর}{\small{\textsf{\textit{noun, verb}}} \\{\fontspec{DejaVu Sans}▪ }\textsf{\textit{noun}}\\ \textbf{1} Each of the several strips across a sail which can be taken in or rolled up to reduce the area exposed to the wind. {\fontspec{DejaVu Sans}◇} \textit{We had to sail her with ‘two reefs in’, a reduced sail area for the rough conditions.} \\{\fontspec{DejaVu Sans}▪ }\textsf{\textit{verb}}\\ \textbf{1} Take in one or more reefs of (a sail) {\fontspec{DejaVu Sans}◇} \textit{reef the mainsail in strong winds}}{}{}{ \colorBullet{ORIGIN} Middle English from Middle Dutch reef, rif, from Old Norse rif, literally ‘rib’, used in the same sense; compare with reef.}%
\par%
\entry{reel}{/riːl/}{ঘুরপাক}{\small{\textsf{\textit{noun, verb}}} \\{\fontspec{DejaVu Sans}▪ }\textsf{\textit{noun}}\\ \textbf{1} A cylinder on which film, wire, thread, or other flexible materials can be wound. {\fontspec{DejaVu Sans}◇} \textit{a cotton reel} \textbf{2} A lively Scottish or Irish folk dance. {\fontspec{DejaVu Sans}◇} \textit{we put on the record player and danced reels} \\{\fontspec{DejaVu Sans}▪ }\textsf{\textit{verb}}\\ \textbf{1} Wind something on to a reel by turning the reel. {\fontspec{DejaVu Sans}◇} \textit{sailplanes are often launched by means of a wire reeled in by a winch} \textbf{2} Lose one's balance and stagger or lurch violently. {\fontspec{DejaVu Sans}◇} \textit{he punched Connolly in the ear, sending him reeling} \colorBulletS{SYN} stagger, lurch, sway, rock, stumble, totter, wobble, falter, waver, swerve, pitch, roll \textbf{3} Dance a reel. {\fontspec{DejaVu Sans}◇} \textit{Anyone who wanted to dance could reel to the sound of the ceilidh band playing at the Butter Cross.}}{}{}{ \colorBullet{ORIGIN} Old English hrēol, denoting a rotatory device on which spun thread is wound; of unknown origin.}%
\par%
\entry{referendum}{/ˌrɛfəˈrɛndəm/}{গণভোট}{ \textsf{\textit{noun}}\ \textbf{1} A general vote by the electorate on a single political question which has been referred to them for a direct decision. {\fontspec{DejaVu Sans}◇} \textit{} \colorBulletS{SYN} public vote, plebiscite, popular vote, ballot, poll}{}{}{ \colorBullet{ORIGIN} Mid 19th century from Latin, gerund (‘referring’) or neuter gerundive (‘something to be brought back or referred’) of referre (see refer).}%
\par%
\entry{refrain}{/rɪˈfreɪn/}{বিরত থাকা}{ \textsf{\textit{verb}}\ \textbf{1} Stop oneself from doing something. {\fontspec{DejaVu Sans}◇} \textit{she refrained from comment} \colorBulletS{SYN} abstain, desist, hold back, stop oneself, withhold}{}{Per our roommate agreement, kindly refrain, from raucous laughter.}{ \colorBullet{ORIGIN} Middle English (in the sense ‘restrain a thought or feeling’): from Old French refrener, from Latin refrenare, from re{-} (expressing intensive force) + frenum ‘bridle’.}%
\par%
\entry{refrain}{/rɪˈfreɪn/}{বিরত থাকা}{ \textsf{\textit{noun}}\ \textbf{1} A repeated line or number of lines in a poem or song, typically at the end of each verse. {\fontspec{DejaVu Sans}◇} \textit{}}{}{Per our roommate agreement, kindly refrain, from raucous laughter.}{ \colorBullet{ORIGIN} Late Middle English from Old French, from refraindre ‘break’, based on Latin refringere ‘break up’ (because the refrain ‘broke’ the sequence).}%
\par%
\entry{refute}{/rɪˈfjuːt/}{খণ্ডন করা}{ \textsf{\textit{verb}}\ \textbf{1} Prove (a statement or theory) to be wrong or false; disprove. {\fontspec{DejaVu Sans}◇} \textit{these claims have not been convincingly refuted} \colorBulletS{SYN} disprove, prove false, prove wrong, prove to be false, prove to be wrong, show to be false, show to be wrong, rebut, confute, give the lie to, demolish, explode, debunk, drive a coach and horses through, discredit, invalidate}{}{}{ \colorBullet{ORIGIN} Mid 16th century : from Latin refutare ‘repel, rebut’.}%
\par%
\entry{regard}{/rɪˈɡɑːd/}{গণ্য করা}{\small{\textsf{\textit{noun, verb}}} \\{\fontspec{DejaVu Sans}▪ }\textsf{\textit{noun}}\\ \textbf{1} Attention to or concern for something. {\fontspec{DejaVu Sans}◇} \textit{the court must have regard to the principle of welfare} \colorBulletS{SYN} consideration, care, concern, sympathy, thought, mind, notice, heed, attention, interest \textbf{2} Best wishes (used to express friendliness in greetings) {\fontspec{DejaVu Sans}◇} \textit{give her my regards} \colorBulletS{SYN} best wishes, good wishes, greetings, kind regards, kindest regards, felicitations, salutations, respects, compliments, best, love \\{\fontspec{DejaVu Sans}▪ }\textsf{\textit{verb}}\\ \textbf{1} Consider or think of in a specified way. {\fontspec{DejaVu Sans}◇} \textit{she regarded London as her base} \colorBulletS{SYN} consider, look on, view, see, hold, think, think of, contemplate, count, judge, deem, estimate, evaluate, interpret, appraise, assess, make of, find, put down as, take for, account, reckon, treat, adjudge, size up, value, rate, gauge, sum up, weigh up \textbf{2} (of a thing) relate to; concern. {\fontspec{DejaVu Sans}◇} \textit{if these things regarded only myself, I could stand it with composure} \colorBulletS{SYN} apply to, be relevant to, have relevance to, concern, refer to, have reference to, belong to, pertain to, be pertinent to, have to do with, bear on, have a bearing on, appertain to, affect, involve, cover, touch}{}{}{ \colorBullet{ORIGIN} Middle English from Old French regarder ‘to watch’, from re{-} ‘back’ (also expressing intensive force) + garder ‘to guard’.}%
\par%
\entry{regarding}{/rɪˈɡɑːdɪŋ/}{সংক্রান্ত}{ \textsf{\textit{preposition}}\ \textbf{1} In respect of; concerning. {\fontspec{DejaVu Sans}◇} \textit{your recent letter regarding the above proposal} \colorBulletS{SYN} concerning, as regards, with regard to, in regard to, with respect to, in respect of, with reference to, relating to, respecting, as for, as to, re, about, apropos, on the subject of, in connection with}{}{}{}%
\par%
\entry{regardless}{/rɪˈɡɑːdləs/}{নির্বিশেষে}{ \textsf{\textit{adverb}}\ \textbf{1} Despite the prevailing circumstances. {\fontspec{DejaVu Sans}◇} \textit{they were determined to carry on regardless} \colorBulletS{SYN} anyway, anyhow, in any case, nevertheless, nonetheless, notwithstanding, despite everything, in spite of everything, for all that, after everything, no matter what, even so, just the same, all the same, be that as it may, in any event, come what may, rain or shine, come rain or shine, whatever the cost}{}{}{}%
\par%
\entry{regime}{/reɪˈʒiːm/}{শাসন}{ \textsf{\textit{noun}}\ \textbf{1} A government, especially an authoritarian one. {\fontspec{DejaVu Sans}◇} \textit{ideological opponents of the regime} \colorBulletS{SYN} government, authorities, system of government, rule, reign, dominion, sovereignty, jurisdiction, authority, control, command, administration, establishment, direction, management, leadership \textbf{2} A system or ordered way of doing things. {\fontspec{DejaVu Sans}◇} \textit{detention centres with a very tough physical regime} \colorBulletS{SYN} system, arrangement, scheme, code}{}{}{ \colorBullet{ORIGIN} Late 15th century (in the sense ‘regimen’): French régime, from Latin regimen ‘rule’ (see regimen). Sense 1 dates from the late 18th century (with original reference to the Ancien Régime).}%
\par%
\entry{regret}{/rɪˈɡrɛt/}{আফসোস}{\small{\textsf{\textit{noun, verb}}} \\{\fontspec{DejaVu Sans}▪ }\textsf{\textit{noun}}\\ \textbf{1} A feeling of sadness, repentance, or disappointment over an occurrence or something that one has done or failed to do. {\fontspec{DejaVu Sans}◇} \textit{she expressed her regret at Virginia's death} \colorBulletS{SYN} sadness, sorrow, disappointment, dismay, unhappiness, dejection, lamentation, grief, mourning, mournfulness \\{\fontspec{DejaVu Sans}▪ }\textsf{\textit{verb}}\\ \textbf{1} Feel sad, repentant, or disappointed over (something that one has done or failed to do) {\fontspec{DejaVu Sans}◇} \textit{she immediately regretted her words} \colorBulletS{SYN} be sorry about, feel contrite about, feel apologetic about, feel remorse about, feel remorse for, be remorseful about, rue, repent, repent of, feel repentant about, be regretful about, be regretful at, have a conscience about, blame oneself for}{}{}{ \colorBullet{ORIGIN} Late Middle English from Old French regreter ‘bewail (the dead)’, perhaps from the Germanic base of greet.}%
\par%
\entry{regrettable}{/rɪˈɡrɛtəb(ə)l/}{অনুশোচীয়}{ \textsf{\textit{adjective}}\ \textbf{1} (of conduct or an event) giving rise to regret; undesirable; unwelcome. {\fontspec{DejaVu Sans}◇} \textit{the loss of this number of jobs is regrettable} \colorBulletS{SYN} undesirable, unfortunate, unwelcome, sad, sorry, woeful, disappointing, distressing, too bad}{}{}{}%
\par%
\entry{rehab}{/ˈriːhab/}{}{\small{\textsf{\textit{noun, verb}}} \\{\fontspec{DejaVu Sans}▪ }\textsf{\textit{noun}}\\ \textbf{1} A course of treatment for drug or alcohol dependence, typically at a residential facility. {\fontspec{DejaVu Sans}◇} \textit{the star has been in rehab for a week} \textbf{2} A building that has been rehabilitated or restored. {\fontspec{DejaVu Sans}◇} \textit{a homeowner who discovers his rehab straddles the San Andreas fault} \colorBulletS{SYN} repair, repairing, fixing, mending, refurbishment, reconditioning, rehabilitation, rebuilding, reconstruction, remodelling, redecoration, revamping, revamp, makeover, overhaul \textbf{3} Financial assistance provided by the Rehabilitation Department, established to support returned servicemen after the Second World War. {\fontspec{DejaVu Sans}◇} \textit{he'd had to bum around for a few years before approaching the Rehab} \\{\fontspec{DejaVu Sans}▪ }\textsf{\textit{verb}}\\ \textbf{1} Rehabilitate or restore. {\fontspec{DejaVu Sans}◇} \textit{they don't rehab you at all in jail} \colorBulletS{SYN} restore to health, restore to normality, reintegrate, readapt, retrain}{}{}{ \colorBullet{ORIGIN} 1940s abbreviation of rehabilitate and rehabilitation.}%
\par%
\entry{rehabilitate}{/riːhəˈbɪlɪteɪt/}{পুনর্বাসন করা}{ \textsf{\textit{verb}}\ \textbf{1} Restore (someone) to health or normal life by training and therapy after imprisonment, addiction, or illness. {\fontspec{DejaVu Sans}◇} \textit{helping to rehabilitate former criminals} \colorBulletS{SYN} restore to health, restore to normality, reintegrate, readapt, retrain}{}{}{ \colorBullet{ORIGIN} Late 16th century (earlier (late 15th century) as rehabilitation) (in the sense ‘restore to former privileges’): from medieval Latin rehabilitat{-}, from the verb rehabilitare(see re{-}, habilitate).}%
\par%
\entry{rehabilitation}{/riːəbɪlɪˈteɪʃ(ə)n/}{পুনর্বাসন}{ \textsf{\textit{noun}}\ \textbf{1} The action of restoring someone to health or normal life through training and therapy after imprisonment, addiction, or illness. {\fontspec{DejaVu Sans}◇} \textit{she underwent rehabilitation and was walking within three weeks}}{}{Rohingya rehabilitation project suicidal}{}%
\par%
\entry{reign}{/reɪn/}{রাজত্ব; আধিপত্য}{\small{\textsf{\textit{noun, verb}}} \\{\fontspec{DejaVu Sans}▪ }\textsf{\textit{noun}}\\ \textbf{1} The period of rule of a monarch. {\fontspec{DejaVu Sans}◇} \textit{the original chapel was built in the reign of Charles I} \colorBulletS{SYN} rule, sovereignty, monarchy \\{\fontspec{DejaVu Sans}▪ }\textsf{\textit{verb}}\\ \textbf{1} Hold royal office; rule as monarch. {\fontspec{DejaVu Sans}◇} \textit{Queen Elizabeth reigns over the UK} \colorBulletS{SYN} ruling, regnant}{}{}{ \colorBullet{ORIGIN} Middle English from Old French reignier ‘to reign’, reigne ‘kingdom’, from Latin regnum, related to rex, reg{-} ‘king’.}%
\par%
\entry{reject}{/rɪˈdʒɛkt/}{প্রত্যাবাসন করান}{\small{\textsf{\textit{noun, verb}}} \\{\fontspec{DejaVu Sans}▪ }\textsf{\textit{noun}}\\ \textbf{1} A person or thing dismissed as inadequate or unacceptable. {\fontspec{DejaVu Sans}◇} \textit{some of the team's rejects have gone on to prove themselves in championships} \colorBulletS{SYN} failure, loser, incompetent \\{\fontspec{DejaVu Sans}▪ }\textsf{\textit{verb}}\\ \textbf{1} Dismiss as inadequate, unacceptable, or faulty. {\fontspec{DejaVu Sans}◇} \textit{union negotiators rejected a 1.5 per cent pay award} \colorBulletS{SYN} banish, put away, set aside, lay aside, abandon, have done with, drop, disregard, brush off, shrug off, forget, think no more of, pay no heed to, put out of one's mind}{}{}{ \colorBullet{ORIGIN} Late Middle English from Latin reject{-} ‘thrown back’, from the verb reicere, from re{-} ‘back’ + jacere ‘to throw’.}%
\par%
\entry{rejoinder}{/rɪˈdʒɔɪndə/}{প্রতিবাদ}{ \textsf{\textit{noun}}\ \textbf{1} A reply, especially a sharp or witty one. {\fontspec{DejaVu Sans}◇} \textit{she would have made some cutting rejoinder but none came to mind} \colorBulletS{SYN} answer, reply, response, retort, riposte, counter, sally}{}{}{ \colorBullet{ORIGIN} Late Middle English from Anglo{-}Norman French rejoindre (infinitive used as a noun) (see rejoin).}%
\par%
\entry{reliant}{/rɪˈlʌɪənt/}{আস্থাবান; নির্ভরশীল}{ \textsf{\textit{adjective}}\ \textbf{1} Dependent on someone or something. {\fontspec{DejaVu Sans}◇} \textit{the company is heavily reliant on the baby market}}{}{}{}%
\par%
\entry{relief}{/rɪˈliːf/}{মুক্তি}{ \textsf{\textit{noun}}\ \textbf{1} A feeling of reassurance and relaxation following release from anxiety or distress. {\fontspec{DejaVu Sans}◇} \textit{much to her relief, she saw the door open} \colorBulletS{SYN} reassurance, consolation, comfort, solace, calmness, relaxation, repose, ease \textbf{2} Financial or practical assistance given to those in special need or difficulty. {\fontspec{DejaVu Sans}◇} \textit{raising money for famine relief} \colorBulletS{SYN} help, aid, assistance, succour, care, sustenance \textbf{3} A person or group of people replacing others who have been on duty. {\fontspec{DejaVu Sans}◇} \textit{the relief nurse was late} \colorBulletS{SYN} replacement, substitute, deputy, reserve, standby, stopgap, cover, stand{-}in, supply, fill{-}in, locum, locum tenens, understudy, proxy, surrogate \textbf{4} The state of being clearly visible or obvious due to being accentuated. {\fontspec{DejaVu Sans}◇} \textit{the setting sun threw the snow{-}covered peaks into relief}}{}{}{ \colorBullet{ORIGIN} Late Middle English from Old French, from relever ‘raise up, relieve’, from Latin relevare ‘raise again, alleviate’.}%
\par%
\entry{relocate}{/riːlə(ʊ)ˈkeɪt/}{নূতন স্থান নির্দেশ করা}{ \textsf{\textit{verb}}\ \textbf{1} Move to a new place and establish one's home or business there. {\fontspec{DejaVu Sans}◇} \textit{sixty workers could face redundancy because the firm is relocating} \colorBulletS{SYN} move, convey, shift, remove, take, carry, fetch, lift, bring, bear, conduct, send, pass on, transport, relay, change, relocate, resettle, transplant, uproot}{}{}{}%
\par%
\entry{reluctance}{/rɪˈlʌkt(ə)ns/}{অনিচ্ছা}{ \textsf{\textit{noun}}\ \textbf{1} Unwillingness or disinclination to do something. {\fontspec{DejaVu Sans}◇} \textit{she sensed his reluctance to continue} \colorBulletS{SYN} unwillingness, disinclination, lack of enthusiasm \textbf{2} The property of a magnetic circuit of opposing the passage of magnetic flux lines, equal to the ratio of the magnetomotive force to the magnetic flux. {\fontspec{DejaVu Sans}◇} \textit{}}{}{}{}%
\par%
\entry{reluctant}{/rɪˈlʌkt(ə)nt/}{অনিচ্ছুক}{ \textsf{\textit{adjective}}\ \textbf{1} Unwilling and hesitant; disinclined. {\fontspec{DejaVu Sans}◇} \textit{she seemed reluctant to answer} \colorBulletS{SYN} unwilling, disinclined, unenthusiastic, grudging, resistant, resisting, opposed, antipathetic}{}{}{ \colorBullet{ORIGIN} Mid 17th century (in the sense ‘writhing, offering opposition’): from Latin reluctant{-} ‘struggling against’, from the verb reluctari, from re{-} (expressing intensive force) + luctari ‘to struggle’.}%
\par%
\entry{remand}{/rɪˈmɑːnd/}{পুনঃপ্রেরণ}{\small{\textsf{\textit{noun, verb}}} \\{\fontspec{DejaVu Sans}▪ }\textsf{\textit{noun}}\\ \textbf{1} A committal to custody. {\fontspec{DejaVu Sans}◇} \textit{the prosecutor applied for a remand to allow forensic evidence to be investigated} \colorBulletS{SYN} custody, imprisonment, confinement, incarceration, internment, captivity, restraint, arrest, house arrest, remand, committal \\{\fontspec{DejaVu Sans}▪ }\textsf{\textit{verb}}\\ \textbf{1} Place (a defendant) on bail or in custody, especially when a trial is adjourned. {\fontspec{DejaVu Sans}◇} \textit{he was remanded in custody for a week} \colorBulletS{SYN} imprison, jail, incarcerate, send to prison, put behind bars, put under lock and key, put in chains, put into irons, throw into irons, clap in irons, hold captive}{}{}{ \colorBullet{ORIGIN} Late Middle English (as a verb in the sense ‘send back again’): from late Latin remandare, from re{-} ‘back’ + mandare ‘commit’. The noun dates from the late 18th century.}%
\par%
\entry{remark}{/rɪˈmɑːk/}{মন্তব্য}{\small{\textsf{\textit{noun, verb}}} \\{\fontspec{DejaVu Sans}▪ }\textsf{\textit{noun}}\\ \textbf{1} A written or spoken comment. {\fontspec{DejaVu Sans}◇} \textit{I decided to ignore his rude remarks} \\{\fontspec{DejaVu Sans}▪ }\textsf{\textit{verb}}\\ \textbf{1} Say something as a comment; mention. {\fontspec{DejaVu Sans}◇} \textit{‘Tom's looking peaky,’ she remarked} \colorBulletS{SYN} comment, say, observe, mention, reflect, state, declare, announce, pronounce, assert \textbf{2} Regard with attention; notice. {\fontspec{DejaVu Sans}◇} \textit{he remarked the man's inflamed eyelids} \colorBulletS{SYN} note, notice, observe, take note of, mark, perceive, discern}{}{}{ \colorBullet{ORIGIN} Late 16th century (in remark (sense 2 of the verb)): from French remarquer ‘note again’, from re{-} (expressing intensive force) + marquer ‘to mark, note’.}%
\par%
\entry{remedial}{/rɪˈmiːdɪəl/}{আরোগ্যকর}{ \textsf{\textit{adjective}}\ \textbf{1} Giving or intended as a remedy or cure. {\fontspec{DejaVu Sans}◇} \textit{remedial surgery} \colorBulletS{SYN} healing, curative, curing, remedial, medicinal, restorative, health{-}giving, tonic, sanative, reparative, corrective, ameliorative, beneficial, good, salubrious, salutary}{}{}{ \colorBullet{ORIGIN} Mid 17th century from late Latin remedialis, from Latin remedium ‘cure, medicine’ (see remedy).}%
\par%
\entry{remedy}{/ˈrɛmɪdi/}{প্রতিকার}{\small{\textsf{\textit{noun, verb}}} \\{\fontspec{DejaVu Sans}▪ }\textsf{\textit{noun}}\\ \textbf{1} A medicine or treatment for a disease or injury. {\fontspec{DejaVu Sans}◇} \textit{herbal remedies for aches and pains} \colorBulletS{SYN} treatment, cure, medicine, medication, medicament, drug, restorative \textbf{2} The margin within which coins as minted may differ from the standard fineness and weight. {\fontspec{DejaVu Sans}◇} \textit{} \\{\fontspec{DejaVu Sans}▪ }\textsf{\textit{verb}}\\ \textbf{1} Set right (an undesirable situation) {\fontspec{DejaVu Sans}◇} \textit{money will be given to remedy the poor funding of nurseries} \colorBulletS{SYN} put right, set right, set to rights, put to rights, right, rectify, retrieve, solve, fix, sort out, put in order, straighten out, resolve, deal with, correct, repair, mend, redress, make good}{}{}{ \colorBullet{ORIGIN} Middle English from Anglo{-}Norman French remedie, from Latin remedium, from re{-} ‘back’ (also expressing intensive force) + mederi ‘heal’.}%
\par%
\entry{repatriation}{/riːpatrɪˈeɪʃ(ə)n/}{প্রত্যাবাসন}{ \textsf{\textit{noun}}\ \textbf{1} The return of someone to their own country. {\fontspec{DejaVu Sans}◇} \textit{the voluntary repatriation of refugees}}{}{}{}%
\par%
\entry{repel}{/rɪˈpɛl/}{প্রতিরোধ করা}{ \textsf{\textit{verb}}\ \textbf{1} Drive or force (an attack or attacker) back or away. {\fontspec{DejaVu Sans}◇} \textit{government units sought to repel the rebels} \colorBulletS{SYN} fight off, repulse, drive away, drive back, put to flight, force back, beat back, push back, thrust back \textbf{2} Be repulsive or distasteful to. {\fontspec{DejaVu Sans}◇} \textit{she was repelled by the permanent smell of drink on his breath} \colorBulletS{SYN} revolt, disgust, repulse, sicken, nauseate, make someone feel sick, turn someone's stomach, be repulsive to, be extremely distasteful to, be repugnant to, make shudder, make someone's flesh creep, make someone's skin crawl, make someone's gorge rise, put off, offend, horrify \textbf{3} Refuse to accept (something, especially an argument or theory) {\fontspec{DejaVu Sans}◇} \textit{the alleged right of lien led by the bankrupt's solicitor was repelled} \colorBulletS{SYN} refuse, decline, say no to, reject, rebuff, scorn, turn down, turn away, repudiate, treat with contempt, disdain, look down one's nose at, despise}{}{}{ \colorBullet{ORIGIN} Late Middle English from Latin repellere, from re{-} ‘back’ + pellere ‘to drive’.}%
\par%
\entry{repetition}{/rɛpɪˈtɪʃ(ə)n/}{পুনরাবৃত্তি}{ \textsf{\textit{noun}}\ \textbf{1} The action of repeating something that has already been said or written. {\fontspec{DejaVu Sans}◇} \textit{her comments are worthy of repetition} \colorBulletS{SYN} reiteration, repeating, restatement, retelling, iteration, recapitulation \textbf{2} The recurrence of an action or event. {\fontspec{DejaVu Sans}◇} \textit{there was to be no repetition of the interwar years} \colorBulletS{SYN} recurrence, reoccurrence, repeat, rerun, replication, duplication}{}{}{ \colorBullet{ORIGIN} Late Middle English from Old French repeticion or Latin repetitio(n{-}), from repetere (see repeat).}%
\par%
\entry{reportedly}{/rɪˈpɔːtɪdli/}{জানা}{ \textsf{\textit{adverb}}\ \textbf{1} According to what some say (used to express the speaker's belief that the information given is not necessarily true) {\fontspec{DejaVu Sans}◇} \textit{he was in El Salvador, reportedly on his way to Texas} \colorBulletS{SYN} supposedly, seemingly, apparently, allegedly, reportedly, professedly, ostensibly, on the face of it, to all appearances, on the surface, to all intents and purposes, outwardly, superficially, purportedly, nominally, by its own account, by one's own account, on paper}{}{}{}%
\par%
\entry{reprieve}{/rɪˈpriːv/}{সাময়িক উপশম}{\small{\textsf{\textit{noun, verb}}} \\{\fontspec{DejaVu Sans}▪ }\textsf{\textit{noun}}\\ \textbf{1} A cancellation or postponement of a punishment. {\fontspec{DejaVu Sans}◇} \textit{he accepted the death sentence and refused to appeal for a reprieve} \colorBulletS{SYN} stay of execution, cancellation of punishment, postponement of punishment, remission, suspension of punishment, respite \\{\fontspec{DejaVu Sans}▪ }\textsf{\textit{verb}}\\ \textbf{1} Cancel or postpone the punishment of (someone, especially someone condemned to death) {\fontspec{DejaVu Sans}◇} \textit{under the new regime, prisoners under sentence of death were reprieved} \colorBulletS{SYN} grant a stay of execution to, cancel someone's punishment, commute someone's punishment, postpone someone's punishment, remit someone's punishment}{}{}{ \colorBullet{ORIGIN} Late 15th century (as the past participle repryed): from Anglo{-}Norman French repris, past participle of reprendre, from Latin re{-} ‘back’ + prehendere ‘seize’. The insertion of {-}v{-} (16th century) remains unexplained. Sense development has undergone a reversal, from the early meaning ‘send back to prison’, via ‘postpone a legal process’, to the current sense ‘rescue from impending punishment’.}%
\par%
\entry{reprisal}{/rɪˈprʌɪz(ə)l/}{প্রত্যধিকার}{ \textsf{\textit{noun}}\ \textbf{1} An act of retaliation. {\fontspec{DejaVu Sans}◇} \textit{three youths died in the reprisals which followed} \colorBulletS{SYN} retaliation, counterattack, counterstroke, comeback}{}{}{ \colorBullet{ORIGIN} Late Middle English from Anglo{-}Norman French reprisaille, from medieval Latin reprisalia (neuter plural), based on Latin repraehens{-} ‘seized’, from the verb repraehendere (see reprehend). The current sense dates from the early 18th century.}%
\par%
\entry{requisite}{/ˈrɛkwɪzɪt/}{প্রয়োজনীয়}{\small{\textsf{\textit{adjective, noun}}} \\{\fontspec{DejaVu Sans}▪ }\textsf{\textit{adjective}}\\ \textbf{1} Made necessary by particular circumstances or regulations. {\fontspec{DejaVu Sans}◇} \textit{the application will not be processed until the requisite fee is paid} \colorBulletS{SYN} necessary, required, prerequisite, essential, indispensable, vital, needed, needful \\{\fontspec{DejaVu Sans}▪ }\textsf{\textit{noun}}\\ \textbf{1} A thing that is necessary for the achievement of a specified end. {\fontspec{DejaVu Sans}◇} \textit{she believed privacy to be a requisite for a peaceful life} \colorBulletS{SYN} necessity, essential requirement, prerequisite, essential, precondition, specification, stipulation}{}{}{ \colorBullet{ORIGIN} Late Middle English from Latin requisitus ‘searched for, deemed necessary’, past participle of requirere (see require).}%
\par%
\entry{resilience}{/rɪˈzɪlɪəns/}{স্থিতিস্থাপকতা}{ \textsf{\textit{noun}}\ \textbf{1} The capacity to recover quickly from difficulties; toughness. {\fontspec{DejaVu Sans}◇} \textit{the often remarkable resilience of so many British institutions} \textbf{2} The ability of a substance or object to spring back into shape; elasticity. {\fontspec{DejaVu Sans}◇} \textit{nylon is excellent in wearability, abrasion resistance and resilience} \colorBulletS{SYN} flexibility, pliability, suppleness, plasticity, elasticity, springiness, spring, give}{}{}{}%
\par%
\entry{resist}{/rɪˈzɪst/}{প্রতিহত করা}{\small{\textsf{\textit{noun, verb}}} \\{\fontspec{DejaVu Sans}▪ }\textsf{\textit{noun}}\\ \textbf{1} A resistant substance applied as a coating to protect a surface during a process, for example to prevent dye or glaze adhering. {\fontspec{DejaVu Sans}◇} \textit{new lithographic techniques require their own special resists} \\{\fontspec{DejaVu Sans}▪ }\textsf{\textit{verb}}\\ \textbf{1} Withstand the action or effect of. {\fontspec{DejaVu Sans}◇} \textit{antibodies help us to resist infection} \colorBulletS{SYN} withstand, be proof against, hold out against, combat, counter}{}{}{ \colorBullet{ORIGIN} Late Middle English from Old French resister or Latin resistere, from re{-} (expressing opposition) + sistere ‘stop’ (reduplication of stare ‘to stand’). The current sense of the noun dates from the mid 19th century.}%
\par%
\entry{respite}{/ˈrɛspʌɪt/}{অবকাশ}{\small{\textsf{\textit{noun, verb}}} \\{\fontspec{DejaVu Sans}▪ }\textsf{\textit{noun}}\\ \textbf{1} A short period of rest or relief from something difficult or unpleasant. {\fontspec{DejaVu Sans}◇} \textit{the refugee encampments will provide some respite from the suffering} \colorBulletS{SYN} rest, break, breathing space, interval, intermission, interlude, recess, lull, pause, time out, hiatus, halt, stop, stoppage, cessation, discontinuation, standstill \\{\fontspec{DejaVu Sans}▪ }\textsf{\textit{verb}}\\ \textbf{1} Postpone (a sentence, obligation, etc.) {\fontspec{DejaVu Sans}◇} \textit{the execution was only respited a few months} \colorBulletS{SYN} postpone, put off, delay, defer, put back, hold off, hold over, carry over, reschedule, do later, shelve, stand over, pigeonhole, hold in abeyance, put in abeyance, mothball}{}{To find some respite from the suffocating heat}{ \colorBullet{ORIGIN} Middle English from Old French respit, from Latin respectus ‘refuge, consideration’.}%
\par%
\entry{restraint}{/rɪˈstreɪnt/}{বাধা}{ \textsf{\textit{noun}}\ \textbf{1} A measure or condition that keeps someone or something under control. {\fontspec{DejaVu Sans}◇} \textit{decisions are made within the financial restraints of the budget} \textbf{2} Unemotional, dispassionate, or moderate behaviour; self{-}control. {\fontspec{DejaVu Sans}◇} \textit{he urged the protestors to exercise restraint} \colorBulletS{SYN} self{-}control, self{-}restraint, self{-}discipline, control, moderation, temperateness, abstemiousness, non{-}indulgence, prudence, judiciousness}{}{}{ \colorBullet{ORIGIN} Late Middle English from Old French restreinte, feminine past participle of restreindre ‘hold back’ (see restrain).}%
\par%
\entry{retain}{/rɪˈteɪn/}{রাখা}{ \textsf{\textit{verb}}\ \textbf{1} Continue to have (something); keep possession of. {\fontspec{DejaVu Sans}◇} \textit{Labour retained the seat} \colorBulletS{SYN} keep, keep possession of, keep hold of, hold on to, hold fast to, keep back, hang on to, cling to \textbf{2} Absorb and continue to hold (a substance) {\fontspec{DejaVu Sans}◇} \textit{limestone is known to retain water} \textbf{3} Keep (something) in place; hold fixed. {\fontspec{DejaVu Sans}◇} \textit{remove the retaining bar} \textbf{4} Keep (someone) engaged in one's service. {\fontspec{DejaVu Sans}◇} \textit{he has been retained as a freelance} \colorBulletS{SYN} employ, commission, contract, pay, keep on the payroll, have in employment}{}{}{ \colorBullet{ORIGIN} Late Middle English via Anglo{-}Norman French from Old French retenir, from Latin retinere, from re{-} ‘back’ + tenere ‘hold’.}%
\par%
\entry{retract}{/rɪˈtrakt/}{প্রত্যাহার করা}{ \textsf{\textit{verb}}\ \textbf{1} Draw or be drawn back or back in. {\fontspec{DejaVu Sans}◇} \textit{she retracted her hand as if she'd been burnt} \colorBulletS{SYN} pull in, draw in, pull back, sheathe, put away \textbf{2} Withdraw (a statement or accusation) as untrue or unjustified. {\fontspec{DejaVu Sans}◇} \textit{he retracted his allegations} \colorBulletS{SYN} take back, withdraw, unsay, recant, disown, disavow, disclaim, abjure, repudiate, renounce, reverse, revoke, rescind, annul, cancel, go back on, backtrack on, do a U{-}turn on, row back on}{}{}{ \colorBullet{ORIGIN} Late Middle English from Latin retract{-} ‘drawn back’, from the verb retrahere (from re{-} ‘back’ + trahere ‘drag’); the senses ‘withdraw (a statement’) and ‘go back on’ via Old French from retractare ‘reconsider’ (based on trahere ‘drag’).}%
\par%
\entry{retreat}{/rɪˈtriːt/}{পশ্চাদপসরণ}{\small{\textsf{\textit{noun, verb}}} \\{\fontspec{DejaVu Sans}▪ }\textsf{\textit{noun}}\\ \textbf{1} An act of moving back or withdrawing. {\fontspec{DejaVu Sans}◇} \textit{a speedy retreat} \colorBulletS{SYN} withdrawal, pulling back, flight \textbf{2} A signal for a military force to withdraw. {\fontspec{DejaVu Sans}◇} \textit{the bugle sounded a retreat} \textbf{3} A quiet or secluded place in which one can rest and relax. {\fontspec{DejaVu Sans}◇} \textit{their country retreat in Ireland} \colorBulletS{SYN} refuge, haven, resort, asylum, sanctuary, sanctum sanctorum \textbf{4} A decline in the value of shares. {\fontspec{DejaVu Sans}◇} \textit{a gloomy stock market forecast sent share prices into a rapid retreat} \\{\fontspec{DejaVu Sans}▪ }\textsf{\textit{verb}}\\ \textbf{1} (of an army) withdraw from enemy forces as a result of their superior power or after a defeat. {\fontspec{DejaVu Sans}◇} \textit{the French retreated in disarray} \colorBulletS{SYN} withdraw, retire, draw back, pull back, pull out, fall back, give way, give ground, recoil, flee, take flight, beat a retreat, beat a hasty retreat, run away, run off, make a run for it, run for it, make off, take off, take to one's heels, make a break for it, bolt, make a quick exit, clear out, make one's getaway, escape, head for the hills}{}{}{ \colorBullet{ORIGIN} Late Middle English from Old French retret (noun), retraiter (verb), from Latin retrahere ‘pull back’ (see retract).}%
\par%
\entry{reveal}{/rɪˈviːl/}{প্রকাশ করা}{\small{\textsf{\textit{noun, verb}}} \\{\fontspec{DejaVu Sans}▪ }\textsf{\textit{noun}}\\ \textbf{1} (in a film or television programme) a final revelation of information that has previously been kept from the characters or viewers. {\fontspec{DejaVu Sans}◇} \textit{the big reveal at the end of the movie answers all questions} \\{\fontspec{DejaVu Sans}▪ }\textsf{\textit{verb}}\\ \textbf{1} Make (previously unknown or secret information) known to others. {\fontspec{DejaVu Sans}◇} \textit{Brenda was forced to reveal Robbie's whereabouts} \colorBulletS{SYN} divulge, disclose, tell, let out, let slip, let drop, let fall, give away, give the game away, give the show away, blurt, blurt out, babble, give out, release, leak, betray, open up, unveil, bring out into the open}{}{}{ \colorBullet{ORIGIN} Late Middle English from Old French reveler or Latin revelare, from re{-} ‘again’ (expressing reversal) + velum ‘veil’.}%
\par%
\entry{reveal}{/rɪˈviːl/}{প্রকাশ করা}{ \textsf{\textit{noun}}\ \textbf{1} Either side surface of an aperture in a wall for a door or window. {\fontspec{DejaVu Sans}◇} \textit{}}{}{}{ \colorBullet{ORIGIN} Late 17th century from obsolete revale ‘to lower’, from Old French revaler, from re{-} ‘back’ + avaler ‘go down, sink’.}%
\par%
\entry{revert}{/rɪˈvəːt/}{প্রত্যাবর্তন করা}{\small{\textsf{\textit{noun, verb}}} \\{\fontspec{DejaVu Sans}▪ }\textsf{\textit{noun}}\\ \textbf{1} A person who has converted to the Islamic faith. {\fontspec{DejaVu Sans}◇} \textit{I am a revert to Islam from a very orthodox Christian family.} \\{\fontspec{DejaVu Sans}▪ }\textsf{\textit{verb}}\\ \textbf{1} Return to (a previous state, practice, topic, etc.) {\fontspec{DejaVu Sans}◇} \textit{he reverted to his native language} \colorBulletS{SYN} return, go back, come back, change back, retrogress, regress, default \textbf{2} Reply or respond to someone. {\fontspec{DejaVu Sans}◇} \textit{we texted both Farah and Shirish, but neither of them reverted} \textbf{3} Turn (one's eyes or steps) back. {\fontspec{DejaVu Sans}◇} \textit{on reverting our eyes, every step presented some new and admirable scene}}{}{}{ \colorBullet{ORIGIN} Middle English from Old French revertir or Latin revertere ‘turn back’. Early senses included ‘recover consciousness’ and ‘return to a position’.}%
\par%
\entry{revive}{/rɪˈvʌɪv/}{পুনরায় জীবত করা}{ \textsf{\textit{verb}}\ \textbf{1} Restore to life or consciousness. {\fontspec{DejaVu Sans}◇} \textit{both men collapsed, but were revived} \colorBulletS{SYN} resuscitate, bring round, bring to life, bring back, bring someone to their senses, bring someone back to their senses, bring back to consciousness, bring back from the edge of death}{}{}{ \colorBullet{ORIGIN} Late Middle English from Old French revivre or late Latin revivere, from Latin re{-} ‘back’ + vivere ‘live’.}%
\par%
\entry{revok}{}{}{\small{\textsf{\textit{}}}}{}{}{}%
\par%
\entry{revoke}{/rɪˈvəʊk/}{রদ করা}{ \textsf{\textit{verb}}\ \textbf{1} Officially cancel (a decree, decision, or promise) {\fontspec{DejaVu Sans}◇} \textit{the men appealed and the sentence was revoked} \colorBulletS{SYN} cancel, repeal, rescind, reverse, abrogate, annul, nullify, declare null and void, make void, void, invalidate, render invalid, quash, abolish, set aside, countermand, retract, withdraw, overrule, override \textbf{2} (in bridge, whist, and other card games) fail to follow suit despite being able to do so. {\fontspec{DejaVu Sans}◇} \textit{}}{}{}{ \colorBullet{ORIGIN} Late Middle English from Old French revoquer or Latin revocare, from re{-} ‘back’ + vocare ‘to call’.}%
\par%
\entry{rhetoric}{/ˈrɛtərɪk/}{অলঙ্কারশাস্ত্র}{ \textsf{\textit{noun}}\ \textbf{1} The art of effective or persuasive speaking or writing, especially the exploitation of figures of speech and other compositional techniques. {\fontspec{DejaVu Sans}◇} \textit{he is using a common figure of rhetoric, hyperbole} \colorBulletS{SYN} oratory, eloquence, power of speech, command of language, expression, way with words, delivery, diction}{}{}{ \colorBullet{ORIGIN} Middle English from Old French rethorique, via Latin from Greek rhētorikē (tekhnē) ‘(art) of rhetoric’, from rhētōr ‘rhetor’.}%
\par%
\entry{riddle}{/ˈrɪd(ə)l/}{হেঁয়ালি}{\small{\textsf{\textit{noun, verb}}} \\{\fontspec{DejaVu Sans}▪ }\textsf{\textit{noun}}\\ \textbf{1} A question or statement intentionally phrased so as to require ingenuity in ascertaining its answer or meaning. {\fontspec{DejaVu Sans}◇} \textit{they started asking riddles and telling jokes} \\{\fontspec{DejaVu Sans}▪ }\textsf{\textit{verb}}\\ \textbf{1} Speak in or pose riddles. {\fontspec{DejaVu Sans}◇} \textit{he who knows not how to riddle}}{}{}{ \colorBullet{ORIGIN} Old English rǣdels, rǣdelse ‘opinion, conjecture, riddle’; related to Dutch raadsel, German Rätsel, also to read.}%
\par%
\entry{riddle}{/ˈrɪd(ə)l/}{হেঁয়ালি}{\small{\textsf{\textit{noun, verb}}} \\{\fontspec{DejaVu Sans}▪ }\textsf{\textit{noun}}\\ \textbf{1} A large coarse sieve, especially one used for separating ashes from cinders or sand from gravel. {\fontspec{DejaVu Sans}◇} \textit{For inside the mill, the shelling stones began to turn, the riddles (large{-}meshed sieves) rhythmically shook and the millstones ground round and round.} \\{\fontspec{DejaVu Sans}▪ }\textsf{\textit{verb}}\\ \textbf{1} Make many holes in (someone or something), especially with gunshot. {\fontspec{DejaVu Sans}◇} \textit{his car was riddled by sniper fire} \colorBulletS{SYN} perforate, hole, make holes in, punch holes in, put holes in, pierce, penetrate, puncture, honeycomb, pepper \textbf{2} Pass (a substance) through a large coarse sieve. {\fontspec{DejaVu Sans}◇} \textit{for final potting, the soil mixture is not riddled} \colorBulletS{SYN} sieve, sift, strain, screen, filter, purify, refine, winnow}{}{}{ \colorBullet{ORIGIN} Late Old English hriddel, of Germanic origin; from an Indo{-}European root shared by Latin cribrum ‘sieve’, cernere ‘separate’, and Greek krinein ‘decide’.}%
\par%
\entry{rival}{/ˈrʌɪv(ə)l/}{প্রতিদ্বন্দ্বী}{\small{\textsf{\textit{noun, verb}}} \\{\fontspec{DejaVu Sans}▪ }\textsf{\textit{noun}}\\ \textbf{1} A person or thing competing with another for the same objective or for superiority in the same field of activity. {\fontspec{DejaVu Sans}◇} \textit{he has no serious rival for the job} \colorBulletS{SYN} competitor, opponent, contestant, contender, challenger \\{\fontspec{DejaVu Sans}▪ }\textsf{\textit{verb}}\\ \textbf{1} Be or seem to be equal or comparable to. {\fontspec{DejaVu Sans}◇} \textit{he was a photographer whose fame rivalled that of his subjects} \colorBulletS{SYN} compete with, vie with, match, be a match for, equal, emulate, measure up to, come up to, compare with, bear comparison with, be comparable to, be comparable with, parallel, be in the same league as, be in the same category as, be on a par with, be on a level with, touch, keep pace with, keep up with}{}{}{ \colorBullet{ORIGIN} Late 16th century from Latin rivalis, originally in the sense ‘person using the same stream as another’, from rivus ‘stream’.}%
\par%
\entry{rivalry}{/ˈrʌɪv(ə)lri/}{দ্বন্দ্ব}{ \textsf{\textit{noun}}\ \textbf{1} Competition for the same objective or for superiority in the same field. {\fontspec{DejaVu Sans}◇} \textit{there always has been intense rivalry between the clubs} \colorBulletS{SYN} competitiveness, competition, contention, vying}{}{}{}%
\par%
\entry{robust}{/rə(ʊ)ˈbʌst/}{শক্তসমর্থ}{ \textsf{\textit{adjective}}\ \textbf{1} Strong and healthy; vigorous. {\fontspec{DejaVu Sans}◇} \textit{the Caplan family are a robust lot} \colorBulletS{SYN} strong, vigorous, sturdy, tough, powerful, powerfully built, solidly built, as strong as a horse, as strong as a ox, muscular, sinewy, rugged, hardy, strapping, brawny, burly, husky \textbf{2} (of wine or food) strong and rich in flavour or smell. {\fontspec{DejaVu Sans}◇} \textit{a robust mixture of fish, onions, capers and tomatoes} \colorBulletS{SYN} strong, full{-}bodied, flavourful, full{-}flavoured, flavoursome, full of flavour, rich}{}{}{ \colorBullet{ORIGIN} Mid 16th century from Latin robustus ‘firm and hard’, from robus, earlier form of robur ‘oak, strength’.}%
\par%
\entry{robustness}{/rə(ʊ)ˈbʌstnəs/}{বলিষ্ঠতা}{ \textsf{\textit{noun}}\ \textbf{1} The quality or condition of being strong and in good condition. {\fontspec{DejaVu Sans}◇} \textit{the overall robustness of national and international financial systems}}{}{}{}%
\par%
\entry{round{-}the{-}clock}{/ˈˌround (T͟H)ə ˈkläk/}{}{ \textsf{\textit{adjective}}\ \textbf{1} Lasting all day and all night. {\fontspec{DejaVu Sans}◇} \textit{round{-}the{-}clock surveillance}}{}{}{}%
\par%
\entry{row}{/rəʊ/}{সারি}{ \textsf{\textit{noun}}\ \textbf{1} A number of people or things in a more or less straight line. {\fontspec{DejaVu Sans}◇} \textit{her villa stood in a row of similar ones} \colorBulletS{SYN} line, column, file, cordon}{}{A woman rows a boat carrying rice straw, to be used as cooking fuel.}{ \colorBullet{ORIGIN} Old English rāw, of Germanic origin; related to Dutch rij and German Reihe.}%
\par%
\entry{row}{/rəʊ/}{সারি}{\small{\textsf{\textit{noun, verb}}} \\{\fontspec{DejaVu Sans}▪ }\textsf{\textit{noun}}\\ \textbf{1} A spell of rowing. {\fontspec{DejaVu Sans}◇} \textit{} \\{\fontspec{DejaVu Sans}▪ }\textsf{\textit{verb}}\\ \textbf{1} Propel (a boat) with oars. {\fontspec{DejaVu Sans}◇} \textit{out in the bay a small figure was rowing a rubber dinghy}}{}{A woman rows a boat carrying rice straw, to be used as cooking fuel.}{ \colorBullet{ORIGIN} Old English rōwan, of Germanic origin; related to rudder; from an Indo{-}European root shared by Latin remus ‘oar’, Greek eretmon ‘oar’.}%
\par%
\entry{row}{/raʊ/}{সারি}{\small{\textsf{\textit{noun, verb}}} \\{\fontspec{DejaVu Sans}▪ }\textsf{\textit{noun}}\\ \textbf{1} A noisy acrimonious quarrel. {\fontspec{DejaVu Sans}◇} \textit{they had a row and she stormed out of the house} \colorBulletS{SYN} argument, quarrel, squabble, fight, contretemps, disagreement, difference of opinion, dissension, falling{-}out, dispute, disputation, contention, clash, altercation, shouting match, exchange, war of words \textbf{2} A loud noise or uproar. {\fontspec{DejaVu Sans}◇} \textit{if he's at home he must have heard that row} \colorBulletS{SYN} din, noise, racket, clamour, uproar, tumult, hubbub, commotion, disturbance, brouhaha, ruckus, rumpus, pandemonium, babel \\{\fontspec{DejaVu Sans}▪ }\textsf{\textit{verb}}\\ \textbf{1} Have a quarrel. {\fontspec{DejaVu Sans}◇} \textit{they rowed about who would receive the money from the sale} \colorBulletS{SYN} argue, quarrel, squabble, bicker, have a fight, have a row, fight, fall out, disagree, fail to agree, differ, be at odds, have a misunderstanding, be at variance, have words, dispute, spar, wrangle, bandy words, cross swords, lock horns, be at each other's throats, be at loggerheads}{}{A woman rows a boat carrying rice straw, to be used as cooking fuel.}{ \colorBullet{ORIGIN} Mid 18th century of unknown origin.}%
\par%
\entry{rumour}{/ˈruːmə/}{গুজব}{\small{\textsf{\textit{noun, verb}}} \\{\fontspec{DejaVu Sans}▪ }\textsf{\textit{noun}}\\ \textbf{1} A currently circulating story or report of uncertain or doubtful truth. {\fontspec{DejaVu Sans}◇} \textit{they were investigating rumours of a massacre} \colorBulletS{SYN} gossip, hearsay, talk, tittle{-}tattle \\{\fontspec{DejaVu Sans}▪ }\textsf{\textit{verb}}\\ \textbf{1} Be circulated as an unverified account. {\fontspec{DejaVu Sans}◇} \textit{it's rumoured that he lives on a houseboat} \colorBulletS{SYN} said to be, reported to be}{}{}{ \colorBullet{ORIGIN} Late Middle English from Old French rumur, from Latin rumor ‘noise’.}%
\par%
\entry{ruse}{/ruːz/}{ছল}{ \textsf{\textit{noun}}\ \textbf{1} An action intended to deceive someone; a trick. {\fontspec{DejaVu Sans}◇} \textit{Emma tried to think of a ruse to get Paul out of the house} \colorBulletS{SYN} ploy, stratagem, tactic, move, device, scheme, trick, gambit, cunning plan, manoeuvre, contrivance, expedient, dodge, subterfuge, machination, game, wile, smokescreen, red herring, blind}{}{}{ \colorBullet{ORIGIN} Late Middle English (as a hunting term): from Old French, from ruser ‘use trickery’, earlier ‘drive back’, perhaps based on Latin rursus ‘backwards’.}%
\par%
\entry{Ruse}{/ˈruːseɪ/}{ছল}{ \textsf{\textit{proper noun}}\ \textbf{1} An industrial city and the principal port of Bulgaria, on the River Danube; population 156,959 (2008). Turkish during the Middle Ages, it was captured by Russia in 1877 and ceded to Bulgaria. {\fontspec{DejaVu Sans}◇} \textit{}}{}{}{}%
\par%
\entry{rush}{/rʌʃ/}{তাড়াহুড়া; ভিড়}{\small{\textsf{\textit{noun, verb}}} \\{\fontspec{DejaVu Sans}▪ }\textsf{\textit{noun}}\\ \textbf{1} A sudden quick movement towards something, typically by a number of people. {\fontspec{DejaVu Sans}◇} \textit{there was a rush for the door} \colorBulletS{SYN} dash, run, sprint, dart, bolt, charge, scramble, bound, break \textbf{2} An act of advancing forward, especially towards the quarterback. {\fontspec{DejaVu Sans}◇} \textit{} \textbf{3} The first prints made of a film after a period of shooting. {\fontspec{DejaVu Sans}◇} \textit{after the shoot the agency team will see the rushes} \\{\fontspec{DejaVu Sans}▪ }\textsf{\textit{verb}}\\ \textbf{1} Move with urgent haste. {\fontspec{DejaVu Sans}◇} \textit{Oliver rushed after her} \colorBulletS{SYN} in a hurry, running about, run off one's feet, rushing about, dashing about, pushed for time, pressed for time, time{-}poor \textbf{2} Dash towards (someone or something) in an attempt to attack or capture. {\fontspec{DejaVu Sans}◇} \textit{to rush the bank and fire willy{-}nilly could be disastrous for everyone} \colorBulletS{SYN} attack, charge, run at, fly at, assail \textbf{3} Entertain (a new student) in order to assess suitability for membership of a college fraternity or sorority. {\fontspec{DejaVu Sans}◇} \textit{} \textbf{4} Make (a customer) pay a particular amount, especially an excessive one. {\fontspec{DejaVu Sans}◇} \textit{how much did they rush you for this heap?}}{}{The rush of dengue patients at the hospital...}{ \colorBullet{ORIGIN} Late Middle English from an Anglo{-}Norman French variant of Old French ruser ‘drive back’, an early sense of the word in English (see ruse).}%
\par%
\entry{rush}{/rʌʃ/}{তাড়াহুড়া; ভিড়}{ \textsf{\textit{noun}}\ \textbf{1} An erect, tufted marsh or waterside plant resembling a sedge or grass, with inconspicuous greenish or brownish flowers. Widely distributed in temperate areas, some kinds are used for matting, chair seats, and baskets. {\fontspec{DejaVu Sans}◇} \textit{} \textbf{2} A thing of no value (used for emphasis) {\fontspec{DejaVu Sans}◇} \textit{not one of them is worth a rush}}{}{The rush of dengue patients at the hospital...}{ \colorBullet{ORIGIN} Old English risc, rysc, of Germanic origin.}%
\par%
\entry{rust}{/rʌst/}{মরিচা}{\small{\textsf{\textit{noun, verb}}} \\{\fontspec{DejaVu Sans}▪ }\textsf{\textit{noun}}\\ \textbf{1} A reddish{-} or yellowish{-}brown flaking coating of iron oxide that is formed on iron or steel by oxidation, especially in the presence of moisture. {\fontspec{DejaVu Sans}◇} \textit{paint protects your car from rust} \colorBulletS{SYN} discoloration, oxidation, rust, tarnishing, blackening, film, patina \textbf{2} A fungal disease of plants which results in reddish or brownish patches. {\fontspec{DejaVu Sans}◇} \textit{} \textbf{3} A reddish{-}brown colour. {\fontspec{DejaVu Sans}◇} \textit{her rust{-}coloured coat} \colorBulletS{SYN} bronze{-}coloured, copper{-}coloured, copper, reddish brown, chestnut, metallic brown, rust{-}coloured, rust, henna, tan \\{\fontspec{DejaVu Sans}▪ }\textsf{\textit{verb}}\\ \textbf{1} Be affected with rust. {\fontspec{DejaVu Sans}◇} \textit{the blades had rusted away} \colorBulletS{SYN} corrode, oxidize, become rusty, tarnish}{}{}{ \colorBullet{ORIGIN} Old English rūst, of Germanic origin; related to Dutch roest, German Rost, also to red.}%
\par%
\end{multicols}%
\pagebreak%
\section*{S}%
\begin{multicols}{2}%
\entry{sabotage}{/ˈsabətɑːʒ/}{অন্তর্ঘাত}{\small{\textsf{\textit{noun, verb}}} \\{\fontspec{DejaVu Sans}▪ }\textsf{\textit{noun}}\\ \textbf{1} The action of sabotaging something. {\fontspec{DejaVu Sans}◇} \textit{a coordinated campaign of sabotage} \colorBulletS{SYN} wrecking, deliberate damage, vandalism, destruction, obstruction, disruption, crippling, impairment, incapacitation \\{\fontspec{DejaVu Sans}▪ }\textsf{\textit{verb}}\\ \textbf{1} Deliberately destroy, damage, or obstruct (something), especially for political or military advantage. {\fontspec{DejaVu Sans}◇} \textit{power lines from South Africa were sabotaged by rebel forces} \colorBulletS{SYN} wreck, deliberately damage, vandalize, destroy, obstruct, disrupt, cripple, impair, incapacitate}{}{}{ \colorBullet{ORIGIN} Early 20th century from French, from saboter ‘kick with sabots, wilfully destroy’ (see sabot).}%
\par%
\entry{sack}{/sak/}{বস্তা}{\small{\textsf{\textit{noun, verb}}} \\{\fontspec{DejaVu Sans}▪ }\textsf{\textit{noun}}\\ \textbf{1} A large bag made of a strong material such as hessian, thick paper, or plastic, used for storing and carrying goods. {\fontspec{DejaVu Sans}◇} \textit{} \colorBulletS{SYN} bag, pack, pouch, pocket \textbf{2}  {\fontspec{DejaVu Sans}◇} \textit{} \textbf{3} Dismissal from employment. {\fontspec{DejaVu Sans}◇} \textit{he got the sack for swearing} \colorBulletS{SYN} dismissal, discharge, redundancy, termination of employment, one's marching orders \textbf{4} Bed, especially as regarded as a place for sex. {\fontspec{DejaVu Sans}◇} \textit{} \colorBulletS{SYN} bed \textbf{5} A base. {\fontspec{DejaVu Sans}◇} \textit{} \textbf{6} An act of tackling of a quarterback behind the line of scrimmage. {\fontspec{DejaVu Sans}◇} \textit{} \\{\fontspec{DejaVu Sans}▪ }\textsf{\textit{verb}}\\ \textbf{1} Dismiss from employment. {\fontspec{DejaVu Sans}◇} \textit{any official found to be involved would be sacked on the spot} \colorBulletS{SYN} dismiss, give someone their notice, throw out, get rid of, lay off, make redundant, let go, discharge, cashier \textbf{2} Tackle (a quarterback) behind the line of scrimmage. {\fontspec{DejaVu Sans}◇} \textit{Oregon intercepted five of his passes and sacked him five times} \textbf{3} Put into a sack or sacks. {\fontspec{DejaVu Sans}◇} \textit{a small part of his wheat had been sacked}}{}{}{ \colorBullet{ORIGIN} Old English sacc, from Latin saccus ‘sack, sackcloth’, from Greek sakkos, of Semitic origin. Sense 1 of the verb dates from the mid 19th century.}%
\par%
\entry{sack}{/sak/}{বস্তা}{\small{\textsf{\textit{noun, verb}}} \\{\fontspec{DejaVu Sans}▪ }\textsf{\textit{noun}}\\ \textbf{1} The pillaging of a town or city. {\fontspec{DejaVu Sans}◇} \textit{the sack of Rome} \colorBulletS{SYN} laying waste, ransacking, plunder, plundering, sacking, looting, ravaging, pillage, pillaging, devastation, depredation, stripping, robbery, robbing, raiding \\{\fontspec{DejaVu Sans}▪ }\textsf{\textit{verb}}\\ \textbf{1} (chiefly in historical contexts) plunder and destroy (a captured town or building) {\fontspec{DejaVu Sans}◇} \textit{the fort was rebuilt in AD 158 and was sacked again in AD 197} \colorBulletS{SYN} ravage, lay waste, devastate, ransack, strip, fleece, plunder, pillage, loot, rob, raid}{}{}{ \colorBullet{ORIGIN} Mid 16th century from French sac, in the phrase mettre à sac ‘put to sack’, on the model of Italian fare il sacco, mettere a sacco, which perhaps originally referred to filling a sack with plunder.}%
\par%
\entry{sack}{/sak/}{বস্তা}{ \textsf{\textit{noun}}\ \textbf{1} A dry white wine formerly imported into Britain from Spain and the Canaries. {\fontspec{DejaVu Sans}◇} \textit{}}{}{}{ \colorBullet{ORIGIN} Early 16th century from the phrase wyne seck, from French vin sec ‘dry wine’.}%
\par%
\entry{saliva}{/səˈlʌɪvə/}{মুখের লালা}{ \textsf{\textit{noun}}\ \textbf{1} Watery liquid secreted into the mouth by glands, providing lubrication for chewing and swallowing, and aiding digestion. {\fontspec{DejaVu Sans}◇} \textit{} \colorBulletS{SYN} spit, spittle, dribble, drool, slaver, slobber, sputum}{}{}{ \colorBullet{ORIGIN} Late Middle English from Latin.}%
\par%
\entry{salvage}{/ˈsalvɪdʒ/}{জাহাজ ও জাহাজের মাল রক্ষা করার কাজ}{\small{\textsf{\textit{noun, verb}}} \\{\fontspec{DejaVu Sans}▪ }\textsf{\textit{noun}}\\ \textbf{1} The rescue of a wrecked or disabled ship or its cargo from loss at sea. {\fontspec{DejaVu Sans}◇} \textit{a salvage operation was under way} \colorBulletS{SYN} rescue, saving, recovery, raising, reclamation, restoration, salvation \\{\fontspec{DejaVu Sans}▪ }\textsf{\textit{verb}}\\ \textbf{1} Rescue (a wrecked or disabled ship or its cargo) from loss at sea. {\fontspec{DejaVu Sans}◇} \textit{an emerald and gold cross was salvaged from the wreck} \colorBulletS{SYN} rescue, save, recover, retrieve, raise, reclaim, get back, restore, reinstate}{}{}{ \colorBullet{ORIGIN} Mid 17th century (as a noun denoting payment for saving a ship or its cargo): from French, from medieval Latin salvagium, from Latin salvare ‘to save’. The verb dates from the late 19th century.}%
\par%
\entry{salvo}{/ˈsalvəʊ/}{ফাঁকি; সামরিক অভিবাদনের অঙ্গস্বরূপ একটানা তোপধ্বনি}{ \textsf{\textit{noun}}\ \textbf{1} A simultaneous discharge of artillery or other guns in a battle. {\fontspec{DejaVu Sans}◇} \textit{a deafening salvo of shots rang out} \colorBulletS{SYN} barrage, volley, shower, deluge, torrent, burst, stream, storm, flood, spate, rain, tide, avalanche, blaze, onslaught}{}{}{ \colorBullet{ORIGIN} Late 16th century (earlier as salve): from French salve, Italian salva ‘salutation’.}%
\par%
\entry{Salvo}{/ˈsalvəʊ/}{ফাঁকি; সামরিক অভিবাদনের অঙ্গস্বরূপ একটানা তোপধ্বনি}{ \textsf{\textit{noun}}\ \textbf{1} A member of the Salvation Army. {\fontspec{DejaVu Sans}◇} \textit{}}{}{}{ \colorBullet{ORIGIN} Late 19th century abbreviation of salvation.}%
\par%
\entry{sanction}{/ˈsaŋ(k)ʃ(ə)n/}{অনুমোদন}{\small{\textsf{\textit{noun, verb}}} \\{\fontspec{DejaVu Sans}▪ }\textsf{\textit{noun}}\\ \textbf{1} A threatened penalty for disobeying a law or rule. {\fontspec{DejaVu Sans}◇} \textit{a range of sanctions aimed at deterring insider abuse} \colorBulletS{SYN} penalty, punishment, deterrent \textbf{2} Official permission or approval for an action. {\fontspec{DejaVu Sans}◇} \textit{he appealed to the bishop for his sanction} \colorBulletS{SYN} authorization, consent, leave, permission, authority, warrant, licence, dispensation, assent, acquiescence, agreement, approval, seal of approval, stamp of approval, approbation, recognition, endorsement, accreditation, confirmation, ratification, validation, blessing, imprimatur, clearance, acceptance, allowance \\{\fontspec{DejaVu Sans}▪ }\textsf{\textit{verb}}\\ \textbf{1} Give official permission or approval for (an action) {\fontspec{DejaVu Sans}◇} \textit{the scheme was sanctioned by the court} \colorBulletS{SYN} authorize, consent to, permit, allow, give leave for, give permission for, warrant, accredit, license, give assent to, endorse, agree to, approve, accept, give one's blessing to, back, support \textbf{2} Impose a sanction or penalty on. {\fontspec{DejaVu Sans}◇} \textit{foreigners in France illegally should be sent home, their employers sanctioned and border controls tightened up} \colorBulletS{SYN} punish, discipline someone for}{}{}{ \colorBullet{ORIGIN} Late Middle English (as a noun denoting an ecclesiastical decree): from French, from Latin sanctio(n{-}), from sancire ‘ratify’. The verb dates from the late 18th century.}%
\par%
\entry{savvy}{/ˈsavi/}{কাণ্ডজ্ঞান}{\small{\textsf{\textit{adjective, noun, verb}}} \\{\fontspec{DejaVu Sans}▪ }\textsf{\textit{adjective}}\\ \textbf{1} Shrewd and knowledgeable; having common sense and good judgement. {\fontspec{DejaVu Sans}◇} \textit{Bob is a savvy veteran who knows all the tricks} \colorBulletS{SYN} shrewd, astute, sharp{-}witted, sharp, acute, intelligent, clever, canny, media{-}savvy, perceptive, perspicacious, sagacious, sage \\{\fontspec{DejaVu Sans}▪ }\textsf{\textit{noun}}\\ \textbf{1} Shrewdness and practical knowledge; the ability to make good judgements. {\fontspec{DejaVu Sans}◇} \textit{the corporate finance bankers lacked the necessary political savvy} \colorBulletS{SYN} shrewdness, astuteness, sharp{-}wittedness, sharpness, acuteness, acumen, acuity, intelligence, wit, canniness, common sense, discernment, insight, understanding, penetration, perception, perceptiveness, perspicacity, perspicaciousness, knowledge, sagacity, sageness \\{\fontspec{DejaVu Sans}▪ }\textsf{\textit{verb}}\\ \textbf{1} Know or understand. {\fontspec{DejaVu Sans}◇} \textit{Charley would savvy what to do} \colorBulletS{SYN} realize, understand, comprehend, grasp, see, know, apprehend}{}{}{ \colorBullet{ORIGIN} Late 18th century originally black and pidgin English imitating Spanish sabe usted ‘you know’.}%
\par%
\entry{scale}{/skeɪl/}{স্কেল}{\small{\textsf{\textit{noun, verb}}} \\{\fontspec{DejaVu Sans}▪ }\textsf{\textit{noun}}\\ \textbf{1} Each of the small, thin horny or bony plates protecting the skin of fish and reptiles, typically overlapping one another. {\fontspec{DejaVu Sans}◇} \textit{} \colorBulletS{SYN} plate \textbf{2} A thick, dry flake of skin. {\fontspec{DejaVu Sans}◇} \textit{} \colorBulletS{SYN} flake \textbf{3} A flaky covering or deposit. {\fontspec{DejaVu Sans}◇} \textit{} \\{\fontspec{DejaVu Sans}▪ }\textsf{\textit{verb}}\\ \textbf{1} Remove scale or scales from. {\fontspec{DejaVu Sans}◇} \textit{he scales the fish and removes the innards} \textbf{2} (especially of the skin) form scales. {\fontspec{DejaVu Sans}◇} \textit{the skin may scale and peel away with itching, stinging, or burning sensations in the infected area}}{}{}{ \colorBullet{ORIGIN} Middle English shortening of Old French escale, from the Germanic base of scale.}%
\par%
\entry{scale}{/skeɪl/}{স্কেল}{\small{\textsf{\textit{noun, verb}}} \\{\fontspec{DejaVu Sans}▪ }\textsf{\textit{noun}}\\ \textbf{1} An instrument for weighing, originally a simple balance (a pair of scales) but now usually a device with an electronic or other internal weighing mechanism. {\fontspec{DejaVu Sans}◇} \textit{bathroom scales} \colorBulletS{SYN} weighing machine, balance, pair of scales \textbf{2} A large drinking container for beer or other alcoholic drink. {\fontspec{DejaVu Sans}◇} \textit{} \\{\fontspec{DejaVu Sans}▪ }\textsf{\textit{verb}}\\ \textbf{1} Weigh a specified weight. {\fontspec{DejaVu Sans}◇} \textit{some men scaled less than ninety pounds}}{}{}{ \colorBullet{ORIGIN} Middle English (in the sense ‘drinking cup’, surviving in South African English): from Old Norse skál ‘bowl’, of Germanic origin; related to Dutch schaal, German Schale ‘bowl’, also to English dialect shale ‘dish’.}%
\par%
\entry{scale}{/skeɪl/}{স্কেল}{\small{\textsf{\textit{noun, verb}}} \\{\fontspec{DejaVu Sans}▪ }\textsf{\textit{noun}}\\ \textbf{1} A graduated range of values forming a standard system for measuring or grading something. {\fontspec{DejaVu Sans}◇} \textit{a new salary scale is planned for all universities} \colorBulletS{SYN} calibrated system, calibration, graduated system, system of measurement, measuring system, register \textbf{2} The relative size or extent of something. {\fontspec{DejaVu Sans}◇} \textit{no one foresaw the scale of the disaster} \colorBulletS{SYN} extent, size, scope, magnitude, dimensions, range, breadth, compass, degree, reach, spread, sweep \textbf{3} An arrangement of the notes in any system of music in ascending or descending order of pitch. {\fontspec{DejaVu Sans}◇} \textit{the scale of C major} \textbf{4}  {\fontspec{DejaVu Sans}◇} \textit{the conversion of the number to the binary scale} \textbf{5} The range of exposures over which a photographic material will give an acceptable variation in density. {\fontspec{DejaVu Sans}◇} \textit{} \\{\fontspec{DejaVu Sans}▪ }\textsf{\textit{verb}}\\ \textbf{1} Climb up or over (something high and steep) {\fontspec{DejaVu Sans}◇} \textit{thieves scaled a high fence} \colorBulletS{SYN} climb, ascend, go up, go over, clamber up, shin, shin up, scramble up, mount \textbf{2} Represent in proportional dimensions; reduce or increase in size according to a common scale. {\fontspec{DejaVu Sans}◇} \textit{} \textbf{3} Estimate the amount of timber that will be produced from (a log or uncut tree) {\fontspec{DejaVu Sans}◇} \textit{the operators were accustomed to having their logs scaled for inventory control}}{}{}{ \colorBullet{ORIGIN} Late Middle English from Latin scala ‘ladder’ (the verb via Old French escaler or medieval Latin scalare ‘climb’), from the base of Latin scandere ‘to climb’.}%
\par%
\entry{scaling}{/ˈskeɪlɪŋ/}{আরোহী}{\small{\textsf{\textit{adjective, noun}}} \\{\fontspec{DejaVu Sans}▪ }\textsf{\textit{adjective}}\\ \textbf{1} (especially of skin or paint) tending to crack and come away in thin pieces. {\fontspec{DejaVu Sans}◇} \textit{do not paint over loose or scaling paint} \\{\fontspec{DejaVu Sans}▪ }\textsf{\textit{noun}}\\ \textbf{1} The removal of the scales from something. {\fontspec{DejaVu Sans}◇} \textit{fresh fish processing is highly labour{-}intensive, mainly in the scaling} \textbf{2} The formation of scales, especially on the skin. {\fontspec{DejaVu Sans}◇} \textit{moisturizers can ease drying and scaling}}{}{}{}%
\par%
\entry{scalp}{/skalp/}{মাথার খুলি}{\small{\textsf{\textit{noun, verb}}} \\{\fontspec{DejaVu Sans}▪ }\textsf{\textit{noun}}\\ \textbf{1} The skin covering the head, excluding the face. {\fontspec{DejaVu Sans}◇} \textit{hair tonics will improve the condition of your hair and scalp} \textbf{2} A bare rock projecting above surrounding water or vegetation. {\fontspec{DejaVu Sans}◇} \textit{} \\{\fontspec{DejaVu Sans}▪ }\textsf{\textit{verb}}\\ \textbf{1} Take the scalp of (an enemy) {\fontspec{DejaVu Sans}◇} \textit{none of the soldiers were scalped} \textbf{2} Resell (shares or tickets) at a large or quick profit. {\fontspec{DejaVu Sans}◇} \textit{tickets were scalped for forty times their face value}}{}{}{ \colorBullet{ORIGIN} Middle English (denoting the skull or cranium): probably of Scandinavian origin.}%
\par%
\entry{scanty}{/ˈskanti/}{অত্যল্প}{\small{\textsf{\textit{adjective, plural noun}}} \\{\fontspec{DejaVu Sans}▪ }\textsf{\textit{adjective}}\\ \textbf{1} Small or insufficient in quantity or amount. {\fontspec{DejaVu Sans}◇} \textit{they paid whatever they could out of their scanty wages to their families} \colorBulletS{SYN} meagre, scant, minimal, limited, modest, restricted, sparse \\{\fontspec{DejaVu Sans}▪ }\textsf{\textit{plural noun}}\\ \textbf{1} Women's skimpy knickers or pants. {\fontspec{DejaVu Sans}◇} \textit{}}{}{}{ \colorBullet{ORIGIN} Late 16th century from scant+ {-}y.}%
\par%
\entry{scare}{/skɛː/}{ভীতি}{\small{\textsf{\textit{noun, verb}}} \\{\fontspec{DejaVu Sans}▪ }\textsf{\textit{noun}}\\ \textbf{1} A sudden attack of fright. {\fontspec{DejaVu Sans}◇} \textit{gosh, that gave me a scare!} \colorBulletS{SYN} fright, shock, start, turn, jump \\{\fontspec{DejaVu Sans}▪ }\textsf{\textit{verb}}\\ \textbf{1} Cause great fear or nervousness in; frighten. {\fontspec{DejaVu Sans}◇} \textit{the rapid questions were designed to scare her into blurting out the truth} \colorBulletS{SYN} frighten, make afraid, make fearful, make nervous, panic, throw into a panic}{}{}{ \colorBullet{ORIGIN} Middle English from Old Norse skirra ‘frighten’, from skjarr ‘timid’.}%
\par%
\entry{scary}{/ˈskɛːri/}{ভীতিকর}{ \textsf{\textit{adjective}}\ \textbf{1} Frightening; causing fear. {\fontspec{DejaVu Sans}◇} \textit{a scary movie} \colorBulletS{SYN} frightening, scaring, hair{-}raising, terrifying, petrifying, spine{-}chilling, blood{-}curdling, chilling, horrifying, alarming, appalling, daunting, formidable, fearsome, nerve{-}racking, unnerving}{}{}{}%
\par%
\entry{scoliosis}{/ˌskɒlɪˈəʊsɪs/}{স্কলায়োসিস}{ \textsf{\textit{noun}}\ \textbf{1} Abnormal lateral curvature of the spine. {\fontspec{DejaVu Sans}◇} \textit{}}{}{}{ \colorBullet{ORIGIN} Early 18th century modern Latin, from Greek, from skolios ‘bent’.}%
\par%
\entry{scour}{/ˈskaʊə/}{পরিমার্জন}{\small{\textsf{\textit{noun, verb}}} \\{\fontspec{DejaVu Sans}▪ }\textsf{\textit{noun}}\\ \textbf{1} The action of scouring or the state of being scoured, especially by swift{-}flowing water. {\fontspec{DejaVu Sans}◇} \textit{the scour of the tide may cause lateral erosion} \textbf{2}  {\fontspec{DejaVu Sans}◇} \textit{} \\{\fontspec{DejaVu Sans}▪ }\textsf{\textit{verb}}\\ \textbf{1} Clean or brighten the surface of (something) by rubbing it hard, typically with an abrasive or detergent. {\fontspec{DejaVu Sans}◇} \textit{she scoured the cooker} \colorBulletS{SYN} scrub, rub, clean, wash, cleanse, wipe \textbf{2} (of livestock) suffer from diarrhoea. {\fontspec{DejaVu Sans}◇} \textit{he went out to deal with piglets who were scouring}}{}{The number of piles has been optimised considering the scour depth at the bridge location and also to make the bridge earthquake resistant.}{ \colorBullet{ORIGIN} Middle English from Middle Dutch, Middle Low German schūren, from Old French escurer, from late Latin excurare ‘clean (off)’, from ex{-} ‘away’ + curare ‘to clean’.}%
\par%
\entry{scour}{/ˈskaʊə/}{পরিমার্জন}{ \textsf{\textit{verb}}\ \textbf{1} Subject (a place, text, etc.) to a thorough search in order to locate something. {\fontspec{DejaVu Sans}◇} \textit{David scoured each newspaper for an article on the murder} \colorBulletS{SYN} search, comb, hunt through, rummage through, sift through, go through with a fine{-}tooth comb, root through, rake through, leave no stone unturned, mine, look all over, look high and low in}{}{The number of piles has been optimised considering the scour depth at the bridge location and also to make the bridge earthquake resistant.}{ \colorBullet{ORIGIN} Late Middle English related to obsolete scour ‘moving hastily’, of unknown origin.}%
\par%
\entry{scrap}{/skrap/}{ছাঁট; বর্জিতাংশ}{\small{\textsf{\textit{noun, verb}}} \\{\fontspec{DejaVu Sans}▪ }\textsf{\textit{noun}}\\ \textbf{1} A small piece or amount of something, especially one that is left over after the greater part has been used. {\fontspec{DejaVu Sans}◇} \textit{I scribbled her address on a scrap of paper} \colorBulletS{SYN} fragment, piece, bit, offcut, oddment, snippet, snip, tatter, wisp, shred, remnant \textbf{2}  {\fontspec{DejaVu Sans}◇} \textit{the steamer was eventually sold for scrap} \\{\fontspec{DejaVu Sans}▪ }\textsf{\textit{verb}}\\ \textbf{1} Discard or remove from service (a redundant, old, or inoperative vehicle, vessel, or machine), especially so as to convert it to scrap metal. {\fontspec{DejaVu Sans}◇} \textit{a bold decision was taken to scrap existing plant}}{}{}{ \colorBullet{ORIGIN} Late Middle English (as a plural noun denoting fragments of uneaten food): from Old Norse skrap ‘scraps’; related to skrapa ‘to scrape’. The verb dates from the late 19th century.}%
\par%
\entry{scrap}{/skrap/}{ছাঁট; বর্জিতাংশ}{\small{\textsf{\textit{noun, verb}}} \\{\fontspec{DejaVu Sans}▪ }\textsf{\textit{noun}}\\ \textbf{1} A fight or quarrel, especially a minor or spontaneous one. {\fontspec{DejaVu Sans}◇} \textit{they were involved in a goalmouth scrap and a player was sent off} \colorBulletS{SYN} quarrel, argument, row, fight, disagreement, difference of opinion, dissension, falling{-}out, dispute, disputation, contention, squabble, contretemps, clash, altercation, exchange, brawl, tussle, conflict, affray, war of words, shouting match, fracas, wrangle, tangle, misunderstanding, passage at arms, passage of arms, battle royal \\{\fontspec{DejaVu Sans}▪ }\textsf{\textit{verb}}\\ \textbf{1} Engage in a minor fight or quarrel. {\fontspec{DejaVu Sans}◇} \textit{the older boys started scrapping with me} \colorBulletS{SYN} quarrel, argue, have a fight, have a row, row, fight, disagree, fail to agree, differ, be at odds, have a misunderstanding, be at variance, fall out, dispute, squabble, brawl, bicker, chop logic, spar, wrangle, bandy words, cross swords, lock horns, be at each other's throats, be at loggerheads}{}{}{ \colorBullet{ORIGIN} Late 17th century (as a noun in the sense ‘sinister plot, scheme’): perhaps from the noun scrape.}%
\par%
\entry{scratchy}{/ˈskratʃi/}{খর্খরে}{ \textsf{\textit{adjective}}\ \textbf{1} (especially of a fabric or garment) having a rough, uncomfortable texture and tending to cause itching or discomfort. {\fontspec{DejaVu Sans}◇} \textit{a cardigan in a scratchy wool}}{}{}{}%
\par%
\entry{scream}{/skriːm/}{চিৎকার}{\small{\textsf{\textit{noun, verb}}} \\{\fontspec{DejaVu Sans}▪ }\textsf{\textit{noun}}\\ \textbf{1} A long, loud, piercing cry expressing extreme emotion or pain. {\fontspec{DejaVu Sans}◇} \textit{they were awakened by screams for help} \colorBulletS{SYN} shriek, screech, yell, howl, shout, bellow, bawl, cry, yawp, yelp, squeal, wail, squawk, squall, caterwaul, whoop \textbf{2} A loud, piercing sound. {\fontspec{DejaVu Sans}◇} \textit{the scream of a falling bomb} \textbf{3} An irresistibly funny person, thing, or situation. {\fontspec{DejaVu Sans}◇} \textit{the movie's a scream} \colorBulletS{SYN} laugh \\{\fontspec{DejaVu Sans}▪ }\textsf{\textit{verb}}\\ \textbf{1} Give a long, loud, piercing cry or cries expressing extreme emotion or pain. {\fontspec{DejaVu Sans}◇} \textit{they could hear him screaming in pain} \textbf{2} Make a loud, high{-}pitched sound. {\fontspec{DejaVu Sans}◇} \textit{sirens were screaming from all over the city} \textbf{3} Turn informer. {\fontspec{DejaVu Sans}◇} \textit{he never got paid and my information is he's ready to scream}}{}{}{ \colorBullet{ORIGIN} Middle English origin uncertain; perhaps from Middle Dutch.}%
\par%
\entry{screen}{/skriːn/}{পর্দা}{\small{\textsf{\textit{noun, verb}}} \\{\fontspec{DejaVu Sans}▪ }\textsf{\textit{noun}}\\ \textbf{1} A fixed or movable upright partition used to divide a room, give shelter from draughts, heat, or light, or to provide concealment or privacy. {\fontspec{DejaVu Sans}◇} \textit{the Special Branch man remained hidden behind the screen for prosecution witnesses} \colorBulletS{SYN} partition, divider, room divider, dividing wall, separator, curtain, arras, blind, awning, shade, shutter, canopy, windbreak \textbf{2} A flat panel or area on an electronic device such as a television, computer, or smartphone, on which images and data are displayed. {\fontspec{DejaVu Sans}◇} \textit{a television screen} \colorBulletS{SYN} display, monitor, visual display unit, VDU, cathode{-}ray tube, CRT \textbf{3} A transparent finely ruled plate or film used in half{-}tone reproduction. {\fontspec{DejaVu Sans}◇} \textit{The halftone screen used to create the greys for the text was terrible, and you could see dots with the naked eye.} \textbf{4} A system of checking a person or thing for the presence or absence of something, typically a disease. {\fontspec{DejaVu Sans}◇} \textit{services offered by the centre include a health screen for people who have just joined the company} \textbf{5} A detachment of troops or ships detailed to cover the movements of the main body. {\fontspec{DejaVu Sans}◇} \textit{HMS Prince Leopold and HMS Prince Charles sailed for Shetland with a screen of four destroyers} \textbf{6} A large sieve or riddle, especially one for sorting substances such as grain or coal into different sizes. {\fontspec{DejaVu Sans}◇} \textit{the material retained on each sieve screen is weighed in turn} \colorBulletS{SYN} sieve, riddle, sifter, strainer, colander, filter, winnow \\{\fontspec{DejaVu Sans}▪ }\textsf{\textit{verb}}\\ \textbf{1} Conceal, protect, or shelter (someone or something) with a screen or something forming a screen. {\fontspec{DejaVu Sans}◇} \textit{her hair swung across to screen her face} \colorBulletS{SYN} conceal, hide, mask, shield, shelter, shade, protect, guard, safeguard, veil, cloak, camouflage, disguise \textbf{2} Show (a film or video) or broadcast (a television programme) {\fontspec{DejaVu Sans}◇} \textit{the show is to be screened by the BBC later this year} \colorBulletS{SYN} show, present, air, broadcast, transmit, televise, put out, put on the air, telecast, relay \textbf{3} Test (a person or substance) for the presence or absence of a disease. {\fontspec{DejaVu Sans}◇} \textit{outpatients were screened for cervical cancer} \textbf{4} Pass (a substance such as grain or coal) through a large sieve or screen, especially so as to sort it into different sizes. {\fontspec{DejaVu Sans}◇} \textit{granulated asphalt—manufactured to 40 mm down or screened to 28 mm \& 14 mm down} \colorBulletS{SYN} sieve, riddle, sift, strain, filter, sort, winnow \textbf{5} Project (a photograph or other image) through a transparent ruled plate so as to be able to reproduce it as a half{-}tone. {\fontspec{DejaVu Sans}◇} \textit{}}{}{}{ \colorBullet{ORIGIN} Middle English shortening of Old Northern French escren, of Germanic origin.}%
\par%
\entry{scrotum}{/ˈskrəʊtəm/}{অণ্ডকোষ}{ \textsf{\textit{noun}}\ \textbf{1} A pouch of skin containing the testicles. {\fontspec{DejaVu Sans}◇} \textit{}}{}{}{ \colorBullet{ORIGIN} Late 16th century from Latin.}%
\par%
\entry{scrumptious}{/ˈskrʌm(p)ʃəs/}{দারুণ}{ \textsf{\textit{adjective}}\ \textbf{1} (of food) extremely tasty; delicious. {\fontspec{DejaVu Sans}◇} \textit{a scrumptious chocolate tart} \colorBulletS{SYN} delicious, gorgeous, tasty, good, mouth{-}watering, appetizing, inviting, palatable, delectable, delightful, succulent, rich, sweet, choice, dainty, savoury, flavoursome, flavourful, piquant, luscious, toothsome}{}{}{ \colorBullet{ORIGIN} Mid 19th century of unknown origin.}%
\par%
\entry{sedate}{/sɪˈdeɪt/}{}{ \textsf{\textit{adjective}}\ \textbf{1} Calm, dignified, and unhurried. {\fontspec{DejaVu Sans}◇} \textit{in the old days, business was carried on at a rather more sedate pace} \colorBulletS{SYN} calm, tranquil, placid, composed, serene, steady, unruffled, imperturbable, unflappable}{}{}{ \colorBullet{ORIGIN} Late Middle English (originally as a medical term meaning ‘not sore or painful’, also ‘calm, tranquil’): from Latin sedatus, past participle of sedare ‘settle’, from sedere ‘sit’.}%
\par%
\entry{sedate}{/sɪˈdeɪt/}{}{ \textsf{\textit{verb}}\ \textbf{1} Calm (someone) or make them sleep by administering a sedative drug. {\fontspec{DejaVu Sans}◇} \textit{she was heavily sedated} \colorBulletS{SYN} tranquillize, give a sedative to, put under sedation, calm down, quieten, pacify, soothe, relax, dope, drug, administer drugs to, administer narcotics to, administer opiates to, knock out, anaesthetize}{}{}{ \colorBullet{ORIGIN} 1960s back{-}formation from sedation.}%
\par%
\entry{sedition}{/sɪˈdɪʃ(ə)n/}{রাজদ্রোহ}{ \textsf{\textit{noun}}\ \textbf{1} Conduct or speech inciting people to rebel against the authority of a state or monarch. {\fontspec{DejaVu Sans}◇} \textit{} \colorBulletS{SYN} incitement, incitement to rebellion, incitement to riot, agitation, rabble{-}rousing, fomentation, fomentation of discontent, troublemaking, provocation, inflaming}{}{}{ \colorBullet{ORIGIN} Late Middle English (in the sense ‘violent strife’): from Old French, or from Latin seditio(n{-}), from sed{-} ‘apart’ + itio(n{-}) ‘going’ (from the verb ire).}%
\par%
\entry{see}{/siː/}{}{ \textsf{\textit{verb}}\ \textbf{1} Perceive with the eyes; discern visually. {\fontspec{DejaVu Sans}◇} \textit{in the distance she could see the blue sea} \colorBulletS{SYN} discern, perceive, glimpse, catch a glimpse of, get a glimpse of, spot, notice, catch sight of, sight \textbf{2} Discern or deduce after reflection or from information; understand. {\fontspec{DejaVu Sans}◇} \textit{I can't see any other way to treat it} \colorBulletS{SYN} understand, grasp, comprehend, follow, take in, realize, appreciate, recognize, work out, get the drift of, make out, conceive, perceive, fathom, fathom out, become cognizant of \textbf{3} Experience or witness (an event or situation) {\fontspec{DejaVu Sans}◇} \textit{I shall not live to see it} \textbf{4} Meet (someone one knows) socially or by chance. {\fontspec{DejaVu Sans}◇} \textit{I saw Colin last night} \colorBulletS{SYN} meet, meet by chance, encounter, run into, run across, stumble across, stumble on, happen on, chance on, come across \textbf{5} Escort or conduct (someone) to a specified place. {\fontspec{DejaVu Sans}◇} \textit{don't bother seeing me out} \colorBulletS{SYN} escort, accompany, show, walk, conduct, lead, take, usher, guide, shepherd, attend \textbf{6} Ensure. {\fontspec{DejaVu Sans}◇} \textit{Lucy saw to it that everyone got enough to eat} \textbf{7} (in poker or brag) equal the bet of (an opponent) and require them to reveal their cards in order to determine who has won the hand. {\fontspec{DejaVu Sans}◇} \textit{If the discarded cards were also equal in rank then the player who was seen wins the tie.}}{ \colorBullet{OTHER} see off: বিদায় দেত্তয়া; to accompany one to the place where they will be departing and wish them farewell.}{John offered to see me off to the train station, but i was so sad to leave that i preferred to go alone. I'm just going to see our guests off. I'll be back shortly.}{ \colorBullet{ORIGIN} Old English sēon, of Germanic origin; related to Dutch zien and German sehen, perhaps from an Indo{-}European root shared by Latin sequi ‘follow’.}%
\par%
\entry{see}{/siː/}{}{ \textsf{\textit{noun}}\ \textbf{1} The place in which a cathedral church stands, identified as the seat of authority of a bishop or archbishop. {\fontspec{DejaVu Sans}◇} \textit{} \colorBulletS{SYN} diocese, bishopric}{ \colorBullet{OTHER} see off: বিদায় দেত্তয়া; to accompany one to the place where they will be departing and wish them farewell.}{John offered to see me off to the train station, but i was so sad to leave that i preferred to go alone. I'm just going to see our guests off. I'll be back shortly.}{ \colorBullet{ORIGIN} Middle English from Anglo{-}Norman French sed, from Latin sedes ‘seat’, from sedere ‘sit’.}%
\par%
\entry{seek}{/siːk/}{চাইতে}{ \textsf{\textit{verb}}\ \textbf{1} Attempt to find (something) {\fontspec{DejaVu Sans}◇} \textit{they came here to seek shelter from biting winter winds} \colorBulletS{SYN} search for, try to find, look for, look about for, look around for, look round for, cast about for, cast around for, cast round for, be on the lookout for, be after, hunt for, be in quest of, quest, quest after, be in pursuit of}{}{}{ \colorBullet{ORIGIN} Old English sēcan, of Germanic origin; related to Dutch zieken and German suchen, from an Indo{-}European root shared by Latin sagire ‘perceive by scent’.}%
\par%
\entry{seem}{/siːm/}{মনে}{ \textsf{\textit{verb}}\ \textbf{1} Give the impression of being something or having a particular quality. {\fontspec{DejaVu Sans}◇} \textit{Dawn seemed annoyed} \colorBulletS{SYN} appear, appear to be, have the air of being, have the appearance of being, give the impression of being, look, look like, look as though one is, look to be, have the look of, show signs of \textbf{2} Be unable to do something, despite having tried. {\fontspec{DejaVu Sans}◇} \textit{he couldn't seem to remember his lines}}{}{}{ \colorBullet{ORIGIN} Middle English (also in the sense ‘suit, befit, be appropriate’): from Old Norse sœma ‘to honour’, from sœmr ‘fitting’.}%
\par%
\entry{seemingly}{/ˈsiːmɪŋli/}{আপাতদৃষ্টিতে}{ \textsf{\textit{adverb}}\ \textbf{1} So as to give the impression of having a certain quality; apparently. {\fontspec{DejaVu Sans}◇} \textit{a seemingly competent and well{-}organized person} \colorBulletS{SYN} apparently, on the face of it, to all appearances, as far as one can see, as far as one can tell, on the surface, to all intents and purposes, outwardly, evidently, superficially, supposedly, avowedly, allegedly, professedly, purportedly}{}{}{}%
\par%
\entry{seize}{/siːz/}{বাজেয়াপ্ত করা}{ \textsf{\textit{verb}}\ \textbf{1} Take hold of suddenly and forcibly. {\fontspec{DejaVu Sans}◇} \textit{she jumped up and seized his arm} \colorBulletS{SYN} grab, grasp, snatch, seize hold of, grab hold of, take hold of, lay hold of, lay hands on, lay one's hands on, get one's hands on, take a grip of, grip, clutch, take, pluck \textbf{2} Take (an opportunity) eagerly and decisively. {\fontspec{DejaVu Sans}◇} \textit{he seized his chance to attack as Carr hesitated} \textbf{3} (of a feeling or pain) affect (someone) suddenly or acutely. {\fontspec{DejaVu Sans}◇} \textit{he was seized by the most dreadful fear} \textbf{4} Strongly appeal to or attract (the imagination or attention) {\fontspec{DejaVu Sans}◇} \textit{the story of the king's escape seized the public imagination} \textbf{5} (of a machine with moving parts) become jammed. {\fontspec{DejaVu Sans}◇} \textit{the engine seized up after only three weeks} \colorBulletS{SYN} stick, become stuck, catch, seize, seize up, become immobilized, become unable to move, become fixed, become wedged, become lodged, become trapped \textbf{6}  {\fontspec{DejaVu Sans}◇} \textit{the court is currently seized of custody applications} \textbf{7} Fasten or attach (someone or something) to something by binding with turns of rope. {\fontspec{DejaVu Sans}◇} \textit{Jack was seized to the gun and had his two dozen lashes}}{}{}{ \colorBullet{ORIGIN} Middle English from Old French seizir ‘give seisin’, from medieval Latin sacire, in the phrase ad proprium sacire ‘claim as one's own’, from a Germanic base meaning ‘procedure’.}%
\par%
\entry{seizure}{/ˈsiːʒə/}{পাকড়}{ \textsf{\textit{noun}}\ \textbf{1} The action of capturing someone or something using force. {\fontspec{DejaVu Sans}◇} \textit{the seizure of the Assembly building} \colorBulletS{SYN} capture, occupation, takeover, overrunning, annexation, annexing, invasion, conquering, subjugation, subjection, colonization \textbf{2} A sudden attack of illness, especially a stroke or an epileptic fit. {\fontspec{DejaVu Sans}◇} \textit{the patient had a seizure} \colorBulletS{SYN} convulsion, spasm, paroxysm, collapse, sudden illness, attack, fit, bout}{}{}{}%
\par%
\entry{semantic}{/sɪˈmantɪk/}{শব্দার্থিক}{ \textsf{\textit{adjective}}\ \textbf{1} Relating to meaning in language or logic. {\fontspec{DejaVu Sans}◇} \textit{} \colorBulletS{SYN} language{-}producing, semantic, lingual, semasiological}{}{}{ \colorBullet{ORIGIN} Mid 17th century from French sémantique, from Greek sēmantikos ‘significant’, from sēmainein ‘signify’, from sēma ‘sign’.}%
\par%
\entry{settle}{/ˈsɛt(ə)l/}{বসতি স্থাপন করা}{ \textsf{\textit{verb}}\ \textbf{1} Resolve or reach an agreement about (an argument or problem) {\fontspec{DejaVu Sans}◇} \textit{the unions have settled their year{-}long dispute with Hollywood producers} \colorBulletS{SYN} resolve, sort out, reach an agreement about, find a solution to, find an answer to, solve, clear up, bring to an end, fix, work out, iron out, smooth over, straighten out, deal with, put right, set right, put to rights, rectify, remedy, reconcile \textbf{2} Pay (a debt or account) {\fontspec{DejaVu Sans}◇} \textit{his bill was settled by charge card} \colorBulletS{SYN} pay, pay in full, settle up, discharge, square, clear, defray, liquidate, satisfy \textbf{3} Adopt a more steady or secure style of life, especially in a permanent job and home. {\fontspec{DejaVu Sans}◇} \textit{one day I will settle down and raise a family} \textbf{4} Sit or come to rest in a comfortable position. {\fontspec{DejaVu Sans}◇} \textit{he settled into an armchair} \colorBulletS{SYN} sit down, seat oneself, install oneself, plant oneself, ensconce oneself, plump oneself, flump}{}{}{ \colorBullet{ORIGIN} Old English setlan ‘to seat, place’, from settle.}%
\par%
\entry{settle}{/ˈsɛt(ə)l/}{বসতি স্থাপন করা}{ \textsf{\textit{noun}}\ \textbf{1} A wooden bench with a high back and arms, typically incorporating a box under the seat. {\fontspec{DejaVu Sans}◇} \textit{}}{}{}{ \colorBullet{ORIGIN} Old English setl ‘a place to sit’, of Germanic origin; related to German Sessel and Latin sella ‘seat’, also to sit.}%
\par%
\entry{settlement}{/ˈsɛt(ə)lm(ə)nt/}{বন্দোবস্ত}{ \textsf{\textit{noun}}\ \textbf{1} An official agreement intended to resolve a dispute or conflict. {\fontspec{DejaVu Sans}◇} \textit{unions succeeded in reaching a pay settlement} \colorBulletS{SYN} agreement, deal, arrangement, resolution, accommodation, bargain, understanding, pact \textbf{2} A place, typically one which has previously been uninhabited, where people establish a community. {\fontspec{DejaVu Sans}◇} \textit{one of the oldest Viking settlements in western Europe} \colorBulletS{SYN} community, colony, outpost, encampment \textbf{3} An arrangement whereby property passes to a succession of people as dictated by the settlor. {\fontspec{DejaVu Sans}◇} \textit{inheritance tax could be due if you make a substantial gift or settlement and then die within the following seven years} \textbf{4} The action or process of settling an account. {\fontspec{DejaVu Sans}◇} \textit{most suppliers will offer early settlement discounts} \colorBulletS{SYN} payment, discharge, defrayal, liquidation, settling, settling up, clearance, clearing, satisfaction \textbf{5} Subsidence of the ground or a structure built on it. {\fontspec{DejaVu Sans}◇} \textit{a boundary wall, which has cracked due to settlement, is to be replaced}}{}{}{}%
\par%
\entry{severe}{/sɪˈvɪə/}{তীব্র}{ \textsf{\textit{adjective}}\ \textbf{1} (of something bad or undesirable) very great; intense. {\fontspec{DejaVu Sans}◇} \textit{a severe shortage of technicians} \colorBulletS{SYN} acute, very bad, serious, grave, critical, dire, drastic, grievous, extreme, dreadful, terrible, awful, frightful, appalling, sore \textbf{2} (of punishment of a person) strict or harsh. {\fontspec{DejaVu Sans}◇} \textit{the charges would have warranted a severe sentence} \colorBulletS{SYN} harsh, hard, bitter, bitterly cold, cold, bleak, freezing, icy, arctic, polar, Siberian, extreme, nasty \textbf{3} Very plain in style or appearance. {\fontspec{DejaVu Sans}◇} \textit{she wore another severe suit, grey this time} \colorBulletS{SYN} plain, simple, restrained, unadorned, undecorated, unembellished, unornamented, austere, chaste, spare, stark, ultra{-}plain, unfussy, without frills, spartan, ascetic, monastic, puritanical}{}{}{ \colorBullet{ORIGIN} Mid 16th century (in severe (sense 2)): from French sévère or Latin severus.}%
\par%
\entry{severity}{/sɪˈvɛrɪti/}{নির্দয়তা}{ \textsf{\textit{noun}}\ \textbf{1} The fact or condition of being severe. {\fontspec{DejaVu Sans}◇} \textit{sentences should reflect the severity of the crime} \colorBulletS{SYN} acuteness, seriousness, gravity, graveness, severeness, grievousness, extremity}{}{}{}%
\par%
\entry{sewer}{/ˈsuːə/}{নর্দমা}{ \textsf{\textit{noun}}\ \textbf{1} An underground conduit for carrying off drainage water and waste matter. {\fontspec{DejaVu Sans}◇} \textit{} \colorBulletS{SYN} drain, sluice, sluiceway, culvert, spillway, flume, sewer}{}{}{ \colorBullet{ORIGIN} Middle English (denoting a watercourse to drain marshy land): from Old Northern French seuwiere ‘channel to drain the overflow from a fish pond’, based on Latin ex{-} ‘out of’ + aqua ‘water’.}%
\par%
\entry{sewer}{/ˈsuːə/}{নর্দমা}{ \textsf{\textit{noun}}\ \textbf{1} A person that sews. {\fontspec{DejaVu Sans}◇} \textit{}}{}{}{}%
\par%
\entry{shag}{/ʃaɡ/}{কোঁকড়া চুল}{ \textsf{\textit{noun}}\ \textbf{1} A carpet or rug with a long, rough pile. {\fontspec{DejaVu Sans}◇} \textit{wall{-}to{-}wall shag carpet} \colorBulletS{SYN} pile, fibres, threads, weave, shag, texture, feel, surface, grain \textbf{2} A thick, tangled hairstyle or mass of hair. {\fontspec{DejaVu Sans}◇} \textit{her hair was cut short in a boyish shag} \textbf{3}  {\fontspec{DejaVu Sans}◇} \textit{}}{}{}{ \colorBullet{ORIGIN} Late Old English sceacga ‘rough matted hair’, of Germanic origin; related to Old Norse skegg ‘beard’ and shaw.}%
\par%
\entry{shag}{/ʃaɡ/}{কোঁকড়া চুল}{ \textsf{\textit{noun}}\ \textbf{1} A western European and Mediterranean cormorant with greenish{-}black plumage and a long curly crest in the breeding season. {\fontspec{DejaVu Sans}◇} \textit{}}{}{}{ \colorBullet{ORIGIN} Mid 16th century perhaps a use of shag, with reference to the bird's ‘shaggy’ crest.}%
\par%
\entry{shag}{/ʃaɡ/}{কোঁকড়া চুল}{ \textsf{\textit{noun}}\ \textbf{1} A dance originating in the US in the 1930s and 1940s, characterized by vigorous hopping from one foot to the other. {\fontspec{DejaVu Sans}◇} \textit{}}{}{}{ \colorBullet{ORIGIN} Of obscure derivation; perhaps from obsolete shag ‘waggle’.}%
\par%
\entry{shag}{/ʃaɡ/}{কোঁকড়া চুল}{ \textsf{\textit{verb}}\ \textbf{1} Chase or catch (fly balls) for practice. {\fontspec{DejaVu Sans}◇} \textit{you run down to the field and hit a few baseballs and shag a few fly balls}}{}{}{ \colorBullet{ORIGIN} Early 20th century of unknown origin.}%
\par%
\entry{shag}{/ʃaɡ/}{কোঁকড়া চুল}{\small{\textsf{\textit{noun, verb}}} \\{\fontspec{DejaVu Sans}▪ }\textsf{\textit{noun}}\\ \textbf{1} An act of having sex. {\fontspec{DejaVu Sans}◇} \textit{} \\{\fontspec{DejaVu Sans}▪ }\textsf{\textit{verb}}\\ \textbf{1} Have sex with (someone). {\fontspec{DejaVu Sans}◇} \textit{} \colorBulletS{SYN} have sexual intercourse, have sexual intercourse with, make love, make love to, sleep together, sleep with, go to bed together, go to bed with}{}{}{ \colorBullet{ORIGIN} Late 18th century of unknown origin.}%
\par%
\entry{sham}{/ʃam/}{মিথ্যা}{\small{\textsf{\textit{adjective, noun, verb}}} \\{\fontspec{DejaVu Sans}▪ }\textsf{\textit{adjective}}\\ \textbf{1} Bogus; false. {\fontspec{DejaVu Sans}◇} \textit{a clergyman who arranged a sham marriage} \colorBulletS{SYN} fake, pretended, feigned, simulated, false, artificial, bogus, synthetic, spurious, ersatz, insincere, not genuine, manufactured, contrived, affected, plastic, make{-}believe, fictitious \\{\fontspec{DejaVu Sans}▪ }\textsf{\textit{noun}}\\ \textbf{1} A thing that is not what it is purported to be. {\fontspec{DejaVu Sans}◇} \textit{our current free health service is a sham} \textbf{2} short for pillow sham {\fontspec{DejaVu Sans}◇} \textit{} \\{\fontspec{DejaVu Sans}▪ }\textsf{\textit{verb}}\\ \textbf{1} Falsely present something as the truth. {\fontspec{DejaVu Sans}◇} \textit{was he ill or was he shamming?}}{}{}{ \colorBullet{ORIGIN} Late 17th century perhaps a northern English dialect variant of the noun shame.}%
\par%
\entry{shatter}{/ˈʃatə/}{ধ্বংস করা}{ \textsf{\textit{verb}}\ \textbf{1} Break or cause to break suddenly and violently into pieces. {\fontspec{DejaVu Sans}◇} \textit{bullets riddled the bar top, glasses shattered, bottles exploded} \colorBulletS{SYN} smash, smash to smithereens, break, break into pieces, burst, blow out \textbf{2} Upset (someone) greatly. {\fontspec{DejaVu Sans}◇} \textit{everyone was shattered by the news} \colorBulletS{SYN} devastating, crushing, staggering, severe, savage, overwhelming, traumatic, very great, dreadful, terrible, awful}{}{}{ \colorBullet{ORIGIN} Middle English (in the sense ‘scatter, disperse’): perhaps imitative; compare with scatter.}%
\par%
\entry{shield}{/ʃiːld/}{ঢাল}{\small{\textsf{\textit{noun, verb}}} \\{\fontspec{DejaVu Sans}▪ }\textsf{\textit{noun}}\\ \textbf{1} A broad piece of metal or another suitable material, held by straps or a handle attached on one side, used as a protection against blows or missiles. {\fontspec{DejaVu Sans}◇} \textit{} \colorBulletS{SYN} buckler, target \textbf{2} A person or thing providing protection. {\fontspec{DejaVu Sans}◇} \textit{a coating of grease provides a shield against abrasive dirt} \colorBulletS{SYN} protection, guard, defence, cover, screen, shade, safety, security, shelter, safeguard, support, bulwark, protector \textbf{3} A large rigid area of the earth's crust, typically of Precambrian rock, which has been unaffected by later orogenic episodes, e.g. the Canadian Shield. {\fontspec{DejaVu Sans}◇} \textit{} \\{\fontspec{DejaVu Sans}▪ }\textsf{\textit{verb}}\\ \textbf{1} Protect from a danger, risk, or unpleasant experience. {\fontspec{DejaVu Sans}◇} \textit{he pulled the cap lower to shield his eyes from the glare}}{}{}{ \colorBullet{ORIGIN} Old English scild (noun), scildan (verb), of Germanic origin; related to Dutch schild and German Schild, from a base meaning ‘divide, separate’.}%
\par%
\entry{shipwreck}{/ˈʃɪprɛk/}{সর্বনাশ}{\small{\textsf{\textit{noun, verb}}} \\{\fontspec{DejaVu Sans}▪ }\textsf{\textit{noun}}\\ \textbf{1} The destruction of a ship at sea by sinking or breaking up, for example in a storm or after striking a rock. {\fontspec{DejaVu Sans}◇} \textit{these islands have a history of shipwrecks and smuggling} \colorBulletS{SYN} wreck, shipwreck, ruin, shell, skeleton, hull, frame, framework, derelict \\{\fontspec{DejaVu Sans}▪ }\textsf{\textit{verb}}\\ \textbf{1} (of a person or ship) suffer a shipwreck. {\fontspec{DejaVu Sans}◇} \textit{the English envoy was shipwrecked off the coast of Sardinia and nearly drowned} \colorBulletS{SYN} foundered, ashore, beached, grounded, stuck, shipwrecked, wrecked, high and dry, on the rocks, on the bottom, on the ground}{}{}{}%
\par%
\entry{shoot{-}out}{/ˈʃuːtaʊt/}{বন্দুকযুদ্ধে}{ \textsf{\textit{noun}}\ \textbf{1} A decisive gun battle. {\fontspec{DejaVu Sans}◇} \textit{we had all got cap pistols for Christmas and gathered in Dr Hadley's backyard for a shoot{-}out} \colorBulletS{SYN} fight, conflict, armed conflict, clash, struggle, skirmish, engagement, dogfight, affray, fray, encounter, confrontation}{}{}{}%
\par%
\entry{shore}{/ʃɔː/}{কূল}{ \textsf{\textit{noun}}\ \textbf{1} The land along the edge of a sea, lake, or other large body of water. {\fontspec{DejaVu Sans}◇} \textit{I made for the shore} \colorBulletS{SYN} seashore, seaside, beach, coast, coastal region, seaboard, sea coast, bank, lakeside, verge, edge, shoreline, waterside, front, shoreside, foreshore, sand, sands}{ \colorBullet{OTHER} shore up: to give someone or something robust support in the face of difficulty or to prevent potential failure. a noun or pronoun can be used between "shore" and "up."}{Workers are trying to shore up the levee to prevent a failure.}{ \colorBullet{ORIGIN} Middle English from Middle Dutch, Middle Low German schōre; perhaps related to the verb shear.}%
\par%
\entry{shore}{/ʃɔː/}{কূল}{\small{\textsf{\textit{noun, verb}}} \\{\fontspec{DejaVu Sans}▪ }\textsf{\textit{noun}}\\ \textbf{1} A prop or beam set obliquely against something weak or unstable as a support. {\fontspec{DejaVu Sans}◇} \textit{} \\{\fontspec{DejaVu Sans}▪ }\textsf{\textit{verb}}\\ \textbf{1} Support or hold up something with props or beams. {\fontspec{DejaVu Sans}◇} \textit{rescue workers had to shore up the building, which was in danger of collapse} \colorBulletS{SYN} prop up, hold up, bolster up, support, brace, buttress, strengthen, fortify, reinforce, underpin, truss, stay}{ \colorBullet{OTHER} shore up: to give someone or something robust support in the face of difficulty or to prevent potential failure. a noun or pronoun can be used between "shore" and "up."}{Workers are trying to shore up the levee to prevent a failure.}{ \colorBullet{ORIGIN} Middle English from Middle Dutch, Middle Low German schore ‘prop’, of unknown origin.}%
\par%
\entry{shore}{/ʃɔː/}{কূল}{\small{\textsf{\textit{}}}}{ \colorBullet{OTHER} shore up: to give someone or something robust support in the face of difficulty or to prevent potential failure. a noun or pronoun can be used between "shore" and "up."}{Workers are trying to shore up the levee to prevent a failure.}{}%
\par%
\entry{shortfall}{/ˈʃɔːtfɔːl/}{ঘাটতি}{ \textsf{\textit{noun}}\ \textbf{1} A deficit of something required or expected. {\fontspec{DejaVu Sans}◇} \textit{they are facing an expected \$10 billion shortfall in revenue} \colorBulletS{SYN} defect, blemish, fault, imperfection, deficiency, weakness, weak point, weak spot, inadequacy, shortcoming, limitation, failing, foible}{}{}{}%
\par%
\entry{shout}{/ʃaʊt/}{চিৎকার}{\small{\textsf{\textit{noun, verb}}} \\{\fontspec{DejaVu Sans}▪ }\textsf{\textit{noun}}\\ \textbf{1} A loud cry expressing a strong emotion or calling attention. {\fontspec{DejaVu Sans}◇} \textit{his words were interrupted by warning shouts} \colorBulletS{SYN} yell, cry, call, roar, howl, bellow, bawl, clamour, bay, cheer, yawp, yelp, wail, squawk, shriek, scream, screech, squeal, squall, caterwaul, whoop \textbf{2} One's turn to buy a round of drinks. {\fontspec{DejaVu Sans}◇} \textit{‘Do you want another drink? My shout.’} \\{\fontspec{DejaVu Sans}▪ }\textsf{\textit{verb}}\\ \textbf{1} (of a person) utter a loud cry, typically as an expression of a strong emotion. {\fontspec{DejaVu Sans}◇} \textit{she shouted for joy} \colorBulletS{SYN} yell, cry, cry out, call, call out, roar, howl, bellow, bawl, call at the top of one's voice, clamour, bay, cheer, yawp, yelp, wail, squawk, shriek, scream, screech, squeal, squall, caterwaul, whoop \textbf{2} Treat (someone) to (something, especially a drink) {\fontspec{DejaVu Sans}◇} \textit{I'll shout you a beer}}{}{}{ \colorBullet{ORIGIN} Late Middle English perhaps related to shoot; compare with Old Norse skúta ‘a taunt’, also with the verb scout.}%
\par%
\entry{shrimp}{/ʃrɪmp/}{চিংড়ি}{\small{\textsf{\textit{noun, verb}}} \\{\fontspec{DejaVu Sans}▪ }\textsf{\textit{noun}}\\ \textbf{1} A small free{-}swimming crustacean with an elongated body, typically marine and frequently of commercial importance as food. {\fontspec{DejaVu Sans}◇} \textit{} \\{\fontspec{DejaVu Sans}▪ }\textsf{\textit{verb}}\\ \textbf{1} Fish for shrimps. {\fontspec{DejaVu Sans}◇} \textit{some families still go shrimping off the coast at Lytham}}{}{}{ \colorBullet{ORIGIN} Middle English probably related to Middle Low German schrempen ‘to wrinkle’, Middle High German schrimpfen ‘to contract’, also to scrimp.}%
\par%
\entry{shrink}{/ʃrɪŋk/}{সঙ্কুচিত করা}{\small{\textsf{\textit{noun, verb}}} \\{\fontspec{DejaVu Sans}▪ }\textsf{\textit{noun}}\\ \textbf{1} A psychiatrist. {\fontspec{DejaVu Sans}◇} \textit{you should see a shrink} \\{\fontspec{DejaVu Sans}▪ }\textsf{\textit{verb}}\\ \textbf{1} Become or make smaller in size or amount. {\fontspec{DejaVu Sans}◇} \textit{the workforce shrank to a thousand} \colorBulletS{SYN} get smaller, become smaller, grow smaller, contract, diminish, lessen, reduce, decrease, dwindle, narrow, shorten, slim, decline, fall off, drop off, condense, deflate, shrivel, wither \textbf{2} Move back or away, especially because of fear or disgust. {\fontspec{DejaVu Sans}◇} \textit{she shrank away from him, covering her face} \colorBulletS{SYN} draw back, recoil, jump back, spring back, jerk back, pull back, start back, back away, retreat, withdraw}{}{}{ \colorBullet{ORIGIN} Old English scrincan, of Germanic origin; related to Swedish skrynka ‘to wrinkle’.}%
\par%
\entry{sigh}{/sʌɪ/}{দীর্ঘশ্বাস}{\small{\textsf{\textit{noun, verb}}} \\{\fontspec{DejaVu Sans}▪ }\textsf{\textit{noun}}\\ \textbf{1} A long, deep audible exhalation expressing sadness, relief, tiredness, or similar. {\fontspec{DejaVu Sans}◇} \textit{she let out a long sigh of despair} \colorBulletS{SYN} breath, breathing out \\{\fontspec{DejaVu Sans}▪ }\textsf{\textit{verb}}\\ \textbf{1} Emit a long, deep audible breath expressing sadness, relief, tiredness, or similar. {\fontspec{DejaVu Sans}◇} \textit{Harry sank into a chair and sighed with relief} \colorBulletS{SYN} breathe out, exhale}{}{}{ \colorBullet{ORIGIN} Middle English (as a verb): probably a back{-}formation from sighte, past tense of siche, sike, from Old English sīcan.}%
\par%
\entry{sight}{/sʌɪt/}{দৃষ্টিশক্তি}{\small{\textsf{\textit{noun, verb}}} \\{\fontspec{DejaVu Sans}▪ }\textsf{\textit{noun}}\\ \textbf{1} The faculty or power of seeing. {\fontspec{DejaVu Sans}◇} \textit{Joseph lost his sight as a baby} \colorBulletS{SYN} eyesight, vision, eyes, faculty of sight, power of sight, ability to see, visual perception, observation \textbf{2} A thing that one sees or that can be seen. {\fontspec{DejaVu Sans}◇} \textit{John was a familiar sight in the bar for many years} \textbf{3} A device on a gun or optical instrument used for assisting a person's precise aim or observation. {\fontspec{DejaVu Sans}◇} \textit{there were reports of a man on the roof aiming a rifle and looking through its sights} \\{\fontspec{DejaVu Sans}▪ }\textsf{\textit{verb}}\\ \textbf{1} Manage to see or observe (someone or something); catch an initial glimpse of. {\fontspec{DejaVu Sans}◇} \textit{tell me when you sight London Bridge} \colorBulletS{SYN} glimpse, catch a glimpse of, get a glimpse of, catch sight of, see, spot, spy, notice, observe, make out, pick out, detect, have sight of \textbf{2} Take aim by looking through the sights of a gun. {\fontspec{DejaVu Sans}◇} \textit{she sighted down the barrel}}{}{}{ \colorBullet{ORIGIN} Old English (ge)sihth ‘something seen’, of West Germanic origin; related to Dutch zicht and German Gesicht ‘sight, face, appearance’. The verb dates from the mid 16th century (in sight (sense 2 of the verb)).}%
\par%
\entry{signatory}{/ˈsɪɡnət(ə)ri/}{দস্তখতকারী}{ \textsf{\textit{noun}}\ \textbf{1} A party that has signed an agreement, especially a state that has signed a treaty. {\fontspec{DejaVu Sans}◇} \textit{Britain is a signatory to the convention}}{}{}{ \colorBullet{ORIGIN} Late 19th century from Latin signatorius ‘of sealing’, from signat{-} ‘marked (with a cross)’, from the verb signare.}%
\par%
\entry{silt}{/sɪlt/}{পলি}{\small{\textsf{\textit{noun, verb}}} \\{\fontspec{DejaVu Sans}▪ }\textsf{\textit{noun}}\\ \textbf{1} Fine sand, clay, or other material carried by running water and deposited as a sediment, especially in a channel or harbour. {\fontspec{DejaVu Sans}◇} \textit{} \colorBulletS{SYN} sediment, deposit, alluvium, mud, slime, ooze, sludge \\{\fontspec{DejaVu Sans}▪ }\textsf{\textit{verb}}\\ \textbf{1} Become filled or blocked with silt. {\fontspec{DejaVu Sans}◇} \textit{the river's mouth had silted up} \colorBulletS{SYN} become blocked, become choked, become clogged, fill up, fill up with silt, become filled, become dammed}{}{}{ \colorBullet{ORIGIN} Late Middle English probably originally denoting a salty deposit and of Scandinavian origin, related to Danish and Norwegian sylt ‘salt marsh’, also to salt.}%
\par%
\entry{sin}{/sɪn/}{পাপ}{\small{\textsf{\textit{noun, verb}}} \\{\fontspec{DejaVu Sans}▪ }\textsf{\textit{noun}}\\ \textbf{1} An immoral act considered to be a transgression against divine law. {\fontspec{DejaVu Sans}◇} \textit{a sin in the eyes of God} \colorBulletS{SYN} immoral act, wrong, wrongdoing, act of evil, act of wickedness, transgression, crime, offence, misdeed, misdemeanour, error, lapse, fall from grace \\{\fontspec{DejaVu Sans}▪ }\textsf{\textit{verb}}\\ \textbf{1} Commit a sin. {\fontspec{DejaVu Sans}◇} \textit{I sinned and brought shame down on us} \colorBulletS{SYN} commit a sin, offend against God, commit an offence, transgress, do wrong, commit a crime, break the law, misbehave, go astray, stray from the straight and narrow, go wrong, fall from grace}{}{}{ \colorBullet{ORIGIN} Old English synn (noun), syngian (verb); probably related to Latin sons, sont{-} ‘guilty’.}%
\par%
\entry{sin}{/sʌɪn/}{পাপ}{ \textsf{\textit{abbreviation}}\ \textbf{1} Sine. {\fontspec{DejaVu Sans}◇} \textit{}}{}{}{}%
\par%
\entry{sixfold}{/ˈsɪksfəʊld/}{ছয় গুণ}{\small{\textsf{\textit{adjective, adverb}}} \\{\fontspec{DejaVu Sans}▪ }\textsf{\textit{adjective}}\\ \textbf{1} Six times as great or as numerous. {\fontspec{DejaVu Sans}◇} \textit{a sixfold increase in their overheads} \\{\fontspec{DejaVu Sans}▪ }\textsf{\textit{adverb}}\\ \textbf{1} By six times; to six times the number or amount. {\fontspec{DejaVu Sans}◇} \textit{coal prices have risen sixfold}}{}{}{}%
\par%
\entry{skid}{/skɪd/}{পিছলাইয়া পড়া}{\small{\textsf{\textit{noun, verb}}} \\{\fontspec{DejaVu Sans}▪ }\textsf{\textit{noun}}\\ \textbf{1} An act of skidding or sliding. {\fontspec{DejaVu Sans}◇} \textit{the Volvo went into a skid} \colorBulletS{SYN} fit of rage, rage, fury, fit of bad temper, fit of ill temper, bad temper, tantrum, passion, paroxysm \textbf{2} A runner attached to the underside of an aircraft for use when landing on snow or grass. {\fontspec{DejaVu Sans}◇} \textit{} \textbf{3} A braking device consisting of a wooden or metal shoe preventing a wheel from revolving. {\fontspec{DejaVu Sans}◇} \textit{} \textbf{4} A beam or plank of wood used to support a ship under construction or repair. {\fontspec{DejaVu Sans}◇} \textit{Contrast that with a gas turbine, which is shipped on a skid and essentially needs only to be hooked up.} \\{\fontspec{DejaVu Sans}▪ }\textsf{\textit{verb}}\\ \textbf{1} (of a vehicle) slide, typically sideways or obliquely, on slippery ground or as a result of stopping or turning too quickly. {\fontspec{DejaVu Sans}◇} \textit{her car skidded and hit the grass verge} \colorBulletS{SYN} glide, move lightly, slide, sail, plane, scud, skate, float, coast \textbf{2} Fasten a skid to (a wheel) as a brake. {\fontspec{DejaVu Sans}◇} \textit{}}{}{}{ \colorBullet{ORIGIN} Late 17th century (as a noun in the sense ‘supporting beam’): perhaps related to Old Norse skíth (see ski).}%
\par%
\entry{skinny}{/ˈskɪni/}{চর্মসার}{\small{\textsf{\textit{adjective, noun}}} \\{\fontspec{DejaVu Sans}▪ }\textsf{\textit{adjective}}\\ \textbf{1} (of a person or part of their body) unattractively thin. {\fontspec{DejaVu Sans}◇} \textit{his skinny arms} \colorBulletS{SYN} thin, scrawny, scraggy, bony, angular, raw{-}boned, hollow{-}cheeked, gaunt, as thin as a rake, skin{-}and{-}bones, sticklike, size{-}zero, emaciated, skeletal, pinched, undernourished, underfed \textbf{2} (of a garment) tight{-}fitting. {\fontspec{DejaVu Sans}◇} \textit{a skinny jumper} \textbf{3} (of coffee) made with skimmed or semi{-}skimmed milk. {\fontspec{DejaVu Sans}◇} \textit{one skinny latte to go, please} \\{\fontspec{DejaVu Sans}▪ }\textsf{\textit{noun}}\\ \textbf{1} A skinny person. {\fontspec{DejaVu Sans}◇} \textit{} \textbf{2} A pair of skinny jeans or trousers. {\fontspec{DejaVu Sans}◇} \textit{if you're tired of squeezing into your skinnies, bell{-}bottoms and flares are back in fashion} \textbf{3} Confidential information on a particular person or topic. {\fontspec{DejaVu Sans}◇} \textit{net managers who want the skinny on the latest in computer security}}{}{}{}%
\par%
\entry{skipper}{/ˈskɪpə/}{অধিনায়ক}{\small{\textsf{\textit{noun, verb}}} \\{\fontspec{DejaVu Sans}▪ }\textsf{\textit{noun}}\\ \textbf{1} The captain of a ship or boat, especially a small trading or fishing vessel. {\fontspec{DejaVu Sans}◇} \textit{the skipper and one other man were convicted of smuggling} \colorBulletS{SYN} commander, master, skipper \\{\fontspec{DejaVu Sans}▪ }\textsf{\textit{verb}}\\ \textbf{1} Act as captain of. {\fontspec{DejaVu Sans}◇} \textit{the course teaches even complete beginners to skipper their own yachts} \colorBulletS{SYN} fly, be at the controls of, control, handle, manoeuvre, drive, operate, steer, regulate, monitor, direct, captain}{}{}{ \colorBullet{ORIGIN} Late Middle English from Middle Dutch, Middle Low German schipper, from schip ‘ship’.}%
\par%
\entry{skipper}{/ˈskɪpə/}{অধিনায়ক}{ \textsf{\textit{noun}}\ \textbf{1} A person or thing that skips. {\fontspec{DejaVu Sans}◇} \textit{eight{-}year{-}old Mary is a tireless skipper} \textbf{2} A small brownish mothlike butterfly with rapid darting flight. {\fontspec{DejaVu Sans}◇} \textit{} \textbf{3} The Atlantic saury (fish). {\fontspec{DejaVu Sans}◇} \textit{}}{}{}{}%
\par%
\entry{skipper}{/ˈskɪpə/}{অধিনায়ক}{ \textsf{\textit{noun}}\ \textbf{1} A long{-}sleeved sweatshirt or T{-}shirt. {\fontspec{DejaVu Sans}◇} \textit{}}{}{}{ \colorBullet{ORIGIN} Of unknown origin.}%
\par%
\entry{skirt}{/skəːt/}{স্কার্ট}{\small{\textsf{\textit{noun, verb}}} \\{\fontspec{DejaVu Sans}▪ }\textsf{\textit{noun}}\\ \textbf{1} A woman's outer garment fastened around the waist and hanging down around the legs. {\fontspec{DejaVu Sans}◇} \textit{} \textbf{2} Women regarded as objects of sexual desire. {\fontspec{DejaVu Sans}◇} \textit{so, Sandro, off to chase some skirt?} \textbf{3} A surface that conceals or protects the wheels or underside of a vehicle or aircraft. {\fontspec{DejaVu Sans}◇} \textit{} \textbf{4} An animal's diaphragm and other membranes as food. {\fontspec{DejaVu Sans}◇} \textit{bits of beef skirt} \textbf{5} A small flap on a saddle, covering the bar from which the stirrup leather hangs. {\fontspec{DejaVu Sans}◇} \textit{I pulled myself slowly into the saddle, arranging the skirts carefully.} \\{\fontspec{DejaVu Sans}▪ }\textsf{\textit{verb}}\\ \textbf{1} Go round or past the edge of. {\fontspec{DejaVu Sans}◇} \textit{he did not go through the city but skirted it} \colorBulletS{SYN} go round, move round, walk round, circle, circumnavigate \textbf{2} Attempt to ignore; avoid dealing with. {\fontspec{DejaVu Sans}◇} \textit{they are both skirting the issue} \colorBulletS{SYN} avoid, evade, steer clear of, sidestep, dodge, circumvent, bypass, pass over, fight shy of}{}{}{ \colorBullet{ORIGIN} Middle English from Old Norse skyrta ‘shirt’; compare with synonymous Old English scyrte, also with short. The verb dates from the early 17th century.}%
\par%
\entry{slain}{/sleɪn/}{নিহত}{\small{\textsf{\textit{}}}}{}{}{}%
\par%
\entry{slaked lime}{}{চুন, জলে ভেজানোর পরে}{\small{\textsf{\textit{}}}}{}{}{}%
\par%
\entry{slide}{/slʌɪd/}{স্লাইড্}{\small{\textsf{\textit{noun, verb}}} \\{\fontspec{DejaVu Sans}▪ }\textsf{\textit{noun}}\\ \textbf{1} A structure with a smooth sloping surface for children to slide down. {\fontspec{DejaVu Sans}◇} \textit{Anna played on the slide} \colorBulletS{SYN} water slide, slide, flume, log flume, hydroslide \textbf{2} An act of moving along a smooth surface while maintaining continuous contact with it. {\fontspec{DejaVu Sans}◇} \textit{use an ice axe to halt a slide on ice and snow} \textbf{3} A decline in value or quality. {\fontspec{DejaVu Sans}◇} \textit{the current slide in house prices} \colorBulletS{SYN} fall, decline, drop, slump, tumble, downturn, downswing \textbf{4} A part of a machine or instrument that slides. {\fontspec{DejaVu Sans}◇} \textit{} \textbf{5} A rectangular piece of glass on which an object is mounted or placed for examination under a microscope. {\fontspec{DejaVu Sans}◇} \textit{} \textbf{6} another term for hairslide {\fontspec{DejaVu Sans}◇} \textit{her hair was held back with a tortoiseshell slide} \textbf{7} A sandal or light shoe without a back. {\fontspec{DejaVu Sans}◇} \textit{} \\{\fontspec{DejaVu Sans}▪ }\textsf{\textit{verb}}\\ \textbf{1} Move smoothly along a surface while maintaining continuous contact with it. {\fontspec{DejaVu Sans}◇} \textit{she slid down the bank into the water} \colorBulletS{SYN} glide, move smoothly, slip, slither, skim, skate, glissade, coast, plane}{}{}{ \colorBullet{ORIGIN} Old English slīdan (verb); related to sled and sledge. The noun, first in the sense ‘act of sliding’, is recorded from the late 16th century.}%
\par%
\entry{slime}{/slʌɪm/}{পাঁক}{\small{\textsf{\textit{noun, verb}}} \\{\fontspec{DejaVu Sans}▪ }\textsf{\textit{noun}}\\ \textbf{1} An unpleasantly thick and slippery liquid substance. {\fontspec{DejaVu Sans}◇} \textit{the cold stone was wet with slime} \colorBulletS{SYN} ooze, sludge, muck, mud, mire \\{\fontspec{DejaVu Sans}▪ }\textsf{\textit{verb}}\\ \textbf{1} Cover with slime. {\fontspec{DejaVu Sans}◇} \textit{what grass remained was slimed over with pale brown mud}}{}{}{ \colorBullet{ORIGIN} Old English slīm, of Germanic origin; related to Dutch slijm and German Schleim ‘mucus, slime’, Latin limus ‘mud’, and Greek limnē ‘marsh’.}%
\par%
\entry{sling}{/slɪŋ/}{গুল্তি ছোড়া}{\small{\textsf{\textit{noun, verb}}} \\{\fontspec{DejaVu Sans}▪ }\textsf{\textit{noun}}\\ \textbf{1} A flexible strap or belt used in the form of a loop to support or raise a hanging weight. {\fontspec{DejaVu Sans}◇} \textit{the horse had to be supported by a sling fixed to the roof} \textbf{2} A simple weapon in the form of a strap or loop, used to hurl stones or other small missiles. {\fontspec{DejaVu Sans}◇} \textit{700 men armed only with slings} \colorBulletS{SYN} catapult, slingshot \textbf{3} A bribe or gratuity. {\fontspec{DejaVu Sans}◇} \textit{} \\{\fontspec{DejaVu Sans}▪ }\textsf{\textit{verb}}\\ \textbf{1} Suspend or arrange (something), especially with a strap or straps, so that it hangs loosely in a particular position. {\fontspec{DejaVu Sans}◇} \textit{a hammock was slung between two trees} \colorBulletS{SYN} hang, suspend, string, dangle, swing, drape \textbf{2} Casually throw or fling. {\fontspec{DejaVu Sans}◇} \textit{sling a few things into your knapsack} \colorBulletS{SYN} throw, toss, fling, hurl, cast, pitch, lob, launch, flip, shy, catapult, send flying, let fly with \textbf{3} Pay a bribe or gratuity. {\fontspec{DejaVu Sans}◇} \textit{they didn't forget to sling when the backhanders came in}}{}{}{ \colorBullet{ORIGIN} Middle English probably from Low German, of symbolic origin; compare with German Schlinge ‘noose, snare’. sling (sense 2 of the verb) is from Old Norse slyngva.}%
\par%
\entry{sling}{/slɪŋ/}{গুল্তি ছোড়া}{ \textsf{\textit{noun}}\ \textbf{1} A sweetened drink of spirits, especially gin, and water. {\fontspec{DejaVu Sans}◇} \textit{}}{}{}{ \colorBullet{ORIGIN} Mid 18th century of unknown origin.}%
\par%
\entry{slum}{/slʌm/}{ঘিঁচি ঘিঁচি বস্তি}{\small{\textsf{\textit{noun, verb}}} \\{\fontspec{DejaVu Sans}▪ }\textsf{\textit{noun}}\\ \textbf{1} A squalid and overcrowded urban street or district inhabited by very poor people. {\fontspec{DejaVu Sans}◇} \textit{inner{-}city slums} \colorBulletS{SYN} hovel \\{\fontspec{DejaVu Sans}▪ }\textsf{\textit{verb}}\\ \textbf{1} Spend time at a lower social level than one's own through curiosity or for charitable purposes. {\fontspec{DejaVu Sans}◇} \textit{he bought some second{-}hand clothes, and slummed among the metropolis's underprivileged}}{}{}{ \colorBullet{ORIGIN} Early 19th century (originally slang, in the sense ‘room’): of unknown origin.}%
\par%
\entry{slumber party}{}{An overnight gathering especially of teenage girls usually at one of their homes}{\small{\textsf{\textit{}}}}{}{}{}%
\par%
\entry{slump}{/slʌmp/}{অতিমন্দা}{\small{\textsf{\textit{noun, verb}}} \\{\fontspec{DejaVu Sans}▪ }\textsf{\textit{noun}}\\ \textbf{1} A sudden severe or prolonged fall in the price, value, or amount of something. {\fontspec{DejaVu Sans}◇} \textit{a slump in profits} \colorBulletS{SYN} steep fall, plunge, drop, collapse, tumble, plummet, downturn, downswing, slide, decline, falling off, decrease, lowering, devaluation, depreciation \\{\fontspec{DejaVu Sans}▪ }\textsf{\textit{verb}}\\ \textbf{1} Sit, lean, or fall heavily and limply. {\fontspec{DejaVu Sans}◇} \textit{she slumped against the cushions} \colorBulletS{SYN} sit heavily, flop, flump, collapse, sink, fall, subside \textbf{2} Undergo a sudden severe or prolonged fall in price, value, or amount. {\fontspec{DejaVu Sans}◇} \textit{land prices slumped} \colorBulletS{SYN} fall steeply, plummet, plunge, tumble, drop, go down, slide, decline, decrease}{}{\newline%
\newline%
1. Analyst say the ongoing slump has been heightened by a surge in sales in recent years… 2. China auto sales slump continues in april.}{ \colorBullet{ORIGIN} Late 17th century (in the sense ‘fall into a bog’): probably imitative and related to Norwegian slumpe ‘to fall’.}%
\par%
\entry{slut}{/slʌt/}{বেশ্যা}{ \textsf{\textit{noun}}\ \textbf{1} A woman who has many casual sexual partners. {\fontspec{DejaVu Sans}◇} \textit{} \colorBulletS{SYN} promiscuous woman \textbf{2} A woman with low standards of cleanliness. {\fontspec{DejaVu Sans}◇} \textit{Although she was handsome in a blowsy way, she was such a slut, with holes in her stockings and grubby straps showing.}}{}{}{ \colorBullet{ORIGIN} Middle English of unknown origin.}%
\par%
\entry{smoldering}{/ˈsmōldəriNG/}{ধিকিধিকি}{\small{\textsf{\textit{adjective, noun}}} \\{\fontspec{DejaVu Sans}▪ }\textsf{\textit{adjective}}\\ \textbf{1} Burning slowly with smoke but no flame. {\fontspec{DejaVu Sans}◇} \textit{a smoldering fire} \\{\fontspec{DejaVu Sans}▪ }\textsf{\textit{noun}}\\ \textbf{1} The process of burning slowly with smoke but no flame. {\fontspec{DejaVu Sans}◇} \textit{the smoldering can go unnoticed for many days before smoke starts to be seen}}{}{}{}%
\par%
\entry{snag}{/snaɡ/}{অপ্রত্যাশিত বাধা}{\small{\textsf{\textit{noun, verb}}} \\{\fontspec{DejaVu Sans}▪ }\textsf{\textit{noun}}\\ \textbf{1} An unexpected or hidden obstacle or drawback. {\fontspec{DejaVu Sans}◇} \textit{there's one small snag} \colorBulletS{SYN} obstacle, difficulty, complication, catch, hitch, stumbling block, pitfall, unseen problem, problem, issue, barrier, impediment, hindrance, inconvenience, setback, hurdle, disadvantage, downside, drawback, minus \textbf{2} A sharp, angular, or jagged projection. {\fontspec{DejaVu Sans}◇} \textit{keep an emery board handy in case of nail snags} \colorBulletS{SYN} sharp projection, jag, jagged bit \textbf{3} A dead tree. {\fontspec{DejaVu Sans}◇} \textit{dozens of species of birds and mammals use standing snags for nesting} \\{\fontspec{DejaVu Sans}▪ }\textsf{\textit{verb}}\\ \textbf{1} Catch or tear (something) on a sharp projection. {\fontspec{DejaVu Sans}◇} \textit{thorns snagged his sweater} \colorBulletS{SYN} tear, rip, ladder, gash \textbf{2} Catch or obtain. {\fontspec{DejaVu Sans}◇} \textit{it's the first time they've snagged the star for a photo}}{}{}{ \colorBullet{ORIGIN} Late 16th century (in snag (sense 2 of the noun)): probably of Scandinavian origin. The early sense ‘stump sticking out from a tree trunk’ gave rise to a US sense ‘submerged piece of timber obstructing navigation’, of which sense 1 is originally a figurative use. Current verb senses arose in the 19th century.}%
\par%
\entry{snag}{/snaɡ/}{অপ্রত্যাশিত বাধা}{ \textsf{\textit{noun}}\ \textbf{1} A sausage. {\fontspec{DejaVu Sans}◇} \textit{I make my own snags, my own pies and pasties}}{}{}{ \colorBullet{ORIGIN} 1940s of unknown origin.}%
\par%
\entry{snail}{/sneɪl/}{শামুক}{ \textsf{\textit{noun}}\ \textbf{1} A mollusc with a single spiral shell into which the whole body can be withdrawn. {\fontspec{DejaVu Sans}◇} \textit{}}{}{Snail's pace: an extremely slow pace}{ \colorBullet{ORIGIN} Old English snæg(e)l, of Germanic origin; related to German Schnecke.}%
\par%
\entry{snap}{/snap/}{ক্ষুদ্র তালা}{\small{\textsf{\textit{adjective, noun, verb}}} \\{\fontspec{DejaVu Sans}▪ }\textsf{\textit{adjective}}\\ \textbf{1} Done or taken on the spur of the moment, unexpectedly, or without notice. {\fontspec{DejaVu Sans}◇} \textit{a snap decision} \colorBulletS{SYN} unrehearsed, unprepared, unscripted, extempore, extemporized, improvised, improvisational, improvisatory, improvisatorial, spontaneous, unstudied, unpremeditated, unarranged, unplanned, on the spot, snap, ad lib \\{\fontspec{DejaVu Sans}▪ }\textsf{\textit{noun}}\\ \textbf{1} A sudden, sharp cracking sound or movement. {\fontspec{DejaVu Sans}◇} \textit{she closed her purse with a snap} \colorBulletS{SYN} click, crack, pop, clink, tick, report, smack, whack, crackle \textbf{2} A hurried, irritable tone or manner. {\fontspec{DejaVu Sans}◇} \textit{‘I'm still waiting,’ he said with a snap} \textbf{3} A snapshot. {\fontspec{DejaVu Sans}◇} \textit{holiday snaps} \colorBulletS{SYN} photograph, picture, photo, shot, snapshot, likeness, image, portrait, study, print, slide, transparency, negative, positive, plate, film, bromide, frame, exposure, still, proof, enprint, enlargement \textbf{4} A card game in which cards from two piles are turned over simultaneously and players call ‘snap’ as quickly as possible when two similar cards are exposed. {\fontspec{DejaVu Sans}◇} \textit{} \textbf{5} A sudden brief spell of cold or otherwise distinctive weather. {\fontspec{DejaVu Sans}◇} \textit{a cold snap} \colorBulletS{SYN} period, spell, time, interval, season, stretch, run \textbf{6} Food, especially food taken to work to be eaten during a break. {\fontspec{DejaVu Sans}◇} \textit{I hurried to get the snap which just meant that I bodged the job and had to do it again.} \textbf{7} An easy task. {\fontspec{DejaVu Sans}◇} \textit{a control panel that makes operation a snap} \colorBulletS{SYN} easy task, easy job, child's play, five{-}finger exercise, gift, walkover, nothing, sinecure, gravy train \textbf{8} A quick backward movement of the ball from the ground that begins a play. {\fontspec{DejaVu Sans}◇} \textit{} \textbf{9} A small fastener on clothing, engaged by pressing its two halves together; a press stud. {\fontspec{DejaVu Sans}◇} \textit{a black cloth jacket with a lot of snaps and attachments} \\{\fontspec{DejaVu Sans}▪ }\textsf{\textit{verb}}\\ \textbf{1} Break suddenly and completely, typically with a sharp cracking sound. {\fontspec{DejaVu Sans}◇} \textit{guitar strings kept snapping} \colorBulletS{SYN} break, break in two, break into two, fracture, splinter, separate, come apart, part, split, crack \textbf{2} (of an animal) make a sudden audible bite. {\fontspec{DejaVu Sans}◇} \textit{a dog was snapping at his heels} \colorBulletS{SYN} bite, gnash its teeth \textbf{3} Suddenly lose one's self{-}control. {\fontspec{DejaVu Sans}◇} \textit{she claims she snapped after years of violence} \colorBulletS{SYN} lose one's self{-}control, crack, freak, freak out, get overwrought, go to pieces, get hysterical, get worked up, flare up \textbf{4} Take a snapshot of. {\fontspec{DejaVu Sans}◇} \textit{he planned to spend the time snapping rare wildlife} \colorBulletS{SYN} photograph, get a photo of, get a photograph of, take a photo of, take a photograph of, take someone's photo, take someone's picture, get a picture of, take a picture of, picture, get a snap of, get a snapshot of, take a snap of, take a snapshot of, take, shoot, get a shot of, take a shot of, take a likeness of, record, film, capture on celluloid, capture on film, record on celluloid, record on film \textbf{5} Put (the ball) into play by a quick backward movement. {\fontspec{DejaVu Sans}◇} \textit{time will not be resumed until the ball is snapped on the next play}}{}{}{ \colorBullet{ORIGIN} Late 15th century (in the senses ‘make a sudden audible bite’ and ‘quick sharp biting sound’): probably from Middle Dutch or Middle Low German snappen ‘seize’; partly imitative.}%
\par%
\entry{snatch}{/snatʃ/}{ছিনান}{\small{\textsf{\textit{noun, verb}}} \\{\fontspec{DejaVu Sans}▪ }\textsf{\textit{noun}}\\ \textbf{1} An act of snatching or quickly seizing something. {\fontspec{DejaVu Sans}◇} \textit{a quick snatch of breath} \textbf{2} The rapid raising of a weight from the floor to above the head in one movement. {\fontspec{DejaVu Sans}◇} \textit{} \textbf{3} A woman's genitals. {\fontspec{DejaVu Sans}◇} \textit{} \\{\fontspec{DejaVu Sans}▪ }\textsf{\textit{verb}}\\ \textbf{1} Quickly seize (something) in a rude or eager way. {\fontspec{DejaVu Sans}◇} \textit{she snatched a biscuit from the plate} \colorBulletS{SYN} grab, seize, seize hold of, grab hold of, take hold of, lay hold of, lay hands on, lay one's hands on, get one's hands on, take, pluck}{}{}{ \colorBullet{ORIGIN} Middle English sna(c)che (verb) ‘suddenly snap at’, (noun) ‘a snare’; perhaps related to snack.}%
\par%
\entry{sneak}{/sniːk/}{ছিঁচকে চোর}{\small{\textsf{\textit{adjective, noun, verb}}} \\{\fontspec{DejaVu Sans}▪ }\textsf{\textit{adjective}}\\ \textbf{1} Acting or done surreptitiously, unofficially, or without warning. {\fontspec{DejaVu Sans}◇} \textit{a sneak thief} \colorBulletS{SYN} furtive, secret, stealthy, sly, surreptitious, clandestine, covert \\{\fontspec{DejaVu Sans}▪ }\textsf{\textit{noun}}\\ \textbf{1} (especially in children's use) someone who informs an adult or person in authority of a companion's misdeeds; a telltale. {\fontspec{DejaVu Sans}◇} \textit{Ethel was the form sneak and goody{-}goody} \colorBulletS{SYN} informer, betrayer, stool pigeon \textbf{2} short for sneaker {\fontspec{DejaVu Sans}◇} \textit{} \\{\fontspec{DejaVu Sans}▪ }\textsf{\textit{verb}}\\ \textbf{1} Move or go in a furtive or stealthy way. {\fontspec{DejaVu Sans}◇} \textit{I sneaked out by the back exit} \colorBulletS{SYN} creep, slink, steal, slip, slide, sidle, edge, move furtively, tiptoe, pussyfoot, pad, prowl \textbf{2} (especially in children's use) inform an adult or person in authority of a companion's misdeeds; tell tales. {\fontspec{DejaVu Sans}◇} \textit{she sneaked on us} \colorBulletS{SYN} inform, inform against, inform on, act as an informer, tell tales, tell tales on, report, give someone away, be disloyal, be disloyal to, sell someone out, stab someone in the back}{}{}{ \colorBullet{ORIGIN} Late 16th century probably dialect; perhaps related to obsolete snike ‘to creep’.}%
\par%
\entry{sneeze}{/sniːz/}{হাঁচি}{\small{\textsf{\textit{noun, verb}}} \\{\fontspec{DejaVu Sans}▪ }\textsf{\textit{noun}}\\ \textbf{1} An act or the sound of sneezing. {\fontspec{DejaVu Sans}◇} \textit{he stopped a sudden sneeze} \\{\fontspec{DejaVu Sans}▪ }\textsf{\textit{verb}}\\ \textbf{1} Make a sudden involuntary expulsion of air from the nose and mouth due to irritation of one's nostrils. {\fontspec{DejaVu Sans}◇} \textit{the smoke made her sneeze}}{}{}{ \colorBullet{ORIGIN} Middle English apparently an alteration of Middle English fnese due to misreading or misprinting (after initial fn{-} had become unfamiliar), later adopted because it sounded appropriate.}%
\par%
\entry{snowflake}{/ˈsnəʊfleɪk/}{তুষারকণা}{ \textsf{\textit{noun}}\ \textbf{1} A flake of snow, especially a feathery ice crystal, typically displaying delicate sixfold symmetry. {\fontspec{DejaVu Sans}◇} \textit{} \textbf{2} An overly sensitive or easily offended person, or one who believes they are entitled to special treatment on account of their supposedly unique characteristics. {\fontspec{DejaVu Sans}◇} \textit{these little snowflakes will soon discover that life doesn't come with trigger warnings} \colorBulletS{SYN} coward, namby{-}pamby, milksop, mouse, weakling, milquetoast \textbf{3} A white{-}flowered Eurasian plant related to and resembling the snowdrop, typically blooming in the summer or autumn. {\fontspec{DejaVu Sans}◇} \textit{}}{}{}{}%
\par%
\entry{so it would seem}{}{}{\small{\textsf{\textit{}}}}{}{}{}%
\par%
\entry{so, listen, fellas, who's up for little party this saturday night?}{}{}{\small{\textsf{\textit{}}}}{}{}{}%
\par%
\entry{soar}{/sɔː/}{উড্ডীন করা}{ \textsf{\textit{verb}}\ \textbf{1} Fly or rise high in the air. {\fontspec{DejaVu Sans}◇} \textit{the bird spread its wings and soared into the air} \colorBulletS{SYN} fly up, wing, wing its way}{}{}{ \colorBullet{ORIGIN} Late Middle English shortening of Old French essorer, based on Latin ex{-} ‘out of’ + aura ‘breeze’.}%
\par%
\entry{sober}{/ˈsəʊbə/}{প্রশান্ত}{\small{\textsf{\textit{adjective, verb}}} \\{\fontspec{DejaVu Sans}▪ }\textsf{\textit{adjective}}\\ \textbf{1} Not affected by alcohol; not drunk. {\fontspec{DejaVu Sans}◇} \textit{} \colorBulletS{SYN} not drunk, not intoxicated, clear{-}headed, as sober as a judge \textbf{2} Serious, sensible, and solemn. {\fontspec{DejaVu Sans}◇} \textit{a sober view of life} \colorBulletS{SYN} serious, sensible, solemn, thoughtful, grave, sombre, severe, earnest, sedate, staid, dignified, steady, level{-}headed, serious{-}minded, businesslike, down{-}to{-}earth, commonsensical, pragmatic, self{-}controlled, restrained, conservative \\{\fontspec{DejaVu Sans}▪ }\textsf{\textit{verb}}\\ \textbf{1} Make or become sober after drinking alcohol. {\fontspec{DejaVu Sans}◇} \textit{that coffee sobered him up} \colorBulletS{SYN} become sober, become clear{-}headed}{}{}{ \colorBullet{ORIGIN} Middle English from Old French sobre, from Latin sobrius.}%
\par%
\entry{soil}{/sɔɪl/}{মাটি}{ \textsf{\textit{noun}}\ \textbf{1} The upper layer of earth in which plants grow, a black or dark brown material typically consisting of a mixture of organic remains, clay, and rock particles. {\fontspec{DejaVu Sans}◇} \textit{blueberries need very acid soil} \colorBulletS{SYN} earth, loam, sod, ground, dirt, clay, turf, topsoil, mould, humus, marl, dust}{}{}{ \colorBullet{ORIGIN} Late Middle English from Anglo{-}Norman French, perhaps representing Latin solium ‘seat’, by association with solum ‘ground’.}%
\par%
\entry{soil}{/sɔɪl/}{মাটি}{\small{\textsf{\textit{noun, verb}}} \\{\fontspec{DejaVu Sans}▪ }\textsf{\textit{noun}}\\ \textbf{1} Waste matter, especially sewage containing excrement. {\fontspec{DejaVu Sans}◇} \textit{} \\{\fontspec{DejaVu Sans}▪ }\textsf{\textit{verb}}\\ \textbf{1} Make dirty. {\fontspec{DejaVu Sans}◇} \textit{he might soil his expensive suit} \colorBulletS{SYN} dirty, get dirty, make dirty, get filthy, make filthy, blacken, grime, begrime, stain, muddy, splash, spot, spatter, splatter, smear, smudge, sully, spoil, defile, pollute, contaminate, foul, befoul}{}{}{ \colorBullet{ORIGIN} Middle English (as a verb): from Old French soiller, based on Latin sucula, diminutive of sus ‘pig’. The earliest use of the noun (late Middle English) was ‘muddy wallow for wild boar’; current noun senses date from the early 16th century.}%
\par%
\entry{soil}{/sɔɪl/}{মাটি}{ \textsf{\textit{verb}}\ \textbf{1} Feed (cattle) on fresh{-}cut green fodder (originally for the purpose of purging them). {\fontspec{DejaVu Sans}◇} \textit{Indian corn makes an exceedingly valuable fodder, both as a means of carrying a herd of milch cows through our severe droughts of summer, and as an article for soiling cows kept in the stall.}}{}{}{ \colorBullet{ORIGIN} Early 17th century perhaps from soil.}%
\par%
\entry{sole}{/səʊl/}{একমাত্র}{\small{\textsf{\textit{noun, verb}}} \\{\fontspec{DejaVu Sans}▪ }\textsf{\textit{noun}}\\ \textbf{1} The undersurface of a person's foot. {\fontspec{DejaVu Sans}◇} \textit{the soles of their feet were nearly black with dirt} \\{\fontspec{DejaVu Sans}▪ }\textsf{\textit{verb}}\\ \textbf{1} Put a new sole on to (a shoe) {\fontspec{DejaVu Sans}◇} \textit{he wanted several pairs of boots to be soled and heeled}}{}{}{ \colorBullet{ORIGIN} Middle English from Old French, from Latin solea ‘sandal, sill’, from solum ‘bottom, pavement, sole’; compare with Dutch zool and German Sohle.}%
\par%
\entry{sole}{/səʊl/}{একমাত্র}{ \textsf{\textit{noun}}\ \textbf{1} A marine flatfish of almost worldwide distribution, important as a food fish. {\fontspec{DejaVu Sans}◇} \textit{}}{}{}{ \colorBullet{ORIGIN} Middle English from Old French, from Provençal sola, from Latin solea (see sole), named from its shape.}%
\par%
\entry{sole}{/səʊl/}{একমাত্র}{ \textsf{\textit{adjective}}\ \textbf{1} One and only. {\fontspec{DejaVu Sans}◇} \textit{my sole aim was to contribute to the national team} \colorBulletS{SYN} only, one, one and only, single, solitary, lone, unique, only possible, individual, exclusive, singular \textbf{2} (especially of a woman) unmarried. {\fontspec{DejaVu Sans}◇} \textit{}}{}{}{ \colorBullet{ORIGIN} Late Middle English (also in the senses ‘secluded’ and ‘unrivalled’): from Old French soule, from Latin sola, feminine of solus ‘alone’.}%
\par%
\entry{Sole}{/səʊl/}{একমাত্র}{ \textsf{\textit{proper noun}}\ \textbf{1} A shipping forecast area in the north{-}eastern Atlantic, covering the western approaches to the English Channel. {\fontspec{DejaVu Sans}◇} \textit{}}{}{}{}%
\par%
\entry{solely}{/ˈsəʊlli/}{কেবলমাত্র}{ \textsf{\textit{adverb}}\ \textbf{1} Not involving anyone or anything else; only. {\fontspec{DejaVu Sans}◇} \textit{he is solely responsible for any debts the company may incur} \colorBulletS{SYN} only, simply, just, merely, uniquely, exclusively, entirely, completely, absolutely, totally, wholly, alone, no more than, to the exclusion of everyone else, to the exclusion of everything else}{}{There was a time when bangladesh was solely depended on importing computer machineries from abroad}{}%
\par%
\entry{solo}{/ˈsəʊləʊ/}{একাকী}{\small{\textsf{\textit{adjective \& adverb, noun, verb}}} \\{\fontspec{DejaVu Sans}▪ }\textsf{\textit{adjective \& adverb}}\\ \textbf{1} For or done by one person alone; unaccompanied. {\fontspec{DejaVu Sans}◇} \textit{a solo album} \colorBulletS{SYN} unaccompanied, single{-}handed, companionless, unescorted, unattended, unchaperoned, independent, lonely, solitary \\{\fontspec{DejaVu Sans}▪ }\textsf{\textit{noun}}\\ \textbf{1} A piece of vocal or instrumental music or a dance, or a part or passage in one, for one performer. {\fontspec{DejaVu Sans}◇} \textit{the opening bassoon solo is relatively bland} \textbf{2} An unaccompanied flight by a pilot in an aircraft. {\fontspec{DejaVu Sans}◇} \textit{his first ride in his aircraft would also be his first solo} \textbf{3}  {\fontspec{DejaVu Sans}◇} \textit{Solo whist is a plain{-}trick game with trumps and bidding, closely related to the more elaborate and now obsolete game of Boston.} \textbf{4} A motorbike without a sidecar. {\fontspec{DejaVu Sans}◇} \textit{50 races—solos and sidecars—should make for a thrilling showdown} \\{\fontspec{DejaVu Sans}▪ }\textsf{\textit{verb}}\\ \textbf{1} Perform an unaccompanied piece of music or a part or passage in one. {\fontspec{DejaVu Sans}◇} \textit{you're in danger of forgetting that you're accompanying rather than soloing} \textbf{2} Fly an aircraft unaccompanied. {\fontspec{DejaVu Sans}◇} \textit{she had been flying for twelve years and had soloed on her seventeenth birthday}}{}{}{ \colorBullet{ORIGIN} Late 17th century (as a musical term): from Italian, from Latin solus ‘alone’.}%
\par%
\entry{sophisticated}{/səˈfɪstɪkeɪtɪd/}{বাস্তববুদ্ধিসম্পন্ন}{ \textsf{\textit{adjective}}\ \textbf{1} Having, revealing, or involving a great deal of worldly experience and knowledge of fashion and culture. {\fontspec{DejaVu Sans}◇} \textit{a chic, sophisticated woman} \colorBulletS{SYN} worldly, worldly{-}wise, experienced, enlightened, cosmopolitan, knowledgeable \textbf{2} (of a machine, system, or technique) developed to a high degree of complexity. {\fontspec{DejaVu Sans}◇} \textit{highly sophisticated computer systems} \colorBulletS{SYN} advanced, highly developed, innovatory, trailblazing, revolutionary}{}{}{}%
\par%
\entry{sordid}{/ˈsɔːdɪd/}{নোংরা}{ \textsf{\textit{adjective}}\ \textbf{1} Involving immoral or dishonourable actions and motives; arousing moral distaste and contempt. {\fontspec{DejaVu Sans}◇} \textit{the story paints a sordid picture of bribes and scams} \colorBulletS{SYN} sleazy, seedy, seamy, unsavoury, shoddy, vile, foul, tawdry, louche, cheap, base, low, low{-}minded, debased, degenerate, corrupt, dishonest, dishonourable, disreputable, despicable, discreditable, contemptible, ignominious, ignoble, shameful, wretched, abhorrent, abominable, disgusting \textbf{2} Dirty or squalid. {\fontspec{DejaVu Sans}◇} \textit{the overcrowded housing conditions were sordid and degrading} \colorBulletS{SYN} dirty, filthy, mucky, grimy, muddy, grubby, shabby, messy, soiled, stained, smeared, smeary, scummy, slimy, sticky, sooty, dusty, unclean, foul, squalid, flea{-}bitten, slummy}{}{}{ \colorBullet{ORIGIN} Late Middle English (as a medical term in the sense ‘purulent’): from French sordide or Latin sordidus, from sordere ‘be dirty’. The current senses date from the early 17th century.}%
\par%
\entry{sought}{/sɔːt/}{চাওয়া}{\small{\textsf{\textit{}}}}{}{Much{-}sought}{}%
\par%
\entry{soul}{/səʊl/}{আত্মা}{ \textsf{\textit{noun}}\ \textbf{1} The spiritual or immaterial part of a human being or animal, regarded as immortal. {\fontspec{DejaVu Sans}◇} \textit{} \colorBulletS{SYN} soul, psyche, inner self, inner being, essential being \textbf{2} Emotional or intellectual energy or intensity, especially as revealed in a work of art or an artistic performance. {\fontspec{DejaVu Sans}◇} \textit{their interpretation lacked soul} \colorBulletS{SYN} inspiration, feeling, emotion, passion, animation, intensity, fervour, ardour, enthusiasm, eagerness, warmth, energy, vitality, vivacity, spirit, spiritedness, commitment \textbf{3} The essence or embodiment of a specified quality. {\fontspec{DejaVu Sans}◇} \textit{he was the soul of discretion} \colorBulletS{SYN} embodiment, personification, incarnation, epitome, quintessence, essence}{}{}{ \colorBullet{ORIGIN} Old English sāwol, sāw(e)l, of Germanic origin; related to Dutch ziel and German Seele.}%
\par%
\entry{sour}{/saʊə/}{টক}{\small{\textsf{\textit{adjective, noun, verb}}} \\{\fontspec{DejaVu Sans}▪ }\textsf{\textit{adjective}}\\ \textbf{1} Having an acid taste like lemon or vinegar. {\fontspec{DejaVu Sans}◇} \textit{she sampled the wine and found it was sour} \colorBulletS{SYN} acid, acidy, acidic, acidulated, tart, bitter, sharp, acetic, vinegary, pungent, acrid, biting, stinging, burning, smarting, unpleasant, distasteful \textbf{2} Feeling or expressing resentment, disappointment, or anger. {\fontspec{DejaVu Sans}◇} \textit{he gave her a sour look} \colorBulletS{SYN} embittered, resentful, nasty, spiteful, sharp{-}tongued, irritable, irascible, peevish, fractious, fretful, cross, crabbed, crabby, crotchety, cantankerous, curmudgeonly, disagreeable, petulant, pettish \textbf{3} (of soil) deficient in lime and usually dank. {\fontspec{DejaVu Sans}◇} \textit{Our soil is on the sour side and lays wet in spots, as the old{-}timers say.} \textbf{4} (of petroleum or natural gas) containing a relatively high proportion of sulphur. {\fontspec{DejaVu Sans}◇} \textit{} \\{\fontspec{DejaVu Sans}▪ }\textsf{\textit{noun}}\\ \textbf{1} A drink made by mixing a spirit with lemon or lime juice. {\fontspec{DejaVu Sans}◇} \textit{a rum sour} \\{\fontspec{DejaVu Sans}▪ }\textsf{\textit{verb}}\\ \textbf{1} Make or become sour. {\fontspec{DejaVu Sans}◇} \textit{water soured with tamarind}}{}{}{ \colorBullet{ORIGIN} Old English sūr, of Germanic origin; related to Dutch zuur and German sauer.}%
\par%
\entry{sovereign}{/ˈsɒvrɪn/}{সার্বভৌম}{\small{\textsf{\textit{adjective, noun}}} \\{\fontspec{DejaVu Sans}▪ }\textsf{\textit{adjective}}\\ \textbf{1} Possessing supreme or ultimate power. {\fontspec{DejaVu Sans}◇} \textit{in modern democracies the people's will is in theory sovereign} \colorBulletS{SYN} supreme, absolute, unlimited, unrestricted, unrestrained, unbounded, boundless, infinite, ultimate, total, unconditional, full, utter, paramount \textbf{2} Very good or effective. {\fontspec{DejaVu Sans}◇} \textit{a sovereign remedy for all ills} \colorBulletS{SYN} effective, efficient, powerful, potent, efficacious, effectual \\{\fontspec{DejaVu Sans}▪ }\textsf{\textit{noun}}\\ \textbf{1} A supreme ruler, especially a monarch. {\fontspec{DejaVu Sans}◇} \textit{the Emperor became the first Japanese sovereign to visit Britain} \colorBulletS{SYN} ruler, monarch, supreme ruler, Crown, crowned head, head of state, potentate, suzerain, overlord, dynast, leader \textbf{2} A former British gold coin worth one pound sterling, now only minted for commemorative purposes. {\fontspec{DejaVu Sans}◇} \textit{}}{}{}{ \colorBullet{ORIGIN} Middle English from Old French soverain, based on Latin super ‘above’. The change in the ending was due to association with reign.}%
\par%
\entry{sow}{/səʊ/}{বুনা}{ \textsf{\textit{verb}}\ \textbf{1} Plant (seed) by scattering it on or in the earth. {\fontspec{DejaVu Sans}◇} \textit{fill a pot with compost and sow a thin layer of seeds on top} \colorBulletS{SYN} scatter, spread, broadcast, disperse, strew, disseminate, distribute \textbf{2} Disseminate or introduce (something undesirable) {\fontspec{DejaVu Sans}◇} \textit{the new policy has sown confusion and doubt} \colorBulletS{SYN} cause, bring about, occasion, create, give rise to, lead to, produce, engender, generate, induce, invite, implant, plant, lodge, prompt, evoke, elicit, initiate, precipitate, instigate, trigger, spark off, provoke}{}{}{ \colorBullet{ORIGIN} Old English sāwan, of Germanic origin; related to Dutch zaaien and German säen.}%
\par%
\entry{sow}{/saʊ/}{বুনা}{ \textsf{\textit{noun}}\ \textbf{1} An adult female pig, especially one which has farrowed. {\fontspec{DejaVu Sans}◇} \textit{} \textbf{2} A large block of metal (larger than a ‘pig’) made by smelting. {\fontspec{DejaVu Sans}◇} \textit{He said most of the stock is ingot, whereas more consumers prefer T{-}bar or sow.}}{}{}{ \colorBullet{ORIGIN} Old English sugu; related to Dutch zeug, German Sau, from an Indo{-}European root shared by Latin sus and Greek hus ‘pig’.}%
\par%
\entry{spank}{/spaŋk/}{পাছায় বেত প্রভৃতি দিয়ে মারা}{\small{\textsf{\textit{noun, verb}}} \\{\fontspec{DejaVu Sans}▪ }\textsf{\textit{noun}}\\ \textbf{1} A slap or series of slaps with one's open hand or a flat object. {\fontspec{DejaVu Sans}◇} \textit{when his father caught him he got a spank} \colorBulletS{SYN} blow, thump, punch, knock, bang, thwack, box, cuff, slap, smack, spank, tap, crack, stroke, welt \\{\fontspec{DejaVu Sans}▪ }\textsf{\textit{verb}}\\ \textbf{1} Slap with one's open hand or a flat object, especially on the buttocks as a punishment. {\fontspec{DejaVu Sans}◇} \textit{she was spanked for spilling ink on the carpet} \colorBulletS{SYN} smack, slap, slipper, put someone over one's knee, thrash, cane, belt, leather, cuff}{}{Physical punishment in our social context is not limited to a light disciplinary spanking.}{ \colorBullet{ORIGIN} Early 18th century perhaps imitative.}%
\par%
\entry{spare}{/spɛː/}{অতিরিক্ত}{\small{\textsf{\textit{adjective, noun, verb}}} \\{\fontspec{DejaVu Sans}▪ }\textsf{\textit{adjective}}\\ \textbf{1} Additional to what is required for ordinary use. {\fontspec{DejaVu Sans}◇} \textit{few people had spare cash for inessentials} \colorBulletS{SYN} extra, supplementary, additional, second, another, alternative, emergency, reserve, backup, relief, fallback, substitute, fresh, auxiliary, ancillary \textbf{2} With no excess fat; thin. {\fontspec{DejaVu Sans}◇} \textit{a spare, bearded figure} \colorBulletS{SYN} slender, lean \textbf{3} Elegantly simple. {\fontspec{DejaVu Sans}◇} \textit{her clothes are smart and spare in style} \\{\fontspec{DejaVu Sans}▪ }\textsf{\textit{noun}}\\ \textbf{1} An item kept in case another item of the same type is lost, broken, or worn out. {\fontspec{DejaVu Sans}◇} \textit{the wheel's broken and it would be suicide to go on without a spare} \textbf{2} (in tenpin bowling) an act of knocking down all the pins with two balls. {\fontspec{DejaVu Sans}◇} \textit{} \\{\fontspec{DejaVu Sans}▪ }\textsf{\textit{verb}}\\ \textbf{1} Give (something of which one has enough) to (someone) {\fontspec{DejaVu Sans}◇} \textit{she asked if I could spare her a bob or two} \colorBulletS{SYN} afford, do without, manage without, get along without, dispense with, part with, give, let someone have, provide \textbf{2} Refrain from killing, injuring, or distressing. {\fontspec{DejaVu Sans}◇} \textit{there was no way the men would spare her} \colorBulletS{SYN} not harm, leave uninjured, leave unhurt \textbf{3} Be frugal. {\fontspec{DejaVu Sans}◇} \textit{but some will spend, and some will spare}}{}{}{ \colorBullet{ORIGIN} Old English spær ‘not plentiful, meagre’, sparian ‘refrain from injuring’, ‘refrain from using’, of Germanic origin; related to Dutch and German sparen ‘to spare’.}%
\par%
\entry{speculate}{/ˈspɛkjʊleɪt/}{ফটকা খেলা}{ \textsf{\textit{verb}}\ \textbf{1} Form a theory or conjecture about a subject without firm evidence. {\fontspec{DejaVu Sans}◇} \textit{my colleagues speculate about my private life} \colorBulletS{SYN} conjecture, theorize, form theories, hypothesize, make suppositions, postulate, guess, make guesses, surmise \textbf{2} Invest in stocks, property, or other ventures in the hope of gain but with the risk of loss. {\fontspec{DejaVu Sans}◇} \textit{he didn't look as though he had the money to speculate in shares} \colorBulletS{SYN} gamble, take a chance, take a risk, venture, take a venture, wager}{}{}{ \colorBullet{ORIGIN} Late 16th century from Latin speculat{-} ‘observed from a vantage point’, from the verb speculari, from specula ‘watchtower’, from specere ‘to look’.}%
\par%
\entry{spill}{/spɪl/}{ঝরা}{\small{\textsf{\textit{noun, verb}}} \\{\fontspec{DejaVu Sans}▪ }\textsf{\textit{noun}}\\ \textbf{1} A quantity of liquid that has spilled or been spilt. {\fontspec{DejaVu Sans}◇} \textit{wipe up spills immediately} \textbf{2} A fall from a horse or bicycle. {\fontspec{DejaVu Sans}◇} \textit{} \colorBulletS{SYN} fall, tumble, accident \textbf{3} A vacating of all or several posts in a cabinet or parliamentary party to allow reorganization after an important change of office. {\fontspec{DejaVu Sans}◇} \textit{} \\{\fontspec{DejaVu Sans}▪ }\textsf{\textit{verb}}\\ \textbf{1} Cause or allow (liquid) to flow over the edge of its container, especially unintentionally. {\fontspec{DejaVu Sans}◇} \textit{you'll spill that tea if you're not careful} \colorBulletS{SYN} knock over, tip over, upset, overturn \textbf{2} Reveal (confidential information) to someone. {\fontspec{DejaVu Sans}◇} \textit{she ought not to be spilling out her troubles to you} \colorBulletS{SYN} reveal, disclose, divulge, let out, leak, blurt out, babble, betray, make known, tell \textbf{3} Cause to fall off a horse or bicycle. {\fontspec{DejaVu Sans}◇} \textit{the horse was wrenched off course, spilling his rider} \colorBulletS{SYN} unseat, throw, dislodge, unhorse}{}{}{ \colorBullet{ORIGIN} Old English spillan ‘kill, destroy, waste, shed (blood’); of unknown origin.}%
\par%
\entry{spill}{/spɪl/}{ঝরা}{ \textsf{\textit{noun}}\ \textbf{1} A thin strip of wood or paper used for lighting a fire, candle, pipe, etc. {\fontspec{DejaVu Sans}◇} \textit{In front of us stood a low oaken table on which there was more mead and wine, and, appropriately for the room, a collection of long clay pipes, loose tobacco and spills.}}{}{}{ \colorBullet{ORIGIN} Middle English (in the sense ‘sharp fragment of wood’): obscurely related to spile. The current sense dates from the early 19th century.}%
\par%
\entry{spoil}{/spɔɪl/}{লুণ্ঠন}{\small{\textsf{\textit{noun, verb}}} \\{\fontspec{DejaVu Sans}▪ }\textsf{\textit{noun}}\\ \textbf{1} Goods stolen or taken forcibly from a person or place. {\fontspec{DejaVu Sans}◇} \textit{the looters carried their spoils away} \colorBulletS{SYN} booty, loot, stolen goods, plunder, ill{-}gotten gains, haul, pickings, takings \textbf{2} Waste material brought up during the course of an excavation or a dredging or mining operation. {\fontspec{DejaVu Sans}◇} \textit{colliery spoil} \\{\fontspec{DejaVu Sans}▪ }\textsf{\textit{verb}}\\ \textbf{1} Diminish or destroy the value or quality of. {\fontspec{DejaVu Sans}◇} \textit{I wouldn't want to spoil your fun} \colorBulletS{SYN} mar, damage, impair, blemish, disfigure, blight, flaw, deface, scar, injure, harm \textbf{2} Harm the character of (someone, especially a child) by being too lenient or indulgent. {\fontspec{DejaVu Sans}◇} \textit{the last thing I want to do is spoil Thomas} \colorBulletS{SYN} overindulge, pamper, indulge, mollycoddle, cosset, coddle, baby, spoon{-}feed, feather{-}bed, wait on hand and foot, cater to someone's every whim, wrap in cotton wool, overparent, kill with kindness \textbf{3} Be extremely or aggressively eager for. {\fontspec{DejaVu Sans}◇} \textit{Cooper was spoiling for a fight} \colorBulletS{SYN} eager for, itching for, looking for, keen to have, raring for, after, bent on, set on, on the lookout for, longing for \textbf{4} Rob (a person or a place) of goods or possessions by force or violence. {\fontspec{DejaVu Sans}◇} \textit{the enemy entered into Hereford, spoiled and fired the city, and razed the walls to the ground} \colorBulletS{SYN} ransack, steal from, plunder, rob, raid, loot, rifle, sack}{}{}{ \colorBullet{ORIGIN} Middle English (in the sense ‘to plunder’): shortening of Old French espoille (noun), espoillier (verb), from Latin spoliare, from spolium ‘plunder, skin stripped from an animal’, or a shortening of despoil.}%
\par%
\entry{spontaneously}{/spɒnˈteɪnɪəsli/}{এমনি}{ \textsf{\textit{adverb}}\ \textbf{1} As a result of a sudden impulse and without premeditation. {\fontspec{DejaVu Sans}◇} \textit{the crowd spontaneously burst into song} \colorBulletS{SYN} without being asked, of one's own accord, voluntarily, on impulse, impulsively, on the spur of the moment, extempore, extemporaneously}{}{}{}%
\par%
\entry{spot}{/spɒt/}{অকুস্থল}{\small{\textsf{\textit{noun, verb}}} \\{\fontspec{DejaVu Sans}▪ }\textsf{\textit{noun}}\\ \textbf{1} A small round or roundish mark, differing in colour or texture from the surface around it. {\fontspec{DejaVu Sans}◇} \textit{ladybirds have black spots on their red wing covers} \colorBulletS{SYN} mark, patch, pop, dot, speck, speckle, fleck, smudge, smear, stain, blotch, blot, splash, daub \textbf{2} A particular place or point. {\fontspec{DejaVu Sans}◇} \textit{a nice secluded spot} \colorBulletS{SYN} place, location, site, position, point, situation, scene, setting, locale, locality, area, neighbourhood, region \textbf{3} A small amount of something. {\fontspec{DejaVu Sans}◇} \textit{a spot of rain} \colorBulletS{SYN} bit, little, some, small amount, morsel, modicum, bite \textbf{4} Denoting a system of trading in which commodities or currencies are delivered and paid for immediately after a sale. {\fontspec{DejaVu Sans}◇} \textit{trading in the spot markets} \textbf{5} short for spotlight {\fontspec{DejaVu Sans}◇} \textit{} \textbf{6}  {\fontspec{DejaVu Sans}◇} \textit{} \textbf{7} A banknote of a specified value. {\fontspec{DejaVu Sans}◇} \textit{a ten{-}spot} \\{\fontspec{DejaVu Sans}▪ }\textsf{\textit{verb}}\\ \textbf{1} See, notice, or recognize (someone or something) that is difficult to detect or that one is searching for. {\fontspec{DejaVu Sans}◇} \textit{Andrew spotted the advert in the paper} \colorBulletS{SYN} notice, see, observe, discern, detect, perceive, make out, pick out, distinguish, recognize, identify, locate \textbf{2} Mark or become marked with spots. {\fontspec{DejaVu Sans}◇} \textit{the velvet was spotted with stains} \colorBulletS{SYN} stain, mark, fleck, speckle, blotch, mottle, smudge, streak, splash, spatter, bespatter \textbf{3} Rain slightly. {\fontspec{DejaVu Sans}◇} \textit{it was still spotting with rain} \colorBulletS{SYN} rain lightly, drizzle \textbf{4} Place (a ball) on its designated starting point on a billiard table. {\fontspec{DejaVu Sans}◇} \textit{} \textbf{5} Give or lend (money) to (someone) {\fontspec{DejaVu Sans}◇} \textit{I'll spot you \$300}}{}{}{ \colorBullet{ORIGIN} Middle English perhaps from Middle Dutch spotte. The sense ‘notice, recognize’ arose from the early 19th century slang use ‘note as a suspect or criminal’.}%
\par%
\entry{spotlight}{/ˈspɒtlʌɪt/}{স্পটলাইট}{\small{\textsf{\textit{noun, verb}}} \\{\fontspec{DejaVu Sans}▪ }\textsf{\textit{noun}}\\ \textbf{1} A lamp projecting a narrow, intense beam of light directly on to a place or person, especially a performer on stage. {\fontspec{DejaVu Sans}◇} \textit{} \\{\fontspec{DejaVu Sans}▪ }\textsf{\textit{verb}}\\ \textbf{1} Illuminate with a spotlight. {\fontspec{DejaVu Sans}◇} \textit{the dancers are spotlighted from time to time throughout the evening}}{}{}{}%
\par%
\entry{spotted}{/ˈspɒtɪd/}{তিলকিত}{ \textsf{\textit{adjective}}\ \textbf{1} Marked or decorated with spots. {\fontspec{DejaVu Sans}◇} \textit{a red spotted handkerchief} \colorBulletS{SYN} mottled, dappled, dapple, pied, piebald, brindled, brindle, speckled, speckly, flecked, specked, stippled}{}{}{}%
\par%
\entry{sprawl}{/sprɔːl/}{টানাটানি করা}{\small{\textsf{\textit{noun, verb}}} \\{\fontspec{DejaVu Sans}▪ }\textsf{\textit{noun}}\\ \textbf{1} An ungainly or carelessly relaxed position in which one's arms and legs are spread out. {\fontspec{DejaVu Sans}◇} \textit{she fell into a sort of luxurious sprawl} \\{\fontspec{DejaVu Sans}▪ }\textsf{\textit{verb}}\\ \textbf{1} Sit, lie, or fall with one's arms and legs spread out in an ungainly way. {\fontspec{DejaVu Sans}◇} \textit{the door shot open, sending him sprawling across the pavement} \colorBulletS{SYN} stretch out, lounge, loll, lie, lie down, lie back, recline, drape oneself, be recumbent, be prostrate, be supine, slump, flop, slouch}{}{}{ \colorBullet{ORIGIN} Old English spreawlian ‘move the limbs convulsively’; related to Danish sprælle ‘kick or splash about’. The noun dates from the early 18th century.}%
\par%
\entry{spread}{/sprɛd/}{বিস্তার}{\small{\textsf{\textit{noun, verb}}} \\{\fontspec{DejaVu Sans}▪ }\textsf{\textit{noun}}\\ \textbf{1} The fact or process of spreading over an area. {\fontspec{DejaVu Sans}◇} \textit{warmer temperatures could help reduce the spread of the disease} \colorBulletS{SYN} expansion, proliferation, extension, growth, mushrooming, increase, escalation, buildout, advance, advancement, development \textbf{2} The extent, width, or area covered by something. {\fontspec{DejaVu Sans}◇} \textit{the male's antlers can attain a spread of six feet} \colorBulletS{SYN} span, width, extent, stretch, reach \textbf{3} The range or variety of something. {\fontspec{DejaVu Sans}◇} \textit{a wide spread of ages} \colorBulletS{SYN} range, span, spectrum, sweep \textbf{4} A soft paste that can be applied in a layer to bread or other food. {\fontspec{DejaVu Sans}◇} \textit{low{-}fat spreads} \colorBulletS{SYN} spread, pâté \textbf{5} An article or advertisement covering several columns or pages of a newspaper or magazine, especially one on two facing pages. {\fontspec{DejaVu Sans}◇} \textit{a double{-}page spread} \textbf{6} A large and impressively elaborate meal. {\fontspec{DejaVu Sans}◇} \textit{his mother laid on a huge spread} \colorBulletS{SYN} elaborate meal, large meal, feast, banquet, repast \textbf{7} A bedspread. {\fontspec{DejaVu Sans}◇} \textit{a patchwork spread} \colorBulletS{SYN} bedspread, bedcover, cover, coverlet, throw, afghan \\{\fontspec{DejaVu Sans}▪ }\textsf{\textit{verb}}\\ \textbf{1} Open out (something) so as to extend its surface area, width, or length. {\fontspec{DejaVu Sans}◇} \textit{I spread a towel on the sand and sat down} \colorBulletS{SYN} lay out, open out, unfurl, unroll, roll out, shake out \textbf{2} Extend over a large or increasing area. {\fontspec{DejaVu Sans}◇} \textit{rain over north{-}west Scotland will spread south{-}east during the day} \colorBulletS{SYN} grow, increase, escalate, advance, develop, broaden, expand, widen, proliferate, mushroom \textbf{3} Apply (a substance) to an object or surface in an even layer. {\fontspec{DejaVu Sans}◇} \textit{he sighed, spreading jam on a croissant} \colorBulletS{SYN} smear, daub, plaster, slather, lather, apply, put \textbf{4} Lay (a table) for a meal. {\fontspec{DejaVu Sans}◇} \textit{On November 25, 2003, we sat down with family and friends around a table spread with food we grew and said thanks.}}{}{}{ \colorBullet{ORIGIN} Old English {-}sprǣdan (used in combinations), of West Germanic origin; related to Dutch spreiden and German spreiten.}%
\par%
\entry{spunk}{/spʌŋk/}{তেজ}{ \textsf{\textit{noun}}\ \textbf{1} Courage and determination. {\fontspec{DejaVu Sans}◇} \textit{she's got no spunk, or she'd have left him long ago} \colorBulletS{SYN} courage, bravery, pluck, pluckiness, courageousness, braveness, valour, mettle, gameness, daring \textbf{2} Semen. {\fontspec{DejaVu Sans}◇} \textit{} \textbf{3} A sexually attractive person. {\fontspec{DejaVu Sans}◇} \textit{}}{}{}{ \colorBullet{ORIGIN} Mid 16th century (in the sense ‘a spark, vestige’): of unknown origin; perhaps a blend of spark and obsolete funk ‘spark’.}%
\par%
\entry{squirt}{/skwəːt/}{ফোয়ারা}{\small{\textsf{\textit{noun, verb}}} \\{\fontspec{DejaVu Sans}▪ }\textsf{\textit{noun}}\\ \textbf{1} A thin stream or small quantity of liquid squirted from something. {\fontspec{DejaVu Sans}◇} \textit{a squirt of perfume} \colorBulletS{SYN} spurt, jet, spray, spritz, fountain, gush, stream, surge, flow \textbf{2} A puny or insignificant person. {\fontspec{DejaVu Sans}◇} \textit{what did he see in this patronizing little squirt?} \colorBulletS{SYN} impudent person, insignificant person, gnat, insect \textbf{3} A compressed radio signal transmitted at high speed. {\fontspec{DejaVu Sans}◇} \textit{The squirt signal is a burst of alternating voltage signal.} \\{\fontspec{DejaVu Sans}▪ }\textsf{\textit{verb}}\\ \textbf{1} Cause (a liquid) to be ejected from a small opening in a thin, fast stream or jet. {\fontspec{DejaVu Sans}◇} \textit{she squirted soda into a glass} \colorBulletS{SYN} squirt, shoot, spray, fountain, jet, erupt \textbf{2} Transmit (information) in highly compressed or speeded{-}up form. {\fontspec{DejaVu Sans}◇} \textit{radio equipment could squirt a million words from one continent to another}}{}{}{ \colorBullet{ORIGIN} Middle English (as a verb): imitative.}%
\par%
\entry{stab}{/stab/}{ছুরিকাঘাত}{\small{\textsf{\textit{noun, verb}}} \\{\fontspec{DejaVu Sans}▪ }\textsf{\textit{noun}}\\ \textbf{1} A thrust with a knife or other pointed weapon. {\fontspec{DejaVu Sans}◇} \textit{multiple stab wounds} \colorBulletS{SYN} lunge, thrust, jab, poke, prod, dig, punch \textbf{2} An attempt to do (something) {\fontspec{DejaVu Sans}◇} \textit{Meredith made a feeble stab at joining in} \colorBulletS{SYN} attempt, try, effort, endeavour \\{\fontspec{DejaVu Sans}▪ }\textsf{\textit{verb}}\\ \textbf{1} Thrust a knife or other pointed weapon into (someone) so as to wound or kill. {\fontspec{DejaVu Sans}◇} \textit{he stabbed her in the stomach} \colorBulletS{SYN} knife, run through, skewer, spear, bayonet, gore, spike, stick, impale, transfix, pierce, prick, puncture, penetrate, perforate, gash, slash, cut, tear, scratch, wound, injure}{}{}{ \colorBullet{ORIGIN} Late Middle English of unknown origin.}%
\par%
\entry{stacked}{/stakt/}{স্তুপীকৃত}{ \textsf{\textit{adjective}}\ \textbf{1} (of a number of things) put or arranged in a stack or stacks. {\fontspec{DejaVu Sans}◇} \textit{the stacked chairs} \textbf{2} (of a pack of cards) shuffled or arranged dishonestly so as to gain an unfair advantage. {\fontspec{DejaVu Sans}◇} \textit{you were playing against a stacked deck} \textbf{3} (of a woman) having large breasts. {\fontspec{DejaVu Sans}◇} \textit{} \colorBulletS{SYN} large{-}breasted, big{-}breasted, full{-}breasted, heavy{-}breasted, bosomy, large{-}bosomed, big{-}bosomed, full{-}bosomed \textbf{4} (of a task) placed in a queue for subsequent processing. {\fontspec{DejaVu Sans}◇} \textit{an operating system that allows for stacked jobs}}{}{}{}%
\par%
\entry{staggering}{/ˈstaɡərɪŋ/}{টলটলায়মান}{ \textsf{\textit{adjective}}\ \textbf{1} Deeply shocking; astonishing. {\fontspec{DejaVu Sans}◇} \textit{the staggering bills for maintenance and repair}}{}{}{}%
\par%
\entry{stain}{/steɪn/}{দাগ}{\small{\textsf{\textit{noun, verb}}} \\{\fontspec{DejaVu Sans}▪ }\textsf{\textit{noun}}\\ \textbf{1} A coloured patch or dirty mark that is difficult to remove. {\fontspec{DejaVu Sans}◇} \textit{there were mud stains on my shoes} \colorBulletS{SYN} mark, spot, spatter, splatter, blotch, blemish, smudge, smear \textbf{2} A penetrative dye or chemical used in colouring a material or object. {\fontspec{DejaVu Sans}◇} \textit{} \colorBulletS{SYN} tint, colour, dye, tinge, shade, pigment, colourant \\{\fontspec{DejaVu Sans}▪ }\textsf{\textit{verb}}\\ \textbf{1} Mark or discolour with something that is not easily removed. {\fontspec{DejaVu Sans}◇} \textit{her clothing was stained with blood} \colorBulletS{SYN} discolour, blemish, soil, mark, muddy, spot, spatter, splatter, smear, splash, smudge, blotch, blacken \textbf{2} Colour (a material or object) by applying a penetrative dye or chemical. {\fontspec{DejaVu Sans}◇} \textit{wood can always be stained to a darker shade} \colorBulletS{SYN} colour, tint, dye, tinge, shade, pigment}{}{}{ \colorBullet{ORIGIN} Late Middle English (as a verb): shortening of archaic distain, from Old French desteindre ‘tinge with a colour different from the natural one’. The noun was first recorded (mid 16th century) in the sense ‘defilement, disgrace’.}%
\par%
\entry{stale}{/steɪl/}{মামুলি}{\small{\textsf{\textit{adjective, verb}}} \\{\fontspec{DejaVu Sans}▪ }\textsf{\textit{adjective}}\\ \textbf{1} (of food) no longer fresh and pleasant to eat; hard, musty, or dry. {\fontspec{DejaVu Sans}◇} \textit{stale bread} \colorBulletS{SYN} dry, dried out, hard, hardened, old, past its best, past its sell{-}by date \\{\fontspec{DejaVu Sans}▪ }\textsf{\textit{verb}}\\ \textbf{1} Make or become stale. {\fontspec{DejaVu Sans}◇} \textit{she would cut up yesterday's leftover bread, staling now}}{}{}{ \colorBullet{ORIGIN} Middle English (describing beer in the sense ‘clear from long standing, strong’): probably from Anglo{-}Norman French and Old French, from estaler ‘to halt’; compare with the verb stall.}%
\par%
\entry{stale}{/steɪl/}{মামুলি}{ \textsf{\textit{verb}}\ \textbf{1} (of an animal, especially a horse) urinate. {\fontspec{DejaVu Sans}◇} \textit{the horse staled while he was riding}}{}{}{ \colorBullet{ORIGIN} Late Middle English perhaps from Old French estaler ‘come to a stand, halt’ (compare with stale).}%
\par%
\entry{stall}{/stɔːl/}{স্থগিত}{\small{\textsf{\textit{noun, verb}}} \\{\fontspec{DejaVu Sans}▪ }\textsf{\textit{noun}}\\ \textbf{1} A stand, booth, or compartment for the sale of goods in a market or large covered area. {\fontspec{DejaVu Sans}◇} \textit{fruit and vegetable stalls} \colorBulletS{SYN} stand, table, counter, booth, kiosk, compartment \textbf{2} An individual compartment for an animal in a stable or cowshed, enclosed on three sides. {\fontspec{DejaVu Sans}◇} \textit{} \colorBulletS{SYN} pen, coop, sty, corral, enclosure, compartment, cubicle \textbf{3} A fixed seat in the choir or chancel of a church, enclosed at the back and sides and often canopied, typically reserved for a particular member of the clergy. {\fontspec{DejaVu Sans}◇} \textit{} \textbf{4} The seats on the ground floor in a theatre. {\fontspec{DejaVu Sans}◇} \textit{a stalls seat} \colorBulletS{SYN} orchestra, parterre \textbf{5} An instance of an engine, vehicle, aircraft, or boat stalling. {\fontspec{DejaVu Sans}◇} \textit{speed must be maintained to avoid a stall and loss of control} \\{\fontspec{DejaVu Sans}▪ }\textsf{\textit{verb}}\\ \textbf{1} (of a motor vehicle or its engine) stop running, typically because of an overload on the engine. {\fontspec{DejaVu Sans}◇} \textit{her car stalled at the crossroads} \textbf{2} Stop or cause to stop making progress. {\fontspec{DejaVu Sans}◇} \textit{his career had stalled, hers taken off} \colorBulletS{SYN} obstruct, impede, interfere with, hinder, hamper, block, interrupt, hold up, hold back, stand in the way of, frustrate, thwart, balk, inhibit, hamstring, sabotage, encumber, restrain, slow, slow down, retard, delay, stonewall, forestall, arrest, check, stop, halt, stay, derail, restrict, limit, curb, put a brake on, bridle, fetter, shackle \textbf{3} Put or keep (an animal) in a stall, especially in order to fatten it. {\fontspec{DejaVu Sans}◇} \textit{the horses were stalled at Upper Bolney Farm}}{}{}{ \colorBullet{ORIGIN} Old English steall ‘stable or cattle shed’, of Germanic origin; related to Dutch stal, also to stand. Early senses of the verb included ‘reside, dwell’ and ‘bring to a halt’.}%
\par%
\entry{standoff}{/ˈstandˌôf/}{বিরোধ নিষ্পত্তিতে}{ \textsf{\textit{noun}}\ \textbf{1} A stalemate or deadlock between two equally matched opponents in a dispute or conflict. {\fontspec{DejaVu Sans}◇} \textit{the 16{-}day{-}old standoff was no closer to being resolved} \colorBulletS{SYN} deadlock, stalemate, impasse, standstill, dead end, draw, tie, dead heat}{}{}{}%
\par%
\entry{stare}{/stɛː/}{অনিমেষনেত্রে}{\small{\textsf{\textit{noun, verb}}} \\{\fontspec{DejaVu Sans}▪ }\textsf{\textit{noun}}\\ \textbf{1} A long fixed or vacant look. {\fontspec{DejaVu Sans}◇} \textit{she gave him a cold stare} \\{\fontspec{DejaVu Sans}▪ }\textsf{\textit{verb}}\\ \textbf{1} Look fixedly or vacantly at someone or something with one's eyes wide open. {\fontspec{DejaVu Sans}◇} \textit{he stared at her in amazement} \colorBulletS{SYN} gaze, gape, goggle, gawk, glare, ogle, leer, peer, look fixedly, look vacantly}{}{}{ \colorBullet{ORIGIN} Old English starian, of Germanic origin, from a base meaning ‘be rigid’.}%
\par%
\entry{station}{/ˈsteɪʃ(ə)n/}{সংস্থিত}{\small{\textsf{\textit{noun, verb}}} \\{\fontspec{DejaVu Sans}▪ }\textsf{\textit{noun}}\\ \textbf{1} A place on a railway line where trains regularly stop so that passengers can get on or off. {\fontspec{DejaVu Sans}◇} \textit{we walked back to the station and caught the train back to Brussels} \colorBulletS{SYN} stopping place, stop, halt, station stop, stage \textbf{2} A place or building where a specified activity or service is based. {\fontspec{DejaVu Sans}◇} \textit{a research station in the rainforest} \colorBulletS{SYN} establishment, base, base camp, camp \textbf{3} A company involved in broadcasting of a specified kind. {\fontspec{DejaVu Sans}◇} \textit{a radio station} \colorBulletS{SYN} channel, broadcasting organization \textbf{4} The place where someone or something stands or is placed on military or other duty. {\fontspec{DejaVu Sans}◇} \textit{the lookout resumed his station in the bow} \colorBulletS{SYN} assigned position, post, area of duty, place, situation, location \textbf{5} A site at which a particular species, especially an interesting or rare one, grows or is found. {\fontspec{DejaVu Sans}◇} \textit{Thus, the southernmost stations for the plant in natural habitats are on Virginia's James and Chickahominy Rivers.} \textbf{6} short for Stations of the Cross {\fontspec{DejaVu Sans}◇} \textit{The stations seem to have originated in the pious practice of pilgrims to the Holy Land who visited the sites of the life, suffering, death and resurrection of Jesus.} \\{\fontspec{DejaVu Sans}▪ }\textsf{\textit{verb}}\\ \textbf{1} Put in or assign to a specified place for a particular purpose, especially a military one. {\fontspec{DejaVu Sans}◇} \textit{troops were stationed in the town} \colorBulletS{SYN} put on duty, post, position, place, set, locate, site}{}{}{ \colorBullet{ORIGIN} Middle English (as a noun): via Old French from Latin statio(n{-}), from stare ‘to stand’. Early use referred generally to ‘position’, especially ‘position in life, status’, and specifically, in ecclesiastical use, to ‘a holy place of pilgrimage (visited as one of a succession’). The verb dates from the late 16th century.}%
\par%
\entry{steep}{/stiːp/}{}{\small{\textsf{\textit{adjective, noun}}} \\{\fontspec{DejaVu Sans}▪ }\textsf{\textit{adjective}}\\ \textbf{1} (of a slope, flight of stairs, or angle) rising or falling sharply; almost perpendicular. {\fontspec{DejaVu Sans}◇} \textit{she pushed the bike up the steep hill} \colorBulletS{SYN} precipitous, sheer, abrupt, sharp, perpendicular, vertical, bluff, vertiginous, dizzy \textbf{2} (of a price or demand) not reasonable; excessive. {\fontspec{DejaVu Sans}◇} \textit{a steep membership fee} \colorBulletS{SYN} expensive, dear, costly, high, stiff \\{\fontspec{DejaVu Sans}▪ }\textsf{\textit{noun}}\\ \textbf{1} A steep mountain slope. {\fontspec{DejaVu Sans}◇} \textit{hair{-}raising steeps}}{}{}{ \colorBullet{ORIGIN} Old English stēap ‘extending to a great height’, of West Germanic origin; related to steeple and stoop.}%
\par%
\entry{steep}{/stiːp/}{}{ \textsf{\textit{verb}}\ \textbf{1} Soak (food or tea) in water or other liquid so as to extract its flavour or to soften it. {\fontspec{DejaVu Sans}◇} \textit{the chillies are steeped in olive oil} \colorBulletS{SYN} marinade, marinate, soak, souse, macerate \textbf{2} Surround or fill with a quality or influence. {\fontspec{DejaVu Sans}◇} \textit{a city steeped in history} \colorBulletS{SYN} imbue with, fill with, permeate with, pervade with, suffuse with, infuse with, perfuse with, impregnate with, soak in}{}{}{ \colorBullet{ORIGIN} Middle English of Germanic origin; related to stoup.}%
\par%
\entry{steer}{/stɪə/}{হাল ধরা}{\small{\textsf{\textit{noun, verb}}} \\{\fontspec{DejaVu Sans}▪ }\textsf{\textit{noun}}\\ \textbf{1} The type of steering of a vehicle. {\fontspec{DejaVu Sans}◇} \textit{some cars boast four{-}wheel steer} \textbf{2} A piece of advice or information concerning the development of a situation. {\fontspec{DejaVu Sans}◇} \textit{the need for the NHS to be given a clear steer as to its future direction} \\{\fontspec{DejaVu Sans}▪ }\textsf{\textit{verb}}\\ \textbf{1} Guide or control the movement of (a vehicle, vessel, or aircraft), for example by turning a wheel or operating a rudder. {\fontspec{DejaVu Sans}◇} \textit{he steered the boat slowly towards the busy quay} \colorBulletS{SYN} guide, direct, manoeuvre}{}{}{ \colorBullet{ORIGIN} Old English stīeran, of Germanic origin; related to Dutch sturen and German steuern.}%
\par%
\entry{steer}{/stɪə/}{হাল ধরা}{\small{\textsf{\textit{}}}}{}{}{ \colorBullet{ORIGIN} Old English stēor, of Germanic origin; related to Dutch stier and German Stier.}%
\par%
\entry{stereotype}{/ˈstɛrɪə(ʊ)tʌɪp/}{ছকের}{\small{\textsf{\textit{noun, verb}}} \\{\fontspec{DejaVu Sans}▪ }\textsf{\textit{noun}}\\ \textbf{1} A widely held but fixed and oversimplified image or idea of a particular type of person or thing. {\fontspec{DejaVu Sans}◇} \textit{the stereotype of the woman as the carer} \colorBulletS{SYN} conventional image, standard image, received idea, cliché, hackneyed idea, formula \textbf{2} A relief printing plate cast in a mould made from composed type or an original plate. {\fontspec{DejaVu Sans}◇} \textit{} \\{\fontspec{DejaVu Sans}▪ }\textsf{\textit{verb}}\\ \textbf{1} View or represent as a stereotype. {\fontspec{DejaVu Sans}◇} \textit{the city is too easily stereotyped as an industrial wasteland} \colorBulletS{SYN} typecast, pigeonhole, conventionalize, standardize, categorize, compartmentalize, label, tag}{}{}{ \colorBullet{ORIGIN} Late 18th century from French stéréotype (adjective).}%
\par%
\entry{sterilize}{/ˈstɛrɪlʌɪz/}{জীবাণুমুক্ত করা}{ \textsf{\textit{verb}}\ \textbf{1} Make (something) free from bacteria or other living microorganisms. {\fontspec{DejaVu Sans}◇} \textit{babies' feeding equipment can be cleaned and sterilized} \colorBulletS{SYN} disinfect, purify, fumigate, decontaminate, sanitize \textbf{2} Deprive (a person or animal) of the ability to produce offspring, typically by removing or blocking the sex organs. {\fontspec{DejaVu Sans}◇} \textit{she fell pregnant despite having been sterilized} \colorBulletS{SYN} vasectomize, hysterectomize}{}{}{}%
\par%
\entry{stern}{/stəːn/}{কঠোর}{ \textsf{\textit{adjective}}\ \textbf{1} (of a person or their manner) serious and unrelenting, especially in the assertion of authority and exercise of discipline. {\fontspec{DejaVu Sans}◇} \textit{a smile transformed his stern face} \colorBulletS{SYN} serious, unsmiling, frowning, poker{-}faced, severe, forbidding, grim, unfriendly, sombre, grave, sober, austere, dour, stony, flinty, steely, unrelenting, unyielding, unforgiving, unbending, unsympathetic, disapproving}{}{}{ \colorBullet{ORIGIN} Old English styrne, probably from the West Germanic base of the verb stare.}%
\par%
\entry{stern}{/stəːn/}{কঠোর}{ \textsf{\textit{noun}}\ \textbf{1} The rearmost part of a ship or boat. {\fontspec{DejaVu Sans}◇} \textit{he stood at the stern of the yacht} \colorBulletS{SYN} rear end, rear, back, tail, poop}{}{}{ \colorBullet{ORIGIN} Middle English probably from Old Norse stjórn ‘steering’, from stýra ‘to steer’.}%
\par%
\entry{stew}{/stjuː/}{ভাপে সিদ্ধ করা}{\small{\textsf{\textit{noun, verb}}} \\{\fontspec{DejaVu Sans}▪ }\textsf{\textit{noun}}\\ \textbf{1} A dish of meat and vegetables cooked slowly in liquid in a closed dish or pan. {\fontspec{DejaVu Sans}◇} \textit{lamb stew} \colorBulletS{SYN} casserole \textbf{2} A state of great anxiety or agitation. {\fontspec{DejaVu Sans}◇} \textit{she's in a right old stew} \colorBulletS{SYN} agitated, anxious, in a state of nerves, nervous, in a state of agitation, in a panic, worked up, keyed up, overwrought, wrought up, flustered, flurried, in a pother \textbf{3} A heated public room used for steam baths. {\fontspec{DejaVu Sans}◇} \textit{} \\{\fontspec{DejaVu Sans}▪ }\textsf{\textit{verb}}\\ \textbf{1} (with reference to meat, fruit, or other food) cook or be cooked slowly in liquid in a closed dish or pan. {\fontspec{DejaVu Sans}◇} \textit{beef stewed in wine} \colorBulletS{SYN} braise, casserole, fricassee, simmer, boil \textbf{2} Remain in a heated or stifling atmosphere. {\fontspec{DejaVu Sans}◇} \textit{sweaty clothes left to stew in a plastic bag} \colorBulletS{SYN} swelter, be very hot, perspire, sweat}{}{}{ \colorBullet{ORIGIN} Middle English (in the sense ‘cauldron’): from Old French estuve (related to estuver ‘heat in steam’), probably based on Greek tuphos ‘smoke, steam’. stew (sense 1 of the noun) (mid 18th century) is directly from the verb (dating from late Middle English).}%
\par%
\entry{stew}{/stjuː/}{ভাপে সিদ্ধ করা}{ \textsf{\textit{noun}}\ \textbf{1} A pond or large tank for keeping fish for eating. {\fontspec{DejaVu Sans}◇} \textit{}}{}{}{ \colorBullet{ORIGIN} Middle English from Old French estui, from estoier ‘confine’.}%
\par%
\entry{stew}{/stjuː/}{ভাপে সিদ্ধ করা}{ \textsf{\textit{noun}}\ \textbf{1} A flight attendant. {\fontspec{DejaVu Sans}◇} \textit{But I'd be in favor of keeping the present policy of no weapon, period if the stews had access to non{-}lethal weapons and were trained in their use.}}{}{}{ \colorBullet{ORIGIN} 1970s abbreviation of stewardess.}%
\par%
\entry{stiff}{/stɪf/}{শক্ত}{\small{\textsf{\textit{adjective, noun, verb}}} \\{\fontspec{DejaVu Sans}▪ }\textsf{\textit{adjective}}\\ \textbf{1} Not easily bent or changed in shape; rigid. {\fontspec{DejaVu Sans}◇} \textit{a stiff black collar} \colorBulletS{SYN} rigid, hard, firm, hardened, inelastic, non{-}flexible, inflexible, ungiving \textbf{2} Severe or strong. {\fontspec{DejaVu Sans}◇} \textit{they face stiff fines and a possible jail sentence} \colorBulletS{SYN} harsh, severe, hard, punitive, punishing, stringent, swingeing, crippling, rigorous, drastic, strong, heavy, draconian \textbf{3} Full of. {\fontspec{DejaVu Sans}◇} \textit{the place is stiff with alarm systems} \textbf{4} Having a specified unpleasant feeling to an extreme extent. {\fontspec{DejaVu Sans}◇} \textit{she was scared stiff} \\{\fontspec{DejaVu Sans}▪ }\textsf{\textit{noun}}\\ \textbf{1} A dead body. {\fontspec{DejaVu Sans}◇} \textit{} \colorBulletS{SYN} corpse, cadaver, dead body, body, remains, skeleton, relics \textbf{2} A boring, conventional person. {\fontspec{DejaVu Sans}◇} \textit{ordinary working stiffs in respectable offices} \textbf{3} A sports club's reserve team. {\fontspec{DejaVu Sans}◇} \textit{And unfortunately that's what we saw from Becks in the Portsmouth game so that explains why I dropped him to play with the stiffs when the first team was at Blackburn.} \\{\fontspec{DejaVu Sans}▪ }\textsf{\textit{verb}}\\ \textbf{1} Cheat (someone) out of something, especially money. {\fontspec{DejaVu Sans}◇} \textit{several workers were stiffed out of their pay} \colorBulletS{SYN} swindle, defraud, deceive, trick, dupe, hoodwink, double{-}cross, gull \textbf{2} Ignore (someone) deliberately; snub. {\fontspec{DejaVu Sans}◇} \textit{the stars are notorious for stiffing their hosts and sponsors at banquets} \colorBulletS{SYN} insult, slight, affront, humiliate, treat disrespectfully \textbf{3} Kill (someone) {\fontspec{DejaVu Sans}◇} \textit{I want to get those pigs who stiffed your doctor} \colorBulletS{SYN} murder, cause the death of, end the life of, take the life of, do away with, make away with, assassinate, do to death, eliminate, terminate, dispatch, finish off, put to death, execute}{}{}{ \colorBullet{ORIGIN} Old English stīf, of Germanic origin; related to Dutch stijf.}%
\par%
\entry{stigmatize}{/ˈstɪɡmətʌɪz/}{কলঙ্কপূর্ণ করা}{ \textsf{\textit{verb}}\ \textbf{1} Describe or regard as worthy of disgrace or great disapproval. {\fontspec{DejaVu Sans}◇} \textit{the institution was stigmatized as a last resort for the destitute} \colorBulletS{SYN} discredit, dishonour, defame, disparage, stigmatize, reproach, censure, blame \textbf{2} Mark with stigmata. {\fontspec{DejaVu Sans}◇} \textit{Francis, stigmatized in fashion as his Lord} \colorBulletS{SYN} condemn, denounce}{}{}{ \colorBullet{ORIGIN} Late 16th century (in the sense ‘mark with a brand’): from French stigmatiser or medieval Latin stigmatizare, from Greek stigmatizein, from stigma (see stigma).}%
\par%
\entry{sting}{/stɪŋ/}{দংশন}{\small{\textsf{\textit{noun, verb}}} \\{\fontspec{DejaVu Sans}▪ }\textsf{\textit{noun}}\\ \textbf{1} A small sharp{-}pointed organ at the end of the abdomen of bees, wasps, ants, and scorpions, capable of inflicting a painful or dangerous wound by injecting poison. {\fontspec{DejaVu Sans}◇} \textit{} \textbf{2} A carefully planned operation, typically one involving deception. {\fontspec{DejaVu Sans}◇} \textit{five blackmailers were jailed last week after they were snared in a police sting} \colorBulletS{SYN} swindle, fraud, piece of deception, trickery, cheat, bit of sharp practice \\{\fontspec{DejaVu Sans}▪ }\textsf{\textit{verb}}\\ \textbf{1} Wound or pierce with a sting. {\fontspec{DejaVu Sans}◇} \textit{he was stung by a jellyfish} \colorBulletS{SYN} prick, wound, injure, hurt \textbf{2} Swindle or exorbitantly overcharge (someone) {\fontspec{DejaVu Sans}◇} \textit{I had to buy some boxer shorts at the last minute and got stung for £42.50!} \colorBulletS{SYN} swindle, defraud, cheat, fleece, gull}{}{}{ \colorBullet{ORIGIN} Old English sting (noun), stingan (verb), of Germanic origin.}%
\par%
\entry{stink}{/stɪŋk/}{দুর্গন্ধ}{\small{\textsf{\textit{adjective, noun, verb}}} \\{\fontspec{DejaVu Sans}▪ }\textsf{\textit{adjective}}\\ \textbf{1} Having a strong unpleasant smell. {\fontspec{DejaVu Sans}◇} \textit{‘What you doing with that stink dog?’} \textbf{2} Contemptible; corrupt. {\fontspec{DejaVu Sans}◇} \textit{the whole episode is so stink that the principal asked for an immediate transfer of the teacher} \\{\fontspec{DejaVu Sans}▪ }\textsf{\textit{noun}}\\ \textbf{1} A strong unpleasant smell; a stench. {\fontspec{DejaVu Sans}◇} \textit{the stink of the place hit me as I went in} \colorBulletS{SYN} stench, reek, foul smell, bad smell, fetidness, effluvium, malodour, malodorousness, miasma \textbf{2} A row or fuss. {\fontspec{DejaVu Sans}◇} \textit{a silly move now would kick up a stink we couldn't handle} \colorBulletS{SYN} fuss, commotion, rumpus, ruckus, trouble, outcry, uproar, brouhaha, furore \\{\fontspec{DejaVu Sans}▪ }\textsf{\textit{verb}}\\ \textbf{1} Have a strong unpleasant smell. {\fontspec{DejaVu Sans}◇} \textit{the place stank like a sewer} \colorBulletS{SYN} reek, smell bad, smell disgusting, smell foul, smell to high heaven, stink to high heaven, give off a bad smell \textbf{2} Be very unpleasant, contemptible, or scandalous. {\fontspec{DejaVu Sans}◇} \textit{he thinks the values of our society stink} \colorBulletS{SYN} be very unpleasant, be abhorrent, be despicable, be contemptible, be disgusting, be vile, be foul}{}{}{ \colorBullet{ORIGIN} Old English stincan, of West Germanic origin; related to Dutch and German stinken, also to stench.}%
\par%
\entry{stipulate}{/ˈstɪpjʊleɪt/}{উপপত্রিক}{ \textsf{\textit{verb}}\ \textbf{1} Demand or specify (a requirement), typically as part of an agreement. {\fontspec{DejaVu Sans}◇} \textit{he stipulated certain conditions before their marriage} \colorBulletS{SYN} specify, set down, set out, lay down, set forth, state clearly}{}{}{ \colorBullet{ORIGIN} Early 17th century from Latin stipulat{-} ‘demanded as a formal promise’, from the verb stipulari.}%
\par%
\entry{stipulate}{/ˈstɪpjʊlət/}{উপপত্রিক}{ \textsf{\textit{adjective}}\ \textbf{1} (of a leaf or plant) having stipules. {\fontspec{DejaVu Sans}◇} \textit{Both have woody trunks and woody roots as well as stipulate leaf bases.}}{}{}{ \colorBullet{ORIGIN} Late 18th century from Latin stipula (see stipule) + {-}ate.}%
\par%
\entry{stirring}{/ˈstəːrɪŋ/}{মন্থন}{\small{\textsf{\textit{adjective, noun}}} \\{\fontspec{DejaVu Sans}▪ }\textsf{\textit{adjective}}\\ \textbf{1} Causing excitement or strong emotion; rousing. {\fontspec{DejaVu Sans}◇} \textit{stirring songs} \colorBulletS{SYN} exciting, thrilling, action{-}packed, gripping, riveting, dramatic, rousing, spirited, stimulating, moving, inspiring, inspirational, electrifying, passionate, impassioned, emotive, emotional, emotion{-}charged, heady, soul{-}stirring \textbf{2} Moving briskly; active. {\fontspec{DejaVu Sans}◇} \textit{a stirring and thriving politician} \\{\fontspec{DejaVu Sans}▪ }\textsf{\textit{noun}}\\ \textbf{1} An initial sign of activity, movement, or emotion. {\fontspec{DejaVu Sans}◇} \textit{the first stirrings of anger}}{}{}{}%
\par%
\entry{stitch}{/stɪtʃ/}{সেলাই}{\small{\textsf{\textit{noun, verb}}} \\{\fontspec{DejaVu Sans}▪ }\textsf{\textit{noun}}\\ \textbf{1} A loop of thread or yarn resulting from a single pass or movement of the needle in sewing, knitting, or crocheting. {\fontspec{DejaVu Sans}◇} \textit{} \textbf{2} A sudden sharp pain in the side of the body, caused by strenuous exercise. {\fontspec{DejaVu Sans}◇} \textit{he was panting and had a stitch} \colorBulletS{SYN} sharp pain, stabbing pain, shooting pain, stab of pain, pang, twinge, spasm \\{\fontspec{DejaVu Sans}▪ }\textsf{\textit{verb}}\\ \textbf{1} Make, mend, or join (something) with stitches. {\fontspec{DejaVu Sans}◇} \textit{stitch a plain seam with right sides together} \colorBulletS{SYN} sew, baste, tack, seam, hem \textbf{2} Manipulate a situation so that someone is placed at a disadvantage or wrongly blamed for something. {\fontspec{DejaVu Sans}◇} \textit{he was stitched up by outsiders and ousted as chairman} \colorBulletS{SYN} falsely incriminate, get someone into trouble}{}{}{ \colorBullet{ORIGIN} Old English stice ‘a puncture, stabbing pain’, of Germanic origin; related to German Stich ‘a sting, prick’, also to stick. The sense ‘loop’ (in sewing etc.) arose in Middle English.}%
\par%
\entry{stockpile}{/ˈstɒkpʌɪl/}{মজুদ}{\small{\textsf{\textit{noun, verb}}} \\{\fontspec{DejaVu Sans}▪ }\textsf{\textit{noun}}\\ \textbf{1} A large accumulated stock of goods or materials, especially one held in reserve for use at a time of shortage or other emergency. {\fontspec{DejaVu Sans}◇} \textit{a stockpile of sandbags was being prepared} \colorBulletS{SYN} stock, store, supply, accumulation, collection, reserve, hoard, cache \\{\fontspec{DejaVu Sans}▪ }\textsf{\textit{verb}}\\ \textbf{1} Accumulate a large stock of (goods or materials) {\fontspec{DejaVu Sans}◇} \textit{he claimed that the weapons were being stockpiled} \colorBulletS{SYN} store up, amass, accumulate, hoard, cache, collect, gather, pile up, heap up, lay in, put away, put aside, set aside, put down, put by, put away for a rainy day, stow away, keep, keep in reserve, save}{}{}{}%
\par%
\entry{stone aggregates}{}{}{\small{\textsf{\textit{}}}}{}{Bhutan exports significant quantity of stone aggregates to bangladesh using the time{-}consuming land route.}{}%
\par%
\entry{stout}{/staʊt/}{স্থুলকায়}{\small{\textsf{\textit{adjective, noun}}} \\{\fontspec{DejaVu Sans}▪ }\textsf{\textit{adjective}}\\ \textbf{1} (of a person) rather fat or of heavy build. {\fontspec{DejaVu Sans}◇} \textit{stout middle{-}aged men} \colorBulletS{SYN} fat, fattish, plump, portly, rotund, roly{-}poly, pot{-}bellied, round, dumpy, chunky, broad in the beam, overweight, fleshy, paunchy, corpulent \textbf{2} (of an object) strong and thick. {\fontspec{DejaVu Sans}◇} \textit{Billy had armed himself with a stout stick} \colorBulletS{SYN} strong, sturdy, heavy, solid, substantial, robust, tough, strongly made, durable, hard{-}wearing \textbf{3} Having or showing courage and determination. {\fontspec{DejaVu Sans}◇} \textit{he put up a stout defence in court} \colorBulletS{SYN} determined, full of determination, vigorous, forceful, spirited, stout{-}hearted \\{\fontspec{DejaVu Sans}▪ }\textsf{\textit{noun}}\\ \textbf{1} A kind of strong, dark beer brewed with roasted malt or barley. {\fontspec{DejaVu Sans}◇} \textit{there is a tradition in England of drinking stout while eating oysters}}{}{}{ \colorBullet{ORIGIN} Middle English from Anglo{-}Norman French and Old French dialect, of West Germanic origin; perhaps related to stilt. The noun (late 17th century) originally denoted any strong beer and is probably elliptical for stout ale.}%
\par%
\entry{strait}{/streɪt/}{প্রণালী}{\small{\textsf{\textit{adjective, noun}}} \\{\fontspec{DejaVu Sans}▪ }\textsf{\textit{adjective}}\\ \textbf{1} (of a place) of limited spatial capacity; narrow or cramped. {\fontspec{DejaVu Sans}◇} \textit{the road was so strait that a handful of men might have defended it} \colorBulletS{SYN} cramped, constricted, restricted, limited, confining, small, narrow, compact, tight, pinched, squeezed, poky, uncomfortable, inadequate, meagre \\{\fontspec{DejaVu Sans}▪ }\textsf{\textit{noun}}\\ \textbf{1}  {\fontspec{DejaVu Sans}◇} \textit{the Straits of Gibraltar} \colorBulletS{SYN} channel, sound, narrows, inlet, stretch of water, arm of the sea, sea passage, neck \textbf{2} Used in reference to a situation characterized by a specified degree of trouble or difficulty. {\fontspec{DejaVu Sans}◇} \textit{the economy is in dire straits} \colorBulletS{SYN} a bad situation, a difficult situation, a sorry condition, difficulty, trouble, crisis, a mess, a predicament, a plight, a tight corner}{}{}{ \colorBullet{ORIGIN} Middle English shortening of Old French estreit ‘tight, narrow’, from Latin strictus ‘drawn tight’ (see strict).}%
\par%
\entry{strand}{/strand/}{তীরভূমি}{\small{\textsf{\textit{noun, verb}}} \\{\fontspec{DejaVu Sans}▪ }\textsf{\textit{noun}}\\ \textbf{1} The shore of a sea, lake, or large river. {\fontspec{DejaVu Sans}◇} \textit{a heron glided to rest on a pebbly strand} \colorBulletS{SYN} seashore, shore, beach, sands, foreshore, shoreline \\{\fontspec{DejaVu Sans}▪ }\textsf{\textit{verb}}\\ \textbf{1} Drive or leave (a boat, sailor, or sea creature) aground on a shore. {\fontspec{DejaVu Sans}◇} \textit{the ships were stranded in shallow water}}{}{}{ \colorBullet{ORIGIN} Old English (as a noun), of unknown origin. The verb dates from the early 17th century.}%
\par%
\entry{strand}{/strand/}{তীরভূমি}{ \textsf{\textit{noun}}\ \textbf{1} A single thin length of something such as thread, fibre, or wire, especially as twisted together with others. {\fontspec{DejaVu Sans}◇} \textit{strands of coloured wool} \colorBulletS{SYN} thread, filament, fibre}{}{}{ \colorBullet{ORIGIN} Late 15th century of unknown origin.}%
\par%
\entry{streak}{/striːk/}{কষ}{\small{\textsf{\textit{noun, verb}}} \\{\fontspec{DejaVu Sans}▪ }\textsf{\textit{noun}}\\ \textbf{1} A long, thin line or mark of a different substance or colour from its surroundings. {\fontspec{DejaVu Sans}◇} \textit{a streak of oil} \colorBulletS{SYN} band, line, strip, stripe, vein, slash, bar \textbf{2} An element of a specified kind in someone's character. {\fontspec{DejaVu Sans}◇} \textit{there's a streak of insanity in the family} \colorBulletS{SYN} element, vein, trace, touch, dash, strain \textbf{3} An act of running naked in a public place so as to shock or amuse others. {\fontspec{DejaVu Sans}◇} \textit{a streak for charity} \\{\fontspec{DejaVu Sans}▪ }\textsf{\textit{verb}}\\ \textbf{1} Cover (a surface) with streaks. {\fontspec{DejaVu Sans}◇} \textit{tears streaking her face, Cynthia looked up} \colorBulletS{SYN} stripe, band, bar, fleck \textbf{2} Move very fast in a specified direction. {\fontspec{DejaVu Sans}◇} \textit{the cat streaked across the street} \colorBulletS{SYN} race, dash, rush, run, sprint, bolt, dart, gallop, career, charge, shoot, hurtle, hare, bound, fly, speed, zoom, go hell for leather, plunge, dive, whisk, scurry, scuttle, scamper, scramble \textbf{3} Run naked in a public place so as to shock or amuse others. {\fontspec{DejaVu Sans}◇} \textit{the singer admitted to streaking in his home town in the seventies}}{}{}{ \colorBullet{ORIGIN} Old English strica, of Germanic origin; related to Dutch streek and German Strich, also to strike. The sense ‘run naked’ was originally US slang.}%
\par%
\entry{strenuous}{/ˈstrɛnjʊəs/}{শ্রমসাধ্য}{ \textsf{\textit{adjective}}\ \textbf{1} Requiring or using great effort or exertion. {\fontspec{DejaVu Sans}◇} \textit{the government made strenuous efforts to upgrade the quality of the teaching profession} \colorBulletS{SYN} arduous, difficult, hard, tough, taxing, demanding, exacting, uphill, stiff, formidable, heavy, exhausting, tiring, fatiguing, gruelling, back{-}breaking, murderous, punishing}{}{}{ \colorBullet{ORIGIN} Early 17th century from Latin strenuus ‘brisk’ + {-}ous.}%
\par%
\entry{stretch}{/strɛtʃ/}{প্রসারণ}{\small{\textsf{\textit{noun, verb}}} \\{\fontspec{DejaVu Sans}▪ }\textsf{\textit{noun}}\\ \textbf{1} An act of stretching one's limbs or body. {\fontspec{DejaVu Sans}◇} \textit{I got up and had a stretch} \colorBulletS{SYN} reach out, hold out, put out, extend, outstretch, thrust out, stick out \textbf{2} A continuous area or expanse of land or water. {\fontspec{DejaVu Sans}◇} \textit{a treacherous stretch of road} \colorBulletS{SYN} expanse, area, tract, belt, sweep, extent, spread, reach \textbf{3} A stretch limo. {\fontspec{DejaVu Sans}◇} \textit{a chauffeur{-}driven stretch} \\{\fontspec{DejaVu Sans}▪ }\textsf{\textit{verb}}\\ \textbf{1} (of something soft or elastic) be made or be capable of being made longer or wider without tearing or breaking. {\fontspec{DejaVu Sans}◇} \textit{my jumper stretched in the wash} \colorBulletS{SYN} be elastic, be stretchy, be stretchable, be tensile \textbf{2} Straighten or extend one's body or a part of one's body to its full length, typically so as to tighten one's muscles or in order to reach something. {\fontspec{DejaVu Sans}◇} \textit{the cat yawned and stretched} \colorBulletS{SYN} extend, straighten, straighten out, unbend \textbf{3} Extend or spread over an area or period of time. {\fontspec{DejaVu Sans}◇} \textit{the beach stretches for over four miles} \colorBulletS{SYN} extend, spread, continue, range, unfold, unroll, be unbroken \textbf{4} Make great demands on the capacity or resources of. {\fontspec{DejaVu Sans}◇} \textit{the cost of the court case has stretched their finances to the limit} \colorBulletS{SYN} put a strain on, put great demands on, overtax, overextend, be too much for}{}{}{ \colorBullet{ORIGIN} Old English streccan, of West Germanic origin; related to Dutch strekken and German strecken. The noun dates from the late 16th century.}%
\par%
\entry{stricken}{/ˈstrɪk(ə)n/}{অভিভূত}{\small{\textsf{\textit{adjective, verb}}} \\{\fontspec{DejaVu Sans}▪ }\textsf{\textit{adjective}}\\ \textbf{1} Seriously affected by an undesirable condition or unpleasant feeling. {\fontspec{DejaVu Sans}◇} \textit{the pilot landed the stricken aircraft} \colorBulletS{SYN} troubled, affected, deeply affected, afflicted, struck, hit, injured, wounded \\{\fontspec{DejaVu Sans}▪ }\textsf{\textit{verb}}\\ \textbf{1} past participle of strike (sense 2 of the verb, {\fontspec{DejaVu Sans}◇} \textit{}}{}{}{ \colorBullet{ORIGIN} Old and feeble.}%
\par%
\entry{strict}{/strɪkt/}{কঠিন}{ \textsf{\textit{adjective}}\ \textbf{1} Demanding that rules concerning behaviour are obeyed and observed. {\fontspec{DejaVu Sans}◇} \textit{my father was very strict} \colorBulletS{SYN} stern, severe, harsh, uncompromising, authoritarian, firm, austere, illiberal, inflexible, unyielding, unbending, no{-}nonsense \textbf{2} (of a person) following rules or beliefs exactly. {\fontspec{DejaVu Sans}◇} \textit{a strict vegetarian} \textbf{3} Exact in correspondence or adherence to something; not allowing or admitting deviation or relaxation. {\fontspec{DejaVu Sans}◇} \textit{a strict interpretation of the law} \colorBulletS{SYN} precise, exact, literal, close, faithful, true, accurate, unerring, scrupulous, careful, meticulous, rigorous, stringent}{}{}{ \colorBullet{ORIGIN} Late Middle English (in the sense ‘restricted in space or extent’): from Latin strictus, past participle of stringere ‘tighten, draw tight’.}%
\par%
\entry{stride}{/strʌɪd/}{দীর্ঘ}{\small{\textsf{\textit{noun, verb}}} \\{\fontspec{DejaVu Sans}▪ }\textsf{\textit{noun}}\\ \textbf{1} A long, decisive step. {\fontspec{DejaVu Sans}◇} \textit{he crossed the room in a couple of strides} \colorBulletS{SYN} step, long step, large step, pace, footstep \textbf{2} A step or stage in progress towards an aim. {\fontspec{DejaVu Sans}◇} \textit{great strides have been made towards equality} \colorBulletS{SYN} make progress, make headway, gain ground, progress, advance, proceed, move, get on, get ahead, come on, come along, shape up, take shape, move forward in leaps and bounds \textbf{3} Trousers. {\fontspec{DejaVu Sans}◇} \textit{} \textbf{4} Denoting or relating to a rhythmic style of jazz piano playing in which the left hand alternately plays single bass notes on the downbeat and chords an octave higher on the upbeat. {\fontspec{DejaVu Sans}◇} \textit{he's a noted stride pianist} \\{\fontspec{DejaVu Sans}▪ }\textsf{\textit{verb}}\\ \textbf{1} Walk with long, decisive steps in a specified direction. {\fontspec{DejaVu Sans}◇} \textit{he strode across the road} \colorBulletS{SYN} march, stalk, pace, tread, step, walk \textbf{2} Cross (an obstacle) with one long step. {\fontspec{DejaVu Sans}◇} \textit{}}{}{}{ \colorBullet{ORIGIN} Old English stride (noun) ‘single long step’, strīdan (verb) ‘stand or walk with the legs wide apart’, probably from a Germanic base meaning ‘strive, quarrel’; related to Dutch strijden ‘fight’ and German streiten ‘quarrel’.}%
\par%
\entry{strike}{/strʌɪk/}{ধর্মঘট}{\small{\textsf{\textit{noun, verb}}} \\{\fontspec{DejaVu Sans}▪ }\textsf{\textit{noun}}\\ \textbf{1} A refusal to work organized by a body of employees as a form of protest, typically in an attempt to gain a concession or concessions from their employer. {\fontspec{DejaVu Sans}◇} \textit{dockers voted for an all{-}out strike} \colorBulletS{SYN} industrial action, walkout \textbf{2} A sudden attack, typically a military one. {\fontspec{DejaVu Sans}◇} \textit{the threat of nuclear strikes} \colorBulletS{SYN} attack, air strike, air attack, assault, bombing, blitz \textbf{3} A discovery of gold, minerals, or oil by drilling or mining. {\fontspec{DejaVu Sans}◇} \textit{the Lena goldfields strike of 1912} \colorBulletS{SYN} find, discovery, unearthing, uncovering \textbf{4} A batter's unsuccessful attempt to hit a pitched ball. {\fontspec{DejaVu Sans}◇} \textit{} \textbf{5} The horizontal or compass direction of a stratum, fault, or other geological feature. {\fontspec{DejaVu Sans}◇} \textit{the mine workings follow the strike of the Bonsor Vein} \textbf{6} short for fly strike {\fontspec{DejaVu Sans}◇} \textit{} \\{\fontspec{DejaVu Sans}▪ }\textsf{\textit{verb}}\\ \textbf{1} Hit forcibly and deliberately with one's hand or a weapon or other implement. {\fontspec{DejaVu Sans}◇} \textit{he raised his hand, as if to strike me} \colorBulletS{SYN} bang, beat, hit, pound \textbf{2} (of a disaster, disease, or other unwelcome phenomenon) occur suddenly and have harmful or damaging effects on. {\fontspec{DejaVu Sans}◇} \textit{a major earthquake struck the island} \colorBulletS{SYN} affect, afflict, attack, hit, come upon, smite \textbf{3} (of a thought or idea) come into the mind of (someone) suddenly or unexpectedly. {\fontspec{DejaVu Sans}◇} \textit{a disturbing thought struck Melissa} \colorBulletS{SYN} occur to, come to, dawn on one, hit \textbf{4} (of a clock) indicate the time by sounding a chime or stroke. {\fontspec{DejaVu Sans}◇} \textit{the church clock struck twelve} \textbf{5} Ignite (a match) by rubbing it briskly against an abrasive surface. {\fontspec{DejaVu Sans}◇} \textit{the match went out and he struck another} \colorBulletS{SYN} ignite, light \textbf{6} (of employees) refuse to work as a form of organized protest, typically in an attempt to obtain a particular concession or concessions from their employer. {\fontspec{DejaVu Sans}◇} \textit{workers may strike over threatened job losses} \colorBulletS{SYN} take industrial action, go on strike, down tools, walk out, work to rule \textbf{7} Cancel, remove, or cross out with or as if with a pen. {\fontspec{DejaVu Sans}◇} \textit{I will strike his name from the list} \colorBulletS{SYN} delete, strike out, strike through, ink out, score out, scratch out, block out, blank out, edit out, blue{-}pencil, cancel, eliminate, obliterate \textbf{8} Make (a coin or medal) by stamping metal. {\fontspec{DejaVu Sans}◇} \textit{they struck similar medals on behalf of the Normandy veterans} \colorBulletS{SYN} mint, stamp, stamp out, strike, cast, punch, die, mould, forge, make, manufacture, produce \textbf{9} Reach, achieve, or agree to (something involving agreement, balance, or compromise) {\fontspec{DejaVu Sans}◇} \textit{the team has struck a deal with a sports marketing agency} \colorBulletS{SYN} achieve, reach, arrive at, find, attain, effect, establish \textbf{10} Discover (gold, minerals, or oil) by drilling or mining. {\fontspec{DejaVu Sans}◇} \textit{if they do strike oil, there will be another test well in a year's time} \colorBulletS{SYN} discover, find, come upon, light on, chance on, happen on, stumble across, stumble on, unearth, uncover, turn up \textbf{11} Move or proceed vigorously or purposefully. {\fontspec{DejaVu Sans}◇} \textit{she struck out into the lake with a practised crawl} \colorBulletS{SYN} go, make one's way, set out, head, direct one's footsteps, move towards \textbf{12} Take down (a tent or the tents of an encampment) {\fontspec{DejaVu Sans}◇} \textit{it took ages to strike camp} \colorBulletS{SYN} take down, pull down, bring down \textbf{13} Insert (a cutting of a plant) in soil to take root. {\fontspec{DejaVu Sans}◇} \textit{best results are obtained from striking them in a propagator} \textbf{14} Secure a hook in the mouth of a fish by jerking or tightening the line after it has taken the bait or fly. {\fontspec{DejaVu Sans}◇} \textit{}}{}{}{ \colorBullet{ORIGIN} Old English strīcan ‘go, flow’ and ‘rub lightly’, of West Germanic origin; related to German streichen ‘to stroke’, also to stroke. The sense ‘deliver a blow’ dates from Middle English.}%
\par%
\entry{string}{/strɪŋ/}{দড়ি}{\small{\textsf{\textit{noun, verb}}} \\{\fontspec{DejaVu Sans}▪ }\textsf{\textit{noun}}\\ \textbf{1} Material consisting of threads of cotton, hemp, or other material twisted together to form a thin length. {\fontspec{DejaVu Sans}◇} \textit{unwieldy packs tied up with string} \colorBulletS{SYN} twine, cord, yarn, thread, strand, fibre \textbf{2} A set of things tied or threaded together on a thin cord. {\fontspec{DejaVu Sans}◇} \textit{she wore a string of agates round her throat} \colorBulletS{SYN} strand, rope, necklace, rosary, chaplet \textbf{3} A tough piece of fibre in vegetables, meat, or other food, such as a tough elongated piece connecting the two halves of a bean pod. {\fontspec{DejaVu Sans}◇} \textit{} \textbf{4} A G{-}string or thong. {\fontspec{DejaVu Sans}◇} \textit{} \textbf{5} short for stringboard {\fontspec{DejaVu Sans}◇} \textit{Each of them is made of beautifully laid rough solid buff Cambridge{-}like brick with very precise precast concrete lintels and strings.} \textbf{6} A hypothetical one{-}dimensional subatomic particle having the dynamical properties of a flexible loop. {\fontspec{DejaVu Sans}◇} \textit{} \\{\fontspec{DejaVu Sans}▪ }\textsf{\textit{verb}}\\ \textbf{1} Hang (something) so that it stretches in a long line. {\fontspec{DejaVu Sans}◇} \textit{lights were strung across the promenade} \colorBulletS{SYN} hang, suspend, sling, stretch \textbf{2} Fit a string or strings to (a musical instrument, a racket, or a bow) {\fontspec{DejaVu Sans}◇} \textit{the harp had been newly strung} \textbf{3} Remove the strings from (a bean). {\fontspec{DejaVu Sans}◇} \textit{String the beans and break into lengths as for cooking.} \textbf{4} Hoax or trick (someone) {\fontspec{DejaVu Sans}◇} \textit{I'm not stringing you—I'll eat my shirt if it's not true} \textbf{5} Work as a stringer in journalism. {\fontspec{DejaVu Sans}◇} \textit{he strings for almost every French radio service} \textbf{6} Determine the order of play by striking the cue ball from baulk to rebound off the top cushion, first stroke going to the player whose ball comes to rest nearer the bottom cushion. {\fontspec{DejaVu Sans}◇} \textit{To begin a game of English billiards, both players "string".}}{}{}{ \colorBullet{ORIGIN} Old English streng (noun), of Germanic origin; related to German Strang, also to strong. The verb (dating from late Middle English) is first recorded in the senses ‘arrange in a row’ and ‘fit with a string’.}%
\par%
\entry{stringent}{/ˈstrɪn(d)ʒ(ə)nt/}{কঠোর}{ \textsf{\textit{adjective}}\ \textbf{1} (of regulations, requirements, or conditions) strict, precise, and exacting. {\fontspec{DejaVu Sans}◇} \textit{stringent guidelines on air pollution} \colorBulletS{SYN} strict, firm, rigid, rigorous, severe, harsh, tough, tight, exacting, demanding, inflexible, stiff, hard and fast, uncompromising, draconian, extreme}{}{}{ \colorBullet{ORIGIN} Mid 17th century (in the sense ‘compelling, convincing’): from Latin stringent{-} ‘drawing tight’, from the verb stringere.}%
\par%
\entry{strip}{/strɪp/}{ফালা}{\small{\textsf{\textit{noun, verb}}} \\{\fontspec{DejaVu Sans}▪ }\textsf{\textit{noun}}\\ \textbf{1} An act of undressing, especially in a striptease. {\fontspec{DejaVu Sans}◇} \textit{she got drunk and did a strip on top of the piano} \textbf{2} The identifying outfit worn by the members of a sports team while playing. {\fontspec{DejaVu Sans}◇} \textit{the team's away strip is a garish mix of red, white, and blue} \colorBulletS{SYN} outfit, clothes, clothing, garments, costume, suit, dress, garb \\{\fontspec{DejaVu Sans}▪ }\textsf{\textit{verb}}\\ \textbf{1} Remove all coverings from. {\fontspec{DejaVu Sans}◇} \textit{they stripped the bed} \textbf{2} Leave bare of accessories or fittings. {\fontspec{DejaVu Sans}◇} \textit{thieves stripped the room of luggage} \colorBulletS{SYN} empty, clear, clean out, plunder, rob, burgle, loot, rifle, pillage, ransack, gut, lay bare, devastate, sack, ravage, raid \textbf{3} Deprive someone of (rank, power, or property) {\fontspec{DejaVu Sans}◇} \textit{the lieutenant was stripped of his rank} \colorBulletS{SYN} take away from, dispossess, deprive, confiscate, divest, relieve, deny, rob \textbf{4} Sell off (the assets of a company) for profit. {\fontspec{DejaVu Sans}◇} \textit{} \textbf{5} Tear the thread or teeth from (a screw, gearwheel, etc.). {\fontspec{DejaVu Sans}◇} \textit{} \textbf{6} (of a bullet) be fired from a rifled gun without spin owing to a loss of surface. {\fontspec{DejaVu Sans}◇} \textit{}}{}{}{ \colorBullet{ORIGIN} Middle English (as a verb): of Germanic origin; related to Dutch stropen. strip (sense 2 of the noun) arose in the late 20th century, possibly from the notion of clothing to which a player ‘strips’ down.}%
\par%
\entry{strip}{/strɪp/}{ফালা}{ \textsf{\textit{noun}}\ \textbf{1} A long, narrow piece of cloth, paper, plastic, or some other material. {\fontspec{DejaVu Sans}◇} \textit{a strip of linen} \colorBulletS{SYN} narrow piece, piece, bit, band, belt, ribbon, sash, stripe, bar, swathe, slip, fillet, shred \textbf{2} A comic strip. {\fontspec{DejaVu Sans}◇} \textit{a strip cartoon} \textbf{3} A programme broadcast regularly at the same time. {\fontspec{DejaVu Sans}◇} \textit{he hosts a weekly two{-}hour advice strip}}{}{}{ \colorBullet{ORIGIN} Late Middle English from or related to Middle Low German strippe ‘strap, thong’, probably also to stripe.}%
\par%
\entry{struck}{/strʌk/}{তাড়িত}{\small{\textsf{\textit{}}}}{}{}{}%
\par%
\entry{strumpet}{/ˈstrʌmpɪt/}{বারাঙ্গনা}{ \textsf{\textit{noun}}\ \textbf{1} A female prostitute. {\fontspec{DejaVu Sans}◇} \textit{} \colorBulletS{SYN} sex worker, call girl}{}{}{ \colorBullet{ORIGIN} Middle English of unknown origin.}%
\par%
\entry{strung}{/strʌŋ/}{অনুবিদ্ধ}{\small{\textsf{\textit{}}}}{}{}{}%
\par%
\entry{stuck}{/stʌk/}{আটকে পড়া}{\small{\textsf{\textit{}}}}{}{}{}%
\par%
\entry{studious}{/ˈstjuːdɪəs/}{অধ্যয়নশীল}{ \textsf{\textit{adjective}}\ \textbf{1} Spending a lot of time studying or reading. {\fontspec{DejaVu Sans}◇} \textit{he was quiet and studious} \colorBulletS{SYN} scholarly, academic, bookish, book{-}loving, intellectual, erudite, learned, donnish, serious, earnest, thoughtful, cerebral \textbf{2} Done deliberately or with a purpose in mind. {\fontspec{DejaVu Sans}◇} \textit{his studious absence from public view} \colorBulletS{SYN} deliberate, wilful, conscious, calculated, intentional, volitional, designed, mannered, measured, studied, knowing, purposeful, contrived, artificial}{}{}{ \colorBullet{ORIGIN} Middle English from Latin studiosus, from studium ‘painstaking application’.}%
\par%
\entry{stumble}{/ˈstʌmb(ə)l/}{পদস্খলন}{\small{\textsf{\textit{noun, verb}}} \\{\fontspec{DejaVu Sans}▪ }\textsf{\textit{noun}}\\ \textbf{1} An act of stumbling. {\fontspec{DejaVu Sans}◇} \textit{he broke a bone in his foot in a stumble down an Alpine pass} \colorBulletS{SYN} fall, trip, spill \\{\fontspec{DejaVu Sans}▪ }\textsf{\textit{verb}}\\ \textbf{1} Trip or momentarily lose one's balance; almost fall. {\fontspec{DejaVu Sans}◇} \textit{her foot caught in the rug and she stumbled} \colorBulletS{SYN} trip, trip over, trip up, lose one's balance, lose one's footing, miss one's footing, founder, slip, pitch}{}{}{ \colorBullet{ORIGIN} Middle English (as a verb): from Old Norse, from the Germanic base of stammer.}%
\par%
\entry{stun}{/stʌn/}{অচেতন করা}{ \textsf{\textit{verb}}\ \textbf{1} Knock unconscious or into a dazed or semi{-}conscious state. {\fontspec{DejaVu Sans}◇} \textit{the man was stunned by a blow to the head} \colorBulletS{SYN} daze, stupefy, knock senseless, knock unconscious, knock out, lay out \textbf{2} Astonish or shock (someone) so that they are temporarily unable to react. {\fontspec{DejaVu Sans}◇} \textit{the community was stunned by the tragedy} \colorBulletS{SYN} astound, amaze, astonish, startle, take someone's breath away, dumbfound, stupefy, overwhelm, stagger, shock, confound, take aback, shake up}{}{}{ \colorBullet{ORIGIN} Middle English shortening of Old French estoner ‘astonish’.}%
\par%
\entry{stutter}{/ˈstʌtə/}{তোতলান}{\small{\textsf{\textit{noun, verb}}} \\{\fontspec{DejaVu Sans}▪ }\textsf{\textit{noun}}\\ \textbf{1} A tendency to stutter while speaking. {\fontspec{DejaVu Sans}◇} \textit{‘She's p{-}perfectly j{-}justified,’ he said with his intermittent stutter} \colorBulletS{SYN} stammer, speech impediment, speech defect \\{\fontspec{DejaVu Sans}▪ }\textsf{\textit{verb}}\\ \textbf{1} Talk with continued involuntary repetition of sounds, especially initial consonants. {\fontspec{DejaVu Sans}◇} \textit{the child was stuttering in fright} \colorBulletS{SYN} stammer, stumble, speak haltingly, falter, speak falteringly, flounder, hesitate, pause, halt}{}{}{ \colorBullet{ORIGIN} Late 16th century (as a verb): frequentative of dialect stut, of Germanic origin; related to German stossen ‘strike against’.}%
\par%
\entry{stymie}{/ˈstʌɪmi/}{কোণঠাসা করা}{ \textsf{\textit{verb}}\ \textbf{1} Prevent or hinder the progress of. {\fontspec{DejaVu Sans}◇} \textit{the changes must not be allowed to stymie new medical treatments} \colorBulletS{SYN} impede, interfere with, hamper, hinder, obstruct, inhibit, frustrate, thwart, foil, spoil, stall, shackle, fetter, stop, check, block, cripple, handicap, scotch}{}{}{ \colorBullet{ORIGIN} Mid 19th century (originally a golfing term, denoting a situation on the green where a ball obstructs the shot of another player): of unknown origin.}%
\par%
\entry{sub{-}par}{}{Below average. Deriving from the term in golf "par" meaning average, and sub meaning below.}{\small{\textsf{\textit{}}}}{}{1. Bangladesh lost by two wickets to new zealand at the oval on wednesday after being all out for a sub{-}par 244. 2. According to stand{-}in captain mahmudullah riyad, bangladesh's sub{-}par performances in the last six months is not about technique or lack of execution in skill.}{}%
\par%
\entry{sublime}{/səˈblʌɪm/}{মহিমান্বিত}{\small{\textsf{\textit{adjective, verb}}} \\{\fontspec{DejaVu Sans}▪ }\textsf{\textit{adjective}}\\ \textbf{1} Of very great excellence or beauty. {\fontspec{DejaVu Sans}◇} \textit{Mozart's sublime piano concertos} \colorBulletS{SYN} exalted, elevated, noble, lofty, awe{-}inspiring, awesome, majestic, magnificent, imposing, glorious, supreme \textbf{2} (of a person's attitude or behaviour) extreme or unparalleled. {\fontspec{DejaVu Sans}◇} \textit{he had the sublime confidence of youth} \colorBulletS{SYN} supreme, total, complete, utter, consummate, extreme \\{\fontspec{DejaVu Sans}▪ }\textsf{\textit{verb}}\\ \textbf{1} (of a solid substance) change directly into vapour when heated, typically forming a solid deposit again on cooling. {\fontspec{DejaVu Sans}◇} \textit{the ice sublimed away, leaving the books dry and undamaged} \textbf{2} Elevate to a high degree of moral or spiritual purity or excellence. {\fontspec{DejaVu Sans}◇} \textit{let your thoughts be sublimed by the spirit of God}}{}{}{ \colorBullet{ORIGIN} Late 16th century (in the sense ‘dignified, aloof’): from Latin sublimis, from sub{-} ‘up to’ + a second element perhaps related to limen ‘threshold’, limus ‘oblique’.}%
\par%
\entry{submerge}{/səbˈməːdʒ/}{নিমজ্জিত}{ \textsf{\textit{verb}}\ \textbf{1} Cause (something) to be under water. {\fontspec{DejaVu Sans}◇} \textit{houses had been flooded and cars submerged} \colorBulletS{SYN} flood, inundate, deluge, engulf, swamp, immerse, drown}{}{New areas submerged in sherpur}{ \colorBullet{ORIGIN} Early 17th century from Latin submergere, from sub{-} ‘under’ + mergere ‘to dip’.}%
\par%
\entry{subsequent}{/ˈsʌbsɪkw(ə)nt/}{পরবর্তী}{ \textsf{\textit{adjective}}\ \textbf{1} Coming after something in time; following. {\fontspec{DejaVu Sans}◇} \textit{the theory was developed subsequent to the earthquake of 1906} \colorBulletS{SYN} following, ensuing, succeeding, successive, later, future, coming, upcoming, to come, next}{}{}{ \colorBullet{ORIGIN} Late Middle English from Old French, or from Latin subsequent{-} ‘following after’ (from the verb subsequi).}%
\par%
\entry{subsequently}{/ˈsʌbsɪkwəntli/}{পরবর্তীকালে}{ \textsf{\textit{adverb}}\ \textbf{1} After a particular thing has happened; afterwards. {\fontspec{DejaVu Sans}◇} \textit{the officer decided to stop and subsequently made an arrest} \colorBulletS{SYN} later, later on, at a later date, at some point in the future, at some time in the future, at a subsequent time, afterwards, in due course, following that, following this, eventually, then, next, by and by}{}{}{}%
\par%
\entry{subside}{/səbˈsʌɪd/}{থিতান}{ \textsf{\textit{verb}}\ \textbf{1} Become less intense, violent, or severe. {\fontspec{DejaVu Sans}◇} \textit{I'll wait a few minutes until the storm subsides} \colorBulletS{SYN} abate, let up, moderate, quieten down, calm, lull, slacken, slacken off, ease, ease up, relent, die down, die out, peter out, taper off, recede, lessen, soften, alleviate, attenuate, remit, diminish, decline, dwindle, weaken, fade, wane, ebb, still, cease, come to a stop, come to an end, terminate \textbf{2} (of water) go down to a lower or the normal level. {\fontspec{DejaVu Sans}◇} \textit{the floods subside almost as quickly as they arise} \colorBulletS{SYN} recede, ebb, fall back, flow back, fall away, fall, go down, get lower, sink, sink lower}{}{}{ \colorBullet{ORIGIN} Late 17th century from Latin subsidere, from sub{-} ‘below’ + sidere ‘settle’ (related to sedere ‘sit’).}%
\par%
\entry{subsidy}{/ˈsʌbsɪdi/}{ভর্তুকি}{ \textsf{\textit{noun}}\ \textbf{1} A sum of money granted by the state or a public body to help an industry or business keep the price of a commodity or service low. {\fontspec{DejaVu Sans}◇} \textit{a farm subsidy} \textbf{2} A parliamentary grant to the sovereign for state needs. {\fontspec{DejaVu Sans}◇} \textit{}}{}{}{ \colorBullet{ORIGIN} Late Middle English from Anglo{-}Norman French subsidie, from Latin subsidium ‘assistance’.}%
\par%
\entry{substandard}{/sʌbˈstandəd/}{নিম্ন মানের}{ \textsf{\textit{adjective}}\ \textbf{1} Below the usual or required standard. {\fontspec{DejaVu Sans}◇} \textit{substandard housing} \colorBulletS{SYN} inferior, second{-}rate, low{-}quality, low{-}grade, poor, poor{-}quality, inadequate, imperfect, faulty, defective, jerry{-}built, shoddy, shabby, crude, unsound, unacceptable, unsatisfactory, unworthy, disappointing \textbf{2} another term for non{-}standard {\fontspec{DejaVu Sans}◇} \textit{sub{-}standard spellings}}{}{The court ordered the authorities concerned to stop production, selling or marketing of these substandard products and to take appropriate legal action against the persons responsible for producing, selling, marketing and supplying the products.}{}%
\par%
\entry{substantial}{/səbˈstanʃ(ə)l/}{সারগর্ভ}{ \textsf{\textit{adjective}}\ \textbf{1} Of considerable importance, size, or worth. {\fontspec{DejaVu Sans}◇} \textit{a substantial amount of cash} \colorBulletS{SYN} considerable, real, material, weighty, solid, sizeable, meaningful, significant, important, notable, major, marked, valuable, useful, worthwhile \textbf{2} Concerning the essentials of something. {\fontspec{DejaVu Sans}◇} \textit{there was substantial agreement on changing policies} \colorBulletS{SYN} fundamental, essential, basic \textbf{3} Real and tangible rather than imaginary. {\fontspec{DejaVu Sans}◇} \textit{spirits are shadowy, human beings substantial} \colorBulletS{SYN} real, true, actual, existing}{}{}{ \colorBullet{ORIGIN} Middle English from Old French substantiel or Christian Latin substantialis, from substantia ‘being, essence’ (see substance).}%
\par%
\entry{subvert}{/səbˈvəːt/}{পরাভূত করা}{ \textsf{\textit{verb}}\ \textbf{1} Undermine the power and authority of (an established system or institution) {\fontspec{DejaVu Sans}◇} \textit{an attempt to subvert democratic government} \colorBulletS{SYN} destabilize, unsettle, overthrow, overturn}{}{}{ \colorBullet{ORIGIN} Late Middle English from Old French subvertir or Latin subvertere, from sub{-} ‘from below’ + vertere ‘to turn’.}%
\par%
\entry{successive}{/səkˈsɛsɪv/}{ধারাবাহিক}{ \textsf{\textit{adjective}}\ \textbf{1} Following one another or following others. {\fontspec{DejaVu Sans}◇} \textit{they were looking for their fifth successive win} \colorBulletS{SYN} consecutive, in a row, straight, solid, sequential, succeeding, in succession, following, serial, running, continuous, unbroken, uninterrupted}{}{}{ \colorBullet{ORIGIN} Late Middle English from medieval Latin successivus, from success{-} ‘followed closely’, from the verb succedere (see succeed).}%
\par%
\entry{suck}{/sʌk/}{স্তন্যপান}{\small{\textsf{\textit{exclamation, noun, verb}}} \\{\fontspec{DejaVu Sans}▪ }\textsf{\textit{exclamation}}\\ \textbf{1} Used to express derision and defiance. {\fontspec{DejaVu Sans}◇} \textit{sucks to them!} \\{\fontspec{DejaVu Sans}▪ }\textsf{\textit{noun}}\\ \textbf{1} An act of sucking something. {\fontspec{DejaVu Sans}◇} \textit{the fish draws the bait into its mouth with a strong suck} \\{\fontspec{DejaVu Sans}▪ }\textsf{\textit{verb}}\\ \textbf{1} Draw into the mouth by contracting the muscles of the lips and mouth to make a partial vacuum. {\fontspec{DejaVu Sans}◇} \textit{they suck mint juleps through straws} \colorBulletS{SYN} sip, sup, siphon, slurp, draw, drink, gulp, lap, guzzle, quaff, swill, swallow, imbibe \textbf{2} Involve (someone) in something without their choosing. {\fontspec{DejaVu Sans}◇} \textit{I didn't want to be sucked into the role of dutiful daughter} \colorBulletS{SYN} implicate in, involve in, draw into \textbf{3} Be very bad or unpleasant. {\fontspec{DejaVu Sans}◇} \textit{I love your country but your weather sucks} \colorBulletS{SYN} be very bad, be awful, be terrible, be dreadful, be horrible, be very unpleasant, be abhorrent, be despicable, be contemptible, be vile, be foul}{}{}{ \colorBullet{ORIGIN} Old English sūcan (verb), from an Indo{-}European imitative root; related to soak.}%
\par%
\entry{sue}{/s(j)uː/}{বিরুদ্ধে মামলা দায়ের}{ \textsf{\textit{verb}}\ \textbf{1} Institute legal proceedings against (a person or institution), typically for redress. {\fontspec{DejaVu Sans}◇} \textit{she is to sue the baby's father} \colorBulletS{SYN} take legal action against, take to court, bring an action against, bring a suit against, proceed against \textbf{2} Appeal formally to a person for something. {\fontspec{DejaVu Sans}◇} \textit{the rebels were forced to sue for peace} \colorBulletS{SYN} appeal, petition, ask, beg, plead, entreat, implore, supplicate}{}{}{ \colorBullet{ORIGIN} Middle English from Anglo{-}Norman French suer, based on Latin sequi ‘follow’. Early senses were very similar to those of the verb follow.}%
\par%
\entry{suffocating}{/ˈsʌfəkeɪtɪŋ/}{শ্বাসরোধী}{ \textsf{\textit{adjective}}\ \textbf{1} Causing difficulty in breathing. {\fontspec{DejaVu Sans}◇} \textit{the suffocating heat}}{}{Please, you are suffocating me.}{}%
\par%
\entry{sully}{/ˈsʌli/}{নোংরা করা}{ \textsf{\textit{verb}}\ \textbf{1} Damage the purity or integrity of. {\fontspec{DejaVu Sans}◇} \textit{they were outraged that anyone should sully their good name} \colorBulletS{SYN} taint, defile, soil, tarnish, stain, blemish, besmirch, befoul, contaminate, pollute, spoil, mar, spot, make impure, disgrace, dishonour, injure, damage}{}{}{ \colorBullet{ORIGIN} Late 16th century perhaps from French souiller ‘to soil’.}%
\par%
\entry{summit}{/ˈsʌmɪt/}{শিখর}{\small{\textsf{\textit{noun, verb}}} \\{\fontspec{DejaVu Sans}▪ }\textsf{\textit{noun}}\\ \textbf{1} The highest point of a hill or mountain. {\fontspec{DejaVu Sans}◇} \textit{she climbed back up the path towards the summit} \colorBulletS{SYN} top, peak, mountaintop, crest, crown, apex, vertex, apogee, tip, cap \textbf{2} A meeting between heads of government. {\fontspec{DejaVu Sans}◇} \textit{two binding treaties were agreed at the summit} \colorBulletS{SYN} meeting, negotiation, conference, talk, talks, discussion, conclave, consultation, deliberation, dialogue, parley, colloquy \\{\fontspec{DejaVu Sans}▪ }\textsf{\textit{verb}}\\ \textbf{1} Reach the summit of (a mountain or hill) {\fontspec{DejaVu Sans}◇} \textit{in 2013, 658 climbers summited Everest}}{}{}{ \colorBullet{ORIGIN} Late Middle English (in the general sense ‘top part’): from Old French somete, from som ‘top’, from Latin summum, neuter of summus ‘highest’.}%
\par%
\entry{suo moto}{}{Suo motu, meaning "on its own motion," is a Latin legal term, approximately equivalent to the term sua sponte. For example, it is used where a government agency acts on its own cognizance, as in "the Commission took suo motu control over the matter." Example {-} "there is no requirement that a court suo motu instruct a jury upon these defenses." State v. Pierson.}{\small{\textsf{\textit{}}}}{}{}{}%
\par%
\entry{superiority complex}{}{An attitude of superiority which conceals actual feelings of inferiority and failure.}{\small{\textsf{\textit{}}}}{}{}{}%
\par%
\entry{surge}{/səːdʒ/}{ঢেউ}{\small{\textsf{\textit{noun, verb}}} \\{\fontspec{DejaVu Sans}▪ }\textsf{\textit{noun}}\\ \textbf{1} A sudden powerful forward or upward movement, especially by a crowd or by a natural force such as the tide. {\fontspec{DejaVu Sans}◇} \textit{flooding caused by tidal surges} \colorBulletS{SYN} gush, rush, outpouring, stream, flow, sweep \\{\fontspec{DejaVu Sans}▪ }\textsf{\textit{verb}}\\ \textbf{1} (of a crowd or a natural force) move suddenly and powerfully forward or upward. {\fontspec{DejaVu Sans}◇} \textit{the journalists surged forward} \colorBulletS{SYN} gush, rush, stream, flow, burst, pour, cascade, spill, overflow, brim over, well, sweep, spout, spurt, jet, spew, discharge, roll, whirl \textbf{2} (of a rope, chain, or windlass) slip back with a jerk. {\fontspec{DejaVu Sans}◇} \textit{}}{}{}{ \colorBullet{ORIGIN} Late 15th century (in the sense ‘fountain, stream’): the noun (in early use) from Old French sourgeon; the verb partly from the Old French stem sourge{-}, based on Latin surgere ‘to rise’. Early senses of the verb included ‘rise and fall on the waves’ and ‘swell with great force’.}%
\par%
\entry{suspected}{/səˈspɛktɪd/}{সন্দেহভাজন}{ \textsf{\textit{adjective}}\ \textbf{1} Believed to exist or to be true, without certain proof. {\fontspec{DejaVu Sans}◇} \textit{a suspected heart condition}}{}{}{}%
\par%
\entry{suspended}{/səˈspɛndɪd/}{স্থগিত}{ \textsf{\textit{adjective}}\ \textbf{1} (of a sentence) imposed by a judge or court but not enforced as long as no further offence is committed within a specified period. {\fontspec{DejaVu Sans}◇} \textit{he was given a suspended jail term of 22 months} \textbf{2} (of solid particles) dispersed through the bulk of a fluid. {\fontspec{DejaVu Sans}◇} \textit{suspended sediments inhibit the sun's energy from being used for reef building} \textbf{3} Supported by attachment from above; hanging. {\fontspec{DejaVu Sans}◇} \textit{small vents in the suspended ceilings supply fresh air}}{}{}{}%
\par%
\entry{suspension}{/səˈspɛnʃ(ə)n/}{সাসপেনশন}{ \textsf{\textit{noun}}\ \textbf{1} The action of suspending someone or something or the condition of being suspended. {\fontspec{DejaVu Sans}◇} \textit{the suspension of military action} \colorBulletS{SYN} adjournment, interruption, postponement, delay, deferral, deferment, shelving, stay, moratorium, arrest, intermission, interlude, prorogation, tabling, abeyance \textbf{2} The system of springs and shock absorbers by which a vehicle is supported on its wheels. {\fontspec{DejaVu Sans}◇} \textit{modifications have been made to the car's rear suspension} \textbf{3} A mixture in which particles are dispersed throughout the bulk of a fluid. {\fontspec{DejaVu Sans}◇} \textit{a suspension of maize starch in arachis oil} \colorBulletS{SYN} mixture, mix, blend, compound, suspension, tincture, infusion, emulsion, colloid, gel, fluid \textbf{4} A discord made by prolonging a note of a chord into the following chord. {\fontspec{DejaVu Sans}◇} \textit{}}{}{}{ \colorBullet{ORIGIN} Late Middle English from French, or from Latin suspensio(n{-}), from the verb suspendere (see suspend).}%
\par%
\entry{sustain}{/səˈsteɪn/}{বজায় রাখা}{\small{\textsf{\textit{noun, verb}}} \\{\fontspec{DejaVu Sans}▪ }\textsf{\textit{noun}}\\ \textbf{1} An effect or facility on a keyboard or electronic instrument whereby a note can be sustained after the key is released. {\fontspec{DejaVu Sans}◇} \textit{} \\{\fontspec{DejaVu Sans}▪ }\textsf{\textit{verb}}\\ \textbf{1} Strengthen or support physically or mentally. {\fontspec{DejaVu Sans}◇} \textit{this thought had sustained him throughout the years} \colorBulletS{SYN} comfort, help, assist, encourage, succour, support, give strength to, be a source of strength to, be a tower of strength to, buoy up, carry, cheer up, hearten, see someone through \textbf{2} Undergo or suffer (something unpleasant, especially an injury) {\fontspec{DejaVu Sans}◇} \textit{he sustained severe head injuries} \colorBulletS{SYN} undergo, experience, go through, suffer, endure \textbf{3} Cause to continue for an extended period or without interruption. {\fontspec{DejaVu Sans}◇} \textit{he cannot sustain a normal conversation} \colorBulletS{SYN} continuous, ongoing, steady, continual, continuing, constant, running, prolonged, persistent, non{-}stop, perpetual, unfaltering, unremitting, unabating, unrelenting, relentless, unrelieved, unbroken, never{-}ending, unending, incessant, unceasing, ceaseless, round the clock \textbf{4} Uphold, affirm, or confirm the justice or validity of. {\fontspec{DejaVu Sans}◇} \textit{the allegations of discrimination were sustained} \colorBulletS{SYN} uphold, validate, ratify, vindicate, confirm, endorse, approve}{}{}{ \colorBullet{ORIGIN} Middle English from Old French soustenir, from Latin sustinere, from sub{-} ‘from below’ + tenere ‘hold’.}%
\par%
\entry{swallow}{/ˈswɒləʊ/}{গেলা}{\small{\textsf{\textit{noun, verb}}} \\{\fontspec{DejaVu Sans}▪ }\textsf{\textit{noun}}\\ \textbf{1} An act of swallowing something, especially food or drink. {\fontspec{DejaVu Sans}◇} \textit{he downed his drink in one swallow} \\{\fontspec{DejaVu Sans}▪ }\textsf{\textit{verb}}\\ \textbf{1} Cause or allow (something, especially food or drink) to pass down the throat. {\fontspec{DejaVu Sans}◇} \textit{she swallowed a mouthful slowly} \colorBulletS{SYN} eat, gulp down, consume, devour, eat up, put away, gobble, gobble up, bolt, bolt down, wolf down, stuff down, gorge oneself on, feast on, polish off \textbf{2} Take in and cause to disappear; engulf. {\fontspec{DejaVu Sans}◇} \textit{the dark mist swallowed her up} \colorBulletS{SYN} engulf, swamp, devour, flood over, overwhelm, overcome, bury, drown, inundate}{}{}{ \colorBullet{ORIGIN} Old English swelgan, of Germanic origin; related to Dutch zwelgen and German schwelgen.}%
\par%
\entry{swallow}{/ˈswɒləʊ/}{গেলা}{ \textsf{\textit{noun}}\ \textbf{1} A migratory swift{-}flying songbird with a forked tail and long pointed wings, feeding on insects in flight. {\fontspec{DejaVu Sans}◇} \textit{}}{}{}{ \colorBullet{ORIGIN} Old English swealwe, of Germanic origin; related to Dutch zwaluw and German Schwalbe.}%
\par%
\entry{swear}{/swɛː/}{শপথ}{ \textsf{\textit{verb}}\ \textbf{1} Make a solemn statement or promise undertaking to do something or affirming that something is the case. {\fontspec{DejaVu Sans}◇} \textit{Maria made me swear I would never tell anyone} \colorBulletS{SYN} promise, vow, promise under oath, solemnly promise, pledge oneself, give one's word, take an oath, swear an oath, swear on the Bible, give an undertaking, undertake, affirm, warrant, state, assert, declare, aver, proclaim, pronounce, profess, attest, guarantee \textbf{2} Use offensive language, especially as an expression of anger. {\fontspec{DejaVu Sans}◇} \textit{Peter swore under his breath} \colorBulletS{SYN} bad language, foul language, strong language}{}{}{ \colorBullet{ORIGIN} Old English swerian of Germanic origin; related to Dutch zweren, German schwören, also to answer.}%
\par%
\entry{sweep}{/swiːp/}{কুড়ান}{\small{\textsf{\textit{noun, verb}}} \\{\fontspec{DejaVu Sans}▪ }\textsf{\textit{noun}}\\ \textbf{1} An act of sweeping something with a brush. {\fontspec{DejaVu Sans}◇} \textit{I was giving the floor a quick sweep} \colorBulletS{SYN} clean, sweep, wipe, dust, mop \textbf{2} A long, swift curving movement. {\fontspec{DejaVu Sans}◇} \textit{a grandiose sweep of his hand} \colorBulletS{SYN} gesture, movement, move, action, stroke, wave \textbf{3} A procedure for inducing labour in a pregnant woman, in which a medical practitioner moves a finger around within the opening of the cervix to detach the amniotic membranes. {\fontspec{DejaVu Sans}◇} \textit{I went in for a sweep at 41 weeks} \textbf{4} A comprehensive search or survey of a place or area. {\fontspec{DejaVu Sans}◇} \textit{the police finished their sweep through the woods} \colorBulletS{SYN} search, hunt, exploration, probe, forage, pursuit, quest \textbf{5} A long, typically curved stretch of road, river, country, etc. {\fontspec{DejaVu Sans}◇} \textit{we could see a wide sweep of country perhaps a hundred miles across} \colorBulletS{SYN} expanse, tract, stretch, space, plain, extent, vastness, vista \textbf{6} A sweepstake. {\fontspec{DejaVu Sans}◇} \textit{} \colorBulletS{SYN} lottery, draw, prize draw, sweepstake, sweep, tombola, ballot \textbf{7} An instance of winning every event, award, or place in a contest. {\fontspec{DejaVu Sans}◇} \textit{a World Series sweep} \textbf{8} A long, heavy oar used to row a barge or other vessel. {\fontspec{DejaVu Sans}◇} \textit{a big, heavy sweep oar} \colorBulletS{SYN} oar, scull, sweep, blade, spoon, spade \textbf{9} A sail of a windmill. {\fontspec{DejaVu Sans}◇} \textit{} \textbf{10} A long pole mounted as a lever for raising buckets from a well. {\fontspec{DejaVu Sans}◇} \textit{} \\{\fontspec{DejaVu Sans}▪ }\textsf{\textit{verb}}\\ \textbf{1} Clean (an area) by brushing away dirt or litter. {\fontspec{DejaVu Sans}◇} \textit{I've swept the floor} \colorBulletS{SYN} brush, clean, scrub, wipe, mop, dust, scour, scrape, rake, buff \textbf{2} Move swiftly and smoothly. {\fontspec{DejaVu Sans}◇} \textit{a large black car swept past the open windows} \colorBulletS{SYN} glide, sail, dash, charge, rush, streak, speed, fly, zoom, swoop, whizz, hurtle \textbf{3} Search (an area) for something. {\fontspec{DejaVu Sans}◇} \textit{the detective swept the room for hair and fingerprints} \colorBulletS{SYN} search, probe, check, explore, hunt through, look through, delve in, go through, sift through, scour, comb, go through with a fine{-}tooth comb, leave no stone unturned in}{}{}{ \colorBullet{ORIGIN} Old English swāpan (verb), of Germanic origin; related to German schweifen ‘sweep in a curve’.}%
\par%
\entry{sweeping}{/ˈswiːpɪŋ/}{সুদূরপ্রসারিত}{\small{\textsf{\textit{adjective, noun}}} \\{\fontspec{DejaVu Sans}▪ }\textsf{\textit{adjective}}\\ \textbf{1} Extending or performed in a long, continuous curve. {\fontspec{DejaVu Sans}◇} \textit{sweeping, desolate moorlands} \colorBulletS{SYN} broad, extensive, expansive, vast, spacious, roomy, boundless, panoramic \textbf{2} Wide in range or effect. {\fontspec{DejaVu Sans}◇} \textit{we cannot recommend any sweeping alterations} \colorBulletS{SYN} extensive, wide{-}ranging, global, broad, wide, comprehensive, all{-}inclusive, all{-}embracing, far{-}reaching, across the board, worldwide, catholic, exhaustive, pervasive \\{\fontspec{DejaVu Sans}▪ }\textsf{\textit{noun}}\\ \textbf{1} Dirt or refuse collected by sweeping. {\fontspec{DejaVu Sans}◇} \textit{the sweepings from the house} \colorBulletS{SYN} debris, waste, waste matter, discarded matter, refuse, rubbish, litter, scrap, flotsam and jetsam, lumber, rubble, wreckage}{}{}{}%
\par%
\entry{sweetmeat}{/ˈswiːtmiːt/}{মোদক}{ \textsf{\textit{noun}}\ \textbf{1} An item of confectionery or sweet food. {\fontspec{DejaVu Sans}◇} \textit{he hurried back to his room like a schoolboy who has stolen a sweetmeat} \colorBulletS{SYN} piece of confectionery, chocolate, bonbon, fondant, toffee}{}{}{}%
\par%
\entry{swell}{/swɛl/}{চিতান}{\small{\textsf{\textit{adjective, adverb, noun, verb}}} \\{\fontspec{DejaVu Sans}▪ }\textsf{\textit{adjective}}\\ \textbf{1} Excellent; very good. {\fontspec{DejaVu Sans}◇} \textit{you're looking swell} \colorBulletS{SYN} excellent, marvellous, wonderful, splendid, magnificent, superb, first{-}rate \\{\fontspec{DejaVu Sans}▪ }\textsf{\textit{adverb}}\\ \textbf{1} Excellently; very well. {\fontspec{DejaVu Sans}◇} \textit{everything was just going swell} \\{\fontspec{DejaVu Sans}▪ }\textsf{\textit{noun}}\\ \textbf{1} A full or gently rounded shape or form. {\fontspec{DejaVu Sans}◇} \textit{the soft swell of her breast} \textbf{2} A gradual increase in amount, intensity, or volume. {\fontspec{DejaVu Sans}◇} \textit{a huge swell in the popularity of one{-}day cricket} \colorBulletS{SYN} increase, rise, growth, expansion, escalation, acceleration, surge, stepping{-}up, proliferation, snowballing, mushrooming, skyrocketing \textbf{3} A slow, regular movement of the sea in rolling waves that do not break. {\fontspec{DejaVu Sans}◇} \textit{there was a heavy swell} \colorBulletS{SYN} billow, billowing, undulation, surge, surging, wave, roll, rolling, bulge, bulging, rush, deluge, movement \textbf{4} A mechanism for producing a crescendo or diminuendo in an organ or harmonium. {\fontspec{DejaVu Sans}◇} \textit{} \textbf{5} A fashionable or stylish person of wealth or high social position. {\fontspec{DejaVu Sans}◇} \textit{a crowd of city swells} \colorBulletS{SYN} fop, beau, man about town, bright young thing, glamour boy, rake \\{\fontspec{DejaVu Sans}▪ }\textsf{\textit{verb}}\\ \textbf{1} (especially of a part of the body) become larger or rounder in size, typically as a result of an accumulation of fluid. {\fontspec{DejaVu Sans}◇} \textit{her bruised knee was already swelling up} \colorBulletS{SYN} expand, bulge, distend, become distended, inflate, become inflated, dilate, become bloated, bloat, blow out, blow up, puff up, balloon, fatten, fill out, tumefy, intumesce \textbf{2} Become or make greater in intensity, number, amount, or volume. {\fontspec{DejaVu Sans}◇} \textit{the low murmur swelled to a roar} \colorBulletS{SYN} grow larger, grow greater, grow, enlarge, increase, expand, rise, wax, mount, escalate, accelerate, step up, accumulate, surge, multiply, proliferate, snowball, mushroom, skyrocket}{}{}{ \colorBullet{ORIGIN} Old English swellan (verb), of Germanic origin; related to German schwellen. Current senses of the noun date from the early 16th century; the informal adjectival use derives from noun swell (sense 5 of the noun) (late 18th century).}%
\par%
\entry{swimmer}{/ˈswɪmə/}{সাঁতারু}{ \textsf{\textit{noun}}\ \textbf{1} A person or animal that swims. {\fontspec{DejaVu Sans}◇} \textit{the fastest freestyle swimmer in the world}}{}{}{}%
\par%
\entry{symposium}{/sɪmˈpəʊzɪəm/}{সম্মেলন}{ \textsf{\textit{noun}}\ \textbf{1} A conference or meeting to discuss a particular subject. {\fontspec{DejaVu Sans}◇} \textit{} \colorBulletS{SYN} meeting, sitting, assembly, conclave, plenary \textbf{2} A drinking party or convivial discussion, especially as held in ancient Greece after a banquet (and notable as the title of a work by Plato). {\fontspec{DejaVu Sans}◇} \textit{} \colorBulletS{SYN} lecture, speech, address, discourse, oration, presentation, report, sermon, disquisition, dissertation, symposium}{}{}{ \colorBullet{ORIGIN} Late 16th century (denoting a drinking party): via Latin from Greek sumposion, from sumpotēs ‘fellow drinker’, from sun{-} ‘together’ + potēs ‘drinker’.}%
\par%
\end{multicols}%
\pagebreak%
\section*{T}%
\begin{multicols}{2}%
\entry{tailoring}{/ˈteɪlərɪŋ/}{দরজির কার্য}{ \textsf{\textit{noun}}\ \textbf{1} The activity or trade of a tailor. {\fontspec{DejaVu Sans}◇} \textit{they learnt woodwork, tailoring, and other trades}}{}{}{}%
\par%
\entry{take a whiz}{}{1. to take a piss; to urinate 2. to send urine out of the body}{\small{\textsf{\textit{}}}}{}{I'll be right back. I have to take a whiz.}{}%
\par%
\entry{takeaway}{/ˈteɪkəweɪ/}{ছাড়াইয়া লত্তয়া}{ \textsf{\textit{noun}}\ \textbf{1} A restaurant or shop selling cooked food to be eaten elsewhere. {\fontspec{DejaVu Sans}◇} \textit{a fast{-}food takeaway} \textbf{2} A key fact, point, or idea to be remembered, typically one emerging from a discussion or meeting. {\fontspec{DejaVu Sans}◇} \textit{the main takeaway for me is that we need to continue to communicate all the things we're doing for our customers} \textbf{3} another term for backswing {\fontspec{DejaVu Sans}◇} \textit{many golfers ruin the swing with a poor takeaway} \textbf{4} (in football and hockey) an act of regaining the ball or puck from the opposing team. {\fontspec{DejaVu Sans}◇} \textit{}}{}{}{}%
\par%
\entry{tangle}{/ˈtaŋɡ(ə)l/}{জট}{\small{\textsf{\textit{noun, verb}}} \\{\fontspec{DejaVu Sans}▪ }\textsf{\textit{noun}}\\ \textbf{1} A confused mass of something twisted together. {\fontspec{DejaVu Sans}◇} \textit{a tangle of golden hair} \colorBulletS{SYN} snarl, mass, mat, cluster, knot, mesh, disorder, thatch, web \textbf{2} A fight, argument, or disagreement. {\fontspec{DejaVu Sans}◇} \textit{she got into a tangle with staff} \\{\fontspec{DejaVu Sans}▪ }\textsf{\textit{verb}}\\ \textbf{1} Twist together into a confused mass. {\fontspec{DejaVu Sans}◇} \textit{the broom somehow got tangled up in my long skirt} \colorBulletS{SYN} ravelled, entangled, snarled, snarled up, entwined, intertwisted, twisted, knotted, knotty, enmeshed, coiled, matted, tangly, messy, muddled \textbf{2} Become involved in a conflict or fight with. {\fontspec{DejaVu Sans}◇} \textit{they usually come a cropper when they tangle with the heavy mobs} \colorBulletS{SYN} come into conflict, become involved, have a dispute, dispute, argue, quarrel, fight, row, wrangle, squabble, contend, cross swords, lock horns}{}{}{ \colorBullet{ORIGIN} Middle English (in the sense ‘entangle, catch in a tangle’): probably of Scandinavian origin and related to Swedish dialect taggla ‘disarrange’.}%
\par%
\entry{tangle}{/ˈtaŋɡ(ə)l/}{জট}{ \textsf{\textit{noun}}\ \textbf{1} Any of a number of brown seaweeds, especially oarweed. {\fontspec{DejaVu Sans}◇} \textit{}}{}{}{ \colorBullet{ORIGIN} Mid 16th century probably from Norwegian tongul.}%
\par%
\entry{tariff}{/ˈtarɪf/}{শুল্ক}{\small{\textsf{\textit{noun, verb}}} \\{\fontspec{DejaVu Sans}▪ }\textsf{\textit{noun}}\\ \textbf{1} A tax or duty to be paid on a particular class of imports or exports. {\fontspec{DejaVu Sans}◇} \textit{the reduction of trade barriers and import tariffs} \colorBulletS{SYN} tax, duty, toll, excise, levy, assessment, imposition, impost, charge, rate, fee, exaction \\{\fontspec{DejaVu Sans}▪ }\textsf{\textit{verb}}\\ \textbf{1} Fix the price of (something) according to a tariff. {\fontspec{DejaVu Sans}◇} \textit{these services are tariffed by volume}}{}{}{ \colorBullet{ORIGIN} Late 16th century (also denoting an arithmetical table): via French from Italian tariffa, based on Arabic ‘arrafa ‘notify’.}%
\par%
\entry{taunt}{/tɔːnt/}{বিদ্রূপ}{\small{\textsf{\textit{noun, verb}}} \\{\fontspec{DejaVu Sans}▪ }\textsf{\textit{noun}}\\ \textbf{1} A remark made in order to anger, wound, or provoke someone. {\fontspec{DejaVu Sans}◇} \textit{pupils will play truant rather than face the taunts of classmates about their ragged clothes} \colorBulletS{SYN} jeer, gibe, sneer, insult, barb, catcall, brickbat, scoff, slap in the face \\{\fontspec{DejaVu Sans}▪ }\textsf{\textit{verb}}\\ \textbf{1} Provoke or challenge (someone) with insulting remarks. {\fontspec{DejaVu Sans}◇} \textit{pupils began taunting her about her weight} \colorBulletS{SYN} jeer at, gibe at, sneer at, scoff at, poke fun at, make fun of, get at, insult, tease, chaff, torment, provoke, goad, ridicule, deride, mock, heckle}{}{}{ \colorBullet{ORIGIN} Early 16th century from French tant pour tant ‘like for like, tit for tat’, from tant ‘so much’, from Latin tantum, neuter of tantus. An early use of the verb was ‘exchange banter’.}%
\par%
\entry{taunting}{/ˈtɔːntɪŋ/}{বিদ্রূপাত্মক}{ \textsf{\textit{adjective}}\ \textbf{1} Intended to provoke someone in an insulting or contemptuous manner. {\fontspec{DejaVu Sans}◇} \textit{taunting comments}}{}{1. The advertisement in a way is taunting the supporters. 2. New zealand cricket have admonished a stadium announcer for taunting pakistan fast bowler mohammad amir during the third t20 on friday.}{}%
\par%
\entry{tear}{/tɛː/}{বিছিন্ন করা}{\small{\textsf{\textit{noun, verb}}} \\{\fontspec{DejaVu Sans}▪ }\textsf{\textit{noun}}\\ \textbf{1} A hole or split in something caused by it having been pulled apart forcefully. {\fontspec{DejaVu Sans}◇} \textit{there was a tear in her dress} \colorBulletS{SYN} rip, hole, split, rent, cut, slash, slit \textbf{2} A brief spell of erratic or unrestrained behaviour; a binge or spree. {\fontspec{DejaVu Sans}◇} \textit{one of my drinking buddies came for the weekend and we went on a tear} \\{\fontspec{DejaVu Sans}▪ }\textsf{\textit{verb}}\\ \textbf{1} Pull (something) apart or to pieces with force. {\fontspec{DejaVu Sans}◇} \textit{I tore up the letter} \colorBulletS{SYN} rip up, rip in two, pull apart, pull to pieces, shred \textbf{2} Move very quickly in a reckless or excited manner. {\fontspec{DejaVu Sans}◇} \textit{she tore along the footpath on her bike} \colorBulletS{SYN} sprint, race, run, dart, rush, dash, hasten, hurry, scurry, scuttle, scamper, hare, bolt, bound, fly, gallop, career, charge, pound, shoot, hurtle, speed, streak, flash, whizz, zoom, sweep, go like lightning, go hell for leather, go like the wind \textbf{3} Be in a state of uncertainty between two conflicting options or parties. {\fontspec{DejaVu Sans}◇} \textit{he was torn between his duty and his better instincts} \colorBulletS{SYN} torment, torture, rack, harrow, wring, lacerate}{}{}{ \colorBullet{ORIGIN} Old English teran, of Germanic origin; related to Dutch teren and German zehren, from an Indo{-}European root shared by Greek derein ‘flay’. The noun dates from the early 17th century.}%
\par%
\entry{tear}{/tɪə/}{বিছিন্ন করা}{\small{\textsf{\textit{noun, verb}}} \\{\fontspec{DejaVu Sans}▪ }\textsf{\textit{noun}}\\ \textbf{1} A drop of clear salty liquid secreted from glands in a person's eye when they cry or when the eye is irritated. {\fontspec{DejaVu Sans}◇} \textit{a tear rolled down her cheek} \colorBulletS{SYN} teardrop \\{\fontspec{DejaVu Sans}▪ }\textsf{\textit{verb}}\\ \textbf{1} (of the eye) produce tears. {\fontspec{DejaVu Sans}◇} \textit{the freezing wind made her eyes tear}}{}{}{ \colorBullet{ORIGIN} Old English tēar, of Germanic origin; related to German Zähre, from an Indo{-}European root shared by Old Latin dacruma (Latin lacrima) and Greek dakru.}%
\par%
\entry{tease}{/tiːz/}{আঁচড়ান}{\small{\textsf{\textit{noun, verb}}} \\{\fontspec{DejaVu Sans}▪ }\textsf{\textit{noun}}\\ \textbf{1} A person who makes fun of someone playfully or unkindly. {\fontspec{DejaVu Sans}◇} \textit{some think of him as a tease who likes to keep others guessing} \colorBulletS{SYN} tease, make fun of, chaff \textbf{2} An act of teasing someone. {\fontspec{DejaVu Sans}◇} \textit{she couldn't resist a gentle tease} \\{\fontspec{DejaVu Sans}▪ }\textsf{\textit{verb}}\\ \textbf{1} Make fun of or attempt to provoke (a person or animal) in a playful way. {\fontspec{DejaVu Sans}◇} \textit{I used to tease her about being so house{-}proud} \colorBulletS{SYN} make fun of, poke fun at, chaff, make jokes about, rag, mock, laugh at, guy, satirize, be sarcastic about \textbf{2} Gently pull or comb (tangled wool, hair, etc.) into separate strands. {\fontspec{DejaVu Sans}◇} \textit{she was teasing out the curls into her usual hairstyle}}{}{}{ \colorBullet{ORIGIN} Old English tǣsan (in tease (sense 2 of the verb)), of West Germanic origin; related to Dutch teezen and German dialect zeisen, also to teasel. Sense 1 is a development of the earlier and more serious ‘irritate by annoying actions’ (early 17th century), a figurative use of the word's original sense.}%
\par%
\entry{tempo}{/ˈtɛmpəʊ/}{লয়}{ \textsf{\textit{noun}}\ \textbf{1} The speed at which a passage of music is or should be played. {\fontspec{DejaVu Sans}◇} \textit{} \colorBulletS{SYN} cadence, speed, rhythm, beat, time, pulse \textbf{2} The rate or speed of motion or activity; pace. {\fontspec{DejaVu Sans}◇} \textit{the tempo of life dictated by a heavy workload} \colorBulletS{SYN} pace, rate, speed, velocity}{}{}{ \colorBullet{ORIGIN} Mid 17th century (as a fencing term denoting the timing of an attack): from Italian, from Latin tempus ‘time’.}%
\par%
\entry{tempo}{/ˈtɛmpəʊ/}{লয়}{ \textsf{\textit{noun}}\ \textbf{1} (in South Asia) a light three{-}wheeled delivery van. {\fontspec{DejaVu Sans}◇} \textit{}}{}{}{ \colorBullet{ORIGIN} An invented word.}%
\par%
\entry{tenure}{/ˈtɛnjə/}{ভোগদখল}{\small{\textsf{\textit{noun, verb}}} \\{\fontspec{DejaVu Sans}▪ }\textsf{\textit{noun}}\\ \textbf{1} The conditions under which land or buildings are held or occupied. {\fontspec{DejaVu Sans}◇} \textit{} \colorBulletS{SYN} tenancy, occupancy, holding, occupation, residence \textbf{2} The holding of an office. {\fontspec{DejaVu Sans}◇} \textit{his tenure of the premiership would be threatened} \colorBulletS{SYN} incumbency, term of office, term, period in office, period of office, time, time in office \textbf{3} Guaranteed permanent employment, especially as a teacher or lecturer, after a probationary period. {\fontspec{DejaVu Sans}◇} \textit{tenure for university staff has been abolished} \\{\fontspec{DejaVu Sans}▪ }\textsf{\textit{verb}}\\ \textbf{1} Give (someone) a permanent post, especially as a teacher or lecturer. {\fontspec{DejaVu Sans}◇} \textit{I had recently been tenured and then promoted to full professor}}{}{}{ \colorBullet{ORIGIN} Late Middle English from Old French, from tenir ‘to hold’, from Latin tenere.}%
\par%
\entry{terrestrial}{/təˈrɛstrɪəl/}{স্থলজ}{\small{\textsf{\textit{adjective, noun}}} \\{\fontspec{DejaVu Sans}▪ }\textsf{\textit{adjective}}\\ \textbf{1} On or relating to the earth. {\fontspec{DejaVu Sans}◇} \textit{increased ultraviolet radiation may disrupt terrestrial ecosystems} \colorBulletS{SYN} earthly, worldly, mundane, earthbound \textbf{2} Of or on dry land. {\fontspec{DejaVu Sans}◇} \textit{a submarine eruption will be much more explosive than its terrestrial counterpart} \\{\fontspec{DejaVu Sans}▪ }\textsf{\textit{noun}}\\ \textbf{1} An inhabitant of the earth. {\fontspec{DejaVu Sans}◇} \textit{}}{}{}{ \colorBullet{ORIGIN} Late Middle English (in the sense ‘temporal, worldly, mundane’): from Latin terrestris (from terra ‘earth’) + {-}al.}%
\par%
\entry{terrible}{/ˈtɛrɪb(ə)l/}{গুরুগম্ভীর}{ \textsf{\textit{adjective}}\ \textbf{1} Extremely bad or serious. {\fontspec{DejaVu Sans}◇} \textit{a terrible crime} \colorBulletS{SYN} dreadful, awful, appalling, horrific, horrifying, horrible, horrendous, atrocious, abominable, abhorrent, frightful, fearful, shocking, hideous, ghastly, grim, dire, hateful, unspeakable, gruesome, monstrous, sickening, heinous, vile \textbf{2} Causing or likely to cause terror; sinister. {\fontspec{DejaVu Sans}◇} \textit{the stranger gave a terrible smile}}{}{}{ \colorBullet{ORIGIN} Late Middle English (in the sense ‘causing terror’): via French from Latin terribilis, from terrere ‘frighten’.}%
\par%
\entry{theremin}{/ˈθɛrəmɪn/}{}{ \textsf{\textit{noun}}\ \textbf{1} An electronic musical instrument in which the tone is generated by two high{-}frequency oscillators and the pitch controlled by the movement of the performer's hand towards and away from the circuit. {\fontspec{DejaVu Sans}◇} \textit{}}{}{}{ \colorBullet{ORIGIN} Early 20th century named after Lev Theremin (1896–1993), its Russian inventor.}%
\par%
\entry{thesaurus}{/θɪˈsɔːrəs/}{জ্ঞানভাণ্ডার}{ \textsf{\textit{noun}}\ \textbf{1} A book that lists words in groups of synonyms and related concepts. {\fontspec{DejaVu Sans}◇} \textit{} \colorBulletS{SYN} wordfinder, wordbook, synonym dictionary, synonym lexicon}{}{}{ \colorBullet{ORIGIN} Late 16th century via Latin from Greek thēsauros ‘storehouse, treasure’. The original sense ‘dictionary or encyclopedia’ was narrowed to the current meaning by the publication of Roget's Thesaurus of English Words and Phrases (1852).}%
\par%
\entry{think about it}{}{1. Take into consideration, have in view; "he entertained the notion of moving to south america" 2. Used when someone says something that, in the hands of someone with a dirty mind, can sound sexual.}{\small{\textsf{\textit{}}}}{}{"pound it!"\newline%
"lol!"\newline%
"what?"\newline%
"think about it!"}{}%
\par%
\entry{thoroughfare}{/ˈθʌrəfɛː/}{রাস্তা; জনসাধারণের যাতায়াতের পথ}{ \textsf{\textit{noun}}\ \textbf{1} A road or path forming a route between two places. {\fontspec{DejaVu Sans}◇} \textit{a scheme to stop the park being used as a thoroughfare} \colorBulletS{SYN} through route, access route, way, passage}{}{Buses clog up major thoroughfare in town}{}%
\par%
\entry{throat}{/θrəʊt/}{গলা}{ \textsf{\textit{noun}}\ \textbf{1} The passage which leads from the back of the mouth of a person or animal. {\fontspec{DejaVu Sans}◇} \textit{her throat was parched with thirst} \colorBulletS{SYN} gullet, oesophagus}{}{}{ \colorBullet{ORIGIN} Old English throte, throtu, of Germanic origin; related to German Drossel. Compare with throttle.}%
\par%
\entry{thug}{/θʌɡ/}{সহযোগী গুণ্ডারা}{ \textsf{\textit{noun}}\ \textbf{1} A violent person, especially a criminal. {\fontspec{DejaVu Sans}◇} \textit{he was attacked by a gang of thugs} \colorBulletS{SYN} ruffian, hoodlum, bully boy, bully, bandit, mugger, gangster, terrorist, gunman, murderer, killer, hitman, assassin, hooligan, vandal, Yardie \textbf{2}  {\fontspec{DejaVu Sans}◇} \textit{}}{}{}{ \colorBullet{ORIGIN} Early 19th century (in thug (sense 2)): from Hindi ṭhag ‘swindler, thief’, based on Sanskrit sthagati ‘he covers or conceals’. thug (sense 1) arose in the mid 19th century.}%
\par%
\entry{tier}{/tɪə/}{স্তর}{ \textsf{\textit{noun}}\ \textbf{1} Each in a series of rows or levels of a structure placed one above the other. {\fontspec{DejaVu Sans}◇} \textit{a tier of seats} \colorBulletS{SYN} row, rank, bank, line}{}{}{ \colorBullet{ORIGIN} Late 15th century from French tire ‘sequence, order’, from tirer ‘elongate, draw’.}%
\par%
\entry{tilt}{/tɪlt/}{হেলানো}{\small{\textsf{\textit{noun, verb}}} \\{\fontspec{DejaVu Sans}▪ }\textsf{\textit{noun}}\\ \textbf{1} A sloping position or movement. {\fontspec{DejaVu Sans}◇} \textit{the tilt of her head} \colorBulletS{SYN} slope, list, camber, gradient, bank, slant, incline, pitch, dip, cant, bevel, angle, heel \textbf{2} A combat for exercise or sport between two men on horseback with lances; a joust. {\fontspec{DejaVu Sans}◇} \textit{} \colorBulletS{SYN} joust, tournament, tourney, lists, combat, contest, fight, duel \textbf{3} A small hut in a forest. {\fontspec{DejaVu Sans}◇} \textit{} \\{\fontspec{DejaVu Sans}▪ }\textsf{\textit{verb}}\\ \textbf{1} Move or cause to move into a sloping position. {\fontspec{DejaVu Sans}◇} \textit{the floor tilted slightly} \colorBulletS{SYN} lean, tip, list, slope, camber, bank, slant, incline, pitch, dip, cant, bevel, angle, cock, heel, careen, bend, be at an angle \textbf{2} (in jousting) thrust at with a lance or other weapon. {\fontspec{DejaVu Sans}◇} \textit{he tilts at his prey} \colorBulletS{SYN} charge, rush, run}{}{}{ \colorBullet{ORIGIN} Late Middle English (in the sense ‘fall or cause to fall, topple’): perhaps related to Old English tealt ‘unsteady’, or perhaps of Scandinavian origin and related to Norwegian tylten ‘unsteady’ and Swedish tulta ‘totter’.}%
\par%
\entry{timid}{/ˈtɪmɪd/}{ভীতু}{ \textsf{\textit{adjective}}\ \textbf{1} Showing a lack of courage or confidence; easily frightened. {\fontspec{DejaVu Sans}◇} \textit{I was too timid to ask for what I wanted} \colorBulletS{SYN} easily frightened, lacking courage, fearful, apprehensive, afraid, frightened, scared, faint{-}hearted}{}{}{ \colorBullet{ORIGIN} Mid 16th century from Latin timidus, from timere ‘to fear’.}%
\par%
\entry{tinkle}{/ˈtɪŋk(ə)l/}{টুংটাং শব্দ করা}{\small{\textsf{\textit{noun, verb}}} \\{\fontspec{DejaVu Sans}▪ }\textsf{\textit{noun}}\\ \textbf{1} A light, clear ringing sound. {\fontspec{DejaVu Sans}◇} \textit{the distant tinkle of a cow bell} \colorBulletS{SYN} ring, chime, peal, ding, ping, clink, chink, jingle, jangle \textbf{2} An act of urinating. {\fontspec{DejaVu Sans}◇} \textit{you have to pay to go in for a tinkle} \\{\fontspec{DejaVu Sans}▪ }\textsf{\textit{verb}}\\ \textbf{1} Make or cause to make a light, clear ringing sound. {\fontspec{DejaVu Sans}◇} \textit{cool water tinkled in the stone fountains} \colorBulletS{SYN} ring, jingle, jangle, chime, peal, ding, ping, clink, chink \textbf{2} Urinate. {\fontspec{DejaVu Sans}◇} \textit{I needed to tinkle}}{}{}{ \colorBullet{ORIGIN} Late Middle English (also in the sense ‘tingle’): frequentative of obsolete tink ‘to chink or clink’, of imitative origin.}%
\par%
\entry{tipsy}{/ˈtɪpsi/}{প্রমত্ত}{ \textsf{\textit{adjective}}\ \textbf{1} Slightly drunk. {\fontspec{DejaVu Sans}◇} \textit{tipsy revellers} \colorBulletS{SYN} merry, mellow, slightly drunk}{}{}{ \colorBullet{ORIGIN} Late 16th century from the verb tip+ {-}sy.}%
\par%
\entry{tire}{/tʌɪə/}{পাগড়ি}{ \textsf{\textit{verb}}\ \textbf{1} Feel or cause to feel in need of rest or sleep. {\fontspec{DejaVu Sans}◇} \textit{soon the ascent grew steeper and he began to tire} \colorBulletS{SYN} exhausting, wearying, fatiguing, enervating, draining, sapping, stressful, wearing, trying, crushing \textbf{2} Lose interest in; become bored with. {\fontspec{DejaVu Sans}◇} \textit{the media will tire of publicizing every protest}}{}{}{ \colorBullet{ORIGIN} Old English tēorian ‘fail, come to an end’, also ‘become physically exhausted’, of unknown origin.}%
\par%
\entry{tire}{/tʌɪə/}{পাগড়ি}{\small{\textsf{\textit{}}}}{}{}{}%
\par%
\entry{toddler}{/ˈtɒdlə/}{শক্তিশালী}{ \textsf{\textit{noun}}\ \textbf{1} A young child who is just beginning to walk. {\fontspec{DejaVu Sans}◇} \textit{} \colorBulletS{SYN} youngster, young one, little one, boy, girl}{}{}{}%
\par%
\entry{toll}{/təʊl/}{উপশুল্ক}{\small{\textsf{\textit{noun, verb}}} \\{\fontspec{DejaVu Sans}▪ }\textsf{\textit{noun}}\\ \textbf{1} A charge payable to use a bridge or road. {\fontspec{DejaVu Sans}◇} \textit{motorway tolls} \colorBulletS{SYN} charge, fee, payment, levy, tariff, dues, tax, duty, impost \textbf{2} The number of deaths or casualties arising from a natural disaster, conflict, accident, etc. {\fontspec{DejaVu Sans}◇} \textit{the toll of dead and injured mounted} \colorBulletS{SYN} number, count, tally, total, running total, sum total, grand total, sum, score, reckoning, enumeration, register, record, inventory, list, listing, account, roll, roster, index, directory \\{\fontspec{DejaVu Sans}▪ }\textsf{\textit{verb}}\\ \textbf{1} Charge a toll for the use of (a bridge or road) {\fontspec{DejaVu Sans}◇} \textit{the report advocates motorway tolling}}{}{}{ \colorBullet{ORIGIN} Old English (denoting a charge, tax, or duty), from medieval Latin toloneum, alteration of late Latin teloneum, from Greek telōnion ‘toll house’, from telos ‘tax’. toll (sense 2 of the noun) (late 19th century) arose from the notion of paying a toll or tribute in human lives (to an adversary or to death).}%
\par%
\entry{toll}{/təʊl/}{উপশুল্ক}{\small{\textsf{\textit{noun, verb}}} \\{\fontspec{DejaVu Sans}▪ }\textsf{\textit{noun}}\\ \textbf{1} A single ring of a bell. {\fontspec{DejaVu Sans}◇} \textit{she heard the Cambridge School bell utter a single toll} \\{\fontspec{DejaVu Sans}▪ }\textsf{\textit{verb}}\\ \textbf{1} (with reference to a bell) sound or cause to sound with a slow, uniform succession of strokes, as a signal or announcement. {\fontspec{DejaVu Sans}◇} \textit{the cathedral bells began to toll for evening service} \colorBulletS{SYN} ring, ring out, chime, chime out, strike, peal, knell}{}{}{ \colorBullet{ORIGIN} Late Middle English probably a special use of dialect toll ‘drag, pull’.}%
\par%
\entry{tongue}{/tʌŋ/}{জিহ্বা}{\small{\textsf{\textit{noun, verb}}} \\{\fontspec{DejaVu Sans}▪ }\textsf{\textit{noun}}\\ \textbf{1} The fleshy muscular organ in the mouth of a mammal, used for tasting, licking, swallowing, and (in humans) articulating speech. {\fontspec{DejaVu Sans}◇} \textit{} \textbf{2} Used in reference to a person's style or manner of speaking. {\fontspec{DejaVu Sans}◇} \textit{he was a redoubtable debater with a caustic tongue} \colorBulletS{SYN} manner of speaking, way of speaking, manner of talking, way of talking, form of expression, mode of expression, choice of words, verbal expression \textbf{3} A strip of leather or fabric under the laces in a shoe, attached only at the front end. {\fontspec{DejaVu Sans}◇} \textit{} \textbf{4} The free{-}swinging metal piece inside a bell which is made to strike the bell to produce the sound. {\fontspec{DejaVu Sans}◇} \textit{} \textbf{5} A long, low promontory of land. {\fontspec{DejaVu Sans}◇} \textit{} \colorBulletS{SYN} promontory, headland, point, head, foreland, cape, peninsula, bluff, ness, naze, horn, spit, tongue \textbf{6} A projecting strip on a wooden board fitting into a groove on another. {\fontspec{DejaVu Sans}◇} \textit{} \textbf{7} The vibrating reed of a musical instrument or organ pipe. {\fontspec{DejaVu Sans}◇} \textit{} \textbf{8} A jet of flame. {\fontspec{DejaVu Sans}◇} \textit{a tongue of flame flashed from the gun} \\{\fontspec{DejaVu Sans}▪ }\textsf{\textit{verb}}\\ \textbf{1} Sound (a note) distinctly on a wind instrument by interrupting the air flow with the tongue. {\fontspec{DejaVu Sans}◇} \textit{Eugene has worked out the correct tonguing} \textbf{2} Lick or caress with the tongue. {\fontspec{DejaVu Sans}◇} \textit{the other horse tongued every part of the colt's mane}}{}{}{ \colorBullet{ORIGIN} Old English tunge, of Germanic origin; related to Dutch tong, German Zunge, and Latin lingua.}%
\par%
\entry{tough}{/tʌf/}{শক্ত}{\small{\textsf{\textit{adjective, noun, verb}}} \\{\fontspec{DejaVu Sans}▪ }\textsf{\textit{adjective}}\\ \textbf{1} (of a substance or object) strong enough to withstand adverse conditions or rough handling. {\fontspec{DejaVu Sans}◇} \textit{tough rucksacks for climbers} \colorBulletS{SYN} durable, strong, resilient, resistant, sturdy, rugged, firm, solid, substantial, sound, stout, indestructible, unbreakable, hard, rigid, stiff, inflexible, toughened \textbf{2} Able to endure hardship or pain. {\fontspec{DejaVu Sans}◇} \textit{she was as tough as old boots} \colorBulletS{SYN} resilient, strong, hardy, gritty, determined, resolute, dogged, stalwart \textbf{3} Demonstrating a strict and uncompromising approach. {\fontspec{DejaVu Sans}◇} \textit{police have been getting tough with drivers} \colorBulletS{SYN} strict, stern, severe, hard, harsh, firm, hard{-}hitting, adamant, inflexible, unyielding, unbending, uncompromising, unsentimental, unsympathetic \textbf{4} Strong and prone to violence. {\fontspec{DejaVu Sans}◇} \textit{tough young teenagers} \colorBulletS{SYN} rough, rowdy, unruly, disorderly, violent, wild, lawless, lawbreaking, criminal \\{\fontspec{DejaVu Sans}▪ }\textsf{\textit{noun}}\\ \textbf{1} A rough and violent man. {\fontspec{DejaVu Sans}◇} \textit{a gang of toughs} \colorBulletS{SYN} ruffian, rowdy, thug, hoodlum, hooligan, brute, bully, bully boy, rough, gangster, desperado \\{\fontspec{DejaVu Sans}▪ }\textsf{\textit{verb}}\\ \textbf{1} Endure a period of hardship or difficulty. {\fontspec{DejaVu Sans}◇} \textit{} \colorBulletS{SYN} put up with it, grin and bear it, keep at it, keep going, stay with it, see it through, see it through to the end}{}{}{ \colorBullet{ORIGIN} Old English tōh, of Germanic origin; related to Dutch taai and German zäh.}%
\par%
\entry{tout}{/taʊt/}{টাউট}{\small{\textsf{\textit{noun, verb}}} \\{\fontspec{DejaVu Sans}▪ }\textsf{\textit{noun}}\\ \textbf{1}  {\fontspec{DejaVu Sans}◇} \textit{} \colorBulletS{SYN} ticket tout, illegal salesman \textbf{2} A person who offers racing tips for a share of any resulting winnings. {\fontspec{DejaVu Sans}◇} \textit{} \textbf{3} An informer. {\fontspec{DejaVu Sans}◇} \textit{} \\{\fontspec{DejaVu Sans}▪ }\textsf{\textit{verb}}\\ \textbf{1} Attempt to sell (something), typically by a direct or persistent approach. {\fontspec{DejaVu Sans}◇} \textit{Sanjay was touting his wares} \textbf{2} Offer racing tips for a share of any resulting winnings. {\fontspec{DejaVu Sans}◇} \textit{}}{}{}{ \colorBullet{ORIGIN} Middle English tute ‘look out’, of Germanic origin; related to Dutch tuit ‘spout, nozzle’. Later senses were ‘watch, spy on’ (late 17th century) and ‘solicit custom’ (mid 18th century). The noun was first recorded (early 18th century) in the slang use ‘thieves' lookout’.}%
\par%
\entry{tout}{/taʊt/}{টাউট}{ \textsf{\textit{determiner}}\ \textbf{1} Used before the name of a city to refer to its high society or people of importance. {\fontspec{DejaVu Sans}◇} \textit{le tout Washington adored him}}{}{}{ \colorBullet{ORIGIN} French, suggested by le tout Paris ‘all (of) Paris’, used to refer to Parisian high society.}%
\par%
\entry{tow}{/təʊ/}{কাতা}{\small{\textsf{\textit{noun, verb}}} \\{\fontspec{DejaVu Sans}▪ }\textsf{\textit{noun}}\\ \textbf{1} An act of towing a vehicle or boat. {\fontspec{DejaVu Sans}◇} \textit{the cruiser got a tow from a warship after its engine failed} \colorBulletS{SYN} tug, towing, haul, pull, drawing, drag, trailing, trawl \\{\fontspec{DejaVu Sans}▪ }\textsf{\textit{verb}}\\ \textbf{1} (of a motor vehicle or boat) pull (another vehicle or boat) along with a rope, chain, or tow bar. {\fontspec{DejaVu Sans}◇} \textit{a pickup van towing a trailer} \colorBulletS{SYN} pull, draw, drag, haul, tug, trail, lug, heave, trawl, hoist, transport}{ \colorBullet{OTHER} towed away: দূরে টেনে }{1. The other ship which came under attack, the norwegian{-}operated front altair, was being towed away from iranian waters and would undergo a damage assessment later saturday, said a spokeswoman for its operator. 2. workers in paris and other cities swept up broken glass and towed away burnt{-}out cars while the government warned of slower economic growth and the judiciary said it would come down hard on looting and attacks on police.}{ \colorBullet{ORIGIN} Old English togian ‘draw, drag’, of Germanic origin; related to tug. The noun dates from the early 17th century.}%
\par%
\entry{tow}{/təʊ/}{কাতা}{ \textsf{\textit{noun}}\ \textbf{1} The coarse and broken part of flax or hemp prepared for spinning. {\fontspec{DejaVu Sans}◇} \textit{In this process, which is much faster than that using guillotine cutters, tow is dyed, finished, cut, dried, screened, and bagged in one continuous operation.}}{ \colorBullet{OTHER} towed away: দূরে টেনে }{1. The other ship which came under attack, the norwegian{-}operated front altair, was being towed away from iranian waters and would undergo a damage assessment later saturday, said a spokeswoman for its operator. 2. workers in paris and other cities swept up broken glass and towed away burnt{-}out cars while the government warned of slower economic growth and the judiciary said it would come down hard on looting and attacks on police.}{ \colorBullet{ORIGIN} Old English (recorded in towcræft ‘spinning’), of Germanic origin.}%
\par%
\entry{TOW}{/təʊ/}{কাতা}{ \textsf{\textit{abbreviation}}\ \textbf{1} Tube{-}launched, optically tracked, wire{-}guided (missile). {\fontspec{DejaVu Sans}◇} \textit{}}{ \colorBullet{OTHER} towed away: দূরে টেনে }{1. The other ship which came under attack, the norwegian{-}operated front altair, was being towed away from iranian waters and would undergo a damage assessment later saturday, said a spokeswoman for its operator. 2. workers in paris and other cities swept up broken glass and towed away burnt{-}out cars while the government warned of slower economic growth and the judiciary said it would come down hard on looting and attacks on police.}{}%
\par%
\entry{trace}{/treɪs/}{চিহ্ন}{\small{\textsf{\textit{noun, verb}}} \\{\fontspec{DejaVu Sans}▪ }\textsf{\textit{noun}}\\ \textbf{1} A mark, object, or other indication of the existence or passing of something. {\fontspec{DejaVu Sans}◇} \textit{remove all traces of the old adhesive} \colorBulletS{SYN} vestige, sign, mark, indication, suggestion, evidence, clue \textbf{2} A very small quantity, especially one too small to be accurately measured. {\fontspec{DejaVu Sans}◇} \textit{his body contained traces of amphetamines} \textbf{3} A procedure to investigate the source of something, such as the place from which a telephone call was made. {\fontspec{DejaVu Sans}◇} \textit{we've got a trace on the call} \textbf{4} A line which represents the projection of a curve or surface on a plane or the intersection of a curve or surface with a plane. {\fontspec{DejaVu Sans}◇} \textit{} \textbf{5} A path or track. {\fontspec{DejaVu Sans}◇} \textit{} \textbf{6} The sum of the elements in the principal diagonal of a square matrix. {\fontspec{DejaVu Sans}◇} \textit{} \\{\fontspec{DejaVu Sans}▪ }\textsf{\textit{verb}}\\ \textbf{1} Find or discover by investigation. {\fontspec{DejaVu Sans}◇} \textit{police are trying to trace a white van seen in the area} \colorBulletS{SYN} track down, find, discover, detect, unearth, uncover, turn up, hunt down, dig up, ferret out, run to ground \textbf{2} Copy (a drawing, map, or design) by drawing over its lines on a superimposed piece of transparent paper. {\fontspec{DejaVu Sans}◇} \textit{trace a map of the world on to a large piece of paper} \colorBulletS{SYN} copy, reproduce, go over, draw over, draw the lines of}{}{}{ \colorBullet{ORIGIN} Middle English (first recorded as a noun in the sense ‘path that someone or something takes’): from Old French trace (noun), tracier (verb), based on Latin tractus (see tract).}%
\par%
\entry{trace}{/treɪs/}{চিহ্ন}{ \textsf{\textit{noun}}\ \textbf{1} Each of the two side straps, chains, or ropes by which a horse is attached to a vehicle that it is pulling. {\fontspec{DejaVu Sans}◇} \textit{Ales broke off in mid{-}explanation to dive into the crowd, reappearing clasping a handkerchief waving teenage girl, and yoking her into the cart's rope traces.}}{}{}{ \colorBullet{ORIGIN} Middle English (denoting a pair of traces): from Old French trais, plural of trait (see trait).}%
\par%
\entry{traffic}{/ˈtrafɪk/}{পাচার}{\small{\textsf{\textit{noun, verb}}} \\{\fontspec{DejaVu Sans}▪ }\textsf{\textit{noun}}\\ \textbf{1} Vehicles moving on a public highway. {\fontspec{DejaVu Sans}◇} \textit{a stream of heavy traffic} \colorBulletS{SYN} vehicles, cars, lorries, trucks \textbf{2} The messages or signals transmitted through a communications system. {\fontspec{DejaVu Sans}◇} \textit{data traffic between remote workstations} \textbf{3} The action of dealing or trading in something illegal. {\fontspec{DejaVu Sans}◇} \textit{the traffic in stolen cattle} \colorBulletS{SYN} trade, trading, trafficking, dealing, commerce, business, peddling, buying and selling \textbf{4} Dealings or communication between people. {\fontspec{DejaVu Sans}◇} \textit{} \colorBulletS{SYN} dealings, association, contact, communication, connection, relations, intercourse \\{\fontspec{DejaVu Sans}▪ }\textsf{\textit{verb}}\\ \textbf{1} Deal or trade in something illegal. {\fontspec{DejaVu Sans}◇} \textit{the government will vigorously pursue individuals who traffic in drugs} \colorBulletS{SYN} trade, deal, do business, peddle, bargain}{}{}{ \colorBullet{ORIGIN} Early 16th century (denoting commercial transportation of merchandise or passengers): from French traffique, Spanish tráfico, or Italian traffico, of unknown origin. Sense 1 dates from the early 19th century.}%
\par%
\entry{trafficker}{/ˈtrafɪkə/}{কারবারী; পাচারকারী}{ \textsf{\textit{noun}}\ \textbf{1} A person who deals or trades in something illegal. {\fontspec{DejaVu Sans}◇} \textit{a convicted drug trafficker}}{}{}{}%
\par%
\entry{tragic}{/ˈtradʒɪk/}{মৃতু্যঘটিত}{\small{\textsf{\textit{adjective, noun}}} \\{\fontspec{DejaVu Sans}▪ }\textsf{\textit{adjective}}\\ \textbf{1} Causing or characterized by extreme distress or sorrow. {\fontspec{DejaVu Sans}◇} \textit{the shooting was a tragic accident} \colorBulletS{SYN} disastrous, calamitous, catastrophic, cataclysmic, devastating, terrible, dreadful, appalling, horrendous, dire, ruinous, gruesome, awful, miserable, wretched, unfortunate \textbf{2} Relating to tragedy in a literary work. {\fontspec{DejaVu Sans}◇} \textit{the same rules apply whether the plot is tragic or comic} \\{\fontspec{DejaVu Sans}▪ }\textsf{\textit{noun}}\\ \textbf{1} A boring or socially inept person, typically having an obsessive and solitary interest. {\fontspec{DejaVu Sans}◇} \textit{at school she's not a complete tragic, but she's not exactly popular either}}{}{}{ \colorBullet{ORIGIN} Mid 16th century from French tragique, via Latin from Greek tragikos, from tragos ‘goat’, but associated with tragōidia (see tragedy).}%
\par%
\entry{trail}{/treɪl/}{লেজ}{\small{\textsf{\textit{noun, verb}}} \\{\fontspec{DejaVu Sans}▪ }\textsf{\textit{noun}}\\ \textbf{1} A mark or a series of signs or objects left behind by the passage of someone or something. {\fontspec{DejaVu Sans}◇} \textit{a trail of blood on the grass} \colorBulletS{SYN} series, stream, string, line, chain, row, succession, train \textbf{2} A long thin part or line stretching behind or hanging down from something. {\fontspec{DejaVu Sans}◇} \textit{smoke trails} \colorBulletS{SYN} wake, tail, stream, slipstream \textbf{3} A beaten path through the countryside. {\fontspec{DejaVu Sans}◇} \textit{country parks with nature trails} \colorBulletS{SYN} path, beaten path, pathway, way, footpath, track, course, road, route \textbf{4} A trailer for a film or broadcast. {\fontspec{DejaVu Sans}◇} \textit{a recent television trail for ‘The Bill’} \textbf{5} The rear end of a gun carriage, resting or sliding on the ground when the gun is unlimbered. {\fontspec{DejaVu Sans}◇} \textit{} \\{\fontspec{DejaVu Sans}▪ }\textsf{\textit{verb}}\\ \textbf{1} Draw or be drawn along behind someone or something. {\fontspec{DejaVu Sans}◇} \textit{Alex trailed a hand through the clear water} \colorBulletS{SYN} drag, sweep, be drawn, draw, stream, dangle, hang, hang down, tow, droop \textbf{2} Walk or move slowly or wearily. {\fontspec{DejaVu Sans}◇} \textit{he baulked at the idea of trailing around the shops} \colorBulletS{SYN} trudge, plod, drag oneself, wander, amble, meander, drift \textbf{3} Follow (a person or animal) by using marks or scent left behind. {\fontspec{DejaVu Sans}◇} \textit{Sam suspected they were trailing him} \colorBulletS{SYN} follow, pursue, track, trace, shadow, stalk, dog, hound, spoor, hunt, hunt down, course, keep an eye on, keep in sight, run to earth, run to ground, run down \textbf{4} Be losing to an opponent in a game or contest. {\fontspec{DejaVu Sans}◇} \textit{the defending champions were trailing 10—5 at half{-}time} \colorBulletS{SYN} lose, be down, be behind, lag behind, fall behind, drop behind \textbf{5} Give advance publicity to (a film, broadcast, or proposal) {\fontspec{DejaVu Sans}◇} \textit{the bank's plans have been extensively trailed} \colorBulletS{SYN} advertise, publicize, announce, proclaim \textbf{6} Apply (slip) through a nozzle or spout to decorate ceramic ware. {\fontspec{DejaVu Sans}◇} \textit{}}{}{}{ \colorBullet{ORIGIN} Middle English (as a verb): from Old French traillier ‘to tow’, or Middle Low German treilen ‘haul a boat’, based on Latin tragula ‘dragnet’, from trahere ‘to pull’. Compare with trawl. The noun originally denoted the train of a robe, later generalized to denote something trailing.}%
\par%
\entry{trample}{/ˈtramp(ə)l/}{দৃঢ়ভাবে আচরণ করা}{\small{\textsf{\textit{noun, verb}}} \\{\fontspec{DejaVu Sans}▪ }\textsf{\textit{noun}}\\ \textbf{1} An act or the sound of trampling. {\fontspec{DejaVu Sans}◇} \textit{destruction's trample treads them down} \\{\fontspec{DejaVu Sans}▪ }\textsf{\textit{verb}}\\ \textbf{1} Tread on and crush. {\fontspec{DejaVu Sans}◇} \textit{the fence had been trampled down} \colorBulletS{SYN} tread, tramp, stamp, walk over}{}{}{ \colorBullet{ORIGIN} Late Middle English (in the sense ‘tread heavily’): frequentative of tramp.}%
\par%
\entry{trance}{/trɑːns/}{সমাধি}{\small{\textsf{\textit{noun, verb}}} \\{\fontspec{DejaVu Sans}▪ }\textsf{\textit{noun}}\\ \textbf{1} A half{-}conscious state characterized by an absence of response to external stimuli, typically as induced by hypnosis or entered by a medium. {\fontspec{DejaVu Sans}◇} \textit{she put him into a light trance} \colorBulletS{SYN} daze, stupor, haze, hypnotic state, half{-}conscious state, dream, daydream, reverie, brown study, suspended animation \\{\fontspec{DejaVu Sans}▪ }\textsf{\textit{verb}}\\ \textbf{1} Put into a trance. {\fontspec{DejaVu Sans}◇} \textit{she's been tranced and may need waking}}{}{}{ \colorBullet{ORIGIN} Middle English (originally as a verb in the sense ‘be in a trance’): from Old French transir ‘depart, fall into trance’, from Latin transire ‘go across’.}%
\par%
\entry{transfusion}{/ˌtransˈfjuːʒ(ə)n/}{পরিব্যাপ্তি}{ \textsf{\textit{noun}}\ \textbf{1} An act of transferring donated blood, blood products, or other fluid into the circulatory system of a person or animal. {\fontspec{DejaVu Sans}◇} \textit{major bleeding necessitating transfusions}}{}{}{ \colorBullet{ORIGIN} Late Middle English from Latin transfusio(n{-}), from the verb transfundere (see transfuse).}%
\par%
\entry{trash talk}{}{ফালতু কথা; in the course of a competitive situation putting down your opponent verbally or saying how good you think you are. 1) verbal abuse used during competition to upset the opposition.  2) to verbally abuse the opponent during competition. 3.  Disparaging, often insulting or vulgar speech about another person or group. – wikipedia.org}{\small{\textsf{\textit{}}}}{}{}{}%
\par%
\entry{tremendous}{/trɪˈmɛndəs/}{অসাধারণ}{ \textsf{\textit{adjective}}\ \textbf{1} Very great in amount, scale, or intensity. {\fontspec{DejaVu Sans}◇} \textit{Penny put in a tremendous amount of time} \colorBulletS{SYN} very great, huge, enormous, immense, colossal, massive, prodigious, stupendous, monumental, mammoth, vast, gigantic, giant, mighty, epic, monstrous, titanic, cosmic, towering, king{-}sized, king{-}size, gargantuan, Herculean, Brobdingnagian \textbf{2} Inspiring awe or dread. {\fontspec{DejaVu Sans}◇} \textit{}}{}{}{ \colorBullet{ORIGIN} Mid 17th century from Latin tremendus (gerundive of tremere ‘tremble’) + {-}ous.}%
\par%
\entry{triumph}{/ˈtrʌɪʌmf/}{জয়জয়কার}{\small{\textsf{\textit{noun, verb}}} \\{\fontspec{DejaVu Sans}▪ }\textsf{\textit{noun}}\\ \textbf{1} A great victory or achievement. {\fontspec{DejaVu Sans}◇} \textit{a garden built to celebrate Napoleon's many triumphs} \colorBulletS{SYN} victory, win, conquest, success \textbf{2} The processional entry of a victorious general into ancient Rome. {\fontspec{DejaVu Sans}◇} \textit{} \\{\fontspec{DejaVu Sans}▪ }\textsf{\textit{verb}}\\ \textbf{1} Achieve a victory; be successful. {\fontspec{DejaVu Sans}◇} \textit{they had no chance of triumphing over the Nationalists} \colorBulletS{SYN} win, succeed, be successful, come first, be the victor, be victorious, gain a victory, carry the day, carry all before one, prevail, take the crown, take the honours, take the prize, come out on top \textbf{2} (of a Roman general) ride into ancient Rome after a victory. {\fontspec{DejaVu Sans}◇} \textit{Caesar triumphed at Rome four times in the same month, with a few days between each triumph.}}{}{}{ \colorBullet{ORIGIN} Late Middle English from Old French triumphe (noun), from Latin triump(h)us, probably from Greek thriambos ‘hymn to Bacchus’. Current senses of the verb date from the early 16th century.}%
\par%
\entry{triumphant}{/trʌɪˈʌmf(ə)nt/}{জয়যুক্ত}{ \textsf{\textit{adjective}}\ \textbf{1} Having won a battle or contest; victorious. {\fontspec{DejaVu Sans}◇} \textit{two of their triumphant Cup team} \colorBulletS{SYN} victorious, successful, winning, prize{-}winning, conquering}{}{Kenya's world 800m record holder david rudisha on friday made a triumphant return to his hometown of kilgoris in western kenya where he was crowned a masai warrior.}{ \colorBullet{ORIGIN} Late Middle English (in the sense ‘victorious’): from Old French, or from Latin triumphant{-} ‘celebrating a triumph’, from the verb triumphare (see triumph).}%
\par%
\entry{troll}{/trɒl/}{দানব}{ \textsf{\textit{noun}}\ \textbf{1} (in folklore) an ugly creature depicted as either a giant or a dwarf. {\fontspec{DejaVu Sans}◇} \textit{} \colorBulletS{SYN} sprite, pixie, elf, imp, brownie, puck}{}{}{ \colorBullet{ORIGIN} Early 17th century from Old Norse and Swedish troll, Danish trold. The first English use is from Shetland; the term was adopted more widely into English in the mid 19th century.}%
\par%
\entry{troll}{/trəʊl/}{দানব}{\small{\textsf{\textit{noun, verb}}} \\{\fontspec{DejaVu Sans}▪ }\textsf{\textit{noun}}\\ \textbf{1} A person who makes a deliberately offensive or provocative online post. {\fontspec{DejaVu Sans}◇} \textit{one solution is to make a troll's postings invisible to the rest of community once they've been recognized} \textbf{2} A line or bait used in trolling for fish. {\fontspec{DejaVu Sans}◇} \textit{} \colorBulletS{SYN} lure, decoy, fly, troll, jig, plug, teaser \\{\fontspec{DejaVu Sans}▪ }\textsf{\textit{verb}}\\ \textbf{1} Make a deliberately offensive or provocative online post with the aim of upsetting someone or eliciting an angry response from them. {\fontspec{DejaVu Sans}◇} \textit{if people are obviously trolling then I'll delete your posts and do my best to ban you} \textbf{2} Carefully and systematically search an area for something. {\fontspec{DejaVu Sans}◇} \textit{a group of companies trolling for partnership opportunities} \textbf{3} Walk in a leisurely way; stroll. {\fontspec{DejaVu Sans}◇} \textit{we all trolled into town} \textbf{4} Sing (something) in a happy and carefree way. {\fontspec{DejaVu Sans}◇} \textit{he trolled a note or two} \colorBulletS{SYN} chant, intone, croon, carol, chorus, warble, trill, pipe, quaver}{}{}{ \colorBullet{ORIGIN} Late Middle English (in the sense ‘stroll, roll’): origin uncertain; compare with Old French troller ‘wander here and there (in search of game)’ and Middle High German trollen ‘stroll’. The computing senses (first recorded in 1992) are probably influenced by troll.}%
\par%
\entry{trombone}{/trɒmˈbəʊn/}{পিতলের বড় বাঁশি}{ \textsf{\textit{noun}}\ \textbf{1} A large brass wind instrument with straight tubing in three sections, ending in a bell over the player's left shoulder, different fundamental notes being made using a forward{-}pointing extendable slide. {\fontspec{DejaVu Sans}◇} \textit{}}{}{}{ \colorBullet{ORIGIN} Early 18th century from French or Italian, from Italian tromba ‘trumpet’.}%
\par%
\entry{tropical}{/ˈtrɒpɪk(ə)l/}{গ্রীষ্মপ্রধান}{ \textsf{\textit{adjective}}\ \textbf{1} Of, typical of, or peculiar to the tropics. {\fontspec{DejaVu Sans}◇} \textit{tropical countries} \textbf{2} Of or involving a trope; figurative. {\fontspec{DejaVu Sans}◇} \textit{} \colorBulletS{SYN} metaphorical, non{-}literal, symbolic, allegorical, representative, emblematic}{}{}{}%
\par%
\entry{troubled}{/ˈtrʌb(ə)ld/}{অস্থির}{ \textsf{\textit{adjective}}\ \textbf{1} Beset by problems or difficulties. {\fontspec{DejaVu Sans}◇} \textit{his troubled private life} \colorBulletS{SYN} difficult, problematic, full of problems, beset by problems, unsettled, hard, tough, stressful, dark}{}{Troubled water: a difficult or confusing situation}{ \colorBullet{ORIGIN} A difficult situation or time.}%
\par%
\entry{truce}{/truːs/}{সাময়িক যুদ্ধবিরতি}{ \textsf{\textit{noun}}\ \textbf{1} An agreement between enemies or opponents to stop fighting or arguing for a certain time. {\fontspec{DejaVu Sans}◇} \textit{the guerrillas called a three{-}day truce} \colorBulletS{SYN} ceasefire, armistice, suspension of hostilities, cessation of hostilities, peace}{}{}{ \colorBullet{ORIGIN} Middle English trewes, trues (plural), from Old English trēowa, plural of trēow ‘belief, trust’, of Germanic origin; related to Dutch trouw and German Treue, also to true.}%
\par%
\entry{trunk}{/trʌŋk/}{ট্রাঙ্ক}{ \textsf{\textit{noun}}\ \textbf{1} The main woody stem of a tree as distinct from its branches and roots. {\fontspec{DejaVu Sans}◇} \textit{} \colorBulletS{SYN} main stem, bole, stock \textbf{2} A person's or animal's body apart from the limbs and head. {\fontspec{DejaVu Sans}◇} \textit{} \colorBulletS{SYN} torso, body \textbf{3} The elongated, prehensile nose of an elephant. {\fontspec{DejaVu Sans}◇} \textit{} \colorBulletS{SYN} proboscis, nose, snout \textbf{4} A large box with a hinged lid for storing or transporting clothes and other articles. {\fontspec{DejaVu Sans}◇} \textit{} \colorBulletS{SYN} chest, box, storage box, crate, coffer}{}{}{ \colorBullet{ORIGIN} Late Middle English from Old French tronc, from Latin truncus.}%
\par%
\entry{turndown}{/ˈtəːndaʊn/}{প্রত্যাখ্যান করা}{\small{\textsf{\textit{adjective, noun}}} \\{\fontspec{DejaVu Sans}▪ }\textsf{\textit{adjective}}\\ \textbf{1} (of a collar) turned down. {\fontspec{DejaVu Sans}◇} \textit{You can wear three basic types of shirts with a tuxedo: wing collar, turndown collar and mandarin collar.} \\{\fontspec{DejaVu Sans}▪ }\textsf{\textit{noun}}\\ \textbf{1} A rejection or refusal. {\fontspec{DejaVu Sans}◇} \textit{no idea should meet a flat turndown if there's a chance of a pay{-}off} \colorBulletS{SYN} rejection, refusal, rebuff, dismissal, spurning, repudiation, repulse, turndown, discouragement \textbf{2} A decline in something; a downturn. {\fontspec{DejaVu Sans}◇} \textit{the company has suffered a dramatic turndown after a storm of bad publicity}}{}{No reason to turn it down}{}%
\par%
\end{multicols}%
\pagebreak%
\section*{U}%
\begin{multicols}{2}%
\entry{ulterior}{/ʌlˈtɪərɪə/}{ভবিষ্য}{ \textsf{\textit{adjective}}\ \textbf{1} Existing beyond what is obvious or admitted; intentionally hidden. {\fontspec{DejaVu Sans}◇} \textit{could there be an ulterior motive behind his request?} \colorBulletS{SYN} secondary, underlying, undisclosed, undivulged, unexpressed, unapparent, under wraps, unrevealed, concealed, hidden, covert, secret, personal, private, selfish}{}{}{ \colorBullet{ORIGIN} Mid 17th century from Latin, literally ‘further, more distant’.}%
\par%
\entry{ulterior motive}{}{অশুভ উদ্দেশ্য; when a person is trying has a hidden motive or hidden objective with another person secretely.}{\small{\textsf{\textit{}}}}{}{}{}%
\par%
\entry{umbrage}{/ˈʌmbrɪdʒ/}{অপমানবোধ}{ \textsf{\textit{noun}}\ \textbf{1} Offence or annoyance. {\fontspec{DejaVu Sans}◇} \textit{she took umbrage at his remarks} \colorBulletS{SYN} take offence, be offended, take exception, bridle, take something personally, be aggrieved, be affronted, take something amiss, be upset, be annoyed, be angry, be indignant, get one's hackles up, be put out, be insulted, be hurt, be wounded, be piqued, be resentful, be disgruntled, get into a huff, go into a huff, get huffy \textbf{2} Shade or shadow, especially as cast by trees. {\fontspec{DejaVu Sans}◇} \textit{} \colorBulletS{SYN} shade, shadowiness, darkness, gathering darkness, dimness, semi{-}darkness, twilight}{}{}{ \colorBullet{ORIGIN} Late Middle English (in umbrage (sense 2)): from Old French, from Latin umbra ‘shadow’. An early sense was ‘shadowy outline’, giving rise to ‘ground for suspicion’, whence the current notion of ‘offence’.}%
\par%
\entry{unabated}{/ʌnəˈbeɪtɪd/}{অখণ্ড, অপ্রতিহত}{ \textsf{\textit{adjective}}\ \textbf{1} Without any reduction in intensity or strength. {\fontspec{DejaVu Sans}◇} \textit{the storm was raging unabated} \colorBulletS{SYN} persistent, continuing, constant, continual, continuous, non{-}stop, lasting, never{-}ending, steady, uninterrupted, unabated, unabating, unbroken, interminable, incessant, unstoppable, unceasing, endless, unending, perpetual, unremitting, unrelenting, unrelieved, sustained}{}{Road collapse as illegal sand lifting goes unabated}{}%
\par%
\entry{unabridged}{/ʌnəˈbrɪdʒd/}{অসংক্ষেপিত}{ \textsf{\textit{adjective}}\ \textbf{1} (of a text) not cut or shortened; complete. {\fontspec{DejaVu Sans}◇} \textit{an unabridged edition} \colorBulletS{SYN} complete, entire, whole, intact, full{-}length, uncut, unshortened, unreduced, uncondensed, unexpurgated}{}{}{}%
\par%
\entry{undaunted}{/ʌnˈdɔːntɪd/}{অকুতোভয়}{ \textsf{\textit{adjective}}\ \textbf{1} Not intimidated or discouraged by difficulty, danger, or disappointment. {\fontspec{DejaVu Sans}◇} \textit{they were undaunted by the huge amount of work needed} \colorBulletS{SYN} unafraid, undismayed, unalarmed, unflinching, unshrinking, unabashed, unfaltering, unflagging, fearless, dauntless, intrepid, bold, valiant, brave, stout{-}hearted, lionhearted, courageous, heroic, gallant, doughty, plucky, game, mettlesome, gritty, steely, indomitable, resolute, determined, confident, audacious, daring, daredevil}{}{}{}%
\par%
\entry{undeniably}{/ʌndɪˈnʌɪəbli/}{অনস্বীকার্য}{ \textsf{\textit{adverb}}\ \textbf{1} Used to emphasize that something cannot be denied or disputed. {\fontspec{DejaVu Sans}◇} \textit{effective, responsive government undeniably benefits businesses}}{}{}{}%
\par%
\entry{undergo}{/ʌndəˈɡəʊ/}{মধ্য দিয়ে যাওয়া; সহ্য করা; বিশেষত চিকিৎসার মধ্যে দিয়ে যাওয়া}{ \textsf{\textit{verb}}\ \textbf{1} Experience or be subjected to (something, typically something unpleasant or arduous) {\fontspec{DejaVu Sans}◇} \textit{he underwent a life{-}saving brain operation} \colorBulletS{SYN} go through, experience, engage in, undertake, live through, face, encounter, submit to, be subjected to, come in for, receive, sustain, endure, brave, bear, tolerate, stand, withstand, put up with, weather, support, brook, suffer, cope with}{}{1. Quader to undergo surgery today 2. Pathao undergoing massive downsizing}{ \colorBullet{ORIGIN} Old English undergān ‘undermine’ (see under{-}, go).}%
\par%
\entry{unearth}{/ʌnˈəːθ/}{মৃত্তিকা খুঁড়িয়া তোলা}{ \textsf{\textit{verb}}\ \textbf{1} Find (something) in the ground by digging. {\fontspec{DejaVu Sans}◇} \textit{workmen unearthed an ancient artillery shell} \colorBulletS{SYN} dig up, excavate, exhume, disinter, bring to the surface, mine, quarry, pull out, root out, scoop out, disentomb, unbury \textbf{2} Drive (an animal, especially a fox) out of a hole or burrow. {\fontspec{DejaVu Sans}◇} \textit{}}{}{}{}%
\par%
\entry{unfold}{/ʌnˈfəʊld/}{বিছান}{ \textsf{\textit{verb}}\ \textbf{1} Open or spread out from a folded position. {\fontspec{DejaVu Sans}◇} \textit{he unfolded the map and laid it out on the table} \colorBulletS{SYN} open out, spread out, stretch out, flatten, straighten out, unfurl, unroll, unravel, uncoil, unwind, extend \textbf{2} (of events or information) gradually develop or be revealed. {\fontspec{DejaVu Sans}◇} \textit{there was a fascinating scene unfolding before me} \colorBulletS{SYN} develop, evolve, happen, take place, occur, transpire, unroll, emerge, grow, progress, mature, work out, untangle, bear fruit, blossom}{}{}{ \colorBullet{ORIGIN} Old English unfealdan(see un{-}, fold).}%
\par%
\entry{unify}{/ˈjuːnɪfʌɪ/}{ঐক্যসাধন করা}{ \textsf{\textit{verb}}\ \textbf{1} Make or become united, uniform, or whole. {\fontspec{DejaVu Sans}◇} \textit{the government hoped to centralize and unify the nation} \colorBulletS{SYN} unite, bring together, join, join together, merge, fuse, amalgamate, coalesce, combine, blend, mix, bind, link up, consolidate, integrate, marry, synthesize, federate, weld together}{}{}{ \colorBullet{ORIGIN} Early 16th century from French unifier or late Latin unificare ‘make into a whole’.}%
\par%
\entry{unlawful}{/ʌnˈlɔːfʊl/}{বেআইনী}{ \textsf{\textit{adjective}}\ \textbf{1} Not conforming to, permitted by, or recognized by law or rules. {\fontspec{DejaVu Sans}◇} \textit{the use of unlawful violence} \colorBulletS{SYN} illegal, illicit, lawbreaking, illegitimate, against the law}{}{}{ \colorBullet{ORIGIN} On the difference between unlawful and illegal, see illegal}%
\par%
\entry{unlikely}{/ʌnˈlʌɪkli/}{অসম্ভাব্য; ঘটার সম্ভাবনা নেই এমন}{ \textsf{\textit{adjective}}\ \textbf{1} Not likely to happen, be done, or be true; improbable. {\fontspec{DejaVu Sans}◇} \textit{an unlikely explanation} \colorBulletS{SYN} improbable, not likely, doubtful, dubious, unexpected, beyond belief, implausible}{}{}{}%
\par%
\entry{unravel}{/ʌnˈrav(ə)l/}{ভেঙে}{ \textsf{\textit{verb}}\ \textbf{1} Undo (twisted, knitted, or woven threads). {\fontspec{DejaVu Sans}◇} \textit{} \colorBulletS{SYN} untangle, disentangle, straighten out, separate out, unsnarl, unknot, unwind, untwist, undo, untie, unkink, unjumble \textbf{2} Investigate and solve or explain (something complicated or puzzling) {\fontspec{DejaVu Sans}◇} \textit{they were attempting to unravel the cause of death} \colorBulletS{SYN} solve, resolve, work out, clear up, puzzle out, find an answer to, get to the bottom of, explain, elucidate, fathom, decipher, decode, crack, penetrate, untangle, unfold, settle, reveal, clarify, sort out, make head or tail of}{}{}{}%
\par%
\entry{unrest}{/ʌnˈrɛst/}{অশান্তি}{ \textsf{\textit{noun}}\ \textbf{1} A state of dissatisfaction, disturbance, and agitation, typically involving public demonstrations or disorder. {\fontspec{DejaVu Sans}◇} \textit{years of industrial unrest} \colorBulletS{SYN} disruption, disturbance, agitation, upset, trouble, turmoil, tumult, disorder, chaos, anarchy, turbulence, uproar}{}{}{}%
\par%
\entry{unruly}{/ʌnˈruːli/}{অবশ}{ \textsf{\textit{adjective}}\ \textbf{1} Disorderly and disruptive and not amenable to discipline or control. {\fontspec{DejaVu Sans}◇} \textit{a group of unruly children} \colorBulletS{SYN} disorderly, rowdy, wild, unmanageable, uncontrollable, disobedient, disruptive, attention{-}seeking, undisciplined, troublemaking, rebellious, mutinous, anarchic, chaotic, lawless, insubordinate, defiant, wayward, wilful, headstrong, irrepressible, unrestrained, obstreperous, difficult, intractable, out of hand, refractory, recalcitrant}{}{}{ \colorBullet{ORIGIN} Late Middle English from un{-}‘not’ + archaic ruly ‘amenable to discipline or order’ (from rule).}%
\par%
\entry{untangle}{/ʌnˈtaŋɡ(ə)l/}{জটিলতামুক্ত করা}{ \textsf{\textit{verb}}\ \textbf{1} Free from a tangled or twisted state. {\fontspec{DejaVu Sans}◇} \textit{fishermen untangled their nets} \colorBulletS{SYN} disentangle, unravel, unsnarl, unjumble, straighten out, sort out, untwist, untwine, untie, unknot, undo}{}{}{}%
\par%
\entry{unveil}{/ʌnˈveɪl/}{প্রকটিত করা}{ \textsf{\textit{verb}}\ \textbf{1} Remove a veil or covering from, in particular uncover (a new monument or work of art) as part of a public ceremony. {\fontspec{DejaVu Sans}◇} \textit{the Princess unveiled a plaque}}{}{}{}%
\par%
\entry{uphill battle}{}{চড়াই যুদ্ধ; a very difficult struggle}{\small{\textsf{\textit{}}}}{}{Egypt faces uphill battle against corruption}{}%
\par%
\entry{uphold}{/ʌpˈhəʊld/}{সমর্থন করা}{ \textsf{\textit{verb}}\ \textbf{1} Confirm or support (something which has been questioned) {\fontspec{DejaVu Sans}◇} \textit{the court upheld his claim for damages} \colorBulletS{SYN} confirm, endorse, sustain, validate, ratify, verify, vindicate, justify, approve}{}{}{}%
\par%
\entry{upscale}{/ˈʌpskeɪl/}{}{\small{\textsf{\textit{adjective, adverb, verb}}} \\{\fontspec{DejaVu Sans}▪ }\textsf{\textit{adjective}}\\ \textbf{1} Relatively expensive and designed to appeal to affluent consumers; upmarket. {\fontspec{DejaVu Sans}◇} \textit{Hawaii's upscale boutique hotels} \colorBulletS{SYN} magnificent, imposing, impressive, awe{-}inspiring, splendid, resplendent, superb, striking, monumental, majestic, glorious \\{\fontspec{DejaVu Sans}▪ }\textsf{\textit{adverb}}\\ \textbf{1} Towards the more expensive or affluent sector of the market. {\fontspec{DejaVu Sans}◇} \textit{once known as the low{-}cost cousin of beef, fish has moved upscale} \\{\fontspec{DejaVu Sans}▪ }\textsf{\textit{verb}}\\ \textbf{1} Increase the size or improve the quality of. {\fontspec{DejaVu Sans}◇} \textit{he needs to extra funds to upscale the business} \textbf{2} Convert (an image or video) so that it displays correctly in a higher resolution format. {\fontspec{DejaVu Sans}◇} \textit{your HDTV will automatically upscale the content you watch}}{}{}{}%
\par%
\entry{urge}{/ˈəːdʒ/}{চালনা করা}{\small{\textsf{\textit{noun, verb}}} \\{\fontspec{DejaVu Sans}▪ }\textsf{\textit{noun}}\\ \textbf{1} A strong desire or impulse. {\fontspec{DejaVu Sans}◇} \textit{he felt the urge to giggle} \colorBulletS{SYN} desire, wish, need, impulse, compulsion, longing, yearning, hankering, craving, appetite, hunger, thirst, lust, fancy \\{\fontspec{DejaVu Sans}▪ }\textsf{\textit{verb}}\\ \textbf{1} Try earnestly or persistently to persuade (someone) to do something. {\fontspec{DejaVu Sans}◇} \textit{he urged her to come and stay with us}}{}{}{ \colorBullet{ORIGIN} Mid 16th century from Latin urgere ‘press, drive’.}%
\par%
\entry{urging}{/ˈəːdʒɪŋ/}{অনুরোধ}{ \textsf{\textit{noun}}\ \textbf{1} The action of urging someone to do something. {\fontspec{DejaVu Sans}◇} \textit{she bought a new one at Gregory's urging} \colorBulletS{SYN} demand, demands, call, calls, urging, insistence}{}{}{}%
\par%
\entry{usher}{/ˈʌʃə/}{উপস্থাপক}{\small{\textsf{\textit{noun, verb}}} \\{\fontspec{DejaVu Sans}▪ }\textsf{\textit{noun}}\\ \textbf{1} A person who shows people to their seats, especially in a cinema or theatre or at a wedding. {\fontspec{DejaVu Sans}◇} \textit{} \colorBulletS{SYN} attendant, escort, guide \textbf{2} An assistant teacher. {\fontspec{DejaVu Sans}◇} \textit{It was modest in size, with perhaps 40 pupils taught by one master, assisted by an usher, in the room above the guildhall, both of which survive and are still used by the school.} \\{\fontspec{DejaVu Sans}▪ }\textsf{\textit{verb}}\\ \textbf{1} Show or guide (someone) somewhere. {\fontspec{DejaVu Sans}◇} \textit{a waiter ushered me to a table} \colorBulletS{SYN} escort, accompany, help, assist, take, show, see, lead, show someone the way, lead the way, conduct, guide, steer, pilot, shepherd, convoy \textbf{2} Cause or mark the start of something new. {\fontspec{DejaVu Sans}◇} \textit{the railways ushered in an era of cheap mass travel} \colorBulletS{SYN} herald, mark the start of, signal, announce, give notice of, ring in, show in, set the scene for, pave the way for, clear the way for, open the way for, smooth the path of}{}{}{ \colorBullet{ORIGIN} Late Middle English (denoting a doorkeeper): from Anglo{-}Norman French usser, from medieval Latin ustiarius, from Latin ostiarius, from ostium ‘door’.}%
\par%
\entry{usurp}{/jʊˈzəːp/}{অন্যায়রূপে অধিকার করা}{ \textsf{\textit{verb}}\ \textbf{1} Take (a position of power or importance) illegally or by force. {\fontspec{DejaVu Sans}◇} \textit{Richard usurped the throne} \colorBulletS{SYN} seize, take over, expropriate, take possession of, take, appropriate, steal, wrest, arrogate, commandeer, annex, assume, lay claim to}{}{}{ \colorBullet{ORIGIN} Middle English (in the sense ‘appropriate a right wrongfully’): from Old French usurper, from Latin usurpare ‘seize for use’.}%
\par%
\entry{utero}{}{}{\small{\textsf{\textit{}}}}{}{}{}%
\par%
\entry{uvula}{/ˈjuːvjʊlə/}{আলজিভ}{ \textsf{\textit{noun}}\ \textbf{1}  {\fontspec{DejaVu Sans}◇} \textit{}}{}{}{ \colorBullet{ORIGIN} Late Middle English from late Latin, diminutive of Latin uva ‘grape’.}%
\par%
\end{multicols}%
\pagebreak%
\section*{V}%
\begin{multicols}{2}%
\entry{vaguely}{/ˈveɪɡli/}{অস্পষ্টভাবে}{ \textsf{\textit{adverb}}\ \textbf{1} In a way that is uncertain, indefinite or unclear; roughly. {\fontspec{DejaVu Sans}◇} \textit{he vaguely remembered talking to her once} \colorBulletS{SYN} roughly, more or less, approximately, nearly, just about, practically, virtually, as near as dammit, for all practical purposes, to all intents and purposes \textbf{2} Slightly. {\fontspec{DejaVu Sans}◇} \textit{he looked vaguely familiar} \colorBulletS{SYN} slightly, a little, a bit, somewhat, rather, moderately, to some degree, to a certain extent, in a way, to a slight extent, faintly, obscurely, dimly}{}{}{}%
\par%
\entry{vain}{/veɪn/}{নিরর্থক}{ \textsf{\textit{adjective}}\ \textbf{1} Having or showing an excessively high opinion of one's appearance, abilities, or worth. {\fontspec{DejaVu Sans}◇} \textit{their flattery made him vain} \colorBulletS{SYN} conceited, narcissistic, self{-}loving, in love with oneself, self{-}admiring, self{-}regarding, wrapped up in oneself, self{-}absorbed, self{-}obsessed, self{-}centred, egotistic, egotistical, egoistic, egocentric, egomaniac \textbf{2} Producing no result; useless. {\fontspec{DejaVu Sans}◇} \textit{a vain attempt to tidy up the room} \colorBulletS{SYN} futile, useless, pointless, worthless, nugatory, to no purpose, in vain}{}{}{ \colorBullet{ORIGIN} Middle English (in the sense ‘devoid of real worth’): via Old French from Latin vanus ‘empty, without substance’.}%
\par%
\entry{valiant}{/ˈvalɪənt/}{বীর}{ \textsf{\textit{adjective}}\ \textbf{1} Possessing or showing courage or determination. {\fontspec{DejaVu Sans}◇} \textit{she made a valiant effort to hold her anger in check} \colorBulletS{SYN} brave, fearless, courageous, valorous, plucky, intrepid, heroic, stout{-}hearted, lionhearted, manly, manful, bold, daring, audacious, gallant, confident, spirited, stout, undaunted, dauntless, doughty, mettlesome, unalarmed, unflinching, unshrinking, unblenching, unabashed, undismayed}{}{}{ \colorBullet{ORIGIN} Middle English (also in the sense ‘robust, well{-}built’): from Old French vailant, based on Latin valere ‘be strong’.}%
\par%
\entry{vandalism}{/ˈvand(ə)lɪz(ə)m/}{খেয়ালের বশে নির্বিচার ধ্বংসাত্মকতা; ধ্বংসোন্মাদনা}{ \textsf{\textit{noun}}\ \textbf{1} Action involving deliberate destruction of or damage to public or private property. {\fontspec{DejaVu Sans}◇} \textit{an act of mindless vandalism} \colorBulletS{SYN} harm, injury, destruction, vandalization, vandalism}{}{}{}%
\par%
\entry{veer}{/vɪə/}{ঢিলা করা}{\small{\textsf{\textit{noun, verb}}} \\{\fontspec{DejaVu Sans}▪ }\textsf{\textit{noun}}\\ \textbf{1} A sudden change of direction. {\fontspec{DejaVu Sans}◇} \textit{In particular, Sword wants to discover what triggers the insects' specific movements {-} a sudden veer or turn or an increase in speed, for example.} \textbf{2} An offensive play using a modified T{-}formation with a split backfield, which allows the quarterback the option of passing to the fullback, pitching to a running back, or running with the ball. {\fontspec{DejaVu Sans}◇} \textit{The veer offensive requires the quarterback to make the decision to run or hand off the ball even faster.} \\{\fontspec{DejaVu Sans}▪ }\textsf{\textit{verb}}\\ \textbf{1} Change direction suddenly. {\fontspec{DejaVu Sans}◇} \textit{an oil tanker that had veered off course} \colorBulletS{SYN} swerve, career, skew, swing, sheer, weave, wheel}{ \colorBullet{OTHER} veer off}{}{ \colorBullet{ORIGIN} Late 16th century from French virer, perhaps from an alteration of Latin gyrare (see gyrate).}%
\par%
\entry{veer}{/vɪə/}{ঢিলা করা}{ \textsf{\textit{verb}}\ \textbf{1} Slacken or let out (a rope or cable) in a controlled way. {\fontspec{DejaVu Sans}◇} \textit{}}{ \colorBullet{OTHER} veer off}{}{ \colorBullet{ORIGIN} Late Middle English from Middle Dutch vieren.}%
\par%
\entry{veil}{/veɪl/}{ঘোমটা}{\small{\textsf{\textit{noun, verb}}} \\{\fontspec{DejaVu Sans}▪ }\textsf{\textit{noun}}\\ \textbf{1} A piece of fine material worn by women to protect or conceal the face. {\fontspec{DejaVu Sans}◇} \textit{a white bridal veil} \colorBulletS{SYN} face covering, veiling \textbf{2} A membrane which is attached to the immature fruiting body of some toadstools and ruptures in the course of development, either (universal veil) enclosing the whole fruiting body or (partial veil) joining the edges of the cap to the stalk. {\fontspec{DejaVu Sans}◇} \textit{Extending from the stem to the margin of the cap, and covering the gills, is the partial veil {-} a membranaceous, white texture of varying thickness.} \\{\fontspec{DejaVu Sans}▪ }\textsf{\textit{verb}}\\ \textbf{1} Cover with or as if with a veil. {\fontspec{DejaVu Sans}◇} \textit{she veiled her face} \colorBulletS{SYN} envelop, surround, swathe, enfold, cover, cover up, conceal, hide, secrete, camouflage, disguise, mask, screen, shield, cloak, blanket, shroud, enwrap, canopy, overlay}{}{}{ \colorBullet{ORIGIN} Middle English from Anglo{-}Norman French veil(e), from Latin vela, plural of velum (see velum).}%
\par%
\entry{verbal}{/ˈvəːb(ə)l/}{মৌখিক}{\small{\textsf{\textit{adjective, noun, verb}}} \\{\fontspec{DejaVu Sans}▪ }\textsf{\textit{adjective}}\\ \textbf{1} Relating to or in the form of words. {\fontspec{DejaVu Sans}◇} \textit{the root of the problem is visual rather than verbal} \textbf{2} Relating to or derived from a verb. {\fontspec{DejaVu Sans}◇} \textit{a verbal adjective} \\{\fontspec{DejaVu Sans}▪ }\textsf{\textit{noun}}\\ \textbf{1} A word or words functioning as a verb. {\fontspec{DejaVu Sans}◇} \textit{} \textbf{2}  {\fontspec{DejaVu Sans}◇} \textit{just a bit of air{-}wave verbals} \colorBulletS{SYN} abuse, stream of abuse, torrent of abuse, teasing, hectoring, jeering, barracking, cursing, scolding, upbraiding, rebuke, reproval, castigation, revilement, vilification, vituperation, defamation, slander, flak \textbf{3} The lyrics of a song or the dialogue of a film. {\fontspec{DejaVu Sans}◇} \textit{it is the responsibility of the directors to do better with the verbals} \textbf{4} A verbal statement containing a damaging admission alleged to have been made to the police, and offered as evidence by the prosecution. {\fontspec{DejaVu Sans}◇} \textit{But the mischief that McKinney, after two decades of cases, the mischief was exactly the problem of verbals.} \\{\fontspec{DejaVu Sans}▪ }\textsf{\textit{verb}}\\ \textbf{1} Attribute a damaging statement to (a suspect), especially dishonestly. {\fontspec{DejaVu Sans}◇} \textit{}}{}{}{ \colorBullet{ORIGIN} Late 15th century (describing a person who deals with words rather than things): from French, or from late Latin verbalis, from verbum ‘word’ (see verb).}%
\par%
\entry{verbally}{/ˈvəːb(ə)li/}{শব্দগতভাবে}{ \textsf{\textit{adverb}}\ \textbf{1} By means of words. {\fontspec{DejaVu Sans}◇} \textit{she claimed to have been verbally abused} \textbf{2} With the function of a verb. {\fontspec{DejaVu Sans}◇} \textit{}}{}{}{}%
\par%
\entry{verge}{/vəːdʒ/}{কিনারা}{\small{\textsf{\textit{noun, verb}}} \\{\fontspec{DejaVu Sans}▪ }\textsf{\textit{noun}}\\ \textbf{1} An edge or border. {\fontspec{DejaVu Sans}◇} \textit{they came down to the verge of the lake} \colorBulletS{SYN} edge, border, margin, side, brink, rim, lip, limit, boundary, outskirts, perimeter, periphery, borderline, frontier \textbf{2} An extreme limit beyond which something specified will happen. {\fontspec{DejaVu Sans}◇} \textit{I was on the verge of tears} \colorBulletS{SYN} brink, threshold, edge, point, dawn \\{\fontspec{DejaVu Sans}▪ }\textsf{\textit{verb}}\\ \textbf{1} Be very close or similar to. {\fontspec{DejaVu Sans}◇} \textit{despair verging on the suicidal} \colorBulletS{SYN} tend towards, incline to, incline towards, border on, approach, near, come near, be close to, be near to, touch on, be tantamount to, be more or less, be not far from, approximate to, resemble, be similar to}{}{}{ \colorBullet{ORIGIN} Late Middle English via Old French from Latin virga ‘rod’. The current verb sense dates from the late 18th century.}%
\par%
\entry{verge}{/vəːdʒ/}{কিনারা}{ \textsf{\textit{noun}}\ \textbf{1} A wand or rod carried before a bishop or dean as an emblem of office. {\fontspec{DejaVu Sans}◇} \textit{‘I will carry on looking after the verges until they (the council) shoot me,’ he said.}}{}{}{ \colorBullet{ORIGIN} Late Middle English from Latin virga ‘rod’.}%
\par%
\entry{verge}{/vəːdʒ/}{কিনারা}{ \textsf{\textit{verb}}\ \textbf{1} Incline in a certain direction or towards a particular state. {\fontspec{DejaVu Sans}◇} \textit{his style verged into the art nouveau school}}{}{}{ \colorBullet{ORIGIN} Early 17th century (in the sense ‘descend to the horizon’): from Latin vergere ‘to bend, incline’.}%
\par%
\entry{vermicelli}{/ˌvəːmɪˈtʃɛli/}{ভার্মিসিলি; সেমাইবিশেষ}{ \textsf{\textit{plural noun}}\ \textbf{1} Pasta in the form of long slender threads. {\fontspec{DejaVu Sans}◇} \textit{} \textbf{2} Shreds of chocolate used to decorate cakes or other sweet foods. {\fontspec{DejaVu Sans}◇} \textit{}}{}{}{ \colorBullet{ORIGIN} Italian, plural of vermicello, diminutive of verme ‘worm’, from Latin vermis.}%
\par%
\entry{vessel}{/ˈvɛs(ə)l/}{বদনা}{ \textsf{\textit{noun}}\ \textbf{1} A ship or large boat. {\fontspec{DejaVu Sans}◇} \textit{} \colorBulletS{SYN} boat, sailing boat, ship, yacht, craft, watercraft \textbf{2} A hollow container, especially one used to hold liquid, such as a bowl or cask. {\fontspec{DejaVu Sans}◇} \textit{} \colorBulletS{SYN} container, receptacle, repository, holder, carrier \textbf{3} A duct or canal holding or conveying blood or other fluid. {\fontspec{DejaVu Sans}◇} \textit{} \colorBulletS{SYN} duct, tube, channel, passage, pipe}{}{}{ \colorBullet{ORIGIN} Middle English from Anglo{-}Norman French vessel(e), from late Latin vascellum, diminutive of vas ‘vessel’.}%
\par%
\entry{veteran}{/ˈvɛt(ə)r(ə)n/}{ঝানু}{ \textsf{\textit{noun}}\ \textbf{1} A person who has had long experience in a particular field. {\fontspec{DejaVu Sans}◇} \textit{a veteran of two world wars} \colorBulletS{SYN} retired soldier}{}{}{ \colorBullet{ORIGIN} Early 16th century from French vétéran or Latin veteranus, from vetus ‘old’.}%
\par%
\entry{vicious}{/ˈvɪʃəs/}{দুশ্চরিত্র}{ \textsf{\textit{adjective}}\ \textbf{1} Deliberately cruel or violent. {\fontspec{DejaVu Sans}◇} \textit{a vicious assault} \colorBulletS{SYN} brutal, ferocious, savage, violent, dangerous, ruthless, remorseless, merciless, heartless, callous, cruel, harsh, cold{-}blooded, inhuman, fierce, barbarous, barbaric, brutish, bestial, bloodthirsty, bloody, fiendish, sadistic, monstrous, villainous, murderous, homicidal, heinous, atrocious, diabolical, terrible, dreadful, awful, grim \textbf{2} Immoral. {\fontspec{DejaVu Sans}◇} \textit{every soul on earth, virtuous or vicious, shall perish} \colorBulletS{SYN} immoral, debauched, dissolute, abandoned, perverted, degenerate, profligate, degraded, wicked, sinful, vile, base, iniquitous, corrupt, corrupted, criminal, vicious, brutal, lewd, licentious, lascivious, lecherous, prurient, obscene, indecent, libertine \textbf{3} (of language or a line of reasoning) imperfect; defective. {\fontspec{DejaVu Sans}◇} \textit{}}{}{}{ \colorBullet{ORIGIN} Middle English (in the sense ‘characterized by immorality’): from Old French vicious or Latin vitiosus, from vitium ‘vice’.}%
\par%
\entry{vintage}{/ˈvɪntɪdʒ/}{মদ}{\small{\textsf{\textit{adjective, noun}}} \\{\fontspec{DejaVu Sans}▪ }\textsf{\textit{adjective}}\\ \textbf{1} Relating to or denoting wine of high quality. {\fontspec{DejaVu Sans}◇} \textit{vintage claret} \colorBulletS{SYN} high{-}quality, quality, prime, choice, select, superior, best \textbf{2} Denoting something from the past of high quality, especially something representing the best of its kind. {\fontspec{DejaVu Sans}◇} \textit{a vintage Sherlock Holmes adventure} \colorBulletS{SYN} high{-}quality, quality, prime, choice, select, superior, best \\{\fontspec{DejaVu Sans}▪ }\textsf{\textit{noun}}\\ \textbf{1} The year or place in which wine, especially wine of high quality, was produced. {\fontspec{DejaVu Sans}◇} \textit{1982 is one of the best vintages of the century} \colorBulletS{SYN} year \textbf{2} The time that something of quality was produced. {\fontspec{DejaVu Sans}◇} \textit{rifles of various sizes and vintages} \colorBulletS{SYN} period, era, epoch, time, origin}{}{}{ \colorBullet{ORIGIN} Late Middle English alteration (influenced by vintner) of earlier vendage, from Old French vendange, from Latin vindemia (from vinum ‘wine’ + demere ‘remove’).}%
\par%
\entry{vow}{/vaʊ/}{ব্রত}{\small{\textsf{\textit{noun, verb}}} \\{\fontspec{DejaVu Sans}▪ }\textsf{\textit{noun}}\\ \textbf{1} A solemn promise. {\fontspec{DejaVu Sans}◇} \textit{} \colorBulletS{SYN} oath, pledge, promise, bond, covenant, commitment, avowal, profession, sworn statement, affirmation, attestation, assurance, word, word of honour, guarantee \\{\fontspec{DejaVu Sans}▪ }\textsf{\textit{verb}}\\ \textbf{1} Solemnly promise to do a specified thing. {\fontspec{DejaVu Sans}◇} \textit{the rebels vowed to continue fighting} \colorBulletS{SYN} swear, state under oath, swear under oath, swear on the Bible, take an oath, pledge, promise, affirm, avow, undertake, give an undertaking, engage, commit, commit oneself, make a commitment, give one's word, give one's word of honour, give an assurance, guarantee \textbf{2} Dedicate to someone or something, especially a deity. {\fontspec{DejaVu Sans}◇} \textit{I vowed myself to this enterprise}}{}{}{ \colorBullet{ORIGIN} Middle English from Old French vou, from Latin votum (see vote); the verb is from Old French vouer.}%
\par%
\entry{vulnerable}{/ˈvʌln(ə)rəb(ə)l/}{জেয়}{ \textsf{\textit{adjective}}\ \textbf{1} Exposed to the possibility of being attacked or harmed, either physically or emotionally. {\fontspec{DejaVu Sans}◇} \textit{we were in a vulnerable position} \colorBulletS{SYN} in danger, in peril, in jeopardy, at risk, endangered, unsafe, unprotected, ill{-}protected, unguarded}{}{}{ \colorBullet{ORIGIN} Early 17th century from late Latin vulnerabilis, from Latin vulnerare ‘to wound’, from vulnus ‘wound’.}%
\par%
\end{multicols}%
\pagebreak%
\section*{W}%
\begin{multicols}{2}%
\entry{wade}{/weɪd/}{}{\small{\textsf{\textit{noun, verb}}} \\{\fontspec{DejaVu Sans}▪ }\textsf{\textit{noun}}\\ \textbf{1} An act of wading. {\fontspec{DejaVu Sans}◇} \textit{} \\{\fontspec{DejaVu Sans}▪ }\textsf{\textit{verb}}\\ \textbf{1} Walk with effort through water or another liquid or viscous substance. {\fontspec{DejaVu Sans}◇} \textit{he waded out to the boat} \colorBulletS{SYN} paddle, wallow, dabble, slop, squelch, trudge, plod \textbf{2} Intervene in (something) or attack (someone) vigorously or forcefully. {\fontspec{DejaVu Sans}◇} \textit{Seb waded into the melee and started to beat off the boys} \colorBulletS{SYN} attack, set upon, assault, launch oneself at, weigh into, fly at, let fly at, turn on, round on, lash out at, hit out at, fall on, jump at, jump on, lunge at, charge, rush, storm}{}{}{ \colorBullet{ORIGIN} Old English wadan ‘move onward’, also ‘penetrate’, from a Germanic word meaning ‘go (through)’, from an Indo{-}European root shared by Latin vadere ‘go’.}%
\par%
\entry{wage}{/weɪdʒ/}{মজুরি}{\small{\textsf{\textit{noun, verb}}} \\{\fontspec{DejaVu Sans}▪ }\textsf{\textit{noun}}\\ \textbf{1} A fixed regular payment earned for work or services, typically paid on a daily or weekly basis. {\fontspec{DejaVu Sans}◇} \textit{we were struggling to get better wages} \colorBulletS{SYN} pay, payment, remuneration, salary, emolument, stipend, fee, allowance, honorarium \\{\fontspec{DejaVu Sans}▪ }\textsf{\textit{verb}}\\ \textbf{1} Carry on (a war or campaign) {\fontspec{DejaVu Sans}◇} \textit{it is necessary to destroy their capacity to wage war} \colorBulletS{SYN} engage in, carry on, conduct, execute, pursue, undertake, prosecute, practise, proceed with, devote oneself to, go on with}{ \colorBullet{OTHER} wages}{}{ \colorBullet{ORIGIN} Middle English from Anglo{-}Norman French and Old Northern French, of Germanic origin; related to gage and wed.}%
\par%
\entry{waist}{/weɪst/}{কোমর}{ \textsf{\textit{noun}}\ \textbf{1} The part of the human body below the ribs and above the hips, often narrower than the areas above and below. {\fontspec{DejaVu Sans}◇} \textit{he put an arm around her waist} \textbf{2} A narrow part in the middle of something, such as a violin or hourglass. {\fontspec{DejaVu Sans}◇} \textit{Wellington, a long almond biscuit, rounded at each end with a narrower waist.} \textbf{3} A blouse or bodice. {\fontspec{DejaVu Sans}◇} \textit{}}{}{}{ \colorBullet{ORIGIN} Late Middle English apparently representing an Old English word from the Germanic root of wax.}%
\par%
\entry{waive}{/weɪv/}{পরিত্যাগ করা}{ \textsf{\textit{verb}}\ \textbf{1} Refrain from insisting on or using (a right or claim) {\fontspec{DejaVu Sans}◇} \textit{he will waive all rights to the money} \colorBulletS{SYN} relinquish, renounce, give up, abandon, reject, surrender, yield, cede, do without, dispense with, put aside, set aside, abdicate, abjure, sacrifice, refuse, turn down, spurn, sign away}{}{}{ \colorBullet{ORIGIN} Middle English (originally as a legal term relating to removal of the protection of the law): from an Anglo{-}Norman French variant of Old French gaiver ‘allow to become a waif, abandon’.}%
\par%
\entry{waiver}{/ˈweɪvə/}{অধিকার পরিত্যাগের ঘোষণা}{ \textsf{\textit{noun}}\ \textbf{1} An act or instance of waiving a right or claim. {\fontspec{DejaVu Sans}◇} \textit{their acquiescence could amount to a waiver} \colorBulletS{SYN} renunciation, surrender, repudiation, rejection, relinquishment, abdication, disavowal, refusal, disaffirmation, dispensation, abandonment, deferral}{}{}{}%
\par%
\entry{wank}{/waŋk/}{হস্তমৈথুন করা}{\small{\textsf{\textit{noun, verb}}} \\{\fontspec{DejaVu Sans}▪ }\textsf{\textit{noun}}\\ \textbf{1} An act of masturbating. {\fontspec{DejaVu Sans}◇} \textit{} \\{\fontspec{DejaVu Sans}▪ }\textsf{\textit{verb}}\\ \textbf{1} (typically used of a man) masturbate. {\fontspec{DejaVu Sans}◇} \textit{}}{}{}{ \colorBullet{ORIGIN} 1940s of unknown origin.}%
\par%
\entry{watchdog}{/ˈwɒtʃdɒɡ/}{রক্ষী কুকুর}{\small{\textsf{\textit{noun, verb}}} \\{\fontspec{DejaVu Sans}▪ }\textsf{\textit{noun}}\\ \textbf{1} A dog kept to guard private property. {\fontspec{DejaVu Sans}◇} \textit{} \colorBulletS{SYN} guard dog, house dog \\{\fontspec{DejaVu Sans}▪ }\textsf{\textit{verb}}\\ \textbf{1} Monitor (a person, activity, or situation) {\fontspec{DejaVu Sans}◇} \textit{how can we watchdog our investments?}}{}{}{}%
\par%
\entry{waterlog}{/ˈwɔːtəlɒɡ/}{জলাবদ্ধতা}{ \textsf{\textit{verb}}\ \textbf{1} Saturate with water; make (something) waterlogged. {\fontspec{DejaVu Sans}◇} \textit{the open roof allowed rain to waterlog the field}}{}{}{ \colorBullet{ORIGIN} Mid 18th century (originally in the sense ‘make (a ship) unmanageable by flooding’): from water + the verb log.}%
\par%
\entry{weave}{/wiːv/}{বুনা}{\small{\textsf{\textit{noun, verb}}} \\{\fontspec{DejaVu Sans}▪ }\textsf{\textit{noun}}\\ \textbf{1} A particular style or manner in which something is woven. {\fontspec{DejaVu Sans}◇} \textit{cloth of a very fine weave} \textbf{2} A hairstyle created by weaving pieces of real or artificial hair into a person's existing hair, typically in order to increase its length or thickness. {\fontspec{DejaVu Sans}◇} \textit{trailers show him with dyed blond hair and, in one scene, a flowing blond weave} \\{\fontspec{DejaVu Sans}▪ }\textsf{\textit{verb}}\\ \textbf{1} Form (fabric or a fabric item) by interlacing long threads passing in one direction with others at a right angle to them. {\fontspec{DejaVu Sans}◇} \textit{textiles woven from linen or wool} \colorBulletS{SYN} entwine, lace, work, twist, knit, interlace, intertwine, interwork, intertwist, interknit, twist together, criss{-}cross, braid, twine, plait \textbf{2} Make (a complex story or pattern) from a number of interconnected elements. {\fontspec{DejaVu Sans}◇} \textit{he weaves colourful, cinematic plots} \colorBulletS{SYN} invent, make up, fabricate, put together, construct, create, contrive, spin}{}{}{ \colorBullet{ORIGIN} Old English wefan, of Germanic origin, from an Indo{-}European root shared by Greek huphē ‘web’ and Sanskrit ūrṇavābhi ‘spider’, literally ‘wool{-}weaver’. The current noun sense dates from the late 19th century.}%
\par%
\entry{weave}{/wiːv/}{বুনা}{ \textsf{\textit{verb}}\ \textbf{1} Twist and turn from side to side while moving somewhere in order to avoid obstructions. {\fontspec{DejaVu Sans}◇} \textit{he had to weave his way through the crowds} \colorBulletS{SYN} thread, thread one's way, wind, wind one's way, work, work one's way, dodge, move in and out, swerve, zigzag, criss{-}cross}{}{}{ \colorBullet{ORIGIN} Late 16th century probably from Old Norse veifa ‘to wave, brandish’.}%
\par%
\entry{weaver}{/ˈwiːvə/}{তাঁতি}{ \textsf{\textit{noun}}\ \textbf{1} A person who weaves fabric. {\fontspec{DejaVu Sans}◇} \textit{} \textbf{2}  {\fontspec{DejaVu Sans}◇} \textit{}}{}{}{}%
\par%
\entry{weepy}{/ˈwiːpi/}{ক্রন্দনশীল}{\small{\textsf{\textit{adjective, noun}}} \\{\fontspec{DejaVu Sans}▪ }\textsf{\textit{adjective}}\\ \textbf{1} Tearful; inclined to weep. {\fontspec{DejaVu Sans}◇} \textit{seeing a bride always made her feel weepy} \colorBulletS{SYN} tearful, in tears, crying, weeping, sobbing, wailing, snivelling, whimpering \\{\fontspec{DejaVu Sans}▪ }\textsf{\textit{noun}}\\ \textbf{1} A sentimental film, book, or song. {\fontspec{DejaVu Sans}◇} \textit{}}{}{}{}%
\par%
\entry{weigh}{/weɪ/}{ওজন}{ \textsf{\textit{verb}}\ \textbf{1} Find out how heavy (someone or something) is, typically using scales. {\fontspec{DejaVu Sans}◇} \textit{weigh yourself on the day you begin the diet} \colorBulletS{SYN} measure the weight of, measure how heavy someone is, measure how heavy something is, put someone on the scales, put something on the scales \textbf{2} Assess the nature or importance of, especially with a view to a decision or action. {\fontspec{DejaVu Sans}◇} \textit{the consequences of the move would need to be very carefully weighed} \colorBulletS{SYN} consider, contemplate, think about, give thought to, entertain the idea of, deliberate about, turn over in one's mind, mull over, chew over, reflect on, ruminate about, muse on}{}{I would like to weigh in here}{ \colorBullet{ORIGIN} Old English wegan, of Germanic origin; related to Dutch wegen ‘weigh’, German bewegen ‘move’, from an Indo{-}European root shared by Latin vehere ‘convey’. Early senses included ‘transport from one place to another’ and ‘raise up’.}%
\par%
\entry{weigh}{/weɪ/}{ওজন}{\small{\textsf{\textit{}}}}{}{I would like to weigh in here}{ \colorBullet{ORIGIN} Late 18th century from an erroneous association with weigh anchor (see anchor).}%
\par%
\entry{weird}{/wɪəd/}{অদ্ভুত}{\small{\textsf{\textit{adjective, noun, verb}}} \\{\fontspec{DejaVu Sans}▪ }\textsf{\textit{adjective}}\\ \textbf{1} Suggesting something supernatural; unearthly. {\fontspec{DejaVu Sans}◇} \textit{weird, inhuman sounds} \colorBulletS{SYN} uncanny, eerie, unnatural, preternatural, supernatural, unearthly, other{-}worldly, unreal, ghostly, mysterious, mystifying, strange, abnormal, unusual \textbf{2} Connected with fate. {\fontspec{DejaVu Sans}◇} \textit{} \\{\fontspec{DejaVu Sans}▪ }\textsf{\textit{noun}}\\ \textbf{1} A person's destiny. {\fontspec{DejaVu Sans}◇} \textit{} \\{\fontspec{DejaVu Sans}▪ }\textsf{\textit{verb}}\\ \textbf{1} Induce a sense of disbelief or alienation in someone. {\fontspec{DejaVu Sans}◇} \textit{blue eyes weirded him out, and Ivan's were especially creepy}}{}{}{ \colorBullet{ORIGIN} Old English wyrd ‘destiny’, of Germanic origin. The adjective (late Middle English) originally meant ‘having the power to control destiny’, and was used especially in the Weird Sisters, originally referring to the Fates, later the witches in Shakespeare's Macbeth; the latter use gave rise to the sense ‘unearthly’ (early 19th century).}%
\par%
\entry{well, look, who it is.}{}{}{\small{\textsf{\textit{}}}}{}{}{}%
\par%
\entry{well{-}being}{/wɛlˈbiːɪŋ/}{মঙ্গল}{ \textsf{\textit{noun}}\ \textbf{1} The state of being comfortable, healthy, or happy. {\fontspec{DejaVu Sans}◇} \textit{an improvement in the patient's well{-}being} \colorBulletS{SYN} welfare, health, good health, happiness, comfort, security, safety, protection, prosperity, profit, good, success, fortune, good fortune, advantage, interest, prosperousness, successfulness}{}{}{}%
\par%
\entry{whisper}{/ˈwɪspə/}{ফিস্ ফিস্ শব্দ}{\small{\textsf{\textit{noun, verb}}} \\{\fontspec{DejaVu Sans}▪ }\textsf{\textit{noun}}\\ \textbf{1} A soft or confidential tone of voice; a whispered word or phrase. {\fontspec{DejaVu Sans}◇} \textit{she spoke in a whisper} \colorBulletS{SYN} murmur, mutter, mumble, low voice, hushed tone, undertone \\{\fontspec{DejaVu Sans}▪ }\textsf{\textit{verb}}\\ \textbf{1} Speak very softly using one's breath rather than one's throat, especially for the sake of secrecy. {\fontspec{DejaVu Sans}◇} \textit{Alison was whispering in his ear} \colorBulletS{SYN} murmur, mutter, mumble, say softly, speak softly, say in muted tones, speak in muted tones, say in hushed tones, speak in hushed tones, say sotto voce, speak sotto voce}{}{}{ \colorBullet{ORIGIN} Old English hwisprian, of Germanic origin; related to German wispeln, from the imitative base of whistle.}%
\par%
\entry{whopping}{/ˈwɒpɪŋ/}{খুব বড়}{ \textsf{\textit{adjective}}\ \textbf{1} Very large. {\fontspec{DejaVu Sans}◇} \textit{a whopping £74 million loss} \colorBulletS{SYN} huge, massive, enormous, gigantic, very big, very large, great, giant, colossal, mammoth, vast, immense, tremendous, mighty, stupendous, monumental, epic, prodigious, mountainous, monstrous, titanic, towering, elephantine, king{-}sized, king{-}size, gargantuan, Herculean, Brobdingnagian, substantial, extensive, hefty, bulky, weighty, heavy, gross}{}{}{}%
\par%
\entry{whore}{/hɔː/}{বেশ্যা}{\small{\textsf{\textit{noun, verb}}} \\{\fontspec{DejaVu Sans}▪ }\textsf{\textit{noun}}\\ \textbf{1} A prostitute. {\fontspec{DejaVu Sans}◇} \textit{} \colorBulletS{SYN} prostitute, promiscuous woman, sex worker, call girl \\{\fontspec{DejaVu Sans}▪ }\textsf{\textit{verb}}\\ \textbf{1} (of a woman) work as a prostitute. {\fontspec{DejaVu Sans}◇} \textit{she was forced to whore in order to support herself} \colorBulletS{SYN} work as a prostitute, prostitute oneself, sell one's body, sell oneself, walk the streets, be on the streets, solicit, work in the sex industry}{}{}{ \colorBullet{ORIGIN} Late Old English hōre, of Germanic origin; related to Dutch hoer and German Hure, from an Indo{-}European root shared by Latin carus ‘dear’.}%
\par%
\entry{wickedness}{/ˈwɪkɪdnəs/}{পাপা}{ \textsf{\textit{noun}}\ \textbf{1} The quality of being evil or morally wrong. {\fontspec{DejaVu Sans}◇} \textit{the wickedness of the regime} \colorBulletS{SYN} evil{-}doing, evil, evilness, sin, sinfulness, iniquity, iniquitousness, vileness, foulness, baseness, badness, wrong, wrongdoing, dishonesty, double{-}dealing, unscrupulousness, roguery, villainy, rascality, delinquency, viciousness, degeneracy, depravity, dissolution, dissipation, immorality, vice, perversion, pervertedness, corruption, corruptness, turpitude, devilry, devilishness, fiendishness}{}{}{}%
\par%
\entry{wig}{/wɪɡ/}{পরচুলা}{ \textsf{\textit{noun}}\ \textbf{1} A covering for the head made of real or artificial hair, typically worn by judges and barristers in law courts or by people trying to conceal their baldness. {\fontspec{DejaVu Sans}◇} \textit{} \colorBulletS{SYN} head of hair, shock of hair, mop of hair, mane}{}{}{ \colorBullet{ORIGIN} Late 17th century shortening of periwig.}%
\par%
\entry{wig}{/wɪɡ/}{পরচুলা}{ \textsf{\textit{verb}}\ \textbf{1} Rebuke (someone) severely. {\fontspec{DejaVu Sans}◇} \textit{I had often occasion to wig him for getting drunk} \colorBulletS{SYN} scold, chastise, upbraid, berate, castigate, lambaste, rebuke, reprimand, reproach, reprove, admonish, remonstrate with, lecture, criticize, censure}{}{}{ \colorBullet{ORIGIN} Early 19th century apparently from wig, perhaps from bigwig and associated with a rebuke given by a person in authority.}%
\par%
\entry{wildebeest}{/ˈwɪldəbiːst/}{নু{-}হরিণ}{\small{\textsf{\textit{}}}}{}{}{ \colorBullet{ORIGIN} Early 19th century from Afrikaans, literally ‘wild beast’.}%
\par%
\entry{willful}{/ˈwilfəl/}{স্বেচ্ছাচারী}{ \textsf{\textit{adjective}}\ \textbf{1} (of an immoral or illegal act or omission) intentional; deliberate. {\fontspec{DejaVu Sans}◇} \textit{willful acts of damage} \colorBulletS{SYN} deliberate, intentional, intended, done on purpose, premeditated, planned, calculated, purposeful, conscious, knowing}{}{}{ \colorBullet{ORIGIN} Middle English from the noun will+ {-}ful.}%
\par%
\entry{wind up}{}{গুটান; 1. verb to tighten the spring inside an item or device, as by twisting a knob. A noun or pronoun can be used between "wind" and "up." 2. verb to twist or coil something onto a particular surface or thing. A noun or pronoun can be used between "wind" and "up." 3. verb to cause someone or something to become more animated. A noun or pronoun can be used between "wind" and "up."}{\small{\textsf{\textit{}}}}{}{1. Let me try winding up your watch—maybe that will get it going again. 2. The cat will keep playing with that yarn, unless you wind it up on the spool. 3. Please don't wind the kids up right before bedtime.}{}%
\par%
\entry{wipe}{/wʌɪp/}{মুছা}{\small{\textsf{\textit{noun, verb}}} \\{\fontspec{DejaVu Sans}▪ }\textsf{\textit{noun}}\\ \textbf{1} An act of wiping. {\fontspec{DejaVu Sans}◇} \textit{Bert was giving the machine a final wipe over with an oily rag} \colorBulletS{SYN} rub, clean, mop, sponge, swab, polish \textbf{2} A disposable cloth treated with a cleansing agent, for wiping things clean. {\fontspec{DejaVu Sans}◇} \textit{} \textbf{3} A cinematographic effect in which an existing picture seems to be wiped out by a new one as the boundary between them moves across the screen. {\fontspec{DejaVu Sans}◇} \textit{} \\{\fontspec{DejaVu Sans}▪ }\textsf{\textit{verb}}\\ \textbf{1} Clean or dry (something) by rubbing with a cloth, a piece of paper, or one's hand. {\fontspec{DejaVu Sans}◇} \textit{Paulie wiped his face with a handkerchief} \colorBulletS{SYN} rub, clean, mop, sponge, swab \textbf{2} Remove or eliminate (something) completely. {\fontspec{DejaVu Sans}◇} \textit{their life savings were wiped out} \colorBulletS{SYN} obliterate, expunge, erase, blot out, remove, remove all traces of, blank out \textbf{3} Pass (a swipe card) over an electronic reader. {\fontspec{DejaVu Sans}◇} \textit{a customer wipes the card across the reader and enters his/her identification number}}{ \colorBullet{OTHER} wipe out}{}{ \colorBullet{ORIGIN} Old English wīpian, of Germanic origin; related to whip.}%
\par%
\entry{wisdom}{/ˈwɪzdəm/}{জ্ঞান}{ \textsf{\textit{noun}}\ \textbf{1} The quality of having experience, knowledge, and good judgement; the quality of being wise. {\fontspec{DejaVu Sans}◇} \textit{listen to his words of wisdom} \colorBulletS{SYN} sagacity, sageness, intelligence, understanding, insight, perception, perceptiveness, percipience, penetration, perspicuity, acuity, discernment, sense, good sense, common sense, shrewdness, astuteness, acumen, smartness, judiciousness, judgement, foresight, clear{-}sightedness, prudence, circumspection}{}{}{ \colorBullet{ORIGIN} Old English wīsdōm(see wise, {-}dom).}%
\par%
\entry{woe}{/wəʊ/}{দুর্ভাগ্য}{ \textsf{\textit{noun}}\ \textbf{1} Great sorrow or distress (often used hyperbolically) {\fontspec{DejaVu Sans}◇} \textit{the Everton tale of woe continued} \colorBulletS{SYN} misery, sorrow, distress, wretchedness, sadness, unhappiness, heartache, heartbreak, despondency, desolation, despair, dejection, depression, gloom, melancholy}{}{}{ \colorBullet{ORIGIN} Natural exclamation of lament: recorded as wā in Old English and found in several Germanic languages.}%
\par%
\entry{woeful}{/ˈwəʊfʊl/}{শোচনীয়}{ \textsf{\textit{adjective}}\ \textbf{1} Characterized by, expressive of, or causing sorrow or misery. {\fontspec{DejaVu Sans}◇} \textit{her face was woeful} \colorBulletS{SYN} sad, unhappy, miserable, woebegone, doleful, forlorn, crestfallen, glum, gloomy, dejected, downcast, disconsolate, downhearted, despondent, depressed, despairing, dismal, melancholy, broken{-}hearted, heartbroken, inconsolable, grief{-}stricken \textbf{2} Very bad; deplorable. {\fontspec{DejaVu Sans}◇} \textit{the remark was enough to establish his woeful ignorance about the theatre} \colorBulletS{SYN} dreadful, very bad, awful, terrible, frightful, atrocious, disgraceful, deplorable, shameful, hopeless, lamentable, laughable, substandard, poor, inadequate, inferior, unsatisfactory}{}{}{}%
\par%
\entry{woo}{/wuː/}{পাণিপ্রার্থনা করা}{ \textsf{\textit{verb}}\ \textbf{1} Seek the favour, support, or custom of. {\fontspec{DejaVu Sans}◇} \textit{pop stars are being wooed by film companies eager to sign them up} \colorBulletS{SYN} seek the support of, seek the favour of, try to win, try to attract, try to cultivate, chase, pursue, try to ingratiate oneself with, curry favour with \textbf{2} Try to gain the love of (someone), especially with a view to marriage. {\fontspec{DejaVu Sans}◇} \textit{he wooed her with quotes from Shakespeare} \colorBulletS{SYN} court, pay court to, pursue, chase, chase after, run after}{}{}{ \colorBullet{ORIGIN} Late Old English wōgian (intransitive), āwōgian (transitive), of unknown origin.}%
\par%
\entry{woo}{/wuː/}{পাণিপ্রার্থনা করা}{ \textsf{\textit{noun \& adjective}}\ \textbf{1} variant form of woo{-}woo {\fontspec{DejaVu Sans}◇} \textit{}}{}{}{}%
\par%
\entry{worrisome}{/ˈwʌrɪs(ə)m/}{ঝামেলাপূর্ণ}{ \textsf{\textit{adjective}}\ \textbf{1} Causing anxiety or concern. {\fontspec{DejaVu Sans}◇} \textit{a worrisome problem} \colorBulletS{SYN} worrying, daunting, alarming, perturbing, trying, taxing, vexatious, niggling, bothersome, troublesome, unsettling, harassing, harrying, harrowing, nerve{-}racking, distressing, dismaying, disquieting, upsetting, traumatic, unpleasant, awkward, difficult, tricky, thorny, problematic, grave}{}{}{}%
\par%
\entry{worth}{/wəːθ/}{মূল্য}{\small{\textsf{\textit{adjective, noun}}} \\{\fontspec{DejaVu Sans}▪ }\textsf{\textit{adjective}}\\ \textbf{1} Equivalent in value to the sum or item specified. {\fontspec{DejaVu Sans}◇} \textit{jewellery worth £450 was taken} \textbf{2} Sufficiently good, important, or interesting to be treated or regarded in the way specified. {\fontspec{DejaVu Sans}◇} \textit{the museums in the district are well worth a visit} \\{\fontspec{DejaVu Sans}▪ }\textsf{\textit{noun}}\\ \textbf{1} The level at which someone or something deserves to be valued or rated. {\fontspec{DejaVu Sans}◇} \textit{they had to listen to every piece of gossip and judge its worth} \textbf{2} The amount that could be achieved or produced in a specified time. {\fontspec{DejaVu Sans}◇} \textit{the companies have debts greater than two years' worth of their sales}}{}{}{ \colorBullet{ORIGIN} Old English w(e)orth (adjective and noun), of Germanic origin; related to Dutch waard and German wert.}%
\par%
\entry{wrath}{/rɒθ/}{ক্রোধ}{ \textsf{\textit{noun}}\ \textbf{1} Extreme anger. {\fontspec{DejaVu Sans}◇} \textit{he hid his pipe for fear of incurring his father's wrath} \colorBulletS{SYN} anger, rage, fury, annoyance, indignation, outrage, pique, spleen, chagrin, vexation, exasperation, dudgeon, high dudgeon, hot temper, bad temper, bad mood, ill humour, irritation, irritability, crossness, displeasure, discontentment, disgruntlement, irascibility, cantankerousness, peevishness, querulousness, crabbiness, testiness, tetchiness, snappishness}{}{}{ \colorBullet{ORIGIN} Old English wrǣththu, from wrāth (see wroth).}%
\par%
\entry{wreak}{/riːk/}{প্রতিহিংসা গ্রহণ করা}{ \textsf{\textit{verb}}\ \textbf{1} Cause (a large amount of damage or harm) {\fontspec{DejaVu Sans}◇} \textit{torrential rainstorms wreaked havoc yesterday} \colorBulletS{SYN} inflict, create, cause, result in, effect, engender, bring about, perpetrate, unleash, vent, bestow, deal out, mete out, serve out, administer, carry out, deliver, apply, lay on, impose, exact}{}{1. flood wreaks havoc on croplands. 2. wreck it ralph }{ \colorBullet{ORIGIN} Old English wrecan ‘drive (out), avenge’, of Germanic origin; related to Dutch wreken and German rächen; compare with wrack, wreck, and wretch.}%
\par%
\entry{wreckage}{/ˈrɛkɪdʒ/}{ধ্বংসাবশেষ}{ \textsf{\textit{noun}}\ \textbf{1} The remains of something that has been badly damaged or destroyed. {\fontspec{DejaVu Sans}◇} \textit{firemen had to cut him free from the wreckage of the car} \colorBulletS{SYN} wreck, debris, detritus, remainder}{}{}{}%
\par%
\entry{writ}{/rɪt/}{লেখন}{ \textsf{\textit{noun}}\ \textbf{1} A form of written command in the name of a court or other legal authority to act, or abstain from acting, in a particular way. {\fontspec{DejaVu Sans}◇} \textit{the two reinstated officers issued a writ for libel against the applicants} \colorBulletS{SYN} summons, subpoena, warrant, arraignment, indictment, court order, process, decree \textbf{2} A piece or body of writing. {\fontspec{DejaVu Sans}◇} \textit{And Percivale took it, and found therein a writ and so he read it, and devised the manner of the spindles and of the ship, whence it came, and by whom it was made.}}{}{}{ \colorBullet{ORIGIN} Old English, as a general term denoting written matter, from the Germanic base of write.}%
\par%
\entry{writ}{/rɪt/}{লেখন}{\small{\textsf{\textit{}}}}{}{}{ \colorBullet{ORIGIN} 1Clear and obvious.}%
\par%
\end{multicols}%
\pagebreak%
\section*{X}%
\begin{multicols}{2}%
\entry{xenophobic}{/zɛnəˈfəʊbɪk/}{}{ \textsf{\textit{adjective}}\ \textbf{1} Having or showing a dislike of or prejudice against people from other countries. {\fontspec{DejaVu Sans}◇} \textit{xenophobic attitudes}}{}{}{}%
\par%
\end{multicols}%
\pagebreak%
\section*{Y}%
\begin{multicols}{2}%
\entry{yam}{/jam/}{রাঙা আলু}{ \textsf{\textit{noun}}\ \textbf{1} The edible starchy tuber of a climbing plant that is widely grown in tropical and subtropical countries. {\fontspec{DejaVu Sans}◇} \textit{} \textbf{2} The cultivated plant that yields the yam. {\fontspec{DejaVu Sans}◇} \textit{} \textbf{3} A sweet potato. {\fontspec{DejaVu Sans}◇} \textit{}}{}{}{ \colorBullet{ORIGIN} Late 16th century from Portuguese inhame or obsolete Spanish iñame, probably of West African origin.}%
\par%
\entry{yam}{/jam/}{রাঙা আলু}{ \textsf{\textit{verb}}\ \textbf{1} (of a cat) miaow. {\fontspec{DejaVu Sans}◇} \textit{a cat slips up the driveway, yamming and trying to talk}}{}{}{}%
\par%
\entry{yarn}{/jɑːn/}{সুতা}{\small{\textsf{\textit{noun, verb}}} \\{\fontspec{DejaVu Sans}▪ }\textsf{\textit{noun}}\\ \textbf{1} Spun thread used for knitting, weaving, or sewing. {\fontspec{DejaVu Sans}◇} \textit{hanks of pale green yarn} \colorBulletS{SYN} thread, cotton, wool, fibre, filament, strand \textbf{2} A long or rambling story, especially one that is implausible. {\fontspec{DejaVu Sans}◇} \textit{he never let reality get in the way of a good yarn} \colorBulletS{SYN} story, tale, anecdote, fable, parable, traveller's tale, fairy story, rigmarole, saga, sketch, narrative, reminiscence, account, report, history \\{\fontspec{DejaVu Sans}▪ }\textsf{\textit{verb}}\\ \textbf{1} Tell a long or implausible story. {\fontspec{DejaVu Sans}◇} \textit{they were yarning about local legends and superstitions}}{}{}{ \colorBullet{ORIGIN} Old English gearn; of Germanic origin, related to Dutch garen.}%
\par%
\entry{yawn}{/jɔːn/}{হাই তোলা}{\small{\textsf{\textit{noun, verb}}} \\{\fontspec{DejaVu Sans}▪ }\textsf{\textit{noun}}\\ \textbf{1} A reflex act of opening one's mouth wide and inhaling deeply due to tiredness or boredom. {\fontspec{DejaVu Sans}◇} \textit{he stretches and stifles a yawn} \\{\fontspec{DejaVu Sans}▪ }\textsf{\textit{verb}}\\ \textbf{1} Involuntarily open one's mouth wide and inhale deeply due to tiredness or boredom. {\fontspec{DejaVu Sans}◇} \textit{he began yawning and looking at his watch} \colorBulletS{SYN} gaping, wide open, wide, cavernous, deep \textbf{2} Be wide open. {\fontspec{DejaVu Sans}◇} \textit{a yawning chasm} \colorBulletS{SYN} gaping, wide open, wide, cavernous, deep}{}{}{ \colorBullet{ORIGIN} Old English geonian, of Germanic origin, from an Indo{-}European root shared by Latin hiare and Greek khainein. Current noun senses date from the early 18th century.}%
\par%
\entry{yell}{/jɛl/}{চিৎকার}{\small{\textsf{\textit{noun, verb}}} \\{\fontspec{DejaVu Sans}▪ }\textsf{\textit{noun}}\\ \textbf{1} A loud, sharp cry of pain, surprise, or delight. {\fontspec{DejaVu Sans}◇} \textit{her foot slipped and she gave a yell of fear} \colorBulletS{SYN} cry, yelp, call, shout, howl, yowl, wail, scream, shriek, screech, squawk, squeal \textbf{2} An extremely amusing person or thing. {\fontspec{DejaVu Sans}◇} \textit{} \colorBulletS{SYN} laugh \\{\fontspec{DejaVu Sans}▪ }\textsf{\textit{verb}}\\ \textbf{1} Shout in a loud, sharp way. {\fontspec{DejaVu Sans}◇} \textit{you heard me losing my temper and yelling at her} \colorBulletS{SYN} cry out, call out, call at the top of one's voice, yelp, shout, howl, yowl, wail, scream, shriek, screech, squawk, squeal}{}{What to do when your boss is yelling at you\newline%
}{ \colorBullet{ORIGIN} Old English g(i)ellan (verb), of Germanic origin; related to Dutch gillen and German gellen.}%
\par%
\entry{yield}{/jiːld/}{উৎপাদ}{\small{\textsf{\textit{noun, verb}}} \\{\fontspec{DejaVu Sans}▪ }\textsf{\textit{noun}}\\ \textbf{1} An amount produced of an agricultural or industrial product. {\fontspec{DejaVu Sans}◇} \textit{the milk yield was poor} \\{\fontspec{DejaVu Sans}▪ }\textsf{\textit{verb}}\\ \textbf{1} Produce or provide (a natural, agricultural, or industrial product) {\fontspec{DejaVu Sans}◇} \textit{the land yields grapes and tobacco} \colorBulletS{SYN} produce, bear, give, supply, provide, afford, return, bring in, pull in, haul in, gather in, fetch, earn, net, realize, generate, furnish, bestow, pay out, contribute \textbf{2} Give way to arguments, demands, or pressure. {\fontspec{DejaVu Sans}◇} \textit{the Western powers now yielded when they should have resisted} \colorBulletS{SYN} surrender, capitulate, submit, relent, admit defeat, accept defeat, concede defeat, back down, climb down, quit, give in, give up the struggle, lay down one's arms, raise the white flag, show the white flag, knuckle under \textbf{3} (of a mass or structure) give way under force or pressure. {\fontspec{DejaVu Sans}◇} \textit{he reeled into the house as the door yielded} \colorBulletS{SYN} bend, give, flex, be flexible, be pliant}{}{The growers successfully overcame the situation by taking additional measures as per instructions of the agro{-}officials and experts, resulting in good yield.       }{ \colorBullet{ORIGIN} Old English g(i)eldan ‘pay, repay’, of Germanic origin. The senses ‘produce, bear’ and ‘surrender’ arose in Middle English.}%
\par%
\entry{yielding}{/ˈjiːldɪŋ/}{প্রদায়ক}{ \textsf{\textit{adjective}}\ \textbf{1} (of a substance or object) giving way under pressure; not hard or rigid. {\fontspec{DejaVu Sans}◇} \textit{she dropped on to the yielding cushions} \colorBulletS{SYN} malleable, easily influenced, impressionable, flexible, adaptable, pliant, compliant, docile, biddable, tractable, like putty in one's hands, yielding, manageable, governable, controllable, amenable, accommodating, susceptible, suggestible, influenceable, persuadable, manipulable, responsive, receptive \textbf{2} Giving a product or generating a financial return of a specified amount. {\fontspec{DejaVu Sans}◇} \textit{higher{-}yielding wheat}}{}{Seed bodies of india and bangladesh yesterday formalised a move to cooperate on expediting trade of high{-}yielding varieties (hyv) of rice seeds for the benefit of the farmers of the two countries, and help boost food security.}{}%
\par%
\entry{you guys have a minute?}{}{}{\small{\textsf{\textit{}}}}{}{}{}%
\par%
\entry{you owe me.}{}{You use this phrase to point out that you're doing something nice for someone that will have to be "paid back" later. You might also hear another version of this phrase, which is even stronger: you owe me, big time.\newline%
}{\small{\textsf{\textit{}}}}{}{A: can you come and pick me up? Please?\newline%
B: ok, but you owe me one.\newline%
}{}%
\par%
\end{multicols}%
\end{document}